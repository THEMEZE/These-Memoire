\chapter{Dynamique hors-équilibre et hydrodynamique généralisée}
\label{chap:GHD}
\minitoc

%\chapter{Hydrodynamique généralisée (GHD)}

\section*{Introduction}

\section{Equation Hydrodynamique Généralisé}

\subsection{Description classique sans interaction}
Considérons une distribution classique de particules dans l’espace des phases, notée $\varphi(x, p, t)$, représentant la densité de particules autour du point $(x, p)$ à l’instant $t$. En l’absence de phénomènes dissipatifs, cette densité est conservée le long du flot hamiltonien, c’est-à-dire \(
\frac{d\varphi}{dt} = \frac{\partial \varphi}{\partial t} + \{ \varphi , H \} = 0,
\)
où $\{ \cdot , \cdot \}$ désigne le crochet de Poisson canonique :
\begin{equation}
\{ \varphi , H \} = \frac{\partial \varphi}{\partial x} \frac{\partial H}{\partial p} - \frac{\partial \varphi}{\partial p} \frac{\partial H}{\partial x}.
\end{equation}
Pour $d \varphi /dt = 0 $ , 
\begin{equation}
	\frac{\partial \varphi}{\partial t} + \{ \varphi , H \} = 0	
\end{equation}


Ce résultat exprime que la distribution $\varphi$ est constante le long des trajectoires dans l’espace des phases générées par le hamiltonien $H$. Sous cette hypothèse, on peut réécrire l’équation de conservation sous forme différentielle :

\begin{equation}
\partial_t \varphi + \partial_x ( \dot{x} \varphi ) + \partial_p ( \dot{p} \varphi ) = 0,
\end{equation}

où les équations du mouvement hamiltonien sont :
\(
\dot{x} = \frac{\partial H}{\partial p}, \qquad \dot{p} = -\frac{\partial H}{\partial x}.
\)

Cette équation prend alors la forme d’une équation de continuité dans l’espace des phases :

\begin{equation}
\partial_t \varphi + \partial_x j_x + \partial_p j_p = 0,
\end{equation}

où les densités de courant sont données par :
\(
j_x = \dot{x} \varphi, \qquad j_p = \dot{p} \varphi.
\)

\paragraph{Exemple : particules libres dans un potentiel externe}
Prenons pour Hamiltonien :

\begin{equation}
H = \varepsilon(p) + V(x), \qquad \text{où } \varepsilon(p) = \frac{p^2}{2m},
\end{equation}

correspondant à un système de particules classiques de masse $m$ soumises à un potentiel externe $V(x)$, sans interaction entre particules.

L'équation de conservation s’écrit alors :

\begin{equation}
\partial_t \varphi + v(p) , \partial_x \varphi - \partial_x V(x) , \partial_p \varphi = 0,
\end{equation}

où $v(p) = \partial_p \varepsilon(p) = p/m$ est la vitesse du flot hamiltonien dans l’espace des phases.

\paragraph{Charges locales conservées et équations hydrodynamiques}
On définit une observable locale (ou charge locale) $q[f](x, t)$ associée à une fonction test $f(p)$ par :

\begin{equation}
q[f](x, t) = \frac{1}{m} \int_{\mathbb{R}} dp \, f(p) \, \varphi(x, p, t).
\end{equation}

Cette quantité représente la moyenne locale de $f(p)$ pondérée par la distribution $\varphi$. En particulier : la densité de particules : $n(x, t) = q[1]$,l’impulsion moyenne locale : $u(x, t) = \frac{q[p]}{n m}$, la pression cinétique : $\mathcal{P}(x, t) = \frac{1}{m} \left( q[p^2] - \frac{q[p]^2}{q[1]} \right)$.

Les courants associés à ces charges s’écrivent :
\begin{equation}
j[f](x, t) = \frac{1}{m} \int dp \, f(p) , \partial_p H(x, p) \, \varphi(x, p, t).
\end{equation}

En prenant la dérivée temporelle de $q[f]$ et en utilisant l’équation de Liouville, on obtient une équation de conservation de la forme :

\begin{equation}
\partial_t q[f] + \partial_x j[f] = \frac{1}{m} \int dp \, f(p) , \partial_p \left( \partial_x V(x) \, \varphi \right),
\end{equation}

qui ne s’annule en général que si $V(x)$ est constant. Toutefois, dans le régime dit hydrodynamique, où $\varphi(x,p,t)$ varie lentement en espace, cette équation devient fermée sur les seules densités $q[f]$, en négligeant les dérivées spatiales d'ordre élevé.

Dans ce cadre, et en ne retenant que les premières charges conservées associées à $f(p) = 1$, $p$, $p^2$, on retrouve les équations d’Euler classiques :

\begin{eqnarray*}
	\partial_t n + \partial_x (n u) &=& 0, \\
	\partial_t (m n u) + \partial_x (m n u^2 + \mathcal{P}) &=& -n \, \partial_x V(x), \\
	\partial_t \mathcal{E} + \partial_x j[\varepsilon(p)] &=& -\partial_x V(x) \cdot q[p],
\end{eqnarray*}

où $\mathcal{E} = q[\varepsilon(p)]$ est la densité d'énergie, et $j[\varepsilon(p)]$ le courant d'énergie.

\paragraph{Remarques sur les charges globales}
En l’absence de potentiel externe ($V = 0$), le système conserve certaines charges globales. Dans un système classique non intégrable, seules ces quelques charges sont conservées. Par exemple dans un système de Gibbs sont conservé
\(
	Q[1] = \int dx \, q[1] \quad \text{(nombre total de particules)}, 
	Q[p] = \int dx \, q[p] \quad \text{(quantité de mouvement totale)}, 
	Q\left[\frac{p^2}{2m}\right] = \int dx \, q\left[\frac{p^2}{2m}\right] \text{(énergie cinétique totale)}.
\)
En revanche, dans un système intégrable, une infinité de charges sont conservées. En particulier, pour tout $p \in \mathbb{R}$:

\begin{equation}
Q[\delta(\cdot - p)] = \frac{1}{m} \int dx \, \varphi(x, p, t),
\end{equation}

%et les charges associées à une observable quelconque $f(p)$ s’écrivent :
%
%\begin{equation}
%Q[f] = \int dp , f(p) , \rho(p, t).
%\end{equation}
%
%Cette structure est l’analogue classique de la description en termes de "rapidity distribution function" dans le cadre quantique. Elle constitue le point de départ naturel pour développer une description hydrodynamique généralisée (GHD) dans le cas intégrable.
%
%

\subsection{Description classique avec interactions}

%Dans la formulation hamiltonienne de la GHD, le champ dynamique est la densité fluide à deux variables  . 
On définit un crochet de Poisson fonctionnel agissant sur les fonctionnelles $F[\rho]$ et $G[\rho]$  de la distribution de rapidité , avec intéraction. Conformément à Bonnemain et al.\cite{bonnemain2024hamiltonian}  :
\begin{equation}
	\{F,G\}\;=\;\iint dx\,d\theta\;\frac{\nu(\theta)}{2\pi}\,\Bigl[\partial_x\frac{\delta F}{\delta \rho(x,\theta)}\,\left(\partial_\theta \left ( \frac{\delta G}{\delta \rho(x,\theta)} \right ) \right)^{\mathrm{dr}} -\partial_x\frac{\delta G}{\delta \rho (x,\theta)}\,\left( \partial_\theta \left ( \frac{\delta F}{\delta \rho (x,\theta)} \right )\right)^{\mathrm{dr}} \Bigr]\,,	
\end{equation}
où $\nu$ est la fonction d’occupation.
%\cite{bonnemain2024hamiltonian,doyon2020lecture}

Pour toute fonction réelle et régulière \( f(x, \theta) \) définie sur \( \mathbb{R}^2 \), on associe le fonctionnel linéaire suivant :
\begin{equation}
	Q[f] = \int_{\mathbb{R}^2} dx\, d\theta\, f(x, \theta)\, \rho(x, \theta).
\end{equation}
Il s'agit de la charge totale associée à une quantité prenant la valeur \( f(x, \theta) \) pour chaque quasi-particule.  Le crochet de Poisson entre deux charges \( Q[f] \) et \( Q[g]\) s’écrit :
\begin{equation}
	\{ Q[f] , Q[g] \} = \int_{\mathbb{R}^2} \frac{dx\, d\theta}{2\pi} \nu  \left( \partial_x f  (\partial_\theta g )^{\mathrm{dr}}  - \partial_x g (\partial_\theta f)^{\mathrm{dr}}  \right),
\end{equation}
or l'application dressing satisfait la relation de symétrie \cite{doyon2020lecture}:
\begin{equation}
	\int_{\mathbb{R}^2}	 dx\, d\theta \, \nu f g^{\mathrm{dr}} = \int_{\mathbb{R}^2}	 dx\, d\theta \, \nu f^{\mathrm{dr}} g,
\end{equation}
soit avec une integration part partie, on réécrit le crochet 
\begin{equation}
	\{ Q[f] , Q[g]\} = \int_{\mathbb{R}^2} \frac{dx\, d\theta}{2\pi} f  \left( \partial_\theta ( \nu   (\partial_x g )^{\mathrm{dr}} )   - \partial_x ( \nu   (\partial_\theta g )^{\mathrm{dr}} )  \right).
\end{equation}

La distribution de rapidité $\rho( x , \theta )  = Q[\delta( \cdot - x )\delta( \cdot - \theta  )  ]$ et pour un hamiltinien $H = Q[h]$ avec $h(x , \theta ) = \varepsilon(\theta) + V(x)$ avec $\varepsilon(\theta) = m \theta^2/2$.

\begin{equation}
	\{ \rho(x, \theta), Q[h] \} + \partial_x (v^{\mathrm{eff}} \rho) + \partial_\theta (a^{\mathrm{eff}} \rho) = 0.
\end{equation}

Nous avons ici utilisé les identités (2.29), ainsi que la définition de la fonction d’occupation (rappelée pour commodité) :

\begin{equation}
	v^{\mathrm{eff}} = \frac{\varepsilon'^{\mathrm{dr}}}{1^{\mathrm{dr}}}, 
	\quad 
	a^{\mathrm{eff}} = -V, 
	\quad 
	\nu = \frac{\rho}{\rho_s}.
\end{equation}

Ainsi, en posant \( \partial_t \rho(x, \theta) = \{ \rho(x, \theta), Q[h] \} \), on retrouve bien les équations de la GHD sous forme hamiltonienne étendue à l’espace :

\begin{equation}
	\partial_t \rho(x, \theta) = -\partial_x (v^{\mathrm{eff}} \rho) - \partial_\theta (a^{\mathrm{eff}} \rho).
\end{equation}












