\chapter{Dispositif expérimental et méthodes d’analyse}
\label{chap:disp.exp}
\minitoc

%\section{Présentation de l’expérience}
%\section*{Introduction}
%
%\section{Refroidissement}
%
%\section{Imagerie}
%\subsection{Prubleme d'iamgerie et idée numerique}
%
%\section{Confinement transverse}
%
%\section{Confinement longitudinale}
%
%\subsection{Evolution logitudinale}
%
%\section{Outil de sélection spatial}
%
%\subsection{Mesure de distribution de rapidités locales $\rho(x , \theta ) $  pour des systèmes en équilibre}
%
%%\subsection{Piégeage transverses et longitudinale}
%%\section{Outil de sélection spatial}
%%%\section{Mesure de $\rho(x , \theta ) $ }
%
%%\section{Mesure de distribution de rapidités locales $\rho(x , \theta ) $  pour des systèmes en équilibre}

\section*{Introduction}

\begin{itemize}
	\item Objectif du chapitre : présentation synthétique de l’expérience
	\item Distinction claire des contributions : mise en place initiale (précédents doctorants), développement (travail de Léa Dubois), contribution personnelle (prise de données, analyses spécifiques, participation à certaines manipulations)
	\item Rôle de l’expérience dans l’étude de la dynamique des gaz de Bose 1D
\end{itemize}

Ce chapitre présente l’expérience utilisée pour étudier les gaz unidimensionnels de rubidium ultra-froids. Nous décrivons l’architecture du dispositif, les méthodes d’imagerie et d’analyse, ainsi que les protocoles expérimentaux auxquels j’ai participé. Le développement initial du refroidissement et du piégeage avant la puce a été réalisé par d’anciens doctorants. La mise en place du piégeage sur la puce et du système de sélection spatiale à l’aide d’un DMD a été initiée par Léa Dubois, alors en première année de doctorat à mon arrivée. Mon travail s’est concentré principalement sur la prise de données, l’analyse et la participation à certaines expériences spécifiques telles que l’expansion longitudinale et la mesure locale de la distribution de rapidité.


\paragraph{Objectif du chapitre}  
Ce chapitre a pour objectif de fournir une présentation synthétique et structurée du dispositif expérimental utilisé pour étudier la dynamique de gaz de Bose unidimensionnels ultra-froids. Il constitue un socle indispensable pour comprendre les protocoles expérimentaux développés au cours de ma thèse et les analyses présentées dans les chapitres suivants.

\paragraph{Architecture générale}  
Nous présentons d'abord l’architecture complète de l’expérience, depuis la production des atomes jusqu’à leur imagerie, en passant par les étapes de refroidissement, de piégeage magnétique sur puce, de manipulation optique, et de génération de potentiels. Cette description s’accompagne d’une mise en contexte des contributions historiques au dispositif.

\paragraph{Contributions successives et personnelles}  
Une attention particulière est portée à la répartition chronologique des contributions. Les étapes initiales (source atomique, MOT, piège DC) ont été développées par d’anciens doctorants. La mise en place du piégeage 1D sur puce ainsi que l’utilisation du DMD pour la sélection spatiale ont été réalisées au cours de la thèse de Léa Dubois. Mon travail s’inscrit dans cette continuité et concerne principalement la prise de données, l’analyse de protocoles dynamiques, ainsi que la participation à certaines opérations de maintenance et d’optimisation du système.

\paragraph{Rôle du dispositif dans la thèse}  
Ce dispositif permet d’explorer des phénomènes hors équilibre dans des gaz quantiques 1D. Il constitue une plateforme particulièrement adaptée à l’étude de protocoles d’expansion, de sondes locales, ou de dynamiques guidées par la théorie hydrodynamique généralisée (GHD), qui sont au cœur de cette thèse.




\section{Présentation générale de l’expérience}
\subsection{Vue d’ensemble du dispositif}
\begin{itemize}
    \item Architecture générale : production, piégeage, manipulation et imagerie.
    \item Systèmes étudiés : gaz de rubidium 87 dans des pièges 1D.
    \item Objectifs : exploration de dynamiques hors équilibre.
\end{itemize}

\subsection{Historique et contributions successives}
\begin{itemize}
    \item Étapes de refroidissement et piégeage initial : travaux antérieurs (voir thèses citées).
    \item Développement du piégeage 1D sur puce et du DMD : thèse de Léa Dubois.
    \item Contributions personnelles : prise de données, protocoles dynamiques, analyse.
\end{itemize}

\section{Le dispositif expérimental}
\subsection{Système laser et contrôle de fréquence}
\label{sec:systeme_laser}

%\paragraph{Laser maître 1 : référence de fréquence}
%La référence principale de fréquence pour l'ensemble des faisceaux utilisés dans l'expérience est fournie par un laser à cavité étendue, développé au SYRTE. Ce laser est asservi par spectroscopie d’absorption saturée sur la transition D2 du $^{87}$Rb, au croisement des transitions $|F=2\rangle \rightarrow |F'=2,3\rangle$. Ce signal de référence est utilisé pour verrouiller les autres sources laser par battement optique.

\paragraph{Laser maître 1 : référence de fréquence}
La stabilité en fréquence de l’ensemble des faisceaux employés dans l’expérience est assurée par un laser à cavité étendue conçu au SYRTE. Ce laser est verrouillé par spectroscopie d’absorption saturée sur la raie D2 du $^{87}$Rb, en ciblant le croisement des transitions $|F=2\rangle \rightarrow |F'=2,3\rangle$. Ce verrouillage fournit la référence absolue de fréquence à partir de laquelle les autres sources laser sont synchronisées par battement optique.

%\paragraph{Laser repompeur}
%Un laser DFB (Distributed Feedback Diode) est utilisé pour produire le faisceau repompeur, permettant de transférer les atomes retombés dans l’état $|F=1\rangle$ vers l’état $|F=2\rangle$. Ce laser est asservi à une fréquence distante de 6\,GHz de celle du maître 1, en utilisant un montage de battement optique et mélange avec un oscillateur à 6.6\,GHz. Une diode Fabry-Perot injectée par la DFB permet d’amplifier la puissance au-delà de 100\,mW.
%
%\paragraph{Laser repompeur}
%Le faisceau de repompage, qui permet de transférer les atomes piégés dans l’état $|F=1\rangle$ vers l’état $|F=2\rangle$, est généré par une diode DFB (Distributed Feedback). Sa fréquence est décalée de 6,GHz par rapport au maître 1 grâce à un système de battement optique combiné à un mélange avec un oscillateur micro-onde à 6.6,GHz. Une diode Fabry–Perot, injectée par la DFB, permet d’augmenter la puissance de sortie au-delà de 100,mW.

\paragraph{Laser repompeur}
Le faisceau de repompage, qui transfère les atomes tombé  dans l’état $|F=1\rangle$ vers l’état $|F=2\rangle$, est produit par une diode DFB (Distributed Feedback). Sa fréquence est décalée de 6 GHz par rapport au maître 1 par battement optique et mélange avec un oscillateur à micro-ondes de 6.6 GHz. Une diode Fabry–Perot, injectée par la DFB, élève la puissance de sortie au-delà de 100 mW.

%\paragraph{Laser maître 2 : laser principal de manipulation}
%Un second laser à cavité étendue, identique au maître 1, est asservi par battement optique à la fréquence du maître 1. Il est amplifié par un amplificateur à semi-conducteur évasé (Tapered Amplifier), permettant d’atteindre une puissance de sortie supérieure à 1\,W. Ce faisceau est ensuite divisé en plusieurs branches pour alimenter :
%\begin{itemize}
%    \item le Piège Magnéto-Optique (PMO),
%    \item la mélasse optique,
%    \item le pompage optique,
%    \item l’imagerie par absorption,
%    \item le faisceau de sélection.
%\end{itemize}

\paragraph{Laser maître 2 : source principale de manipulation}
Un second laser à cavité étendue, est verrouillé par battement optique sur la fréquence du maître 1. L’émission est amplifiée au moyen d’un amplificateur à semi-conducteur évasé (Tapered Amplifier), fournissant plus de 1\,W en sortie. Le faisceau ainsi produit est distribué vers différentes parties de l’installation expérimentale : alimentation du piège magnéto-optique (PMO), formation de la mélasse optique, réalisation du pompage optique, imagerie par absorption,génération du faisceau de sélection.


%\paragraph{Contrôle de fréquence et polarisation}
%Les fréquences des différents faisceaux sont ajustées via des Modulateurs Acousto-Optiques (AOM), tandis que leur polarisation et leur intensité sont contrôlées à l’aide de cubes PBS en combinaison avec des lames demi-onde motorisées ou fixes. Cette configuration assure une grande flexibilité dans la mise en œuvre des différentes phases expérimentales.

\paragraph{Gestion des fréquences et polarisations}
%Les ajustements de fréquence des divers faisceaux sont réalisés à l’aide de modulateurs acousto-optiques (AOM).
Les faisceaux peuvent être interrompus soit à l’aide d’obturateurs mécaniques, soit via des modulateurs acousto-optiques (AOM). Ces derniers offrent un temps de commutation beaucoup plus court que les systèmes mécaniques, car ils permettent de sélectionner uniquement un ordre de diffraction non nul et d’éteindre instantanément le faisceau en interrompant l’alimentation radiofréquence. L’intensité et la polarisation sont réglées via des cubes séparateurs PBS associés à des lames demi-onde, fixes ou motorisées. Ce dispositif offre une grande souplesse pour adapter la configuration optique aux différentes étapes de l’expérience.

%\paragraph{Remarque}
%Une description plus détaillée du montage laser et de son verrouillage peut être trouvée dans la thèse de A.~Johnson~\cite{Johnson2016}. L’ensemble a été maintenu et utilisé sans modifications majeures au cours de ma thèse.

\paragraph{Note}
Une présentation plus exhaustive du montage laser et de son système de verrouillage est disponible dans la thèse de A.Johnson\cite{Johnson2016}. Le dispositif a été conservé dans son architecture d’origine tout au long de mes travaux, avec seulement un entretien régulier.


\subsection{Production et refroidissement des atomes (non détaillé ici, renvoi à d'autres travaux)}
{\color{blue}
\begin{itemize}
    \item Source chaude de rubidium, MOT, molasses optique.
    \item Refroidissement à des températures sub-$\mu~K$ Refroidissement sub-Doppler (détails renvoyés aux travaux précédents).
\end{itemize}
}
%Le dispositif expérimental permet de produire des gaz de rubidium ultra-froids, avec pour objectif final l’obtention de gaz unidimensionnels dans le régime quantique dégénéré. La production suit une séquence expérimentale déjà bien établie, initialement développée par d’anciens doctorants (voir par exemple la thèse d’A. Johnson~\cite{Johnson2016}), puis réoptimisée au début de la thèse de Léa-Dubois ~\cite{L.Dubois2024} sous la supervision d’I. Bouchoule.

Le dispositif expérimental permet de produire des gaz ultra-froids de rubidium, en vue d’obtenir des gaz unidimensionnels dans le régime quantique dégénéré. La séquence expérimentale suit un protocole établi, initialement développé par d’anciens doctorants (voir par exemple la thèse d’A. Johnson~\cite{Johnson2016}) et réoptimisé au début de la thèse de Léa. Dubois~\cite{L.Dubois2024} sous la supervision d’I. Bouchoule.

%\paragraph{Libération des atomes de rubidium}
%Les atomes de $^{87}$Rb sont libérés à partir d’un \emph{dispenser}, placé directement dans l’enceinte à vide, sur le côté de la monture de la puce atomique. Ce composant, parcouru par un courant de \( 4.5\,\mathrm{A} \) pendant environ \( 5\,\mathrm{s} \), émet un flux d’atomes thermiques dans la chambre à vide.

\paragraph{Libération des atomes de rubidium}
Les atomes de $^{87}$Rb sont émis à partir d’un \emph{dispenser} placé directement dans l’enceinte à vide, à proximité de la monture de la puce atomique. Un courant de \( 4.5\,\mathrm{A} \)  est appliqué pendant environ \( 5\,\mathrm{s} \), générant un flux d’atomes thermiques dans la chambre à vide.

%
%\paragraph{Capture par piège magnéto-optique (PMO)}
%Les atomes thermiques sont ralentis et piégés à l’aide d’un piège magnéto-optique. Celui-ci utilise quatre faisceaux laser (dont deux sont réfléchis par la puce) et un champ quadrupolaire magnétique généré par des bobines. Le nuage ainsi formé se situe à quelques millimètres de la surface de la puce.

\paragraph{Capture par le piège magnéto-optique (PMO)}
Les atomes thermiques sont ralentis et confinés dans un piège magnéto-optique. Quatre faisceaux laser (dont deux réfléchis par la puce) combinés à un champ quadrupolaire magnétique produit par des bobines permettent de former un nuage atomique situé à quelques millimètres de la surface de la puce.

%\paragraph{Rapprochement vers la puce}
%Pour rapprocher les atomes de la puce, on transfère le champ quadrupolaire depuis les bobines vers un champ généré par le fil en forme de U de la puce (fil bleu dans la Fig.~\ref{fig:puce}). Ce fil est parcouru par un courant variant de \( 3.6\,\mathrm{A} \) à \( 1.5\,\mathrm{A} \), ce qui rapproche le nuage à quelques centaines de micromètres de la surface.

\paragraph{Rapprochement vers la puce}
Le nuage est rapproché de la surface de la puce en transférant le champ quadrupolaire depuis les bobines vers le champ produit par le fil en forme de U de la puce (fil bleu, Fig.~\ref{fig:puce}). Le courant dans ce fil est ajusté lentement de \( 3.6\,\mathrm{A} \) à \( 1.5\,\mathrm{A} \), ce qui positionne le nuage à quelques centaines de micromètres de la surface.

%\paragraph{Mélasse optique}
%Une phase de mélasse optique permet un refroidissement sub-Doppler des atomes capturés. Un système d’imagerie provisoire est utilisé à cette étape pour visualiser le nuage atomique, dont la taille dépasse le champ d’observation du système d’imagerie final.

%\paragraph{Mélasse optique}
%Une étape de mélasse optique est ensuite appliquée pour atteindre un refroidissement sub-Doppler des atomes capturés. %Un système d’imagerie provisoire permet de visualiser le nuage, dont la taille dépasse le champ d’observation du dispositif final.

%\paragraph{Pompage optique}
%Afin de polariser les atomes dans l’état magnétique \( |F=2,\,m_F=2\rangle \), un pompage optique est effectué avec un faisceau circulairement polarisé \( \sigma^+ \), résonant sur la transition \( |F=2\rangle \rightarrow |F'=2\rangle \).

\paragraph{Pompage optique}
Enfin, les atomes sont préparés dans l’état magnétique \( |F=2,\,m_F=2\rangle \) par pompage optique. Un faisceau circulairement polarisé \( \sigma^+ \), résonant sur la transition \( |F=2\rangle \rightarrow |F'=2\rangle \), assure la polarisation du nuage.

\paragraph{Mélasse optique}
Après la capture dans le PMO, une étape de mélasse optique est appliquée pour refroidir davantage le nuage atomique, au-delà de la limite de Doppler. La mélasse optique repose sur l’utilisation de faisceaux laser légèrement désaccordés en fréquence et polarisés de manière appropriée, qui interagissent avec les atomes selon le mécanisme de refroidissement sub-Doppler.

Le principe physique est le suivant : les atomes en mouvement voient les faisceaux laser avec un décalage Doppler, ce qui modifie la probabilité d’absorption selon leur vitesse et leur position. Combiné avec les effets de polarisation (notamment les forces de type Sisyphus dans un champ de polarisation variable), cela crée un potentiel de friction optique qui ralentit les atomes. Contrairement au refroidissement Doppler standard, la mélasse optique permet de réduire l’énergie cinétique des atomes en dessous de la limite Doppler, atteignant des températures beaucoup plus basses.

Ainsi, cette étape permet d’obtenir un nuage plus dense et plus froid, condition essentielle pour les manipulations ultérieures et la formation de gaz unidimensionnels dans le régime quantique dégénéré.






\subsection{Piégeage magnétique sur puce}
{\color{blue}
\begin{itemize}
    \item Présentation de la puce atomique.
    \item Confinement transverse et longitudinal.
    \item Régime 1D : conditions d’accès (\(\hbar \omega_\perp \gg k_B T\)).
    \item Problèmes de rugosité, stabilité magnétique.
\end{itemize}
}

\subsubsection{Piégeage magnétique sur puce}
\label{sec:piegeage_puce}

%On peut utiliser des piégeage optique pour produire des stracture atomique longitudinale alongé. Certaines groupe de recherche utilise un redeau optique 2D pour obtenir un réseau 2D de tube longitudinaaux \cite{Kinoshita2004,LaburtheTolra2004,Paredes2004,Moritz2003}. Ce raseaux 2D produit un grand nombre de systéme atomique propise è l'étude de de gase 1D. Avec ce genre de dispositif on peux etudier des gas 1D peut dense car les densité peut etre moyenné sur tous les tudes. Mais avec ce genre de dispositif on ne peut pas étudier experimentalement les fluctudation dans le systéme. Nous pour gièger les atomes on utilise une puce atomique.
%
%\paragraph{Principe général}
%Les atomes de rubidium sont piégés grâce à une puce atomique intégrée dans l’enceinte à vide. Une puce atomique est un circuit microfabriqué contenant des micro-fils dans lesquels circulent des courants permettant de générer des champs magnétiques à géométrie contrôlée. Ce dispositif, développé dans les années 1990 \cite{Denschlag1999,Fortagh1998}, permet une miniaturisation du système de piégeage \cite{Folman2000,Reichel1999}, les premiers condensats sur puce ont été obtenus en 2001 \cite{Haensel2001,Ott2001} et la premièref fois aux laboratoir Charles Fabry (LCF) \cite{Aussibal2003} et un accès à des confinements forts, particulièrement adaptés à l'étude de gaz de Bose unidimensionnels \cite{Schumm2005,Trebbia2006}.

-------

On peut créer des structures atomiques allongées en utilisant des techniques de piégeage optique. Par exemple, plusieurs groupes de recherche ont recours à des réseaux optiques bidimensionnels (2D) pour former un ensemble de tubes atomiques longitudinaux \cite{Kinoshita2004,LaburtheTolra2004,Paredes2004,Moritz2003}. Ces réseaux 2D permettent de produire un grand nombre de systèmes atomiques quasi-unidimensionnels, offrant ainsi une plateforme idéale pour l’étude des gaz 1D. Ce type de dispositif est particulièrement adapté à l’étude de gaz faiblement denses, car les densités peuvent être moyennées sur l’ensemble des tubes. Cependant, l’étude expérimentale des fluctuations locales dans chaque tube reste difficile avec ce genre de configuration. Pour surmonter cette limitation, on utilise le piégeage à l’aide de puces atomiques.

\paragraph{Principe général}
Les atomes de rubidium sont confinés par une puce atomique intégrée dans l’enceinte à vide. Une puce atomique est un circuit microfabriqué comportant de fins micro-fils parcourus par des courants électriques, ce qui permet de générer des champs magnétiques à géométrie contrôlée. Cette technologie, développée dans les années 1990 \cite{Denschlag1999,Fortagh1998}, offre une miniaturisation significative des dispositifs de piégeage \cite{Folman2000,Reichel1999}. Les premiers condensats de Bose–Einstein sur puce ont été réalisés en 2001 \cite{Haensel2001,Ott2001}, puis ultérieurement au Laboratoire Charles Fabry \cite{Aussibal2003}. Les puces atomiques permettent d’accéder à des confinements très forts, particulièrement adaptés à l’étude des gaz de Bose unidimensionnels et à l’exploration de leurs propriétés quantiques locales \cite{Schumm2005,Trebbia2006}.


-----
Des structures atomiques allongées peuvent être réalisées par piégeage optique. Dans ce cadre, des réseaux optiques bidimensionnels (2D) permettent de créer un ensemble de tubes atomiques quasi-unidimensionnels \cite{Kinoshita2004,LaburtheTolra2004,Paredes2004,Moritz2003}. Ces réseaux offrent un grand nombre de systèmes atomiques identiques, facilitant l’étude statistique de gaz 1D faiblement dense. Toutefois, l’accès expérimental aux fluctuations locales dans chaque tube reste limité.

Pour contourner cette contrainte, les puces atomiques offrent une solution efficace. Ces dispositifs microfabriqués intègrent de fins micro-fils parcourus par des courants, générant des champs magnétiques de géométrie contrôlée et permettant des confinements très forts \cite{Denschlag1999,Fortagh1998,Folman2000,Reichel1999}. La miniaturisation ainsi obtenue a permis l’obtention des premiers condensats de Bose–Einstein sur puce dès 2001 \cite{Haensel2001,Ott2001}, et dés 2003 au Laboratoire Charles Fabry \cite{Aussibal2003}. Grâce à ces confinements, il devient possible d’étudier expérimentalement les propriétés de gaz de Bose unidimensionnels et leurs fluctuations locales \cite{Schumm2005,Trebbia2006}.

-------

\paragraph{Structure de la puce utilisée}
La puce utilisée au cours de cette expérience a été conçue en collaboration avec S.~Bouchoule, A.~Durnez et A.~Harouri (C2N). Elle repose sur un substrat de carbure de silicium sur lequel est déposé le circuit électrique. Ce dernier est recouvert d’une couche de résine BCB, aplanie par des cycles d’enduction et d’attaque plasma. Une fine couche d’or (\(\sim200\,\mathrm{nm}\)) est finalement évaporée afin de permettre l’utilisation de la puce comme miroir pour l’imagerie à \(780\,\mathrm{nm}\). La puce est soudée à l’indium sur une monture en cuivre inclinée à \(45^\circ\) par rapport à l’axe optique.

%\paragraph{Fils de piégeage et géométrie des champs}
%Plusieurs fils sont intégrés à la puce pour assurer les différentes étapes du piégeage et du transport des atomes : un fil en forme de Z est utilisé pour le piégeage initial (DC), tandis que trois micro-fils (symétriques et parallèles) sont utilisés pour former un guide unidimensionnel par courants alternatifs (AC). La géométrie des fils a été optimisée pour minimiser la dissipation de chaleur, limiter les couplages parasites et améliorer la symétrie du piège. Dans la zone d’intérêt, les atomes sont piégés à environ \(15\,\mu\mathrm{m}\) au-dessus des fils, soit à \(8\,\mu\mathrm{m}\) au-dessus de la surface de la puce.

%\paragraph{Fils de piégeage et géométrie des champs}
%La puce atomique comporte plusieurs ensembles de fils, chacun jouant un rôle précis dans les différentes étapes de la capture, du transport et du confinement des atomes.
\paragraph{Fils de piégeage et géométrie des champs}
La puce atomique intègre plusieurs ensembles de conducteurs, chacun conçu pour une étape spécifique de la capture, du transport et du confinement des atomes. L’ensemble de la séquence de transfert, depuis le piège magnéto-optique (PMO) jusqu’au guide unidimensionnel, repose sur une succession de configurations magnétiques générées par ces différents fils.

%\medskip
%\subparagraph{Fil en forme de U .}
%Après la phase de pré-refroidissement, le nuage est initialement capturé dans un piège magnéto-optique (PMO) situé au-dessus de la puce. Il est ensuite approché de la surface en transférant progressivement le champ quadrupolaire des bobines externes vers celui produit par un fil en forme de U intégré à la puce (phase \textit{U} : transfert du PMO vers la puce + mélace optique + ponpage optique). 

\subparagraph{Phase U : approche de la surface}
Après la phase de pré-refroidissement, le nuage est initialement capturé dans un PMO situé au-dessus de la puce. Il est ensuite rapproché de la surface en transférant progressivement le champ quadrupolaire des bobines externes vers celui produit par un fil en forme de U intégré à la puce (fils bleus dans la Fig.~\ref{fig:puce}). Cette étape (\textit{phase U}) est accompagnée d’un mélange optique et d’un pompage optique afin de préparer les atomes pour le piégeage magnétique.


%\medskip
%\subparagraph{Fil en forme de Z : Chargement dans le piège DC .}
%Après le pompage optique, les atomes sont transférés dans un piège magnétique combinant un courant continu circulant dans le fil en forme de Z de la puce (fil orange dans la Fig.~\ref{fig:puce}) et un champ magnétique externe. Ce piège, noté \emph{piège DC}, permet un confinement transverse important. Un refroidissement par évaporation radiofréquence est alors réalisé pendant environ \( 2.3\,\mathrm{s} \), ce qui abaisse la température du nuage à environ \( 1\,\mu\mathrm{K} \), pour un nombre d’atomes typiquement autour de \( 2.5 \times 10^5 \).

\subparagraph{Phase Z : piège DC et refroidissement}
À l’issue du pompage optique, les atomes sont transférés dans un piège magnétique combinant un courant continu circulant dans un fil en forme de Z (fil orange) et un champ magnétique externe. Ce \emph{piège DC} assure un confinement transverse fort. Un refroidissement par évaporation radiofréquence, d’une durée d’environ \(2.3\,\mathrm{s}\), abaisse la température du nuage à environ \(1\,\mu\mathrm{K}\), pour un nombre typique d’atomes de l’ordre de \(2.5\times 10^5\).
%\medskip
%Une fois chargé dans ce piège intermédiaire, le nuage est transporté vers la zone expérimentale. Dans cette région, trois micro-fils parallèles et symétriques (jaune), parcourus par des courants alternatifs (AC), créent un guide magnétique unidimensionnel assurant le confinement transversal des atomes. Le confinement longitudinal est obtenu grâce à deux paires de fils : d/d′ (rose) et D/D′ (vert).

\subparagraph{Transfert vers le guide unidimensionnel}
Une fois refroidi, le nuage est acheminé vers la zone expérimentale où trois micro-fils parallèles et symétriques (fils jaunes) parcourus par des courants alternatifs (AC) génèrent un guide magnétique unidimensionnel assurant le confinement transverse. Le confinement longitudinal est fourni par deux paires de fils : $d/d'$ (rose) et $D/D'$ (vert).

Le passage du piège DC au guide 1D est réalisé de manière adiabatique grâce à cinq rampes linéaires de courant d’une durée comprise entre \(50\) et \(60\,\mathrm{ms}\) chacune. Durant cette opération :  
(i) le courant dans le fil Z est progressivement réduit,  
(ii) le courant dans les micro-fils du guide est augmenté jusqu’à environ \(50\,\mathrm{mA}\),  
(iii) un courant initial de \(0.5\,\mathrm{A}\) est appliqué dans les fils $D$ et $D'$, puis ajusté pour maintenir fixe la position du centre de masse du nuage.  

Ce protocole minimise les oscillations résiduelles dans le guide et assure un découplage efficace entre la dynamique longitudinale et le confinement transverse. Ce dispositif a été développé au cours de la thèse de Léa Dubois~\cite{TheseLea} et a été utilisé dans le cadre de mes protocoles expérimentaux sur l’expansion longitudinale et les sondes locales de distribution de rapidité.

\subparagraph{Optimisation géométrique}
La géométrie des conducteurs de la puce a été conçue pour réduire la dissipation thermique, limiter les couplages parasites et garantir une bonne symétrie des champs magnétiques. Dans la zone expérimentale, les atomes sont piégés à environ \(15\,\mu\mathrm{m}\) au-dessus des fils, soit \(8\,\mu\mathrm{m}\) au-dessus de la surface de la puce.


\paragraph{Refroidissement final et accès au régime unidimensionnel}
Une dernière phase de refroidissement par évaporation radiofréquence est effectuée directement dans le guide AC. Grâce à l’anisotropie marquée du piège, le confinement transverse atteint une fréquence \(\omega_\perp\) telle que l’énergie quantique \(\hbar \omega_\perp\) dépasse largement les énergies thermique et chimique du système. On atteint ainsi le régime unidimensionnel, caractérisé par la hiérarchie d’énergies :
\[
k_B T, \mu \ll \hbar \omega_\perp,
\]
où \(\mu\) désigne le potentiel chimique et \(T\) la température du gaz.

Dans ce régime, le confinement transverse est assuré principalement par la géométrie des micro-fils et la présence de champs magnétiques externes, tandis que le confinement longitudinal, plus faible, est ajustable via une combinaison de champs magnétiques externes et de courants circulant dans des fils additionnels ($d/d'$ et $D/D'$). 

Les gaz obtenus contiennent typiquement entre \(3\times 10^3\) et \(1.5\times 10^4\) atomes, pour des températures de l’ordre de \(50\) à \(200\,\mathrm{nK}\). La Fig.~\ref{fig:gaz1D} illustre un exemple de nuage dans ce régime, observé avec le système d’imagerie final.



%\paragraph{Confinement transverse et longitudinal}
%Le confinement transverse est assuré principalement par la géométrie des fils et la présence de champs magnétiques externes. Sa fréquence élevée permet d’atteindre des énergies de confinement \(\hbar \omega_\perp\) bien supérieures aux énergies thermiques et chimiques du système, condition nécessaire à l’accès au régime 1D :
%\[
%k_B T, \mu \ll \hbar \omega_\perp.
%\]
%Le confinement longitudinal, plus faible, est modulable par combinaison de champs magnétiques externes et courants dans les fils additionnels.

\paragraph{Avantages du piégeage sur puce}
Comparé aux systèmes utilisant des réseaux optiques 2D, le piégeage sur puce ne fournit qu’un seul tube, ce qui permet un meilleur accès aux fluctuations locales de densité et aux observables résolues spatialement. Ce type de dispositif est ainsi particulièrement adapté à l'étude de la thermodynamique et de la dynamique de gaz 1D isolés.

\paragraph{Limitations et effets parasites}
Parmi les limitations spécifiques au piégeage sur puce figurent la rugosité des potentiels magnétiques due aux imperfections des fils, qui peut induire des modulations parasites du confinement longitudinal. De plus, la stabilité du dispositif est sensible aux champs parasites magnétiques externes ainsi qu’aux échauffements dus aux courants continus.





\paragraph{Imagerie finale}
À l’issue de ce refroidissement, les atomes sont observés avec le système d’imagerie final (voir Fig.~\ref{fig:imagerieFinale}), adapté aux tailles caractéristiques du gaz dans le piège. Une image typique de ce nuage est présentée en Fig.~\ref{fig:nuageDC}.



%\paragraph{Refroidissement final et accès au régime unidimensionnel}
%Une dernière phase de refroidissement par évaporation radiofréquence est ensuite réalisée dans le guide AC. Ce refroidissement, mené dans le piège à forte anisotropie, permet d’atteindre le régime unidimensionnel, caractérisé par la hiérarchie d’énergies :
%\[
%k_B T, \mu \ll \hbar \omega_\perp
%\]
%où \( \omega_\perp \) est la fréquence de confinement transverse, \( \mu \) le potentiel chimique et \( T \) la température du gaz.
%
%Les gaz obtenus contiennent typiquement entre \( 3 \times 10^3 \) et \( 1.5 \times 10^4 \) atomes, pour des températures de l’ordre de \( 50 \text{ à } 200\,\mathrm{nK} \). La Fig.~\ref{fig:gaz1D} montre un exemple de tel gaz observé avec le système d’imagerie final.


\paragraph{Remarques expérimentales}
Lorsque j’ai rejoint l’équipe, la première année thèse de Léa Dubois touchait à sa fin et le dispositif expérimental était en fonctionnement stable. Les différentes étapes du cycle (dispenser, PMO, mélasse, pompage optique, piège DC, transfert vers le guide, évaporation finale) avaient été mises en place et optimisées pendant les premières années de sa thèse, sous la supervision d’I. Bouchoule.Le cycle expérimental complet dure environ 15 secondes. Une description plus détaillée peut être trouvée dans la thèse d’A. Johnson~\cite{Johnson2016}.


Pendant ma première année, j’ai principalement participé à la prise de données en collaboration avec Léa. Grâce à la qualité de son travail, le dispositif était globalement très fiable, ce qui a permis de mener des campagnes expérimentales riches sans intervention lourde. Néanmoins, cette stabilité avait pour contrepartie que je n’ai pas été directement impliqué dans la résolution des pannes complexes ou dans le reconditionnement complet de la manipulation, ce qui a limité ma formation sur les aspects de maintenance approfondie du dispositif.

En revanche, peu avant la fin de la thèse de Léa et au début de ma troisième année, nous avons observé une chute significative du nombre d’atomes capturés. Sous la supervision d’I. Bouchoule, une intervention lourde a alors été décidée : nous avons cassé le vide pour diagnostiquer le problème. Il s’est avéré que les connecteurs du dispenser étaient endommagés. L’opération a été mise à profit pour installer un nouveau dispenser et remplacer la puce atomique.

Cette opération a mobilisé plusieurs personnes du laboratoire et de ses partenaires : S. Bouchoule (C2N) et Anne [Nom complet à préciser] ont participé à la manipulation et à la pose de la puce, tandis que j’ai pu assister à l’étuvage de l’enceinte à vide avec F. Nogrette. Après cette intervention, j’ai suivi avec I. Bouchoule le réajustement progressif de la séquence de refroidissement : alignement des faisceaux, réglages de la mélasse, optimisation du chargement dans le piège DC, puis dans le guide.

Cet épisode m’a permis de me confronter plus directement aux paramètres critiques du cycle d’évaporation et à la reprise d’une séquence complète. Toutefois, le départ de Léa, qui maîtrisait tous les aspects de la manipulation, a marqué une rupture importante dans la continuité des savoir-faire pratiques liés à cette expérience.


\begin{center}
	({fig:puce} — Schéma de la puce atomique avec fils U, Z, AC, D et D'.)
\end{center}
\begin{center}
	({fig:imagerieFinale} — Schéma optique du système d’imagerie final)
\end{center}
\begin{center}
	[{fig:nuageDC} — Image du gaz dans le piège DC après évaporation]
\end{center}
\begin{center}
	[{fig:gaz1D} — Image typique d’un gaz dans le régime 1D]
\end{center}



\subsection{Génération de potentiels modulés}
\begin{itemize}
    \item Courants modulés pour créer des pièges harmoniques ou quartiques.
    \item Découplage transverse/longitudinal.
\end{itemize}

\paragraph{Champ des micro-fils.}
Puisque que $m_F = 2 $, (état assuré par pompage optique), le potentiel magnétique $-\vec{\mu} \vec{B}(\vec{r}) $ (avec moment dipolaire magnétique alors $\vec{\mu}$ et le champs magnetque totale que resente les atomes$\vec{B}(\vec{r})$) est proportionnel à $\vert \vec{B}(\vec{r}) \vert$  de sorte que les atomes, en état low-field seeking, sont attirés vers les régions de champ magnétique minimal. Les micro-fils, alignés selon l’axe horizontal $\vec{e}_x$, sont parcourus par des courants alternatifs $\pm I$ (déphasés) produisant le champ magnétique de confinement : un fil central parcouru par un courant \( I \), et deux fils latéraux par des courants opposés \(-I\). 

\paragraph{Champ de biais.}
Un champ de biais transverse $\vec{B}_{\mathrm{biais}} = {B}_{\mathrm{biais}} \, \vec{e}_y$ , avec l'axe verticale par $\vec{e}_y$ , est appliqué afin de régler la distance des atomes par rapport aux micro-fils. En notant $\vec{e}_z$ l’axe horizontal perpendiculaire à $\vec{e}_x$ et $\vec{e}_y$ l’annulation du champ total a lieu en
%Dans cette configuration, un champ de biais transverse est appliqué pour ajuster la distance des atomes au-dessus des micro-fils. Pour y avoir une idée notons l'axe verticale par $\vec{e}_y$, et $\vec{B}_{biais} = {B}_{biais} \, \vec{e}_y$. Alors en notan $\vec{e}_z$ l'axe hortisontale perpetdiculaire à $\vec{e}_x$ et $\vec{e}_y$, le champs totale s'anume en ​
  %permet ainsi de positionner précisément le minimum du potentiel à une hauteur
$z_0 = \mu_0 I / (2 \pi {B}_{\mathrm{biais}} ) $ avec $\mu_0$ la perméabilité du vide . La modulation de ${B}_{\mathrm{biais}}$ permet de déplacer le point où le champ total s’annule, ce qui permet de positionner précisément le minimum du potentiel à une distance $d$ du plan des fils. 

\paragraph{Champ d’Ioffe.}
Afin d’éviter les pertes de Majorana liées à la présence d’un champ nul, un champ longitudinal $B_0 \, \vec{e}_x$ est ajouté, garantissant que le minimum de champ reste non nul.%.Un champ longitudinal (selon $\vec{e}_x$) $B_0$ est ajouté afin que ce minimum ne corresponde pas à un champ nul, ce qui supprime les pertes de Majorana dues aux inversions de spin au voisinage d’un zéro de champ. 

%L’intérêt de ces pièges est que les atomes peuvent être confinés très près des micro-fils — ici à $ d = 15\, \mu m$ , soit l’espacement entre deux fils — ce qui maximise le gradient de champ et donc la fréquence de piégeage transverse

\paragraph{Fréquence de piégeage transverse.}
Dans la configuration étudiée, les atomes sont confinés à $ d = 15\, \mu m$ au-dessus de la puce, soit l’espacement entre deux micro-fils. Cette faible distance maximise le gradient de champ et donc la fréquence de piégeage transverse, qui s’écrit
\begin{eqnarray*}
	\omega_\perp^{(0)} =  \sqrt{\frac{\mu_B}{mB_0}} \frac{\mu_0 I }{2\pi d^2} 
\end{eqnarray*}
avec $\mu_B$ le magnéton de Bohr, $m$ la masse atomique et $\mu_0$ la perméabilité du vide.
%Pour éviter que les atomes ne perçoivent les rugosités magnétiques dues aux défauts des conducteurs, on fait circuler dans les fils un courant alternatif à haute fréquence ($\sim 400\,KHz$) : le potentiel est alors moyenné temporellement, produisant un confinement plus lisse. À $15\, \mu m$ au-dessus de la puce, le profil de champ est localement harmonique, et la fréquence de piégeage transverse devient

\paragraph{Rugosité et suppression par modulation}
Les imperfections géométriques des micro-fils engendrent des fluctuations parasites du champ magnétique le long du guide, créant une rugosité du potentiel. Pour la supprimer, les courants sont modulés à haute fréquence ($\sim 400\,KHz$), bien au-delà des fréquences de piégeage. Dans ce régime, les atomes ne perçoivent que le potentiel moyenné temporellement, où la composante parasite longitudinale est fortement réduite. Le confinement effectif reste harmonique, avec une fréquence transverse donnée par
\begin{eqnarray*}
	\omega_\perp = \frac{\omega_\perp^{(0)}}{\sqrt{2}}.		
\end{eqnarray*}




%\paragraph{Découplage des confinements transverses et longitudinaux.}
%Les courants qui parcourent les fils D, D', d, d' sons selon $\vec{e}_u$ donc les chanps induit sont selon $\vec{e}_x$ noté $B_\parallel^x$ et $\vec{e}_v$ (axex normale à la puce) , noté $B_\parallel^v$. Si les champs selon $\vec{e}_x$ est négligeable devant $B_0$ alors la moyenne de pottenstelle presente une partie transverce et longitudinale decouplés : $\braket{V} = V_\perp ( y , z ) + V_\parallel(x) $.
\paragraph{Découplage des confinements transverses et longitudinaux.}
Les courants qui parcourent les fils $D$, $D'$, $d$ et $d'$ sont orientés selon $\vec{e}_u$. 
Les champs magnétiques induits possèdent alors une composante selon $\vec{e}_x$, notée $B_\parallel^x$, et une composante selon $\vec{e}_v$ (axe normal à la puce), notée $B_\parallel^v$. 
Si le champ selon $\vec{e}_x$ est négligeable devant $B_0$, alors le potentiel moyen se sépare en une partie transverse et une partie longitudinale découplées : 
\(
\braket{V} = V_\perp(y,z) + V_\parallel(x) .
\)


\paragraph{Potentiel longitudinal harmonique.}
Dans la configuration où seuls les fils $D$ et $D'$ sont utilisés, le potentiel longitudinal peut, à l’ordre 2 en $x$, être considéré comme harmonique :
\begin{eqnarray*}
	V_\parallel (x) = V_0 + \frac{1}{2} m \omega_\parallel^2 x^2 ,
\end{eqnarray*}
On note  $2L=1.89 \,mm$ est la distance séparant les fils $D$ et $D'$. Les courants circulant dans ces deux fils sont identiques et notés $I_D = I_{D'}$. Si la condition $B_0 \gg \mu_0 I_D d /(\pi L)^2 $ est vérifiée, alors le terme constant du potentiel vaut approximativement $V_0 \simeq \mu_B B_0$.

\medskip

La pulsation longitudinale totale $\omega_\parallel$ se décompose en deux contributions : (i) une pulsation $\omega_\parallel^x = \sqrt{\frac{6\, d \, \mu_B \, \mu_0 \,I_D }{\pi \, L^4 \, m}}$ induite par le champ longitudinal $B_\parallel^x$ et (ii) une pulsation $\omega_\parallel^v = \sqrt{\frac{\mu_B }{m \, B_0}}\frac{\mu_0 \, I_D }{\pi \, L^2}$ liée au champ  $B_\parallel^v$. Pour des courants $I>1A$ , on a $\omega_\parallel^v \gg \omega_\parallel^x$, et ainsi :  
\begin{eqnarray*}
	\omega_\parallel \propto \frac{I_D}{\sqrt{B_0} L^2}.
\end{eqnarray*} 
La fréquence longitudinale est donc réglée expérimentalement en ajustant $I_D$.

\medskip

Avec les dimensions caractéristiques de la puce et des fils, il est possible d’atteindre des confinements longitudinaux de fréquence $f_\parallel = \omega_\parallel/ 2 \pi$ allant jusqu’à $\sim 150 \, H_z$, la limite étant imposée par le chauffage des fils pour $I_D \leq =4 \, A$.
 
 \medskip
 
 \subparagraph{Mesure de la fréquence transverse et longitudinale}
Pour la caractérisation, la pulsation transverse $\omega_\perp$ a été mesurée par la méthode du mode de respiration transverse \cite{Kagan1996}, tandis que $\omega_\parallel$ a été obtenue à partir des oscillations dipolaires longitudinales. Les détails expérimentaux de ces méthodes figurent dans le manuscrit de thèse de Léa Dubois \cite{L.Dubois2024}, p. 73 et p. 78.

\medskip

 \paragraph{Potentiel longitudinal quartic.}
 Si on ajoute du courand  dans les fils $d$ et $d'$.  Alors on peux avoir un potentiel non gégligeable à l'ordre 4 . Pour simmplifier, les courants dans ces fils $I_d$ et $I_{d'}$ sont identique. et le potentiel s'écrit : 
 \paragraph{Potentiel longitudinal quartique.}
Si l’on ajoute un courant dans les fils $d$ et $d'$, on peut générer un potentiel longitudinal comportant un terme significatif à l’ordre 4 en $x$ . Pour simplifier, on suppose $I_d=I_{d'}$. On obtient alors : 
 \begin{eqnarray*}
 	V_\parallel(x) \, = \, \mu_B B_0  & + & 	 \frac{\mu_B \, \mu_0}{\pi} d  \left [ \frac{I_D}{L^2} + \frac{I_d}{l^2} + 3 \left ( \frac{I_D}{L^4} + \frac{I_d}{l^4} \right ) x^2  +  5 \left ( \frac{I_D}{L^6} + \frac{I_d}{l^6} \right ) x^4 \right ] \\
 	& + & \frac{\mu_B}{B_0} \left ( \frac{\mu_0}{\pi} \right )^2  \left [ \left ( \frac{I_D}{L^2} + \frac{I_d}{l^2} \right ) x^2  + 2 \left ( \frac{I_D}{L^2} + \frac{I_d}{l^2} \right )\left ( \frac{I_D}{L^4} + \frac{I_d}{l^4} \right ) x^4 \right ].
 \end{eqnarray*}
 
 En ajustant $I_D$ et $i_d$, on peut réaliser par exemple un double puits \cite{Schemmer2019}, ou bien supprimer le terme quadratique $x^2$ afin d’obtenir un potentiel quartique pur :
\begin{eqnarray*}
	V_\parallel(x) = a_0 + a_4 x^4 	
\end{eqnarray*}
comme on le fais dans \cite{Dubois2025}.

En pratique, la puce présente des dimensions finies et n’est pas parfaitement symétrique. Un calcul plus précis, prenant en compte la géométrie exacte (disposition et épaisseur des fils), est présenté en annexe de la thèse de Thibault Jacqmin \cite{???}, p. 151. Cela impose un ajustement fin et asymétrique des courants $I_D$, $I_{D'}$, $I_d$ et $I_{d'}$.


On ajuste les courant $I_D$ et $i_d$ pour par exemple fais des douple puit \cite{Schemmer2019} ou en supriment le terme en $x^2$  d’obtenir un potentiel longitudinal quartique de la forme $V_\parallel(x) = a_0 + a_4 x^4$ \cite{Dubois2025}.\\

En réalité la puce presente des dimention finie, Un calcul plus précis prenant en compte la géométrie exacte des fils (disposition sur la
puce, épaisseur finie) se trouve en appendice de la thèse de Thibault Jacqmin [112] , page 151. De plus la pude n'est pas pardetement symetrique donc on doit ajuster les courant $I_D$, $I_{D'}$, $I_d$ et $I_{d'}$.


\paragraph{Caractérisation des potentiels longitudinal et transverse.}
Pour atteindre le régime unidimensionnel, les confinements doivent être fortement anisotropes : un piégeage transverse très fort et un piégeage longitudinal faible. La condition \(\mu, k_B T \ll \hbar \omega_\perp\) garantit le gel des degrés de liberté transverses.

\medskip

Cette configuration est particulièrement adaptée pour obtenir des profils de densité homogènes, nécessaires à certaines expériences de transport. Le transfert des atomes du piège harmonique vers le piège quartique est réalisé de manière \emph{diabatique} (changement rapide du potentiel), car un transfert adiabatique entraîne des pertes importantes.

\paragraph{Caractérisation des potentiels longitudinal et transverse.}
Pour atteindre le régime unidimensionnel, les potentiels de piégeage doivent être très asymétriques : un confinement transverse fort et un confinement longitudinal faible. La fréquence transverse \(\omega_\perp\) doit être suffisamment élevée pour geler les degrés de liberté dans cette direction, avec la condition \(\mu, k_B T \ll \hbar \omega_\perp\).

%\paragraph{Potentiel longitudinal}
%
%Le confinement longitudinal est produit par des courants continus ou modulés dans certains fils. Dans certains protocoles spécifiques, on utilise un potentiel quartique \( V_\parallel(x) = c_4 x^4 \). Le système reste dans le régime 1D tant que la longueur caractéristique longitudinale reste beaucoup plus grande que la transverse.
%
%\paragraph{Potentiel transverse}
%
%Le confinement transverse est réalisé à l’aide de trois micro-fils parallèles situés sur la puce : un fil central parcouru par un courant \( I \), et deux fils latéraux par des courants opposés \(-I\). Cette configuration crée un piège transverse harmonique avec une fréquence \(\omega_\perp\) contrôlable par la valeur du champ \( B_0 \) et le courant. Les atomes sont piégés à environ \( d = 15~\mu\text{m} \) au-dessus de la puce. La fréquence maximale accessible expérimentalement est de l’ordre de \( \sim 100~\text{kHz} \).
%
%\paragraph{Effet de rugosité et suppression par modulation}
%
%La rugosité des micro-fils induit des fluctuations parasites du champ magnétique le long du guide. Pour supprimer cet effet, les courants sont modulés à haute fréquence (environ 400~kHz). Grâce à cette modulation rapide, les atomes ne ressentent que le potentiel moyen, dans lequel la composante parasite longitudinale du champ s’annule. Ce procédé permet d’obtenir un potentiel transverse régulier et stable, avec une fréquence efficace \[ f_\perp = \frac{f_\perp^{(0)}}{\sqrt{2}}. \]
%
%\paragraph{Découplage des confinements transverse et longitudinal.}
%Dans notre dispositif, le confinement transverse est assuré par les micro-fils modulés, tandis que le confinement longitudinal est généré par quatre fils extérieurs (D, D', d, d'). L’analyse du potentiel magnétique moyen montre que, sous l’hypothèse d’un champ de bobine homogène et dominant, les contributions transverse et longitudinale du potentiel sont découplées. Cette propriété est cruciale pour nos expériences : elle permet de modifier la géométrie du potentiel longitudinal sans perturber le confinement transverse, facilitant ainsi l’exploration de différentes configurations dynamiques.
%
%\paragraph{Piégeage longitudinal harmonique.}
%Un piège longitudinal harmonique est réalisé en appliquant des courants égaux dans les fils D et D', disposés de manière symétrique. Le champ magnétique longitudinal produit conduit à un potentiel quadratique local :
%\[
%V_\parallel(x) = V_0 + \frac{1}{2} m \omega_\parallel^2 x^2,
%\]
%avec une fréquence $\omega_\parallel$ contrôlée par le courant et la géométrie de la puce. En pratique, des fréquences jusqu’à 150 Hz sont atteintes pour des courants de 4 A. Une correction peut être nécessaire pour prendre en compte un champ magnétique résiduel $B_{0v}$, responsable d’un déplacement du centre du nuage atomique.
%
%\paragraph{Piégeage longitudinal quartique.}
%L’ajout de deux fils supplémentaires (d et d') permet de modifier la forme du potentiel longitudinal jusqu’à l’ordre 4. En ajustant les courants dans les quatre fils, on peut annuler le terme quadratique et obtenir un potentiel quartique :
%\[
%V_\parallel(x) = a_0 + a_4 x^4.
%\]
%Cette configuration est particulièrement adaptée pour générer des profils de densité homogènes, comme requis dans certaines expériences de transport. Le transfert des atomes du piège harmonique vers le piège quartique est réalisé de manière diabatique (changement rapide du potentiel), car un transfert adiabatique entraînait des pertes importantes.



\section{Sélection spatiale avec DMD}
\subsection{Motivation et principe}
{\color{blue}
\begin{itemize}
    \item Besoin de préparer des tranches homogènes.
    \item Intérêt dans les protocoles hors équilibre.
\end{itemize}
}

\paragraph{Objectif du dispositif de sélection}

L’outil de sélection spatiale a été conçu pour permettre une action locale sur le gaz atomique. Il présente deux objectifs principaux. D’une part, il permet de mesurer la distribution de rapidité localement résolue, en sélectionnant une tranche du gaz avant de la libérer et de suivre son expansion. D’autre part, il offre la possibilité de créer des situations hors équilibre en retirant une partie du gaz à l’équilibre, ce qui perturbe la configuration initiale et initie une dynamique.

\paragraph{Intérêt pour les protocoles hors équilibre}

Ce dispositif permet ainsi de générer des protocoles analogues à des configurations classiques comme le pendule de Newton, ou de sonder directement la dynamique d’un gaz de Lieb-Liniger dans des conditions contrôlées. Il constitue une brique essentielle pour les expériences de dynamique et de transport quantique.


\subsection{Mise en place technique (initiée par Léa Dubois)}

{\color{blue}
\begin{itemize}
    \item Dispositif optique de projection.
    \item Contrôle numérique des motifs.
    \item Calibration et stabilité.
\end{itemize}
}

\paragraph{Principe de sélection par pression de radiation}

La sélection repose sur l’illumination d’une zone définie du gaz avec un faisceau quasi-résonant avec la transition cyclique \( F=2 \rightarrow F'=3 \) de la ligne D2 du rubidium. Les atomes subissent une pression de radiation due aux cycles absorption/émission spontanée, ce qui les pousse hors du piège ou les amène dans un état non piégé.

\paragraph{Façonnage spatial du faisceau}

La sélection doit être spatialement résolue. Le profil d’intensité dans le plan des atomes est de type binaire :
\[
I(x) = 
\begin{cases}
0 & \text{si } x \in [x_1, x_2] \\
I_0 & \text{sinon}
\end{cases}
\]
ce qui permet de préserver ou d’éjecter les atomes selon leur position longitudinale.

\paragraph{Utilisation du DMD}

Pour générer ce profil, un DMD (Digital Micromirror Device) est utilisé. Il s’agit d’une matrice de \(1024 \times 768\) micro-miroirs orientables individuellement (±12°). En inclinant ces miroirs, on contrôle localement la réflexion de la lumière. L’image du DMD est projetée directement sur le plan des atomes, en imagerie directe.

\paragraph{Avantages du DMD}

Le DMD permet une reconfiguration rapide et programmable du motif de lumière. Cette technologie est largement utilisée dans les expériences d’atomes froids pour produire des potentiels structurés, homogénéiser un faisceau ou adresser localement les atomes.

\paragraph{Alternatives possibles}

Il est possible, en théorie, d’atteindre un effet similaire par un transfert cohérent des atomes vers un état anti-piégé via un pulse micro-onde ou une transition Raman. Cependant, la méthode par pression de radiation est plus simple à mettre en œuvre et adaptée à nos objectifs expérimentaux.

\paragraph{Principe de l’expulsion par pression de radiation}

Un atome illuminé par un faisceau proche de la résonance peut être expulsé du piège soit par transition vers un état anti-piégé, soit par effet de pression de radiation. Cette dernière génère une accélération suffisante pour fournir une énergie cinétique supérieure à la profondeur du puits magnétique. Le nombre de photons diffusés nécessaire peut être estimé à partir de la conservation de l’impulsion : une vingtaine de photons suffisent typiquement à extraire un atome du piège dans nos conditions.

\paragraph{Modèle de diffusion et estimation du seuil}

Le taux de diffusion de photons est modélisé à l’aide d’un taux \(\Gamma_{\mathrm{sc}}\), dépendant de l’intensité \(I\), de l’intensité de saturation \(I_{\mathrm{sat}}\), d’un paramètre \(\alpha\) (lié à la polarisation et au champ magnétique) et du désaccord \(\delta\). À résonance, et pour un temps d’illumination \(\tau_p\), on peut estimer le nombre total de photons diffusés par atome par \(N_{\mathrm{sc}} = \tau_p \Gamma_{\mathrm{sc}}\).

\paragraph{Mesures expérimentales de la puissance nécessaire}

La puissance minimale nécessaire pour éjecter tous les atomes d’une zone illuminée est déterminée en fixant un temps d’illumination donné, puis en variant l’intensité du faisceau. L’analyse est réalisée après un délai d’attente de \(\sim 10\) ms, pour s’assurer que seuls les atomes encore piégés soient détectés. Il est observé que 99$\%$ des atomes sont retirés à partir d’un rapport \(I/I_{\mathrm{sat}} \simeq 0.12\).

\paragraph{Mesures de photons diffusés par fluorescence}

La quantité de photons diffusés est également mesurée par l’analyse du signal de fluorescence capté par la caméra. En calibrant le rapport entre photons détectés et photons diffusés (en tenant compte de l’efficacité optique du système), le nombre moyen de photons nécessaires pour éjecter un atome est confirmé expérimentalement autour de 20. Un ajustement du modèle de diffusion permet d’estimer le paramètre \(\alpha \simeq 0.4\).

\paragraph{Saturation et effets Doppler}

À fort temps d’illumination (\(\tau_p > 150\,\mu\)s), une saturation du nombre de photons diffusés est observée, interprétée comme un effet géométrique : les atomes accélérés atteignent physiquement la puce atomique et cessent de contribuer au signal. Une correction Doppler peut être introduite dans le modèle, mais reste négligeable (\(< 5\%\)) dans les régimes expérimentaux utilisés.

\paragraph{Limitations expérimentales de la sélection}

Plusieurs effets peuvent limiter l'efficacité ou la propreté de la sélection :
\begin{itemize}
    \item La diffraction liée à la taille finie de l’objectif entraîne un flou de l’ordre de \(1{-}2\,\mu\)m au bord des zones éclairées.
    \item Une diffusion parasite par la puce peut se produire à forte intensité si tout le DMD est illuminé ; cela est évité en réduisant la taille transverse du faisceau à quelques micro-miroirs seulement.
    \item Des inhomogénéités d’éclairement dues à la gaussienne du faisceau et au speckle peuvent conduire à une sur-illumination de certaines zones. Un effort a été fait pour homogénéiser l’intensité en sortie de fibre.
    \item La réabsorption des photons diffusés pourrait entraîner un échauffement du gaz restant. Un désaccord en fréquence de 15 MHz a été testé pour éviter ce phénomène, sans effet visible sur la température du gaz.
\end{itemize}

\paragraph{Mesures de l’impact sur le gaz restant}

La température du gaz sélectionné est comparée avant et après sélection via l’analyse des fluctuations de densité après temps de vol. Aucun changement significatif de température ni d’élargissement n’a été observé. Ces résultats suggèrent que, dans les conditions expérimentales utilisées, la sélection ne perturbe pas significativement les atomes restants.




\subsection{Utilisation dans les protocoles}

{\color{blue}
\begin{itemize}
    \item Formes utilisées : boîtes, barrières, coupures.
    \item Préparation initiale contrôlée du gaz.
    \item Exemples de protocoles expérimentaux utilisant le DMD
\end{itemize}
}

\paragraph{Sélection locale et mesure de rapidité}

En sélectionnant une tranche du gaz, on peut ensuite couper le confinement longitudinal et laisser cette tranche s’étendre. Le profil de densité asymptotique obtenu après un long temps d’expansion est proportionnel à la distribution de rapidité locale du gaz initial. Ce protocole permet ainsi une mesure résolue de \(\rho(x,t \to \infty) \sim \rho(v)\).

\paragraph{Génération d’états hors équilibre}

La sélection permet également de créer des discontinuités dans le profil de densité, et donc d’initier une dynamique hors équilibre. Par exemple, on peut ne conserver que deux paquets séparés de gaz, qui vont alors osciller l’un vers l’autre. Cette configuration est analogue à un pendule de Newton quantique.

\paragraph{Formes utilisées}

Les motifs projetés par le DMD peuvent prendre différentes formes : boîtes, barrières, coupures, etc. Cette flexibilité rend l’outil extrêmement précieux pour explorer diverses configurations initiales et protocoles dynamiques.

\paragraph{Contrôle logiciel du DMD}

Le pilotage du DMD repose sur l’utilisation d’un module intégré fourni par Vialux (V7001-SuperSpeed), qui comprend les bibliothèques logicielles ALP-4. Plusieurs configurations du DMD peuvent être chargées en mémoire au début de chaque cycle expérimental, puis sélectionnées en cours de séquence à l’aide d’un signal digital. Le temps de commutation des miroirs est inférieur à \(30\,\mu\mathrm{s}\), ce qui est compatible avec les protocoles étudiés.

\paragraph{Partage du faisceau avec la voie d’imagerie}

Le faisceau utilisé pour la sélection spatiale est prélevé à partir du faisceau sonde déjà accordé sur la transition \(F=2 \rightarrow F'=3\) de la raie D2. Le partage est réalisé à l’aide d’un cube séparateur de polarisation placé en aval d’une lame demi-onde, permettant de contrôler la puissance injectée dans la fibre optique. Ce choix simplifie la mise en œuvre en évitant d’ajouter une source laser supplémentaire.

\paragraph{Blocage du faisceau de sélection}

Deux systèmes permettent de couper le faisceau de sélection pendant le cycle expérimental :
\begin{itemize}
    \item un cache mécanique (type électro-aimant), utilisé pour un blocage longue durée ;
    \item un modulateur acousto-optique (AOM), permettant de produire des impulsions brèves de quelques dizaines de \(\mu\mathrm{s}\), en amont du séparateur.
\end{itemize}
Pour garantir que le faisceau ne perturbe pas l’imagerie, le cache mécanique reste fermé pendant l’utilisation du faisceau sonde.

\paragraph{Montage optique de projection}

Le faisceau façonné par le DMD est projeté dans le plan des atomes à l’aide d’un système optique permettant de sélectionner l’ordre 0 de diffraction. L’ensemble des optiques est dimensionné (diamètre \(50\,\mathrm{mm}\)) pour limiter la diffraction. L’alignement est effectué en superposant le faisceau de sélection à la voie d’imagerie.

\paragraph{Grandissement et champ couvert}

Le montage permet de couvrir une zone de l’ordre de \(600\,\mu\mathrm{m}\) dans le plan des atomes, soit plus que la longueur typique d’un nuage (\(\sim 400\,\mu\mathrm{m}\) pour \(f_{\parallel}=5\,\mathrm{Hz}\)). Le grandissement est déterminé par les focales utilisées : une focale \(f_1 = 750\,\mathrm{mm}\) du côté du DMD, et \(f = 32\,\mathrm{mm}\) pour l’objectif côté atomes, donnant \(G = f/f_1 \approx 0.043\).

\paragraph{Visualisation et interface}

Le contrôle du DMD s’effectue via une interface graphique permettant de prévisualiser les configurations de miroirs. Une capture d’écran de cette interface est présentée dans la Fig.~\ref{fig:dmd_interface}, où la zone active réfléchie est visualisée en rouge. Cette interface est pilotée de manière automatisée pendant le déroulement de la séquence expérimentale.


\section{Techniques d’imagerie et d’analyse}
\subsection{Imagerie par absorption}
{\color{blue}
\begin{itemize}
    \item Imagerie \textit{in situ} et après temps de vol.
    \item Résolution, limites instrumentales.
\end{itemize}
}

\paragraph{Système d’imagerie par absorption}

L’imagerie est réalisée à l’aide d’une caméra CCD à déplétion profonde, optimisée pour une grande efficacité quantique à la longueur d’onde de 780 nm. On utilise des techniques d’imagerie par absorption permettant d’extraire la densité optique \( D(x, z) \), elle-même reliée à la densité atomique 3D via la loi de Beer-Lambert. Le profil de densité linéaire \( n(x) \) est obtenu par intégration sur les directions transverses.

\paragraph{Imagerie après temps de vol}

En appliquant un champ magnétique vertical (\( B = 8\,\mathrm{G} \)), la polarisation du faisceau peut être rendue circulaire (\( \sigma^+ \)) pour adresser la transition fermée \( |F=2, m_F=2\rangle \rightarrow |F'=3, m_F'=3\rangle \). Cette configuration assure une meilleure définition de la section efficace d’absorption. Un temps de vol de quelques ms est utilisé avant l’imagerie, permettant également de décomprimer le nuage.

\paragraph{Imagerie in situ}

Sans champ magnétique, les atomes sont imagés à $7~\mu m$ de la puce, ce qui implique une double absorption du faisceau incident et réfléchi. Dans ce cas, la transition n’est pas fermée, ce qui nécessite une calibration du facteur de conversion entre la densité mesurée et la densité réelle. Un ajustement linéaire permet de relier les profils in situ aux profils obtenus après temps de vol.

\paragraph{Choix des paramètres d’imagerie}

L’intensité du faisceau sonde est choisie typiquement à \( I_0/I_{\mathrm{sat}} \approx 0.3 \) pour optimiser le rapport signal sur bruit tout en restant dans une zone de linéarité acceptable. Dans ces conditions, le nombre de photons diffusés est de l’ordre de \( N_{\mathrm{sc}} \approx 230 \) et le rayon de diffusion reste comparable à la résolution du système d’imagerie (\( \sim 2.6\,\mu \mathrm{m} \)).

\paragraph{Limites du modèle de Beer-Lambert}

La validité de la loi de Beer-Lambert repose sur une approximation à une particule. Dans le cas des gaz fortement denses ou quasi 1D, les effets collectifs, les réabsorptions et les couplages dipolaires peuvent invalider ce modèle. Pour cette raison, même pour l’imagerie in situ, un temps de vol court (\( \sim 1\,\mathrm{ms} \)) est souvent appliqué afin de diluer le gaz transversalement.

\paragraph{Défauts et instabilités expérimentales}

Plusieurs limitations instrumentales ont été identifiées :
\begin{itemize}
    \item La caméra initialement utilisée montrait des motifs parasites aléatoires ainsi qu’un offset variant au cours du temps. Le remplacement de la caméra a permis de résoudre ces problèmes.
    \item Des franges d’interférences apparaissaient lors de la division des images d’absorption, probablement dues à des effets Fabry-Pérot dans les optiques. Le désaxage du faisceau d’imagerie a permis d’en limiter l’impact.
    \item Des photons résiduels, même en l’absence de faisceau sonde, ont été détectés. Ces derniers proviennent vraisemblablement de diffusions multiples dans le système optique.
\end{itemize}

\paragraph{Conclusion}

La combinaison de l’imagerie in situ et après temps de vol, ainsi qu’une calibration soigneuse des paramètres optiques et expérimentaux, permettent d’accéder à des profils de densité fiables malgré les limites intrinsèques du système d’imagerie. Une attention particulière a été portée à la réduction des artefacts expérimentaux afin de garantir la précision des mesures.


\subsection{Analyse des profils}

{\color{blue}
\begin{itemize}
    \item Extraction des densités, tailles, températures.
    \item Distribution longitudinale.
    \item Estimation de la température par ajustement Yang-Yang (optionnel si pertinent).
\end{itemize}
}


\section{Expériences et protocoles étudiés}
Cette section peut être la plus personnelle, en précisant ton rôle à chaque fois.
\subsection{Expansion longitudinale}
\begin{itemize}
    \item Protocole d’expansion (libération longitudinale, maintien du confinement transverse).
    \item Suivi de l’évolution du profil.
    \item Analyse à différents temps d’expansion
    \item Comparaison aux modèles analytiques : solutions homothétiques, GP, asymptotiques.
\end{itemize}

\subsection{Motivation et protocole expérimental d’expansion longitudinale}

\paragraph{Motivation.}
Une partie essentielle de mon travail de thèse a consisté à sonder la distribution de rapidités résolue spatialement, ce qui constitue une information clé pour comprendre la dynamique hors équilibre d’un gaz quantique unidimensionnel. Pour accéder à cette observable, il est nécessaire de réaliser un protocole qui relie la distribution de rapidités à des profils de densité mesurables expérimentalement. L’expansion longitudinale dans le guide 1D s’impose alors comme un outil naturel : en laissant le nuage se dilater librement dans la direction longitudinale, on convertit en partie l’information contenue dans les phases et les excitations collectives du système en une dynamique de densité directement accessible par imagerie. Ce protocole permet ainsi de comparer les prédictions issues des équations effectives, comme l’équation de Gross–Pitaevskii dans différents régimes de confinement, avec des mesures expérimentales résolues spatialement.

\paragraph{Considérations physiques.}
Au-delà de son intérêt pratique, l’expansion longitudinale offre une fenêtre unique sur la physique des gaz bosoniques 1D. Elle permet d’étudier comment un système initialement confiné évolue vers un état dilué, révélant à la fois l’impact du régime transverse (TF 3D vs TF 1D) et l’influence des fluctuations de phase. Dans le régime TF 1D, ces fluctuations deviennent dominantes et se traduisent par des ondulations de densité mesurables. Leur analyse expérimentale, via le spectre de puissance, fournit un accès direct aux corrélations de phase et à la thermodynamique effective du gaz.

\paragraph{Protocole expérimental.}
Concrètement, l’expansion longitudinale est réalisée selon la séquence illustrée en Fig.~??? :
\begin{itemize}
  \item Le nuage est initialement piégé dans un potentiel magnétique caractérisé par une fréquence longitudinale $f_{\parallel} = 5.0$ ou $9.4\,\mathrm{Hz}$ selon les jeux de données, et une fréquence transverse $f_{\perp} = 2.56\,\mathrm{kHz}$.
  \item À $t=0$, le confinement longitudinal est éteint en annulant les courants $I_D=I_{D'}=0$. La coupure est réalisée sur un temps fini $t_{\parallel} = 70\,\mu\mathrm{s} \ll 1/f_{\parallel}$, ce qui évite un pic de courant parasite tout en préservant la dynamique du gaz.
  \item Le nuage se dilate librement dans la direction longitudinale pendant une durée $\tau$. Ensuite, le confinement transverse est relâché en annulant $I_{\perp}$, avec un temps de coupure $t_{\perp} = 5\,\mu\mathrm{s} \ll 1/f_{\perp}$.
  \item Une image par absorption est enfin prise après un temps de vol $t_v$. Pour l’étude des profils de densité, on utilise typiquement $t_v = 1\,\mathrm{ms}$.
\end{itemize}

%\paragraph{Découplage des confinements.}
%Comme discuté en Section~\ref{chap:...}, l’architecture expérimentale rend ce protocole particulièrement simple à mettre en œuvre. La modulation des courants transverses $I_{\perp}$ garantit que le potentiel longitudinal est découplé de celui transverse, ce qui permet un contrôle précis et indépendant des deux confinements.

\paragraph{Perspective.}
La mise en œuvre de ce protocole d’expansion longitudinale ne répond donc pas seulement à un besoin technique de mesure, mais s’inscrit dans une stratégie plus générale : relier les prédictions théoriques de la GHD et des modèles effectifs à des observables accessibles, et sonder directement l’évolution des fluctuations et des corrélations dans un système quantique 1D.

\paragraph{Équations Gross-Pitaevskii dépendantes du temps.}
La dynamique du système étudié est décrite par l’équation de Gross-Pitaevskii (GP) \eqref{chap.1:eq.GP.1} :
\begin{eqnarray*}
	i \partial_\tau\phi = \left \{ - \frac{1}{2}\Delta_{\vec{r}} + V(\vec{r}) + g_{\mathrm{3D}} N \vert \phi \vert^2 \right \} \phi,
\end{eqnarray*}
avec $g_{\mathrm{3D}} = 4 \pi a_{\mathrm{3D}}$ et en présence d’un potentiel externe (voir \eqref{} et \eqref{}) :
\begin{eqnarray*}
	V(\vec{r}) = V_\perp(\vec{r}_\perp) + V_\parallel(x), 
	\qquad 
	V_\perp(\vec{r}_\perp) = \tfrac{1}{2} \, \omega_\perp^2 \, \vec{r}_\perp^2, 
	\qquad 
	V_\parallel(x) = \tfrac{1}{2} \, \omega_\parallel^2 \, x^2.
\end{eqnarray*} 


\paragraph{Séparation des degrés de liberté.}
Dans un piège de type cigare, caractérisé par $\omega_\perp \gg \omega_\parallel$, la dynamique transverse se déroule sur des temps caractéristiques beaucoup plus courts que la dynamique longitudinale. On fait alors l’hypothèse d’un \emph{suivi adiabatique transverse} : l’état reste en permanence dans son état fondamental transverse. Ainsi, les degrés de liberté transverses et longitudinaux se découplent et la fonction d’onde peut se factoriser sous la forme
\begin{equation}
    \phi(r,\tau) = \psi(x,\tau)\,\Phi\!\left(\vec{r}_\perp, n(x,\tau)\right),
\end{equation}
où $\psi(x,\tau)$ décrit la dynamique longitudinale et $\Phi$ est la fonction d’onde transverse dépendant paramétriquement de la densité linéaire $n(x,\tau)$. La condition de normalisation 
\(
\int d \vec{r}_\perp \, \big|\Phi\!\left(\vec{r}_\perp, n\right)\big|^2 = 1
\)
permet de réécrire la densité linéaire définie par 
\(
n \doteq N \int d \vec{r}_\perp \, |\phi|^2
\)
sous la forme
\begin{eqnarray*}
	n(x,\tau) = N \, |\psi(x,\tau)|^2.
\end{eqnarray*}
L’équation de Gross-Pitaevskii se réécrit alors
\begin{eqnarray}
	\left( i \partial_\tau + \tfrac{1}{2} \partial_x^2 - V_\parallel(x) - \mu(n) \right) \psi = 0, 
	\qquad 
	\mu(n)\,\Phi = \left( - \tfrac{1}{2} \Delta_{\vec{r}_\perp} + V_\perp + g_{\mathrm{3D}} \, n \, \big|\Phi(\vec{r}_\perp,n)\big|^2 \right)\Phi.
\end{eqnarray}

\paragraph{Équations hydrodynamiques.}
En utilisant la transformation de Madelung 
\(
\psi(x,\tau) = \sqrt{n(x,\tau)} \, e^{i \vartheta(x,\tau)},
\)
et en introduisant la vitesse $u = \partial_x \vartheta$, on obtient les équations hydrodynamiques associées :
\begin{eqnarray}\label{chap:5:eq.hydro.1}
	\left\{
	\begin{array}{rcl}
		\partial_\tau n + \partial_x ( n u )	 & = & 0, \\[0.3em]
		\partial_\tau u + \partial_x \left( \tfrac{u^2}{2} + V_\parallel(x) + \mu(n) + Q(n) \right) & = & 0,
	\end{array} 
	\right.
\end{eqnarray}
où le terme de pression quantique est donné par
\(
Q(n) = - \frac{1}{2} \, \frac{\partial_x^2 \sqrt{n}}{\sqrt{n}}.
\)
Ces équations sont équivalentes aux deux premières de \eqref{chap:3:eq:hydro.1}, en tenant compte de la relation thermodynamique $dP = n \, d\mu$ et en négligeant le terme de pression quantique $Q(n)$.

\medskip

Pour notre protocole, pour $\tau < 0$ le système est à l’équilibre, avec la condition
\(
\mu(n) + V_\parallel(x) = \mu\bigl(n(x=0)\bigr).
\)
Pour $\tau \geq 0$, le potentiel longitudinal est éteint : $V_\parallel(x) = 0$.

\medskip

\paragraph{Solutions analytiques homothétique.}
Si $n$ est solution des equation hydrodynamique \eqref{chap:5:eq.hydro.1} , pour $\tau \geq 0$. On fais l'hypothèse que la densité linéaire suit une forme homothétique
\begin{eqnarray}
	n(x,\tau) = \frac{1}{\lambda(\tau)} n_0 \left ( \frac{x}{\lambda(\tau)} \right ) ,	
\end{eqnarray}
avec $n_0$ le profil de densité à $\tau = 0 $ et $\lambda(\tau)$ le facter d'echelle à une temps d'expension $\tau$. Avec les containtes $\lambda(0) = 1$ et $\lambda'(0) = 0$ et $N = \int dx \, n(x , \tau ) $. En injectant dans \eqref{chap:5:eq.hydro.1} il vient que 
\begin{eqnarray}\label{chap:5:eq.hydro.2}
	\left\{
	\begin{array}{rcl}
		u(x, \tau ) & = & \displaystyle \frac{\dot\lambda(\tau)}{\lambda(\tau)} x , \\[0.3em]
		\partial_x \mu ( n ( x , \tau ))  & = & - \displaystyle \frac{\ddot\lambda(\tau)}{\lambda(\tau)} x,
	\end{array} 
	\right.
\end{eqnarray}
(car \(\partial_\tau u=(\ddot\lambda/\lambda-\dot\lambda^2/\lambda^2)x\) et \(v\partial_x v=(\dot\lambda/\lambda)^2 x\), leur somme donne \((\ddot\lambda/\lambda)x\)) et initialement $\mu( n_0 ( x ) ) = \mu( n_0 ( x = 0  ) ) - \frac{1}{2} \omega_\parallel^2 x^2 $.

\medskip

Calculons maintenant \(\partial_x\mu(n(x))\). D'abord
\[
\partial_x n(x)=\frac{1}{\lambda^2}\,n_0'\!\Big(\frac{x}{\lambda}\Big).
\]
À l'équilibre \(\mu\big(n_0(y)\big)=\mu_0-\tfrac12 \omega_\parallel^2 y^2\), d'où
\[
\mu'(n_0(y))\,n_0'(y)=-\omega_\parallel^2 y
\quad\Rightarrow\quad
n_0'(y)=-\frac{\omega_\parallel^2\,y}{\mu'(n_0(y))}.
\]
En prenant \(y=x/\lambda\) on obtient
\[
n_0'\!\Big(\frac{x}{\lambda}\Big)
= -\frac{\omega_\parallel^2}{\lambda}\,\frac{x}{\mu'\big(n_0(x/\lambda)\big)}.
\]
Donc
\[
\partial_x n(x) = -\frac{\omega_\parallel^2\,x}{\lambda^3}\;
\frac{1}{\mu'\big(n_0(x/\lambda)\big)}.
\]
Puis
\[
\partial_x\mu(n(x))=\mu'\big(n(x)\big)\,\partial_x n(x)
= -\frac{m\omega_\parallel^2\,x}{\lambda^3}\;
\frac{\mu'\big(n(x)\big)}{\mu'\big(n_0(x/\lambda)\big)}.
\]
Or \(n_0(x/\lambda)=\lambda\,n(x)\), donc on définit
\[
f(\lambda)\equiv\frac{\mu'(n)}{\mu'(\lambda n)}.
\]
On obtient finalement
\[
\partial_x\mu(n(x)) = -\frac{\omega_\parallel^2}{\lambda^3}\,f(\lambda)\,x.
\]

%On souhaite calculer $\partial_x \mu\bigl(n(x,\tau)\bigr)$ en utilisant la règle de la chaîne et la forme homothétique. On a
%\[
%	\partial_x \mu\bigl(n(x)\bigr) 
%	= \mu'(n(x)) \, \partial_x n(x) 
%	= \frac{1}{\lambda^2} \, \mu'(n(x)) \, \partial_x n_0\!\left(\tfrac{x}{\lambda}\right),
%\]
%où le dernier terme s’écrit
%\[
%	\left. \frac{\partial n_0}{\partial x} \right|_{x/\lambda} 
%	= \left. \frac{\partial n_0}{\partial \mu} \right|_{\mu(n_0(x/\lambda))} 
%	\left. \frac{\partial \mu}{\partial x} \right|_{n_0(x/\lambda)} .
%\]
%On utilise alors 
%\[
%	\left. \frac{\partial \mu}{\partial x} \right|_{n_0(x/\lambda)} = - \frac{\omega_\parallel^2}{\lambda^2} \, x,
%	\qquad 
%	\left. \frac{\partial n_0}{\partial \mu} \right|_{\mu(n_0(x/\lambda))} 
%	= \left. \frac{\partial n}{\partial \mu} \right|_{\mu(\lambda n)} .
%\]
%Il vient donc, en utilisant de plus la deuxième équation de 
En remplaçant dans la deuxième d'Euler \eqref{chap:5:eq.hydro.2} et en simplifiant  \(x\),
\begin{eqnarray}\label{chap:5:eq.hydro.3}
	\frac{\ddot\lambda}{\lambda}  
	 =  \frac{\omega_\parallel^2}{\lambda^3} \, f(\lambda)  .
\end{eqnarray}

\paragraph{Proposition.}
Si le facteur
\(
f(\lambda)
\)
est bien défini indépendamment de \(n>0\) (ce qui est le cas pour les solutions homothétiques),
alors \(f\) est une loi de puissance.

\paragraph{Preuve.}
Posons \(g(n) = \mu'(n)>0\) ou \(<0\) (\ie $\mu$ strictement monotone). La définition de \(f\) équivaut à l’existence d’une fonction
\(\chi(\lambda)=1/f(\lambda)\) telle que
\[
g(\lambda n)=\chi(\lambda)\,g(n)\qquad(\forall\,\lambda,n>0).
\]
En prenant \(n=1\), on a \(\chi(\lambda)=g(\lambda)/g(1)\).
Donc, pour tous \(a,b>0\),
\[
\chi(ab)=\frac{g(ab)}{g(1)}=\frac{\chi(a)\,g(b)}{g(1)}=g(a)\,g(b),
\]
c’est-à-dire que \(\chi\) est \emph{multiplicative}. Sous une hypothèse physique très faible
(continuité/mesurabilité ou simple localement bornée), toute fonction multiplicative sur
\(\mathbb{R}_+^\ast\) est de la forme
\[
\chi(\lambda)=\lambda^{\alpha-1}
\quad\Rightarrow\quad
f(\lambda)=\lambda^{1-\alpha}.
\]
%En réintégrant \(g=\mu'\propto n^{\alpha-1}\), on retrouve \(\mu(n)\propto n^\alpha\) pour \(\alpha\neq 0\)
%(et \(\mu(n)\propto \ln n\) pour \(\alpha=0\)).
\qed


% --- Démonstration que f(λ)=λ^{1-\alpha} et réciproque ---
\paragraph{Proposition.}
$f(\lambda) = \lambda^{1-\alpha}$ et $\mu (n) \propto n^\alpha $ sont equivalents.
%
%On définit
%\[
%f(\lambda)=\frac{\mu'(n)}{\mu'(\lambda n)},
%\]
%en supposant \(\mu\in C^1\) et \(\mu'(n)>0\) pour \(n>0\).

\paragraph{1. Si \(\mu(n)=C\,n^\alpha\) (avec \(C\neq0\)) :}
Alors \(\mu'(n)=C\alpha\,n^{\alpha-1}\). Par conséquent
\[
f(\lambda)=\frac{C\alpha\,n^{\alpha-1}}{C\alpha\,(\lambda n)^{\alpha-1}}
=\lambda^{1-\alpha}.
\]

\paragraph{2. Réciproque : si \(f(\lambda)=\lambda^{1-\alpha}\) pour tout \(\lambda>0\) (et tout \(n>0\)) :}
Posons \(g(n)=\mu'(n)\). L'hypothèse s'écrit
\[
\frac{g(n)}{g(\lambda n)}=\lambda^{1-\alpha}
\quad\Longleftrightarrow\quad
g(\lambda n)=\lambda^{\alpha-1}\,g(n),
\]
pour tout \(n>0\) et tout \(\lambda>0\).

Fixons \(n_0>0\) et définissons \(\varphi(\lambda)\equiv g(\lambda n_0)\). La relation ci-dessus donne
\[
\varphi(\lambda)=\lambda^{\alpha-1}\,\varphi(1).
\]
Autrement dit \(\varphi(\lambda)=C_1\,\lambda^{\alpha-1}\) pour une constante \(C_1=\varphi(1)=g(n_0)\). En remplaçant \(\lambda=x/n_0\) on obtient pour tout \(x>0\)
\[
g(x)=C_1\,x^{\alpha-1}.
\]
Ainsi \(g(n)=\mu'(n)=C\,n^{\alpha-1}\) avec \(C\) constant.

En intégrant (en supposant \(\alpha\neq 0\)),
%\[
%\mu(n)=\int \mu'(n)\,dn = \int C\,n^{\alpha-1}\,dn = \frac{C}{\alpha}\,n^\alpha + \text{const},
%\]
%donc 
\(\mu(n)\propto n^\alpha\). (Pour \(\alpha=0\) on obtient \(\mu'(n)=C\,n^{-1}\) et \(\mu(n)=C\ln n+\text{const}\).)

\paragraph{Remarque sur les hypothèses.}
La démonstration utilise la propriété fonctionnelle multiplicative
\(g(\lambda n)=\lambda^{\alpha-1}g(n)\). Sous une hypothèse faible de continuité (ou dérivabilité) en \(n\) cette équation force la forme de puissance \(g(n)\propto n^{\alpha-1}\). Sans régularité, des solutions pathologiques peuvent exister mais ne sont pas physiquement pertinentes dans le contexte thermodynamique.

\qed

%% --- Bref argument (à insérer) ---
%En posant \(g(n)=\mu'(n)\) et \(f(\lambda)=\dfrac{g(n)}{g(\lambda n)}=\lambda^{1-\alpha}\), on obtient
%\[
%g(\lambda n)=\lambda^{\alpha-1}g(n)\quad\forall\,\lambda,n>0,
%\]
%d'où \(g(n)=C\,n^{\alpha-1}\) et, pour \(\alpha\neq0\), \(\mu(n)=\dfrac{C}{\alpha}n^\alpha+\mathrm{const}\), i.e. \(\mu\propto n^\alpha\).
%
%
%% --- Encadré pour le cas alpha = 0 ---
%\medskip
%\noindent\textbf{Remarque (cas \(\alpha=0\)).} Si \(\alpha=0\) alors la relation fonctionnelle donne \(g(n)=\mu'(n)=C\,n^{-1}\). En intégrant on obtient
%\[
%\mu(n)=C\ln n + \mathrm{const}.
%\]
%Ce cas correspond physiquement, par exemple, au gaz isotherme idéal en 1D (ou plus généralement à une dépendance logarithmique du potentiel chimique), où la compressibilité \(\mu'(n)\propto 1/n\).
%
%--------------------------------------------
%
%avec $f(\lambda) = \frac{\mu'(n)}{\mu'(\lambda n )}$ avec  $\mu(n)$ est continue et strictement monitone (donc inversible). Puisque $f(1)= 1$ et $f(\lambda_1 \lambda_2) =  f(\lambda_1) f( \lambda_2)$ alors $f$ est une fonction de puissace $f(\lambda) = \lambda^{1-\beta}$. Ainssi une solution de l'éqtation hydrondynamique homothètique donne  
%\begin{eqnarray}
%	\mu(n) = a n^\beta + b,  	
%\end{eqnarray}
%avec $a, b$ et $\beta$ des réelles. 

\paragraph{Cas particulier.}
Dans le régime quasi-1D on utilise l'expression d'interpolation (cf. Salasnich et al.)
\[
\mu(n)=\hbar\omega_\perp\Big(\sqrt{1+4\,a_{\mathrm{3D}}\,n}-1\Big),
\]
où \(n\) est la densité linéique et \(a_{\mathrm{3D}}\) le scattering length. De cette formule on obtient deux limites asymptotiques :

\begin{itemize}
\item \emph{Régime transverse Thomas--Fermi (TF), \(4a_{\mathrm{3D}}n\gg1\).} 
Alors \(\sqrt{1+4a_{\mathrm{3D}}n}\simeq 2\sqrt{a_{\mathrm{3D}}n}\) et
\[
\mu(n)\simeq 2\hbar\omega_\perp\sqrt{a_{\mathrm{3D}}\,n},
\]
ce qui correspond à \(\mu\propto n^{1/2}\) (donc \(\alpha=\tfrac12\)). Ce régime décrit la situation où \(\mu\gg\hbar\omega_\perp\) et de nombreux niveaux transverses sont excités.
\item \emph{Régime quasi-1D (transverse fondamental), \(4a_{\mathrm{3D}}n\ll1\).} 
Alors \(\sqrt{1+4a_{\mathrm{3D}}n}\simeq 1+2a_{\mathrm{3D}}n\) et
\[
\mu(n)\simeq 2\hbar\omega_\perp\,a_{\mathrm{3D}}\,n \equiv g\,n,
\]
avec \(g=2\hbar\omega_\perp a_{\mathrm{3D}}\). Ici \(\mu\propto n\) (donc \(\alpha=1\)) ; on est proche de l'état fondamental transverse (gaussien).
\end{itemize}

Les deux formes ci-dessus sont bien les limites asymptotiques de l'expression d'interpolation donnée plus haut.

\medskip

Enfin, l'équation d'évolution du facteur d'échelle obtenue précédemment s'écrit correctement
\[
\boxed{\qquad \ddot\lambda\,\lambda^{\alpha+1}=\omega_\parallel^2 \qquad}
\]


% --- Table révisée (petites corrections typographiques) ---
\begin{table}[h]
\centering
\begin{tabular}{l c c c}
\hline
Système & loi pour $\mu(n)$ & $\beta$ (avec $f(\lambda)=\lambda^{-\beta}$) & $\displaystyle \omega_{\rm breath}/\omega_\parallel$ \\
\hline
Gaz classique isotherme (1D, $\mu\propto\ln n$) 
& $\mu'(n)\propto 1/n$ 
& $-1$ 
& $\sqrt{3}\approx1.732$ \\[4pt]

Gaz de Bose 1D en régime moyen (GP, $\mu\propto n$) 
& $\alpha=1$ 
& $0$ 
& $2$ \\[4pt]

Tonks--Girardeau (1D, $\mu\propto n^2$) 
& $\alpha=2$ 
& $1$ 
& $\sqrt{5}\approx2.236$ \\[4pt]

Gaz de Fermi unitaire (ex. 3D, $\mu\propto n^{2/3}$) 
& $\alpha=\tfrac{2}{3}$ 
& $-\tfrac{1}{3}$ 
& $\sqrt{3+\tfrac{2}{3}}\approx1.915$ \\[4pt]

Cas général (loi de puissance) 
& $\mu\propto n^\alpha$ 
& $\beta=\alpha-1$ 
& $\displaystyle \sqrt{3+\alpha}$ \\
\hline
\end{tabular}
\caption{Valeurs de $\beta$ et fréquences du mode de souffle pour quelques régimes usuels.}
\label{tab:breathing}
\end{table}
 



\subsection{Sonde locale de distribution de rapidité}
\begin{itemize}
    \item Principe de la mesure : coupure d’une tranche puis expansion.
    \item Rôle du DMD dans la sélection.
    \item Accès à la distribution de vitesse locale.
    \item Comparaison avec les prédictions GHD.
    \item Limites et incertitudes
\end{itemize}

\section{Discussion sur les limites et les perspectives}
\begin{itemize}
    \item Contraintes techniques (bruit, alignement, stabilité de la puce…).
    \item Améliorations potentielles (résolution, contrôle du potentiel, automatisation).
    \item Perspectives pour d’autres types d’expériences (étude de chocs, turbulence quantique, etc.)
\end{itemize}

\section*{Conclusion}
\begin{itemize}
    \item Résumé de l’architecture du dispositif).
    \item Méthodes d’analyse utilisées et robustesse.
    \item Importance de l’expérience dans le contexte de l’étude des gaz quantiques unidimensionnels
\end{itemize}
Ce chapitre a présenté les éléments essentiels du dispositif expérimental, les méthodes d’imagerie, ainsi que les expériences auxquelles j’ai participé. L’ensemble constitue une plateforme performante pour l’étude de la dynamique de gaz 1D hors équilibre.

\paragraph{Résumé de l’architecture expérimentale}  
Nous avons décrit les éléments clés du dispositif utilisé : un système de refroidissement laser basé sur trois sources couplées, un piégeage magnétique sur puce optimisé pour réaliser des géométries unidimensionnelles, une plateforme de modulation de potentiel via un DMD, et un système d’imagerie haute résolution. L’ensemble permet une manipulation fine des nuages atomiques dans un cadre reproductible et stable.

\paragraph{Méthodes d’analyse et robustesse}  
L’imagerie par absorption, couplée à une analyse rigoureuse des profils atomiques, fournit des outils fiables pour extraire les grandeurs pertinentes : densités, tailles, températures, distributions de vitesses. Ces méthodes ont permis de confronter les résultats expérimentaux à des prédictions théoriques de type GHD ou Yang-Yang.

\paragraph{Importance du dispositif pour la thèse}  
Ce dispositif a été essentiel pour mener à bien les expériences présentées dans cette thèse. Il offre à la fois un contrôle local (grâce au DMD), un bon confinement transverse (grâce à la puce) et une imagerie précise. La plateforme est ainsi bien adaptée pour étudier des systèmes 1D fortement corrélés hors équilibre, et pour tester les prédictions de la physique statistique intégrable.

\paragraph{Perspectives}  
Malgré ses atouts, le dispositif présente des limitations techniques (rugosité magnétique, sensibilité à l’alignement, etc.) qui laissent entrevoir des pistes d’amélioration. Des développements futurs pourraient notamment viser à augmenter la résolution spatiale, automatiser davantage les séquences, ou explorer d'autres régimes dynamiques comme la turbulence ou les collisions de chocs quantiques.



%\appendix
\section*{Annexes}
\begin{itemize}
    \item Schémas techniques (puce, DMD, optique).
    \item Tableaux de paramètres expérimentaux.
    \item Exemples de motifs DMD utilisés.
\end{itemize}