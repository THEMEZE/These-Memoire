\chapter{Modèle de Lieb-Liniger et approche Bethe Ansatz}\label{chap:LL-BA}
\minitoc


%\section*{Introduction}
%
%Dans ce chapitre, nous introduisons progressivement le modèle de Lieb-Liniger et l'Ansatz de Bethe, outils fondamentaux pour décrire un gaz de bosons unidimensionnel avec interactions delta. L'objectif est d'accompagner pas à pas le lecteur depuis la formulation du problème quantique en champ de bosons jusqu'aux solutions exactes obtenues par l'Ansatz de Bethe.
%
%Pour des raisons pédagogiques, nous commençons par aborder le cas d’une seule particule, sans interaction, avec des condition aux bord périodique. Cela permet d’introduire naturellement les fonction d'onde à une particule, leur évolution sous l’action du Hamiltonien libre et l'équation qui reluste de la condition de limite periodique ( equation de bethe à une. particule) : la premier quantification.
%
%Puis nous continions par écrire l'équation du champ de bosons, exprimée à l’aide des opérateurs de création et d’annihilation en représentation de position : la seconde quantification. Cela permet de mettre au clair les operateur à un corps et à 2 corps.
%
%Une fois les notations bien établies, nous généralisons le raisonnement au cas de \(N\) particules, pour obtenir l’Hamiltonien de Lieb-Liniger complet ainsi que la forme générale de l’Ansatz de Bethe. Les solutions ainsi construites permettent non seulement de déterminer le spectre de l’Hamiltonien, mais aussi de calculer des observables physiques importantes, telles que l’impulsion totale ou le nombre de particules.
%
%Ensuite, nous étudions le cas de deux particules, cette fois en tenant compte de l’interaction locale. Ce qui nous permet de metre en évidence les consequences de des interaction ponctiels sur entre autre la continuté de la fonction d'onde et que l'on a pour une équatuion de Bethe à deux particule.%Cela nous amène à considérer les états de position dans le cas général, y compris lorsque les deux particules peuvent occuper la même position. Cette situation, bien plus subtile qu’il n’y paraît, met en évidence la complexité introduite par l’interaction, et justifie que l’on commence par analyser les configurations où les particules sont à des positions distinctes.
%
%%Dans le référentiel du centre de masse, le problème à deux corps avec interaction devient équivalent à un problème à une seule particule en interaction avec une barrière delta au centre. Cette reformulation permet d’interpréter l’effet de l’interaction comme une condition de raccord sur la fonction d’onde, tout en respectant la symétrie bosonique.
%
%%Nous revenons ensuite aux coordonnées du laboratoire afin d’introduire naturellement la forme des solutions imposée par l’Ansatz de Bethe. Cela nous conduit aux équations dites de Bethe, qui relient les quasimoments des particules à travers des conditions de périodicité modifiées par l’interaction.
%
%Ensuitre nous généralisont la fonction d'on à deux particule, à la fonction d'onde à N particule. Et les les quation de bethe à N particume
%
%Enfin, nous introduisons la notion de distribution de rapidité, outil essentiel dans l’étude des états d’énergie minimale (états fondamentaux) et au delà de l'éta fondamentale. Ce cadre servira de base aux développements ultérieurs sur les gaz bosons intégrables.

\section*{Introduction}

Ce chapitre est consacré à la présentation progressive du modèle de Lieb-Liniger et de l’Ansatz de Bethe, outils centraux pour la description d’un gaz de bosons unidimensionnel en interaction via un potentiel de type delta. L’objectif est d’accompagner rigoureusement le lecteur depuis la formulation quantique du système jusqu’aux solutions exactes obtenues par l’approche de Bethe.

\medskip

Nous commençons, pour des raisons pédagogiques, par le cas le plus simple : une particule libre, sans interaction, dans un espace unidimensionnel avec conditions aux bords périodiques. Cette première étape permet d’introduire naturellement les fonctions d’onde à une particule, leur évolution sous l’action du Hamiltonien libre, ainsi que la quantification résultant des conditions de périodicité — autrement dit, la version élémentaire des équations de Bethe.

\medskip

Nous passons ensuite à la formulation du problème en champ quantique, en exprimant le Hamiltonien en termes d’opérateurs de création et d’annihilation dans la représentation positionnelle : il s’agit du passage à la seconde quantification. Cette étape permet de formaliser clairement les termes à un corps et à deux corps dans l’Hamiltonien, et d’établir les notations qui seront utilisées tout au long du chapitre.

\medskip

Une fois ce cadre posé, nous généralisons le raisonnement au cas de \(N\) particules pour introduire le modèle complet de Lieb-Liniger. Nous présentons alors l’Ansatz de Bethe dans sa forme générale, qui fournit les états propres de l’Hamiltonien. Ce formalisme permet d’accéder explicitement au spectre du système, ainsi qu’à diverses quantités physiques telles que l’impulsion totale et le nombre de particules.

\medskip

Nous traitons d'abord le cas à seulement deux particules, cette fois en tenant compte de l’interaction locale. L’analyse de ce système met en lumière les effets de l’interaction ponctuelle sur la régularité de la fonction d’onde et les conditions de raccord, ainsi que sur les modifications des équations de Bethe. Ce cas constitue une étape clé vers la généralisation à \(N\) particules.

\medskip

La fonction d’onde est ensuite étendue au cas général de \(N\) particules, ce qui nous permet de dériver les équations de Bethe pour un système entièrement interactif. Ces équations encapsulent toute l’information sur les états propres du système.

\medskip

Enfin, nous introduisons la notion de \emph{distribution de rapidité}, concept fondamental pour la description des états dans la limite thermodynamique. Elle permet non seulement de caractériser les états d’énergie minimale (états fondamentaux), mais aussi d’analyser des configurations excitées au-delà de l’état fondamental. Ce formalisme constituera le socle des développements ultérieurs sur les propriétés thermodynamiques et dynamiques des gaz bosoniques intégrables.


\section{Description du modèle de Lieb-Liniger}

\subsection{Introduction au modèle de gaz de Bose unidimensionnel}% et Hamiltonien du modèle}

\subsubsection{De la première à la seconde quantification}

\paragraph{Introduction.}

La mécanique quantique se développe historiquement en deux grandes étapes : la \emph{première quantification}, aussi appelée quantification canonique, et la \emph{seconde quantification}. Comprendre ces deux cadres est essentiel pour aborder les systèmes quantiques complexes, en particulier ceux où le nombre de particules peut varier.

%La mécanique quantique s’est historiquement développée en deux étapes : la \emph{première quantification}, aussi appelée quantification canonique, puis la \emph{seconde quantification}. Comprendre ces deux cadres est essentiel pour aborder les systèmes à nombre de particules variable.


%\vspace{0.5cm}

\paragraph{Première quantification (quantification canonique, particule unique).}

La première quantification est la mécanique quantique standard, celle que vous avez rencontrée dès vos premiers cours. Elle consiste à quantifier un système classique décrit par des variables dynamiques telles que la position $x$ et la quantité de mouvement $p$. On procède en remplaçant ces variables par des {\bf opérateurs hermitiens} $\operator{x}$ et %$\operator{p}$
\begin{eqnarray}
	\operator{p} \doteq -i\hbar \operator{\partial}_x,	\label{chap.1.rapel.1}
\end{eqnarray}
où $\hbar$ est la constante de Planck réduite, satisfaisant la {\bf relation de commutation canonique} fondamentale $[\operator{x}, \operator{p}] = i\hbar$. L’état du système est alors décrit par une {\bf fonction d’onde} $\psi(x,t)$, solution de {\bf l’équation de  Schrödinger} indépendante du nombre de particules :
\begin{eqnarray}
\quad i \hbar \frac{\partial \psi }{\partial t}  &= \operator{\mathcal{H}} \psi,\label{chap.1.rapel.2}
\end{eqnarray}

avec $\operator{\mathcal{H}}$ l’opérateur hamiltonien. 

\begin{mdframed}[
	linewidth=0.5pt, 
	backgroundcolor=gray!5, 
	roundcorner=50pt,	
	innerleftmargin=5pt,
    innerrightmargin=5pt,
    innertopmargin=-10pt,
    innerbottommargin=2pt,
    leftmargin=2pt,
    rightmargin=2pt
	]
\subparagraph{Exemple : particule libre en une boite à une dimension.} 
	{~}\\
	
	Dans le cas d’une particule libre de masse $m$ se déplaçant en une dimension, l’Hamiltonien est constitué uniquement du terme cinétique $\operator{\mathcal{H}} = \operator{p}^2 / 2m$. En représentation position, où l’opérateur quantité de mouvement s’écrit comme dans l’équation \eqref{chap.1.rapel.1}, l’Hamiltonien prend alors la forme différentielle :
	\begin{eqnarray}
		\operator{\mathcal{H}} = -\frac{\hbar^2}{2m} \partial_x^2.\label{chap.1.rapel.libre.1}
	\end{eqnarray}
	Les états propres stationnaires de \eqref{chap.1.rapel.2} dépendant du temps sont de la forme $\psi_k(x,t) = \varphi_k(x)\,e^{-i\varepsilon(k)t/\hbar}$ où $\varphi_k(x)$ est une fonction propre de l’hamiltonien,  soit de  l’équation stationnaire  $\operator{\mathcal{H}}\varphi_k = \varepsilon(k)\varphi_k$ \ie pour une particule libre:
	\begin{eqnarray}
		\frac{\hbar^2}{2m} \partial_x^2 \varphi_k = \varepsilon(k) \varphi_k,\label{chap.1.rapel.libre.2}
	\end{eqnarray}
	avec $\varepsilon(k)$ l’énergie associée à une onde plane de nombre d’onde $k$
	\begin{eqnarray}
		\varepsilon(k) = \frac{\hbar^2 k^2 }{2 m}\label{chap.1.rapel.libre.3}.
	\end{eqnarray}
	Les fonctions propres spatiales $\varphi_k(x)$ de l’hamiltonien libre s’écrivent comme des combinaisons linéaires d’ondes planes  
	\begin{eqnarray}
		\varphi_k(x) = a e^{-i k x} + b e^{i k x},~~ \mbox{avec}\quad  (a,b) \in \mathbb{C}^2\label{chap.1.rapel.libre.4}.
	\end{eqnarray}
\subparagraph{Périodisité.}
	Si la particule est confinée dans une boîte de longueur $L$ avec des conditions aux limites périodiques (ie $\varphi_k(x+L) = \varphi_k(x)$), alors le spectre de $k$ est quantifié : 
	\begin{eqnarray}
		e^{kL}= 1 \quad\mbox{ou encore} kL \in 2\pi\mathbb{Z}\label{chap.1.rapel.libre.5}.
	\end{eqnarray}
	Le problème est équivalent à celui d’une particule libre sur un cercle de périmètre $L$.\\
	
	\medskip
	
	La particule est délocalisée sur tout l’espace (le cercle), sans structure particulière \ie le solutions \eqref{chap.1.rapel.libre.4} correspondent à des {\bf états non liés} (ou états de diffusion).

%On résume :
%\begin{eqnarray}
%	,~~ , ~~\varphi_k(x) = a e^{-i k x} + b e^{i k x},~~ kL \in 2\pi\mathbb{Z}.\label{chap.1.recap}
%\end{eqnarray}
\end{mdframed}

\medskip

Pour $k \neq 0 $ (respectivement pour $k = 0$), la fonction propre $\varphi_k(x)$ de l’équation \eqref{chap.1.rapel.libre.4} appartient à un sous-espace propre associé à $k$ de dimension 2 (respectivement de dimension 1) engendré par $x \mapsto e^{-ikx}$ et $x \mapsto e^{ikx}$ (respectivement par $x \mapsto 1$).
L’espace engendré par l’ensemble des sous-espaces propres forme un {\bf espace de Hilbert} , muni du {\bf produit scalaire} défini par :
\begin{eqnarray}
	( \varphi_{k'} , \varphi_{k} ) = \int_0^L \varphi_{k'}^\ast(x) \varphi_{k}(x) \, dx .\label{chap.1.rapel.libre.6}
\end{eqnarray}
Les sous-espaces propres sont orthogonaux entre eux \ie en utilisant les conséquences de la condition de périodicité \eqref{chap.1.rapel.libre.5}, $( \varphi_{k'} , \varphi_{k} ) = 0$ pour $\vert k' \vert \neq \vert k \vert  $ . 
Pour chaque sous-espace propre on impose que les états propres forment une base orthonormale \ie en utilisant \eqref{chap.1.rapel.libre.5}, les fonctions propres $\varphi_{k}$ écrit sous la forme \eqref{chap.1.rapel.libre.4}, sont orthogonaux avec $\varphi_{\overline{k}} \colon x \mapsto \pm ( b^\ast e^{-ikx} - a^\ast e^{ikx} )$  soit $(\varphi_{\overline{k}} , \varphi_{k} ) = 0$, et on impose que  $ \vert a \vert^2 + \vert b \vert^2 = L^{-1}$ pour assuré la normalité  de $\varphi_{k}$  et de $\varphi_{\overline{k}}$ soit  $( \varphi_{k} , \varphi_{k} ) = (\varphi_{\overline{k}} , \varphi_{\overline{k}})   = 1$. 

\medskip

Les solutions générales de l’équation de Schrödinger s’écrivent alors comme une superposition d’états propres  $\psi = c_0 \psi_0 +  \sum_{\vert k \vert > 0 } (c_k \psi_k  + c_{\overline{k}} \psi_{\overline{k}}) $. 

\medskip
Il y a deux base de vecteur propre particulier : 
\begin{enumerate}[label=\roman*)]
	\item {\bf Base de chiralité / impulsion :}
%	\begin{eqnarray}
%		\left \{ \begin{array}{rcll} \varphi_+  & = & \displaystyle \frac{1}{\sqrt{L}} e^{+ikx} & \mbox{: état avec impulsion $+\hbar k $ } \\ \varphi_-  & = & \displaystyle \frac{1}{\sqrt{L}}e^{-ikx} & \mbox{: état avec impulsion $-\hbar k $ }\end{array}\right.
%	\end{eqnarray}
	\begin{eqnarray}\label{chap.1.rapel.libre.7}
		\varphi_\pm  & = & \displaystyle \frac{1}{\sqrt{L}} e^{\pm ikx} 
	\end{eqnarray}
	Ces derniers de plus d'être états propres de l’opérateur énergie $\operator{\mathcal{H}}$, sont des états propres de l’opérateur impulsion $\operator{p}$, avec valeurs propres opposées $\pm \hbar k$.
	\item {\bf Base symétrique / antisymétrique :} En appliquant la matrice de passage unitaire  $\frac{1}{\sqrt{2}}\left (\begin{matrix}1 & 1 \\ -i & + i\end{matrix}\right)$ à la base $\{ \varphi_+ , \varphi_- \}$ , on passer dans las base    
%	\begin{eqnarray}
%		\left \{ \begin{array}{rcllll} \varphi_k  &= & \displaystyle\frac{1}{\sqrt{2L}}(e^{+ikx} + e^{-ikx})  & = & \displaystyle \sqrt{\frac{2}{L}} \cos (kx)  & \mbox{type Neumann  : $\varphi_k'(0) = \varphi_k'(L) = 0 $ } \\ \varphi_{\overline{k}}  & = & \displaystyle\frac{1}{\sqrt{2L}i}(e^{+ikx} - e^{-ikx}) & = & \displaystyle \sqrt{\frac{2}{L}} \sin (kx)  & \mbox{type Dirichlet : $\varphi_{\overline{k}}(0) = \varphi_{\overline{k}}(L) = 0 $}\end{array} \right.
%	\end{eqnarray}
	\begin{eqnarray}\label{chap.1.rapel.libre.8}
		\left \{ \begin{array}{rcllll} \varphi_S  & = & \displaystyle \sqrt{\frac{2}{L}} \cos (kx)  & \mbox{type Neumann  : $\varphi_S'(0) = \varphi_S'(L) = 0 $ } \\ \varphi_A   & = & \displaystyle \sqrt{\frac{2}{L}} \sin (kx)  & \mbox{type Dirichlet : $\varphi_A(0) = \varphi_A(L) = 0 $}\end{array} \right.
	\end{eqnarray}
\end{enumerate}


\medskip

Cette condition d’orthonormalité est imposée afin de garantir l’indépendance linéaire des états quantiques, et d'assurer que toute fonction d’onde de l’espace de Hilbert puisse être développée de manière unique sur cette base. 


\medskip

Avec le formalisme de Dirac, la fonction d’onde $\varphi_k$ est représentée par le ket $\ket{k}$ normé (\ie $\langle k' \vert k \rangle = \delta_{k', k}$, où $\delta_{p,q}$ est le symbole de Kronecker)
, et l’équation de Schrödinger s’écrit :
\(
\operator{\mathcal{H}} \ket{k} = \varepsilon(k) \ket{k}.
\)
En appliquant le bra $\bra{x}$ de part et d’autre, on obtient :
\(
\bra{x} \operator{\mathcal{H}} \ket{k} = \varepsilon(k) \langle x \vert k \rangle,
\)
où $\ket{x}$ est normé (\ie $\braket{x'\vert x} = \delta ( x' - x ) $ avec $\delta ( y - x )$ une distribution de Dirac) et $\varphi_k(x) = \langle x \vert k \rangle$ est la représentation positionnelle de l’état $\ket{k}$.


\begin{mdframed}[
	linewidth=0.5pt, 
	backgroundcolor=gray!5, 
	roundcorner=50pt,	
	innerleftmargin=5pt,
    innerrightmargin=5pt,
    innertopmargin=1pt,
    innerbottommargin=2pt,
    leftmargin=2pt,
    rightmargin=2pt
	]
La base $\{\ket{x}\}$ étant continue, et les états $\{\ket{k}\}$ quantifiés (par exemple dans une boîte de taille finie avec conditions aux limites périodiques), les relations de changement de base s’écrivent :
\begin{eqnarray}\label{chap.1:eq.rapel.etat.prim.1}
	\ket{k} = \int_0^L dx \, \varphi_k(x) \ket{x}, \qquad   
	\ket{x} = \sum_k \varphi_k^\ast(x) \ket{k},
\end{eqnarray}
avec $\varphi_k^\ast(x) = \langle k \vert x \rangle$. L’état $\ket{x}$ est relié aux états $\ket{k}$ par une transformation de Fourier discrète. Ces formules montrent que les états $\ket{k}$ sont les composantes de Fourier de l’état $\ket{x}$.
\end{mdframed}


\subparagraph{De la particule unique aux systèmes à $N$ particules.}

Pour un système composé de $N$ particules identiques, une approche naturelle consiste à introduire une fonction d’onde $\varphi(x_1, \dots, x_N)$ dépendant de $N$ variables , symétrique pour des bosons ou antisymétrique pour des fermions sous l’échange de deux coordonnées $x_i \leftrightarrow x_j$, solution de l’équation de Schrödinger à $N$ corps. 
Toutefois, cette description devient rapidement inextricable lorsque le nombre de particules augmente, ou lorsque le système permet la création et l’annihilation de particules, comme dans un milieu ouvert ou en contact avec un bain thermique.


\subsubsection{Seconde quantification}

Pour dépasser ces limitations, on adopte le \textbf{formalisme de la seconde quantification}, dans lequel l’état du système est décrit non plus par une fonction d’onde mais par un vecteur dans un espace de Fock. Les opérateurs de création et d’annihilation remplacent alors les variables dynamiques classiques et permettent une description unifiée et élégante des systèmes à nombre variable de particules.

\paragraph{Structure de l’espace des états de Fock.}
Dans ce formalisme, l’espace des états est une {\bf somme directe d’espaces à $N$ particules}, et chaque état est décrit par l’occupation des différents modes quantiques. Les opérateurs $\operator{a}_k^\dagger$ et $\operator{a}_k$ créent et annihilent une particule dans l’état d’onde plane de moment $k$ :
\begin{eqnarray}
	\ket{k} & = & \operator{a}_k^\dagger \ket{\emptyset} ,%~\equiv~ \text{état avec une particule dans le mode } k,	
\end{eqnarray}
état avec une particule dans le mode $k$ , où \(\ket{\emptyset}\) désigne le vide quantique de Fock, défini par :
\begin{eqnarray}
	\forall k \in \mathbb{R}\colon \qquad \operator{a}_k \ket{\emptyset} = 0 ,\quad  \langle \emptyset \vert \emptyset \rangle = 1. \label{chap:eq.vide.fock.k}
\end{eqnarray}
Le symbole \( \operator{a}_\lambda \) représente ici de manière générique soit l’opérateur \( \operator{b}_\lambda \) pour les bosons, soit \( \operator{c}_\lambda \) pour les fermions, et satisfait respectivement les relations de commutation (pour les bosons) ou d’anticommutation (pour les fermions). Dans ce qui suit, nous nous restreignons au cas bosonique.

\subparagraph{Relations de commutation bosoniques.} Les relations de commutation fondamentales pour les bosons sont :
\begin{eqnarray}
	[\operator{b}_k, \operator{b}_{k'}] = [\operator{b}_k^\dagger, \operator{b}_{k'}^\dagger] = 0 ,\qquad [\operator{b}_k, \operator{b}_{k'}^\dagger] = \operator{\delta}_{k,k'}, \label{chap:1:com.1.k}
\end{eqnarray}
où $\operator{\delta}_{k,k'}$ est le symbole de Kronecker, valant $1$ si $k = k'$ et $0$ sinon.


%\vspace{1em}
\paragraph{Nature du champ quantique.}
La seconde quantification généralise ce cadre en permettant de traiter des systèmes où le nombre de particules n’est pas fixé, ce qui est fréquent en physique des particules, des champs quantiques, ou des gaz quantiques.

L’idée principale est de ne plus quantifier directement les particules, mais le \emph{champ quantique} associé. Les états d’une particule unique deviennent alors des états d’occupation dans un espace de Fock, qui décrit l’ensemble des configurations possibles avec zéro, une, ou plusieurs particules.



\subparagraph{Champs de Bose.}
Le gaz de Bose unidimensionnel est décrit dans le cadre de la théorie quantique des champs par un champ bosonique canonique \( \operator{\Psi}(x) \), qui agit sur l’espace de Fock des états du système. Ce champ quantique encode l’annihilation d’une particule en \( x \), et son adjoint \( \operator{\Psi}^\dag(x) \) correspond à la création d’une particule en ce point. 
\begin{eqnarray}
	\vert x \rangle  & = & \operator{\Psi}^\dag (x)\ket{\emptyset} ,
\end{eqnarray}
état avec une particule en  $x$ et \(\ket{\emptyset}\) est le vide quantique de Fock défini par :
\begin{eqnarray}
	\forall x \in \mathbb{R}, \qquad \operator{\Psi}(x) \ket{\emptyset} = 0. \label{chap:eq.vide.fock}
\end{eqnarray}

\subparagraph{Relations de commutation bosoniques.}
Ces champs satisfont les relations de commutation canoniques à temps égal :
%\begin{eqnarray}
%	\left . \begin{array}{rcl}
%		[ \operator{\Psi}(x),  \operator{\Psi}^\dagger(y) ]  &=&  \operator{\delta}(x - y) \\
%		\left [ \operator{\Psi}(x),  \operator{\Psi}(y) \right ]   =  [ \operator{\Psi}^\dag(x),  \operator{\Psi}^\dag(y) ]  &=&  0 
%	\end{array} \right . \label{chap:1:com.1}
%\end{eqnarray}
\begin{eqnarray}
	 [ \operator{\Psi}(x),  \operator{\Psi}(y)  ]   =  [ \operator{\Psi}^\dag(x),  \operator{\Psi}^\dag(y) ]  =  0,   & & [ \operator{\Psi}(x),  \operator{\Psi}^\dagger(y) ]  =  \operator{\delta}(x - y) ,\label{chap:1:com.1}
\end{eqnarray}
où $\operator{\delta}(x - y)$ est la fonction delta de Dirac.  
Ces relations expriment le caractère bosonique des excitations du champ.

\paragraph{État à $N$ particules.} Soient $N$ bosons dans les états $\{ k_1 , \cdots , k_N \}$ (un boson dans l’état $k_1$, un autre dans $k_2$, etc.) et aux positions $\{ x_1 , \cdots , x_N \}$ (un boson en $x_1$, un autre en $x_2$, etc.). Leurs états s’écrivent alors :
\begin{eqnarray}
	\ket{ \{ k_1 , \cdots , k_N \}} = \frac{1}{\sqrt{N!}} \operator{b}_{k_1}^\dag\, \cdots \, \operator{b}_{k_N}^\dag \ket{\emptyset}, \quad \ket{\{x_1 , \cdots , x_N\}} = \frac{1}{\sqrt{N!}} \operator{\Psi}^\dag(x_1)\, \cdots \, \operator{\Psi}^\dag(x_N) \ket{\emptyset}	, \label{eq.chap.1.ket.N}
\end{eqnarray}
où le facteur \( 1/\sqrt{N!} \) traduit le caractère d’indiscernabilité des bosons et garantit la symétrisation correcte de l’état.

\begin{mdframed}[
	linewidth=0.5pt, 
	backgroundcolor=gray!5, 
	roundcorner=50pt,	
	innerleftmargin=5pt,
    innerrightmargin=5pt,
    innertopmargin=-10pt,
    innerbottommargin=2pt,
    leftmargin=2pt,
    rightmargin=2pt
	]
\subparagraph{Changement de base.}
On peut relier les opérateurs de création/annihilation dans la base des ondes planes aux opérateurs de champ via :
\begin{eqnarray}\label{chap.1:eq.rapel.etat.second.1}
	\operator{b}_k^\dagger = \int_0^L dx \, \varphi_k(x) \operator{\Psi}^\dagger(x), \qquad 
	\operator{\Psi}^\dagger(x) = \sum_k \varphi_k^\ast(x)\operator{b}_k^\dagger.\label{eq.chap.1.TF.1}
\end{eqnarray}
Le champ quantique $\operator{\Psi}(x)$ est relié aux opérateurs de moment $\operator{b}_k$ par une transformation de Fourier. Ces formules montrent que les opérateurs $\operator{b}_{k}$ sont les composantes de Fourier du champ $\operator{\Psi}(x)$.
\end{mdframed}
%où $\varphi_k(x)$ est la fonction d’onde d’un état d’énergie bien définie \( \ket{k} \) dans la représentation positionnelle.
Ainsi, un état à \(N\) bosons dans la base \( \ket{k}^{\otimes N} \) peut s’écrire :
\begin{eqnarray}
	\ket{\{k_a\}} = \frac{1}{\sqrt{N!}} \int dx_1 \cdots dx_N \, \varphi_{\{k_a\}} ( \{x_a\} ) \, \operator{\Psi}^\dag(x_1) \cdots \operator{\Psi}^\dag(x_N) \ket{\emptyset},
\end{eqnarray}
où on note \( \{k_a\} \equiv \{k_1, \dots, k_N\} \) et \( \{x_a\} \equiv \{x_1, \dots, x_N\} \), et la fonction d’onde symétrisée s’écrit :
\(
	\varphi_{\{k_a\}} ( \{x_a\} ) = \frac{1}{\sqrt{N!}} \sum_{\sigma \in \operator{S}_N } \prod_{i=1}^N \varphi_{k_{\sigma(i)}}(x_i),
\) 
avec $\operator{S}_N $  le groupe symétrique d'ordre $N$ mais aussi :
\begin{eqnarray}\label{chap.1:eq.rapel.fonction.propre.N}
	\varphi_{\{k_a\}} (\{x_a\}) = \frac{1}{\sqrt{N!}} \bra{\emptyset} \operator{\Psi}(x_1) \cdots \operator{\Psi}(x_N) \ket{\{k_a\}}.
\end{eqnarray}



\subsubsection{Operateur. }


\paragraph{Opérateur à un corps.}

\subparagraph{Dans la base discrètes des modes \( \{ \ket{k} \} \).}

Soit \( \operator{f} \) un opérateur à une particule, dont les éléments de matrice dans une base orthonormée \( \{ \ket{k} \} \) sont donnés par \( f_{\lambda\nu} = \langle \lambda \vert \operator{f} \vert \nu \rangle \). Un opérateur symétrique à \( N \) particules correspondant à la somme des actions de \( \operator{f} \) sur chacune des particules s’écrit en première configuration  :
\(
	\operator{F} = \sum_{i=1}^N \operator{f}^{(i)},
\)
où \( \operator{f}^{(i)} \) désigne l’action de \( \operator{f} \) sur la $i^\text{e}$ particule uniquement. En base de Dirac, cela donne :
\(
	\operator{f}^{(i)} = \sum_{\lambda, \nu} f_{\lambda\nu} \, \ket{i\!:\!\lambda} \bra{i\!:\!\nu},
\)
où \( \ket{i\!:\!\lambda} \) représente un état où seule la $i^\text{e}$ particule est dans l’état \( \lambda \). %(Par construction, l’opérateur \( \operator{F} \) commute avec les projecteurs de symétrisation \( \operator{S}_N \) (bosons) et d’antisymétrisation \( \operator{A}_N \) (fermions).)
On peut montrer que la somme des projecteurs agissant sur chaque particule s’identifie à une combinaison d’opérateurs de création et d’annihilation :
\(
	\sum_{i=1}^N \ket{i\!:\!\lambda} \bra{i\!:\!\nu} = \operator{a}^\dagger_\lambda \operator{a}_\nu^{},
\)
(où \( \operator{a}_\lambda \) est une notation générique désignant \( \operator{b}_\lambda \) pour les bosons, ou \( \operator{c}_\lambda \) pour les fermions).

On en déduit que l’opérateur à un corps \( \operator{F} \) peut se réécrire dans le formalisme de la seconde quantification comme :
\begin{eqnarray}\label{chap.1:eq.rapel.opp.1.prim.1}
	\operator{F} = \sum_{\lambda, \nu} \bra{\lambda} \operator{f} \ket{\nu} \operator{a}^\dagger_\lambda \operator{a}_\nu^{}.
\end{eqnarray}

L'opérateur $\operator{a}^\dagger_\lambda \operator{a}_\nu^{}$ fais la transition d'une particule de l'état $\nu$ à vers l'état $\lambda$. Si $\lambda = \nu$ cette opérateur est l'opérateur nombre de particule dans le mode $\lambda$.

\begin{mdframed}[
	linewidth=0.5pt, 
	backgroundcolor=gray!5, 
	roundcorner=50pt,	
	innerleftmargin=5pt,
    innerrightmargin=5pt,
    innertopmargin=-10pt,
    innerbottommargin=2pt,
    leftmargin=2pt,
    rightmargin=2pt
	]
\subparagraph{Exemple : Énergie cinétique totale des particules libres.}

Si l’on sait diagonaliser l’opérateur à une particule \( \operator{f} \), c’est-à-dire si l’on peut écrire :
\(
	\operator{f} = \sum_k f_k \ket{k} \bra{k},
\)
alors l’opérateur à $N$ corps associé s’écrit :
\(
	\operator{F} = \sum_k \bra{k} \operator{f} \ket{k} \operator{a}^\dagger_k \operator{a}_k^{}.
\)
On obtient ainsi une forme diagonale de \( \operator{F} \) en seconde quantification.

Un exemple immédiat est celui des \textbf{particules libres sans interaction}. On rappelle que :
\(
	\operator{\mathcal{H}} \ket{k} = \varepsilon(k) \ket{k},
\)
avec $\varepsilon(k)$ l’énergie du mode $k$ \eqref{chap.1.rapel.libre.3}.
En injectant $\operator{f} = \operator{\mathcal{H}} \,( = \frac{\operator{p}^2}{2m})$ dans \eqref{chap.1:eq.rapel.opp.1.prim.1}, on obtient l’énergie cinétique totale du système :
\begin{equation}
	\operator{K} = \sum_{k} \varepsilon(k) \, \operator{b}^\dagger_k \operator{b}_k^{}.\label{eq.chap.1.cinietique.1}
\end{equation}

Pour $N$ particules libres, en écrivant l’état sous la forme~\eqref{eq.chap.1.ket.N}, en utilisant les relations de commutation~\eqref{chap:1:com.1.k} et la définition de l’état de Fock~\eqref{chap:eq.vide.fock.k}, on trouve que $\ket{\{k_a\}}$ est un état propre de $\operator{K}$ associé à l’énergie
\(
	\sum_{i = 1}^N \varepsilon(k_i),
\)
c’est-à-dire :
\begin{eqnarray}
	\operator{K} \ket{\{k_a\}} = \left( \sum_{i = 1}^N \varepsilon(k_i) \right) \ket{\{k_a\}}.\label{eq.chap.1.cinietique.2}
\end{eqnarray}

%On insiste sur le fait que cette diagonalisation n’est valable que pour un gaz de particules \textbf{libres sans interaction}. 
%Dès que des interactions sont présentes, les états propres du système ne sont plus simplement donnés par les produits d’ondes planes $\ket{\{k_a\}}$, et l’opérateur énergie cinétique $\operator{K}$ n’est plus diagonal dans cette base.

\end{mdframed}

\subparagraph{Dans la base continue des positions \( \{ \ket{x} \} \).}
%Les opérateurs à plusieurs corps peuvent être exprimés de manière remarquable à l’aide des opérateurs de champ, d’une façon physiquement transparente qui rappelle les formules bien connues du cas à une particule.
En injectant les relation des changement de base d'état
\eqref{chap.1:eq.rapel.etat.prim.1} et de champ \eqref{chap.1:eq.rapel.etat.second.1} (qui prend la même forme pour $\operator{a}_\lambda^{}$ et pour $\operator{a}_\lambda^\dagger$), dans \eqref{chap.1:eq.rapel.opp.1.prim.1} on obtient :
\begin{eqnarray}\label{chap.1:eq.rapel.opp.1.second.2}
\operator{F} = \iint_0^L dx \, dy \, \operator{\Psi}^\dagger(x) \, \bra{ x} \operator{f} \ket{y} \, \operator{\Psi}(y).
\end{eqnarray}%où \( \hat{f} \) est l’opérateur à un corps exprimé dans la base position, et \( \hat{\psi}^\dagger(\vec{r}) \), \( \hat{\psi}(\vec{r}) \) sont les opérateurs de création et d’annihilation d’une particule au point \( \vec{r} \).
\begin{mdframed}[
	linewidth=0.5pt, 
	backgroundcolor=gray!5, 
	roundcorner=50pt,	
	innerleftmargin=5pt,
    innerrightmargin=5pt,
    innertopmargin=-10pt,
    innerbottommargin=2pt,
    leftmargin=2pt,
    rightmargin=2pt
]

\subparagraph{Exemple : Énergie cinétique totale des particules libres.}

Reprenons l’exemple de l’énergie cinétique totale avec  
\(
	\operator{f} = \frac{\operator{p}^2}{2m}.
\) 
En injectant 
\(
	\bra{x} \operator{f} \ket{x'} = - \frac{\hbar^2}{2m} \partial_{y}^2 \delta ( y - x )
\) 
dans \eqref{chap.1:eq.rapel.opp.1.second.2}, on réécrit l’opérateur énergie cinétique total $\operator{K}$ de l’équation \eqref{eq.chap.1.cinietique.1} :
\begin{eqnarray}
\operator{K} =  -\frac{\hbar^2}{2m} \int_0^L dx \, \operator{\Psi}^\dagger(x) \, \operator{\partial}_x^2 \operator{\Psi}(x)
= \frac{\hbar^2}{2m} \int_0^L dx \, \operator{\partial}_x \operator{\Psi}^\dagger(x) \cdot \operator{\partial}_x \operator{\Psi}(x). \label{eq.chap.1.cinietique.3}
\end{eqnarray}

Lorsque cet Hamiltonien agit sur l’état de Fock à $N$ particules libres $\ket{\{k_a\}}$, les règles de commutation (\ref{chap:1:com.1}) ainsi que la définition des états de Fock (\ref{chap:eq.vide.fock}) impliquent (cf. Annexe \ref{annex:N.part}) :
\begin{eqnarray}\label{eq.chap.1.cinietique.4}
\operator{K}\ket{ \{k_a\}} =  \int_0^L d^N z \, \operator{\mathcal{K}}_N \, \varphi_{\{k_a\}}(\{z_a\} ) \operator{\Psi}(z_1) \cdots \operator{\Psi}^\dag(z_N) \ket{\emptyset}\, , \quad \mbox{avec} \quad \operator{\mathcal{K}}_N = -\sum_{i=1}^N \frac{\hbar^2\operator{\partial}_{z_i}^2}{2m}, 
\end{eqnarray}
où \( -i \hbar \operator{\partial}_{z_i} \)  désigne l’opérateur impulsion associé à la coordonnée \( z_i \) de la $i$-ème particule.

\medskip



\end{mdframed}

\subparagraph{Quantité de mouvement totale et nombre total de particules.}
Dans l'exemple de l'énergie cinétique, l'opérateur  $\operator{f}$ est proportionnel à l'opérateur impulsion au carré $\operator{p}^2$. On peut appliquer un raisonnement similaire à d'autres observables : pour le {\bf nombre total de particules}, on prend  $\operator{f} = \operator{p}^0$, c’est-à-dire l’identité et pour la {\bf quantité de mouvement totale}, on choisit $\operator{f} = \operator{p}$ (puissance 1). On note $\operator{Q}$ l'opérateur nombre total de particule et $\operator{P}$ l'opérateur quantité mouvement totale . 

\medskip

Ces observables sont liées à des symétries fondamentales du système : $\operator{Q}$ est associé à la symétrie $U(1)$ (conservation du nombre de particules) ; $\operator{P}$ est associé à la translation (conservation de la quantité de mouvement).

Pour ces observables, les états $\ket{\{k_a\}}$ sont des états propres, et cela reste vrai \textbf{avec ou sans interaction}, tant que le système conserve ces symétries fondamentales. Ainsi, $\ket{\{k_a\}}$ diagonalise directement $\operator{Q}$ et $\operator{P}$, même en présence d’interactions locales qui respectent $U(1)$ et la translation.

\medskip


En seconde quantification, ces opérateurs s'écrivent dans la base \( \{ \ket{k} \} \) : 
\begin{equation}
	\operator{Q} = \sum_{k}  \operator{b}^\dagger_k \operator{b}_k^{} , \quad  \operator{P} = i\hbar\sum_{k} k \, \operator{b}^\dagger_k \operator{b}_k^{},\label{eq.chap.1.Q.P.k.1}
\end{equation}
Lorsqu’on les applique à un état de Fock à $N$ particules \(\ket{\{k_a\}}\), on obtient :
\begin{eqnarray}\label{eq.chap.1.Q.P.k.2}
	\operator{Q} \ket{\{k_a\}} = \left( \sum_{i = 1}^N 1 \right) \ket{\{k_a\}},\quad \operator{P} \ket{\{k_a\}} = -i\hbar \left( \sum_{i = 1}^N k \right) \ket{\{k_a\}}.
\end{eqnarray}
Dans la base position \( \{ \ket{x} \} \), les opérateurs s’écrivent : 
\begin{eqnarray}\label{eq.chap.1.Q.P.k.3}
	\operator{Q}  =  \int_0^L dx \, \operator{\Psi}^\dag (x) \operator{\Psi} (x) \, , \quad 
	\operator{P}  =  \frac{i\hbar}2 \int_0^L dx \, \left \{  \operator{\Psi}^\dag(x) \operator{\partial}_x \operator{\Psi}(x) - \left [\operator{\partial}_x \operator{\Psi}^\dag(x)\right ] \operator{\Psi}(x)\right \} 
\end{eqnarray}
où l'expression symétrisée de $\operator{P}$ assure son hermiticité.\\ 
Lorsqu'on applique ces opérateurs à l’état \(\ket{\{k_a\}}\), on obtient (comme pour l’énergie cinétique) :
\begin{eqnarray}
	\operator{Q}\ket{ \{k_a\}} =  \int_0^L d^N z \, \operator{\mathcal{N}}_N \, \varphi_{\{k_a\}}(\{z_a\} ) \operator{\Psi}(z_1) \cdots \operator{\Psi}^\dag(z_N) \ket{\emptyset}\, , \quad &\mbox{avec}& \quad \operator{\mathcal{N}}_N = \sum_{i=1}^N 1,\label{eq.chap.1.Q.1}\\
	\operator{P}\ket{ \{k_a\}} =  \int_0^L d^N z \, \operator{\mathcal{P}}_N \, \varphi_{\{k_a\}}(\{z_a\} ) \operator{\Psi}(z_1) \cdots \operator{\Psi}^\dag(z_N) \ket{\emptyset}\, , \quad &\mbox{avec}& \quad \operator{\mathcal{P}}_N = \hbar \sum_{i=1}^N \operator{\partial}_{z_i}\label{eq.chap.1.P.1}.
\end{eqnarray}

\begin{mdframed}[
	linewidth=0.5pt, 
	backgroundcolor=gray!5, 
	roundcorner=50pt,	
	innerleftmargin=5pt,
    innerrightmargin=5pt,
    innertopmargin=-10pt,
    innerbottommargin=2pt,
    leftmargin=2pt,
    rightmargin=2pt
	]
\subparagraph{On s'avance sur le chapitre (\ref{chap:relaxation}).} , en voulant généraliser avec $\operator{f} = \operator{p}^q$ où $q$ est un entier. Soit dans la base \(\{\ket{k}\}\) : \(\operator{F} = \hbar^q \sum_k k^q \operator{b}_k^\dagger \operator{b}_k^{} \) et en l'appliquant à \(\ket{\{k_a\}}\) :
On peut généraliser cette construction en considérant des opérateurs à une particule de la forme $\operator{f} = \operator{p}^q$, où $q$ est entier. Dans la base impulsion \(\{\ket{k}\}\), l’opérateur à $N$ corps associé s’écrit : \(\operator{F} = \hbar^q \sum_k k^q \operator{b}_k^\dagger \operator{b}_k^{} \) et son action sur un état de Fock libre est immédiatement 
\begin{eqnarray}\label{chap.1:eq.rapel.opp.1.second.3}
	 \operator{F} \ket{\{k_a\}} = \hbar^q \left( \sum_{i = 1}^N k_i^q \right) \ket{\{k_a\}},
\end{eqnarray}
En représentation position \(\{\ket{x}\}\) , on obtient l’opérateur hermitisé \\	
\(
	\operator{F}  =  \frac{\hbar^q}2 \int_0^L \left \{  \operator{\Psi}^\dag(x) \operator{\partial}_x^q \operator{\Psi}(x) + (-1)^q \left [\operator{\partial}_x^q \operator{\Psi}^\dag(x)\right ]\operator{\Psi}(x)\right \} dx
\)
et son action sur $\ket{\{k_a\}}$ se traduit par :\\ 
\(
	\operator{F}\ket{ \{k_a\}} =  \int_0^L d^N z \, \operator{\mathcal{F}}_N \, \varphi_{\{k_a\}}(\{z_a\} ) \operator{\Psi}(z_1) \cdots \operator{\Psi}^\dag(z_N) \ket{\emptyset}\, 
\) \mbox{avec} 
\(\operator{\mathcal{F}}_N = \hbar^q \sum_{i=1}^N (\operator{\partial}_{z_i})^q.
\)

\medskip



\textbf{Remarque :} 

Cette construction n’est valable que pour un gaz \textbf{de particules libres, sans interaction}.  
En présence d’un Hamiltonien avec interaction (par exemple dans le modèle de  \gls{LL} Sec \ref{chap1:sssec:LL}), les états propres ne sont plus des produits d’ondes planes $\ket{\{k_a\}}$, et l’opérateur énergie cinétique $\operator{K}$ n’est plus diagonalisé dans cette base. Ce sont alors les états de Bethe qui diagonalise l’Hamiltonien complet (cinétique + interaction).

\medskip

Pour $q \ge 2$, si l’on ajoute une interaction locale (comme dans le modèle de Lieb–Liniger), l’opérateur $\operator{F}$ à une particule n’est plus suffisant pour définir une charge conservée : il faut inclure des termes à deux corps (ou davantage) pour assurer la commutation avec l’Hamiltonien.  
On passe ainsi progressivement des observables simples $\operator{Q}$ et $\operator{P}$, liées à des symétries fondamentales, aux \textbf{charges conservées d’intégrabilité} à plusieurs corps, qui sont précisément celles qui caractérisent les états de Bethe.

\end{mdframed}



\paragraph{Opérateurs à deux corps}

\subparagraph{Dans la base discrètes des modes \( \{ \ket{k} \} \).}

Nous considérons à présent les termes d’interaction impliquant deux particules , $\operator{v}$ , dont les éléments de matrices sont donnés par $v_{\alpha \beta \gamma \delta} = \bra{ 1 : \alpha; 2 : \beta } \operator{v}\ket{ 1 : \gamma; 2 : \delta }$ , où $\ket{ i : \gamma; j : \delta }$ représente l'état où la $i^\text{e}$  particules est dans l'état $\gamma$ et la $j^\text{e}$ dans l'état $\delta$  . Ceux-ci correspondent à des opérateurs de la forme :
\(
    \operator{V} = \sum_{j < i} \operator{v}^{(i, j)} = \frac{1}{2} \sum_{i, j \ne i} \operator{v}^{(i, j)}
    \label{eq:V2corps}.
\)
avec $\operator{v}^{(i, j)}$ désigne l’interaction à deux corps entre les $i^\text{e}$ et $j^\text{e}$ particules , exprimés dans la base à deux états :
\(
	\operator{v}^{(i, j)} = \sum_{\alpha,\beta,\delta,\gamma} \ket{i : \alpha; j : \beta }v_{\alpha \beta \gamma \delta} \bra{ i : \gamma; j : \delta }.
    %v_{\alpha \beta \gamma \delta} = \bra{ i : \alpha; j : \beta } \operator{v}^{(i,j)} \ket{ i : \gamma; j : \delta }.
    \label{eq:matriceV}
\)
On peut réécrire l’opérateur \( \operator{V} \) en termes d’opérateurs de création et d’annihilation comme suit :
\begin{equation}
    \operator{V} = \frac{1}{2} \sum_{\alpha, \beta, \gamma, \delta} \bra{ 1 : \alpha; 2 : \beta } \operator{v}\ket{ 1 : \gamma; 2 : \delta } \, \operator{a}^\dagger_\alpha \operator{a}^\dagger_\beta \operator{a}_\delta^{} \operator{a}_\gamma^{}.
    \label{chap.1:eq.rapel.opp.2.prim.1}
\end{equation}

Cette forme est particulièrement utile pour le traitement des interactions dans l’espace de Fock.

\subparagraph{Dans la base continue des positions \( \{ \ket{x} \} \).}

En injectant les relation des changement de base d'état
\eqref{chap.1:eq.rapel.etat.prim.1} et de champ \eqref{chap.1:eq.rapel.etat.second.1}, dans \eqref{chap.1:eq.rapel.opp.2.prim.1} on obtient :
\begin{equation}
    \operator{V} = \frac{1}{2} \iiiint_0^L dx_1^{} \, dx_2^{} \, dx_1' \, dx_2' \; 
    \bra{ 1 : x_1^{}, 2 : x_2^{} } \operator{v} \ket{ 1 : x_1', 2 : x_2' } \,
    \operator{\Psi}^\dagger(x_1^{}) \, \operator{\Psi}^\dagger(x_2^{}) \, 
    \operator{\Psi}(x_2') \, \operator{\Psi}(x_1')
    \label{chap.1:eq.rapel.opp.2.second.1}
\end{equation}

\begin{mdframed}[
	linewidth=0.5pt, 
	backgroundcolor=gray!5, 
	roundcorner=50pt,	
	innerleftmargin=5pt,
    innerrightmargin=5pt,
    innertopmargin=-10pt,
    innerbottommargin=2pt,
    leftmargin=2pt,
    rightmargin=2pt
	]
\subparagraph{Exemple : Interactions ponctuelles.} 
Dans le cas d’une interaction ne dépendant que de la distance relative entre deux particules, $\bra{ 1 : x_1^{}, 2 : x_2^{} } \operator{v} \ket{ 1 : x_1', 2 : x_2' } = v(x_1 -x_2)\delta(x_1 - x_1')\delta(x_2 - x_2') $,  l'expression \eqref{chap.1:eq.rapel.opp.2.second.1} se simplifie :
\begin{eqnarray}
     \operator{V} =  
    \frac{1}{2} \int dx_1^{} \, dx_2^{} \; v(x_1^{} - x_2^{}) \,
    \operator{\Psi}^\dagger(x_1^{}) \, \operator{\Psi}^\dagger(x_2^{}) \, 
    \operator{\Psi}(x_2^{}) \, \operator{\Psi}(x_1^{})
    \label{eq:V_local}
\end{eqnarray} 
soit pour des interactions ponctuelles :	
\begin{eqnarray}\label{chap.1:eq.rapel.opp.2.second.ponctuel.1}
	\quad \operator{V}  = \frac{g}{2} \int dx \,
    \operator{\Psi}^\dagger(x) \, \operator{\Psi}^\dagger(x) \, 
    \operator{\Psi}(x) \, \operator{\Psi}(x)  		
\end{eqnarray}
et quand on l'applique à l'état $\ket{\{k_a\}}$, les règles de commutations (\ref{chap:1:com.1}) et la définition d'état de Fock (\ref{chap:eq.vide.fock}) impliquent que (cf Annex \ref{annex:N.part})
\begin{eqnarray}\label{chap.1:eq.rapel.opp.2.second.ponctuel.2}
\operator{V}\ket{\{k_a \}} =  \int d^Nz \, \operator{\mathcal{V}}_N \varphi_{\{k_a\}}( \{z_a\} )\operator{\Psi}(z_1)\cdots \operator{\Psi}^\dag(z_N) \ket{\emptyset} \, \quad \mbox{avec} \quad  \operator{\mathcal{V}}_N 	= g\sum_{1\leq i < j \leq N } \operator{\delta}(z_i - z_j)	
\end{eqnarray}
%avec 
%\(
%	\operator{\mathcal{V}}_N 	
% = g\sum_{1\leq i < j \leq N } \operator{\delta}(x_i - x_j)	
%\)
où \( g \) est la constante de couplage.
\end{mdframed}




\subsubsection{Expression de l’Hamiltonien de Lieb-Liniger \acrshort{LL}. }\label{chap1:sssec:LL}

À partir d’ici, on fixe $\hbar = m = 1 $. Ainsi, les dimensions (unités) des nombres d’onde $k$ et des vitesses ne sont plus différenciées.
Dans le formalisme des opérateurs de champs, l’Hamiltonien d’un système soumis à des interactions ponctuelles est la somme de l’énergie cinétique $\operator{K}$ donnée par \eqref{eq.chap.1.cinietique.3}, et du terme d’interaction $\operator{V}$ introduit dans \eqref{chap.1:eq.rapel.opp.2.second.ponctuel.1} :
\begin{eqnarray}\label{chap.1:eq.LL.champ.1}
	\operator{H} & = & \int dx \, \operator{\Psi}^\dag (x)\left [-\frac{1}{2}\operator{\partial}_x^2 + \frac{g}{2}  \operator{\Psi}^\dag (x) \operator{\Psi} (x) \right ] \operator{\Psi} (x) \label{chap:1:ham.mod}.
\end{eqnarray}
%Lorsqu’on applique cet Hamiltonien à un état de Fock à $N$ particules $\ket{\{\theta_1 , \cdots , \theta_N \}} $, où chaque paramètre $\theta_i$ est homogène à un nombre d’onde ou à une vitesse (mais pas nécessairement homogéne à un nombre d'onde $k$ d’où la notation $\theta$), on obtient — en utilisant les équations \eqref{eq.chap.1.cinietique.4} et \eqref{chap.1:eq.rapel.opp.2.second.ponctuel.2} : \\

\medskip


Lorsqu’on applique cet Hamiltonien à un état à $N$ particules, il est important de distinguer deux situations :
\begin{itemize}[label = $\bullet$]
	\item {\bf Sans interaction ( $g =0$ )} : les états propres sont simplement les états de Fock en impulsion $\ket{\{k_a \}}$ définie en \eqref{eq.chap.1.ket.N}. Ce sont des produits d’ondes planes symétrisées.
	\item {\bf Avec interaction ( $g \neq 0$ )} : les états propres ne sont plus des produits d’ondes planes. Ce sont les états de Bethe, que l’on note 
		\begin{eqnarray}
			\ket{\{\theta_a \}}	
		\end{eqnarray}
		où les paramètres $\theta_a$ jouent le rôle de {\bf quasi-moments (rapidités)}.Ceux-ci sont homogènes à un nombre d’onde ou à une vitesse, mais ne coïncident pas directement avec les impulsions libres $k_a$. La fonction d’onde correspondante est une combinaison linéaire de morceaux d’ondes planes, reliés par des phases de diffusion fixées par l’interaction locale.
\end{itemize}

\medskip

Ainsi, la notation $\ket{\{\theta_a \}}$  est choisie pour rappeler que :
\begin{itemize}[label = $\bullet$]
	\item en absence d’interaction, $\theta_a = k_a$ et l’on retrouve les états de Fock,
	\item en présence d’interaction, $\theta_a$ sont les quasi-moments de Bethe, qui généralisent les nombres d’onde libres.
\end{itemize}

\medskip

En utilisant les équations \eqref{eq.chap.1.cinietique.4} et \eqref{chap.1:eq.rapel.opp.2.second.ponctuel.2}, on obtient : 
\begin{eqnarray}\label{chap.1:eq.LL.champ.2}
	\operator{H}\ket{\{\theta_a\}} =  \int d^Nz \, \operator{\mathcal{H}}_N \varphi_{\{\theta_a\}}(\{z_a\})\operator{\Psi}(z_1)\cdots \operator{\Psi}^\dag(z_N) \ket{\emptyset}, 
\end{eqnarray}
avec $\{\theta_a\} \equiv \{ \theta_1 , \cdots , \theta_N \} $ et .
\begin{eqnarray}\label{chap.1:eq.LL.eta.1}
\operator{\mathcal{H}}_N  =  \operator{\mathcal{K}}_N  +  \operator{\mathcal{V}}_N \, \quad \mbox{où on rappelle} \quad \operator{\mathcal{K}}_N = -\frac{1}2\sum_{i=1}^N \partial_{z_i} , \,  \mbox{et} \, 	\operator{\mathcal{V}}_N = g\sum_{1\leq i < j \leq N } \operator{\delta}(z_i - z_j)	. 
\end{eqnarray}

%\(
%	\operator{\mathcal{H}}_N 	
% =  \operator{\mathcal{K}}_N  +  \operator{\mathcal{V}}_N .	
%\)


%où \( g \) est la constante de couplage. %Dans ce chapitre, nous considérons uniquement les propriétés du système à un instant donné, de sorte que la dépendance temporelle des champs est omise pour alléger l’écriture.

%Ce formalisme est ainsi adapté pour décrire des condensats de Bose, des gaz quantiques, ou la création/annihilation de particules dans les champs quantiques.

\paragraph{Équation du mouvement associée.}

L’équation du mouvement du champ \( \operator{\Psi}(x) \) s'obtient à partir de l’équation de Heisenberg :
\begin{eqnarray}\label{chap.1:eq.GP.1}
	i\operator{\partial}_t	\operator{\Psi} & = & [ \operator{\Psi} , \operator{H} ].
\end{eqnarray}
En évaluant explicitement le commutateur (\ref{chap:1:com.1}), on trouve :
\begin{eqnarray}
	i \operator{\partial}_t \operator{\Psi}	 & = & - \frac{1}{2}\operator{\partial}_x^2 \operator{\Psi} + g \operator{\Psi}^\dag\operator{\Psi} \operator{\Psi}.
\end{eqnarray}
Il est important de souligner que cette équation est encore {\bf quantique} : elle décrit l’évolution de l’opérateur de champ $\operator{\Psi}$.

\medskip

En revanche, si l’on prend l’approximation dite de {\bf champ moyen}, où l’on remplace l’opérateur de champ par son espérance de valeur dans un état cohérent ou condensé
\begin{eqnarray}
	\Psi = \braket{\operator{\Psi}},		
\end{eqnarray}
on obtient alors l’équation de \gls{NS} (ou équation de  \gls{GP}) pour une fonction d’onde classique :
\begin{eqnarray}
	i \partial_t \Psi	 & = & - \frac{1}{2}\partial_x^2 \Psi + g \vert \Psi \vert^2  \Psi.
\end{eqnarray}
Ainsi, l’appellation « équation de Schrödinger non linéaire » n’est pertinente qu’au niveau {\bf effectif classique} (champ moyen), et non au niveau fondamental de l’équation de Heisenberg pour les champs quantiques.

%Pour $g > 0$, l'état fondamental à température nulle est une sphère de Fermi. Seul ce cas sera considéré par la suite.

%\vspace{0.5cm}


\paragraph{Conservation et commutation.}

\subparagraph{À propos d’autres interactions.}

La conservation des opérateurs nombre total de particules $\operator{Q}$ et quantité totale de mouvement $\operator{P}$ dépend uniquement des \textbf{symétries} de l’interaction, et non de sa force :  

\begin{itemize}[label = $\bullet$]
    \item Une interaction \textbf{locale et à deux corps} de type $g \, \operator{\Psi}^\dagger \operator{\Psi}^\dagger \operator{\Psi} \operator{\Psi}$ (cas du modèle de Lieb--Liniger) est invariante par $U(1)$ et par translation. Ainsi, $\operator{Q}$ et $\operator{P}$ sont conservés.  

    \item Une interaction \textbf{à longue portée mais translationalement invariante} (par exemple un potentiel dipolaire uniforme, ou un potentiel $V(x-y)$ dépendant seulement de la distance relative) conserve encore $\operator{P}$, et si elle respecte la symétrie $U(1)$, alors $\operator{Q}$ reste aussi conservé.  

    \item Une interaction \textbf{non uniforme ou dépendant explicitement de la position} (par exemple un potentiel externe $V(x)$, ou des bords durs) brise l’invariance par translation : $\operator{P}$ n’est alors plus conservé.  

    \item Une interaction qui \textbf{crée ou détruit des particules} brise la symétrie $U(1)$ : $\operator{Q}$ n’est plus conservé.  
\end{itemize}

En résumé, la conservation de $\operator{Q}$ et $\operator{P}$ ne dépend pas de la nature microscopique de l’interaction, mais uniquement des \textbf{symétries correspondantes} de l’Hamiltonien.  

\medskip

\subparagraph{Interactions de contact.}

Dans le cas particulier du modèle de Lieb--Liniger, les opérateurs nombre total de particules $\operator{Q}$ et quantité totale de mouvement $\operator{P}$ commutent avec l’Hamiltonien $\operator{H}$ :  
\begin{eqnarray}
[ \operator{H} , \operator{Q} ] = 0, 
\qquad 
[ \operator{H} , \operator{P} ] = 0.
\end{eqnarray}
Ils constituent donc des \textbf{intégrales du mouvement}. Cette propriété reflète directement la conservation du nombre de particules (symétrie $U(1)$) et l’invariance par translation du système (symétrie spatiale).  

\begin{mdframed}[
	linewidth=0.5pt, 
	backgroundcolor=gray!5, 
	roundcorner=50pt,	
	innerleftmargin=5pt,
    innerrightmargin=5pt,
    innertopmargin=5pt,
    innerbottommargin=2pt,
    leftmargin=2pt,
    rightmargin=2pt
	]	
	Nous verrons au chapitre~\ref{chap:relaxation} que, dans les systèmes intégrables comme le modèle de Lieb--Liniger, cette situation s’étend à une {\bf \em infinité d’intégrales du mouvement}. 
	Cette richesse supplémentaire de conservation conduit naturellement à l’introduction de l’ensemble de Gibbs généralisé, appelé en anglais \gls{GGE}.

\end{mdframed}


\paragraph{États propres et valeurs propres.}
Les états propres $\ket{\{\theta_a\}}$, construits dans le cadre de la seconde quantification à partir de la solution du modèle de Lieb-Liniger, sont simultanément fonctions propres des opérateurs $\operator{Q}$, $\operator{P}$ et $\operator{H}$ :
\begin{eqnarray}\label{chap1:eq.Q.P.K.theta.1}
\operator{Q} \ket{\{\theta_a\}} = N \ket{\{\theta_a\}}, \quad
\operator{P} \ket{\{\theta_a\}} = \left( \sum_{a=1}^N \theta_a \right) \ket{\{\theta_a\}}, \
\operator{H} \ket{\{\theta_a\}} = \left( \frac{1}{2} \sum_{a=1}^N \theta_a^2 \right) \ket{\{\theta_a\}}.
\end{eqnarray}

\paragraph*{Conclusion.}

La première quantification constitue la base indispensable pour comprendre le comportement quantique d’un nombre fixé de particules. La seconde quantification en est une extension naturelle, nécessaire pour décrire des systèmes plus complexes où le nombre de particules peut varier. Elle repose sur la quantification des champs et l’introduction d’opérateurs de création et d’annihilation, ouvrant ainsi la voie à la physique quantique des champs et à de nombreuses applications modernes.

\medskip

Les opérateurs $\operator{Q}$, $\operator{P}$ (\eqref{eq.chap.1.Q.P.k.3}) et $\operator{H}$ \eqref{chap.1:eq.LL.champ.1} possèdent une structure diagonale commune dans la base des états propres $\ket{\{\theta_a\}}$, révélant la nature intégrable du modèle de Lieb-Liniger. Leurs valeurs propres sont respectivement les 0\textsuperscript{e}, 1\textsuperscript{er} et 2\textsuperscript{e} moments des $\theta_a$. Cette structure permet de généraliser la construction à une hiérarchie complète d’observables conservées, qui seront présentées au chapitre suivant.

\medskip

En passant par la seconde quantification, nous avons pu relier l’Hamiltonien à une particule $\operator{\mathcal{H}}_1$ \eqref{chap.1.rapel.libre.1} à un Hamiltonien à $N$ particules $\operator{\mathcal{H}}_N$ \eqref{chap.1:eq.LL.eta.1} dans le modèle de Lieb-Liniger (système avec interactions ponctuelles). Les états $\varphi_{\{ \theta_a \}}$ sont des états propres du nombre total de particules $\operator{\mathcal{N}}_N$ \eqref{eq.chap.1.Q.1}, de la quantité de mouvement totale $\operator{\mathcal{P}}_N$ \eqref{eq.chap.1.P.1} et de l’hamiltonien $\operator{\mathcal{H}}_N$.

\medskip

Dans le cas d’un système à une particule, les états de la forme \eqref{chap.1.rapel.libre.7} sont déjà propres de $\operator{\mathcal{N}}_1$, $\operator{\mathcal{P}}_1$ et $\operator{\mathcal{H}}_1$ (cf. \eqref{chap.1.rapel.libre.2}). Mais quelle est leur forme pour $N$ particules ? Et comment généraliser les conditions périodiques \eqref{chap.1.rapel.libre.5}, valables pour une particule, à un système à $N$ particules ?\\

\medskip

Nous allons étudier le cas de $N=2$ corps, afin de simplifier l’analyse tout en capturant les premiers effets non triviaux des interactions. Cela permettra de comprendre plus facilement les implications physiques des interactions ponctuelles, en particulier leur influence sur la structure des états propres, les conditions de continuité des fonctions d’onde, ainsi que les relations de dispersion modifiées. Ce cas servira de base pour introduire les concepts clés de la résolution exacte par la méthode de Bethe ansatz, avant de les généraliser au cas $N$-corps.

 




%\subsection{Opérateurs nombre de particules et moment dans la formulation quantique du gaz de Lieb-Liniger}
%
%Dans cette section, nous nous intéressons aux opérateurs fondamentaux que sont le {\em nombre total de particules} $\operator{Q}$ et le {\em moment total} $\operator{P}$, dans le cadre du gaz de bosons unidimensionnel décrit par l’Hamiltonien de Lieb-Liniger. Après avoir introduit ces opérateurs dans le langage de la seconde quantification, nous montrons qu’ils sont {\em conservés} par la dynamique, et qu’ils admettent les {\em mêmes états propres} que l’Hamiltonien. Nous donnons ensuite leur expression dans la représentation à  $N$ particules, ainsi que la forme explicite de leurs valeurs propres en fonction des {\em rapidités} $\theta_a$ , illustrant la structure polynomiale typique des intégrales du mouvement dans les systèmes intégrables.
%
%\subsubsection{Définition en seconde quantification}
%
%Les opérateurs du nombre total de particules $\operator{Q}$ et du moment total $\operator{P}$ s’écrivent en seconde quantification comme suit :
%\begin{eqnarray}
%	\operator{Q}  =  \int \operator{\Psi}^\dag (x) \operator{\Psi} (x) \, d x, \quad 
%	\operator{P}  =  - \frac{i}2 \int \left \{  \operator{\Psi}^\dag(x) \operator{\partial}_x \operator{\Psi}(x) - \left [ \operator{\partial}_x \operator{\Psi}^\dag(x)\right ] \operator{\Psi}(x)\right \} dx \label{eq.1.7}
%\end{eqnarray}
%Ces deux opérateurs sont {\em hermitiens}, et représentent des observables physiques fondamentales : le nombre de particules et la quantité de mouvement du système.
%
%\subsubsection{Conservation et commutation}
%Ces opérateurs commutent avec l’Hamiltonien $\operator{H}$ du modèle de Lieb-Liniger :
%\begin{eqnarray}
%[ \operator{H} , \operator{Q} ] = 0, \quad [ \operator{H} , \operator{P} ] = 0.
%\end{eqnarray}
%Ils constituent ainsi des intégrales du mouvement. Cette propriété est une manifestation de la symétrie translationnelle du système (pour $\operator{P}$) et de la conservation du nombre total de particules (pour $\operator{Q}$).
%
%\begin{mdframed}[
%	linewidth=0.5pt, 
%	backgroundcolor=gray!5, 
%	roundcorner=50pt,	
%	innerleftmargin=5pt,
%    innerrightmargin=5pt,
%    innertopmargin=5pt,
%    innerbottommargin=2pt,
%    leftmargin=2pt,
%    rightmargin=2pt
%	]
%	Nous verrons au chapitre (\ref{chap:relaxation}) que cette situation s’étend à une {\bf \em infinité d’intégrales du mouvement} dans les systèmes intégrables, ce qui permettra de construire l’ensemble de Gibbs généralisé (GGE).
%\end{mdframed}
%
%\subsubsection{États propres et valeurs propres}
%Les états propres $\ket{\{\theta_a\}}$, construits dans le cadre de la seconde quantification à partir de la solution du modèle de Lieb-Liniger, sont simultanément fonctions propres des opérateurs $\operator{Q}$, $\operator{P}$ et $\operator{H}$ :
%\begin{eqnarray}
%\operator{Q} \ket{\{\theta_a\}} = N \ket{\{\theta_a\}}, \quad
%\operator{P} \ket{\{\theta_a\}} = \left( \sum_{a=1}^N \theta_a \right) \ket{\{\theta_a\}}, \
%\operator{H} \ket{\{\theta_a\}} = \left( \frac{1}{2} \sum_{a=1}^N \theta_a^2 \right) \ket{\{\theta_a\}}.
%\end{eqnarray}
%Autrement dit, les valeurs propres associées à ces trois opérateurs sont données par :
%\begin{eqnarray}
%N = \sum_{a = 1}^N \theta_a^0, \quad p = \sum_{a = 1}^N \theta_a, \quad e = \frac{1}{2} \sum_{a = 1}^N \theta_a^2.
%\end{eqnarray}
%Cela illustre que les trois premières intégrales du mouvement du système — nombre, moment, énergie — peuvent être exprimées comme des {\bf \em moments successifs} des rapidités.	
%
%\subsubsection{Forme en première quantification}
%En utilisant la représentation en espace de configuration $\{z_a\} \equiv \{z_1 , \cdots , z_N \}$, les opérateurs $\operator{Q}$ et $\operator{P}$ agissent comme suit sur les fonctions d’onde $\varphi_{\{\theta_a\}}(\{z_a\})$ :
%\begin{eqnarray}
%	\operator{Q}\ket{\{\theta_a\}} =  \sqrt{N!}\int d^Nz \, \operator{\mathcal{N}} \varphi_{\{\theta_a\}}(\{z_a\} )\ket{\{z_a\}}, \, \operator{P}\ket{\{\theta_a\}} =  \sqrt{N!}\int d^Nz \, \operator{\mathcal{P}}_N \varphi_{\{\theta_a\}}(\{z_a\} )\ket{\{z_a\}} 
%\end{eqnarray}
%où les opérateurs associés agissant sur les fonctions d’onde à $N$ particules sont :
%\begin{eqnarray}
%	\operator{ \mathcal{N}}  =  \sum_{k = 1}^N 1 = N ,~\operator{ \mathcal{P}}_N  = -i \sum_{k = 1}^N k =- i\sum_{k = 1}^N \operator{\partial}_{z_k}	
%\end{eqnarray}
%
%Ces formes découlent directement des règles de commutation canonique (\ref{chap:1:com.1}) et de la définition des opérateurs en seconde quantification (\ref{chap:eq.vide.fock}) (cf. annexes \ref{annex:N.part}).
%
%\subsubsection{Conclusion}
%Ainsi, les opérateurs $\operator{Q}$ , $\operator{P}$ et $\operator{H}$ possèdent une structure diagonale commune dans la base des états propres $\ket{\{\theta_a\}}$, révélant la nature intégrable du modèle de Lieb-Liniger. Leurs valeurs propres sont respectivement les 0e, 1er et 2e moments des rapidités. Cette structure permet de généraliser la construction à une hiérarchie complète d’observables conservées, qui seront présentées au chapitre suivant.
%

\subsection{Fonction d’onde et Hamiltonien et moment à 2 corps}

\paragraph{Introduction au système de deux bosons avec interaction de contact.}

Considérons maintenant un système de deux bosons confinés dans une boîte unidimensionnelle de longueur \(L\), avec des conditions aux limites périodiques. Contrairement au cas à une seule particule, une interaction de contact intervient ici dans la dynamique. L’Hamiltonien à deux particules s’écrit :
\begin{eqnarray}
	\operator{\mathcal{H}}_2 = \operator{\mathcal{K}}_2 + \operator{\mathcal{V}}_2, \quad \text{avec} \quad \operator{\mathcal{K}}_2 = - \frac{1}{2} \partial_{z_1}^2 - \frac{1}{2} \partial_{z_2}^2, \, \text{et} \,\, \operator{\mathcal{V}}_2 = g \, \delta(z_1 - z_2). \label{chap:1:hal.mod.2.part.3}
\end{eqnarray}

On rappelle que, pour des particules de masse unitaire (i.e., \(\hbar = m = 1\)), les énergies propres de l’opérateur cinétique \(\operator{\mathcal{K}}_2\), associées aux fonctions d’onde symétrisées \(\varphi_{\{ \theta_1 , \theta_2 \}}\), sont données par :
\begin{eqnarray}
	\varepsilon(\theta_1) + \varepsilon(\theta_2) = \frac{\theta_1^2}{2} + \frac{\theta_2^2}{2}.
\end{eqnarray}

Afin de simplifier le problème, nous nous plaçons dans le référentiel du centre de masse.

\paragraph{Changement de variables : coordonnées du centre de masse et relative.}

En première quantification, on introduit les nouvelles variables :
\(
Z = \frac{z_1 + z_2}{2} \, \text{(centre de masse)}, \, Y = z_1 - z_2 \, \text{(coordonnée relative)}.
\)
Dans ce changement de variables, l’opérateur laplacien total 
\(
\partial_{z_1}^2 + \partial_{z_2}^2 
\)
devient
\(
\frac{1}{2} \partial_Z^2 + 2 \, \partial_Y^2.
\)
L’Hamiltonien~\eqref{chap:1:hal.mod.2.part.3} se décompose alors en la somme de deux Hamiltoniens agissant respectivement sur \(Z\) et \(Y\) :
\begin{eqnarray}\label{chap:1:hal.mod.2.part.4}
	\operator{\mathcal{H}}_2 = -\frac{1}{4} \partial_Z^2 + \operator{\mathcal{H}}_{\text{rel}}, \qquad \text{avec} \quad \operator{\mathcal{H}}_{\text{rel}} = - \partial_Y^2 + g \, \delta(Y).
\end{eqnarray}

\paragraph{Résolution du problème du centre de masse et de la coordonnée relative.}

L’Hamiltonien du centre de masse, \(-\frac{1}{4} \partial_Z^2\), décrit une particule de masse totale \(\bar{m} = 2\). Ses états propres sont des ondes planes associées à une énergie \(\overline{\theta}^2\), avec :
\(
\overline{\theta} = \frac{\theta_1 + \theta_2}{2},
\)
jouant ici un rôle analogue à celui d’un pseudo-moment associé dans le référentielle de laboratoire.
Le Hamiltonien relatif, \(\operator{\mathcal{H}}_{\text{rel}}\), correspond quant à lui à une particule de masse réduite \(\tilde{m} = \frac{1}{2}\) soumise à un potentiel delta centré en \(Y = 0\). Son équation propre s’écrit :
\begin{eqnarray}\label{chap:1:hal.mod.2.part.5}
	- \partial_Y^2 \, \tilde{\varphi}(Y) + g \, \delta(Y) \, \tilde{\varphi}(Y) = \tilde{\varepsilon} \, \tilde{\varphi}(Y),
\end{eqnarray}
où \(\tilde{\varepsilon}\) désigne l’énergie associée au mouvement relatif.
%%%%%%%%%%%%%%

\paragraph{Forme symétrique de la fonction d'onde pour bosons.}
Dans le référentiel du centre de masse. Le système est le même que que celuis d'un particules de masse $\tilde{m}= \frac{1}{2}$.
Le système étant composé de particules bosoniques, on cherche une solution symétrique que l’on écrit sous la forme  :
\begin{eqnarray}
	\tilde{\varphi}(Y) ~=~a~e^{i\frac{1}{2} \tilde{\theta} \vert Y \vert } + b~e^{-i\frac{1}{2} \tilde{\theta}\vert Y \vert } ~\propto~  \sin\left( \frac{1}{2} (\tilde{\theta} |Y| + \Phi ) \right). \label{eq:ansatz.boson}
\end{eqnarray}
Le paramètre \( \tilde{\theta} = \theta_1 - \theta_2 \) joue ici un rôle analogue à celui d’un pseudo-moment associé à la coordonnée relative,
est  la phase s'écrit
\begin{eqnarray}
	\Phi(\tilde{\theta}) &=& 2 \arctan\left (\frac{1}{i} \frac{a+b}{a-b}\right),	\label{chap:1:dif.mod.2.part.1} 
\end{eqnarray}
car \( a\exp(ix)+b\exp(-ix) = 2\sqrt{ab}\sin\left(x+\arctan\left(-i\, \frac{a+b}{a-b}\right)\right) \). Pour $\tilde{\theta}<0$, les termes exponentiels \( \exp(i\tilde{\theta} \vert Y \vert/2 ) \) et \( \exp(-i\tilde{\theta} \vert Y \vert/2 ) \) correspondent aux paires de particules entrantes et sortantes d’un processus de diffusion à deux corps.

En réinjectant l’ansatz~\eqref{eq:ansatz.boson} dans l’équation relative
\eqref{chap:1:hal.mod.2.part.5}, on obtient l’énergie propre
\(\tilde{\varepsilon}\) du problème réduit.  
Elle prend la forme cinétique usuelle
\(\tfrac{1}{2}\times\text{masse}\times\text{vitesse}^{2}\).  
La masse réduite vaut ici \(\tilde{m}= \frac{1}{2}\) et le paramètre
\(\tilde{\theta}\) joue le rôle d’une impulsion ; ainsi
\begin{equation}
   \tilde{\varepsilon}(\tilde{\theta})
   \;=\;\frac{\tilde{\theta}^{2}}{4}.
   \label{eq:energie_relative}
\end{equation}

Cette énergie gouverne la décroissance exponentielle de la fonction
d’onde dans la coordonnée relative : plus \(\tilde{\theta}\) est grand,
plus l’état est localisé autour de \(Y=0\), signe d’une interaction
attractive plus forte entre les deux bosons.

La fonction d’onde relative présente des oscillations de fréquence \(\tilde{\theta}/2\), et son énergie croît avec \(\tilde{\theta}^2\). Cette solution correspond à un état de diffusion à deux corps en interaction ponctuelle. En revanche, une décroissance exponentielle autour de $Y=0$ n’apparaît que dans le cas d’un couplage attractif ($g<0$), où des états liés peuvent se former.\\


L’énergie totale se décompose enfin en la somme du mouvement du centre
de masse et du mouvement relatif :
\(
   \overline{\theta}^{2}
   \;+\;
   \tilde{\varepsilon}(\tilde{\theta})
   \;=\;
   \varepsilon(\theta_{1})
   \;+\;
   \varepsilon(\theta_{2}),
\)
où \(\overline{\theta}= \tfrac{\theta_{1}+\theta_{2}}{2}\).


%%%%%%%%%%%%%%%%%%%%%%%%%%%
\paragraph{Condition de discontinuité à cause du potentiel delta.}
En raison de la présence du potentiel delta centré en $Y = 0$, la dérivée première de la fonction d’onde $\tilde{\varphi}(Y)$ présente une discontinuité en ce point. En effet, le potentiel étant infini en $Y = 0$, la phase $\Phi$ du régime symétrique est déterminée en intégrant l’équation du mouvement autour de la singularité. En intégrant entre $- \epsilon$ et $+ \epsilon$ et en faisant tendre $\epsilon \to 0$, on obtient la condition de saut de la dérivée :
\begin{eqnarray*}
	\underset{ \epsilon \to 0 }{\lim} \int_{-\epsilon}^{+\epsilon}  - 	\partial_Y^2\tilde{\varphi}(Y) + g \delta ( Y )\tilde{\varphi}(Y) \, dY  & = & \underset{ \epsilon \to 0 }{\lim}  \int_{-\epsilon}^{+\epsilon}  \tilde{\varepsilon}(\tilde{\theta})d Y ,\\
	\\
	\tilde{\varphi}'(0^+) - \tilde{\varphi}'(0^-) - g \tilde{\varphi} (  0 ) & = & 0 .
\end{eqnarray*}


%%%%%%%%%%%%%%%
\paragraph{Détermination de la phase $\Phi$.}
Et en évaluant la discontinuité de sa dérivée au point $Y = 0$, on trouve que la phase $\Phi$ satisfait la condition :
%\begin{equation}
%	\tan\left( \frac{\Phi}{2} \right) = \frac{\tilde{\theta}}{c}.
%\end{equation}
\begin{eqnarray}\label{chap:1:dif.mod.2.part.2}
	\Phi(\tilde{\theta}) & = & 2 \arctan (\tilde{\theta}/g) \in [ - \pi , +\pi ].
\end{eqnarray}
%Cette relation exprime l’impact de l’interaction de type delta sur le déphasage de la fonction d’onde liée.
Cette relation exprime l’effet de l’interaction delta sur la phase de la fonction d’onde à deux particules. On en déduit que plus le couplage \( g \) est fort (\( g \to \infty \)), plus la phase \( \Phi \) se rapproche de zéro. Cela correspond à une fonction d’onde qui s’annule en \( Y = 0 \), caractéristique d’un régime d’imperméabilité totale.

À l’inverse, dans la limite d’une interaction faible (\( g \to 0 \)), la phase \( \Phi \) tend vers \( \pi \) (ou \( -\pi \), selon le signe de \( \tilde{\theta} \)). Dans ce cas, la discontinuité de la dérivée de la fonction d’onde au point \( Y = 0 \) devient négligeable, ce qui traduit une interaction absente entre les deux particules.


%%%%%%%%%%%%%%%%%%%%%%%%%%%%%%%%%%%
%\paragraph{Phase de diffusion à un corp.}
%Les équations \eqref{chap:1:dif.mod.2.part.1} et \eqref{chap:1:dif.mod.2.part.2}  et en remarquant que pour $z \in \mathbb{C} \backslash \{ \pm i \} 2\artan(z) = i \ln \left( \frac{ 1 - i z }{1+iz} \right ) $ soit $\exp(2i\arctan(x)) = (1 + ix)/(1 - ix)$ et $\Phi(\tilde{\theta}) = i \ln ( - b/a ) $  donne rapport entre les amplitudes $a$ et $b$ de la fonction d'onde \eqref{eq:ansatz.boson} définit la phase de diffusion / {\em matrice diffusion} $S( \tilde{\theta}) \doteq e^{i\Phi ( \tilde{\theta}  ) }$  :

%\begin{eqnarray}
%	e^{i\Phi ( \tilde{\theta}  ) } &=& -\frac{a}{b} ~=~\frac{1 +i\tilde{\theta}/g} { 1 - i\tilde{\theta}/g} .\label{chap:1:dif.mod.2.part.3}
%\end{eqnarray}

\paragraph{Phase de diffusion à deux corps.}

En combinant les équations~\eqref{chap:1:dif.mod.2.part.1} et~\eqref{chap:1:dif.mod.2.part.2} avec l’identité analytique valable pour tout
\(z \in \mathbb{C}\setminus\{\pm i\}\),
\(
2\arctan(z)=i\ln\!\left(\frac{1-iz}{1+iz}\right)
\)
\ie 
\(
e^{2i\arctan(z)}=\frac{1+iz}{1-iz},
\)
on obtient que le rapport des amplitudes \(a\) et \(b\) de la fonction
d’onde relative~\eqref{eq:ansatz.boson} définit la {\em phase de diffusion }
\(
\Phi(\tilde{\theta}) = i\ln\!\left(-\frac{b}{a}\right).
\)
On introduit alors la {\em matrice de diffusion} (ou {\em facteur de diffusion}) noté  \(S(\theta) \) , définie comme une phase complexe : 
\begin{eqnarray}\label{chap:1:def.mat.dif.1}
	S(\theta) \;\doteq\; e^{i\Phi(\theta)}	
\end{eqnarray}
Dans le cas d’une interaction de type delta, cette fonction prend la forme explicite :
\begin{eqnarray}\label{chap:1:dif.mod.2.part.3}
	S(\tilde{\theta}) \; = \; \frac{1 + i\,\tilde{\theta}/g}{1 - i\,\tilde{\theta}/g}.%\tag{\ref{chap:1:dif.mod.2.part.3}}
\end{eqnarray}
%où \(g\) est le paramètre d’interaction et
%\(\tilde{\theta} = \theta_1 - \theta_2\) le pseudo‑moment relatif.  
Cette expression, unitaire et analytique, caractérise entièrement la diffusion élastique à deux corps dans le modèle considéré.



\paragraph{Lien entre la phase de diffusion et le décalage temporel — interprétation semi-classique}
%
%Il a été souligné par {\color{black}Eisenbud (1948)} et {\color{black}Wigner (1955)} que la phase de diffusion peut être interprétée, de manière semi-classique, comme un {\em décalage temporel}. Esquissons brièvement l'argument de {\color{black}Wigner (1955)}.Tout d'abord, notons que, pour une particule unique, une approximation simple d’un paquet d’ondes peut être obtenue en superposant deux ondes planes avec des moments $\tilde{\theta}/2$ et $\tilde{\theta}/2 + \delta \tilde{\theta}$, respectivement :
%
%Comme l'ont montré Eisenbud (1948) et Wigner (1955), la phase de diffusion $\delta(\theta)$ peut être interprétée, dans une approche semi-classique, comme un décalage temporel. Pour en comprendre l’origine, considérons l’argument esquissé par Wigner.
%
%L’idée est de construire un paquet d’ondes incident à partir de deux ondes planes proches, de moments $\tilde{\theta}/2$ et $\tilde{\theta}/2 + \delta \tilde{\theta}$ :

Wigner (1955), à la suite d’Eisenbud (1948), a mis en évidence un lien entre la {\bf phase de diffusion} et un {\em décalage temporel}, interprétation qui peut être éclairée dans une perspective semi-classique. L’idée de Wigner repose sur l’analyse d’un paquet d’ondes incident, constitué de la superposition de deux ondes planes de moments voisins, $\tilde{\theta}/2$ et $\tilde{\theta}/2 + \delta \tilde{\theta}$ :
\begin{eqnarray}
	\tilde{\varphi}_{\text{inc}}(Y) & \propto & e^{i\frac{1}{2}\tilde{\theta} \vert Y\vert} + e^{i\frac{1}{2}\left(\tilde{\theta} + 2\delta \tilde{\theta} \right) \vert Y\vert}.
\end{eqnarray}
Cette superposition évolue dans le temps comme :
\begin{eqnarray}
\tilde{\varphi}_{\text{inc}}(Y, t) &\propto &  e^{i\left( \frac{1}{2} \tilde{\theta}\vert Y\vert - t\,\tilde{\varepsilon}(\tilde{\theta}) \right)} + e^{i\left( \frac{1}{2}\left(  \tilde{\theta} + 2\delta \tilde{\theta} \right) \vert Y\vert - t\,\tilde{\varepsilon}(\tilde{\theta} + 2\delta \tilde{\theta}) \right)}.
\end{eqnarray}
%où l'on a utilisé l'expression de l'énergie réduite : $\tilde{\varepsilon}(\tilde{\theta}) = \tilde{\theta}^2 / 4$.
Le centre de ce 'paquet d'ondes' se situe à la position où les phases des deux termes coïncident, c'est-à-dire au point où $\vert Y\vert\delta \tilde{\theta}  - t[\tilde{\varepsilon}(\tilde{\theta} + 2\delta \tilde{\theta} ) - \tilde{\varepsilon}(\tilde{\theta})] = 0$, ce qui donne $\vert Y\vert \simeq \tilde{\theta} t$ avec la vitesse réduite $\tilde{\theta} = 2 \tilde{\varepsilon}'(\tilde{\theta}) $. %Ainsi, il s'agit effectivement d'un 'paquet d'ondes' se déplaçant à la vitesse $\theta$. Ensuite, considérons deux particules entrantes dans un état tel que le centre de masse $Z = (z_1 + z_2)/2$ ait une impulsion $\theta_1 - \theta_2$, tandis que la coordonnée relative $Y = z_1 - z_2$ se trouve dans un 'paquet d'ondes' se déplaçant à la vitesse $ (\theta_1 - \theta_2)/2$,
Selon les équations (\ref{eq:ansatz.boson}) et (\ref{chap:1:dif.mod.2.part.3}), l'état sortant de la diffusion correspondant serait :
\begin{eqnarray}
	\tilde{\varphi}_{outc} ( Y, t ) & \propto & -e^{i\Phi(\tilde{\theta})}e^{-i\frac{1}{2}\tilde{\theta} \vert Y\vert} - e^{i\Phi(\tilde{\theta} + 2 \delta \tilde{\theta} )}e^{-i\frac{1}{2}\left(\tilde{\theta} + 2\delta \tilde{\theta} \right) \vert Y\vert}. %\tag{2}
\end{eqnarray}
En répétant l'argument précédent de la stationnarité de phase, on trouve que la coordonnée relative est à la position $\vert Y \vert  \simeq \tilde{\theta} t - 2\Phi'( \tilde{\theta})$ au moment $t$. %Étant donné que le centre de masse n'est pas affecté par la collision et se déplace à la vitesse de groupe $\tilde{\theta} =(\theta_1 + \theta_2)/2$, nous constatons que la position des deux particules semiclassiques après la collision sera
\begin{eqnarray}
	\vert Y \vert & \simeq & 	\tilde{\theta} t  - 2 \Delta (\tilde{\theta} )
\end{eqnarray}
où le {\bf déplacement de diffusion} $\Delta (\theta)$ est donné par la dérivée de la {\em phase de diffusion},
\begin{eqnarray}\label{eq:I-1-16}
	\Delta ( \theta ) & \doteq & \frac{ d \Phi }{ d \theta } ( \theta )= \frac{ 2 g }{ g^2 + \theta^2} . 	
\end{eqnarray}


%\paragraph{Retour aux coordonnées du laboratoire.}
%En revenant aux coordonnées d'origine (celles du laboratoire), on constate que la fonction d'onde à deux corps 
%\(
%	\varphi_{\{\theta_1 , \theta_2\}} (z_1, z_2) = \langle \emptyset \vert \operator{\Psi} (z_1)\operator{\Psi} (z_2) \vert \{\theta_1, \theta_2\} \rangle,
%\)
%avec \(z_1 < z_2\) , (ie $Y>0$) . Et le centre de masse sur le mouvement
%\(
%	Z  =  \overline{\theta} t.	
%\)
%avec,  on rappelle , $\overline{\theta}$ la vitesse de groupe dans le référentielle de laboratoire.\\
%Nous constatons que la position des deux particules semiclassiques après la collision sera
%\begin{eqnarray}
%	z_1 ~=~ Z + \frac{Y}2 ~\simeq ~ \theta_1 t - \Delta(\theta_1 - \theta_2), & & 	z_2 ~=~ Z - \frac{Y}2 ~\simeq ~ \theta_2t + \Delta(\theta_1 - \theta_2),
%\end{eqnarray}

%avec  $\theta_1$ et $\theta_2$ on rappelle définie tel que 
%\(
%	\tilde{\theta} ~=~\theta_1 - \theta_2 , \,	\overline{\theta}~=~\frac{\theta_1 + \theta_2}{2}.	
%\)
%On remarquant que 
%\begin{eqnarray*}
%	z_1 \theta_1  + z_2  \theta_2 ~=~ 2Z\overline{\theta} + \frac{1}{2}Y\tilde{\theta}, & & z_1 \theta_2  + z_2  \theta_1 ~=~ 2Z\overline{\theta} - \frac{1}{2}Y\tilde{\theta}. 
%\end{eqnarray*}
%Ce qui est en accod avec la masse total $\overline{m} = 2$ et la masse résuite $\tilde{m} = \frac{1}{2}$ \\
%Ce qui nous motive à multiplier la fonction d'onde dans le référentiel du centre de masse \eqref{eq:ansatz.boson} par $\exp(2iZ\overline{\theta})$ pour obtenir 

%\begin{eqnarray}\label{eq:I-1-10}
%	\varphi_{\{\theta_1 , \theta_2\}}(z_1 , z_2) & \propto &  \left \{ \begin{array} { c cl} ( \theta_2 - \theta_1 - ic) e^{ i z_1 \theta_1 + iz_2 \theta_2 } - ( \theta_1 - \theta_2 - ic) e^{ i z_1 \theta_2 + iz_2 \theta_1} & \mbox{si} & z_1 < z_2 \\ (z_1 \leftrightarrow z_2) & \mbox{si} & z_1 > z_2 \end{array} \right.
%\end{eqnarray}

%correspondant aux valeurs propres

%\begin{eqnarray}
%	\varepsilon(\theta_1 , \theta_2) ~=~ \overbrace{ \overline{\theta}^2}^{\overline{\varepsilon}(\overline{\theta})}	 + \overbrace{\frac{1}{4} \tilde{\theta}^2}^{\tilde{\varepsilon}(\tilde{\theta})} ~=~ \frac{\theta_1}{2} + \frac{\theta_2}{2}.	
%\end{eqnarray}

%Pour $\theta_1 > \theta_2$, les deux termes $e^{iz_1 \theta_1 + iz_2 \theta_2 }$ et $e^{iz_1 \theta_2 + iz_2 \theta_1 }$ correspondent aux paires de particules entrantes et sortantes dans un processus de diffusion à deux corps. Le rapport de leurs amplitudes est la phase de diffusion à deux corps \eqref{chap:1:dif.mod.2.part.3} reste inchangé

%\begin{eqnarray}\label{chap:1:dif.mod.2.part.4}
%	e^{i\Phi ( \theta_1 - \theta_2  ) }~=~ -\frac{a}{b} ~=~\frac{\theta_1 - \theta_2  -ic} { \theta_2 - \theta_1  - ic}. 
%\end{eqnarray}


%%%%%%%%%%%%%%%%%%%%%%%%%%
\paragraph{Retour aux coordonnées du laboratoire.}

En revenant aux coordonnées du laboratoire, la fonction d’onde à deux corps s’écrit
\(
	\varphi_{\{\theta_1 , \theta_2\}} (z_1, z_2) 
	= \langle \emptyset \vert \operator{\Psi} (z_1)\operator{\Psi} (z_2) \vert \{\theta_1, \theta_2\} \rangle/\sqrt{2},
\)
dans le cas \(z_1 < z_2\), c’est-à-dire pour une séparation relative \(Y = z_1 - z_2 < 0\) (on pourra symétriser ultérieurement).  
Dans le référentiel du laboratoire, le centre de masse évolue selon
\(
	Z = \frac{z_1 + z_2}{2} = \overline{\theta}\,t.
\)
%où l’on rappelle que \(\overline{\theta} = \frac{\theta_1 + \theta_2}{2}\) est la vitesse de groupe du système dans le référentiel laboratoire.
Ainsi, la position semi-classique des deux particules après la collision s’écrit
\begin{eqnarray}
	z_1 = Z + \frac{Y}{2} \;\simeq\; \theta_1 t - \Delta(\theta_1 - \theta_2),\quad
	z_2 = Z - \frac{Y}{2} \;\simeq\; \theta_2 t + \Delta(\theta_1 - \theta_2),
\end{eqnarray}
%où \(\Delta(\theta_1 - \theta_2)\) représente le décalage dû à l’interaction entre les deux particules.
%On rappelle les définitions :
%\[
%	\tilde{\theta} = \theta_1 - \theta_2, 
%	\quad
%	\overline{\theta} = \frac{\theta_1 + \theta_2}{2}.
%\]
On peut vérifier les identités utiles suivantes :
\begin{eqnarray*}
	z_1 \theta_1 + z_2 \theta_2 = 2Z \overline{\theta} + \frac{1}{2} Y \tilde{\theta}, \quad
	z_1 \theta_2 + z_2 \theta_1 &=& 2Z \overline{\theta} - \frac{1}{2} Y \tilde{\theta},
\end{eqnarray*}
ce qui est en accord avec les masses associées : masse totale \(\overline{m} = 2\), masse réduite \(\tilde{m} = \frac{1}{2}\).

Cela nous motive à multiplier l’ansatz dans le référentiel du centre de masse (équation~\eqref{eq:ansatz.boson}) par un facteur de phase globale \(\exp(2iZ\overline{\theta})\) pour revenir à la représentation dans le laboratoire. On obtient alors l’expression de la fonction d’onde :
\begin{eqnarray}\label{eq:I-1-10}
	\varphi_{\{\theta_1 , \theta_2\}}(z_1 , z_2) & \propto &  \left \{ \begin{array} { c cl} ( \theta_2 - \theta_1 - ig) e^{ i z_1 \theta_1 + iz_2 \theta_2 } - ( \theta_1 - \theta_2 - ig) e^{ i z_1 \theta_2 + iz_2 \theta_1} & \mbox{si} & z_1 < z_2 \\ (z_1 \leftrightarrow z_2) & \mbox{si} & z_1 > z_2 \end{array} \right.
\end{eqnarray}

%Cette fonction d’onde correspond à une valeur propre d’énergie donnée par la somme des énergies associées aux deux degrés de liberté :

%\begin{equation}
%	\varepsilon(\theta_1 , \theta_2) 
%	= \underbrace{\overline{\theta}^2}_{\overline{\varepsilon}(\overline{\theta})}
%	+ \underbrace{\frac{1}{4} \tilde{\theta}^2}_{\tilde{\varepsilon}(\tilde{\theta})}
%	= \frac{\theta_1^2}{2} + \frac{\theta_2^2}{2}.
%\end{equation}

Pour \(\theta_1 > \theta_2\), les deux termes exponentiels 
\(e^{i z_1 \theta_1 + i z_2 \theta_2}\) et \(e^{i z_1 \theta_2 + i z_2 \theta_1}\)
correspondent respectivement aux ondes entrantes et sortantes dans le canal de diffusion à deux corps \cite{Bouchoule_2022}.  
Le rapport de leurs amplitudes définit la {\bf phase de diffusion} $\Phi$ et  {\bf matrice diffusion} $S$  à deux corps \eqref{chap:1:dif.mod.2.part.3} , reste inchangé mais voici une autre écriture :

\begin{equation}\label{chap:1:dif.mod.2.part.4}
	S(\theta_1- \theta_2)  
	= \frac{\theta_1 - \theta_2 - ig}{\theta_2 - \theta_1 - ig}.
\end{equation}

Cette phase caractérise entièrement le processus de diffusion dans le modèle de Lieb-Liniger à deux particules.

\paragraph{Conditions périodiques et équations de Bethe pour deux bosons {\color{red}(à révoir)}.}

%La fonction d’onde obtenue par Bethe ansatz (voir
%\eqref{eq:I-1-10}) est, pour $z_{1}<z_{2}$,
%\[
%	\varphi_{\{\theta_{1},\theta_{2}\}}(z_{1},z_{2})
%		= a\,e^{i\theta_{1}z_{1}+i\theta_{2}z_{2}}
%		+b\,e^{i\theta_{2}z_{1}+i\theta_{1}z_{2}},
%	\quad
%	a=\theta_{2}-\theta_{1}-ic,\;
%	b=-(\theta_{1}-\theta_{2}-ic).
%\]

%\medskip
%\subparagraph{Périodicité sur $z_{2}$.}  
%On impose à la fonction d’onde obtenue par Bethe ansatz (voir
%\eqref{eq:I-1-10})
%\(
%	\varphi_{\{\theta_{1},\theta_{2}\}}(z_{1},z_{2}\!=\!L)
%	=
%	\varphi_{\{\theta_{1},\theta_{2}\}}(z_{1},z_{2}\!=\!0)
%\)
%avec $0<z_{1}<z_{2}=L$.  
%Au point $z_{2}=L$ on reste dans le secteur $z_{1}<z_{2}$, tandis qu’au point $z_{2}=0$ le domaine pertinent devient $z_{2}<z_{1}$;  la fonction d’onde y est obtenue en échangeant $z_{1}\leftrightarrow z_{2}$ , soit 
%\(
%	\varphi_{\{\theta_{1},\theta_{2}\}}(z_{1},\!L)
%	=
%	\varphi_{\{\theta_{1},\theta_{2}\}}(0 , z_{1})
%\)
%.  
%On obtient ainsi
%\begin{eqnarray*}
%	a\,e^{i\theta_{1}z_{1}+i\theta_{2}L}+b\,e^{i\theta_{2}z_{1}+i\theta_{1}L} & = &
%	a\,e^{i\theta_{2}z_{1}}\,e^{i\theta_{1}\! \cdot 0} + b \,e^{i\theta_{1}z_{1}}\,e^{i\theta_{2}\! \cdot 0},	
%\end{eqnarray*}
%avec la condition $z_1< z_2$, avec le rapport $a$ et $b$ vérifiant \eqref{chap:1:dif.mod.2.part.4} de la sorte $-b/a = e^{i\Phi(\theta_1 - \theta_2)}$ .

%%%%%%%%%%%%%%%%

\subparagraph{Périodicité en \( z_2 \).}  
On impose une condition de périodicité sur la fonction d’onde obtenue par ansatz de Bethe (voir équation~\eqref{eq:I-1-10}) :
\(
	\varphi_{\{\theta_1,\theta_2\}}(z_1, z_2 = L) = \varphi_{\{\theta_1,\theta_2\}}(z_1, z_2 = 0),
\)
avec \( 0 < z_1 < z_2 = L \).  
Au point \( z_2 = L \), la configuration reste dans le secteur \( z_1 < z_2 \), tandis qu’à \( z_2 = 0 \), on entre dans le secteur \( z_2 < z_1 \). La continuité de la fonction d’onde impose alors d’échanger les coordonnées \( z_1 \leftrightarrow z_2 \) :
\(
	\varphi_{\{\theta_1,\theta_2\}}(z_1, L) = \varphi_{\{\theta_1,\theta_2\}}(0, z_1).
\)
En utilisant l’expression explicite de l’ansatz dans les deux secteurs, on obtient l’égalité suivante :
\begin{eqnarray*}
	a\,e^{i\theta_1 z_1 + i\theta_2 L} + b\,e^{i\theta_2 z_1 + i\theta_1 L}
	&=& a\,e^{i\theta_2 z_1} + b\,e^{i\theta_1 z_1}.
\end{eqnarray*}
%où le second membre correspond à la fonction d’onde dans le secteur \( z_2 < z_1 \), évaluée en \( z_2 = 0 \) et \( z_1 = z_1 \).  
%La condition de périodicité impose donc :
%\[
%	a\,e^{i\theta_1 z_1 + i\theta_2 L} + b\,e^{i\theta_2 z_1 + i\theta_1 L}
%	= a\,e^{i\theta_2 z_1} + b\,e^{i\theta_1 z_1}.
%\]
Cette relation, valable pour tout \( z_1 \in [0,L] \), fixe une contrainte sur le rapport \( b/a \). En utilisant l’expression de la phase de diffusion introduite en \eqref{chap:1:dif.mod.2.part.4} pour $z_1<z_2$ :
\begin{eqnarray*}
	-\frac{b}{a} = e^{i\Phi(\theta_1 - \theta_2)},
\end{eqnarray*}
on obtient une condition sur les phases \( \theta_1 \) et \( \theta_2 \), cœur de la quantification imposée par le formalisme de Bethe.

%\[
%	( \theta_2 - \theta_1 - ig)\,e^{i\theta_{1}z_{1}+i\theta_{2}L}
%	- ( \theta_1 - \theta_2 - ig)\,e^{i\theta_{2}z_{1}+i\theta_{1}L}
%	=
%	( \theta_2 - \theta_1 - ig)\,e^{i\theta_{2}z_{1}}\,e^{i\theta_{1}\! \cdot 0}
%	- ( \theta_1 - \theta_2 - ig)\,e^{i\theta_{1}z_{1}}\,e^{i\theta_{2}\! \cdot 0}.
%\]
En identifiant les coefficients de $e^{i\theta_{1}z_{1}}$ et
$e^{i\theta_{2}z_{1}}$ indépendamment, on obtient
\(
	e^{i\theta_{2}L}\;a = b, 
	\,
	e^{i\theta_{1}L}\;b = a,
\)
c’est‑à‑dire l'équations de Bethe
%\begin{equation}\label{eq:PC2}
%	e^{i\theta_{2}L} = \frac{b}{a}
%	= \frac{\theta_{1}-\theta_{2}+ic}{\theta_{2}-\theta_{1}+ic},
%\quad
%	e^{i\theta_{1}L} = \frac{a}{b}
%	= \frac{\theta_{2}-\theta_{1}+ic}{\theta_{1}-\theta_{2}+ic}.
%\end{equation}
\begin{eqnarray}\label{eq:PC2}
	e^{i\theta_{1}L}\,e^{i\Phi(\theta_{1}-\theta_{2})} = -1,
	\qquad
	e^{i\theta_{2}L}\,e^{i\Phi(\theta_{2}-\theta_{1})} = -1.	
\end{eqnarray}
En prenant le logarithme on obtient les \emph{équations de Bethe à deux
particules} :
\begin{equation}\label{eq:Bethe2}
	\theta_{1}L + \Phi(\theta_{1}-\theta_{2}) = 2\pi I_{1}, 
	\qquad
	\theta_{2}L + \Phi(\theta_{2}-\theta_{1}) = 2\pi I_{2},
\end{equation}
où $I_{1},I_{2}\in\mathbb{Z}/2$ sont les nombres demis entiers. 

\subparagraph{Périodicité sur $z_{1}$.}  Le raisonnement symétrique conduit exactement aux mêmes égalités \eqref{eq:PC2} et \eqref{eq:Bethe2}.  Ces équations  constituent la quantification complète du gaz de Lieb–Liniger à deux bosons sur un cercle de longueur $L$ et seront le point de départ pour l’étude de l’état fondamental et des excitations.



\begin{figure}[H]
	\centering
  %\includegraphics[width=0.5\textwidth]{}
  %\caption{Gauche : La fonction d'onde (\ref{eq:I-1-10}) sur la ligne infinie correspond à un processus de diffusion à deux corps. Semiclassiquement, la phase de diffusion dans ce processus à deux corps se reflète dans le décalage de diffusion (\ref{eq:I-1-16}) : après la collision, la position de la particule a été déplacée d'une distance $\Delta ( \theta_1 - \theta_2 )$ . Droite : La fonction d'onde de Bethe (\ref{eq:I-2-17}) sur la ligne infinie correspond à un processus de diffusion à $N$-corps qui se factorise en des processus à deux corps (le décalage de diffusion $\Delta$ est également présent ici, mais il n'est pas représenté dans la caricature). Dans ce processus à $N$-corps, les rapidités $\theta_j$ sont les moments asymptotiques des bosons.}
  \label{}	
\end{figure}

\subsubsection{Interprétation physique pour deux particules et rôle de la rapidité}

Pour bien comprendre le sens physique des équations de Bethe~\eqref{eq:Bethe2}, nous avons commencer par le cas de deux particules. Dans ce cadre, les particules interagissent lorsqu'elles se croisent, et à chaque interaction elles acquièrent une {\bf phase de diffusion} $\Phi$. L’état propre du système est obtenu en imposant que l’onde multi-corps soit périodique sur un cercle de longueur $L$ : chaque particule effectue une rotation complète, accumule une phase cinématique $e^{i\theta L}$ liée à son mouvement libre, ainsi que des contributions de phase dues aux diffusions avec l’autre particule. Cela donne lieu à une quantification des {\bf pseudo-impulsions} $\theta$ via l’équation de Bethe.

\medskip

Pour décrire ces {\bf excitations} , ces {\bf quasi-particules}, on introduit le paramètre $\theta$ appelé {\bf rapidité}. Ce terme vient de la théorie relativiste \cite{ZAMOLODCHIKOV1979253,Babelon2003}, mais reste pertinent même dans des modèles non relativistes, comme ici. La rapidité est choisie comme {\bf paramètre spectral naturel} : elle étiquette les états propres, linéarise les relations de dispersion dans certains régimes, et elle simplifie considérablement les équations de Bethe.

\medskip

On peut interpréter chaque $\theta$ comme la {\bf vitesse d’une quasi-particule} : une entité collective qui se comporte comme une particule libre, mais qui tient compte des effets des interactions avec les autres. Contrairement à une particule élémentaire, une {\bf quasi-particule} est une excitation émergente du système à plusieurs corps : elle résume de façon efficace le comportement collectif d’une particule « habillée » par son environnement d’interactions. Cette notion est centrale en physique des systèmes quantiques à N corps, où les excitations ne sont plus des particules indépendantes, mais des objets collectifs.

Ainsi, dans l’image de Bethe, les états propres sont des configurations stables de $N$ quasi-particules de rapidités $\theta_1,\dots,\theta_N$, dont les valeurs sont quantifiées par les conditions d’interférence imposées par les équations de Bethe.



\section{Équation de Bethe et distribution de rapidité}

\subsection{Fonction d’onde dans le secteur ordonné et représentation de Gaudin}

Sans en donner ici la démonstration, donnons une forme généralisée de la fonction propre à $N$ particules [cf. équation \eqref{chap.1:eq.rapel.fonction.propre.N}], qui prolonge naturellement l’expression obtenue pour le cas à deux corps [cf. équation \eqref{eq:I-1-10}].
Dans le domaine $z_1 < z_2 < \cdots < z_N$, la fonction d’onde pour un état de Bethe à $N$ particules s’écrit ({\color{blue} \cite{Korepin_1997,Gaudin_2014,LL_1963_1,Franchini_2016,Wouters2015}}) :
\begin{eqnarray}
	\varphi_{\{\theta_a\}} ( z_1 , \cdots , z_N ) & = &  \frac{1}{\sqrt{N!}}\langle \emptyset \vert \operator{\Psi} ( z_1 ) \cdots \operator{\Psi} (z_N ) \vert \{ \theta_a \} \rangle \notag\\
	& \propto & \sum_\sigma (-1)^{|\sigma|} \left( \prod_{1 \leq a < b \leq N} (\theta_{\sigma(b)} - \theta_{\sigma(a)} - i g) \right) e^{i \sum_{j=1}^{N} z_j \theta_{\sigma(j)}},\label{eq:I-2-17}
\end{eqnarray}
où la somme s'étend sur toutes les permutations $\sigma$ de $\{1,\dots,N\}$. Le facteur $(-1)^{|\sigma|}$ est la signature de la permutation, et les amplitudes dépendent des différences de quasi-moments $\theta_j$ ainsi que du couplage $g$.
Cette fonction d’onde est ensuite étendue par symétrie aux autres domaines du type $z_{\pi(1)} < z_{\pi(2)} < \cdots < z_{\pi(N)}$ via des propriétés d’échange symétriques.

\vspace{1em}

\subsection{Conditions aux bords périodiques}

Les équations précédentes ont été établies pour un système défini sur la droite réelle. Cependant, dans une perspective thermodynamique, il est essentiel de considérer une densité finie $ N/L$. Cela peut être obtenu en compactifiant l’espace sur un cercle de longueur $L$, i.e. en imposant les {\em conditions aux bords périodiques}.

Concrètement, cela consiste à identifier $x = 0$ et $x = L$ et à exiger que la fonction d’onde soit périodique lorsqu’une particule fait le tour du système :
\begin{equation}\label{eq:periodic}
\varphi_{\{\theta_a\}}(x_1, \dots, x_{N-1}, L) = \varphi_{\{\theta_a\}}(0, x_1, \dots, x_{N-1}).
\end{equation}
Cette condition doit être satisfaite pour chaque particule. Or, déplacer la $j$-ième particule de $x_j$ à $x_j + L$ revient à la faire passer devant toutes les autres : cela introduit un facteur de diffusion à chaque croisement.

%\vspace{1em}

\subsubsection{Équations de Bethe exponentielles}

En imposant les conditions de périodicité sur la fonction d’onde de type Bethe~\eqref{eq:I-2-17}, on généralise l'éqution\eqref{eq:PC2} pour $N$ particules. On obtient que chaque moment $\theta_a$ doit satisfaire l’équation :
\begin{equation}
	e^{i \theta_a L} \prod_{b \ne a} S(\theta_a - \theta_b) = (-1)^{N-1}, \quad a = 1, \dots, N,
	\label{eq:bethe_exp}
\end{equation}
où la matrice diffusion $S(\theta)$ définie en \eqref{chap:1:def.mat.dif.1}, \eqref{chap:1:dif.mod.2.part.3} et \eqref{chap:1:dif.mod.2.part.4} est l’amplitude de diffusion à deux corps. Le signe $(-1)^{N-1}$ vient du fait que chaque permutation change la signature du déterminant dans la représentation de Gaudin.
%\vspace{1em}

\subsubsection{Équations de Bethe logarithmiques}

En prenant le logarithme du membre gauche et du membre droit de l’équation~\eqref{eq:bethe_exp}, on généralise l'équation \eqref{eq:Bethe2}. On obtient sa forme {\em logarithmique}:
\begin{equation}\label{chap:1:eq:EBA}
	L \theta_a + \sum_{b=1}^N \Phi(\theta_a - \theta_b) = 2\pi I_a, \qquad a = 1, \dots, N,
\end{equation}
où les $I_a$ sont des nombres quantiques fermioniques, c’est-à-dire des entiers $I_a \in \mathbb{Z}$ si $N$ est impair, et des demi-entiers $I_a \in \mathbb{Z} + \tfrac{1}{2}$ si $N$ est pair.

\medskip

Cette écriture révèle un lien direct avec un gaz de {\bf fermions libres} en une dimension : en interaction forte (\ie $g \to \infty$ c’est-à-dire si $\Phi = 0$), les équations \eqref{chap:1:eq:EBA} se réduisent à $L \theta_a = 2\pi I_a$, soit $\theta_a = 2\pi I_a / L$. Cela correspond exactement aux quantifications de l'impulsion pour des {\bf fermions libres sans spin}, dans une boîte de taille $L$, avec conditions aux bords périodiques.

\medskip

L’interprétation est alors la suivante : les solutions de Bethe ${ \theta_a }$ décrivent des quasi-particules interagissantes, dont la configuration est déterminée par l’ensemble des nombres quantiques ${ I_a }$, eux-mêmes analogues aux {\bf moments quantiques d’un gaz de fermions libres}. C’est pourquoi on parle de {\em nombres fermioniques} $I_a$ dans ce contexte.
%où les $I_a \in \mathbb{Z}$ (resp. $\mathbb{Z} + \tfrac{1}{2}$) sont des nombres entiers (resp. demi-entiers) si $N$ est impair (resp. pair).L'equation de Bethe \eqref{chap:1:eq:EBA} nous donne le lien entre la configuration des rapidité $\{ \theta_a \} $ et $N$ fermions sans interaction, ni spin et de vecteur d'onde $K_a = 2\pi I_a/L$. Dans la suite, les $I_a$ seront appelés {\bf nombres fermionique}. 

\medskip

Dans la configuration d’état fondamental (ou de type “mer de Fermi”), ces {\em nombres fermionique} sont pris de manière symétrique autour de zéro :
\begin{equation}\label{chap:1:eq:EBA.1}
	I_a = a - \frac{N+1}{2}, \quad \text{pour } a \in \llbracket 1 , N \rrbracket.
\end{equation}
ce qui correspond au choix symétrique des nombres quantiques pour l’état fondamental. Il en résulte une distribution uniforme des $\theta_a$ dans l’intervalle autour de zéro $[-\theta_{\mbox{\tiny max}} , \theta_{\mbox{\tiny max}} ]$ où \( \theta_{\mbox{\tiny max}} \) est le paramètre de Fermi (ou rapidité maximale).
%Ce choix garantit une distribution uniforme des $\theta_a$ à l’état fondamental.
%\vspace{1em}

\subsubsection{Interprétation physique}

Les équations de Bethe~\eqref{chap:1:eq:EBA} représentent une {\em quantification des pseudo‑impulsions $\theta_a$} des particules en interaction, résultant d’un {\em interféromètre multi‑corps sur le cercle} : chaque particule accumule une phase $e^{i \theta_a L}$ due au mouvement libre, ainsi que des phases de diffusion lorsqu’elle croise les autres.

Ce système d'équations détermine les états propres du système de Lieb–Liniger en volume fini, et joue un rôle fondamental dans la description exacte de ses propriétés thermodynamiques et dynamiques.


\subsection{Thermodynamique du gaz de Lieb–Liniger à l'état fondamental}

Dans la limite thermodynamique, le nombre de particules \( N \) et la longueur \( L \) du système tendent vers l'infini de telle sorte que leur rapport reste fini :
\begin{eqnarray*}
	\lim_{\underset{ N \to \infty}{L \to \infty}} \frac{N}{L} = n < \infty,
\end{eqnarray*}
où \( n \) désigne la densité linéique de particules.

\medskip

Considérons désormais le système à température nulle. L’état fondamental dans le secteur à nombre de particules fixé correspond à la configuration d’énergie minimale parmi les solutions des équations de Bethe \eqref{chap:1:eq:EBA}.

Dans la limite thermodynamique $(\underset{\text{\tiny therm}}{\lim} \equiv \underset{\underset{ N \to \infty}{L \to \infty}}{\lim} )$ , les valeurs de \( \theta_a \) deviennent quasi-continues, avec un espacement \( \theta_{a+1} - \theta_a = \mathcal{O}(1/L) \), et se condensent dans un intervalle symétrique autour de zéro $[-\theta_{\mbox{\tiny max}} , \theta_{\mbox{\tiny max}} ]$ . En supposant l'ordre \( I_a \geq I_b  \) implique \(\theta_a \geq \theta_b \), cet intervalle constitue ce qu'on appelle la {\em mer de Dirac} (ou sphère de Fermi en dimension un).

Nous introduisons la {\bf densité d’états}  \( \rho_s(\theta) \), définie par
\begin{eqnarray}
	 \rho_s(\theta_a) &\doteq & \frac{1}{L} \lim_{\text{\tiny therm}} \frac{|I_{a+1} - I_a|}{|\theta_{a+1} - \theta_a|},
\end{eqnarray}
soit en notant la fonction \( I(\theta_a) = I_a \)
\begin{eqnarray}
	2\pi \rho_s(\theta_a) &=&  \frac{2\pi}{L} \frac{\partial I}{\partial \theta}(\theta_a).
\end{eqnarray}
L’application des équations de Bethe sous forme logarithmique \eqref{chap:1:eq:EBA} conduit alors à
\begin{eqnarray}
	2\pi \rho_s(\theta_a) = 1 + \frac{1}{L} \sum_{b = 1}^N \Delta(\theta_a - \theta_b),
\end{eqnarray}
ce qui relie \( \rho_s \) à le déplacement de diffusion  \( \Delta \) définie dans l'équation  \eqref{eq:I-1-16}.

\medskip

Intéressons-nous maintenant à la {\em densité de particules dans l’espace des moments}, que l'on nome la {\bf distribution de rapidité macroscopique} et notée \( \rho(\theta) \), et définie par
\begin{eqnarray}
	L \rho(\theta) \delta \theta & \doteq & \mbox{nombre de quasi-particules ayant une rapidité dans } [\theta , \theta + \delta \theta].	
\end{eqnarray}
Autrement dit, dans un petit intervalle $\delta \theta $, le nombre total de particules ayant une rapidité dans cet intervalle est approximativement :
\begin{eqnarray}\label{chap.1:eq.dN.1}
	\delta N (\theta ) & = & L \rho (\theta) \delta \theta 	
\end{eqnarray}
Avec les rapidité $\theta_a$ ordonnées \ie $\theta_1 < \theta_1 < \cdots < \theta_N$. Pour des intervalle $[\theta_a , \theta_a + \delta \theta_a]$ assez petit $\delta \theta_a = \theta_{a+1} - \theta_a$, est correspond à $\delta N (\theta_a) = 1 $ quasi-particule (car une seule particule occupe l’intervalle entre deux rapidités consécutives). Dans la limite thermodynamique la dernier équation \eqref{chap.1:eq.dN.1} se réécrit comme :
\begin{eqnarray}
	\rho(\theta_a) & = & \underset{\text{\tiny therm}}{\lim} \frac{1}{L} \cdot \frac{1}{\theta_{a+1} - \theta_a} > 0.
\end{eqnarray}
On peut aussi écrire la distribution de rapidité sous la forme :
\begin{eqnarray}
	\rho(\theta) & = & \frac{1}{L} \sum_{a = 1}^N \delta ( \theta - \theta_a ) .	
\end{eqnarray}

Dans l’état fondamental, toutes les positions disponibles dans l’intervalle \( [-\theta_{\mbox{\tiny max}}, +\theta_{\mbox{\tiny max}}] \) sont occupées. On a donc :
\begin{eqnarray}\label{chap.1.rho.2}
	\rho(\theta) = \rho_s(\theta).
\end{eqnarray}

La quantité \( L \rho(\theta) d\theta \) représente le nombre de rapidités dans la cellule infinitésimale \( [\theta, \theta + d\theta] \), tandis que
\(
	N = L \int_{-\theta_{\mbox{\tiny max}}}^{+\theta_{\mbox{\tiny max}}} \rho(\theta)\, d\theta
\)
donne le nombre total de particules dans le système. Le passage de la somme discrète à l'intégrale dans le second membre de l'équation de Bethe permet d’écrire :
\begin{eqnarray}
	\frac{1}{L} \sum_{b = 1}^N \Delta(\theta_a - \theta_b) \underset{\text{\tiny therm}}{\longrightarrow }\int_{-\theta_{\mbox{\tiny max}}}^{+\theta_{\mbox{\tiny max}}} \Delta(\theta_a - \theta)\, \rho(\theta)\, d\theta.
\end{eqnarray}
Ainsi, l'équation pour la densité d'états devient :
\begin{eqnarray}\label{chap.1.rho.s.2}
	2\pi \rho_s(\theta) = 1 + \int_{-\theta_{\mbox{\tiny max}}}^{+\theta_{\mbox{\tiny max}}} \Delta(\theta - \theta')\, \rho(\theta')\, d\theta',
\end{eqnarray}
et, comme \( \rho = \rho_s \), à l'état fondamental, on obtient l’équation linéaire intégrale satisfaite par la densité de rapidités :
\begin{eqnarray}\label{chap.1.rho.3}
	\rho(\theta) - \int_{-\theta_{\mbox{\tiny max}}}^{+\theta_{\mbox{\tiny max}}} \frac{\Delta(\theta - \theta')}{2\pi} \rho(\theta')\, d\theta' = \frac{1}{2\pi}.
\end{eqnarray}


\subsection{Excitations élémentaires}

À partir de l’état fondamental dans le régime d’interaction forte, les excitations élémentaires du modèle de Lieb-Liniger ont été classifiées par Lieb en deux types distincts~\cite{LL_1963_2}. Ces excitations peuvent être comprises comme des perturbations de la mer de Fermi formée par les quasi-particules.

\begin{itemize}
    \item \textbf{Excitations de type I :} Il s’agit remplacer une quasi-particule avec un nombre fermionique au bord de Fermi $I_{N}$ (ou $I_{-N}$) par une quasi-particule avec un nombre fermionique $I'> I_N$ (resp.$I' <I_{-N}$), c’est-à-dire au-delà du bord de la mer de Fermi. Ces excitations sont analogues à des excitations de particules libres et, dans la limite des faibles interactions ($g \to 0$), leur relation de dispersion reproduit celle prédite par la théorie de Bogoliubov.

    \item \textbf{Excitations de type II :} Ces excitations sont de type particule-trou. Elles sont obtenues en remplaçant une quasi-particule de la mer de Fermi avec une nombre fermionique $I_a$ (\ie, en créant un trou dans l’état fondamental) et de le remplacer par une quasi-particule avec $I' = I_{N+1} = N/2$ ou $I' = I_{N-1} = -N/2$ et en réarrangeant les autres nombres fermionique. Cela correspond à une excitation interne du Fermi pseudo-mer, avec conservation du nombre de particules. Pour de petits $I'$, la dispersion est linéaire, correspondant à des modes phononiques. Dans la limite d’interaction faible \( g \to 0 \), ces excitations peuvent être interprétées comme des solitons sombres~\cite{Ishikawa1980SolitonsIA,Karpiuk_2015}.
\end{itemize}

Ces deux types d’excitations définissent ensemble le spectre complet du modèle de Lieb-Liniger et permettent d’accéder aux propriétés dynamiques du système, telles que les fonctions de réponse ou la structure du spectre d’énergie.

\paragraph{Équation de Bethe continue.}

À température non nulle (hors de l’état fondamental), il n’y a plus de mer de Fermi définie, et les équations \eqref{chap.1.rho.2} et \eqref{chap.1.rho.3} ne sont plus valides (en particulier $\rho \neq \rho_s$). Les équations discrètes de Bethe \eqref{chap.1.rho.s.2} se condensent alors en une équation intégrale pour les densités de rapidité :

\begin{equation}
	2\pi \rho_s \;=\; 1 \;+\;\Delta \star \rho,
\label{eq:TBA-rhos}
\end{equation}
où le symbole $\star$ désigne la \emph{convolution} :
\(
	[\Delta \star \rho](\theta) = \int_{-\infty}^{\infty} d\theta' \, \Delta(\theta - \theta') \, \rho(\theta').
\)
Et on peut définir la distribution de trou $\rho_h$ , tel que $\rho_s = \rho + \rho_h$. \\

\paragraph{Opération de \emph{dressing}.}
\subparagraph{Définition.}
À toute fonction $f(\theta)$ on associe sa version \emph{habillée} (ou \emph{dressed}) $f^{\mathrm{dr}}_{[\nu]}(\theta)$, définie comme la solution de l’équation intégrale suivante :
\begin{eqnarray}\label{eq:dessing}
	f^{\mathrm{dr}}_{[\nu]} & = & f  \;+\;\tfrac{\Delta}{2\pi}\star\bigl(\nu\,f^{\mathrm{dr}}\bigr) \label{eq:dressing}	
\end{eqnarray}
où pour notre système 
\begin{eqnarray}
	\nu = \frac{\rho}{\rho_s	}\label{eq:TBA-nu}
\end{eqnarray}
est le {\bf facteur d’occupation}, et $\Delta/2\pi$ est le {\bf noyau de diffusion} du modèle.
Dans la suite, il est pratique de décrire la thermodynapique et la dynamique des système à l'aide de la {\em fonction d'occumation}.

%où $\Delta/2\pi$ est un noyau de convolution spécifique et 
%\(
%\nu=\dfrac{\rho}{\rho_s}
%\)
%est le \emph{facteur d’occupation}.

\subparagraph{Interprétation physique}

Le dressing incorpore à tous ordres les effets de rétrodiffusion entre quasi-particules. Il encode ainsi les corrections d’interaction aux grandeurs physiques initiales $f(\theta)$. %Dans le modèle de Lieb–Liniger, cette opération permet de déterminer : l’énergie habillée $\varepsilon^{\mathrm{dr}}(\theta)$ , l’impulsion habillée $p^{\mathrm{dr}}(\theta)$ , les susceptibilités thermodynamiques (cf. section~\ref{chap:GGE}).
%
%{\color{blue}
%\paragraph{Susceptibilités thermodynamiques.}
%
%Les susceptibilités thermodynamiques décrivent la réponse linéaire du système à une variation infinitésimale de paramètres thermodynamiques conjugués aux charges conservées. Pour un système intégrable, elles mesurent la sensibilité des valeurs moyennes $\langle Q_i \rangle$ des charges conservées $Q_i$ par rapport aux potentiels thermodynamiques $\mu_j$ associés à ces charges :
%
%\begin{equation}
%    \chi_{ij} = \frac{\partial \langle Q_i \rangle}{\partial \mu_j}.
%\end{equation}
%
%Dans le cadre de la thermodynamique de Bethe, ces susceptibilités s’expriment à l’aide des fonctions habillées. Si $q_i(\theta)$ est la densité de charge $Q_i$ portée par une quasi-particule de rapidité $\theta$, alors la densité totale de charge est donnée par :
%
%\begin{equation}
%    \langle Q_i \rangle = \int d\theta\, \rho(\theta)\, q_i^{\mathrm{dr}}(\theta),
%\end{equation}
%
%où $q_i^{\mathrm{dr}}(\theta)$ est la charge habillée, solution de l’équation de dressing :
%
%\begin{equation}
%    q_i^{\mathrm{dr}}(\theta) = q_i(\theta) + \int \frac{d\theta'}{2\pi}\, \Delta(\theta - \theta')\, \nu(\theta')\, q_i^{\mathrm{dr}}(\theta').
%\end{equation}
%
%Par différentiation par rapport aux $\mu_j$, on obtient :
%
%\begin{equation}
%    \chi_{ij} = \int d\theta\, \rho_s(\theta)\, \nu(\theta)\, q_i^{\mathrm{dr}}(\theta)\, q_j^{\mathrm{dr}}(\theta),
%\end{equation}
%
%où $\rho_s(\theta)$ est la densité de sites disponibles, $\nu(\theta) = \rho(\theta)/\rho_s(\theta)$ est le facteur d’occupation, et $\Delta(\theta)$ est le noyau issu de la phase de diffusion du modèle considéré.
%
%Ces susceptibilités interviennent dans la théorie hydrodynamique généralisée (GHD) comme coefficients de la métrique thermodynamique et des corrélations à longue distance. Elles permettent également d’exprimer les fluctuations thermiques et les coefficients de transport linéaire (formules de Kubo généralisées).
%
%%\begin{tcolorbox}[colback=gray!5,colframe=gray!40!black,title=Exemple : susceptibilités dans le modèle de Lieb–Liniger]
%\begin{mdframed}[
%	linewidth=0.5pt, 
%	backgroundcolor=gray!5, 
%	roundcorner=50pt,	
%	innerleftmargin=5pt,
%    innerrightmargin=5pt,
%    innertopmargin=5pt,
%    innerbottommargin=2pt,
%    leftmargin=2pt,
%    rightmargin=2pt
%	]
%	
%
%Dans le modèle de Lieb–Liniger à couplage $g > 0$, les quasi-particules sont caractérisées par leur rapidité $\theta$ (proportionnelle à l’impulsion).
%
%\vspace{1mm}
%\textbf{Phase de diffusion et noyau} : la phase de diffusion entre deux particules de rapidité $\theta$ et $\theta'$ est :
%\[
%\phi(\theta - \theta') = 2 \arctan\left( \frac{\theta - \theta'}{g} \right),
%\]
%ce qui donne, par dérivation, le noyau de Bethe :
%\[
%\Delta(\theta) = \frac{2g}{\theta^2 + g^2}.
%\]
%
%\vspace{1mm}
%\textbf{Charge de nombre de particules :} la densité de charge associée au nombre total de particules est $q(\theta) = 1$. Sa version habillée $q^{\mathrm{dr}} = 1^{\mathrm{dr}}$ satisfait :
%\[
%1^{\mathrm{dr}}(\theta) = 1 + \int \frac{d\theta'}{2\pi}\, \Delta(\theta - \theta')\, \nu(\theta')\, 1^{\mathrm{dr}}(\theta').
%\]
%
%\vspace{1mm}
%\textbf{Susceptibilité de compressibilité :}
%La susceptibilité associée à cette charge, notée $\chi_{NN}$ (compressibilité isotherme), est alors donnée par :
%\[
%\chi_{NN} = \int d\theta\, \rho_s(\theta)\, \nu(\theta)\, [1^{\mathrm{dr}}(\theta)]^2.
%\]
%
%\vspace{1mm}
%Cette quantité mesure la variation du nombre de particules à l'équilibre lorsqu'on change le potentiel chimique, et encode les effets d’interactions à tous les ordres dans la phase d’équilibre.
%%\end{tcolorbox}
%
%\end{mdframed}
%\begin{mdframed}[
%	linewidth=0.5pt, 
%	backgroundcolor=gray!5, 
%	roundcorner=50pt,	
%	innerleftmargin=5pt,
%    innerrightmargin=5pt,
%    innertopmargin=5pt,
%    innerbottommargin=2pt,
%    leftmargin=2pt,
%    rightmargin=2pt
%	]
%
%%\begin{tcolorbox}[colback=blue!3,colframe=blue!50!black,title=Exemple : susceptibilité énergétique (capacité thermique)]
%Pour la charge énergie, la densité associée est $q(\theta) = \epsilon(\theta)$, avec :
%\[
%\epsilon(\theta) = \theta^2,
%\]
%dans le modèle de Lieb–Liniger (masse $m=1/2$). Sa version habillée est $\epsilon^{\mathrm{dr}}(\theta)$, solution de :
%\[
%\epsilon^{\mathrm{dr}}(\theta) = \epsilon(\theta) + \int \frac{d\theta'}{2\pi}\, \Delta(\theta - \theta')\, \nu(\theta')\, \epsilon^{\mathrm{dr}}(\theta').
%\]
%
%\vspace{1mm}
%La \textbf{capacité thermique} (susceptibilité $\chi_{EE}$) s’écrit :
%\[
%\chi_{EE} = \int d\theta\, \rho_s(\theta)\, \nu(\theta)\, [\epsilon^{\mathrm{dr}}(\theta)]^2.
%\]
%
%Cela mesure la variation de l’énergie en réponse à un changement de température — incluant les effets d’interaction via le dressing.
%%\end{tcolorbox}
%\end{mdframed}
%\begin{mdframed}[
%	linewidth=0.5pt, 
%	backgroundcolor=gray!5, 
%	roundcorner=50pt,	
%	innerleftmargin=5pt,
%    innerrightmargin=5pt,
%    innertopmargin=5pt,
%    innerbottommargin=2pt,
%    leftmargin=2pt,
%    rightmargin=2pt
%	]
%	
%%\begin{tcolorbox}[colback=green!2,colframe=green!50!black,title=Exemple : susceptibilité d’impulsion]
%La densité d’impulsion est $q(\theta) = p(\theta)$, avec :
%\[
%p(\theta) = \theta,
%\]
%(à masse $m=1/2$ dans le Lieb–Liniger). Le dressing $p^{\mathrm{dr}}$ obéit à :
%\[
%p^{\mathrm{dr}}(\theta) = p(\theta) + \int \frac{d\theta'}{2\pi}\, \Delta(\theta - \theta')\, \nu(\theta')\, p^{\mathrm{dr}}(\theta').
%\]
%
%\vspace{1mm}
%La susceptibilité $\chi_{PP}$ (fluctuation de l’impulsion totale) vaut :
%\[
%\chi_{PP} = \int d\theta\, \rho_s(\theta)\, \nu(\theta)\, [p^{\mathrm{dr}}(\theta)]^2.
%\]
%
%Cette quantité intervient dans la description hydrodynamique et les corrélations à grande échelle des systèmes intégrables.
%%\end{tcolorbox}
%\end{mdframed}
%
%----------
%\subparagraph{Interprétation physique}Le dressing incorpore à tous ordres les effets de rétrodiffusion entre quasi-particules.1 Il encode ainsi les corrections d’interaction aux grandeurs physiques initiales f(θ).4 Dans le modèle de Lieb–Liniger, cette opération permet de déterminer : l’énergie habillée εdr(θ) , l’impulsion habillée pdr(θ) 6 , les susceptibilités thermodynamiques (cf. section~\ref{chap:GGE}).{\color{blue}\paragraph{Susceptibilités thermodynamiques.}Les susceptibilités thermodynamiques décrivent la réponse linéaire du système à une variation infinitésimale de paramètres thermodynamiques conjugués aux charges conservées.8 Pour un système intégrable, elles mesurent la sensibilité des valeurs moyennes ⟨Qi​⟩ des charges conservées Qi​ par rapport aux potentiels thermodynamiques μj​ associés à ces charges 9 :\begin{equation}    \chi_{ij} = \frac{\partial \langle Q_i \rangle}{\partial \mu_j}.\end{equation}Dans le cadre de la thermodynamique de Bethe, ces susceptibilités s’expriment à l’aide des fonctions habillées.4 Si qi​(θ) est la densité de charge Qi​ portée par une quasi-particule de rapidité θ, alors la densité totale de charge est donnée par :\begin{equation}    \langle Q_i \rangle = \int d\theta, \rho(\theta), q_i^{\mathrm{dr}}(\theta),\end{equation}où qidr​(θ) est la charge habillée 4, solution de l’équation de dressing 6 :\begin{equation}    q_i^{\mathrm{dr}}(\theta) = q_i(\theta) + \int \frac{d\theta'}{2\pi}, \Delta(\theta - \theta'), \nu(\theta'), q_i^{\mathrm{dr}}(\theta').\end{equation}Par différentiation par rapport aux μj​, on obtient :\begin{equation}    \chi_{ij} = \int d\theta, \rho_s(\theta), \nu(\theta), q_i^{\mathrm{dr}}(\theta), q_j^{\mathrm{dr}}(\theta),\end{equation}où ρs​(θ) est la densité de sites disponibles 10, ν(θ)=ρ(θ)/ρs​(θ) est le facteur d’occupation 10, et Δ(θ) est le noyau issu de la phase de diffusion du modèle considéré.11Ces susceptibilités interviennent dans la théorie hydrodynamique généralisée (GHD) comme coefficients de la métrique thermodynamique et des corrélations à longue distance.9 Elles permettent également d’exprimer les fluctuations thermiques et les coefficients de transport linéaire (formules de Kubo généralisées).8$\%$\begin{tcolorbox}[colback=gray!5,colframe=gray!40!black,title=Exemple : susceptibilités dans le modèle de Lieb–Liniger]\begin{mdframed}[linewidth=0.5pt, backgroundcolor=gray!5, roundcorner=50pt,	innerleftmargin=5pt,    innerrightmargin=5pt,    innertopmargin=5pt,    innerbottommargin=2pt,    leftmargin=2pt,    rightmargin=2pt]Dans le modèle de Lieb–Liniger à couplage g>0, les quasi-particules sont caractérisées par leur rapidité θ (proportionnelle à l’impulsion).12\vspace{1mm}\textbf{Phase de diffusion et noyau} : la phase de diffusion entre deux particules de rapidité θ et θ′ est :[\phi(\theta - \theta') = 2 \arctan\left( \frac{\theta - \theta'}{g} \right),]ce qui donne, par dérivation, le noyau de Bethe :13\vspace{1mm}\textbf{Charge de nombre de particules :} la densité de charge associée au nombre total de particules est q(θ)=1. Sa version habillée qdr=1dr satisfait :\\vspace{1mm}\textbf{Susceptibilité de compressibilité :}La susceptibilité associée à cette charge, notée χNN​ (compressibilité isotherme), est alors donnée par :[\chi_{NN} = \int d\theta, \rho_s(\theta), \nu(\theta), [1^{\mathrm{dr}}(\theta)]^2.]\vspace{1mm}Cette quantité mesure la variation du nombre de particules à l'équilibre lorsqu'on change le potentiel chimique, et encode les effets d’interactions à tous les ordres dans la phase d’équilibre.\end{tcolorbox}\end{mdframed}\begin{mdframed}[linewidth=0.5pt, backgroundcolor=gray!5, roundcorner=50pt,	innerleftmargin=5pt,    innerrightmargin=5pt,    innertopmargin=5pt,    innerbottommargin=2pt,    leftmargin=2pt,    rightmargin=2pt]\begin{tcolorbox}[colback=blue!3,colframe=blue!50!black,title=Exemple : susceptibilité énergétique (capacité thermique)]Pour la charge énergie, la densité associée est q(θ)=ϵ(θ), avec :[\epsilon(\theta) = \theta^2,]dans le modèle de Lieb–Liniger (masse m=1/2). Sa version habillée est ϵdr(θ), solution de :6\vspace{1mm}La \textbf{capacité thermique} (susceptibilité χEE​) s’écrit :[\chi_{EE} = \int d\theta, \rho_s(\theta), \nu(\theta), [\epsilon^{\mathrm{dr}}(\theta)]^2.]Cela mesure la variation de l’énergie en réponse à un changement de température — incluant les effets d’interaction via le dressing.\end{tcolorbox}\end{mdframed}\begin{mdframed}[linewidth=0.5pt, backgroundcolor=gray!5, roundcorner=50pt,	innerleftmargin=5pt,    innerrightmargin=5pt,    innertopmargin=5pt,    innerbottommargin=2pt,    leftmargin=2pt,    rightmargin=2pt]\begin{tcolorbox}[colback=green!2,colframe=green!50!black,title=Exemple : susceptibilité d’impulsion]La densité d’impulsion est q(θ)=p(θ), avec :[p(\theta) = \theta,](à masse m=1/2 dans le Lieb–Liniger). Le dressing pdr obéit à :6\vspace{1mm}La susceptibilité χPP​ (fluctuation de l’impulsion totale) vaut :[\chi_{PP} = \int d\theta, \rho_s(\theta), \nu(\theta), [p^{\mathrm{dr}}(\theta)]^2.]Cette quantité intervient dans la description hydrodynamique et les corrélations à grande échelle des systèmes intégrables.9\end{tcolorbox}\end{mdframed}}
%
%
%}

\subparagraph{Exemple\,: densité de sites}

En prenant $f(\theta) = 1$ dans l’équation~\eqref{eq:dressing}, on obtient :
\(
1^{\mathrm{dr}}_{[\nu]}=1+\frac{\Delta}{2\pi}\star\bigl(\nu\,1^{\mathrm{dr}}_{[\nu]}\bigr)
\) soit directement : 
\begin{eqnarray}
	2\pi\rho_s = 1^{\mathrm{dr}}_{[\nu]},\label{eq:TBA-rhos-2}
\end{eqnarray}
ce qui n’est autre que la relation constitutive~\eqref{eq:TBA-rhos}.


