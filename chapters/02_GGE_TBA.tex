\chapter{Relaxation et Équilibre dans les Systèmes Quantiques Intégrables : Une Approche par la Thermodynamique de Bethe}\label{chap:relaxation}
\minitoc

%------------------------------------------------------------------
\section*{Introduction générale}

Dans les modèles quantiques intégrables, l’évolution vers l’équilibre, à partir d’un état initial arbitraire (et typiquement hors d’équilibre), ne conduit pas à une thermique de Gibbs classique.  
En effet, du fait de l’existence d’une infinité de charges conservées en involution, les systèmes intégrables n’explorent qu’une sous-partie contrainte de l’espace des états accessibles.  
Ils relaxent alors vers un état stationnaire décrit par une \emph{Ensemble Thermodynamique Généralisé} (GGE), qui encode la conservation de toutes ces quantités.

Cette section pose les fondations nécessaires à la description de ces états stationnaires dans le cadre de la \textbf{thermodynamique de Bethe} (TBA), qui généralise l’analyse au-delà de l’état fondamental.  
Nous considérons ici un régime macroscopique à température (ou entropie) finie, correspondant à des états hautement excités du spectre, mais toujours décrits dans le formalisme intégrable exact.

Notre point de départ est la relation constitutive entre la \emph{densité de quasi-particules} (ou \emph{rapidités}) $\rho(\theta)$ et la \emph{densité d’états} disponibles $\rho_s(\theta)$, qui encode le spectre accessible en présence d’interactions.  
Nous introduisons ensuite une opération clé de la TBA, appelée \emph{habillage} (\emph{dressing}), qui intervient systématiquement dans le calcul des observables physiques et permet de prendre en compte de manière non perturbative les effets des interactions.  
Cette construction sera illustrée dans le cadre du modèle intégrable de Lieb–Liniger, qui décrit un gaz unidimensionnel de bosons avec interaction delta répulsive.

Les outils développés ici seront fondamentaux pour formuler dans la section suivante le concept d’ensemble généralisé (GGE), et pour décrire la dynamique de relaxation des systèmes intégrables.



\section{Notion d’état d’équilibre généralisé (GGE)}

\paragraph{Introduction.}


\paragraph{Configuration des états.}\label{sec:config-etats}.
On désigne par $\boldsymbol{\{ \theta_a \}}\equiv \{ \theta_1 , \cdots , \theta_{N} \}$ la \emph{configuration de rapidités} caractérisant un état propre à $N\!\equiv\!N(\{ \theta_a \})$ particules – le nombre de particules n’est donc pas fixé \emph{a priori} mais dépend de la configuration.  
L’état propre correspondant est noté $\ket{\{ \theta_a \}}\;=\;\ket{\{\theta_1,\dots,\theta_N \}}$.

%%%%%%%%%%%%%%%%%%%%%%%%%%%%%%%%%%%%%%%%%%%%%%%%%%
\paragraph{Observables diagonales dans la base des états propres.}
Dans le chapitre précédent (\ref{chap:LL-BA}), on a vu que l'état $\ket{\{ \theta_a \}}$ associé à cette configuration est une état propre des observables nombre et quantité de mouvement  et  énergie cinétique \eqref{chap1:eq.Q.P.K.theta.1}. Ces observables sont diagonales dans la base des états propres :
\begin{eqnarray}
	\operator{Q}  =  \sum_{ \{\theta_a\} } \left ( \sum_{a = 1}^{N}  1 \right )  \vert \{ \theta_a\}\rangle	\langle \{ \theta_a \}\vert, \, 
	\operator{P}  =  \sum_{\{ \theta_a\}}\left( \sum_{a = 1}^{N}  \theta_a \right )   \vert \{ \theta_a\}\rangle	\langle \{ \theta_a \}\vert,\,\operator{K}  =  \sum_{\{ \theta_a\}}\left ( \sum_{a = 1}^{N} \frac{\theta_a^2}{2} \right )   \vert \{ \theta_a\}\rangle	\langle \{ \theta_a \}\vert.\label{chap.2.gge.1}		
\end{eqnarray}
avec $ \sum_{\{ \theta_a\}}$ une somme sur tous les configurations.\\
%\begin{eqnarray}
%	\operator{Q} \ket{\{ \theta_a\}}  =  \sum_{ \{\theta_a\} } \left ( \sum_{a = 1}^{N}  1 \right ) \ket{\{ \theta_a\}}, \, 
%	\operator{P} \ket{\{ \theta_a\}}  =  \sum_{\{ \theta_a\}}\left( \sum_{a = 1}^{N}  \theta_a \right ) \ket{\{ \theta_a\}},\,\operator{H} \ket{\{ \theta_a\}}  =  \sum_{\{ \theta_a\}}\left ( \sum_{a = 1}^{N} \frac{\theta_a^2}{2} \right )   \ket{\{ \theta_a\}}.		
%\end{eqnarray}

Nous avons introduit ces observables en injectant des opérateurs $\operator{f}$ proportionnels à des puissances de la quantité de mouvement d’une particule $\operator{p}$, respectivement $\propto \operator{p}^0$, $\propto \operator{p}^1$ et $\propto \operator{p}^2$, dans l’opérateur à un corps $\operator{F}$ défini dans l’équation \eqref{chap.1:eq.rapel.opp.1.second.2}. Écrit de cette manière, nous avons vu dans l’équation \eqref{chap.1:eq.rapel.opp.1.second.3} que pour $\operator{f} = \operator{p}^q$ avec $q$ entier, l’état de Bethe $\ket{\{ \theta_a \} }$ est un état propre de $\operator{F}$ :
\begin{eqnarray}\label{chap.2:eq.rapel.opp.1.second.1}
	 \operator{F} \ket{\{\theta_a\}} =   \sum_{ \{\theta_a\} }\left( \sum_{a = 1}^N \theta_a^q \right) \ket{\{\theta_a\}},
\end{eqnarray}
avec des valeurs propres données par des puissances de $\theta$. Cela motive l’étude d’états d’équilibre statistique au-delà de l’équilibre thermique, c’est-à-dire au-delà de l’ensemble de Gibbs.
   




%%%%%%%%%%%%%%%%%%%%%%%%%%%%%%%%%%%%%%%%%%%%
\paragraph{Contexte et GGE dans les systèmes intégrables.}

Dans un système quantique {\bf intégrable}, il existe une infinité de charges conservées locales $\operator{Q}_i$ commutant entre elles et avec l’Hamiltonien $\operator{H}$ ([Rigol et al. 2007] ) \cite{??}. Concrètement, chaque charge se présente sous la forme $\operator{Q}_i = \int dx \,\operator{q}_i(x)$, où $\operator{q}_i(x)$ est une densité d’observable locale à support borné. L’intégrabilité implique ainsi une caractérisation complète des états propres par un ensemble de paramètres (rapidités $\{\theta_j\}$ dans le modèle de Lieb-Liniger) \cite{??}. En particulier, contrairement aux systèmes génériques, un système intégrable ne thermalise pas au sens canonique classique, car la présence de toutes ces contraintes empêche l’oubli complet des conditions initiales. Les points clés sont alors :

\begin{itemize}[label = $\bullet$]
	\item {\bf Charges conservées} : infinité de locales $\operator{Q}_i$ satisfaisant et $[\operator{Q}_i , \operator{H} ] = 0$ et $[\operator{Q}_i , \operator{Q}_j ] = 0$.
	\item {\bf Densités locales} : chaque $\operator{Q}_i$ s’écrit $\operator{Q}_i = \int_\mathbb{R} dx \, \operator{q}_i(x)$ avec $\operator{q}_i(x)$ à support fini.
	\item {\bf Relaxation non canonique} : après un {\em quench} (changement brutal de paramètre), le système évolue vers un état stationnaire qui n’est pas décrit par l’ensemble canonique habituel.
\end{itemize}

Pour décrire cet état, on introduit l’{\bf ensemble de Gibbs généralisé (GGE)}. Rigol et al. ont montré qu’une « extension naturelle de l’ensemble de Gibbs aux systèmes intégrables » prédit correctement les valeurs moyennes des observables après relaxation \cite{??}.  Formellement, pour une région finie du système $\mathcal{S} \subset \mathbb{R}$, on définit la matrice densité locale :
\begin{eqnarray}
	\operator{\rho}^{(\mathcal{S})}_{\mathrm{GGE}} = \frac{1}{Z^{(\mathcal{S})}}\exp \left ( - \sum_i \beta_i \operator{Q}_i^{(\mathcal{S})} \right), \quad \operator{Q}_i^{(\mathcal{S})} = \int_\mathcal{S} dx \, \operator{q}_i(x), \label{chap.TBA.op.rho.S}	
\end{eqnarray}

où $\beta_i \in \mathbb{R}$ sont les multiplicateurs de Lagrange (ou « températures généralisées ») associés aux charges locales conservées $\{\operator{Q}_i\}$. La fonction de partition 
\begin{eqnarray}
	Z^{(\mathcal{S})} = \bm{\mathrm{Tr}}\left [\exp \left( - \sum_i \beta_i \operator{Q}_i^{(\mathcal{S})} \right ) \right ]  \label{chap.TBA.op.Z.S}	
\end{eqnarray}
 assure la normalisation. L’{\bf état GGE} ainsi défini est le seul permettant de prédire de manière cohérente les observables locales de $\mathcal{S}$ à long temps \cite{??}. Autrement dit, l’équilibre local après quench est un état stationnaire faisant perdurer la mémoire de chaque charge conservée, ce qui conduit à un nombre macroscopique de paramètres $\beta_i$ thermodynamiques (une « température » par charge) \cite{??}.

 \subparagraph{Interprétation des multiplicateurs de Lagrange.}
Les multiplicateurs de Lagranges $\beta_i$ apparaissent naturellement lors de l'optimisation sous contraintes, par exemple dans le formalisme de l'{\bf ensemble de Gibbs généralisé (GGE)}, oû il imposent la conservation des valeurs moyennes des charges $\langle \operator{Q}_i^{(\mathcal{S})} \rangle_{\operator{\rho}^{(\mathcal{S})}_{\mathrm{GGE}}} = \bm{\mathrm{Tr}}[\operator{\rho}^{(\mathcal{S})}_{\mathrm{GGE}} \operator{Q}_i^{(\mathcal{S})}]   $.\\

En résumé, la GGE généralise les ensembles canoniques standard : au lieu de retenir uniquement l’énergie, on impose la conservation de l’ensemble complet $\{\operator{Q}_i \}$. Cette construction rend compte du fait que, dans un système intégrable, les observables locaux convergent vers les valeurs moyennes de $\operator{\rho}^{(\mathcal{S})}_{\mathrm{GGE}}$ , et non vers celles d’un Gibbs thermique ordinaire \cite{??}\cite{??}. On comprend ainsi pourquoi la {\em thermalisation habituelle} (canonique ou microcanonique) échoue : seul l’ensemble de Gibbs généralisé peut intégrer toutes les contraintes locales.

\paragraph{Rappel sur le modèle de Lieb-Liniger et distribution de rapidités.}
Comme rappelé au chapitre précédent, {\bf le modèle de  Lieb-Liniger} (gaz bosonique 1D à interactions de contact) est un exemple paradigmatique d’un système intégrable \cite{??}. Ses états propres sont caractérisés par un ensemble de $N$  rapidités $\{ \theta_a \}$ , qui jouent le rôle de quasi-momenta ({\bf Bethe ansatz}). Dans ce contexte, l’état macroscopique du gaz après relaxation unitaire est entièrement déterminé par la {\bf distribution des rapidités}. Formellement, on définit $\rho(\theta)$ la distribution intensive des rapidités telle que $\rho(\theta) d \theta$ donne la fraction de particules par unité de longueur ayant une rapidité dans la cellule $[\theta , \theta + d \theta ] $.\\

Cette « distribution de rapidités » est d’autant plus pertinente qu’elle est {\em accessible expérimentalement}. En effet, lorsque le gaz bosonique 1D est libéré et laissé s’étendre, la distribution asymptotique des vitesses des atomes coïncide avec la distribution initiale des rapidités \cite{??} . Autrement dit, la GGE prédit un profil de vitesses observables en laboratoire. Léa Dubois souligne dans sa thèse que " la distribution de rapidités est la distribution asymptotique des vitesses des atomes après une expansion dans le guide 1D ", et qu’elle peut être extraite par l’hydrodynamique généralisée \cite{??}. \\

Dans la GGE, cette distribution macroscopique $\rho(\theta)$ est fixée par l’ensemble des charges conservées. Par exemple, on ajuste les $\beta_i$ de sorte que les valeurs moyennes $\langle \operator{Q}_i \rangle_{\operator{\rho}^{(\mathcal{S})}_{\mathrm{GGE}}}$ correspondent aux valeurs initiales. Ce processus détermine donc la fonction $\rho(\theta)$ décrivant l’état d’équilibre local. Les observables locaux du gaz (densité, corrélations, etc.) en découlent alors via les équations de Bethe ansatz. 


\paragraph{Convention pour les moyennes d'observables.}
Dans la suite du chapitre, nous noterons la moyenne d’une observable $\operator{\mathcal{O}}$ dans un état décrit par une matrice densité (ici noté) $\operator{\rho}$ par :
\begin{eqnarray}\label{chap.TBA.moy.dens}	
	\braket{\operator{\mathcal{O}}}_{\operator{\rho}} \doteq \bm{\mathrm{Tr}}[\operator{\rho} \, \operator{\mathcal{O}}],
\end{eqnarray}
En particulier, si la matrice densité est un projecteur, comme $\ket{\{\theta_a \}}\!\bra{\{\theta_a \}}$, $\bm{\mathrm{Tr}}[\ket{\{\theta_a \}}\!\bra{\{\theta_a \}} \operator{\mathcal{O}}] =  \bra{\{\theta_a \}}\operator{\mathcal{O}}\ket{\{\theta_a \}}$. dans ce cas on notera la moyenne :
\begin{eqnarray}\label{chap.TBA.moy.dens.pur}
	\braket{\operator{\mathcal{O}}}_{\{\theta_a \}} = \bra{\{\theta_a \}} \operator{\mathcal{O}} \ket{\{\theta_a \}},
\end{eqnarray}
où l’on note simplement l’ensemble des rapidité ${\theta_a}$ pour désigner l’état pur.

%%%%%%%%%%%%%%%%%%%%%%%%%%%%%%%%%%%%%%%%%%%%%%%%%%
\paragraph{Charges conservées locales diagonales dans la base des états propres.}
Les charges conservées locales $\operator{Q}_i^{(\mathcal{S})}$ est diagonale dans la base des  états propres $\ket{ \{ \theta_a \}}$ , avec pour valeurs propres $\langle \operator{Q}_i^{(\mathcal{S})} \rangle_{\{\theta_a \}} $ 	 :
%\begin{eqnarray}
%	\operator{Q}_i^{(\mathcal{S})} & = & \sum_{ \{\theta_a\} } \langle \operator{Q}_i^{(\mathcal{S})} \rangle_{\{\theta_a \}}  \ket{\{\theta_a \}}\!\bra{\{\theta_a \}}.		
%\end{eqnarray}
\begin{eqnarray}\label{chap.TBA.Qi.diag}
	\operator{Q}_i^{(\mathcal{S})}\ket{\{\theta_a \}} & = &  \langle \operator{Q}_i^{(\mathcal{S})} \rangle_{\{\theta_a \}}  \ket{\{\theta_a \}}.		
\end{eqnarray}
%%%%%%%%%%%%%%%%%%%%%%%%%%%%%%%%%%%%%%%%
\paragraph{Probabilité d’un état à rapidités fixées.}
On peut alors définir la probabilité d’occurrence d’un état $\ket{\{ \theta_a \} }$ comme la moyenne de la matrice densité locale $\operator{\rho}^{(\mathcal{S})}_{\mathrm{GGE}}$ définie dans \eqref{chap.TBA.op.rho.S}:
\begin{eqnarray}
	\mathbb{P}^{(\mathcal{S})}_{\{ \theta_a \}}  & \equiv &  \langle \operator{\rho}^{(\mathcal{S})}_{\mathrm{GGE}} \rangle_{\{\theta_a \}}, \label{chap.TBA.P.1}\\
	& = & 
	\frac{1}{Z^{(\mathcal{S})}} \exp \left (- \sum_i \beta_i \langle \operator{Q}_i^{(\mathcal{S})} \rangle_{\{\theta_a \}} \right ) \label{chap.TBA.P.2}.
\end{eqnarray}

%%%%%%%%%%%%%%%%%%%%%%%%%%%
\paragraph{Moyenne d’un charges conservées locales et dérivées de $Z^{(\mathcal{S})}$.} Les charges locales $\operator{Q}_i^{(\mathcal{S})}$ sont diagonale dans la bases \( \{ \ket{\{\theta_a \}} \}  \) [cf eq~ ~\eqref{chap.TBA.Qi.diag}]. 
On peut donc  écrire la moyenne d’une observable comme une somme pondérée par cette probabilité [cf eqs ~\eqref{chap.TBA.P.1}-\eqref{chap.TBA.P.2}] , ou encore comme une dérivée de la fonction de partition définie dans l'équation \eqref{chap.TBA.op.Z.S} :
\begin{eqnarray}
	\langle \operator{Q}_i^{(\mathcal{S})} \rangle_{\operator{\rho}^{(\mathcal{S})}_{\mathrm{GGE}}} &= & \sum_{\{ \theta_a\}} \langle \operator{Q}_i^{(\mathcal{S})} \rangle_{\{\theta_a \}} \mathbb{P}^{(\mathcal{S})}_{\{ \theta_a \}} \label{chap.TBA.moy.1}\\
	 & = &  \left. \frac{1}{Z^{(\mathcal{S})}} \frac{\partial Z^{(\mathcal{S})}}{\partial \beta_i} \right )_{\beta_{j \neq i }}	 \label{chap.TBA.moy.2}
\end{eqnarray}

Par le même raisonnement le moment non centré s'écrit :
\begin{eqnarray}
	\braket{ \operator{Q}_{i_1}^{(\mathcal{S})} \, \operator{Q}_{i_2}^{(\mathcal{S})} \cdots \operator{Q}_{i_q}^{(\mathcal{S})} }_{\operator{\rho}^{(\mathcal{S})}_{\mathrm{GGE}}} &= &  (-1)^q \frac{1}{Z^{(\mathcal{S})}} \left.\frac{\partial}{\partial \beta_{i_1}} \right )_{\beta_{j \neq i_1 }} \left.\frac{\partial}{\partial \beta_{i_2}} \right )_{\beta_{j \neq i_2 }} \cdots \left.\frac{\partial}{\partial \beta_{i_q}} \right )_{\beta_{j \neq i_q }} Z^{(\mathcal{S})} \label{chap.TBA.mom.1}.	
\end{eqnarray}

%%%%%%%%%%%%%%%%%%%%%%%%%%%%%%%
\paragraph{Moments d’ordre supérieur et fluctuations.} On s'avance sur le chapitre (\ref{chap:Fluctu}).
Le premier et second moments permettent d’accéder à la variance 
\begin{eqnarray}
	 \left \langle \left (\operator{Q}_i^{(\mathcal{S})} - \langle\operator{Q}_i^{(\mathcal{S})} \rangle_{\operator{\rho}^{(\mathcal{S})}_{\mathrm{GGE}}} \right )^2  \right \rangle_{\operator{\rho}^{(\mathcal{S})}_{\mathrm{GGE}}} = \langle(\operator{Q}_i^{(\mathcal{S})})^2 \rangle_{\operator{\rho}^{(\mathcal{S})}_{\mathrm{GGE}}}  -  \langle\operator{Q}_i^{(\mathcal{S})} \rangle_{\operator{\rho}^{(\mathcal{S})}_{\mathrm{GGE}}}^2	
\end{eqnarray}
de le charge locale $\operator{Q}_i^{(\mathcal{S})}$, en injectant \eqref{chap.TBA.moy.2} et \eqref{chap.TBA.mom.1} et en utilisant $\frac{1}{f} \partial_x^2 f - ( \frac{1}{f} \partial_x f ) = \partial_x^2 \ln f  $:
\begin{eqnarray}
	\left \langle \left (\operator{Q}_i^{(\mathcal{S})} - \langle\operator{Q}_i^{(\mathcal{S})} \rangle_{\operator{\rho}^{(\mathcal{S})}_{\mathrm{GGE}}} \right )^2  \right \rangle_{\operator{\rho}^{(\mathcal{S})}_{\mathrm{GGE}}}  &=&	  \left . \frac{\partial^2 \ln Z^{(\mathcal{S})}  }{{\partial \beta_i}^2 }  \right )_{\beta_{j\neq i}},\\
	& = &  - \left . 	\frac{\partial \langle\operator{Q}_i^{(\mathcal{S})} \rangle_{\operator{\rho}^{(\mathcal{S})}_{\mathrm{GGE}}} }{\partial \beta_i } \right )_{\beta_{j\neq i}}.	
\end{eqnarray}

%%%%%%%%%%%%%%%%%%%%%%%%%%%%%%
\paragraph{Cas particulier de l’équilibre thermique.}

Dans le cas particulier de l’équilibre thermique standard (\ie Gibbsien), le système est décrit par une seule contrainte d’énergie (ou d’énergie et de particule, dans le cas d’un grand canonique). Les multiplicateurs de Lagrange associés aux charges conservées peuvent alors être identifiés à des grandeurs thermodynamiques classiques.

\begin{itemize}[label=$\bullet$]
	\item Si la seule charge conservée est le nombre de particules $\operator{Q}_0^{(\mathcal{S})} = \operator{Q}$, le multiplicateur associé est $\beta_0 = -\beta \mu$, où $\mu$ est le potentiel chimique et $\beta = T^{-1}$ l’inverse de la température (avec $k_B = 1$).
	
	\item Si la charge conservée est $\operator{Q}_2^{(\mathcal{S})}-\mu\operator{Q}_0^{(\mathcal{S})}  = \operator{K} - \mu \operator{Q} $ (ensemble grand canonique), alors le multiplicateur est simplement $ \beta$.
\end{itemize}

Dans le cadre de l’équilibre thermique , les moyennes et les fluctuations thermodynamiques usuelles s’expriment naturellement comme dérivées du logarithme de la fonction de partition $Z^{(\mathcal{S})}$ :
\begin{eqnarray}
	\langle \operator{Q} \rangle_{\operator{\rho}^{(\mathcal{S})}_{\mathrm{GGE}}}  = \left .\frac{1}{\beta} \frac{ \partial \ln Z^{(\mathcal{S})}}{\partial \mu } \right )_{T},  & &  \left . \frac{1}{\beta} \frac{ \partial \langle \operator{Q} \rangle_{\operator{\rho}^{(\mathcal{S})}_{\mathrm{GGE}}}}{\partial \mu } \right )_{T} =  \left . \frac{1}{\beta^2} \frac{ \partial^2 \ln Z^{(\mathcal{S})}}{{\partial \mu}^2 } \right )_{T} \\
	\langle \operator{H} - \mu\operator{Q}  \rangle_{\operator{\rho}^{(\mathcal{S})}_{\mathrm{GGE}}}  = -\left . \frac{ \partial \ln Z^{(\mathcal{S})}}{\partial \beta } \right )_{\mu} ,  & & -\left .  \frac{ \partial \langle \operator{H} - \mu\operator{Q} \rangle_{\operator{\rho}^{(\mathcal{S})}_{\mathrm{GGE}}}}{\partial \beta } \right )_{\mu } = \left .  \frac{ \partial^2 \ln Z^{(\mathcal{S})}}{{\partial \beta}^2 } \right )_{\mu}   .		
\end{eqnarray}
En combinant ces relations, on peut également exprimer l’énergie moyenne et ses fluctuations comme :
\begin{eqnarray}
	\langle \operator{H} \rangle_{\operator{\rho}^{(\mathcal{S})}_{\mathrm{GGE}}}  = \left [ \left .\frac{\mu}{\beta} \frac{ \partial}{\partial \mu } \right )_{T} -\left . \frac{ \partial }{\partial \beta } \right )_{\mu}   \right ]\ln Z^{(\mathcal{S})},  \quad  -\left .  \frac{ \partial \langle \operator{H} \rangle_{\operator{\rho}^{(\mathcal{S})}_{\mathrm{GGE}}}}{\partial \beta } \right )_{-\mu \beta } = \left [ \left .\frac{\mu}{\beta} \frac{ \partial}{\partial \mu } \right )_{T} -\left . \frac{ \partial }{\partial \beta } \right )_{\mu}  \right ]^2\ln Z^{(\mathcal{S})}.		
\end{eqnarray}

%%%%%%%%%%%%%

\section{Remarques sur le formalisme}




%% !TEX encoding = IsoLatin

%\documentclass[11pt,a4paper]{report}
%\documentclass[11pt,a4paper]{book}
\documentclass[10pt, titlepage]{book} % Taille de base des caractères (12pt recommandée pour lecture)


% -------------------------------------
% Encodage et langue
% -------------------------------------
\usepackage[utf8]{inputenc}
\usepackage[T1]{fontenc}
\usepackage[french]{babel}

% -------------------------------------
% Marges et dimensions
% -------------------------------------
\usepackage[a4paper, top=1.0cm, bottom=1.0cm, left=1cm, right=1cm]{geometry} 
% Ajuste ici les marges selon tes préférences

% -------------------------------------
% Interligne
% -------------------------------------
\usepackage{setspace}
%\onehalfspacing  % Interligne 1.5 
%\doublespacing %(utilise \doublespacing pour double interligne)

% -------------------------------------
% Police (facultatif)
% -------------------------------------
%\usepackage{mathptmx} % Police Times (ancienne)
%\usepackage{libertine} % Police élégante

\usepackage{newtxtext,newtxmath} % Times moderne pour texte et maths

% -------------------------------------
% Paquets utiles
% -------------------------------------
\let\Bbbk\relax
\let\openbox\relax
\usepackage{amsmath, amssymb, amsthm}
\usepackage{graphicx}
\usepackage{hyperref}
\usepackage{xcolor}
\usepackage{braket}
\usepackage{tikz}
\usepackage{pgfplots}
\usepackage{float}
\usepackage{enumitem}
\usepackage{caption}
\usepackage{subcaption}
\usepackage{algorithm2e}
\usepackage{cancel}
\usepackage{bm}
\usepackage{listings}
\usepackage{pdfpages}
\usepackage{mdframed}
\usepackage{braket}
\usepackage{stmaryrd} 
\usetikzlibrary {datavisualization}
\usetikzlibrary {arrows.meta,bending,positioning}
\usetikzlibrary {datavisualization.formats.functions}
%PREAMBULE pour schÃéma
\usepackage{pgfplots}
\usepackage{tikz}
\usepackage[european resistor, european voltage, european current]{circuitikz}
\usetikzlibrary{arrows,shapes,positioning}
\usetikzlibrary{decorations.markings,decorations.pathmorphing,
decorations.pathreplacing}
\usetikzlibrary{calc,patterns,shapes.geometric}
\usepackage{anyfontsize}


% -------------------------------------
% Pour les chapitres
% -------------------------------------
\usepackage[Glenn]{fncychap} % Style de chapitres


% -------------------------------------
% Largeur du texte (évite de le redéfinir si tu utilises geometry)
% -------------------------------------
%\setlength\textwidth{20.5cm}
%\setlength\textheight{22cm}

% -------------------------------------
% Optionnel : si tu veux jouer avec les marges manuellement
% -------------------------------------
% \setlength\topmargin{-1cm}
% \setlength\evensidemargin{-2cm}
% \setlength\oddsidemargin{\evensidemargin}

\usepackage{mdframed}

\usepackage{scalerel}
\usepackage{xcolor}
\usepackage{stackengine}
\usepgflibrary {shadings}


\usetikzlibrary {decorations.pathmorphing}

\usepackage{tikz}

\usepackage{marvosym}
\usepackage{changepage}

\usepackage{minitoc}
\usepackage{tocloft}
%\renewcommand{\cfttoctitle}{\hspace{-2em}}
% Nastaveni obsahu
% Nastaveni obsahu

\usepackage{imakeidx}
\usepackage{fancyhdr}

%\usepackage{makeidx}
\makeindex[intoc=true]
\makeindex[name=pers, title=Index of person names, intoc=true]

\usepackage{xcolor}

\usepackage{hyperref}

%%%%%%%%%%%%%%%%%%%%%
%\definecolor{linkcolor}{RGB}{0,0,180}
\usepackage{titlesec}

\usepackage{tocloft}
\usepackage{datetime} % Pour une date personnalisée
\usepackage[useregional]{datetime2}

\usepackage{mathrsfs}

% -------------------------------------
% Pour les mini-tables des matières
% -------------------------------------
\usepackage{minitoc}
\dominitoc

%\usepackage[most]{tcolorbox}

%%%%%%%%%%%%%%%%%%%%%%%%%%%%%
%\usepackage[utf8]{inputenc}
%\usepackage[T1]{fontenc}
%\usepackage[french]{babel}
%\usepackage{amsmath, amssymb}
%\usepackage{graphicx}
%\usepackage{hyperref}
%\usepackage{tikz}
%\usepackage{physics}
%\usepackage{float}


%\newcommand{\ket}[1]{\left|#1\right\rangle}
%\newcommand{\bra}[1]{\left\langle#1\right|}
%\newcommand{\mean}[1]{\left\langle#1\right\rangle}
%\newcommand{\dd}{\mathrm{d}}

% Activer \frontmatter, \mainmatter et \appendix pour la classe report
%\newcommand{\frontmatter}{%
%  \pagenumbering{roman}%
%  \setcounter{page}{1}%
%  \renewcommand{\chaptermark}[1]{\markboth{##1}{}}
%  \renewcommand{\sectionmark}[1]{\markright{##1}}
%}
%
%\newcommand{\mainmatter}{%
%  \pagenumbering{arabic}%
%  \setcounter{page}{1}
%}


% \appendix est déjà défini dans report, inutile de le redéfinir


% Figures flottantes:
% fraction maximale d'une page pouvant etre occupe par une figure:
\renewcommand{\topfraction}{0.8}
% fraction minimale d'une page reservee pour le texte:
\renewcommand{\textfraction}{0.2}
% fraction minimale d'occupation de la page par une figure pleine page:
\renewcommand{\floatpagefraction}{0.7}

%%%%%%%%%%%%%%%%%%%%%%%%%%%%%%%%%%%%%%%%
%         D\'ecoupage des mots           %
%%%%%%%%%%%%%%%%%%%%%%%%%%%%%%%%%%%%%%%%
\hyphenation{}

%%%%%%%%%%%%%%%%%%%%%%%%%%%%%%%%%%%%%%%%
%%%%  Th\'eor\`emes, d\'efinitions, etc.
%%%%%%%%%%%%%%%%%%%%%%%%%%%%%%%%%%%%%%%%


% Il y a diffÃérents types d'ÃénoncÃés qui mÃéritent un environnement spÃécifique, voici une liste assez exhaustive.
\theoremstyle{plain}
    \newtheorem{Theo}{Th\'eor\`eme}[section] %compteur commençant par le numÃéro de la section (on pourrait aussi faire commencer par le numÃéro de la sous-section - remplacer "section" par "subsection")
    \newtheorem{Prop}[Theo]{Proposition}        %mÃême compteur que pour les thÃéorÃèmes
    \newtheorem{Prob}[Theo]{Probl\`eme}        %idem
    \newtheorem{Lemm}[Theo]{Lemme}            %etc...
    \newtheorem{Coro}[Theo]{Corollaire}
    \newtheorem{Propr}[Theo]{Propri\'et\'e}
    \newtheorem{Conj}[Theo]{ Conjecture}
    \newtheorem{Aff}[Theo]{Affirmation}

    \newtheorem{TheoPrinc}{Th\'eor\`eme}     %compteur spÃécifique pour les thÃéorÃèmes les plus importants du papier
        
\theoremstyle{definition}
    \newtheorem{Defi}[Theo]{D\'efinition}
    \newtheorem{Exem}[Theo]{Exemple}
    \newtheorem{Nota}[Theo]{\Large Notation}

\theoremstyle{remark}
    \newtheorem{Rema}[Theo]{Remarque}
    \newtheorem{NB}[Theo]{N.B.}
    \newtheorem{Comm}[Theo]{Commentaire}
    \newtheorem{question}[Theo]{$\ast$ Question}
    \newtheorem{exer}[Theo]{Exercice}
    \newtheorem{Consequence}[Theo]{Conséquence}
    \newtheorem{Rap}[Theo]{Rappel}
    \newtheorem*{Merci}{Remerciements}

\mdfdefinestyle{propstyle}{%
linecolor=black,linewidth=2pt,%
hidealllines=true,
frametitlerule=true,%
frametitlebackgroundcolor=gray!20,
backgroundcolor=gray!10!white,
roundcorner=5pt,
innertopmargin=\topskip,
}

%\mdtheorem[style=propstyle]{prop}{Property}[chapter]
\mdtheorem[style=propstyle]{lemma}[prop]{Lemma}
\mdtheorem[style=propstyle]{TheoPrinc}{Th\'eor\`eme}[chapter]

% Définition d'un style personnalisé pour les Affirmations
\mdfdefinestyle{affirmestyle}{%
    linecolor=gray, % Couleur de la bordure
    linewidth=1pt, % Épaisseur de la bordure
    backgroundcolor=gray!10, % Couleur de fond (gris clair)
    roundcorner=5pt, % Coins arrondis
    innertopmargin=0pt, % Marge intérieure au-dessus du cadre
    innerbottommargin=10pt, % Marge intérieure en-dessous du cadre
    innerleftmargin=10pt, % Marge intérieure à gauche
    innerrightmargin=10pt, % Marge intérieure à droite
    skipabove=10pt, % Espace au-dessus du cadre
    skipbelow=10pt % Espace en-dessous du cadre
}

% Définition de l'environnement Affirmation
\theoremstyle{definition} % Style de théorème pour les affirmations
\newmdtheoremenv[style=affirmestyle]{aff}{Point clé n$^{\circ}$} % Environnement Affirmation avec le style personnalisé
    
\newcommand\dangersign[1]{%
    \renewcommand\stacktype{L}%
    \scaleto{\stackon[1.3pt]{\color{red}$\triangle$}{\tiny !}}{#1}%
}

\tikzset{every picture/.style={execute at begin picture={\shorthandoff{:;!?};}}}
\tikzstyle{every picture}+=[remember picture]
\tikzstyle{na} = [shape=rectangle,inner sep=0pt]

% Commandes pour les flèches textuelles
\newcommand{\ptFleche}[2]{        % Déclaration d'une extrémité de flèche
    \tikz[baseline=(#1.base)]\node[na](#1){#2};
  }
%\newcommand{\Fleche}[5][thick]{    % Dessin de la flèche
%    \begin{tikzpicture}[overlay]
%        \path[->,#1](#2) edge [out=#4, in=#5] (#3);
%    \end{tikzpicture}
%  }
  
% \newcommand{\Flecheprim}[5][thick]{    % Dessin de la flèche
%    \begin{tikzpicture}[overlay]
%        \path[->,#1](#2) edge [out=#4, in=#5] (#3);
%    \end{tikzpicture}
%  }
%



\definecolor{linkcolor}{RGB}{0,0,180}
\PassOptionsToPackage{
    colorlinks=true,
    linkcolor=linkcolor,
    citecolor=linkcolor,
    urlcolor=linkcolor
}{hyperref}

% Appliquer la couleur à tous les niveaux de titre
\titleformat{\section}{\normalfont\color{colorSix!90!black}\Large\bfseries}{\thesection}{1em}{}
\titleformat{\subsection}{\normalfont\color{colorSix!70!black}\large\bfseries}{\thesubsection}{1em}{}
\titleformat{\subsubsection}{\normalfont\color{colorSix!50!black}\normalsize\bfseries}{\thesubsubsection}{1em}{}
\titleformat{\paragraph}[runin]{\normalfont\color{colorOne!30!black}\bfseries}{\theparagraph}{1em}{}
\titleformat{\subparagraph}[runin]{\normalfont\color{colorOne!10!black}\itshape}{\thesubparagraph}{1em}{}
%%%%%%%%%%%%%%%%%%%%%%%

%%Couleurs dans la table des matières

% Modifier la couleur des entrées de la TOC
\renewcommand{\cftsecfont}{\color{linkcolor!90!black}}
\renewcommand{\cftsubsecfont}{\color{linkcolor!70!black}}
\renewcommand{\cftsubsubsecfont}{\color{linkcolor!50!black}}
\renewcommand{\cftparafont}{\color{linkcolor!30!black}}
\renewcommand{\cftsubparafont}{\color{linkcolor!10!black}}
%%%%%%%%%%%%%%%%%%%%%%%%%%%%%
% Reglages:
%
%\pagestyle{fancyplain}
%\addtolength{\headwidth}{\marginparsep}
%\addtolength{\headwidth}{\marginparwidth}
%\renewcommand{\chaptermark}[1]{\markboth{#1}{}}
%\renewcommand{\sectionmark}[1]{\markright{\thesection\ #1}}
%\lhead[\fancyplain{}{\bfseries\thepage}]{}
%\rhead[]{\fancyplain{}{\bfseries\thepage}}
%\chead[\fancyplain{}{\bfseries\leftmark}]{\fancyplain{}{\bfseries\rightmark}}
%\cfoot{}
%

%usepackage{titlesec}
% Changer la couleur des paragraphes en rouge par exemple :
%\titleformat{\paragraph}[runin] % ou [block] selon ce que tu veux
%  {\normalfont\color{red}\bfseries}
%  {\theparagraph}{1em}{}

% Définition des couleurs avec les codes HTML
\definecolor{colorOne}{HTML}{443E46}
\definecolor{colorTwo}{HTML}{F6DEB8}
\definecolor{colorThree}{HTML}{908CA4}
\definecolor{colorFour}{HTML}{57659E}
\definecolor{colorFive}{HTML}{C57284}
\definecolor{colorSix}{HTML}{FF5B69}

% Raccourcis pour les couleurs
\def\colorOne{colorOne}
\def\colorTwo{colorTwo}
\def\colorThree{colorThree}
\def\colorFour{colorFour}
\def\colorFive{colorFive}
\def\colorSix{colorSix}

%%% ===== Index principal + index secondaire (noms propres) =====
\makeindex[intoc=true]
\makeindex[name=pers, title=Index des noms propres, intoc=true]

%%% ===== Couleur des liens =====
\definecolor{linkcolor}{RGB}{0,0,180}
\PassOptionsToPackage{
    colorlinks=true,
    linkcolor=linkcolor,
    citecolor=linkcolor,
    urlcolor=linkcolor
}{hyperref}
\usepackage{hyperref}

%%% ===== Réglages hyperref =====
\hypersetup{
  pdftitle={Étude de la dynamique hors équilibre de bosons unidimensionnels},
  pdfsubject={Quantum Physics},
  pdfauthor={Guillaume THEMEZE <guillaume.themeze@gmail.fr>},
  pdfkeywords={LaTeX, quantum, bosons, dynamique},
  colorlinks=true
}

%%% ===== Style des titres (colorés) =====
\titleformat{\chapter}[display]{\normalfont\sffamily\huge\bfseries\color{colorSix}}{\chaptertitlename\ \thechapter}{20pt}{\Huge}
\titleformat{\section}{\normalfont\color{colorSix!90!colorFour}\Large\bfseries}{\thesection}{1em}{}
\titleformat{\subsection}{\normalfont\color{colorSix!70!colorFour}\large\bfseries}{\thesubsection}{1em}{}
\titleformat{\subsubsection}{\normalfont\color{colorSix!50!colorFour}\normalsize\bfseries}{\thesubsubsection}{1em}{}
\titleformat{\paragraph}[runin]{\normalfont\color{colorSix!30!colorFour}\bfseries}{\theparagraph}{1em}{}
\titleformat{\subparagraph}[runin]{\normalfont\color{colorSix!10!colorFour}\itshape}{\thesubparagraph}{1em}{}

%%% ===== Couleurs de la table des matières =====
\renewcommand{\cftsecfont}{\color{linkcolor!90!black}}
\renewcommand{\cftsubsecfont}{\color{linkcolor!70!black}}
\renewcommand{\cftsubsubsecfont}{\color{linkcolor!50!black}}
\renewcommand{\cftparafont}{\color{linkcolor!30!black}}
\renewcommand{\cftsubparafont}{\color{linkcolor!10!black}}

%%% ===== En-têtes et pieds de page =====
\pagestyle{fancy}
\fancyhf{}
\setlength{\headheight}{14pt}

\fancyhead[RO,LE]{\thepage}
\fancyhead[LO]{\scshape \nouppercase{\rightmark}}  % Section
\fancyhead[RE]{\scshape \nouppercase{\leftmark}}  % Chapitre
\renewcommand{\headrulewidth}{.4pt}


\newdateformat{mydate}{\THEDAY~\monthname[\THEMONTH]~\THEYEAR}
\newdateformat{mydatetime}{\THEDAY~\monthname[\THEMONTH]~\THEYEAR~à~\currenttime}

%\DTMsetstyle{french} % ou autre style
%\DTMsetup{showtimezone=false}

\fancyfoot[L]{Thèse}
%\fancyfoot[R]{Paris, \mydatetime\today{} -- Période 2022--2025}
\fancyfoot[R]{Paris, \DTMnow -- Période 2022--2025}
%\fancyfoot[R]{Paris, le \DTMdate\today{} à \DTMcurrenttime -- Période 2022--2025}
\renewcommand{\footrulewidth}{.4pt}

% Supprimer les numéros sur la première page de chaque chapitre
\makeatletter
\let\ps@plain=\ps@empty
\makeatother

%%% ===== Réglages des titres de sections dans les en-têtes =====
\renewcommand{\chaptermark}[1]{\markboth{#1}{}}
\renewcommand{\sectionmark}[1]{\markright{\thesection\ #1}}

%%% ===== Notes de bas de page à la française =====
\usepackage[french]{babel}
%\usepackage[frenchfootnotes]{french}
%\FrenchFootnotes
%\AddThinSpaceBeforeFootnotes

%%%%%%%%%%%%%%%%%%%%%%%%%%%%%%%%%%
\newcommand{\operatorvec}[1]{\vec{{\bm{#1}}}} % pour les operateur
\newcommand{\operator}[1]{\hat{\bm{#1}}} % pour les operaeur vecteur
\newcommand{\operatormat}[1]{\operatorname{#1}} % pour les operaeur vecteur
\newcommand{\operatortilde}[1]{\tilde{\bm{#1}}} % pour les opetateur avec un tilde
\newcommand{\operatortildevec}[1]{\tilde{\bm{#1}}}% pour les opetateur avec un tilde et vecteur
\newcommand{\dfonc}[1]{\mathscr{D}_{[#1]}}
%%%%%%%%%%%%%%%%%%%%%%%%%%%%%%%%%%

%🔤 2. Abréviations classiques
% Mathématiques générales
\newcommand{\dd}{\mathrm{d}}           % différentielle droite
\newcommand{\ii}{\mathrm{i}}           % unité imaginaire
\newcommand{\ee}{\mathrm{e}}           % exponentielle

% Pour les ensembles usuels
\newcommand{\R}{\mathbb{R}}            % réels
\newcommand{\C}{\mathbb{C}}            % complexes
\newcommand{\Z}{\mathbb{Z}}            % entiers
\newcommand{\N}{\mathbb{N}}            % naturels

% Délimiteurs automatiques
%\newcommand{\abs}[1]{\left|#1\right|}
%\newcommand{\norm}[1]{\left\lVert#1\right\rVert}
%\newcommand{\paren}[1]{\left(#1\right)}
%\newcommand{\bracket}[1]{\left[#1\right]}
%\newcommand{\set}[1]{\left\{#1\right\}}

%⚛️ 3. Physique quantique
% Bra-ket
%\newcommand{\ketbra}[2]{\ket{#1}\!\bra{#2}}
%\newcommand{\braket}[2]{\left\langle #1 \middle| #2 \right\rangle}
%\newcommand{\ketproj}[1]{\ket{#1}\!\bra{#1}}

% Hamiltonien, opérateurs
\newcommand{\Ham}{\mathcal{H}}
\newcommand{\Op}[1]{\hat{#1}}
\newcommand{\Tr}{\mathrm{Tr}}

% Commutateurs et anticommutateurs
\newcommand{\comm}[2]{\left[#1, #2\right]}
\newcommand{\acomm}[2]{\left\{#1, #2\right\}}

%🌡️ 4. GHD ou dynamique intégrable
\newcommand{\rhoP}{\rho_{\mathrm{p}}}       % densité de particules
\newcommand{\rhoT}{\rho_{\mathrm{t}}}       % densité totale
\newcommand{\veff}{v^{\mathrm{eff}}}        % vitesse efficace
\newcommand{\dr}{\partial}                  % dérivée
\newcommand{\nustar}{\nu^\ast}              % solution auto-similaire

%✍️ 5. Utilisation typographique
\newcommand{\eg}{\emph{e.g.}\xspace}
\newcommand{\ie}{\emph{i.e.}\xspace}
\newcommand{\etal}{\emph{et al.}\xspace}



% Commandes spécifiques ou pour la mise en forme

\makeatletter
\newcommand\xleftrightarrow[2][]{%
  \ext@arrow 9999{\longleftrightarrowfill@}{#1}{#2}}
\newcommand\longleftrightarrowfill@{%
  \arrowfill@\leftarrow\relbar\rightarrow}
\makeatother


\newacronym{LL}{LL}{Lieb-Liniger}
\newacronym{NS}{NS}{Schrödinger non linéaire}
\newacronym{GP}{GP}{Gross–Pitaevskii}
\newacronym{GGE}{GGE}{Generalized Gibbs Ensemble}

\title{Titre de la thèse}
\author{Prénom NOM}
\date{\today}



\begin{document}

\frontmatter
%\chapter*{Introduction}
\addcontentsline{toc}{chapter}{Introduction}
\minitoc

Ceci est l’introduction de la thèse.


\tableofcontents
\mainmatter

\chapter{Modèle de Lieb-Liniger et approche Bethe Ansatz}
\minitoc

\section*{Introduction}

Dans ce chapitre, nous introduisons progressivement le modèle de Lieb-Liniger et l'Ansatz de Bethe, outils fondamentaux pour décrire un gaz de bosons unidimensionnel avec interactions delta. L'objectif est d'accompagner pas à pas le lecteur depuis la formulation du problème quantique en champ de bosons jusqu'aux solutions exactes obtenues par l'Ansatz de Bethe.

Nous commençons par écrire l'équation du champ de bosons, exprimée à l’aide des opérateurs de création et d’annihilation en représentation de position. Pour des raisons pédagogiques, nous abordons d’abord le cas d’une seule particule, sans interaction. Cela permet d’introduire naturellement les états de position et leur évolution sous l’action du Hamiltonien libre.

Ensuite, nous étudions le cas de deux particules, cette fois en tenant compte de l’interaction locale. Cela nous amène à considérer les états de position dans le cas général, y compris lorsque les deux particules peuvent occuper la même position. Cette situation, bien plus subtile qu’il n’y paraît, met en évidence la complexité introduite par l’interaction, et justifie que l’on commence par analyser les configurations où les particules sont à des positions distinctes.

Dans le référentiel du centre de masse, le problème à deux corps avec interaction devient équivalent à un problème à une seule particule en interaction avec une barrière delta au centre. Cette reformulation permet d’interpréter l’effet de l’interaction comme une condition de raccord sur la fonction d’onde, tout en respectant la symétrie bosonique.

Nous revenons ensuite aux coordonnées du laboratoire afin d’introduire naturellement la forme des solutions imposée par l’Ansatz de Bethe. Cela nous conduit aux équations dites de Bethe, qui relient les quasimoments des particules à travers des conditions de périodicité modifiées par l’interaction.

Une fois les notations bien établies, nous généralisons le raisonnement au cas de \(N\) particules, pour obtenir l’Hamiltonien de Lieb-Liniger complet ainsi que la forme générale de l’Ansatz de Bethe. Les solutions ainsi construites permettent non seulement de déterminer le spectre de l’Hamiltonien, mais aussi de calculer des observables physiques importantes, telles que l’impulsion totale ou le nombre de particules.

Enfin, nous introduisons la notion de distribution de rapidité, outil essentiel dans l’étude des états d’énergie minimale (états fondamentaux) et dans la description thermodynamique du système. Ce cadre servira de base aux développements ultérieurs sur les gaz intégrables à température finie et les états stationnaires après quench quantique.

\section{Description du modèle de Lieb-Liniger}

\subsection{Introduction au modèle de gaz de Bose unidimensionnel et Hamiltonien du modèle}

\subsubsection{De la première à la seconde quantification}

\paragraph{Introduction.}

La mécanique quantique se développe historiquement en deux grandes étapes : la \emph{première quantification}, aussi appelée quantification canonique, et la \emph{seconde quantification}. Comprendre ces deux cadres est essentiel pour aborder les systèmes quantiques complexes, en particulier ceux où le nombre de particules peut varier.

%La mécanique quantique s’est historiquement développée en deux étapes : la \emph{première quantification}, aussi appelée quantification canonique, puis la \emph{seconde quantification}. Comprendre ces deux cadres est essentiel pour aborder les systèmes à nombre de particules variable.


%\vspace{0.5cm}

\paragraph{Première quantification (quantification canonique, particule unique).}

La première quantification est la mécanique quantique standard, celle que vous avez rencontrée dès vos premiers cours. Elle consiste à quantifier un système classique décrit par des variables dynamiques telles que la position $x$ et la quantité de mouvement $p$. On procède en remplaçant ces variables par des {\bf opérateurs hermitiens} $\operator{x}$ et %$\operator{p}$
\begin{eqnarray}
	\operator{p} \doteq -i\hbar \operator{\partial}_x,	\label{chap.1.rapel.1}
\end{eqnarray}
où $\hbar$ est la constante de Planck réduite, satisfaisant la {\bf relation de commutation canonique} fondamentale $[\operator{x}, \operator{p}] = i\hbar$. L’état du système est alors décrit par une {\bf fonction d’onde} $\psi(x,t)$, solution de {\bf l’équation de  Schrödinger} indépendante du nombre de particules :
\begin{eqnarray}
\quad i \hbar \frac{\partial \psi }{\partial t}  &= \operator{\mathcal{H}} \psi,\label{chap.1.rapel.2}
\end{eqnarray}

avec $\operator{\mathcal{H}}$ l’opérateur hamiltonien. 

\begin{mdframed}[
	linewidth=0.5pt, 
	backgroundcolor=gray!5, 
	roundcorner=50pt,	
	innerleftmargin=5pt,
    innerrightmargin=5pt,
    innertopmargin=-10pt,
    innerbottommargin=2pt,
    leftmargin=2pt,
    rightmargin=2pt
	]
\subparagraph{Exemple : particule libre en une boite à une dimension.} 
	{~}\\
	
	Dans le cas d’une particule libre de masse $m$ se déplaçant en une dimension, l’Hamiltonien est constitué uniquement du terme cinétique $\operator{\mathcal{H}} = \operator{p}^2 / 2m$. En représentation position, où l’opérateur quantité de mouvement s’écrit comme dans l’équation \eqref{chap.1.rapel.1}, l’Hamiltonien prend alors la forme différentielle :
	\begin{eqnarray}
		\operator{\mathcal{H}} = -\frac{\hbar^2}{2m} \partial_x^2.
	\end{eqnarray}
	Les états propres stationnaires de \eqref{chap.1.rapel.2} dépendant du temps sont de la forme $\psi_k(x,t) = \varphi_k(x)\,e^{-i\varepsilon(k)t/\hbar}$ où $\varphi_k(x)$ est une fonction propre de l’hamiltonien,  soit de  l’équation stationnaire  $\operator{\mathcal{H}}\varphi_k = \varepsilon(k)\varphi_k$ \ie pour une particule libre:
	\begin{eqnarray}
		\frac{\hbar^2}{2m} \partial_x^2 \varphi_k = \varepsilon(k) \varphi_k,
	\end{eqnarray}
	avec $\varepsilon(k)$ l’énergie associée à une onde plane de nombre d’onde $k$
	\begin{eqnarray}
		\varepsilon(k) = \frac{\hbar^2 k^2 }{2 m}.
	\end{eqnarray}
	Les fonctions propres spatiales $\varphi_k(x)$ de l’hamiltonien libre s’écrivent comme des combinaisons linéaires d’ondes planes  
	\begin{eqnarray}
		\varphi_k(x) = a e^{-i k x} + b e^{i k x},~~ \mbox{avec}\quad  (a,b) \in \mathbb{C}^2.
	\end{eqnarray}
	Si la particule est confinée dans une boîte de longueur $L$ avec des conditions aux limites périodiques (ie $\varphi_k(x+L) = \varphi_k(x)$), alors le spectre de $k$ est quantifié : 
	\begin{eqnarray}
		kL \in 2\pi\mathbb{Z}.
	\end{eqnarray}
	Le problème est équivalent à celui d’une particule libre sur un cercle de périmètre $L$.\\
	
	\medskip
	
	Les solutions générales de l’équation de Schrödinger s’écrivent alors comme une superposition d’états propres  $\psi = \sum_k c_k \psi_k $.  

%On résume :
%\begin{eqnarray}
%	,~~ , ~~\varphi_k(x) = a e^{-i k x} + b e^{i k x},~~ kL \in 2\pi\mathbb{Z}.\label{chap.1.recap}
%\end{eqnarray}
\end{mdframed}

Ces solutions correspondent à des {\bf états non liés} (ou états de diffusion) : la particule est délocalisée sur tout l’espace (le cercle), sans structure particulière.

La fonction $\varphi_k(x)$ est supposée normalisée dans l’espace des états quantifiés (boîte finie) :
\(
\int_0^L dx \, \varphi_{k'}^\ast(x)\, \varphi_k(x) = \delta_{k,\pm k'}.
\)
avec  $ \vert a \vert^2 + \vert b \vert^2 = L^{-1}$.
Et dans le sous espace engendré pas $x \mapsto e^{-ikx}$ et $x \mapsto e^{ikx}$ (de deux dimension si $k \neq 0$ et une dimension si $k$ =0), $x \mapsto \pm ( b^\ast e^{-ikx} - a^\ast e^{ikx} )$ est orthogonale à  $\varphi_k$ pour $k \neq 0$.
\begin{mdframed}[
	linewidth=0.5pt, 
	backgroundcolor=gray!5, 
	roundcorner=50pt,	
	innerleftmargin=5pt,
    innerrightmargin=5pt,
    innertopmargin=-10pt,
    innerbottommargin=2pt,
    leftmargin=2pt,
    rightmargin=2pt
	]
\subparagraph{Remarque.} On choisie  \( a = \frac{1}{\sqrt{L}} \) et \( b = 0 \)), alors
\(
\varphi_k(x) = \frac{1}{\sqrt{L}}\, e^{i k x}
\)
est une onde plane. 

\end{mdframed}

Avec le formalisme de Dirac, la fonction d’onde $\varphi_k$ est représentée par le ket $\ket{k}$ normé (i.e. $\langle k' \vert k \rangle = \delta_{k, k'}$, où $\delta_{p,q}$ est le symbole de Kronecker)
, et l’équation de Schrödinger s’écrit :
\(
\operator{\mathcal{H}}_1 \ket{k} = \varepsilon(k) \ket{k}.
\)
En appliquant le bra $\bra{x}$ de part et d’autre, on obtient :
\(
\bra{x} \operator{\mathcal{H}}_1 \ket{k} = \varepsilon(k) \langle x \vert k \rangle,
\)
où $\varphi_k(x) = \langle x \vert k \rangle$ est la représentation positionnelle de l’état $\ket{k}$.



%\begin{mdframed}[
%	linewidth=0.5pt, 
%	backgroundcolor=gray!5, 
%	roundcorner=50pt,	
%	innerleftmargin=5pt,
%    innerrightmargin=5pt,
%    innertopmargin=-10pt,
%    innerbottommargin=2pt,
%    leftmargin=2pt,
%    rightmargin=2pt
%	]
%\subparagraph{Remarque.} Si l’on choisit une base orthonormée d’états propres (par exemple en fixant \( a = \frac{1}{\sqrt{L}}, b = 0 \)), alors
%\(
%\varphi_k(x) = \frac{1}{\sqrt{L}}\, e^{i k x}, \quad \text{et donc} \quad \langle k \vert x \rangle = \varphi_k^\ast(x) = \frac{1}{\sqrt{L}}\, e^{-i k x},
%\)
%ce qui est bien une onde plane. 
%En revanche, dans l’écriture générale \( \varphi_k(x) = a\, e^{i k x} + b\, e^{-i k x} \), la fonction \( \langle k \vert x \rangle = \varphi_k^\ast(x) \) n’est \emph{pas nécessairement} une onde plane, car \( \varphi_k(x) \) n’est pas normalisée.
%\end{mdframed}

\begin{mdframed}[
	linewidth=0.5pt, 
	backgroundcolor=gray!5, 
	roundcorner=50pt,	
	innerleftmargin=5pt,
    innerrightmargin=5pt,
    innertopmargin=1pt,
    innerbottommargin=2pt,
    leftmargin=2pt,
    rightmargin=2pt
	]
La base $\{\ket{x}\}$ étant continue, et les états $\{\ket{k}\}$ quantifiés (par exemple dans une boîte de taille finie avec conditions aux limites périodiques), les relations de changement de base s’écrivent :
\begin{eqnarray}
	\ket{k} = \int dx \, \varphi_k(x) \ket{x}, \qquad   
	\ket{x} = \sum_k \varphi_k^\ast(x) \ket{k},
\end{eqnarray}
avec $\varphi_k^\ast(x) = \langle k \vert x \rangle$. L’état $\ket{x}$ est relié aux états $\ket{k}$ par une transformation de Fourier discrète. Ces formules montrent que les états $\ket{k}$ sont les composantes de Fourier de l’état $\ket{x}$.
\end{mdframed}






\subparagraph{De la particule unique aux systèmes à $N$ particules.}

Pour un système composé de $N$ particules identiques, une approche naturelle consiste à introduire une fonction d’onde $\varphi(x_1, \dots, x_N)$ dépendant de $N$ variables , symétrique pour des bosons ou antisymétrique pour des fermions sous l’échange de deux coordonnées $x_i \leftrightarrow x_j$, solution de l’équation de Schrödinger à $N$ corps. %Dans le cas bosonique, des interactions à courte portée peuvent être modélisées par un potentiel de type Dirac :

%\begin{equation}
%V_{\text{int}}(x_1, \dots, x_N) = g \sum_{i<j} \delta(x_i - x_j),
%\end{equation}

%où $g$ est un paramètre d’interaction contrôlant l’intensité des collisions. 
Toutefois, cette description devient rapidement inextricable lorsque le nombre de particules augmente, ou lorsque le système permet la création et l’annihilation de particules, comme dans un milieu ouvert ou en contact avec un bain thermique.





%{\color{blue} \paragraph{Seconde quantification.}
%
%%Dans ce formalisme, l’espace des états est une **somme directe d’espaces à $N$ particules**, et chaque état est décrit par son occupation des modes quantiques. Les opérateurs $\hat{a}_k^\dagger$ et $\hat{a}_k$ créent et détruisent une particule dans l’état d’onde plane de moment $k$, satisfaisant les relations de commutation (bosons) ou d’anticommutation (fermions) :
%%\begin{equation}
%%[\hat{a}_k, \hat{a}_{k'}^\dagger] = \delta_{k,k'} \quad \text{(bosons)}.
%%\end{equation}
%
%%L’hamiltonien d’un gaz de particules libres s’écrit alors simplement :
%%\begin{equation}
%%\hat{\mathcal{H}} = \sum_k \varepsilon(k) \, \hat{a}_k^\dagger \hat{a}_k,
%%\end{equation}
%%avec $\varepsilon(k) = \frac{\hbar^2 k^2}{2m}$ comme dans la quantification canonique.
%
%\paragraph{Vers le Bethe ansatz.}
%
%Ce formalisme devient particulièrement utile lorsque des interactions entre particules sont introduites. Dans le cas d’un **gaz de bosons en une dimension avec interactions de contact**, le système reste exactement soluble : la solution repose sur une **superposition cohérente d’ondes planes symétrisées**, ajustées par les conditions d’interaction.
%
%C’est l’idée fondamentale du **Bethe ansatz**, qui généralise la solution d’une particule libre sur un cercle à $N$ particules **avec collisions élastiques**. On y retrouve des relations de quantification du type :
%\begin{equation}
%k_j L + \sum_{\substack{\ell=1 \\ \ell \neq j}}^N \theta(k_j - k_\ell) = 2\pi n_j,
%\end{equation}
%où $\theta$ est une phase de diffusion et les $n_j \in \mathbb{Z}$.
%
%On passe ainsi des conditions de périodicité simples à des **conditions de type Bethe**, qui encodent les effets des interactions sous forme de **conditions de compatibilité entre les moments**.
%
%}

\subsubsection{Seconde quantification}

Pour dépasser ces limitations, on adopte le \textbf{formalisme de la seconde quantification}, dans lequel l’état du système est décrit non plus par une fonction d’onde mais par un vecteur dans un espace de Fock. Les opérateurs de création et d’annihilation remplacent alors les variables dynamiques classiques et permettent une description unifiée et élégante des systèmes à nombre variable de particules.

%\paragraph{Structure de l’espace des états de Fock.}
%Dans ce formalisme, l’espace des états est une {\bf somme directe d’espaces à $N$ particules}, et chaque état est décrit par l’occupation des différents modes quantiques. Les opérateurs $\operator{a}_k^\dagger$ et $\operator{a}_k^{}$ créent et annihilent une particule dans l’état d’onde plane de moment $k$ :
%\begin{eqnarray*}
%	\ket{k} & = & \operator{a}_k^\dagger \ket{\emptyset} ~=~ \text{état avec une particule dans le mode } k,	
%\end{eqnarray*}
%où \(\ket{\emptyset}\) est le vide quantique de Fock, défini par :
%\begin{eqnarray}
%	\forall k \in \mathbb{R}\colon \qquad \operator{a}_k \ket{\emptyset} = 0 ,\quad  \langle \emptyset\vert \emptyset \rangle = 1, \label{chap:eq.vide.fock.k}
%\end{eqnarray}
%où \( \operator{a}_\lambda \) peut ici estre une notation générique désignant \( \operator{b}_\lambda \) pour les bosons, ou \( \operator{c}_\lambda \) pour les fermions,
%et satisfaisant les relations de commutation (pour les bosons) ou d’anticommutation (pour les fermions). Dans ce qui suit, nous nous restreignons au cas bosonique. \subparagraph{Relations de commutation bosoniques.} Les relations de commutation bosoniques fondamentales sont alors :
%\begin{eqnarray}
%[\operator{a}_k^{}, \operator{a}_{k'}^{}]= [\operator{a}_k^\dagger, \operator{a}_{k'}^\dagger]= 0 ,\qquad [\operator{a}_k^{}, \operator{a}_{k'}^\dagger] = \operator{\delta}_{k,k'},\label{chap:1:com.1.k}
%\end{eqnarray}
%où $\operator{\delta}_{k,k'}$ est le symbole de Kronecker, qui vaut $1$ si $k = k'$ et $0$ sinon.\\

%%%%%%%%%%%%%%%%%%%%%%%%
\paragraph{Structure de l’espace des états de Fock.}
Dans ce formalisme, l’espace des états est une {\bf somme directe d’espaces à $N$ particules}, et chaque état est décrit par l’occupation des différents modes quantiques. Les opérateurs $\operator{a}_k^\dagger$ et $\operator{a}_k$ créent et annihilent une particule dans l’état d’onde plane de moment $k$ :
\begin{eqnarray*}
	\ket{k} & = & \operator{a}_k^\dagger \ket{\emptyset} ~=~ \text{état avec une particule dans le mode } k,	
\end{eqnarray*}
où \(\ket{\emptyset}\) désigne le vide quantique de Fock, défini par :
\begin{eqnarray}
	\forall k \in \mathbb{R}\colon \qquad \operator{a}_k \ket{\emptyset} = 0 ,\quad  \langle \emptyset \vert \emptyset \rangle = 1. \label{chap:eq.vide.fock.k}
\end{eqnarray}
Le symbole \( \operator{a}_\lambda \) représente ici de manière générique soit l’opérateur \( \operator{b}_\lambda \) pour les bosons, soit \( \operator{c}_\lambda \) pour les fermions, et satisfait respectivement les relations de commutation (pour les bosons) ou d’anticommutation (pour les fermions). Dans ce qui suit, nous nous restreignons au cas bosonique.

\subparagraph{Relations de commutation bosoniques.} Les relations de commutation fondamentales pour les bosons sont :
\begin{eqnarray}
	[\operator{b}_k, \operator{b}_{k'}] = [\operator{b}_k^\dagger, \operator{b}_{k'}^\dagger] = 0 ,\qquad [\operator{b}_k, \operator{b}_{k'}^\dagger] = \operator{\delta}_{k,k'}, \label{chap:1:com.1.k}
\end{eqnarray}
où $\operator{\delta}_{k,k'}$ est le symbole de Kronecker, valant $1$ si $k = k'$ et $0$ sinon.
%%%%%%%%%%%%%%%%%%%%%%%%%%%%%%%%%%%%%%%%

%\vspace{1em}
\paragraph{Nature du champ quantique.}
La seconde quantification généralise ce cadre en permettant de traiter des systèmes où le nombre de particules n’est pas fixé, ce qui est fréquent en physique des particules, des champs quantiques, ou des gaz quantiques.

L’idée principale est de ne plus quantifier directement les particules, mais le \emph{champ quantique} associé. Les états d’une particule unique deviennent alors des états d’occupation dans un espace de Fock, qui décrit l’ensemble des configurations possibles avec zéro, une, ou plusieurs particules.



\subparagraph{Champs de Bose.}
Le gaz de Bose unidimensionnel est décrit dans le cadre de la théorie quantique des champs par un champ bosonique canonique \( \operator{\Psi}(x) \), qui agit sur l’espace de Fock des états du système. Ce champ quantique encode l’annihilation d’une particule en \( x \), et son adjoint \( \operator{\Psi}^\dag(x) \) correspond à la création d’une particule en ce point. 
\begin{eqnarray}
	\vert x \rangle  & = & \operator{\Psi}^\dag (x)\ket{\emptyset} ~=~ \text{état avec une particule en } x,
\end{eqnarray}
et \(\ket{\emptyset}\) est le vide quantique de Fock défini par :
\begin{eqnarray}
	\forall x \in \mathbb{R}, \qquad \operator{\Psi}(x) \ket{\emptyset} = 0. \label{chap:eq.vide.fock}
\end{eqnarray}

\subparagraph{Relations de commutation bosoniques.}
Ces champs satisfont les relations de commutation canoniques à temps égal :
%\begin{eqnarray}
%	\left . \begin{array}{rcl}
%		[ \operator{\Psi}(x),  \operator{\Psi}^\dagger(y) ]  &=&  \operator{\delta}(x - y) \\
%		\left [ \operator{\Psi}(x),  \operator{\Psi}(y) \right ]   =  [ \operator{\Psi}^\dag(x),  \operator{\Psi}^\dag(y) ]  &=&  0 
%	\end{array} \right . \label{chap:1:com.1}
%\end{eqnarray}
\begin{eqnarray}
	 [ \operator{\Psi}(x),  \operator{\Psi}(y)  ]   =  [ \operator{\Psi}^\dag(x),  \operator{\Psi}^\dag(y) ]  =  0,   & & [ \operator{\Psi}(x),  \operator{\Psi}^\dagger(y) ]  =  \operator{\delta}(x - y) ,\label{chap:1:com.1}
\end{eqnarray}
où $\operator{\delta}(x - y)$ est la fonction delta de Dirac.  
Ces relations expriment le caractère bosonique des excitations du champ.

\paragraph{État à $N$ particules.} Soient $N$ bosons dans les états $\{ k_1 , \cdots , k_N \}$ (un boson dans l’état $k_1$, un autre dans $k_2$, etc.) et aux positions $\{ x_1 , \cdots , x_N \}$ (un boson en $x_1$, un autre en $x_2$, etc.). Leurs états s’écrivent alors :
\begin{eqnarray}
	\ket{ \{ k_1 , \cdots , k_N \}} = \frac{1}{\sqrt{N!}} \operator{b}_{k_1}^\dag\, \cdots \, \operator{b}_{k_N}^\dag \ket{\emptyset}, \quad \ket{\{x_1 , \cdots , x_N\}} = \frac{1}{\sqrt{N!}} \operator{\Psi}^\dag(x_1)\, \cdots \, \operator{\Psi}^\dag(x_N) \ket{\emptyset}	, \label{eq.chap.1.ket.N}
\end{eqnarray}
où le facteur \( 1/\sqrt{N!} \) traduit le caractère d’indiscernabilité des bosons et garantit la symétrisation correcte de l’état.

\subparagraph{Changement de base.}
On peut relier les opérateurs de création/annihilation dans la base des ondes planes aux opérateurs de champ via :
\begin{eqnarray}
	\operator{b}_k^\dagger = \int dx \, \varphi_k(x) \operator{\Psi}^\dagger(x), \qquad 
	\operator{\Psi}^\dagger(x) = \sum_k \varphi_k^\ast(x)\operator{b}_k^\dagger.\label{eq.chap.1.TF.1}
\end{eqnarray}
Le champ quantique $\operator{\Psi}(x)$ est relié aux opérateurs de moment $\operator{b}_k$ par une transformation de Fourier. Ces formules montrent que les opérateurs $\operator{b}_{k}$ sont les composantes de Fourier du champ $\operator{\Psi}(x)$.

%où $\varphi_k(x)$ est la fonction d’onde d’un état d’énergie bien définie \( \ket{k} \) dans la représentation positionnelle.
Ainsi, un état à \(N\) bosons dans la base \( \ket{k}^{\otimes N} \) peut s’écrire :
\begin{eqnarray}
	\ket{\{k_1 , \cdots , k_N\}} = \frac{1}{\sqrt{N!}} \int dx_1 \cdots dx_N \, \varphi_{\{k_a\}} ( x_1 , \cdots , x_N ) \, \hat{\Psi}^\dag(x_1) \cdots \hat{\Psi}^\dag(x_N) \ket{\emptyset},
\end{eqnarray}
où \( \{k_a\} \equiv \{k_1, \dots, k_N\} \), et la fonction d’onde symétrisée s’écrit :
\(
	\varphi_{\{k_a\}} ( x_1 , \cdots , x_N ) = \frac{1}{\sqrt{N!}} \sum_{\sigma \in \operator{S}_N } \prod_{i=1}^N \varphi_{k_{\sigma(i)}}(x_i),
\) 
avec $\operator{S}_N $  le groupe symétrique d'ordre $N$ mais aussi :
\begin{eqnarray}
	\varphi_{\{k_a\}} ( x_1 , \cdots , x_N ) = \frac{1}{\sqrt{N!}} \bra{\emptyset} \hat{\Psi}(x_1) \cdots \hat{\Psi}(x_N) \ket{\{k_1, \cdots , k_N\}}.
\end{eqnarray}



\subsubsection{Operateur. }


\paragraph{Opérateur à un corps.}

Soit \( \operator{f} \) un opérateur à une particule, dont les éléments de matrice dans une base orthonormée \( \{ \ket{k} \} \) sont donnés par \( f_{\lambda\nu} = \langle \lambda \vert \operator{f} \vert \nu \rangle \). Un opérateur symétrique à \( N \) particules correspondant à la somme des actions de \( \operator{f} \) sur chacune des particules s’écrit en première configuration  :
\(
	\operator{F} = \sum_{i=1}^N \operator{f}^{(i)},
\)
où \( \operator{f}^{(i)} \) désigne l’action de \( \operator{f} \) sur la $i^\text{e}$ particule uniquement. En base de Dirac, cela donne :
\(
	\operator{f}^{(i)} = \sum_{\lambda, \nu} f_{\lambda\nu} \, \ket{i\!:\!\lambda} \bra{i\!:\!\nu},
\)
où \( \ket{i\!:\!\lambda} \) représente un état où seule la $i^\text{e}$ particule est dans l’état \( \lambda \). (Par construction, l’opérateur \( \operator{F} \) commute avec les projecteurs de symétrisation \( \operator{S}_N \) (bosons) et d’antisymétrisation \( \operator{A}_N \) (fermions).)
On peut montrer que la somme des projecteurs agissant sur chaque particule s’identifie à une combinaison d’opérateurs de création et d’annihilation :
\(
	\sum_{i=1}^N \ket{i\!:\!\lambda} \bra{i\!:\!\nu} = \operator{a}^\dagger_\lambda \operator{a}_\nu^{},
\)
(où \( \operator{a}_\lambda \) peut ici estre une notation générique désignant \( \operator{b}_\lambda \) pour les bosons, ou \( \operator{c}_\lambda \) pour les fermions).

On en déduit que l’opérateur à un corps \( \operator{F} \) peut se réécrire dans le formalisme de la seconde quantification comme :
\begin{eqnarray}
	\operator{F} = \sum_{\lambda, \nu} f_{\lambda\nu} \, \operator{a}^\dagger_\lambda \operator{a}_\nu^{}.
\end{eqnarray}


\subparagraph{Exemples d’opérateurs à un corps.}

Si l’on sait diagonaliser l’opérateur \( \operator{f} \), c’est-à-dire si l’on peut écrire :
\(
	\operator{f} = \sum_k f_k \ket{k} \bra{k},
\)
alors l’opérateur à $N$ corps associé s’écrit :
\(
	\operator{F} = \sum_k f_k \, \operator{a}^\dagger_k \operator{a}_k^{} = \sum_k f_k \, \operator{n}_k,
\)
où \( \operator{n}_k = \operator{a}^\dagger_k \operator{a}_k \) est l’opérateur nombre de particules dans le mode \( k \). On obtient ainsi une forme diagonale de \( \operator{F} \) en seconde quantification.
\begin{mdframed}[linewidth=0.5pt, backgroundcolor=gray!5, roundcorner=5pt]
Un exemple immédiat est celui des particules libres. Si l’on diagonalise le problème à une particule selon :
\(
	\operator{\mathcal{H}}_1 \ket{k} = \varepsilon(k) \ket{k},
\)
alors l’énergie totale du système correspond ici uniquement à son énergie cinétique, et s’écrit :
\begin{equation}
	\operator{K} = \sum_{k} \varepsilon(k) \, \operator{a}^\dagger_k \operator{a}_k^{}.\label{eq.chap.1.cinietique.1}
\end{equation}

Et pour $N$ particules, en écrivant l’état sous la forme~\eqref{eq.chap.1.ket.N}, en utilisant les relations de commutation~\eqref{chap:1:com.1.k} et la définition de l’état de Fock~\eqref{chap:eq.vide.fock.k}, on trouve que $\ket{\{k_1, \cdots, k_N\}}$ est un état propre de $\operator{K}$ associé à l'énergie $\left( \sum_{i = 1}^N \varepsilon(k_i) \right)$, c’est-à-dire :
\begin{eqnarray}
	\operator{K} \ket{\{k_1, \cdots, k_N\}} = \left( \sum_{i = 1}^N \varepsilon(k_i) \right) \ket{\{k_1, \cdots, k_N\}}.\label{eq.chap.1.cinietique.2}
\end{eqnarray}
\end{mdframed}

\paragraph{Forme champ des opérateurs à un corps.}

Les opérateurs à plusieurs corps peuvent être exprimés de manière remarquable à l’aide des opérateurs de champ, d’une façon physiquement transparente qui rappelle les formules bien connues du cas à une particule.

La forme générale d’un opérateur à un corps s’écrit :
\begin{eqnarray}
\operator{F} = \int dx \, dx' \, \operator{\Psi}^\dagger(x) \, \bra{ x} \operator{f} \ket{x'} \, \operator{\Psi}(x').
\end{eqnarray}%où \( \hat{f} \) est l’opérateur à un corps exprimé dans la base position, et \( \hat{\psi}^\dagger(\vec{r}) \), \( \hat{\psi}(\vec{r}) \) sont les opérateurs de création et d’annihilation d’une particule au point \( \vec{r} \).
\begin{mdframed}[
	linewidth=0.5pt, 
	backgroundcolor=gray!5, 
	roundcorner=50pt,	
	innerleftmargin=5pt,
    innerrightmargin=5pt,
    innertopmargin=-10pt,
    innerbottommargin=2pt,
    leftmargin=2pt,
    rightmargin=2pt
]
\subparagraph{Énergie cinétique totale.}

Pour des particules non relativistes, l’énergie cinétique élémentaire s’écrit $\operator{f} = \frac{\hbar^2 \operator{p}^2}{2m}$. À l’échelle du champ quantique, l’énergie cinétique totale prend la forme opératorielle :
\begin{eqnarray}
\operator{K} =  -\frac{\hbar^2}{2m} \int dx \, \operator{\Psi}^\dagger(x) \, \operator{\partial}_x^2 \operator{\Psi}(x)
= \frac{\hbar^2}{2m} \int dx \, \operator{\partial}_x \operator{\Psi}^\dagger(x) \cdot \operator{\partial}_x \operator{\Psi}(x). \label{eq.chap.1.cinietique.3}
\end{eqnarray}

Le champ quantique $\operator{\Psi}(x)$ est relié aux opérateurs de moment $\operator{b}_k$ par une transformation de Fourier. En injectant l'expression \eqref{eq.chap.1.TF.1} dans \eqref{eq.chap.1.cinietique.3}, on retrouve la forme discrète \eqref{eq.chap.1.cinietique.1}, cette fois exprimée en termes des opérateurs $\operator{b}_k$.

Lorsque cet Hamiltonien agit sur l’état de Fock à $N$ particules $\ket{\{k_1 , \cdots , k_N\}}$, les règles de commutation (\ref{chap:1:com.1}) ainsi que la définition des états de Fock (\ref{chap:eq.vide.fock}) impliquent (cf. Annexe \ref{annex:N.part}) :
\begin{eqnarray}
\operator{K}\ket{k_1 , \cdots , k_N } =  \int d^N z \, \operator{\mathcal{K}}_N \, \varphi_{\{k_a\}}(z_1 , \cdots , z_N ) \operator{\Psi}(z_1) \cdots \operator{\Psi}^\dag(z_N) \ket{\emptyset}
\end{eqnarray}
avec :
\[
	\operator{\mathcal{K}}_N = \sum_{i=1}^N \frac{\operator{p}_i^2}{2m},
\]
où \( \operator{p}_i \) désigne l’opérateur impulsion de la \( i^\text{ème} \) particule.
\end{mdframed}




\paragraph{Opérateurs à deux corps}

Nous considérons à présent les termes d’interaction impliquant deux particules , $\operator{v}$ , dont les éléments de matrices sont donnés par $v_{\alpha \beta \gamma \delta} = \bra{ 1 : \alpha; 2 : \beta } \operator{v}\ket{ 1 : \gamma; 2 : \delta }$ , où $\ket{ i : \gamma; j : \delta }$ représente l'état où la $i^\text{e}$  particules est dans l'état $\gamma$ et la $j^\text{e}$ dans l'état $\delta$  . Ceux-ci correspondent à des opérateurs de la forme :
\(
    \operator{V} = \sum_{j < i} \operator{v}^{(i, j)} = \frac{1}{2} \sum_{i, j \ne i} \operator{v}^{(i, j)}
    \label{eq:V2corps}.
\)
avec $\operator{v}^{(i, j)}$ désigne l’interaction à deux corps entre les $i^\text{e}$ et $j^\text{e}$ particules , exprimés dans la base à deux états :
\(
	\operator{v}^{(i, j)} = \sum_{\alpha,\beta,\delta,\gamma} \ket{i : \alpha; j : \beta }v_{\alpha \beta \gamma \delta} \bra{ i : \gamma; j : \delta }.
    %v_{\alpha \beta \gamma \delta} = \bra{ i : \alpha; j : \beta } \operator{v}^{(i,j)} \ket{ i : \gamma; j : \delta }.
    \label{eq:matriceV}
\)
On peut réécrire l’opérateur \( \operator{V} \) en termes d’opérateurs de création et d’annihilation comme suit :
\begin{equation}
    \operator{V} = \frac{1}{2} \sum_{\alpha, \beta, \gamma, \delta} v_{\alpha \beta \gamma \delta} \, \operator{a}^\dagger_\alpha \operator{a}^\dagger_\beta \operator{a}_\delta^{} \operator{a}_\gamma^{}.
    \label{eq:Vcreation}
\end{equation}

Cette forme est particulièrement utile pour le traitement des interactions dans l’espace de Fock, notamment en théorie des champs et en physique des gaz quantiques.

\subparagraph{Expression générale d’un terme à deux corps. }

Un terme d’interaction à deux corps général peut s’écrire :
\begin{equation}
    \operator{V} = \frac{1}{2} \int dx_1^{} \, dx_2^{} \, dx_1' \, dx_2' \; 
    \bra{ 1 : x_1^{}, 2 : x_2^{} } \operator{v} \ket{ 1 : x_1', 2 : x_2' } \,
    \operator{\Psi}^\dagger(x_1^{}) \, \operator{\Psi}^\dagger(x_2^{}) \, 
    \operator{\Psi}(x_2') \, \operator{\Psi}(x_1')
    \label{eq:V_general}
\end{equation}

\begin{mdframed}[
	linewidth=0.5pt, 
	backgroundcolor=gray!5, 
	roundcorner=50pt,	
	innerleftmargin=5pt,
    innerrightmargin=5pt,
    innertopmargin=-10pt,
    innerbottommargin=2pt,
    leftmargin=2pt,
    rightmargin=2pt
	]
\subparagraph{Interactions ponctuelles.} 
Dans le cas d’une interaction ne dépendant que de la distance relative entre deux particules, cette expression se simplifie :
\(
     \operator{V} = \frac{1}{2} \sum_{i, j \ne i}  \operator{v}(x_i^{} - x_j^{}) = 
    \frac{1}{2} \int dx_1^{} \, dx_2^{} \; v(x_1^{} - x_2^{}) \,
    \operator{\Psi}^\dagger(x_1^{}) \, \operator{\Psi}^\dagger(x_2^{}) \, 
    \operator{\Psi}(x_2^{}) \, \operator{\Psi}(x_1^{})
    \label{eq:V_local}
\) soit pour des interactions ponctuelles :	
\begin{eqnarray}
	\quad \operator{V}  = \frac{g}{2} \int dx \,
    \operator{\Psi}^\dagger(x) \, \operator{\Psi}^\dagger(x) \, 
    \operator{\Psi}(x) \, \operator{\Psi}(x)  		
\end{eqnarray}
et quand on l'applique à l'état $\ket{\{k_1 , \cdots , k_N\}} $ , les règles de commutations (\ref{chap:1:com.1}) et la définition d'état de Fock (\ref{chap:eq.vide.fock}) impliquent que (cf Annex \ref{annex:N.part})
\begin{eqnarray}
\operator{V}\ket{\{k_1 , \cdots , k_N\}} =  \int d^Nz \, \operator{\mathcal{V}}_N \varphi_{\{k_a\}}(z_1 , \cdots , z_N )\operator{\Psi}(z_1)\cdots \operator{\Psi}^\dag(z_N) \ket{\emptyset} 
\end{eqnarray}
avec 
\(
	\operator{\mathcal{V}}_N 	
 = g\sum_{1\leq i < j \leq N } \operator{\delta}(x_i - x_j)	
\)
où \( g \) est la constante de couplage.
\end{mdframed}


%Le hamiltonien général décrivant des particules identiques en interaction s’écrit alors :
%\begin{equation}
%    \hat{H} = \int d\vec{r} \; \hat{\psi}^\dagger(\vec{r}) 
%    \left( -\frac{\hbar^2}{2m} \Delta + u(\vec{r}) - \mu \right) 
%    \hat{\psi}(\vec{r})
%    + \frac{1}{2} \int d\vec{r} \, d\vec{r}' \; v(\vec{r} - \vec{r}') \,
%    \hat{\psi}^\dagger(\vec{r}') \, \hat{\psi}^\dagger(\vec{r}) \,
%    \hat{\psi}(\vec{r}) \, \hat{\psi}(\vec{r}')
%    \label{eq:H_general}
%\end{equation}

%\noindent
%Bien que cette expression ait une interprétation physique très claire, il est important de garder à l'esprit que \( \hat{H} \) et \( \hat{\psi} \) sont des objets du formalisme à plusieurs corps.



%%%%%%%%%%%%%%%%
%........................

%\subsubsection{Seconde quantification}



%\paragraph{Hamiltoniens en seconde quantification.}
%\subparagraph{Terme à un corps.}
%Un hamiltonien à un corps, correspondant à une énergie cinétique ou un potentiel externe, s’écrit :
%\[
%\hat{\mathcal{H}}_1 = \int dx\, \operator{\Psi}^\dagger(x) \hat{h}(x) \operator{\Psi}(x),
%\]
%où \( \hat{h}(x) \) est l’opérateur d’un corps (ex. : \( -\frac{\hbar^2}{2m} \partial_x^2 + V(x) \)).

%\subparagraph{Terme à deux corps.}
%Les interactions entre particules, modélisées par une interaction à deux corps \( V(x - y) \), s’expriment comme :
%\[
%\hat{\mathcal{H}}_2 = \frac{1}{2} \int dx\,dy\, \operator{\Psi}^\dagger(x) \operator{\Psi}^\dagger(y) V(x - y) \operator{\Psi}(y) \operator{\Psi}(x).
%\]


%.......................


\paragraph{Expression de l’Hamiltonien. }
L’hamiltonien dans ce formalisme s’écrit en termes des opérateurs de champ, par exemple pour l’énergie cinétique et les interactions ponctuelles avec $\hbar = m = 1 $  :

%Le Hamiltonien du modèle est donné par

%\begin{eqnarray}
%	\operator{H} & = & \int dx \, \left [ \operator{\partial}_x \operator{\Psi}^\dag (x)\operator{\partial}_x \operator{\Psi}(x) + c \operator{\Psi}^\dag (x) \operator{\Psi}^\dag (x) \operator{\Psi} (x) \operator{\Psi} (x) \right ] \label{chap:1:ham.mod}
%\end{eqnarray}

\begin{eqnarray}
	\operator{H} & = & \int dx \, \operator{\Psi}^\dag (x)\left [-\frac{1}{2}\operator{\partial}_x^2 + \frac{g}{2}  \operator{\Psi}^\dag (x) \operator{\Psi} (x) \right ] \operator{\Psi} (x) \label{chap:1:ham.mod}.
\end{eqnarray}

Quand on l'applique à l'état $\ket{\{\theta_1 , \cdots , \theta_N \}} $, avec $\theta_i$ homogène à des nombres d'onde ou à des vitesse , il vient que %, les règles de commutations (\ref{chap:1:com.1}) et la définition d'état de Fock (\ref{chap:eq.vide.fock}) impliquent que (cf Annex \ref{annex:N.part})
\begin{eqnarray}
\operator{H}\ket{\{\theta_1 , \cdots , \theta_N\}} =  \int d^Nz \, \operator{\mathcal{H}}_N \varphi_{\{\theta_a\}}(z_1 , \cdots , z_N )\operator{\Psi}(z_1)\cdots \operator{\Psi}^\dag(z_N) \ket{\emptyset} 
\end{eqnarray}
avec 
\(
	\operator{\mathcal{H}}_N 	
 =  \operator{\mathcal{K}}_N  +  \operator{\mathcal{V}}_N .	
\)


%où \( g \) est la constante de couplage. %Dans ce chapitre, nous considérons uniquement les propriétés du système à un instant donné, de sorte que la dépendance temporelle des champs est omise pour alléger l’écriture.

Ce formalisme est ainsi adapté pour décrire des condensats de Bose, des gaz quantiques, ou la création/annihilation de particules dans les champs quantiques.

\paragraph{Équation du mouvement associée.}

L’équation du mouvement du champ \( \Psi(x) \) est obtenue à partir de l’équation de Heisenberg :

\begin{eqnarray}
	i\operator{\partial}_t	\operator{\Psi} & = & [ \operator{\Psi} , \operator{H} ]
\end{eqnarray}

ce qui, après évaluation explicite du commutateur (\ref{chap:1:com.1}), conduit à :


%\begin{eqnarray}
%	i \operator{\partial}_t \operator{\Psi}	 & = & - \operator{\partial}_x^2 \operator{\Psi} + 2c \operator{\Psi}^\dag\operator{\Psi} \operator{\Psi}
%\end{eqnarray}

\begin{eqnarray}
	i \operator{\partial}_t \operator{\Psi}	 & = & - \frac{1}{2}\operator{\partial}_x^2 \operator{\Psi} + g \operator{\Psi}^\dag\operator{\Psi} \operator{\Psi}
\end{eqnarray}

est appelée l'équation de \textbf{Schrödinger non linéaire (NS)}.

Pour $g > 0$, l'état fondamental à température nulle est une sphère de Fermi. Seul ce cas sera considéré par la suite.

%\vspace{0.5cm}

\subsubsection*{Conclusion}

La première quantification est la base indispensable qui permet de comprendre le comportement quantique d’un nombre fixé de particules. La seconde quantification en est une extension naturelle, nécessaire pour décrire des systèmes plus complexes où le nombre de particules peut varier. Elle repose sur la quantification des champs, et l’introduction d’opérateurs créant ou détruisant ces particules, ouvrant ainsi la voie à la physique quantique des champs et à de nombreuses applications modernes.


\subsection{Opérateurs nombre de particules et moment dans la formulation quantique du gaz de Lieb-Liniger}

Dans cette section, nous nous intéressons aux opérateurs fondamentaux que sont le {\em nombre total de particules} $\operator{Q}$ et le {\em moment total} $\operator{P}$, dans le cadre du gaz de bosons unidimensionnel décrit par l’Hamiltonien de Lieb-Liniger. Après avoir introduit ces opérateurs dans le langage de la seconde quantification, nous montrons qu’ils sont {\em conservés} par la dynamique, et qu’ils admettent les {\em mêmes états propres} que l’Hamiltonien. Nous donnons ensuite leur expression dans la représentation à  $N$ particules, ainsi que la forme explicite de leurs valeurs propres en fonction des {\em rapidités} $\theta_a$ , illustrant la structure polynomiale typique des intégrales du mouvement dans les systèmes intégrables.

\subsubsection{Définition en seconde quantification}

Les opérateurs du nombre total de particules $\operator{Q}$ et du moment total $\operator{P}$ s’écrivent en seconde quantification comme suit :
\begin{eqnarray}
	\operator{Q}  =  \int \operator{\Psi}^\dag (x) \operator{\Psi} (x) \, d x, \quad 
	\operator{P}  =  - \frac{i}2 \int \left \{  \operator{\Psi}^\dag(x) \operator{\partial}_x \operator{\Psi}(x) - \left [ \operator{\partial}_x \operator{\Psi}^\dag(x)\right ] \operator{\Psi}(x)\right \} dx \label{eq.1.7}
\end{eqnarray}
Ces deux opérateurs sont {\em hermitiens}, et représentent des observables physiques fondamentales : le nombre de particules et la quantité de mouvement du système.

\subsubsection{Conservation et commutation}
Ces opérateurs commutent avec l’Hamiltonien $\operator{H}$ du modèle de Lieb-Liniger :
\begin{eqnarray}
[ \operator{H} , \operator{Q} ] = 0, \quad [ \operator{H} , \operator{P} ] = 0.
\end{eqnarray}
Ils constituent ainsi des intégrales du mouvement. Cette propriété est une manifestation de la symétrie translationnelle du système (pour $\operator{P}$) et de la conservation du nombre total de particules (pour $\operator{Q}$).

\begin{mdframed}[
	linewidth=0.5pt, 
	backgroundcolor=gray!5, 
	roundcorner=50pt,	
	innerleftmargin=5pt,
    innerrightmargin=5pt,
    innertopmargin=5pt,
    innerbottommargin=2pt,
    leftmargin=2pt,
    rightmargin=2pt
	]
	Nous verrons au chapitre 2 que cette situation s’étend à une {\bf \em infinité d’intégrales du mouvement} dans les systèmes intégrables, ce qui permettra de construire l’ensemble de Gibbs généralisé (GGE).
\end{mdframed}

\subsubsection{États propres et valeurs propres}
Les états propres $\ket{\{\theta_a\}}$, construits dans le cadre de la seconde quantification à partir de la solution du modèle de Lieb-Liniger, sont simultanément fonctions propres des opérateurs $\operator{Q}$, $\operator{P}$ et $\operator{H}$ :
\begin{eqnarray}
\operator{Q} \ket{\{\theta_a\}} = N \ket{\{\theta_a\}}, \quad
\operator{P} \ket{\{\theta_a\}} = \left( \sum_{a=1}^N \theta_a \right) \ket{\{\theta_a\}}, \
\operator{H} \ket{\{\theta_a\}} = \left( \frac{1}{2} \sum_{a=1}^N \theta_a^2 \right) \ket{\{\theta_a\}}.
\end{eqnarray}
Autrement dit, les valeurs propres associées à ces trois opérateurs sont données par :
\begin{eqnarray}
N = \sum_{a = 1}^N \theta_a^0, \quad p = \sum_{a = 1}^N \theta_a, \quad e = \frac{1}{2} \sum_{a = 1}^N \theta_a^2.
\end{eqnarray}
Cela illustre que les trois premières intégrales du mouvement du système — nombre, moment, énergie — peuvent être exprimées comme des {\bf \em moments successifs} des rapidités.	

\subsubsection{Forme en première quantification}
En utilisant la représentation en espace de configuration $\{z_a\} \equiv \{z_1 , \cdots , z_N \}$, les opérateurs $\operator{Q}$ et $\operator{P}$ agissent comme suit sur les fonctions d’onde $\varphi_{\{\theta_a\}}(\{z_a\})$ :
\begin{eqnarray}
	\operator{Q}\ket{\{\theta_a\}} =  \sqrt{N!}\int d^Nz \, \operator{\mathcal{N}} \varphi_{\{\theta_a\}}(\{z_a\} )\ket{\{z_a\}}, \, \operator{P}\ket{\{\theta_a\}} =  \sqrt{N!}\int d^Nz \, \operator{\mathcal{P}}_N \varphi_{\{\theta_a\}}(\{z_a\} )\ket{\{z_a\}} 
\end{eqnarray}
où les opérateurs associés agissant sur les fonctions d’onde à $N$ particules sont :
\begin{eqnarray}
	\operator{ \mathcal{N}}  =  \sum_{k = 1}^N 1 = N ,~\operator{ \mathcal{P}}_N  = -i \sum_{k = 1}^N k =- i\sum_{k = 1}^N \operator{\partial}_{z_k}	
\end{eqnarray}

Ces formes découlent directement des règles de commutation canonique (\ref{chap:1:com.1}) et de la définition des opérateurs en seconde quantification (\ref{chap:eq.vide.fock}) (cf. annexes \ref{annex:N.part}).

\subsubsection{Conclusion}
Ainsi, les opérateurs $\operator{Q}$ , $\operator{P}$ et $\operator{H}$ possèdent une structure diagonale commune dans la base des états propres $\ket{\{\theta_a\}}$, révélant la nature intégrable du modèle de Lieb-Liniger. Leurs valeurs propres sont respectivement les 0e, 1er et 2e moments des rapidités. Cette structure permet de généraliser la construction à une hiérarchie complète d’observables conservées, qui seront présentées au chapitre suivant.


\subsection{Fonction d’onde et Hamiltonien et moment à 2 corps}

%Nous considérons à présent le cas de deux bosons quantiques dans la même boîte unidimensionnelle de longueur \(L\), avec des conditions aux limites périodiques. Contrairement au cas à une particule, le terme d’interaction à contact intervient dans la dynamique. L'hamiltonien à 2 particule s'écrit :
%En première quantification, en utilisant les coordonnées du centre de masse et relatives $Z = (z_1 + z_2)/2$ et $Y = z_1 - z_2$, il vient que
%l'hamiltonien (\ref{chap:1:hal.mod.2.part.3}) se divise en une somme de deux problèmes indépendants à une seule particule.
%Les états propres de l'hamiltonien du centre de masse de masse $\overline{m}= 2$, $-\frac{1}{4} \partial_Z^2$, sont des ondes planes, et l'hamiltonien pour la coordonnée relative $Y$ correspond à celui d'une particule de masse réduite $\tilde{m} = 1/2$ en présence d'un potentiel delta en $Y = 0$. 
%\paragraph{Introduction au système à deux bosons avec interaction de contact.}
%Nous considérons à présent le cas de deux bosons quantiques dans une même boîte unidimensionnelle de longueur \(L\), avec des conditions aux limites périodiques. Contrairement au cas à une particule, un terme d’interaction de contact intervient ici dans la dynamique. L’Hamiltonien à deux particules s’écrit :
%\begin{eqnarray}
%	\operator{\mathcal{H}}_2  =  \operator{\mathcal{K}}_2 +\operator{\mathcal{V}}_2  & avec & \operator{\mathcal{K}}_2 =   - \frac{1}{2} \partial_{z_1}^2 - \frac{1}{2} \partial_{z_2}^2,  \quad \mbox{et} \quad  \operator{\mathcal{V}}_2  =  	g  \delta(z_1 - z_2). \label{chap:1:hal.mod.2.part.3} 		
%\end{eqnarray}
%On rappelle que l'énergies propres de  $\operator{\mathcal{K}}_2$ associées aux fonction d'ondes $\varphi_{\{ \theta_1 , \theta_2 \}}$ , la masse des particule étant égale à 1 (ie $\hbar= m=1$) s'écrit 
%\begin{eqnarray}
%	\varepsilon(\theta_1) + 	\varepsilon(\theta_2) & = & \frac{\theta_1^2}{2} + \frac{\theta_2^2}{2} 
%\end{eqnarray}
%On vas travailler dans le centre de masse.

%\paragraph{Changement de variables : coordonnées du centre de masse et relatives.}
 
%En première quantification, en introduisant les coordonnées du centre de masse \(Z = \frac{z_1 + z_2}{2}\) et relative \(Y = z_1 - z_2\), on obtient :
%\(
%	\partial_{z_1}^2 + \partial_{z_2}^2 = \frac{1}{2} \partial_Z^2 + 	2\partial_Y^2.  
%\)
%L’Hamiltonien~\eqref{chap:1:hal.mod.2.part.3} se décompose alors en une somme de deux problèmes indépendants à une seule variable :

%\begin{eqnarray}\label{chap:1:hal.mod.2.part.4}
%	\operator{\mathcal{H}}_2  =  	-\frac{1}{4} \partial_Z^2 + \operator{\mathcal{H}}_{rel} , \quad \mbox{avec}\quad  \operator{\mathcal{H}}_{rel} =  - 	\partial_Y^2 + g \delta ( Y ). 
%\end{eqnarray}

%\paragraph{Résolution du problème de centre de masse et de coordonnée relative.}

%Les états propres de l’Hamiltonien associé au centre de masse, \(-\frac{1}{4} \partial_Z^2\), correspondant à une particule de masse totale \(\bar{m} = 2\), sont des ondes planes associés à l'énergie $\overline{\theta}^2$ avec $\overline{\theta} = \frac{ \theta_1 + \theta_2}{2}$. L’Hamiltonien, $\operator{\mathcal{H}}_{rel}$, associé à la coordonnée relative \(Y\) correspond quant à lui à celui d’une particule de masse réduite \(\tilde{m} = \frac{1}{2}\), soumise à un potentiel delta en \(Y = 0\) :
%\begin{eqnarray}\label{chap:1:hal.mod.2.part.5}
%	- 	\partial_Y^2 \tilde{\varphi}(Y) + g \delta ( Y )\tilde{\varphi}(Y) & = & \tilde{\varepsilon}\,\tilde{\varphi}(Y),
%\end{eqnarray}
%où $\tilde{\varepsilon}$ est l’énergie propre du problème relatif.

%%%%%%
\paragraph{Introduction au système de deux bosons avec interaction de contact.}

Considérons maintenant un système de deux bosons quantiques confinés dans une boîte unidimensionnelle de longueur \(L\), avec des conditions aux limites périodiques. Contrairement au cas à une seule particule, une interaction de contact intervient ici dans la dynamique. L’Hamiltonien à deux particules s’écrit :
\begin{eqnarray}
	\operator{\mathcal{H}}_2 = \operator{\mathcal{K}}_2 + \operator{\mathcal{V}}_2, \quad \text{avec} \quad \operator{\mathcal{K}}_2 = - \frac{1}{2} \partial_{z_1}^2 - \frac{1}{2} \partial_{z_2}^2, \quad \text{et} \quad \operator{\mathcal{V}}_2 = g \, \delta(z_1 - z_2). \label{chap:1:hal.mod.2.part.3}
\end{eqnarray}

On rappelle que, pour des particules de masse unitaire (i.e., \(\hbar = m = 1\)), les énergies propres de l’opérateur cinétique \(\operator{\mathcal{K}}_2\), associées aux fonctions d’onde symétrisées \(\varphi_{\{ \theta_1 , \theta_2 \}}\), sont données par :
\begin{eqnarray}
	\varepsilon(\theta_1) + \varepsilon(\theta_2) = \frac{\theta_1^2}{2} + \frac{\theta_2^2}{2}.
\end{eqnarray}

Afin de simplifier le problème, nous nous plaçons dans le référentiel du centre de masse.

\paragraph{Changement de variables : coordonnées du centre de masse et relative.}

En première quantification, on introduit les nouvelles variables :
\(
Z = \frac{z_1 + z_2}{2} \, \text{(centre de masse)}, \qquad Y = z_1 - z_2 \, \text{(coordonnée relative)}.
\)
Dans ce changement de variables, l’opérateur laplacien total devient :
\(
\partial_{z_1}^2 + \partial_{z_2}^2 = \frac{1}{2} \partial_Z^2 + 2 \, \partial_Y^2.
\)
L’Hamiltonien~\eqref{chap:1:hal.mod.2.part.3} se décompose alors en la somme de deux Hamiltoniens agissant respectivement sur \(Z\) et \(Y\) :
\begin{eqnarray}\label{chap:1:hal.mod.2.part.4}
	\operator{\mathcal{H}}_2 = -\frac{1}{4} \partial_Z^2 + \operator{\mathcal{H}}_{\text{rel}}, \qquad \text{avec} \quad \operator{\mathcal{H}}_{\text{rel}} = - \partial_Y^2 + g \, \delta(Y).
\end{eqnarray}

\paragraph{Résolution du problème du centre de masse et de la coordonnée relative.}

L’Hamiltonien du centre de masse, \(-\frac{1}{4} \partial_Z^2\), décrit une particule de masse totale \(\bar{m} = 2\). Ses états propres sont des ondes planes associées à une énergie \(\overline{\theta}^2\), avec :
\(
\overline{\theta} = \frac{\theta_1 + \theta_2}{2},
\)
jouant ici un rôle analogue à celui d’un pseudo-moment associé dans le référentielle de laboratoire.
Le Hamiltonien relatif, \(\operator{\mathcal{H}}_{\text{rel}}\), correspond quant à lui à une particule de masse réduite \(\tilde{m} = \frac{1}{2}\) soumise à un potentiel delta centré en \(Y = 0\). Son équation propre s’écrit :
\begin{eqnarray}\label{chap:1:hal.mod.2.part.5}
	- \partial_Y^2 \, \tilde{\varphi}(Y) + g \, \delta(Y) \, \tilde{\varphi}(Y) = \tilde{\varepsilon} \, \tilde{\varphi}(Y),
\end{eqnarray}
où \(\tilde{\varepsilon}\) désigne l’énergie associée au mouvement relatif.
%%%%%%%%%%%%%%

\paragraph{Forme symétrique de la fonction d'onde pour bosons.}
Dans le référentiel du centre de masse. Le système est le même que que celuis d'un particules de masse $\tilde{m}= \frac{1}{2}$.
Le système étant composé de particules bosoniques, on cherche une solution symétrique que l’on écrit sous la forme  :
\begin{eqnarray}
	\tilde{\varphi}(Y) ~=~a~e^{i\frac{1}{2} \tilde{\theta} \vert Y \vert } + b~e^{-i\frac{1}{2} \tilde{\theta}\vert Y \vert } ~\propto~  \sin\left( \frac{1}{2} (\tilde{\theta} |Y| + \Phi ) \right). \label{eq:ansatz.boson}
\end{eqnarray}
Le paramètre \( \tilde{\theta} = \theta_1 - \theta_2 \) joue ici un rôle analogue à celui d’un pseudo-moment associé à la coordonnée relative,
est  la phase s'écrit
\begin{eqnarray}
	\Phi(\tilde{\theta}) &=& 2 \arctan\left (\frac{1}{i} \frac{a+b}{a-b}\right),	\label{chap:1:dif.mod.2.part.1} 
\end{eqnarray}
car \( a\exp(ix)+b\exp(-ix) = 2\sqrt{ab}\sin\left(x+\arctan\left(-i\, \frac{a+b}{a-b}\right)\right) \). Pour $\tilde{\theta}<0$, les termes exponentiels \( \exp(i\tilde{\theta} \vert Y \vert/2 ) \) et \( \exp(-i\tilde{\theta} \vert Y \vert/2 ) \) correspondent aux paires de particules entrantes et sortantes d’un processus de diffusion à deux corps.


%En réinjectant l'équation \eqref{eq:ansatz.boson} dans l’équation \eqref{chap:1:hal.mod.2.part.5}, on obtient l’énergie propre du problème réduit $\tilde{\varepsilon}$ associé à l’état lié. Celle-ci prend la forme classique de l’énergie cinétique d’une particule, \( \frac{1}{2} \times \text{masse} \times \text{vitesse}^2 \), la masse réduite du problème étant ici \( \tilde{m} = \frac{1}{2} \), et où \( \tilde{\theta} \) joue un rôle analogue à celui d’une vitesse. On en déduit :
%\begin{eqnarray}\tilde{\varepsilon}(\tilde{\theta})  & = &  \frac{1}{2} \cdot \tilde{m} \cdot \tilde{\theta}^2 = \frac{1}{2} \cdot \frac{1}{2} \cdot \tilde{\theta}^2 = \frac{\tilde{\theta}^2}{4}.\end{eqnarray}
%\begin{eqnarray}
%	\tilde{\varepsilon}(\tilde{\theta})  & = &  \frac{\tilde{\theta}^2}{4}.
%\end{eqnarray}
% Il encode la décroissance exponentielle de la fonction d’onde liée dans l’espace relatif, et sa valeur est directement reliée à la profondeur de l’état lié. Une valeur plus grande de \( \tilde{\theta} \) correspond à un état plus fortement lié, c’est-à-dire plus localisé autour de \( Y = 0 \), ce qui reflète une interaction plus attractive entre les deux particules. $\overline{\theta}^2 +  \tilde{\varepsilon}(\tilde{\theta}) = \varepsilon{\theta_1} + \varepsilon{\theta_2}$.
En réinjectant l’ansatz~\eqref{eq:ansatz.boson} dans l’équation relative
\eqref{chap:1:hal.mod.2.part.5}, on obtient l’énergie propre
\(\tilde{\varepsilon}\) du problème réduit.  
Elle prend la forme cinétique usuelle
\(\tfrac{1}{2}\times\text{masse}\times\text{vitesse}^{2}\).  
La masse réduite vaut ici \(\tilde{m}= \frac{1}{2}\) et le paramètre
\(\tilde{\theta}\) joue le rôle d’une impulsion ; ainsi
\begin{equation}
   \tilde{\varepsilon}(\tilde{\theta})
   \;=\;
   \frac{1}{2}\,\tilde{m}\,\tilde{\theta}^{2}
   \;=\;
   \frac{1}{2}\times\frac{1}{2}\,\tilde{\theta}^{2}
   \;=\;
   \frac{\tilde{\theta}^{2}}{4}.
   \label{eq:energie_relative}
\end{equation}

Cette énergie gouverne la décroissance exponentielle de la fonction
d’onde dans la coordonnée relative : plus \(\tilde{\theta}\) est grand,
plus l’état est localisé autour de \(Y=0\), signe d’une interaction
attractive plus forte entre les deux bosons.

L’énergie totale se décompose enfin en la somme du mouvement du centre
de masse et du mouvement relatif :
\(
   \overline{\theta}^{2}
   \;+\;
   \tilde{\varepsilon}(\tilde{\theta})
   \;=\;
   \varepsilon(\theta_{1})
   \;+\;
   \varepsilon(\theta_{2}),
\)
où \(\overline{\theta}= \tfrac{\theta_{1}+\theta_{2}}{2}\) et
\(\varepsilon(\theta)=\theta^{2}/2\).






%%%%%%%%%%%%%%%%%%%%%%%%%%%
\paragraph{Condition de discontinuité à cause du potentiel delta.}
En raison de la présence du potentiel delta centré en $Y = 0$, la dérivée première de la fonction d’onde $\tilde{\varphi}(Y)$ présente une discontinuité en ce point. En effet, le potentiel étant infini en $Y = 0$, la phase $\Phi$ du régime symétrique est déterminée en intégrant l’équation du mouvement autour de la singularité. En intégrant entre $- \epsilon$ et $+ \epsilon$ et en faisant tendre $\epsilon \to 0$, on obtient la condition de saut de la dérivée :

%avec $\Phi$ une phase à déterminer. %\begin{equation}
%	E = \frac{\tilde{m} \theta^2}{2}.
%\end{equation}

%La dérivée de la fonction d’onde n’est pas continue en $Y = 0$. Le potentiel étant infini en $Y = 0$, la phase $\Phi$ est obtenue en intégrant l’équation du mouvement entre $- \epsilon$ et $+ \epsilon$ et en faisant tendre $\epsilon$ vers zéro :


%En raison de ce potentiel delta, la dérivée première de la fonction d'onde $\varphi(Y)$ doit avoir une discontinuité en $Y = 0$ : 

%{\color{lightgray} 
%\begin{eqnarray*}
%	\underset{ \epsilon \to 0 }{\lim} \int_{-\epsilon}^{+\epsilon}  	-\underbrace{\cancel{\frac{1}{4} \partial_Z^2\varphi(Y)}}_{0} - 	\partial_Y^2\varphi(Y) + c \delta ( Y )\varphi(Y) \, dY  & = & \underset{ \epsilon \to 0 }{\lim}  \int_{-\epsilon}^{+\epsilon}  E d Y , \\
%	\underset{ \epsilon \to 0 }{\lim}  \left [ \varphi'(\epsilon) - \varphi'(-\epsilon) \right ] - c \varphi (  0 ) & =  &  -\underset{ \epsilon \to 0 }{\lim}  \int_{-\epsilon}^{+\epsilon}  E d Y,\\
%	 \varphi'(0^+) - \varphi'(0^-) - c \varphi (  0 ) & = & 0 .
%\end{eqnarray*}


%}

\begin{eqnarray*}
	\underset{ \epsilon \to 0 }{\lim} \int_{-\epsilon}^{+\epsilon}  - 	\partial_Y^2\tilde{\varphi}(Y) + g \delta ( Y )\tilde{\varphi}(Y) \, dY  & = & \underset{ \epsilon \to 0 }{\lim}  \int_{-\epsilon}^{+\epsilon}  \tilde{\varepsilon}(\tilde{\theta})d Y ,\\
	\\
	\tilde{\varphi}'(0^+) - \tilde{\varphi}'(0^-) - g \tilde{\varphi} (  0 ) & = & 0 .
\end{eqnarray*}


%soit $\tilde{\varphi}'(0^+) - \tilde{\varphi}'(0^-) - c \tilde{\varphi} (  0 )  =  0 $ .

%%%%%%%%%%%%%%%
\paragraph{Détermination de la phase $\Phi$.}
Et en évaluant la discontinuité de sa dérivée au point $Y = 0$, on trouve que la phase $\Phi$ satisfait la condition :

%\begin{equation}
%	\tan\left( \frac{\Phi}{2} \right) = \frac{\tilde{\theta}}{c}.
%\end{equation}

\begin{eqnarray}\label{chap:1:dif.mod.2.part.2}
	\Phi(\tilde{\theta}) & = & 2 \arctan (\tilde{\theta}/g) \in [ - \pi , +\pi ].
\end{eqnarray}

%{\color{red}( à revoir)} Cette relation exprime l’impact de l’interaction delta sur le déphasage de la solution liée. On en déduit que plus le couplage $g$ est fort ($g \to \infty$), plus la phase $\Phi$ se rapproche de $0$, ce qui correspond à une fonction d’onde présentant s'annulant en $Y = 0$. En revanche, dans la limite d’interaction faible ($g \to 0$), la phase $\Phi$ tend vers $\pm \pi$ et la discontinué de la dérivé de la fonction d'onde devient négligeable.
%Cette relation exprime l’impact de l’interaction de type delta sur le déphasage de la fonction d’onde liée.On en déduit que plus le couplage $g$ est fort ($g \to \infty$), la phase $\Phi$ se rapproche de $0$, ce qui correspond à une fonction d’onde présentant s'annulant en $Y = 0$, à l’image du régime d’imperméabilité totale.
%À l’inverse, dans la limite d’interaction faible (\( g \to 0 \)), si bien que \( \Phi \) tend vers $\pi$ (ou \( -\pi \), selon le signe de \( \tilde{\theta} \)). Dans ce cas, la discontinuité de la dérivée de la fonction d’onde au point \( Y = 0 \) devient négligeable, ce qui traduit un couplage quasi inexistant entre les deux particules.
%Cette relation exprime l’impact de l’interaction de type delta sur le déphasage de la fonction d’onde liée. Lorsque le couplage \( g \) devient très fort (\( g \to \infty \)), la fraction \( \tilde{\theta}/g \to 0 \), et la phase \( \Phi(\tilde{\theta}) \to 0 \). Cela correspond à une situation dans laquelle la fonction d’onde est fortement contrainte à s’annuler en \( Y = 0 \), à l’image du régime d’imperméabilité totale.
%À l’inverse, dans la limite d’interaction faible (\( g \to 0 \)), la fraction \( \tilde{\theta}/g \to \infty \), si bien que \( \Phi(\tilde{\theta}) \to \pi \) (ou \( -\pi \), selon le signe de \( \tilde{\theta} \)). Dans ce cas, la discontinuité de la dérivée de la fonction d’onde au point \( Y = 0 \) devient négligeable, ce qui traduit un couplage quasi inexistant entre les deux particules.

Cette relation exprime l’impact de l’interaction de type delta sur le déphasage de la fonction d’onde liée. On en déduit que plus le couplage \( g \) est fort (\( g \to \infty \)), plus la phase \( \Phi \) se rapproche de zéro. Cela correspond à une fonction d’onde qui s’annule en \( Y = 0 \), caractéristique d’un régime d’imperméabilité totale.

À l’inverse, dans la limite d’une interaction faible (\( g \to 0 \)), la phase \( \Phi \) tend vers \( \pi \) (ou \( -\pi \), selon le signe de \( \tilde{\theta} \)). Dans ce cas, la discontinuité de la dérivée de la fonction d’onde au point \( Y = 0 \) devient négligeable, ce qui traduit une interaction presque absente entre les deux particules.


%%%%%%%%%%%%%%%%%%%%%%%%%%%%%%%%%%%
%\paragraph{Phase de diffusion à un corp.}
%Les équations \eqref{chap:1:dif.mod.2.part.1} et \eqref{chap:1:dif.mod.2.part.2}  et en remarquant que pour $z \in \mathbb{C} \backslash \{ \pm i \} 2\artan(z) = i \ln \left( \frac{ 1 - i z }{1+iz} \right ) $ soit $\exp(2i\arctan(x)) = (1 + ix)/(1 - ix)$ et $\Phi(\tilde{\theta}) = i \ln ( - b/a ) $  donne rapport entre les amplitudes $a$ et $b$ de la fonction d'onde \eqref{eq:ansatz.boson} définit la phase de diffusion / {\em matrice diffusion} $S( \tilde{\theta}) \doteq e^{i\Phi ( \tilde{\theta}  ) }$  :

%\begin{eqnarray}
%	e^{i\Phi ( \tilde{\theta}  ) } &=& -\frac{a}{b} ~=~\frac{1 +i\tilde{\theta}/g} { 1 - i\tilde{\theta}/g} .\label{chap:1:dif.mod.2.part.3}
%\end{eqnarray}

\paragraph{Phase de diffusion à deux corps.}

En combinant les équations~\eqref{chap:1:dif.mod.2.part.1} et~\eqref{chap:1:dif.mod.2.part.2} avec l’identité analytique valable pour tout
\(z \in \mathbb{C}\setminus\{\pm i\}\),
\(
2\arctan(z)=i\ln\!\left(\frac{1-iz}{1+iz}\right)
\Leftrightarrow
e^{2i\arctan(z)}=\frac{1+iz}{1-iz},
\)
on obtient que le rapport des amplitudes \(a\) et \(b\) de la fonction
d’onde relative~\eqref{eq:ansatz.boson} définit la {\em phase de diffusion }
\(
\Phi(\tilde{\theta}) = i\ln\!\left(-\frac{b}{a}\right).
\)
On introduit alors la {\em matrice de diffusion} (ou facteur de diffusion)
\begin{eqnarray}
	S(\tilde{\theta}) \;\doteq\; e^{i\Phi(\tilde{\theta})}= -\frac{a}{b}= \frac{1 + i\,\tilde{\theta}/g}{1 - i\,\tilde{\theta}/g}.%\tag{\ref{chap:1:dif.mod.2.part.3}}
\end{eqnarray}
%où \(g\) est le paramètre d’interaction et
%\(\tilde{\theta} = \theta_1 - \theta_2\) le pseudo‑moment relatif.  
Cette expression, unitaire et analytique, caractérise entièrement la diffusion élastique à deux corps dans le modèle considéré.



\paragraph{Lien entre phase de diffusion et décalage temporel : interprétation semi-classique. {\color{red}(à revoir)}}

Il a été souligné par {\color{black}Eisenbud (1948)} et {\color{black}Wigner (1955)} que la phase de diffusion peut être interprétée, de manière semi-classique, comme un {\em décalage temporel}. Esquissons brièvement l'argument de {\color{black}Wigner (1955)}.Tout d'abord, notons que, pour une particule unique, une approximation simple d’un paquet d’ondes peut être obtenue en superposant deux ondes planes avec des moments $\tilde{\theta}/2$ et $\tilde{\theta}/2 + \delta \tilde{\theta}$, respectivement :
\begin{eqnarray}
	\tilde{\varphi}_{\text{inc}}(Y) & \propto & e^{i\frac{1}{2}\tilde{\theta} \vert Y\vert} + e^{i\frac{1}{2}\left(\tilde{\theta} + 2\delta \tilde{\theta} \right) \vert Y\vert}.
\end{eqnarray}
Cette superposition évolue dans le temps comme :
\begin{eqnarray}
\tilde{\varphi}_{\text{inc}}(Y, t) &\propto &  e^{i\left( \frac{1}{2} \tilde{\theta}\vert Y\vert - t\,\tilde{\varepsilon}(\tilde{\theta}) \right)} + e^{i\left( \frac{1}{2}\left(  \tilde{\theta} + 2\delta \tilde{\theta} \right) \vert Y\vert - t\,\tilde{\varepsilon}(\tilde{\theta} + 2\delta \tilde{\theta}) \right)}.
\end{eqnarray}
%où l'on a utilisé l'expression de l'énergie réduite : $\tilde{\varepsilon}(\tilde{\theta}) = \tilde{\theta}^2 / 4$.
Le centre de ce 'paquet d'ondes' se situe à la position où les phases des deux termes coïncident, c'est-à-dire au point où $\vert Y\vert\delta \tilde{\theta}  - t[\tilde{\varepsilon}(\tilde{\theta} + 2\delta \tilde{\theta} ) - \tilde{\varepsilon}(\tilde{\theta})] = 0$, ce qui donne $\vert Y\vert \simeq \tilde{\theta} t$ avec la vitesse réduite $\tilde{\theta} = 1/2 \varepsilon'(\tilde{\theta}) $. %Ainsi, il s'agit effectivement d'un 'paquet d'ondes' se déplaçant à la vitesse $\theta$. Ensuite, considérons deux particules entrantes dans un état tel que le centre de masse $Z = (z_1 + z_2)/2$ ait une impulsion $\theta_1 - \theta_2$, tandis que la coordonnée relative $Y = z_1 - z_2$ se trouve dans un 'paquet d'ondes' se déplaçant à la vitesse $ (\theta_1 - \theta_2)/2$,
Selon les équations (\ref{eq:ansatz.boson}) et (\ref{chap:1:dif.mod.2.part.3}), l'état sortant de la diffusion correspondant serait :
\begin{eqnarray}
	\tilde{\varphi}_{outc} ( Y, t ) & \propto & -e^{i\Phi(\tilde{\theta})}e^{-i\frac{1}{2}\tilde{\theta} \vert Y\vert} - e^{i\Phi(\tilde{\theta} + 2 \delta \tilde{\theta} )}e^{-i\frac{1}{2}\left(\tilde{\theta} + 2\delta \tilde{\theta} \right) \vert Y\vert}. %\tag{2}
\end{eqnarray}
En répétant l'argument précédent de la stationnarité de phase, on trouve que la coordonnée relative est à la position $\vert Y \vert  \simeq \tilde{\theta} t - 2\Phi'( \tilde{\theta})$ au moment $t$. %Étant donné que le centre de masse n'est pas affecté par la collision et se déplace à la vitesse de groupe $\tilde{\theta} =(\theta_1 + \theta_2)/2$, nous constatons que la position des deux particules semiclassiques après la collision sera
\begin{eqnarray}
	\vert Y \vert & \simeq & 	\tilde{\theta} t  - 2 \Delta (\tilde{\theta} )
\end{eqnarray}
où le déplacement de diffusion $\Delta (\theta)$ est donné par la dérivée de la phase de diffusion,
\begin{eqnarray}\label{eq:I-1-16}
	\Delta ( \theta ) & \doteq & \frac{ d \Phi }{ d \theta } ( \theta )= \frac{ 2 g }{ g^2 + \theta^2} . 	
\end{eqnarray}


%\paragraph{Retour aux coordonnées du laboratoire.}
%En revenant aux coordonnées d'origine (celles du laboratoire), on constate que la fonction d'onde à deux corps 
%\(
%	\varphi_{\{\theta_1 , \theta_2\}} (z_1, z_2) = \langle \emptyset \vert \operator{\Psi} (z_1)\operator{\Psi} (z_2) \vert \{\theta_1, \theta_2\} \rangle,
%\)
%avec \(z_1 < z_2\) , (ie $Y>0$) . Et le centre de masse sur le mouvement
%\(
%	Z  =  \overline{\theta} t.	
%\)
%avec,  on rappelle , $\overline{\theta}$ la vitesse de groupe dans le référentielle de laboratoire.\\
%Nous constatons que la position des deux particules semiclassiques après la collision sera
%\begin{eqnarray}
%	z_1 ~=~ Z + \frac{Y}2 ~\simeq ~ \theta_1 t - \Delta(\theta_1 - \theta_2), & & 	z_2 ~=~ Z - \frac{Y}2 ~\simeq ~ \theta_2t + \Delta(\theta_1 - \theta_2),
%\end{eqnarray}

%avec  $\theta_1$ et $\theta_2$ on rappelle définie tel que 
%\(
%	\tilde{\theta} ~=~\theta_1 - \theta_2 , \,	\overline{\theta}~=~\frac{\theta_1 + \theta_2}{2}.	
%\)
%On remarquant que 
%\begin{eqnarray*}
%	z_1 \theta_1  + z_2  \theta_2 ~=~ 2Z\overline{\theta} + \frac{1}{2}Y\tilde{\theta}, & & z_1 \theta_2  + z_2  \theta_1 ~=~ 2Z\overline{\theta} - \frac{1}{2}Y\tilde{\theta}. 
%\end{eqnarray*}
%Ce qui est en accod avec la masse total $\overline{m} = 2$ et la masse résuite $\tilde{m} = \frac{1}{2}$ \\
%Ce qui nous motive à multiplier la fonction d'onde dans le référentiel du centre de masse \eqref{eq:ansatz.boson} par $\exp(2iZ\overline{\theta})$ pour obtenir 

%\begin{eqnarray}\label{eq:I-1-10}
%	\varphi_{\{\theta_1 , \theta_2\}}(z_1 , z_2) & \propto &  \left \{ \begin{array} { c cl} ( \theta_2 - \theta_1 - ic) e^{ i z_1 \theta_1 + iz_2 \theta_2 } - ( \theta_1 - \theta_2 - ic) e^{ i z_1 \theta_2 + iz_2 \theta_1} & \mbox{si} & z_1 < z_2 \\ (z_1 \leftrightarrow z_2) & \mbox{si} & z_1 > z_2 \end{array} \right.
%\end{eqnarray}

%correspondant aux valeurs propres

%\begin{eqnarray}
%	\varepsilon(\theta_1 , \theta_2) ~=~ \overbrace{ \overline{\theta}^2}^{\overline{\varepsilon}(\overline{\theta})}	 + \overbrace{\frac{1}{4} \tilde{\theta}^2}^{\tilde{\varepsilon}(\tilde{\theta})} ~=~ \frac{\theta_1}{2} + \frac{\theta_2}{2}.	
%\end{eqnarray}

%Pour $\theta_1 > \theta_2$, les deux termes $e^{iz_1 \theta_1 + iz_2 \theta_2 }$ et $e^{iz_1 \theta_2 + iz_2 \theta_1 }$ correspondent aux paires de particules entrantes et sortantes dans un processus de diffusion à deux corps. Le rapport de leurs amplitudes est la phase de diffusion à deux corps \eqref{chap:1:dif.mod.2.part.3} reste inchangé

%\begin{eqnarray}\label{chap:1:dif.mod.2.part.4}
%	e^{i\Phi ( \theta_1 - \theta_2  ) }~=~ -\frac{a}{b} ~=~\frac{\theta_1 - \theta_2  -ic} { \theta_2 - \theta_1  - ic}. 
%\end{eqnarray}


%%%%%%%%%%%%%%%%%%%%%%%%%%
\paragraph{Retour aux coordonnées du laboratoire.}

En revenant aux coordonnées du laboratoire, la fonction d’onde à deux corps s’écrit
\(
	\varphi_{\{\theta_1 , \theta_2\}} (z_1, z_2) 
	= \langle \emptyset \vert \operator{\Psi} (z_1)\operator{\Psi} (z_2) \vert \{\theta_1, \theta_2\} \rangle/\sqrt{2},
\)
dans le cas \(z_1 < z_2\), c’est-à-dire pour une séparation relative \(Y = z_1 - z_2 < 0\) (on pourra symétriser ultérieurement).  
Dans le référentiel du laboratoire, le centre de masse évolue selon
\(
	Z = \frac{z_1 + z_2}{2} = \overline{\theta}\,t.
\)
%où l’on rappelle que \(\overline{\theta} = \frac{\theta_1 + \theta_2}{2}\) est la vitesse de groupe du système dans le référentiel laboratoire.
Ainsi, la position semi-classique des deux particules après la collision s’écrit
\begin{eqnarray}
	z_1 = Z + \frac{Y}{2} \;\simeq\; \theta_1 t - \Delta(\theta_1 - \theta_2),\quad
	z_2 = Z - \frac{Y}{2} \;\simeq\; \theta_2 t + \Delta(\theta_1 - \theta_2),
\end{eqnarray}
%où \(\Delta(\theta_1 - \theta_2)\) représente le décalage dû à l’interaction entre les deux particules.
%On rappelle les définitions :
%\[
%	\tilde{\theta} = \theta_1 - \theta_2, 
%	\quad
%	\overline{\theta} = \frac{\theta_1 + \theta_2}{2}.
%\]
On peut vérifier les identités utiles suivantes :
\begin{eqnarray*}
	z_1 \theta_1 + z_2 \theta_2 = 2Z \overline{\theta} + \frac{1}{2} Y \tilde{\theta}, \quad
	z_1 \theta_2 + z_2 \theta_1 &=& 2Z \overline{\theta} - \frac{1}{2} Y \tilde{\theta},
\end{eqnarray*}
ce qui est en accord avec les masses associées : masse totale \(\overline{m} = 2\), masse réduite \(\tilde{m} = \frac{1}{2}\).

Cela nous motive à multiplier l’ansatz dans le référentiel du centre de masse (équation~\eqref{eq:ansatz.boson}) par un facteur de phase globale \(\exp(2iZ\overline{\theta})\) pour revenir à la représentation dans le laboratoire. On obtient alors l’expression de la fonction d’onde :
\begin{eqnarray}\label{eq:I-1-10}
	\varphi_{\{\theta_1 , \theta_2\}}(z_1 , z_2) & \propto &  \left \{ \begin{array} { c cl} ( \theta_2 - \theta_1 - ig) e^{ i z_1 \theta_1 + iz_2 \theta_2 } - ( \theta_1 - \theta_2 - ig) e^{ i z_1 \theta_2 + iz_2 \theta_1} & \mbox{si} & z_1 < z_2 \\ (z_1 \leftrightarrow z_2) & \mbox{si} & z_1 > z_2 \end{array} \right.
\end{eqnarray}

%Cette fonction d’onde correspond à une valeur propre d’énergie donnée par la somme des énergies associées aux deux degrés de liberté :

%\begin{equation}
%	\varepsilon(\theta_1 , \theta_2) 
%	= \underbrace{\overline{\theta}^2}_{\overline{\varepsilon}(\overline{\theta})}
%	+ \underbrace{\frac{1}{4} \tilde{\theta}^2}_{\tilde{\varepsilon}(\tilde{\theta})}
%	= \frac{\theta_1^2}{2} + \frac{\theta_2^2}{2}.
%\end{equation}

Pour \(\theta_1 > \theta_2\), les deux termes exponentiels 
\(e^{i z_1 \theta_1 + i z_2 \theta_2}\) et \(e^{i z_1 \theta_2 + i z_2 \theta_1}\)
correspondent respectivement aux ondes entrantes et sortantes dans le canal de diffusion à deux corps.  
Le rapport de leurs amplitudes définit la phase de diffusion / matrice diffusion $e^{i\Phi ( \tilde{\theta}  ) }$  à deux corps \eqref{chap:1:dif.mod.2.part.3} , reste inchangé :

\begin{equation}\label{chap:1:dif.mod.2.part.4}
	S(\theta_1- \theta_2) \doteq e^{i\Phi(\theta_1 - \theta_2)} 
	= \frac{\theta_1 - \theta_2 - ig}{\theta_2 - \theta_1 - ig}.
\end{equation}

Cette phase caractérise entièrement le processus de diffusion dans le modèle de Lieb-Liniger à deux particules.

\paragraph{Conditions périodiques et équations de Bethe pour deux bosons {\color{red}(à révoir)}.}

%La fonction d’onde obtenue par Bethe ansatz (voir
%\eqref{eq:I-1-10}) est, pour $z_{1}<z_{2}$,
%\[
%	\varphi_{\{\theta_{1},\theta_{2}\}}(z_{1},z_{2})
%		= a\,e^{i\theta_{1}z_{1}+i\theta_{2}z_{2}}
%		+b\,e^{i\theta_{2}z_{1}+i\theta_{1}z_{2}},
%	\quad
%	a=\theta_{2}-\theta_{1}-ic,\;
%	b=-(\theta_{1}-\theta_{2}-ic).
%\]

%\medskip
%\subparagraph{Périodicité sur $z_{2}$.}  
%On impose à la fonction d’onde obtenue par Bethe ansatz (voir
%\eqref{eq:I-1-10})
%\(
%	\varphi_{\{\theta_{1},\theta_{2}\}}(z_{1},z_{2}\!=\!L)
%	=
%	\varphi_{\{\theta_{1},\theta_{2}\}}(z_{1},z_{2}\!=\!0)
%\)
%avec $0<z_{1}<z_{2}=L$.  
%Au point $z_{2}=L$ on reste dans le secteur $z_{1}<z_{2}$, tandis qu’au point $z_{2}=0$ le domaine pertinent devient $z_{2}<z_{1}$;  la fonction d’onde y est obtenue en échangeant $z_{1}\leftrightarrow z_{2}$ , soit 
%\(
%	\varphi_{\{\theta_{1},\theta_{2}\}}(z_{1},\!L)
%	=
%	\varphi_{\{\theta_{1},\theta_{2}\}}(0 , z_{1})
%\)
%.  
%On obtient ainsi
%\begin{eqnarray*}
%	a\,e^{i\theta_{1}z_{1}+i\theta_{2}L}+b\,e^{i\theta_{2}z_{1}+i\theta_{1}L} & = &
%	a\,e^{i\theta_{2}z_{1}}\,e^{i\theta_{1}\! \cdot 0} + b \,e^{i\theta_{1}z_{1}}\,e^{i\theta_{2}\! \cdot 0},	
%\end{eqnarray*}
%avec la condition $z_1< z_2$, avec le rapport $a$ et $b$ vérifiant \eqref{chap:1:dif.mod.2.part.4} de la sorte $-b/a = e^{i\Phi(\theta_1 - \theta_2)}$ .

%%%%%%%%%%%%%%%%

\subparagraph{Périodicité en \( z_2 \).}  
On impose une condition de périodicité sur la fonction d’onde obtenue par ansatz de Bethe (voir équation~\eqref{eq:I-1-10}) :
\(
	\varphi_{\{\theta_1,\theta_2\}}(z_1, z_2 = L) = \varphi_{\{\theta_1,\theta_2\}}(z_1, z_2 = 0),
\)
avec \( 0 < z_1 < z_2 = L \).  
Au point \( z_2 = L \), la configuration reste dans le secteur \( z_1 < z_2 \), tandis qu’à \( z_2 = 0 \), on entre dans le secteur \( z_2 < z_1 \). La continuité de la fonction d’onde impose alors d’échanger les coordonnées \( z_1 \leftrightarrow z_2 \) :
\(
	\varphi_{\{\theta_1,\theta_2\}}(z_1, L) = \varphi_{\{\theta_1,\theta_2\}}(0, z_1).
\)
En utilisant l’expression explicite de l’ansatz dans les deux secteurs, on obtient l’égalité suivante :
\begin{eqnarray*}
	a\,e^{i\theta_1 z_1 + i\theta_2 L} + b\,e^{i\theta_2 z_1 + i\theta_1 L}
	&=& a\,e^{i\theta_2 z_1} + b\,e^{i\theta_1 z_1}.
\end{eqnarray*}
%où le second membre correspond à la fonction d’onde dans le secteur \( z_2 < z_1 \), évaluée en \( z_2 = 0 \) et \( z_1 = z_1 \).  
%La condition de périodicité impose donc :
%\[
%	a\,e^{i\theta_1 z_1 + i\theta_2 L} + b\,e^{i\theta_2 z_1 + i\theta_1 L}
%	= a\,e^{i\theta_2 z_1} + b\,e^{i\theta_1 z_1}.
%\]
Cette relation, valable pour tout \( z_1 \in (0,L) \), fixe une contrainte sur le rapport \( b/a \). En utilisant l’expression de la phase de diffusion introduite en \eqref{chap:1:dif.mod.2.part.4} pour $z_1<z_2$ :
\begin{eqnarray*}
	-\frac{b}{a} = e^{i\Phi(\theta_1 - \theta_2)},
\end{eqnarray*}
on obtient une condition quantique sur les phases \( \theta_1 \) et \( \theta_2 \), cœur de la quantification imposée par le formalisme de Bethe.

%\[
%	( \theta_2 - \theta_1 - ig)\,e^{i\theta_{1}z_{1}+i\theta_{2}L}
%	- ( \theta_1 - \theta_2 - ig)\,e^{i\theta_{2}z_{1}+i\theta_{1}L}
%	=
%	( \theta_2 - \theta_1 - ig)\,e^{i\theta_{2}z_{1}}\,e^{i\theta_{1}\! \cdot 0}
%	- ( \theta_1 - \theta_2 - ig)\,e^{i\theta_{1}z_{1}}\,e^{i\theta_{2}\! \cdot 0}.
%\]
En identifiant les coefficients de $e^{i\theta_{1}z_{1}}$ et
$e^{i\theta_{2}z_{1}}$ indépendamment, on obtient
\(
	e^{i\theta_{2}L}\;a = b, 
	\,
	e^{i\theta_{1}L}\;b = a,
\)
c’est‑à‑dire l'équations de Bethe
%\begin{equation}\label{eq:PC2}
%	e^{i\theta_{2}L} = \frac{b}{a}
%	= \frac{\theta_{1}-\theta_{2}+ic}{\theta_{2}-\theta_{1}+ic},
%\quad
%	e^{i\theta_{1}L} = \frac{a}{b}
%	= \frac{\theta_{2}-\theta_{1}+ic}{\theta_{1}-\theta_{2}+ic}.
%\end{equation}
\begin{eqnarray*}\label{eq:PC2}
	e^{i\theta_{1}L}\,e^{i\Phi(\theta_{1}-\theta_{2})} = -1,
	\qquad
	e^{i\theta_{2}L}\,e^{i\Phi(\theta_{2}-\theta_{1})} = -1.	
\end{eqnarray*}
En prenant le logarithme on obtient les \emph{équations de Bethe à deux
particules} :
\begin{equation}\label{eq:Bethe2}
	\theta_{1}L + \Phi(\theta_{1}-\theta_{2}) = 2\pi I_{1}, 
	\qquad
	\theta_{2}L + \Phi(\theta_{2}-\theta_{1}) = 2\pi I_{2},
\end{equation}
où $I_{1},I_{2}\in\mathbb{Z}$ sont les nombres quantiques entiers
(caractère bosonique). 

\subparagraph{Périodicité sur $z_{1}$.}  Le raisonnement symétrique conduit exactement aux mêmes égalités \eqref{eq:PC2}.  
\bigskip
Les équations \eqref{eq:Bethe2} constituent la quantification complète
du gaz de Lieb–Liniger à deux bosons sur un cercle de longueur $L$ et
seront le point de départ pour l’étude de l’état fondamental et des
excitations.



\begin{figure}[H]
	\centering
  %\includegraphics[width=0.5\textwidth]{}
  %\caption{Gauche : La fonction d'onde (\ref{eq:I-1-10}) sur la ligne infinie correspond à un processus de diffusion à deux corps. Semiclassiquement, la phase de diffusion dans ce processus à deux corps se reflète dans le décalage de diffusion (\ref{eq:I-1-16}) : après la collision, la position de la particule a été déplacée d'une distance $\Delta ( \theta_1 - \theta_2 )$ . Droite : La fonction d'onde de Bethe (\ref{eq:I-2-17}) sur la ligne infinie correspond à un processus de diffusion à $N$-corps qui se factorise en des processus à deux corps (le décalage de diffusion $\Delta$ est également présent ici, mais il n'est pas représenté dans la caricature). Dans ce processus à $N$-corps, les rapidités $\theta_j$ sont les moments asymptotiques des bosons.}
  \label{}	
\end{figure}



\section{Équation de Bethe et distribution de rapidité}

\subsection{Fonction d’onde dans le secteur ordonné et représentation de Gaudin}

Dans le domaine $z_1 < z_2 < \cdots < z_N$, la fonction d’onde pour un état de Bethe à $N$ particules s’écrit ({\color{blue}Gaudin 2014}, {\color{blue}Korepin et al. 1997}, {\color{black}Lieb et Liniger 1963}) :
\begin{eqnarray}
	\varphi_{\{\theta_a\}} ( z_1 , \cdots , z_N ) & = &  \frac{1}{\sqrt{N!}}\langle \emptyset \vert \operator{\Psi} ( z_1 ) \cdots \operator{\Psi} (z_N ) \vert \{ \theta_a \} \rangle \notag\\
	& \propto & \sum_\sigma (-1)^{|\sigma|} \left( \prod_{1 \leq a < b \leq N} (\theta_{\sigma(b)} - \theta_{\sigma(a)} - i g) \right) e^{i \sum_{j=1}^{N} z_j \theta_{\sigma(j)}},\label{eq:I-2-17}
\end{eqnarray}
où la somme s'étend sur toutes les permutations $\sigma$ de $\{1,\dots,N\}$. Le facteur $(-1)^{|\sigma|}$ est la signature de la permutation, et les amplitudes dépendent des différences de quasi-moments $\theta_j$ ainsi que du couplage $c$.
Cette fonction d’onde est ensuite étendue par symétrie aux autres domaines du type $z_{\pi(1)} < z_{\pi(2)} < \cdots < z_{\pi(N)}$ via des propriétés d’échange symétriques.

\vspace{1em}

\subsection{Conditions aux bords périodiques}

Les équations précédentes ont été établies pour un système défini sur la droite réelle. Cependant, dans une perspective thermodynamique, il est essentiel de considérer une densité finie $ N/L$. Cela peut être obtenu en compactifiant l’espace sur un cercle de longueur $L$, i.e. en imposant les {\em conditions aux bords périodiques}.

Concrètement, cela consiste à identifier $x = 0$ et $x = L$ et à exiger que la fonction d’onde soit périodique lorsqu’une particule fait le tour du système :
\begin{equation}\label{eq:periodic}
\varphi_{\{\theta_a\}}(x_1, \dots, x_{N-1}, L) = \varphi_{\{\theta_a\}}(0, x_1, \dots, x_{N-1}).
\end{equation}
Cette condition doit être satisfaite pour chaque particule. Or, déplacer la $j$-ième particule de $x_j$ à $x_j + L$ revient à la faire passer devant toutes les autres : cela introduit un facteur de diffusion à chaque croisement.

%\vspace{1em}

\subsection{Équations de Bethe exponentielles}

En imposant les conditions de périodicité sur la fonction d’onde de type Bethe~\eqref{eq:I-2-17}, on obtient que chaque moment $\theta_a$ doit satisfaire l’équation :
\begin{equation}
	e^{i \theta_a L} \prod_{b \ne a} S(\theta_a - \theta_b) = (-1)^{N-1}, \quad a = 1, \dots, N,
	\label{eq:bethe_exp}
\end{equation}
où la matrice diffusion $S(\theta) = \frac{\theta - i g}{-\theta - i g} = e^{i \Phi(\theta)}$ est l’amplitude de diffusion à deux corps, et $\Phi(\theta) = 2 \arctan\left( \frac{\theta}{c} \right)$ est la phase associée~\eqref{chap:1:eq:Phi}. Le signe $(-1)^{N-1}$ vient du fait que chaque permutation change la signature du déterminant dans la représentation de Gaudin.
%\vspace{1em}

\subsection{Équations de Bethe logarithmiques}

En prenant le logarithme du membre gauche et du membre droit de l’équation~\eqref{eq:bethe_exp}, on obtient :
\begin{equation}\label{chap:1:eq:EBA}
	L \theta_a + \sum_{b=1}^N \Phi(\theta_a - \theta_b) = 2\pi I_a, \qquad a = 1, \dots, N,
\end{equation}
où les $I_a \in \mathbb{Z}$ (ou $\mathbb{Z} + \tfrac{1}{2}$) sont des nombres quantiques entiers (ou demis entiers) . Dans la configuration d’état fondamental (ou de type “mer de Fermi”), ces nombres sont pris de manière symétrique autour de zéro :
\[
I_a = a - \frac{N+1}{2}, \quad \text{pour } a \in \llbracket 1 , N \rrbracket.
\]
Ce choix garantit une distribution uniforme des $\theta_a$ à l’état fondamental.
%\vspace{1em}

\subsection{Interprétation physique}

Les équations de Bethe~\eqref{chap:1:eq:EBA} représentent une {\em quantification des pseudo‑impulsions $\theta_a$} des particules en interaction, résultant d’un {\em interféromètre multi‑corps sur le cercle} : chaque particule accumule une phase $e^{i \theta_a L}$ due au mouvement libre, ainsi que des phases de diffusion lorsqu’elle croise les autres.

Ce système d'équations détermine les états propres du système de Lieb–Liniger en volume fini, et joue un rôle fondamental dans la description exacte de ses propriétés thermodynamiques et dynamiques.


\subsection{Thermodynamique du gaz de Lieb–Liniger à température nulle}

Dans la limite thermodynamique, le nombre de particules \( N \) et la longueur \( L \) du système tendent vers l'infini de telle sorte que leur rapport reste fini :
\begin{eqnarray*}
	\lim_{N,\, L \to \infty} \frac{N}{L} = D < \infty,
\end{eqnarray*}
où \( D \) désigne la densité linéique de particules.

Considérons désormais le système à température nulle. L’état fondamental dans le secteur à nombre de particules fixé correspond à la configuration d’énergie minimale parmi les solutions des équations de Bethe \eqref{chap:1:eq:EBA}.

Dans la limite thermodynamique, les valeurs de \( \theta_a \) deviennent quasi-continues, avec un espacement \( \theta_{a+1} - \theta_a = \mathcal{O}(1/L) \), et se condensent dans un intervalle symétrique autour de zéro :
\[
\theta_a \in [-K, K],
\]
où \( K \) est le paramètre de Fermi (ou rapidité maximale), défini par \( K = \theta_N \). En supposant l'ordre \( I_a \geq I_b \Rightarrow \theta_a \geq \theta_b \), cet intervalle constitue ce qu'on appelle la {\em mer de Dirac} (ou sphère de Fermi en dimension un).

Nous introduisons la densité d’états \( \rho_s(\theta) \), définie par
\begin{eqnarray*}
	2\pi \rho_s(\theta_a) &=& \frac{2\pi}{L} \lim_{\text{therm}} \frac{|I_{a+1} - I_a|}{|\theta_{a+1} - \theta_a|} = \frac{2\pi}{L} \frac{\partial I}{\partial \theta}(\theta_a),
\end{eqnarray*}
où \( I(\theta_a) = I_a \). L’application des équations de Bethe sous forme logarithmique conduit alors à
\begin{eqnarray*}
	2\pi \rho_s(\theta_a) = 1 + \frac{1}{L} \sum_{b = 1}^N \Delta(\theta_a - \theta_b),
\end{eqnarray*}
ce qui relie \( \rho_s \) à la fonction d’interaction \( \Delta \) entre les rapidités.

Intéressons-nous maintenant à la {\em densité de particules dans l’espace des moments}, notée \( \rho(\theta) \), définie par
\begin{eqnarray*}
	\rho(\theta_a) = \lim_{L \to \infty} \frac{1}{L} \cdot \frac{1}{\theta_{a+1} - \theta_a} > 0.
\end{eqnarray*}
Dans l’état fondamental, toutes les positions disponibles dans l’intervalle \( [-K, K] \) sont occupées. On a donc :
\begin{eqnarray}\label{chap.1.rho.2}
	\rho(\theta) = \rho_s(\theta).
\end{eqnarray}

La quantité \( L \rho(\theta) d\theta \) représente le nombre de rapidités dans la cellule infinitésimale \( [\theta, \theta + d\theta] \), tandis que
\(
	N = L \int_{-K}^{K} \rho(\theta)\, d\theta
\)
donne le nombre total de particules dans le système. Le passage de la somme discrète à l'intégrale dans le second membre de l'équation de Bethe permet d’écrire :
\begin{eqnarray*}
	\frac{1}{L} \sum_{b = 1}^N \Delta(\theta_a - \theta_b) \longrightarrow \int_{-K}^{K} \Delta(\theta_a - \theta)\, \rho(\theta)\, d\theta.
\end{eqnarray*}
Ainsi, l'équation pour la densité d'états devient :
\begin{eqnarray}\label{chap.1.rho.s.2}
	2\pi \rho_s(\theta) = 1 + \int_{-K}^{K} \Delta(\theta - \theta')\, \rho(\theta')\, d\theta',
\end{eqnarray}
et, comme \( \rho = \rho_s \), on obtient l’équation linéaire intégrale satisfaite par la densité de rapidités :
\begin{eqnarray}\label{chap.1.rho.3}
	\rho(\theta) - \int_{-K}^{K} \frac{\Delta(\theta - \theta')}{2\pi} \rho(\theta')\, d\theta' = \frac{1}{2\pi}.
\end{eqnarray}


\subsection{Excitations élémentaires à température nulle}




\chapter{Relaxation et Équilibre dans les Systèmes Quantiques Intégrables : Une Approche par la Thermodynamique de Bethe}\label{chap:relaxation}
\minitoc

%------------------------------------------------------------------
\section*{Introduction générale}

Dans les modèles quantiques intégrables, l’évolution vers l’équilibre, à partir d’un état initial arbitraire (et typiquement hors d’équilibre), ne conduit pas à une thermique de Gibbs classique.  
En effet, du fait de l’existence d’une infinité de charges conservées en involution, les systèmes intégrables n’explorent qu’une sous-partie contrainte de l’espace des états accessibles.  
Ils relaxent alors vers un état stationnaire décrit par une \emph{Ensemble Thermodynamique Généralisé} (GGE), qui encode la conservation de toutes ces quantités.

Cette section pose les fondations nécessaires à la description de ces états stationnaires dans le cadre de la \textbf{thermodynamique de Bethe} (TBA), qui généralise l’analyse au-delà de l’état fondamental.  
Nous considérons ici un régime macroscopique à température (ou entropie) finie, correspondant à des états hautement excités du spectre, mais toujours décrits dans le formalisme intégrable exact.

Notre point de départ est la relation constitutive entre la \emph{densité de quasi-particules} (ou \emph{rapidités}) $\rho(\theta)$ et la \emph{densité d’états} disponibles $\rho_s(\theta)$, qui encode le spectre accessible en présence d’interactions.  
Nous introduisons ensuite une opération clé de la TBA, appelée \emph{habillage} (\emph{dressing}), qui intervient systématiquement dans le calcul des observables physiques et permet de prendre en compte de manière non perturbative les effets des interactions.  
Cette construction sera illustrée dans le cadre du modèle intégrable de Lieb–Liniger, qui décrit un gaz unidimensionnel de bosons avec interaction delta répulsive.

Les outils développés ici seront fondamentaux pour formuler dans la section suivante le concept d’ensemble généralisé (GGE), et pour décrire la dynamique de relaxation des systèmes intégrables.



\section{Notion d’état d’équilibre généralisé (GGE)}

\paragraph{Introduction.}


\paragraph{Configuration des états.}\label{sec:config-etats}.
On désigne par $\boldsymbol{\{ \theta_a \}}\equiv \{ \theta_1 , \cdots , \theta_{N} \}$ la \emph{configuration de rapidités} caractérisant un état propre à $N\!\equiv\!N(\{ \theta_a \})$ particules – le nombre de particules n’est donc pas fixé \emph{a priori} mais dépend de la configuration.  
L’état propre correspondant est noté $\ket{\{ \theta_a \}}\;=\;\ket{\{\theta_1,\dots,\theta_N \}}$.

%%%%%%%%%%%%%%%%%%%%%%%%%%%%%%%%%%%%%%%%%%%%%%%%%%
\paragraph{Observables diagonales dans la base des états propres.}
Dans le chapitre précédent (\ref{chap:LL-BA}), on a vu que l'état $\ket{\{ \theta_a \}}$ associé à cette configuration est une état propre des observables nombre et quantité de mouvement  et  énergie cinétique \eqref{chap1:eq.Q.P.K.theta.1}. Ces observables sont diagonales dans la base des états propres :
\begin{eqnarray}
	\operator{Q}  =  \sum_{ \{\theta_a\} } \left ( \sum_{a = 1}^{N}  1 \right )  \vert \{ \theta_a\}\rangle	\langle \{ \theta_a \}\vert, \, 
	\operator{P}  =  \sum_{\{ \theta_a\}}\left( \sum_{a = 1}^{N}  \theta_a \right )   \vert \{ \theta_a\}\rangle	\langle \{ \theta_a \}\vert,\,\operator{K}  =  \sum_{\{ \theta_a\}}\left ( \sum_{a = 1}^{N} \frac{\theta_a^2}{2} \right )   \vert \{ \theta_a\}\rangle	\langle \{ \theta_a \}\vert.\label{chap.2.gge.1}		
\end{eqnarray}
avec $ \sum_{\{ \theta_a\}}$ une somme sur tous les configurations.\\
%\begin{eqnarray}
%	\operator{Q} \ket{\{ \theta_a\}}  =  \sum_{ \{\theta_a\} } \left ( \sum_{a = 1}^{N}  1 \right ) \ket{\{ \theta_a\}}, \, 
%	\operator{P} \ket{\{ \theta_a\}}  =  \sum_{\{ \theta_a\}}\left( \sum_{a = 1}^{N}  \theta_a \right ) \ket{\{ \theta_a\}},\,\operator{H} \ket{\{ \theta_a\}}  =  \sum_{\{ \theta_a\}}\left ( \sum_{a = 1}^{N} \frac{\theta_a^2}{2} \right )   \ket{\{ \theta_a\}}.		
%\end{eqnarray}

Nous avons introduit ces observables en injectant des opérateurs $\operator{f}$ proportionnels à des puissances de la quantité de mouvement d’une particule $\operator{p}$, respectivement $\propto \operator{p}^0$, $\propto \operator{p}^1$ et $\propto \operator{p}^2$, dans l’opérateur à un corps $\operator{F}$ défini dans l’équation \eqref{chap.1:eq.rapel.opp.1.second.2}. Écrit de cette manière, nous avons vu dans l’équation \eqref{chap.1:eq.rapel.opp.1.second.3} que pour $\operator{f} = \operator{p}^q$ avec $q$ entier, l’état de Bethe $\ket{\{ \theta_a \} }$ est un état propre de $\operator{F}$ :
\begin{eqnarray}\label{chap.2:eq.rapel.opp.1.second.1}
	 \operator{F} \ket{\{\theta_a\}} =   \sum_{ \{\theta_a\} }\left( \sum_{a = 1}^N \theta_a^q \right) \ket{\{\theta_a\}},
\end{eqnarray}
avec des valeurs propres données par des puissances de $\theta$. Cela motive l’étude d’états d’équilibre statistique au-delà de l’équilibre thermique, c’est-à-dire au-delà de l’ensemble de Gibbs.
   




%%%%%%%%%%%%%%%%%%%%%%%%%%%%%%%%%%%%%%%%%%%%
\paragraph{Contexte et GGE dans les systèmes intégrables.}

Dans un système quantique {\bf intégrable}, il existe une infinité de charges conservées locales $\operator{Q}_i$ commutant entre elles et avec l’Hamiltonien $\operator{H}$ ([Rigol et al. 2007] ) \cite{??}. Concrètement, chaque charge se présente sous la forme $\operator{Q}_i = \int dx \,\operator{q}_i(x)$, où $\operator{q}_i(x)$ est une densité d’observable locale à support borné. L’intégrabilité implique ainsi une caractérisation complète des états propres par un ensemble de paramètres (rapidités $\{\theta_j\}$ dans le modèle de Lieb-Liniger) \cite{??}. En particulier, contrairement aux systèmes génériques, un système intégrable ne thermalise pas au sens canonique classique, car la présence de toutes ces contraintes empêche l’oubli complet des conditions initiales. Les points clés sont alors :

\begin{itemize}[label = $\bullet$]
	\item {\bf Charges conservées} : infinité de locales $\operator{Q}_i$ satisfaisant et $[\operator{Q}_i , \operator{H} ] = 0$ et $[\operator{Q}_i , \operator{Q}_j ] = 0$.
	\item {\bf Densités locales} : chaque $\operator{Q}_i$ s’écrit $\operator{Q}_i = \int_\mathbb{R} dx \, \operator{q}_i(x)$ avec $\operator{q}_i(x)$ à support fini.
	\item {\bf Relaxation non canonique} : après un {\em quench} (changement brutal de paramètre), le système évolue vers un état stationnaire qui n’est pas décrit par l’ensemble canonique habituel.
\end{itemize}

Pour décrire cet état, on introduit l’{\bf ensemble de Gibbs généralisé (GGE)}. Rigol et al. ont montré qu’une « extension naturelle de l’ensemble de Gibbs aux systèmes intégrables » prédit correctement les valeurs moyennes des observables après relaxation \cite{??}.  Formellement, pour une région finie du système $\mathcal{S} \subset \mathbb{R}$, on définit la matrice densité locale :
\begin{eqnarray}
	\operator{\rho}^{(\mathcal{S})}_{\mathrm{GGE}} = \frac{1}{Z^{(\mathcal{S})}}\exp \left ( - \sum_i \beta_i \operator{Q}_i^{(\mathcal{S})} \right), \quad \operator{Q}_i^{(\mathcal{S})} = \int_\mathcal{S} dx \, \operator{q}_i(x), \label{chap.TBA.op.rho.S}	
\end{eqnarray}

où $\beta_i \in \mathbb{R}$ sont les multiplicateurs de Lagrange (ou « températures généralisées ») associés aux charges locales conservées $\{\operator{Q}_i\}$. La fonction de partition 
\begin{eqnarray}
	Z^{(\mathcal{S})} = \bm{\mathrm{Tr}}\left [\exp \left( - \sum_i \beta_i \operator{Q}_i^{(\mathcal{S})} \right ) \right ]  \label{chap.TBA.op.Z.S}	
\end{eqnarray}
 assure la normalisation. L’{\bf état GGE} ainsi défini est le seul permettant de prédire de manière cohérente les observables locales de $\mathcal{S}$ à long temps \cite{??}. Autrement dit, l’équilibre local après quench est un état stationnaire faisant perdurer la mémoire de chaque charge conservée, ce qui conduit à un nombre macroscopique de paramètres $\beta_i$ thermodynamiques (une « température » par charge) \cite{??}.

 \subparagraph{Interprétation des multiplicateurs de Lagrange.}
Les multiplicateurs de Lagranges $\beta_i$ apparaissent naturellement lors de l'optimisation sous contraintes, par exemple dans le formalisme de l'{\bf ensemble de Gibbs généralisé (GGE)}, oû il imposent la conservation des valeurs moyennes des charges $\langle \operator{Q}_i^{(\mathcal{S})} \rangle_{\operator{\rho}^{(\mathcal{S})}_{\mathrm{GGE}}} = \bm{\mathrm{Tr}}[\operator{\rho}^{(\mathcal{S})}_{\mathrm{GGE}} \operator{Q}_i^{(\mathcal{S})}]   $.\\

En résumé, la GGE généralise les ensembles canoniques standard : au lieu de retenir uniquement l’énergie, on impose la conservation de l’ensemble complet $\{\operator{Q}_i \}$. Cette construction rend compte du fait que, dans un système intégrable, les observables locaux convergent vers les valeurs moyennes de $\operator{\rho}^{(\mathcal{S})}_{\mathrm{GGE}}$ , et non vers celles d’un Gibbs thermique ordinaire \cite{??}\cite{??}. On comprend ainsi pourquoi la {\em thermalisation habituelle} (canonique ou microcanonique) échoue : seul l’ensemble de Gibbs généralisé peut intégrer toutes les contraintes locales.

\paragraph{Rappel sur le modèle de Lieb-Liniger et distribution de rapidités.}
Comme rappelé au chapitre précédent, {\bf le modèle de  Lieb-Liniger} (gaz bosonique 1D à interactions de contact) est un exemple paradigmatique d’un système intégrable \cite{??}. Ses états propres sont caractérisés par un ensemble de $N$  rapidités $\{ \theta_a \}$ , qui jouent le rôle de quasi-momenta ({\bf Bethe ansatz}). Dans ce contexte, l’état macroscopique du gaz après relaxation unitaire est entièrement déterminé par la {\bf distribution des rapidités}. Formellement, on définit $\rho(\theta)$ la distribution intensive des rapidités telle que $\rho(\theta) d \theta$ donne la fraction de particules par unité de longueur ayant une rapidité dans la cellule $[\theta , \theta + d \theta ] $.\\

Cette « distribution de rapidités » est d’autant plus pertinente qu’elle est {\em accessible expérimentalement}. En effet, lorsque le gaz bosonique 1D est libéré et laissé s’étendre, la distribution asymptotique des vitesses des atomes coïncide avec la distribution initiale des rapidités \cite{??} . Autrement dit, la GGE prédit un profil de vitesses observables en laboratoire. Léa Dubois souligne dans sa thèse que " la distribution de rapidités est la distribution asymptotique des vitesses des atomes après une expansion dans le guide 1D ", et qu’elle peut être extraite par l’hydrodynamique généralisée \cite{??}. \\

Dans la GGE, cette distribution macroscopique $\rho(\theta)$ est fixée par l’ensemble des charges conservées. Par exemple, on ajuste les $\beta_i$ de sorte que les valeurs moyennes $\langle \operator{Q}_i \rangle_{\operator{\rho}^{(\mathcal{S})}_{\mathrm{GGE}}}$ correspondent aux valeurs initiales. Ce processus détermine donc la fonction $\rho(\theta)$ décrivant l’état d’équilibre local. Les observables locaux du gaz (densité, corrélations, etc.) en découlent alors via les équations de Bethe ansatz. 


\paragraph{Convention pour les moyennes d'observables.}
Dans la suite du chapitre, nous noterons la moyenne d’une observable $\operator{\mathcal{O}}$ dans un état décrit par une matrice densité (ici noté) $\operator{\rho}$ par :
\begin{eqnarray}\label{chap.TBA.moy.dens}	
	\braket{\operator{\mathcal{O}}}_{\operator{\rho}} \doteq \bm{\mathrm{Tr}}[\operator{\rho} \, \operator{\mathcal{O}}],
\end{eqnarray}
En particulier, si la matrice densité est un projecteur, comme $\ket{\{\theta_a \}}\!\bra{\{\theta_a \}}$, $\bm{\mathrm{Tr}}[\ket{\{\theta_a \}}\!\bra{\{\theta_a \}} \operator{\mathcal{O}}] =  \bra{\{\theta_a \}}\operator{\mathcal{O}}\ket{\{\theta_a \}}$. dans ce cas on notera la moyenne :
\begin{eqnarray}\label{chap.TBA.moy.dens.pur}
	\braket{\operator{\mathcal{O}}}_{\{\theta_a \}} = \bra{\{\theta_a \}} \operator{\mathcal{O}} \ket{\{\theta_a \}},
\end{eqnarray}
où l’on note simplement l’ensemble des rapidité ${\theta_a}$ pour désigner l’état pur.

%%%%%%%%%%%%%%%%%%%%%%%%%%%%%%%%%%%%%%%%%%%%%%%%%%
\paragraph{Charges conservées locales diagonales dans la base des états propres.}
Les charges conservées locales $\operator{Q}_i^{(\mathcal{S})}$ est diagonale dans la base des  états propres $\ket{ \{ \theta_a \}}$ , avec pour valeurs propres $\langle \operator{Q}_i^{(\mathcal{S})} \rangle_{\{\theta_a \}} $ 	 :
%\begin{eqnarray}
%	\operator{Q}_i^{(\mathcal{S})} & = & \sum_{ \{\theta_a\} } \langle \operator{Q}_i^{(\mathcal{S})} \rangle_{\{\theta_a \}}  \ket{\{\theta_a \}}\!\bra{\{\theta_a \}}.		
%\end{eqnarray}
\begin{eqnarray}\label{chap.TBA.Qi.diag}
	\operator{Q}_i^{(\mathcal{S})}\ket{\{\theta_a \}} & = &  \langle \operator{Q}_i^{(\mathcal{S})} \rangle_{\{\theta_a \}}  \ket{\{\theta_a \}}.		
\end{eqnarray}
%%%%%%%%%%%%%%%%%%%%%%%%%%%%%%%%%%%%%%%%
\paragraph{Probabilité d’un état à rapidités fixées.}
On peut alors définir la probabilité d’occurrence d’un état $\ket{\{ \theta_a \} }$ comme la moyenne de la matrice densité locale $\operator{\rho}^{(\mathcal{S})}_{\mathrm{GGE}}$ définie dans \eqref{chap.TBA.op.rho.S}:
\begin{eqnarray}
	\mathbb{P}^{(\mathcal{S})}_{\{ \theta_a \}}  & \equiv &  \langle \operator{\rho}^{(\mathcal{S})}_{\mathrm{GGE}} \rangle_{\{\theta_a \}}, \label{chap.TBA.P.1}\\
	& = & 
	\frac{1}{Z^{(\mathcal{S})}} \exp \left (- \sum_i \beta_i \langle \operator{Q}_i^{(\mathcal{S})} \rangle_{\{\theta_a \}} \right ) \label{chap.TBA.P.2}.
\end{eqnarray}

%%%%%%%%%%%%%%%%%%%%%%%%%%%
\paragraph{Moyenne d’un charges conservées locales et dérivées de $Z^{(\mathcal{S})}$.} Les charges locales $\operator{Q}_i^{(\mathcal{S})}$ sont diagonale dans la bases \( \{ \ket{\{\theta_a \}} \}  \) [cf eq~ ~\eqref{chap.TBA.Qi.diag}]. 
On peut donc  écrire la moyenne d’une observable comme une somme pondérée par cette probabilité [cf eqs ~\eqref{chap.TBA.P.1}-\eqref{chap.TBA.P.2}] , ou encore comme une dérivée de la fonction de partition définie dans l'équation \eqref{chap.TBA.op.Z.S} :
\begin{eqnarray}
	\langle \operator{Q}_i^{(\mathcal{S})} \rangle_{\operator{\rho}^{(\mathcal{S})}_{\mathrm{GGE}}} &= & \sum_{\{ \theta_a\}} \langle \operator{Q}_i^{(\mathcal{S})} \rangle_{\{\theta_a \}} \mathbb{P}^{(\mathcal{S})}_{\{ \theta_a \}} \label{chap.TBA.moy.1}\\
	 & = &  \left. \frac{1}{Z^{(\mathcal{S})}} \frac{\partial Z^{(\mathcal{S})}}{\partial \beta_i} \right )_{\beta_{j \neq i }}	 \label{chap.TBA.moy.2}
\end{eqnarray}

Par le même raisonnement le moment non centré s'écrit :
\begin{eqnarray}
	\braket{ \operator{Q}_{i_1}^{(\mathcal{S})} \, \operator{Q}_{i_2}^{(\mathcal{S})} \cdots \operator{Q}_{i_q}^{(\mathcal{S})} }_{\operator{\rho}^{(\mathcal{S})}_{\mathrm{GGE}}} &= &  (-1)^q \frac{1}{Z^{(\mathcal{S})}} \left.\frac{\partial}{\partial \beta_{i_1}} \right )_{\beta_{j \neq i_1 }} \left.\frac{\partial}{\partial \beta_{i_2}} \right )_{\beta_{j \neq i_2 }} \cdots \left.\frac{\partial}{\partial \beta_{i_q}} \right )_{\beta_{j \neq i_q }} Z^{(\mathcal{S})} \label{chap.TBA.mom.1}.	
\end{eqnarray}

%%%%%%%%%%%%%%%%%%%%%%%%%%%%%%%
\paragraph{Moments d’ordre supérieur et fluctuations.} On s'avance sur le chapitre (\ref{chap:Fluctu}).
Le premier et second moments permettent d’accéder à la variance 
\begin{eqnarray}
	 \left \langle \left (\operator{Q}_i^{(\mathcal{S})} - \langle\operator{Q}_i^{(\mathcal{S})} \rangle_{\operator{\rho}^{(\mathcal{S})}_{\mathrm{GGE}}} \right )^2  \right \rangle_{\operator{\rho}^{(\mathcal{S})}_{\mathrm{GGE}}} = \langle(\operator{Q}_i^{(\mathcal{S})})^2 \rangle_{\operator{\rho}^{(\mathcal{S})}_{\mathrm{GGE}}}  -  \langle\operator{Q}_i^{(\mathcal{S})} \rangle_{\operator{\rho}^{(\mathcal{S})}_{\mathrm{GGE}}}^2	
\end{eqnarray}
de le charge locale $\operator{Q}_i^{(\mathcal{S})}$, en injectant \eqref{chap.TBA.moy.2} et \eqref{chap.TBA.mom.1} et en utilisant $\frac{1}{f} \partial_x^2 f - ( \frac{1}{f} \partial_x f ) = \partial_x^2 \ln f  $:
\begin{eqnarray}
	\left \langle \left (\operator{Q}_i^{(\mathcal{S})} - \langle\operator{Q}_i^{(\mathcal{S})} \rangle_{\operator{\rho}^{(\mathcal{S})}_{\mathrm{GGE}}} \right )^2  \right \rangle_{\operator{\rho}^{(\mathcal{S})}_{\mathrm{GGE}}}  &=&	  \left . \frac{\partial^2 \ln Z^{(\mathcal{S})}  }{{\partial \beta_i}^2 }  \right )_{\beta_{j\neq i}},\\
	& = &  - \left . 	\frac{\partial \langle\operator{Q}_i^{(\mathcal{S})} \rangle_{\operator{\rho}^{(\mathcal{S})}_{\mathrm{GGE}}} }{\partial \beta_i } \right )_{\beta_{j\neq i}}.	
\end{eqnarray}

%%%%%%%%%%%%%%%%%%%%%%%%%%%%%%
\paragraph{Cas particulier de l’équilibre thermique.}

Dans le cas particulier de l’équilibre thermique standard (\ie Gibbsien), le système est décrit par une seule contrainte d’énergie (ou d’énergie et de particule, dans le cas d’un grand canonique). Les multiplicateurs de Lagrange associés aux charges conservées peuvent alors être identifiés à des grandeurs thermodynamiques classiques.

\begin{itemize}[label=$\bullet$]
	\item Si la seule charge conservée est le nombre de particules $\operator{Q}_0^{(\mathcal{S})} = \operator{Q}$, le multiplicateur associé est $\beta_0 = -\beta \mu$, où $\mu$ est le potentiel chimique et $\beta = T^{-1}$ l’inverse de la température (avec $k_B = 1$).
	
	\item Si la charge conservée est $\operator{Q}_2^{(\mathcal{S})}-\mu\operator{Q}_0^{(\mathcal{S})}  = \operator{K} - \mu \operator{Q} $ (ensemble grand canonique), alors le multiplicateur est simplement $ \beta$.
\end{itemize}

Dans le cadre de l’équilibre thermique , les moyennes et les fluctuations thermodynamiques usuelles s’expriment naturellement comme dérivées du logarithme de la fonction de partition $Z^{(\mathcal{S})}$ :
\begin{eqnarray}
	\langle \operator{Q} \rangle_{\operator{\rho}^{(\mathcal{S})}_{\mathrm{GGE}}}  = \left .\frac{1}{\beta} \frac{ \partial \ln Z^{(\mathcal{S})}}{\partial \mu } \right )_{T},  & &  \left . \frac{1}{\beta} \frac{ \partial \langle \operator{Q} \rangle_{\operator{\rho}^{(\mathcal{S})}_{\mathrm{GGE}}}}{\partial \mu } \right )_{T} =  \left . \frac{1}{\beta^2} \frac{ \partial^2 \ln Z^{(\mathcal{S})}}{{\partial \mu}^2 } \right )_{T} \\
	\langle \operator{H} - \mu\operator{Q}  \rangle_{\operator{\rho}^{(\mathcal{S})}_{\mathrm{GGE}}}  = -\left . \frac{ \partial \ln Z^{(\mathcal{S})}}{\partial \beta } \right )_{\mu} ,  & & -\left .  \frac{ \partial \langle \operator{H} - \mu\operator{Q} \rangle_{\operator{\rho}^{(\mathcal{S})}_{\mathrm{GGE}}}}{\partial \beta } \right )_{\mu } = \left .  \frac{ \partial^2 \ln Z^{(\mathcal{S})}}{{\partial \beta}^2 } \right )_{\mu}   .		
\end{eqnarray}
En combinant ces relations, on peut également exprimer l’énergie moyenne et ses fluctuations comme :
\begin{eqnarray}
	\langle \operator{H} \rangle_{\operator{\rho}^{(\mathcal{S})}_{\mathrm{GGE}}}  = \left [ \left .\frac{\mu}{\beta} \frac{ \partial}{\partial \mu } \right )_{T} -\left . \frac{ \partial }{\partial \beta } \right )_{\mu}   \right ]\ln Z^{(\mathcal{S})},  \quad  -\left .  \frac{ \partial \langle \operator{H} \rangle_{\operator{\rho}^{(\mathcal{S})}_{\mathrm{GGE}}}}{\partial \beta } \right )_{-\mu \beta } = \left [ \left .\frac{\mu}{\beta} \frac{ \partial}{\partial \mu } \right )_{T} -\left . \frac{ \partial }{\partial \beta } \right )_{\mu}  \right ]^2\ln Z^{(\mathcal{S})}.		
\end{eqnarray}

%%%%%%%%%%%%%

\section{Remarques sur le formalisme}




%\input{preamble}

\begin{document}

\frontmatter
%\input{chapters/00_intro}
\tableofcontents
\mainmatter

\input{chapters/01_LL_BA}
\input{chapters/02_GGE_TBA}
\input{chapters/03_GHD}
%\input{chapters/97_GHD}
\input{chapters/04_GGE_Fluctuation}
\input{chapters/05_Disp_Exp}
\input{chapters/06_Bipart}
\input{chapters/07_Dipolaire}

%\input{chapters/08_conclusion}
%\appendix
%\input{chapters/99_annexes}

\bibliographystyle{abbrv}
\bibliography{thesis}

%\printbibliography

\end{document}

%| Style     | Description                                                             |
%| --------- | ----------------------------------------------------------------------- |
%| `plain`   | Tri alphabétique, numérotation croissante                               |
%| `unsrt`   | Même que `plain` mais sans tri, respecte l’ordre d’apparition           |
%| `abbrv`   | Comme `plain` mais avec prénoms et noms abrégés                         |
%| `alpha`   | Les références sont étiquetées par une combinaison du nom et de l’année |
%| `apalike` | Style APA simplifié                                                     |
%| `ieeetr`  | Style IEEE, tri par ordre d’apparition                                  |
%| `siam`    | Style SIAM (mathématiques appliquées)                                   |
%| `acm`     | Style ACM (informatique)                                                |
%



\section{Rôle des charges conservées extensives et quasi-locales}
%Dans les systèmes intégrables, l’état stationnaire atteint après une évolution hors d’équilibre n’est généralement pas décrit par un état de Gibbs classique, mais par un ensemble généralisé de Gibbs (GGE). Celui-ci est construit à partir de toutes les charges conservées du système

\paragraph{Écriture des observables thermodynamiques comme sommes sur les rapidités.}

%Dans le cas thermique, les valeurs moyennes des observables classiques telles que le nombre de particules et l'énergie peuvent s'exprimer comme des sommes de puissances des rapidités :
Dans un système à $N$ particules caractérisé par des rapidités $\{ \theta_a \}_{a = 1}^N$, les charges conservées classiques — telles que le nombre de particules, l’impulsion ou l’énergie — s’écrivent comme des sommes de puissances des rapidités :
\(
	\langle \operator{Q} \rangle_{\{ \theta_a\} } \propto \sum_{a = 1}^N \theta_a^0 , \,  \langle \operator{P} \rangle_{\{ \theta_a\} } \propto \sum_{a = 1}^N \theta_a^1  ,\,  \mbox{et} \langle \operator{K} \rangle_{\{ \theta_a\} } \propto \sum_{a = 1}^N \theta_a^2 .	
\)
(cf. équations \eqref{chap.2.gge.1})
Dans ce paragraphe précédent, nous avons sous-entendu — sans l’expliciter — qu’il est montré que l’ensemble des charges locales conservées forme une famille donnée par :
\begin{eqnarray}
	\operator{Q}_i^{(\mathcal{S})} \ket{\{\theta_a\} } & \propto & \sum_a \theta_a^i \ket{\{\theta_a\} }.
\end{eqnarray}
Ces charges agissent donc de manière diagonale sur les états de Bethe, avec des valeurs propres correspondant aux moments des rapidités.
%%%%%%%%%%%%%%%%%%%%%%%%%%%%%%%%%%%%%%%%%%%%%%%%%%
\paragraph{Charges locales conservées .\label{sec:charges-gen}}

%Les états propres du Hamiltonien de Lieb–Liniger~\eqref{eq:LL} sont les états de Bethe
%\(
%  \ket{\boldsymbol{\theta}}
%  =\ket{\theta_1,\dots,\theta_N}\!,
%\)
%déterminés par leurs rapidités \(\boldsymbol{\theta}\).

À toute fonction régulière
\(
  f:\mathbb R\!\to\!\mathbb R
\)
on associe un opérateur-charge loclal :
\begin{eqnarray}\label{chap.2.charge.f.1}
	\operator{\mathcal{Q}}^{(\mathcal{S})}[f] & = &  L \int_0^L d\theta \, f(\theta) \operator{\rho}^{(\mathcal{S})}(\theta).	
\end{eqnarray}
où $\operator{\rho}(\theta)$ agit sur une état de Bethe comme 
\begin{eqnarray}\label{chap.2.rho.1}
	 \operator{\rho}(\theta) \ket{ \{ \theta_a \} } &=& \frac{1}{L} \sum_{a = 1 }^N  \delta ( \theta - \theta_a ) \ket{ \{ \theta_a \} }.	
\end{eqnarray}
De sorte que $\operator{\mathcal{Q}}^{(\mathcal{S})}[f]$ agit sur une état de Bethe comme
\begin{eqnarray}\label{chap.2.charge.1}
	\operator{\mathcal{Q}}^{(\mathcal{S})}[f]\,\ket{\{\theta_a\} } =  \sum_{a=1}^{N}f(\theta_a)\,\ket{\{\theta_a\} } \quad \mbox{de sorte que} \quad \braket{\operator{\mathcal{Q}}^{(\mathcal{S})}[f]}_{\{\theta_a\}} = \sum_{a=1}^N f(\theta_a)
\end{eqnarray}
Les choix particuliers
\(
  f_0(\theta)=1
\)
,
\(
  f_1(\theta)=\theta
\)
et
\(
  f_2(\theta)=\theta^{2}/2
\)
redonnent respectivement l'opérateur nombre \(\operator{Q}=\operator{Q}_0^{(\mathcal{S})} = \operator{\mathcal{Q}}^{(\mathcal{S})}[1]\) , impulsion \(\operator{P}=\operator{Q}_1^{(\mathcal{S})} = \operator{\mathcal{Q}}^{(\mathcal{S})}[\theta]\) et énergie cinétique
\(\operator{K}=\operator{Q}_2^{(\mathcal{S})} = \operator{\mathcal{Q}}^{(\mathcal{S})}[\theta^2/2]\). Et dans le cadre des (GGE), pour tous les ordres $i$ on note :
\begin{eqnarray}\label{chap.2.charge.ordre.i.1}
	\operator{Q}^{(\mathcal{S})}_i = \operator{\mathcal{Q}}^{(\mathcal{S})}[f_i]	, \quad \mbox{de sorte que} \quad \braket{\operator{Q}^{(\mathcal{S})}_i}_{\{\theta_a\}} = \sum_{a=1}^N f_i(\theta_a)  
\end{eqnarray}
avec les densités spectrales $f_i(\theta) \propto \theta^i$ . 

Ces charges sont extensives : leur densité locale $\operator{q}^{(\mathcal{S})}_{[f]}$ permet d’écrire
\(
  \operator{\mathcal{Q}}^{(\mathcal{S})}[f]=\int_0^{L}\!dx\;\operator{q}^{(\mathcal{S})}_{[f]}(x).
\)

\paragraph{Charges conservées généralisée.\label{sec:charges-gen}}
Les fonction $f_i$ étant fixées, on note la fonction régulière
\(
  w:\mathbb R\!\to\!\mathbb R
\)
–– dorénavant appelée \emph{poids spectral}, ou \emph{potentiel spectral} ––
\begin{eqnarray}
	w = \sum_i \beta_i f_i \label{chap.2.w.1},	
\end{eqnarray}
on associe un opérateur-charge généralisé $\operator{\mathcal{Q}}^{(\mathcal{S})}[w]$ :
\begin{eqnarray}\label{chap.2.charge.gen.1}
	\operator{\mathcal{Q}}^{(\mathcal{S})}[w]\,\ket{\{\theta_a\} } =  \sum_{a=1}^{N}w(\theta_a)\,\ket{\{\theta_a\} } \quad \mbox{de sorte que} \quad \braket{\operator{\mathcal{Q}}^{(\mathcal{S})}[w]}_{\{\theta_a\}} = \sum_{i} \beta_i  \braket{\operator{Q}^{(\mathcal{S})}_i}_{\{\theta_a\}}
\end{eqnarray}

%%%%%%%%%%%%%%%%%%%%%%%%%%%%%%%%%%%%%%%%%
\paragraph{Expression de la matrice densité généralisée.}
La matrice densité  s’écrit sous la forme :
L’ensemble général défini par $\operator{\varrho}^{(\mathcal{S})}[w]$ 
\begin{eqnarray}\label{chap.2.densite.1}
	\operator{\varrho}^{(\mathcal{S})}[w]  =  \frac{e^{-\operator{\mathcal{Q}}^{(\mathcal{S})}[w]}}{Z^{(\mathcal{S})}[w]}, \, \mbox{avec} \quad e^{-\operator{\mathcal{Q}}^{(\mathcal{S})}[w]}  = 	\sum_{\{\theta_a \}} e^{- \sum_{a = 1}^N w(\theta_a) } \vert \{ \theta_a\} \rangle \langle  \{ \theta_a\}  \vert, 
\end{eqnarray}	
	%pour une certaine fonction $w$ relié à la charge% $\operator{\mathcal{Q}} [w]  = \sum_{\{\theta_a \}} \left ( \sum_{a = 1}^N w ( \theta_a )  \right ) \vert \{ \theta_a \} \rangle \langle \{ \theta_a \} \vert $.
%où l'opérateur de charge associé à $w$ s’écrit :
%\begin{eqnarray}
%	\operator{\mathcal{Q}} [w]   & = &  \sum_{\{\theta_a \}} \left ( \sum_{a = 1}^N w ( \theta_a )  \right ) \vert \{ \theta_a \} \rangle \langle \{ \theta_a \} \vert,	
%\end{eqnarray}
et la fonction de partition \eqref{chap.TBA.op.Z.S} s'écrit $Z^{(\mathcal{S})}[w]\doteq \bm{\mathrm{Tr}}\left [ e^{-\operator{\mathcal{Q}}^{(\mathcal{S})}[w]}\right ] $ vaux :
\begin{eqnarray}
	Z^{(\mathcal{S})}[w]   =  \sum_{\{\theta_a \}} e^{-\sum_{a = 1}^N w(\theta_a)},\label{chap.TBA.op.Z.S.1}	
\end{eqnarray}
devient un Generalized Gibbs Ensemble (GGE), $\operator{\rho}^{(\mathcal{S})}_{\mathrm{GGE}}$ (de l'équation \eqref{chap.TBA.op.rho.S})	 dès lors que $w(\theta) = \sum_i \beta_i f_i(\theta)$ (de l'équation \eqref{chap.2.w.1}) où $f_i$ sont les densités spectrales associées aux charges locales conservées (de l'équation \eqref{chap.2.charge.ordre.i.1}).


%%%%%%%%%%%%%%%%%%%%%%%%%%%%%%%%%%
\paragraph{Probabilité associée à une configuration de rapidités.}
	%Et on peut réecrire la probabilité de la configuration $\{\theta_a\}$ :% $ P_{\{ \theta_a \}} = \langle \{ \theta_a \}\vert \operator{\rho}_{GGE}[w] \vert  \{ \theta_a \} \rangle = e^{-\sum_{a = 1}^N w(\theta_a)} / Z $ avec $Z = \sum_{\{\theta_a \}} e^{-\sum_{a = 1}^N w(\theta_a)}$.\\
	%La probabilité d’occuper un état à $N$ particules caractérisé par les rapidités ${\theta_a}$ est alors :
Dans ce formalisme, la probabilité d’occuper l’état $\ket{\{\theta \}}$ \eqref{chap.TBA.P.1} est donc
\begin{eqnarray}
	\mathbb{P}^{(\mathcal{S})}_{\{ \theta_a \}} & = &  Z^{(\mathcal{S})}[w]^{-1}e^{-\sum_{a = 1}^N w(\theta_a)}\label{chap.TBA.P.w.2}. 		
\end{eqnarray}
%Cela montre que le poids statistique d’une configuration factorise naturellement sur les pseudo-moments, avec un poids spectrale / energie génralisé $w(\theta)$ attribué à chaque particule.
On voit ainsi que le poids statistique factorise naturellement sur les
pseudo‑moments, chaque particule étant pondérée par $w(\theta_a)$.

%avec 
%\begin{eqnarray}
%	Z  & = & \sum_{\{\theta_a \}} e^{-\sum_{a = 1}^N w(\theta_a)}.		
%\end{eqnarray}


%%%%%%%%%%%%%%%%%%%%%%%%
\paragraph{Moyennes d'observables dans le GGE.}
%La valeur moyenne d’un observable locale $\operator{\mathcal{O}}$ dans l’ensemble généralisé s’écrit :
Pour tout opérateur local $\operator{\mathcal{O}}$ diagonal dans la base de Bethe,
la moyenne généralisée vaut
\begin{eqnarray}\label{chap.2.moyenne.1}
	\langle \operator{\mathcal{O}}\rangle_{\operator{\varrho}^{(\mathcal{S})}[w]} & = & \displaystyle   \frac{\sum_{\{\theta_a \}} \braket{ \operator{\mathcal{O}}}_{\{ \theta_a\}} e^{- \sum_{a = 1}^N w(\theta_a) }  }{\sum_{\{\theta_a  \}} e^{- \sum_{a = 1}^N  w(\theta_a) } }
\end{eqnarray}
%Cette expression formelle montre que la connaissance de $w(\theta)$ suffit à déterminer les propriétés statistiques de toutes les observables diagonales dans cette base, incluant les charges conservées elles-mêmes.
Ainsi, la connaissance de la fonction $w(\theta)$ suffit à déterminer
les propriétés statistiques de toute observable diagonale,
y compris les charges conservées elles‑mêmes.	
	% Nous aimerions calculer les valeurs d'attente par rapport à cette matrice de densité, par exemple
	%La moyenne GGE d'un observable s'écrit ,
	%\begin{aff}
	%\begin{eqnarray}
	%	\langle \operator{\mathcal{O}} \rangle_{GGE} & \doteq & \displaystyle  \text{Tr} (\operator{\mathcal{O}}\operator{\rho}[w]) = \frac{\text{Tr} (\operator{\mathcal{O}}e^{-\operator{\mathcal{Q}}[w]})}{\text{Tr} (e^{-\operator{\mathcal{Q}}[w]})}	 = \frac{\sum_{\{\theta_a \}} \langle  \{ \theta_a\}  \vert   \operator{\mathcal{O}} \vert \{ \theta_a\} \rangle e^{- \sum_{a = 1}^N w(\theta_a) }  }{\sum_{\{\theta_a  \}} e^{- \sum_{a = 1}^N  f(\theta_a) } }
		%& =  & \frac{ \sum_{\pi} \sum_{\vert \{\theta_a \}\rangle \vert \Pi } \langle  \{ \theta_a\}  \vert   \operator{\mathcal{O}} \vert \{ \theta_a\} \rangle e^{- \sum_{a = 1}^N f(\theta_a) }  }{\sum_{\pi} \sum_{\vert \{\theta_a \}\rangle \vert \Pi }  e^{- \sum_{a = 1}^N  f(\theta_a) } }
	%\end{eqnarray}
	%pour une certaine observable $\operator{\mathcal{O}}$.\\
	%\end{aff}
	

\paragraph{Conclusion de la section : vers la thermodynamique de Bethe.}

Nous avons vu que, dans un système intégrable, la description correcte de l’équilibre stationnaire requiert l’introduction d’une \emph{famille infinie de charges conservées}, comprenant à la fois des charges strictement locales et des charges quasi‑locales.
Toutes ces charges se réunissent dans l’opérateur fonctionnel
\(
\operator{\mathcal{Q}}^{(\mathcal{S})}[w]
\)
, défini par un \emph{poids spectral}  $w(\theta)$ (cf. équations~\eqref{chap.2.charge.1}).
Cette construction conduit naturellement à la matrice densité généralisée
\(
\operator{\rho}^{(\mathcal{S})}_{\mathrm{GGE}}  \propto  e^{-\operator{\mathcal{Q}}^{(\mathcal{S})}[w]}
\) 
(cf. équations~\eqref{chap.2.densite.1}), et à la moyenne d’un opérateur local $\operator{\mathcal{O}}$ donnée par
\(
\langle \operator{\mathcal{O}}\rangle_{\operator{\rho}^{(\mathcal{S})}_{\mathrm{GGE}}}  =  \displaystyle  \text{Tr} (\operator{\mathcal{O}}\operator{\varrho}^{(\mathcal{S})}[w])
\)
(cf. équations~\eqref{chap.2.moyenne.1}).
La connaissance de $w(\theta)$ suffit donc pour prédire les valeurs moyennes de toutes les observables diagonales, y compris celles des charges elles‑mêmes ; c’est le cœur du {\bf Ensemble de Gibbs Généralisé (GGE pour Generalized Gibbs Ensemble)} .

\medskip
Cette base est désormais posée : dans la section suivante, nous passerons au \emph{thermodynamique de Bethe}.
Nous verrons comment, dans la limite thermodynamique, les sommes sur les configurations de rapidités se transforment en intégrales sur des densités continues, comment apparaît l’entropie de Yang–Yang, et comment les moyennes de l’ensemble généralisé se réexpriment à l’aide de ces densités macroscopiques.
C’est ce formalisme qui permettra d’analyser finement la relaxation post‑quench et de relier microscopie intégrable et hydrodynamique généralisée.



%\input{preamble}

\begin{document}

\frontmatter
%\input{chapters/00_intro}
\tableofcontents
\mainmatter

\input{chapters/01_LL_BA}
\input{chapters/02_GGE_TBA}
\input{chapters/03_GHD}
%\input{chapters/97_GHD}
\input{chapters/04_GGE_Fluctuation}
\input{chapters/05_Disp_Exp}
\input{chapters/06_Bipart}
\input{chapters/07_Dipolaire}

%\input{chapters/08_conclusion}
%\appendix
%\input{chapters/99_annexes}

\bibliographystyle{abbrv}
\bibliography{thesis}

%\printbibliography

\end{document}

%| Style     | Description                                                             |
%| --------- | ----------------------------------------------------------------------- |
%| `plain`   | Tri alphabétique, numérotation croissante                               |
%| `unsrt`   | Même que `plain` mais sans tri, respecte l’ordre d’apparition           |
%| `abbrv`   | Comme `plain` mais avec prénoms et noms abrégés                         |
%| `alpha`   | Les références sont étiquetées par une combinaison du nom et de l’année |
%| `apalike` | Style APA simplifié                                                     |
%| `ieeetr`  | Style IEEE, tri par ordre d’apparition                                  |
%| `siam`    | Style SIAM (mathématiques appliquées)                                   |
%| `acm`     | Style ACM (informatique)                                                |
%




\section{Thermodynamique de Bethe et relaxation}

%------------------------------------------------------------------
\subsection{Limite thermodynamique}

\paragraph{Observables locales dans la limite thermodynamique.}
%Lorsque l'observable $\operator{\mathcal{O}}$ est suffisamment local, on croit que la valeur d'attente $\langle  \{ \theta_a\}  \vert   \mathcal{O} \vert \{ \theta_a\} \rangle$ ne dépend pas de l'état microscopique spécifique du système, de sorte qu'elle devient une fonctionnelle de $\Pi$ dans la limite thermodynamique.
Dans la suite de ce chapitre, nous omettrons l’exposant $(\mathcal{S})$.
\vspace{0.2em}
Dans la base des états de Bethe \( \{ \ket{\{ \theta_a \}} \} \), l’opérateur \( \hat{\rho}(\theta) \) défini en \eqref{chap.2.rho.1} est diagonal, et agit comme un projecteur sur les valeurs de rapidité.

\vspace{0.5em}

Dans la limite thermodynamique, différentes configurations microscopiques \( \{ \theta_a \} \) peuvent correspondre à la même distribution de rapidité macroscopique \( \rho(\theta) \). Autrement dit, plusieurs états \( \ket{\{ \theta_a \}} \) partagent la même valeur propre \( \rho(\theta) \) de l’opérateur \( \operator{\rho}(\theta) \). Cela reflète une {\em dégénérescence macroscopique} induite par le passage à la limite thermodynamique (\( N, L \to \infty \) avec \( N/L \to \text{const} \)).

\vspace{0.5em}

Si l’observable $\mathcal{O}$ est suffisamment locale, sa valeur d’attente dans un état propre ne dépend pas des détails microscopiques, mais uniquement de la distribution de rapidité. On écrit alors :
\begin{eqnarray}
	\underset{\mbox{\tiny therm.}}{\lim} \braket{  \operator{\mathcal{O}} }_{\{ \theta_a\}}  & = & \langle \operator{\mathcal{O}}\rangle_{[\rho]},
\end{eqnarray}
où $\underset{\mbox{\tiny therm.}}{\lim}$ est la limite thermodynamique ($N,L \to \infty$ avec $N/L \to $ const) et où \( \langle \mathcal{O} \rangle_{[\rho]} \) désigne la valeur d’attente de \( \mathcal{O} \) dans un état macroscopique caractérisé par la distribution de rapidité \( \rho(\theta) \).


\medskip
Dans un ensemble général (GGE), la valeur moyenne de l’observable \eqref{chap.2.moyenne.1} devient alors :		
\begin{eqnarray}\label{chap.2.moyenne.2}
	\underset{\mbox{\tiny therm.}}{\lim} \langle \operator{\mathcal{O}} \rangle_{\operator{\varrho}[w]} & =  & \frac{  \displaystyle \sum_{\rho }  \langle \operator{\mathcal{O}}\rangle_{[\rho]} \Omega[\rho] e^{- \sum_{a = 1}^N  w(\theta_a)    }}{ \displaystyle \sum_{\rho}   \Omega[\rho]\,e^{- \sum_{a = 1}^N  w(\theta_a) } } ,
\end{eqnarray}
où $\sum_{\rho }$ est une somme sus tous les distribution de rapidité $\rho$ et 
où $\Omega[\rho]$ désigne le nombre de micro-états compatibles avec la distribution de rapidité $\rho$.

%où $\# \mbox{micro-états.}$ est les nombre de micro état associée àa la distribution de rapidité $\rho$.
%Avant de se plonger sur $\# \mbox{micro-états.}$, regardons le changement des équation de Bethes. 

\medskip
Pour établir la fonction $\Omega[\rho]$, reppelons-nons de la transformation des équations de Bethe dans dans la limite thermodynamique, hors état fondamentale \eqref{eq:TBA-nu} et \eqref{eq:TBA-rhos-2}.
\begin{equation}
	\nu = \frac{\rho}{\rho_s} \, , \qquad 2\pi \rho_s = 1^{\mathrm{dr}}_{[\nu]} 
\label{chap.2:eq:TBA-rhos}
\end{equation}
où $f^{\mathrm{dr}}_{[\nu]}$ est définie en \eqref{eq:dessing}.

\medskip

Cette formalisation constitue la brique de base de la \textbf{hydrodynamique généralisée} et, dans la section suivante, permet de définir rigoureusement l’\textbf{entropie de Yang–Yang}, indispensable pour décrire la relaxation hors d’équilibre des systèmes intégrables.

%\vspace{1ex}
%La formalisation ci‑dessus fournit la brique de base pour la
%\textbf{hydrodynamique généralisée} et, dans la section suivante, pour la
%définition précise de l’\textbf{entropie de Yang-Yang}
%assurant la relaxation des systèmes intégrables hors‑équilibre.

%\input{preamble}

\begin{document}

\frontmatter
%\input{chapters/00_intro}
\tableofcontents
\mainmatter

\input{chapters/01_LL_BA}
\input{chapters/02_GGE_TBA}
\input{chapters/03_GHD}
%\input{chapters/97_GHD}
\input{chapters/04_GGE_Fluctuation}
\input{chapters/05_Disp_Exp}
\input{chapters/06_Bipart}
\input{chapters/07_Dipolaire}

%\input{chapters/08_conclusion}
%\appendix
%\input{chapters/99_annexes}

\bibliographystyle{abbrv}
\bibliography{thesis}

%\printbibliography

\end{document}

%| Style     | Description                                                             |
%| --------- | ----------------------------------------------------------------------- |
%| `plain`   | Tri alphabétique, numérotation croissante                               |
%| `unsrt`   | Même que `plain` mais sans tri, respecte l’ordre d’apparition           |
%| `abbrv`   | Comme `plain` mais avec prénoms et noms abrégés                         |
%| `alpha`   | Les références sont étiquetées par une combinaison du nom et de l’année |
%| `apalike` | Style APA simplifié                                                     |
%| `ieeetr`  | Style IEEE, tri par ordre d’apparition                                  |
%| `siam`    | Style SIAM (mathématiques appliquées)                                   |
%| `acm`     | Style ACM (informatique)                                                |
%






\subsection{Statistique des macro-états : entropie de Yang-Yang}

%\paragraph{Macro-états et entropie dans la TBA.}

%Dans la limite thermodynamique, dans le modèle statistique (GGE) , les moyenne, observables physiques deviennent des fonctionnelles de la {\bf distribution de rapidité}  $\rho(\theta)$ et du {\bf poing spectrale} $w(\theta)$ . Cette description est efficace car elle permet d’échapper au détail de chaque état propre. 
%Toutefois, cette simplification laisse en suspens une question cruciale : 
%Mais dans ce modelle qui simplifie on veux {\bf la distribution de rapidité d’un système à l'équilibre thermique à température finie} que l'on notera $\langle \rho \rangle$ pour dire la dansité moyenne. Et les lien entre  $w$ et $\langle \rho \rangle$.  Le problème est étudier par par Yang et Yang en 1969. Pour saisir l'enssentielle, nous devons comprendre la {\bf structure statistique des états propres} associés à une même distribution $\rho(\theta)$. Nous nous interrensons comme promis plus haut : à $\Omega(\theta)$ dans l'équation de moyenne \eqref{chap.2.moyenne.2}  ,  {\bf  nombre états propres microscopiques correspondent à une même distribution $\rho(\theta)$}.
%{\bf quelle est la distribution de rapidité d’un système à l'équilibre thermique à température finie ?}. 
%La question a été répondue dans les travaux pionniers de Yang et Yang (1969), que nous allons maintenant examiner brièvement. Pour répondre à cette question, nous devons comprendre la {\bf structure statistique des états propres} associés à une même distribution $\rho(\theta)$.

\paragraph{Motivation.}

Dans la limite thermodynamique, une observable locale dans un \textit{Generalized Gibbs Ensemble} (GGE) dépend uniquement de deux objets continus :  (i)  la \textbf{distribution de rapidité} $\rho(\theta)$, (ii) le \textbf{poids spectral} $w(\theta)$, c.-à-d.\ la " température généralisée " assignée à chaque quasi‑particule.
Cette reformulation est puissante car elle fait disparaître les détails d’un état propre individuel.  

\medskip
Cependant, pour décrire un \emph{vrai} équilibre à température finie, il faut la distribution à l'équilibre :
\begin{eqnarray}\label{chap.2:eq.rho.eq.1}
	\rho_{\mathrm{eq}}(\theta)\;\doteq\;\braket{\operator{\rho}(\theta)}_{\operator{\varrho}[w]}	,  
\end{eqnarray}
donc le lien entre $\rho_{\mathrm{eq}}$ et $w$.
La réponse fut donnée dans les travaux pionniers de \textsc{Yang \& Yang} (1969).  
Leur approche repose sur l’analyse de la \textbf{structure statistique des états propres} partageant la même distribution $\rho(\theta)$.

% : combien d’états microscopiquement distincts correspondent à ce même « macro‑état » ?

\paragraph{Distribution de rapidité comme macro-état.}

Chaque distribution de rapidité $\rho(\theta)$ ne correspond pas à un état propre unique, mais à un grand {\bf ensemble de micro-états} : différents choix des ensembles de quasi-moments $(\{\theta_a\}_{a \in \llbracket 1 , N \rrbracket })_{N \in \mathbb{Z}} $ peuvent conduire à la même densité de distribution à l’échelle macroscopique. Ainsi, $\rho(\theta)$ doit être interprétée comme un {\bf macro-état}, qui agrège un très grand nombre d’états propres microscopiques.

La question thermodynamique devient alors : {\bf Combien de micro-états microscopiquement distincts sont compatibles avec un même macro-état $\rho(\theta)$ ?} 

\medskip
Plus précisément, dans l’expression de moyenne des operateurs locaux \eqref{chap.2.moyenne.2}, apparaît le facteur
\(
\Omega[\rho]
\),
qui compte ces états propres.  
La détermination de $\Omega[\rho]$ (ou équivalemment de l’entropie de Yang–Yang $\mathcal{S}_{YY}[\rho]$ car 
\(
\Omega[\rho] = e^{L\mathcal{S}_{YY}[\rho]}
\)
avec $L$ la taille du système
) est donc la clé pour relier \emph{(i)} le poids spectral $w(\theta)$ imposé dans le GGE et \emph{(ii)} la distribution de rapidité moyenne $\rho_{\mathrm{eq}}(\theta)$ observée à l’équilibre.

\paragraph{Dénombrement local des configurations microcanoniques.}
Pour répondre à cette question, on subdivise l’axe des rapidités en petites tranches ou cellules de largeur $\delta \theta$, chacune centrée en un point $\theta_a$. Dans une tranche $[\theta_a, \theta_a + \delta\theta]$, on suppose que la densité $\rho(\theta)$ est à peu près constante. Le nombre de quasi-particules dans cette tranche est alors approximativement :
\begin{eqnarray*}
	N_a = L\rho(\theta_a) \delta \theta,
\end{eqnarray*}
et le nombre total d'états disponibles (\ie, le nombre d’états possibles si toutes les positions en moment étaient disponibles) est donné par la densité totale de niveaux 
\begin{eqnarray*}
	M_a = L\rho_s(\theta_a) \delta \theta.
\end{eqnarray*}
%La densité de niveaux $\rho_s(\theta)$ tient compte du fait que les moments sont quantifiés de manière discrète, en raison des équations de Bethe (voir équation (??)).

Les particules occupent ces niveaux de manière analogue à des fermions libres (principe d’exclusion de Pauli), le nombre de manières différentes de choisir $N_a$ niveaux parmi $M_a$ est donné par :
	
	
	\begin{figure}[H]
		\centering 
		\begin{tikzpicture}
			%\input{figures/04_GGE_Fluctuation/Occupation_code}	
			\begin{scope}[transform canvas={scale=0.6}]
			\input{figures/04_GGE_Fluctuation/Occupation_theta_code}	
			\end{scope}
			
			\draw[color = red , scale = 0.5 , draw = none ] (-13.5 , -1) rectangle (13 , 10) ; 
				
			
		\end{tikzpicture}	
		\captionsetup{skip=10pt} % Ajoute de l’espace après la légende
	\end{figure}
	
	
\begin{eqnarray}
	\Omega(\theta_a) & \approx  & \binom{M_a}{N_a} ~= ~   \frac{[ L\rho_s ( \theta ) \delta \theta ] ! }{ [ L\rho ( \theta ) \delta \theta ] ! [( L\rho_s ( \theta ) - L\rho ( \theta ) )  \delta \theta ] ! }. 	
\end{eqnarray}

\paragraph{Estimation asymptotique à l’aide de Stirling.}

En utilisant la formule de Stirling :
\begin{eqnarray}
	n! & \underset{n \to \infty}{\sim} &  n^n e^{-n} \sqrt{2\pi n}.,
\end{eqnarray}	
composé du fonction logarithmique, il vient cette équivalence : 
\begin{eqnarray}
	\ln n! & \underset{n \to \infty}{\rightarrow} & n \ln n \underbrace{- n + \ln \sqrt{2 \pi n }}_{o \left ( n \ln n \right ) } ,\\
	&  \underset{n \to \infty}{\sim} & n \ln n  
\end{eqnarray}
	
$\# \mbox{conf.}$ est jamais null donc on peut approximer, pour de grandes valeurs de $L$ et de $\delta\theta$  : 
\begin{eqnarray}
    \ln \Omega(\theta) & \underset{\underset{\rho (\theta )\leq  \rho_s (\theta )}{\rho \delta \theta  \to \infty}}{\sim}   & L [ \rho_s\ln \rho_s - \rho \ln \rho - (\rho_s - \rho ) \ln ( \rho_s - \rho) ] (\theta )\delta \theta .
\end{eqnarray}

Cette expression donne la contribution par unité de $\theta$ à l’{\bf entropie}  associée à la cellule autour de $\theta_a$.

\paragraph{Entropie de Yang-Yang : définition .}
%L'entropie totale du macro-état $\rho(\theta)$, notée $\mathcal{S}_{YY}[\rho]$, est obtenue en sommant sur toutes les tranches. Pour alléger la notation, nous écrivons cette somme comme :
%Le nombre total de micro-états est le produit de toutes ces configurations pour toutes les cellules de rapidité $[\theta, \theta + \delta \theta]$. %En prenant le logarithme et en remplaçant la somme par une intégrale sur $ \theta$, nous obtenons l'entropie de Yang-Yang :

%L'entropie totale du macro-état $\rho(\theta)$, notée $\mathcal{S}_{YY}[\rho]$, est obtenue en sommant sur toutes les tranches. Pour alléger la notation, nous écrivons cette somme comme :
%Le nombre total de micro-états compatibles avec une distribution macroscopique $\rho(\theta)$ est donné par le produit des nombres de configurations pour chaque cellule de rapidité $[\theta, \theta + \delta \theta]$.

%En prenant la sum le logarithme des $\Omega(\theta)$ , on obtient l'entropie totale de Yang-Yang. Pour alléger la notation, cette somme sur les tranches est notée :

Le nombre total de micro-états compatibles avec une distribution macroscopique donnée $\rho(\theta)$ est obtenu en prenant le produit des nombres de configurations pour chaque cellule de rapidité $[\theta_a, \theta_a + \delta \theta]$ : $ \Omega(\theta_a)$ .
En prenant le logarithme de ce produit, on accède à l'entropie totale. Pour alléger la notation, cette somme sur les cellules est notée
\(
	\sum_a^{\theta-\mbox{\tiny cellules}}	
\)
où chaque $a$ indexe une cellule de rapidité $[\theta_a, \theta_a + \delta\theta]$.
On écrit alors :
\begin{eqnarray}
    \ln \Omega[\rho] & = & \sum_a^{\theta-\mbox{\tiny cellules}} \ln \Omega(\theta_a), \\
    & \approx &   L\mathcal{S}_{YY} [ \rho ] , 	
\end{eqnarray}
où l’on définit l’\textbf{entropie de Yang–Yang} par la formule discrétisée :
\begin{eqnarray}
    \mathcal{S}_{YY}[\rho] & \doteq & \sum_a^{\theta-\mbox{\tiny cellules}} \, [ \rho_s\ln \rho_s - \rho \ln \rho - ( \rho_s - \rho ) \ln ( \rho_s - \rho ) ] (\theta_a) \delta \theta .
\end{eqnarray}

%\paragraph{Énergie généralisée.}	
%Les variations de $w(\theta)$ étant négligeables sur chaque tranche de largeur $\delta\theta$, on peut approximer l’énergie généralisée comme :%  $\sum_{a = 1}^N  f(\theta_a) = \sum_{a \vert tranche } f(\theta_a) \Pi( \theta_a)\delta \theta$.

%\begin{eqnarray}
%	 \mathcal{W} & = & \sum_{a = 1}^N  w(\theta_a)	 ~ \sim ~ L\mathcal{W}[\rho] ~=~ L \sum_a^{\theta-\mbox{\tiny tranches}}	 w(\theta_a) \rho(\theta_a) \delta \theta.
%\end{eqnarray}

\paragraph{Énergie généralisée par unité de longueur : définition.}

Dans le cadre du Generalized Gibbs Ensemble (GGE), l’\textbf{énergie généralisée} associée à une distribution de rapidité $\rho(\theta)$ et à un poids spectral $w(\theta)$ est définie comme la somme des poids assignés à chaque quasi-particule. 
Dans la limite thermodynamique, en supposant que $w(\theta)$ varie lentement sur chaque tranche $[\theta_a, \theta_a + \delta\theta]$ ,  cette somme soit l’\textbf{énergie généralisée par unité de longueur} $\mathcal{W}$ se se définit par :
\begin{eqnarray}
	L \mathcal{W}(\{\theta_a\}) \doteq  \sum_{a = 1}^N w(\theta_a) 
	 \underset{\mbox{\tiny therm .}}{\sim}  L \mathcal{W}[\rho]  \doteq  L \sum_a^{\theta\text{-cellules}} w(\theta_a) \rho(\theta_a)\, \delta\theta. 
\end{eqnarray} 
%La fonctionnelle
%\(
%\mathcal{W}[\rho] = \int d\theta\, w(\theta)\, \rho(\theta)
%\)
%représente donc l’énergie généralisée par unité de longueur, dans l’état macroscopique défini par la distribution $\rho$.


\paragraph{Moyenne des Observables locales dans la limite thermodynamique.}

Dans un ensemble général (GGE), la valeur moyenne de l’observable \eqref{chap.2.moyenne.2} devient :	
	
\begin{eqnarray}\label{chap.2.moyenne.3}
	\underset{\mbox{\tiny therm.}}{\lim} \langle \operator{\mathcal{O}} \rangle_{\operator{\varrho}[w]} &  \approx &  ~ \frac{  \displaystyle \sum_{\rho }  \langle \operator{\mathcal{O}}\rangle_{[\rho]}  e^{L(\mathcal{S}_{YY}[\rho] -  \mathcal{W}[\rho]) }}{ \displaystyle \sum_{\rho } e^{L(\mathcal{S}_{YY}[\rho] -  \mathcal{W}[\rho]) } },
\end{eqnarray}
où la somme $\sum\rho$ porte sur toutes les distributions possibles de rapidité $\rho$

%%%%%%%%%%%%%%%%%%%%%%%%%%%%%%%%%%%%%%%%%
\paragraph{Passage à la limite continue.}
%En faisant tandre $\delta \theta \to 0 $ , les somme devienen des integrales 
En faisant tendre $\delta\theta \to 0$, les sommes deviennent des intégrales 
%\(
%\sum_a^{\theta-\mbox{\tiny tranches}}\delta \theta   \underset{\delta \theta \to 0 }{\rightarrow}  \int d \theta ,	
%\)
et l'entropie de Yang-Yang ainsi que l’énergie généralisée par unité de longueur prennent la forme :
\begin{eqnarray}
	\mathcal{S}_{YY}[\rho] & = & \int d \theta  \, [ \rho_s\ln \rho_s - \rho \ln \rho - ( \rho_s - \rho ) \ln ( \rho_s - \rho ) ] (\theta) , \label{chap.2.entropi.int}\\
	\mathcal{W}[\rho] & = & \int   w(\theta) \rho(\theta) \, d \theta \label{chap.2.W.int}		
\end{eqnarray}

%%%%%%%%%%%%%%%%%%%%%%%%%%%%%%%%%%%%%%%%%%
\paragraph{Formule fonctionnelle pour les moyennes.}

%et la valeur moyenne des opservables $\langle \operator{\mathcal{O}} \rangle$ s'écrit commes une intégrale de chemin/formelle
Dans la limite thermodynamique $L \to \infty$, la somme sur les distributions de rapidité $\rho$ admissibles peut être approximée par une intégrale fonctionnelle sur l’espace des densités de rapidité continues, munie d’une mesure fonctionnelle $\mathcal{D}\rho$ : 
\(
\sum_{\rho } \sim \int \mathcal{D} \rho .
\)
Cette correspondance repose sur l’idée que les macro-états admissibles deviennent denses dans l’espace fonctionnel, et que le poids statistique associé à chaque configuration est donné par l’entropie de Yang–Yang.
La mesure fonctionnelle $\mathcal{D}\rho$ parcourt l’espace des densités
$\rho(\theta)$ continues, \emph{chaque configuration étant pondérée par le
facteur exponentiel}
\(
e^{\,L(\mathcal{S}_{YY}[\rho]-\mathcal{W}[\rho])}.
\)
Finalement, la moyenne d'une observable dans le GGE \eqref{chap.2.moyenne.3} s’écrit comme une intégrale fonctionnelle/de chemin :
\begin{eqnarray}
	\underset{\mbox{\tiny therm.}}{\lim} \langle \operator{\mathcal{O}} \rangle_{\operator{\varrho}[w]} & = & \frac{\int \mathcal{D} \rho \; e^{L (\mathcal{S}_{YY}[\rho] - \mathcal{W}[\rho])} \, \langle\operator{\mathcal{O}}\rangle_{[\rho]}}{\int \mathcal{D} \rho \; e^{L (\mathcal{S}_{YY}[\rho] - \mathcal{W}[\rho])}}. \label{chap:TBA:eq:ensemble_average}
\end{eqnarray}


%----------------------
%------------------------------------------------------------------
%\paragraph{Passage de la somme discrète à l’intégrale fonctionnelle.}

%Dans la limite thermodynamique $L\to\infty$, l’ensemble (discret) des
%distributions de rapidité admissibles devient dense dans l’espace
%fonctionnel ; la somme correspondante peut donc s’approximer par une
%intégrale fonctionnelle :
%\[
%\sum_{\rho}\; \longrightarrow\; \int\! \mathcal{D}\rho .
%\]
%La mesure fonctionnelle $\mathcal{D}\rho$ parcourt l’espace des densités
%$\rho(\theta)$ continues, \emph{chaque configuration étant pondérée par le
%facteur exponentiel}
%\(
%e^{\,L\bigl[\mathcal{S}_{YY}[\rho]-\mathcal{W}[\rho]\bigr]},
%\)
%qui combine
%\begin{itemize}
%\item l’\textbf{entropie de Yang–Yang}
%      $\displaystyle
%        \mathcal{S}_{YY}[\rho]
%        =\!\int d\theta\,
%          \bigl[
%            \rho_s\ln\rho_s
%            -\rho\ln\rho
%            -(\rho_s-\rho)\ln(\rho_s-\rho)
%          \bigr]$,
%\item le \textbf{coût énergétique généralisé}
%      $\displaystyle
%        \mathcal{W}[\rho]
%        =\!\int d\theta\, w(\theta)\,\rho(\theta)$,
%\end{itemize}
%où $w(\theta)$ est le \emph{poids spectral} fixé par le GGE.

%------------------------------------------------------------------
%\paragraph{Moyenne d’une observable dans le GGE.}

%On obtient alors la formule de champ moyen
%\begin{equation}\label{eq:GGE-functional-average}
%\bigl\langle\mathcal{O}\bigr\rangle_{\!{\rm GGE}}
%=
%\frac{\displaystyle
%      \int \mathcal{D}\rho\;
%      e^{L\bigl[\mathcal{S}_{YY}[\rho]-\mathcal{W}[\rho]\bigr]}\,
%      \langle\mathcal{O}\rangle_{[\rho]}}
%     {\displaystyle
%      \int \mathcal{D}\rho\;
%      e^{L\bigl[\mathcal{S}_{YY}[\rho]-\mathcal{W}[\rho]\bigr]}}.
%\end{equation}

%------------------------------------------------------------------
\paragraph{Interprétation thermodynamique.}

\begin{itemize}[label = $\bullet$] 
\item $\mathcal{S}_{YY}[\rho]$ \emph{compte} le logarithme du nombre de
      micro-états réalisant la distribution $\rho(\theta)$ :
      c’est l’\textbf{entropie combinatoire}.
\item $\mathcal{W}[\rho]$ mesure le \emph{coût énergétique généralisé}
      associé à cette distribution, dicté par le poids spectral $w(\theta)$.
\end{itemize}

Leur différence
\[
(\mathcal{S}_{YY}-\mathcal{W})[\rho]
\]
joue donc le rôle d’une \emph{fonction thermodynamique effective}
(analogue à une entropie libre).  
L’exposant $e^{L(\mathcal{S}_{YY}-\mathcal{W})[\rho]}$ fixe la \textbf{probabilité relative} d’un
macro-état $\rho(\theta)$ dans le GGE : le terme entropique favorise la
multiplicité des états, tandis que le terme énergétique pénalise les
configurations coûteuses — d’où la compétition caractéristique de
l’équilibre statistique.





%avec $\mathcal{O}[\rho]$ la valeur de l’observable dans un état propre caractérisé par la distribution de rapidité $\rho$.	
%où $\mathcal{O}[\rho]$ est la valeur de l’observable dans un état propre caractérisé par la distribution $\rho$.

%\input{preamble}

\begin{document}

\frontmatter
%\input{chapters/00_intro}
\tableofcontents
\mainmatter

\input{chapters/01_LL_BA}
\input{chapters/02_GGE_TBA}
\input{chapters/03_GHD}
%\input{chapters/97_GHD}
\input{chapters/04_GGE_Fluctuation}
\input{chapters/05_Disp_Exp}
\input{chapters/06_Bipart}
\input{chapters/07_Dipolaire}

%\input{chapters/08_conclusion}
%\appendix
%\input{chapters/99_annexes}

\bibliographystyle{abbrv}
\bibliography{thesis}

%\printbibliography

\end{document}

%| Style     | Description                                                             |
%| --------- | ----------------------------------------------------------------------- |
%| `plain`   | Tri alphabétique, numérotation croissante                               |
%| `unsrt`   | Même que `plain` mais sans tri, respecte l’ordre d’apparition           |
%| `abbrv`   | Comme `plain` mais avec prénoms et noms abrégés                         |
%| `alpha`   | Les références sont étiquetées par une combinaison du nom et de l’année |
%| `apalike` | Style APA simplifié                                                     |
%| `ieeetr`  | Style IEEE, tri par ordre d’apparition                                  |
%| `siam`    | Style SIAM (mathématiques appliquées)                                   |
%| `acm`     | Style ACM (informatique)                                                |
%



\subsection{Équations intégrales de la TBA}

\paragraph{Moyenne des observables dans l’ensemble généralisé de Gibbs.}

\paragraph{Approximation au point selle («\,méthode de la selle statique\,»)}

Dans la limite thermodynamique \( L \to \infty \), cette intégrale est dominée par la configuration \( \rho_{eq} \) qui maximise le poids exponentiel $e^{L(\mathcal{S}_{YY}-\mathcal{W})[\rho]}$  dans l'expression \eqref{chap:TBA:eq:ensemble_average}. Il s’agit de la densité de rapidité la plus probable, solution d’un problème de maximisation. On obtient à l’ordre principal
\begin{eqnarray}
	\underset{\mbox{\tiny therm.}}{\lim} \langle \operator{\mathcal{O}} \rangle_{\operator{\varrho}[w]} & \approx &  \langle\operator{\mathcal{O}}\rangle_{[\rho_{eq} ]},	
	\label{chap:TBA:eq:ensemble_average:approx}
\end{eqnarray}
où $\rho_{eq}$ est la distribution de rapidité à l'équilibre \eqref{chap.2:eq.rho.eq.1}.
Cette approximation correspond à une méthode de \textit{selle statique}, où l’on développe la \emph{fonction thermodynamique effective}, $\mathcal{S}_{YY}-\mathcal{W}$  au voisinage de la distribution dominante.


\paragraph{Développement fonctionnel au premier ordre.}

%On effectue un développement de Taylor fonctionnel de l'action à l’ordre linéaire en $\rho = \rho_{eq} + \delta \rho$ :
Écrivons
\(
\rho=\rho_{\text{eq}}+\delta\rho
\)
et développons $(\mathcal{S}_{YY}-\mathcal{W})[\rho]$ à l’ordre linéaire :
\begin{eqnarray*}
	\mathcal{S}_{YY}[\rho] - \mathcal{W}[\rho] & \approx & \mathcal{S}_{YY}[ \rho_{eq}] - \mathcal{W}[ \rho_{eq}] +  \left. \frac{\delta (\mathcal{S}_{YY}[\rho] - \mathcal{W}[\rho]) }{\delta \rho} \right|_{\rho = \rho_{eq} }	(\delta \rho) + \mathcal{O}(\delta \rho^2 ) ,
	\label{chap:TBA:eq:action}	
\end{eqnarray*}	
La condition de stationnarité au point selle impose :
\(
	\left. \frac{\delta (\mathcal{S}_{YY}[\rho] - \mathcal{W}[\rho]) }{\delta \rho} \right|_{\rho = \rho_{eq} }	  =  0  	
\)
soit 
\begin{equation}
\left. \frac{\delta \mathcal{S}_{YY}}{\delta \rho} \right|_{\rho = \rho_{eq}} = \left. \frac{\delta \mathcal{W}}{\delta \rho} \right|_{\rho = \rho_{eq}}. \label{chap:TBA:eq:stationnarite}
\end{equation}

%%%%%%%%%%%%%%%%
%-----------------------------------------------------

%------------------------------------------------------------------
%\subsection{Équations intégrales de la TBA}

%\paragraph{Moyenne des observables dans le Generalized Gibbs Ensemble.}

%Dans la limite thermodynamique, la moyenne d’une observable locale
%s’écrit formellement comme une intégrale fonctionnelle sur les densités de
%rapidité\,\footnote{%
%La mesure fonctionnelle $\mathcal{D}\rho$ est la limite continue de la
%somme discrète sur les macro-états admissibles, chacun étant pondéré par
%le facteur combinatoire $e^{L\mathcal{S}_{YY}[\rho]}$.}
%
%\begin{equation}\label{eq:TBA:ensemble_average}
%\left\langle \mathcal{O} \right\rangle_{\!\text{GGE}}
%=\frac{\displaystyle
%      \int\!\mathcal{D}\rho\;
%      e^{L\bigl[\mathcal{S}_{YY}[\rho]-\mathcal{W}[\rho]\bigr]}\;
%      \langle\mathcal{O}\rangle_{[\rho]}}
%     {\displaystyle
%      \int\!\mathcal{D}\rho\;
%      e^{L\bigl[\mathcal{S}_{YY}[\rho]-\mathcal{W}[\rho]\bigr]}} .
%\end{equation}

%------------------------------------------------------------------
%\paragraph{Approximation au point selle («\,méthode de la selle statique\,»).}

%Lorsque $L\to\infty$, les intégrales \eqref{eq:TBA:ensemble_average}
%sont dominées par la distribution
%$\rho_{\text{eq}}$ qui \emph{maximise} l’exposant
%\(
%\Phi[\rho]=\mathcal{S}_{YY}[\rho]-\mathcal{W}[\rho].
%\)
%On obtient à l’ordre principal
%\begin{equation}
%\left\langle \mathcal{O} \right\rangle_{\!\text{GGE}}
%\;\simeq\;
%\langle \mathcal{O} \rangle_{[\rho_{\text{eq}}]} .
%\label{eq:TBA:saddle_average}
%\end{equation}

%------------------------------------------------------------------
%\paragraph{Condition de stationnarité et équation variationnelle.}

%Écrivons
%\(
%\rho=\rho_{\text{eq}}+\delta\rho
%\)
%et développons $\Phi[\rho]$ à l’ordre linéaire :
%\[
%\Phi[\rho]\;=\;
%\Phi[\rho_{\text{eq}}]
%+
%\int d\theta\,
%\left.
%\frac{\delta\Phi}{\delta\rho(\theta)}
%\right|_{\rho_{\text{eq}}}
%\delta\rho(\theta)
%+O(\delta\rho^{2}).
%\]
%La stationnarité impose
%\(
%\dfrac{\delta\Phi}{\delta\rho(\theta)}\bigl|_{\rho_{\text{eq}}}=0,
%\)
%soit
%\begin{equation}
%\left.
%\frac{\delta\mathcal{S}_{YY}}{\delta\rho(\theta)}
%\right|_{\rho_{\text{eq}}}
%=
%\left.
%\frac{\delta\mathcal{W}}{\delta\rho(\theta)}
%\right|_{\rho_{\text{eq}}}.
%\label{eq:TBA:variational_condition}
%\end{equation}

%------------------------------------------------------------------
%\paragraph{Forme explicite : introduction de la pseudo-énergie.}

%Pour le modèle de Lieb–Liniger (et, plus généralement, pour un modèle
%intégrable à noyau $\Delta$), on introduit la \emph{pseudo-énergie}
%\[
%\varepsilon(\theta)
%\;=\;
%w(\theta)
%\;+\;\Bigl[\Delta\star\ln\!\bigl(1+e^{-\varepsilon}\bigr)\Bigr](\theta),
%\]
%obtenue en réécrivant \eqref{eq:TBA:variational_condition}.
%Le \emph{facteur d’occupation}
%\(
%\nu(\theta)=\rho(\theta)/\rho_s(\theta)
%\)
%se donne alors par la statistique de type Fermi-Dirac
%\[
%\nu(\theta)=\frac1{1+e^{\varepsilon(\theta)}}.
%\]

%Les équations intégrales complètes de la \textbf{Thermodynamique de Bethe}
%(TBA) sont donc
%\begin{align}
%2\pi\rho_s(\theta) &= 1 + \bigl[\Delta \star \rho\bigr](\theta),
%\label{eq:TBA:rho_s}\\[4pt]
%\rho(\theta) &= \frac{\rho_s(\theta)}{1+e^{\varepsilon(\theta)}},
%\qquad
%\varepsilon(\theta)=w(\theta)+\bigl[\Delta\star\ln(1+e^{-\varepsilon})\bigr](\theta).
%\label{eq:TBA:epsilon}
%\end{align}
%Elles déterminent sans ambiguïté la distribution d’équilibre
%$\rho_{\text{eq}}(\theta)$ en fonction du poids spectral $w(\theta)$.

%\medskip
%Ainsi, la méthode du point selle relie le \emph{poids spectral}
%(caractéristique du GGE) à la distribution de rapidité la plus probable,
%et permet d’évaluer les observables par la formule
%\label{chap:TBA:eq:ensemble_average:approx}.


%-----------------------------------------------------
%%%%%%%%%%%%%%%%

%\paragraph{Équation intégrale de la TBA.}

%Cette égalité donne naissance à une équation intégrale pour le poids spectral \( w \), défini comme la dérivée fonctionnelle de l'énergie généralisée pris en $\rho_{eq}$ :
%\(
%w ~=~ \left. \frac{\delta \mathcal{W}[\rho]}{\delta \rho} \right|_{\rho =  \rho_{eq} }
%\)
%qui par stationnarité (cf équation \eqref{chap:TBA:eq:stationnarite}) est égale à la dérivée fonctionnelle de l'entropie de Yang-Yang pris en $\rho_{eq}$ :
%\(
%\left. \frac{\delta \mathcal{S}_{YY}[\rho]}{\delta \rho} \right|_{\rho = \rho_{eq} }
%\) 
%qui lui vaux 
%\(
%\ln ( \nu_{eq}^{-1}  - 1 ) - \frac{\Delta}{2\pi} \star \ln ( 1 -  \nu_{eq })
%\)
%avec le facteur d'ocupation à l'équilibre $\nu_{eq} = \rho_{eq}/{\rho_{eq}}_s$. Ainci on peux s'arreter sur l'équation 
%\begin{eqnarray}
%	w & = & \ln ( \nu_{eq}^{-1}  - 1 ) - \frac{\Delta}{2\pi} \star \ln ( 1 -  \nu_{eq }).\label{chap:TBA:eq:w}
%\end{eqnarray}

%\medskip
%Ainsi, la méthode du point selle relie le \emph{poids spectral}
%(caractéristique du GGE) à la distribution de rapidité la plus probable,
%et permet d’évaluer les observables par la formule
%\eqref{chap:TBA:eq:ensemble_average:approx}.\\

%\paragraph{Forme explicite : introduction de la pseudo-énergie.}

%Le \emph{facteur d’occupation}
%\(
%\nu_{eq}
%\)
%se donne alors par la statistique de type Fermi-Dirac
%\begin{eqnarray}
%	\nu_{eq}=\frac1{1+e^{\epsilon}},\label{chap:TBA:eq:nu_eq}
%\end{eqnarray}
%où \emph{pseudo-énergie} 
%\(
%\epsilon
%\)
%se définie en intectant \eqref{chap:TBA:eq:nu_eq} dans \eqref{chap:TBA:eq:w} : 
%\begin{eqnarray}
%	\epsilon & = & w + \frac{\Delta}{2\pi} \star \ln ( 1  + e^{-\epsilon}).\label{chap:TBA:eq:e}	
%\end{eqnarray}


%---------------------------------
%------------------------------------------------------------------
\paragraph{Équation intégrale de la TBA.}

La condition de stationnarité au point selle \(\rho=\rho_{\mathrm{eq}}\) \eqref{chap:TBA:eq:stationnarite} implique :
\begin{eqnarray}
	\left.\frac{\delta\mathcal{S}_{YY}}{\delta\rho(\theta)}\right|_{\rho_{\mathrm{eq}}} = \left.\frac{\delta\mathcal{W}}{\delta\rho(\theta)}\right|_{\rho_{\mathrm{eq}}}\;\doteq\;w(\theta),
\end{eqnarray}
En utilisant l’expression explicite de l’entropie de Yang–Yang \eqref{chap.2.entropi.int}, on obtient l’identité fonctionnelle
\begin{eqnarray}
	w & = & \ln ( \nu_{\!eq}^{-1}  - 1 ) - \frac{\Delta}{2\pi} \star \ln ( 1 -  \nu_{\!eq}).\label{chap:TBA:eq:w}
\end{eqnarray}
où
\(
\nu_{\!eq}=\rho_{\!eq}/\rho_{s,\!eq}
\)
est le \textbf{facteur d’occupation} à l’équilibre.
%------------------------------------------------------------------
\paragraph{Forme pseudo-énergie.}
La \textbf{pseudo-énergie} $\epsilon$ se donne alors par la statistique de type Fermi-Dirac
\begin{eqnarray}
	\epsilon =\ln(\nu^{-1}_{\!eq}-1),\qquad\nu_{\!eq}=\frac{1}{1+e^{\epsilon}}.\label{chap:TBA:eq:nu}%\tag{\text{TBA--$\nu$}} 
\end{eqnarray}
En réinjectant \eqref{chap:TBA:eq:nu} dans \eqref{chap:TBA:eq:w} on obtient
l’équation intégrale canonique de la thermodynamique de Bethe :
\begin{eqnarray}
	\epsilon & = & w - \frac{\Delta}{2\pi} \star \ln ( 1  + e^{-\epsilon}).\label{chap:TBA:eq:e}%\tag{\text{TBA–-$\varepsilon$}}	
\end{eqnarray}
%\[
%\boxed{\;
%\varepsilon(\theta)
%=
%w(\theta)
%+\frac{\Delta}{2\pi}\star\ln\!\bigl[1+e^{-\varepsilon(\theta)}\bigr]
%\;}
%\tag{TBA–$\varepsilon$}\label{eq:TBA:eq:e}
%\]

Les relations \eqref{chap:TBA:eq:nu}–\eqref{chap:TBA:eq:e} déterminent de façon univoque la distribution de rapidité d’équilibre \(\rho_{\!eq}\) à partir du poids spectral \(w\), caractéristique du GGE.

\medskip
Ainsi, la méthode du point selle relie \emph{explicitement} le {\em poids spectral}, $w$  (caractéristique du GGE) au \emph{macro-état le plus probable}, $\rho_{eq}$ , et permet d’évaluer les observables par la formule d’ensemble \eqref{chap:TBA:eq:ensemble_average:approx}.


\paragraph{Résolution numérique de l’équation TBA.}\label{para-algho-TBA}

Prenons un poids spectrale quelconque, par exemple : 
\begin{equation}
  w(\theta)= \theta^2 .\label{eq:TBA:w:quadra}
\end{equation} 
En injectant $w$ dans l’équation intégrale pour lapseudo-énergie \eqref{chap:TBA:eq:e}, on obtient l’équation non linéaire.
Cette équation définit un opérateur contractant sur l’espace des fonctions
\( \epsilon(\theta) \) ; son Jacobien a une norme strictement
inférieure à 1, garantissant la convergence de l’itération de Picard.

\medskip
\subparagraph{Algorithme d’itération.}  
La structure contractante de l’équation garantit l’absence de cycles ou de points fixes multiples, assurant la convergence de l’itération vers l’unique solution admissible.
L’équation \eqref{eq:num:TBA} est non linéaire ; pour la résoudre numériquement, on utilise une méthode itérative de type Picard. On initialise
\(
  \epsilon_0 = w ,
\)
puis on construit une suite de fonctions \(\varepsilon_n\) définie par
\begin{eqnarray*}
	\epsilon_{n+1} & = & \epsilon_0 -   \frac{\Delta}{2\pi} \star \ln \left( 1 + e^{-\epsilon_n} \right) ,\quad n\ge0
\end{eqnarray*}
L’itération est poursuivie jusqu’à convergence, que l’on peut tester via le critère numérique
\(
  \beta \left\| \varepsilon_{n+1} - \varepsilon_n \right\|_\infty < 10^{-12},
\)
où \(\|\cdot\|_\infty\) désigne la norme \(L^\infty\) (ou un maximum discret après discrétisation).


\medskip
\subparagraph{Facteur d’occupation et densités.}  
Une fois la pseudo-énergie \( \epsilon(\theta) \) convergée, le facteur d’occupation  à l'équilibre est obtenu en injectant $\epsilon$ dans l’équation \eqref{chap:TBA:eq:nu}, ce qui donne  $\nu_{\!eq}$.
 
On en déduit ensuite la densité d'état à l'équilibre $\rho_{s,eq}$ via le {\bf dressing}  de la fonction constante $f(\theta) = 1$, selon \eqref{eq:TBA-rhos-2}, rappelée ici pour mémoire : $ 2\pi \rho_{s,eq}  =  1^{\mathrm{dr}}_{[\nu_{\! eq}]}$.\\

L’opérateur de dressing \eqref{eq:dressing} étant linéaire, il se résout numériquement sous la forme :
\begin{eqnarray*}
	\left\{ \mathrm{id} - \frac{\Delta}{2\pi} \star ( \nu \ast \cdot ) \right\} f^{\mathrm{dr}}_{[\nu]} & = & f,\label{eq:TBA:rho_s:num}
\end{eqnarray*}
où $\mathrm{id} \colon f \mapsto f$ est l’identité fonctionnelle, et $\ast$ désigne la multiplication.
Après discrétisation de la variable $\theta$, cette équation devient un système linéaire de type $Ax=b$ , facilement résoluble numériquement.

La distribution de rapidité est alors obtenue par $\rho_{\!\mathrm{eq}} = \nu_{\!\mathrm{eq}} \ast \rho_{\! s,\mathrm{eq}}$.\\

\medskip
Ainsi en fixant le poids spectral $w(\theta)$, l’algorithme fournit la pseudo-énergie \( \epsilon \), le facteur d’occupation \( \nu_{\mathrm{eq}} \) et la distribution de rapidité \( \rho_{\!\mathrm{eq}} \).

\medskip
\subparagraph{À l'équilibre thermique.} 
Si on se place à l’équilibre canonique, caractérisé par la température \( T \) et le potentiel chimique \( \mu \).  Dans ce cadre, le poids spectral vaut
\begin{equation}
  w(\theta)=\beta\bigl[\varepsilon(\theta)-\mu\bigr],\qquad\beta=\tfrac1T\; (k_B = 1 ),\quad\varepsilon(\theta)=\tfrac{\theta^{2}}{2}\;(m=1).\label{eq:TBA:w:canonical}
\end{equation}
%En injectant \eqref{eq:TBA:w:canonical} dans l’équation intégrale pour lapseudo-énergie \eqref{chap:TBA:eq:e}, on obtient l’équation non linéaire :
%\begin{eqnarray*}
%	\epsilon & = & \beta(\varepsilon - \mu)  -  \frac{\Delta}{2\pi} \star \ln \left( 1 + e^{-\epsilon} \right) ,\label{eq:num:TBA}
%\end{eqnarray*}
%Ainsi, pour tout couple \((T,\mu)\), l’algorithme fournit la pseudo-énergie \( \epsilon \), le facteur d’occupation \( \nu_{\mathrm{eq}} \) et la distribution de rapidité \( \rho_{\mathrm{eq}} \) à l’équilibre thermique, prêts à être utilisés pour le calcul des observables.
%
%\medskip
%Pour $w$ quelconque , l'algorythme est identique.




		


\chapter{Dynamique hors-équilibre et hydrodynamique généralisée}
\label{chap:GHD}
\minitoc

%\chapter{Hydrodynamique généralisée (GHD)}

\section*{Introduction}


\paragraph{De l’état stationnaire à la dynamique}  
Après avoir étudié les propriétés stationnaires des gaz de bosons unidimensionnels, nous nous tournons désormais vers leur évolution temporelle. Ce chapitre s’appuie sur une approche hydrodynamique adaptée aux systèmes intégrables : la théorie dite d’Hydrodynamique Généralisée (GHD). Celle-ci est largement documentée dans la littérature (voir par exemple [50, 24, 51, 52]) et nous en présentons ici les concepts essentiels.

\paragraph{Principe général d’une approche hydrodynamique}  
De manière générale, l’hydrodynamique vise à décrire la dynamique à grande échelle (\emph{coarse grained dynamics}) d’un système, également appelée « échelle d’Euler ». L’idée consiste à découper l’espace-temps d’un système de taille $L$ en cellules de dimensions $\ell \times \tau$, comme illustré en Fig.~???.  
La longueur $\ell$ est choisie de sorte que $L \gg \ell \gg \ell_c$, où $\ell_c$ désigne une longueur microscopique caractéristique, par exemple la distance inter-particule. On peut alors considérer que la densité est uniforme à l’intérieur de chaque cellule, ce qui correspond à l’Approximation de Densité Locale.

\paragraph{Choix des échelles spatio-temporelles}  
Le temps $\tau$ est fixé pour être beaucoup plus grand que le temps caractéristique de relaxation. Ainsi, chaque cellule de l’espace-temps est supposée décrire un état localement relaxé. La notion de relaxation occupe donc une place centrale dans la construction des approches hydrodynamiques.

\paragraph{Particularités pour les systèmes quantiques isolés}  
Dans le cadre de systèmes quantiques isolés, la relaxation n’est pas un concept trivial, qu’il s’agisse de systèmes chaotiques ou intégrables. La section suivante s’attache à définir plus précisément cette notion, avant de présenter les approches hydrodynamiques adaptées à chaque cas. Pour les systèmes intégrables, une attention particulière est portée à la formulation et aux implications de l’Hydrodynamique Généralisée.

\paragraph{Équations hydrodynamiques de type Euler}  
Les équations hydrodynamiques de type Euler sont des équations hyperboliques qui décrivent la dynamique émergente des systèmes à plusieurs corps à grandes échelles d’espace et de temps~\cite{ref1}. Elles rendent compte de la propagation de la relaxation locale, c’est-à-dire la séparation entre une dynamique lente, émergente, et la projection rapide des observables locales sur les quantités conservées. En une dimension d’espace, elles prennent la forme locale de conservation
\begin{equation}\label{chap:GHD:eq.conserv.1}
	\partial_t q_i + \partial_x j_i = F_i,	
\end{equation}
où l’indice $i$ énumère les lois de conservation locales admises, et où $F_i$ représente les contributions provenant de champs de force externes, qui rompent en général la conservation stricte.

\paragraph{Relations constitutives et exemples}  
Les flux $j_i$ et les termes de force $F_i$ dépendent uniquement des densités conservées $q_i$ (équations d’état), et sont déterminés à partir de considérations thermodynamiques, telles que la maximisation de l’entropie. Les équations d’Euler pour un fluide galiléen, ou encore l’hydrodynamique relativiste, constituent des exemples classiques de ce type d’équations.

\paragraph{Cas intégrable et hydrodynamique généralisée}  
En dimension un, de nombreux systèmes à plusieurs corps présentent une propriété d’intégrabilité~\cite{ref2,ref3}. Dans ce contexte, il existe une infinité de lois de conservation, et la théorie universelle qui décrit leur hydrodynamique à l’échelle d’Euler est l’Hydrodynamique Généralisée (GHD)~\cite{ref4,ref5}. Cette approche englobe les équations connues pour les bâtons durs~\cite{ref1,ref6} et les gaz de solitons~\cite{ref7,ref8,ref9}, tout en s’appliquant plus largement, aussi bien à des systèmes classiques que quantiques : particules en interaction, chaînes de spins ou théories des champs quantiques (voir~\cite{ref10} pour des revues).

\paragraph{Paramétrisation spectrale et densité conservée}  
La GHD reformule l’infinité de lois de conservation (éventuellement rompues) en une famille indexée par un paramètre spectral continu $\theta$, plutôt que par un indice discret $i$. On note $\rho(x,\theta,t)$ la densité conservée en espace réel, espace spectral et temps. Le paramètre spectral énumère les objets asymptotiques issus de la théorie de diffusion correspondante (particules, solitons, etc.), incluant leur quantité de mouvement et leurs éventuels degrés internes. Dans de nombreux cas simples, $\theta$ appartient à un sous-ensemble de $\mathbb{R}$, représentant les moments asymptotiques, et les coordonnées $(x,\theta)$ forment un « espace des phases spectral » sur lequel $\rho$ joue le rôle de densité.

\paragraph{Prise en compte des champs de force}  
L’inclusion de champs de force externes couplés aux densités conservées a été introduite dans~\cite{ref11}, où il est montré que la GHD s’écrit
\begin{equation}\label{chap:GHD:eq.GHD.1}
	\partial_t \rho + \partial_x(v^{\text{eff}} \rho) + \partial_\theta(a^{\text{eff}} \rho) = 0.
\end{equation}
Ici, $v^{\text{eff}}$ et $a^{\text{eff}}$ sont des fonctionnels appropriés de $\rho(x,\cdot,t)$, et le dernier terme représente la contribution des champs de force. D’autres types de forces ont été étudiés~\cite{Bastianello2019a,Bastianello2019b}, mais ne seront pas considérés ici.

%\section{Formulation hamiltonienne de la GHD}
%
%\subsection{Crochet de Poisson fonctionnel}
%
%\paragraph{Définition générale}
%Bonnemain \emph{et al.}~\cite{bonnemain2024hamiltonian} définissent un crochet de Poisson fonctionnel agissant sur les fonctionnelles $F$ et $G$ de la distribution de rapidité, avec interactions :
%\begin{equation}\label{chap:GHD:eq.chochet.bonnemain.1}
%	\{F,G\}=\iint dx\,d\theta\;\frac{\nu}{2\pi}\,\left[\partial_x \left ( \frac{\delta F}{\delta \rho(x,\theta)} \right )\,\left(\partial_\theta \left ( \frac{\delta G}{\delta \rho(x,\theta)} \right ) \right)^{\mathrm{dr}}_{[\nu]} -\partial_x \left ( \frac{\delta G}{\delta \rho (x,\theta)} \right ) \,\left( \partial_\theta \left ( \frac{\delta F}{\delta \rho (x,\theta)} \right )\right)^{\mathrm{dr}}_{[\nu]} \right],
%\end{equation}
%où $\nu$ est la fonction d’occupation. L'application de l’opérateur de \emph{dressing} dans ce crochet traduit les interactions entre particules.
%
%\paragraph{Cas des charges globales}
%Les charges locales conservées ont été définies en \eqref{chap.2.charge.f.1}.  
%Avec le même formalisme, les charges globales conservées se définissent comme fonctionnelles linéaires d’une fonction réelle et régulière \( f(x, \theta) \) définie sur \( \mathbb{R}^2 \) :
%\begin{equation}\label{chap:GHD:eq.charge.global.1}
%	\mathcal{Q}[f] = \int_{\mathbb{R}^2} dx\, d\theta\, f(x, \theta)\, \rho(x, \theta),
%\end{equation}
%qui représente la charge totale associée à une quantité prenant la valeur \( f(x, \theta) \) pour chaque quasi-particule.
%
%Dans notre étude de la dynamique, nous n’avons pas besoin de l’information sur le poids spectral.  
%On notera donc, dans la limite thermodynamique, les moyennes d’opérateurs simplement en retirant leur chapeau :
%\[
%\underset{\mathrm{therm}}{\lim} \braket{\mathcal{O}}_{\varrho[w]} \equiv \mathcal{O}.
%\]
%Ainsi, dans cette limite, la charge globale \eqref{chap:GHD:eq.charge.global.1} s’écrit directement comme ci-dessus.
%
%Le crochet de Poisson \eqref{chap:GHD:eq.chochet.bonnemain.1} appliqué à deux charges globales \( \mathcal{Q}[f] \) et \( \mathcal{Q}[g]\) s’écrit :
%\begin{equation}\label{chap:GHD:eq.chochet.bonnemain.2}
%	\{\mathcal{Q}[f], \mathcal{Q}[g]\} = \int_{\mathbb{R}^2} dx\, d\theta \frac{\nu}{2\pi}  \left( \partial_x f  (\partial_\theta g )^{\mathrm{dr}}_{[\nu]}  - \partial_x g (\partial_\theta f)^{\mathrm{dr}}_{[\nu]}  \right).
%\end{equation}
%L’application du dressing satisfait la symétrie~\cite{doyon2020lecture} :
%\begin{equation}\label{chap:GHD:eq.sym.dr.1}
%	\int_{\mathbb{R}^2}	 dx\, d\theta \, \nu f g^{\mathrm{dr}}_{[\nu]} = \int_{\mathbb{R}^2}	 dx\, d\theta \, \nu f^{\mathrm{dr}}_{[\nu]} g.
%\end{equation}
%Par intégration par parties, le crochet \eqref{chap:GHD:eq.chochet.bonnemain.2} devient :
%\begin{equation}\label{chap:GHD:eq.chochet.bonnemain.3}
%	\{ \mathcal{Q}[f] , \mathcal{Q}[g]\} = \int_{\mathbb{R}^2} dx\, d\theta \,   f  \left( \partial_\theta \left ( \frac{\nu }{2\pi}  (\partial_x g )^{\mathrm{dr}}_{[\nu]} \right )   - \partial_x  \left ( \frac{\nu}{2\pi}  (\partial_\theta g )^{\mathrm{dr}}_{[\nu]} \right )  \right).
%\end{equation}
%
%\subsection{Crochet avec l’Hamiltonien}
%
%\paragraph{Densité hamiltonienne et grandeurs effectives}
%On note $h(x,\theta)$ la densité associée à la moyenne de l’Hamiltonien :
%\begin{equation}\label{chap:GHD:eq.ham.1}
%	H = \mathcal{Q}[h].
%\end{equation}
%La fonction d’occupation $\nu$, la vitesse effective $v^{\mathrm{eff}}$ et l’accélération effective $a^{\mathrm{eff}}$ sont définies par :
%\begin{equation}\label{chap:GHD:eq.nu.v.a.1}
%	\nu = 2\pi \frac{\rho}{1^{\mathrm{dr}}_{[\nu]}}, \quad  
%	v^{\mathrm{eff}} = \frac{(\partial_\theta h )^{\mathrm{dr}}_{[\nu]}}{1^{\mathrm{dr}}_{[\nu]}}, \quad  
%	a^{\mathrm{eff}} = -\frac{(\partial_x h )^{\mathrm{dr}}_{[\nu]}}{1^{\mathrm{dr}}_{[\nu]}},
%\end{equation}
%fonctions de $\rho(x,\theta,t)$.
%
%Le crochet \eqref{chap:GHD:eq.chochet.bonnemain.3} appliqué à $(f,h)$ devient :
%\begin{equation}\label{chap:GHD:eq.chochet.bonnemain.4}
%	\{\mathcal{Q}[f] , \mathcal{Q}[h]\} = -\int_{\mathbb{R}^2} dx\, d\theta \,   f  \left[ \partial_x \left ( \rho  v^{\mathrm{eff}} \right )   +  \partial_\theta   \left ( \rho  a^{\mathrm{eff}} \right )  \right].
%\end{equation}
%
%\paragraph{Forme locale : densités conservées}
%En choisissant $f(x,\theta) \mapsto \delta(\cdot - x)f(\theta)$ dans \eqref{chap:GHD:eq.charge.global.1}, on obtient la densité conservée :
%\[
%q_{[f]}(x) = \mathcal{Q}[(x,\theta) \mapsto \delta(\cdot - x) f(\theta)].
%\]
%Appliquée à \eqref{chap:GHD:eq.chochet.bonnemain.4}, cette prescription donne :
%\begin{equation}\label{chap:GHD:eq.chochet.bonnemain.5}
%	\{ q_{[f]}(x) , \mathcal{Q}[h]\} = - \partial_x \left ( \int_{\mathbb{R}} d\theta \,   f  \,  \rho  \,  v^{\mathrm{eff}} \right ) + \int_{\mathbb{R}} d\theta \, f' \,    \rho \, a^{\mathrm{eff}}.
%\end{equation}
%En utilisant l’équation de Liouville \eqref{chap:GHD:eq.Liouv.1}, on retrouve la forme de convection :
%\begin{equation}\label{chap:GHD:eq.conserv.2}
%	\partial_t q_{[f]} + \partial_x j_{[f]} = F_{[f]},
%\end{equation}
%avec
%\begin{equation}\label{chap:GHD:eq.conserv.2.1}
%	j_{[f]} = \int_{\mathbb{R}} d\theta \,v^{\mathrm{eff}} \, f \, \rho, 
%	\quad F_{[f]} = \int_{\mathbb{R}} d\theta \,  a^{\mathrm{eff}} \, f' \, \rho.
%\end{equation}
%
%\paragraph{Forme locale : équation sur \texorpdfstring{$\rho$}{rho}}
%En prenant $\rho(x,\theta) = \mathcal{Q}[\delta(\cdot - x)\delta(\cdot - \theta)]$ et en l’appliquant à \eqref{chap:GHD:eq.chochet.bonnemain.4}, on obtient :
%\begin{equation}\label{chap:GHD:eq.chochet.bonnemain.6}
%	\{ \rho ( x , \theta ) , \mathcal{Q}[h]\} = - \partial_x \left (  v^{\mathrm{eff}} \,  \rho   \right ) - \partial_\theta \left (  a^{\mathrm{eff}}  \,  \rho  \right).
%\end{equation}
%En appliquant l’équation de Liouville \eqref{chap:GHD:eq.Liouv.1}, on retrouve l’équation GHD :
%\begin{equation}\label{chap:GHD:eq.conserv.3}
%	\partial_t \rho + \partial_x(v^{\mathrm{eff}} \rho) + \partial_\theta(a^{\mathrm{eff}} \rho) = 0.
%\end{equation}
%
%---------------------
\section{Formulation hamiltonienne de la GHD}

\subsection{Crochet de Poisson fonctionnel}

\paragraph{Interprétation et limite non-interactive}  
À ce niveau de généralité, l'équation de l’Hydrodynamique Généralisée (GHD) \eqref{chap:GHD:eq.GHD.1} peut être interprétée comme la dynamique hydrodynamique d’un fluide bidimensionnel dont la densité est conservée dans l’espace des phases spectral.  
Les effets d’interaction se traduisent par un couplage non local dans la direction des rapidités $\theta$, reflétant les processus de diffusion élastique entre quasi-particules possédant des paramètres spectraux distincts.

\medskip

Dans le cas limite d’un système \emph{sans interactions}, l’espace spectral coïncide avec l’espace des phases classique, et l’équation de GHD se réduit alors à l’équation de Liouville (ou, de façon équivalente, à l’équation de Boltzmann sans terme de collisions) issue de la théorie cinétique élémentaire.

\medskip

En l’absence de phénomènes dissipatifs, la densité de distribution $\rho$ est conservée le long du flot hamiltonien associé à l’énergie $H$, ce qui s’exprime par
\begin{equation}\label{chap:GHD:eq.Liouv.1}
	\frac{d \rho}{dt} 
	= \frac{\partial \rho}{\partial t } + \{ \rho , H \} = 0,
\end{equation}
où $\{\cdot , \cdot\}$ désigne le crochet de Poisson canonique dans l’espace des phases.  
Dans cette perspective, l’Hydrodynamique Généralisée apparaît comme une extension naturelle de l’équation de Liouville aux systèmes intégrables, incorporant les effets collectifs induits par les interactions tout en préservant une description exacte à grande échelle.


\paragraph{Structure hamiltonienne et crochet de Poisson fonctionnel}  
Bonnemain \emph{et al.} \cite{bonnemain2024hamiltonian} introduisent un crochet de Poisson fonctionnel agissant sur des fonctionnelles $F$ et $G$ de la distribution de rapidité $\rho(x,\theta)$ en présence d’interactions. Celui-ci s’écrit
\begin{equation}\label{chap:GHD:eq.chochet.bonnemain.1}
	\{F,G\}
	=\iint dx\,d\theta\;\frac{\nu}{2\pi}\,
	\left[
		\partial_x \left( \frac{\delta F}{\delta \rho(x,\theta)} \right)
		\left( \partial_\theta \left( \frac{\delta G}{\delta \rho(x,\theta)} \right) \right)^{\mathrm{dr}}_{[\nu]}
		-
		\partial_x \left( \frac{\delta G}{\delta \rho(x,\theta)} \right)
		\left( \partial_\theta \left( \frac{\delta F}{\delta \rho(x,\theta)} \right) \right)^{\mathrm{dr}}_{[\nu]}
	\right],
\end{equation}
où $\nu$ désigne la fonction d’occupation. Dans ce crochet l'application de l’opérateur de \emph{dressing} $(\cdot)^{\mathrm{dr}}_{[\nu]}$ (introduit dans \eqref{eq:dessing})  traduit les interactions entre particules.

%L’opérateur de \emph{dressing} $(\cdot)^{\mathrm{dr}}_{[\nu]}$ agit ici sur les dérivées fonctionnelles dans la variable spectrale $\theta$ ; il encode les effets des interactions à longue portée dans l’espace des rapidités. Cette structure hamiltonienne permet de reformuler la GHD comme une équation de type Liouville sur l’espace fonctionnel des distributions $\rho$, mais avec un crochet de Poisson modifié par le \emph{dressing}, traduisant la nature intégrable et non-locale des interactions.

\medskip

\paragraph{Charges globales conservées}  
Les charges locales conservées ont été définies dans les équations~\eqref{chap.2.charge.f.1}.  
Dans le même formalisme, on définit les \emph{charges globales conservées} comme des fonctionnelles linéaires agissant sur une fonction réelle et régulière $f(x,\theta)$ définie sur $\mathbb{R}^2$, selon
\begin{equation}\label{chap:GHD:eq.charge.global.0}
	\operator{\mathcal{Q}}[f] 
	= \int_{\mathbb{R}^2} dx\, d\theta\, f(x, \theta)\, \operator{\rho}(x, \theta),
\end{equation}
où $\operator{\rho}(x,\theta)$ est l'opérateur distribution de rapidité.  
Cette quantité correspond à la charge totale associée à une observable prenant la valeur $f(x,\theta)$ pour chaque quasi-particule.

\medskip

La valeur moyenne $\langle \operator{\mathcal{Q}}[f] \rangle_{\operator{\varrho}[w]}$ a été définie en~\eqref{chap.TBA.moy.dens}.  
La matrice densité locale $\operator{\varrho}^{(\mathcal{S})}[w]$ a été introduite en~\eqref{chap.2.densite.1}.  
De manière analogue, la \emph{matrice densité globale} $\operator{\varrho}[w]$ s’écrit
\begin{equation}\label{chap:GHD:eq.charge.global.2}
	\operator{\varrho}[w] 
	= \frac{1}{Z[w]}\, e^{-\operator{\mathcal{Q}}[w]}, 
	\qquad  
	Z[w] = \mathrm{Tr} \left[ e^{-\operator{\mathcal{Q}}[w]} \right],
\end{equation}
où la charge globale $\operator{\mathcal{Q}}[w]$ est définie par~\eqref{chap:GHD:eq.charge.global.0}, et $w$ désigne le poids spectral.  

%Cette formulation met en évidence le lien entre la description statistique du système et la conservation des charges globales, en généralisant le principe de Gibbs aux systèmes intégrables par l’introduction de l’ensemble d’observables $\operator{\mathcal{Q}}[f]$ sur l’espace spectral.

\medskip

\paragraph{Crochet de Poisson entre charges globales}  
Dans notre étude de la dynamique, nous n’avons pas besoin de l’information détaillée sur le poids spectral $w$.  
Nous noterons donc, dans ce chapitre, et dans la limite thermodynamique, les moyennes des opérateurs en supprimant leur chapeau, \emph{i.e.}
\begin{equation}
\underset{\mathrm{therm}}{\lim} \, \langle \operator{\mathcal{O}} \rangle_{\varrho[w]} \; \equiv \; \mathcal{O},
\end{equation}
de sorte que, dans cette limite, la moyenne de la charge globale s’écrit
\begin{equation}\label{chap:GHD:eq.charge.global.1}
	\mathcal{Q}[f] 
	= \int_{\mathbb{R}^2} dx\, d\theta\, f(x, \theta)\, \rho(x, \theta),
\end{equation}
où $f$ est une fonction régulière sur $\mathbb{R}^2$.

\medskip

Le crochet de Poisson (défini en~\eqref{chap:GHD:eq.chochet.bonnemain.1}) entre deux charges $\mathcal{Q}[f]$ et $\mathcal{Q}[g]$ prend la forme
\begin{equation}\label{chap:GHD:eq.chochet.bonnemain.2}
	\{\mathcal{Q}[f], \mathcal{Q}[g]\}
	= \int_{\mathbb{R}^2} dx\, d\theta\, \frac{\nu}{2\pi} 
	\left[ \partial_x f \, (\partial_\theta g)^{\mathrm{dr}}_{[\nu]} 
	     - \partial_x g \, (\partial_\theta f)^{\mathrm{dr}}_{[\nu]} \right].
\end{equation}
%où $\nu$ est la fonction d’occupation et $(\cdot)^{\mathrm{dr}}_{[\nu]}$ désigne l’application de \emph{dressing} associée à $\nu$.

\medskip

Cette application de \emph{dressing} satisfait la relation de symétrie~\cite{doyon2020lecture} :
\begin{equation}\label{chap:GHD:eq.sym.dr.1}
	\int_{\mathbb{R}^2} dx\, d\theta \; \nu \, f \, g^{\mathrm{dr}}_{[\nu]} 
	= \int_{\mathbb{R}^2} dx\, d\theta \; \nu \, f^{\mathrm{dr}}_{[\nu]} \, g.
\end{equation}

Pour appliquer la relation de symétrie~\eqref{chap:GHD:eq.sym.dr.1} au crochet~\eqref{chap:GHD:eq.chochet.bonnemain.2}, il est nécessaire de vérifier que les fonctions impliquées satisfont les conditions requises sur leurs types tensoriels.
\footnote{
La relation de symétrie~\eqref{chap:GHD:eq.sym.dr.1} est valable lorsque la somme des types tensoriels de $f$ et $g$ est $(1,1)$ dans le sens de~\cite{doyon2020lecture}. Dans ce formalisme, le type $(a,b)$ caractérise la transformation d'un objet vis-à-vis de $x$ (première entrée) et de $\theta$ (seconde entrée). Si $f$ est de type $(p,q)$ et $g$ de type $(r,s)$, alors leur somme est $(p+r,q+s)$. La condition $(1,1)$ garantit que l'intégrande $\nu\, f\, g^{\mathrm{dr}}$ est un scalaire invariant, rendant l'intégrale bien définie. Dans~\eqref{chap:GHD:eq.chochet.bonnemain.2}, $\partial_x f$ est de type $(1,0)$ et $\partial_\theta g$ de type $(0,1)$, ce qui satisfait cette condition et permet l'utilisation de~\eqref{chap:GHD:eq.sym.dr.1}.
}

En utilisant cette symétrie ainsi qu’une intégration par parties, le crochet~\eqref{chap:GHD:eq.chochet.bonnemain.2} se réécrit
\begin{equation}\label{chap:GHD:eq.chochet.bonnemain.3}
	\{\mathcal{Q}[f], \mathcal{Q}[g]\}
	= \int_{\mathbb{R}^2} dx\, d\theta \; f \,
	\left[
		\partial_\theta \left( \frac{\nu}{2\pi} \, (\partial_x g)^{\mathrm{dr}}_{[\nu]} \right)
		- \partial_x \left( \frac{\nu}{2\pi} \, (\partial_\theta g)^{\mathrm{dr}}_{[\nu]} \right)
	\right].
\end{equation}

\medskip

\subsection{Crochet avec l’Hamiltonien}

\paragraph{Densité hamiltonienne et grandeurs effectives} 
On note $h(x,\theta)$ la densité associée à la moyenne de l’Hamiltonien, telle que
\begin{equation}\label{chap:GHD:eq.ham.1}
	H = \mathcal{Q}[h].
\end{equation}

La fonction d’occupation $\nu$, la vitesse effective $v^{\mathrm{eff}}$ et l’accélération effective $a^{\mathrm{eff}}$ sont définies par
%\begin{equation}\label{chap:GHD:eq.nu.v.a.1}
%	\nu = 2\pi \frac{\rho}{1^{\mathrm{dr}}_{[\nu]}}, 
%	\quad v^{\mathrm{eff}} = \frac{(\partial_\theta h )^{\mathrm{dr}}_{[\nu]}}{1^{\mathrm{dr}}_{[\nu]}}, 
%	\quad a^{\mathrm{eff}} = -\frac{(\partial_x h )^{\mathrm{dr}}_{[\nu]}}{1^{\mathrm{dr}}_{[\nu]}},
%\end{equation}
\begin{equation}\label{chap:GHD:eq.nu.v.a.1}
	2 \pi \rho =  1^{\mathrm{dr}}_{[\nu]} \, \nu , 
	\quad 2 \pi \, v^{\mathrm{eff}} \, \rho  =(\partial_\theta h )^{\mathrm{dr}}_{[\nu]} \, \nu , 
	\quad 2 \pi \, a^{\mathrm{eff}} \, \rho  = -(\partial_x h )^{\mathrm{dr}}_{[\nu]}\, \nu ,
\end{equation}
toutes trois étant des fonctions de $\rho(\cdot,\cdot,t)$. Ces quantités interviennent dans les équations de mouvement
\begin{equation}
	\dot{x} = v^{\mathrm{eff}}, \qquad \dot{\theta} = a^{\mathrm{eff}},
\end{equation}
montrant que les dérivées $\partial_x$ et $\partial_\theta$ présentes dans le crochet de Poisson correspondent respectivement à l'action de l'accélération effective sur $\theta$ et de la vitesse effective sur $x$.

\medskip 

%Avec ces définitions, le crochet~\eqref{chap:GHD:eq.chochet.bonnemain.3} s’écrit
Le crochet \eqref{chap:GHD:eq.chochet.bonnemain.3} appliqué à $(f,h)$ devient :
\begin{equation}\label{chap:GHD:eq.chochet.bonnemain.4}
	\{\mathcal{Q}[f], \mathcal{Q}[h]\} 
	= - \int_{\mathbb{R}^2} dx\, d\theta \; f \left[ \partial_x \big( \rho \, v^{\mathrm{eff}} \big) 
	+ \partial_\theta \big( \rho \, a^{\mathrm{eff}} \big) \right].
\end{equation}

\paragraph{Forme locale : densités conservées .} 
En choisissant $(x,\theta) \mapsto \delta(\cdot - x)f(\theta)$ dans \eqref{chap:GHD:eq.charge.global.1}, on obtient la moyenne de la densité conservée \ie
%On remarque que les moyennes des densités conservées $q_{[f]}(x)$ s’obtiennent en appliquant la prescription
%\[
%(x,\theta) \mapsto \delta(\cdot - x) \, f(\theta)
%\]
%dans~\eqref{chap:GHD:eq.charge.global.1}, \emph{i.e.}
\begin{equation}
	q_{[f]}(x) = \mathcal{Q} \big[ (x,\theta) \mapsto \delta(\cdot - x) \, f(\theta) \big].
\end{equation}

Appliqué à~\eqref{chap:GHD:eq.chochet.bonnemain.4}, on obtient
\begin{equation}\label{chap:GHD:eq.chochet.bonnemain.5}
	\{q_{[f]}(x), \mathcal{Q}[h]\} 
	= - \partial_x \left[ \int_{\mathbb{R}} d\theta \; f \, \rho \, v^{\mathrm{eff}} \right]
	+ \int_{\mathbb{R}} d\theta \; f' \, \rho \, a^{\mathrm{eff}}.
\end{equation}

En appliquant l’équation de Liouville~\eqref{chap:GHD:eq.Liouv.1}, on retrouve la forme de convection~\eqref{chap:GHD:eq.conserv.1} :
\begin{equation}\label{chap:GHD:eq.conserv.2}
	\partial_t q_{[f]} + \partial_x j_{[f]} = F_{[f]},
\end{equation}
où le flux $j_{[f]}$ et le terme de force $F_{[f]}$ sont donnés par
\begin{equation}\label{chap:GHD:eq.conserv.2.1}
	j_{[f]} = \int_{\mathbb{R}} d\theta \; v^{\mathrm{eff}} \, f \, \rho,
	\quad F_{[f]} = \int_{\mathbb{R}} d\theta \; a^{\mathrm{eff}} \, f' \, \rho.
\end{equation}

\paragraph{Forme locale : équation sur \texorpdfstring{$\rho$}{rho}} 
De manière analogue, pour la distribution de rapidité à l’équilibre thermodynamique, on note
\begin{equation}
	\rho(x,\theta) = \mathcal{Q}\big[ \delta(\cdot - x) \, \delta(\cdot - \theta) \big].
\end{equation}
Appliqué à~\eqref{chap:GHD:eq.chochet.bonnemain.4}, on obtient
\begin{equation}\label{chap:GHD:eq.chochet.bonnemain.6}
	\{\rho(x,\theta), \mathcal{Q}[h]\} 
	= - \partial_x \big( v^{\mathrm{eff}} \, \rho \big)
	  - \partial_\theta \big( a^{\mathrm{eff}} \, \rho \big).
\end{equation}

En appliquant l’équation de Liouville~\eqref{chap:GHD:eq.Liouv.1}, on retrouve l’équation GHD~\eqref{chap:GHD:eq.GHD.1} :
\begin{equation}\label{chap:GHD:eq.conserv.3}
	\partial_t \rho + \partial_x \big( v^{\mathrm{eff}} \rho \big)
	+ \partial_\theta \big( a^{\mathrm{eff}} \rho \big) = 0.
\end{equation}

Le résultat remarquable \eqref{chap:GHD:eq.conserv.3} a été obtenu pour la première fois par Bertini et al. (2016) et Castro-Alvaredo et al. (2016). Cette observation clé a déclenché l’ensemble des développements ultérieurs de la dynamique hydrodynamique généralisée (GHD) dans les systèmes quantiques intégrables. Les travaux de Bertini et al. (2016) s’appuient en partie sur ceux de Bonnes et al. (2014), où la formule donnant la vitesse effective \eqref{chap:GHD:eq.nu.v.a.1} était apparue pour la première fois dans le contexte d’un système quantique intégrable.\\

Les équation \eqref{chap:GHD:eq.Liouv.1},  \eqref{chap:GHD:eq.conserv.2} et \eqref{chap:GHD:eq.conserv.3} décrivent la dynamique au régime d’Euler. En dehors de cette approximation, il est nécessaire de prendre en compte des contributions supplémentaires liées aux effets diffusifs \cite{DeNardis2018}.




\section{Cas particuliers et interpretations}


\subsection{Cas sans interaction}

En l’absence d’interaction, l’opérateur de \emph{dressing} se réduit à l’identité.  
Dans ce cas, la fonction d’occupation \eqref{chap:GHD:eq.nu.v.a.1} devient :
\begin{equation}
	\nu = 2\pi \rho,
\end{equation}
et le crochet \eqref{chap:GHD:eq.chochet.bonnemain.1} se simplifie en :
\begin{equation}
	\{F,G\} = \iint dx\,d\theta\;\rho \,\left[\partial_x \!\left( \frac{\delta F}{\delta \rho (x , \theta)} \right)\, \partial_\theta \!\left( \frac{\delta G}{\delta \rho (x , \theta)} \right) - \partial_x \!\left( \frac{\delta G}{\delta \rho (x , \theta)} \right) \, \partial_\theta \!\left( \frac{\delta F}{\delta \rho(x , \theta) } \right) \right].
\end{equation}

Les flux et termes de force \eqref{chap:GHD:eq.conserv.2.1} s’expriment alors en remplaçant la vitesse effective $v^{\mathrm{eff}}$ et l’accélération effective $a^{\mathrm{eff}}$ (de \eqref{chap:GHD:eq.nu.v.a.1}) par leurs expressions issues de la dynamique hamiltonienne libre :
\begin{equation}
	v^{\mathrm{eff}} \to \partial_\theta h, 
	\quad a^{\mathrm{eff}} \to  -\partial_x h.
\end{equation}

Dans le cadre de \eqref{chap:GHD:eq.ham.2} \(\partial_\theta h = \theta\) et   \(\partial_x h  = V' \). De plus en ne considérant que les premières charges conservées associées à $f(\theta) = 1$, $\theta$ et $\theta^2/2$ dans \eqref{chap:GHD:eq.conserv.2} et \eqref{chap:GHD:eq.conserv.2.1}, on retrouve les équations d’Euler classiques :
\begin{eqnarray}\label{chap:3:eq:hydro.1}
	\left\{
	\begin{array}{rcl}
	\partial_t n + \partial_x (n u) &=& 0, \\
	\partial_t (n u) + \partial_x (n u^2 + \mathcal{P}) &=& -n \, \partial_x V(x), \\
	\partial_t E + \partial_x (u(E+\mathcal{P})) &=& 0,
	\end{array}
	\right .  
\end{eqnarray}
avec la densité de particule $n(x, t) = q_{[1]}$, la vitesse moyenne du fluide $u = \frac{q_{[\theta]}}{n}$ , la pression cinétique du fluide $\mathcal{P}(x, t) = \left( q_{[\theta^2]} - \frac{q_{[\theta]}^2}{n} \right)$ , l'énergie totale $E = nu^2/2 + nV + ne$ où $e(x,t)$ est l'énergie interne d'une particule.



\paragraph{Remarques sur les charges globales}
En l’absence de potentiel externe ($V = 0$), le système conserve certaines charges globales. Dans un système classique non intégrable, seules ces quelques charges sont conservées. Par exemple dans un système de Gibbs sont conservé , nombre total de particules , quantité de mouvement totale , énergie cinétique totale soit respectivement $Q[1]  =  \int dx \, q[1]$ , $Q[\theta] = \int dx \, q[\theta$ et $Q\left[\frac{\theta^2}{2}\right] = \int dx \, q\left[\frac{\theta^2}{2}\right] $. 
%\begin{equation}
%	Q[1]  &= & \int dx \, q[1] \, \text{(nombre total de particules)},\\ 
%	Q[\theta] = \int dx \, q[\theta] \, \text{(quantité de mouvement totale)},\\ 
%	Q\left[\frac{\theta^2}{2}\right] = \int dx \, q\left[\frac{\theta^2}{2}\right] \text{(énergie cinétique totale)}.
%\end{equation}

\medskip

En revanche, dans un système intégrable, une infinité de charges sont conservées. En particulier, pour tout $\theta \in \mathbb{R}$:
\(
\rho(\theta , t ) = Q[\delta(\cdot - \theta)] = \int dx \, \rho(x, \theta, t),
\)
et les charges associées à une observable quelconque $f(\theta)$ s’écrivent :
\(
Q[f] = \int d\theta \, f(\theta) \, \rho(\theta, t).
\)

\medskip

Cette structure est l’analogue classique de la description en termes de {\bf distribution de rapidité} dans le cadre intégrable. Elle constitue le point de départ naturel pour développer une description hydrodynamique généralisée (GHD) dans le cas intégrable.


\subsection{la vitesse effectif}

En partant de la définition de la vitesse effectif en \eqref{chap:GHD:eq.nu.v.a.1} et en utilisant la définition de l'appliocation \emph{dressing} \eqref{eq:dessing}, il vient que 
\begin{eqnarray}\label{chap:GHD:veff.1}
	2 \pi \, v^{\mathrm{eff}} \,   \rho   = \nu  \,  \partial_\theta h   + \nu  \,  \left ( \Delta \star ( \rho \,  v^{\mathrm{eff}} )  \right ) ,
\end{eqnarray}
où $\Delta$ désigne le décalage en diffusion défini dans le modèle LL  en Eq.\eqref{eq:I-1-16}
et en soustraiant $v^{\mathrm{eff}} ( \theta ) \nu (\theta) \left ( \Delta \star  \rho  \right )( \theta )$ des deux cotés et on obtiens 

\begin{eqnarray}\label{chap:GHD:veff.2}
	v^{\mathrm{eff}} ( \theta )  \left (  2\pi \, \rho ( \theta ) - \nu (\theta) \left ( \Delta \star  \rho  \right )( \theta ) \right )    = \nu( \theta )  \,  \left (  \partial_\theta h ( \theta )  +  \int d \theta' \, \Delta(\theta - \theta') \rho ( \theta') ( v^{\mathrm{eff}} ( \theta' ) - v^{\mathrm{eff}} ( \theta )  )   \right )  ,
\end{eqnarray}

En partant de la l'écriture de la fonction d'ocumation en \eqref{chap:GHD:eq.nu.v.a.1} et en utilisant la définition de l'appliocation \emph{dressing} \eqref{eq:dessing}, il vient que 
\begin{eqnarray}\label{chap:GHD:veff.3}
	2 \pi \rho - \nu \, 	\Delta \star  \rho = \nu, 
\end{eqnarray}
On obtient 
\begin{eqnarray}\label{chap:GHD:veff.4}
	v^{\mathrm{eff}} ( \theta )      =  \partial_\theta h ( \theta )   +  \int d \theta' \, \Delta(\theta - \theta') \rho ( \theta') ( v^{\mathrm{eff}} ( \theta' ) - v^{\mathrm{eff}} ( \theta )  )   ,
\end{eqnarray}
et dans le modèle de Lieb-Liniger $\partial_\theta h ( \theta )   = \theta$.
Sur le plan physique, le premier terme peut être interprété comme un décalage spatial induit par un processus de diffusion à deux corps. Le second terme quantifie le taux de ces processus de diffusion par unité de temps. L’équation ainsi obtenue correspond à l’équation hydrodynamique généralisée (GHD), formulée pour la première fois en 2016\cite{Bertini2016,CastroAlvaredo2016}.

Le résultat remarquable~\eqref{eq:46} a été obtenu pour la première fois par Bertini \emph{et al.}~\cite{Bertini2016} et Castro-Alvaredo \emph{et al.}~\cite{CastroAlvaredo2016}. Cette observation a constitué le point de départ des développements ultérieurs de l’hydrodynamique généralisée (GHD) dans les systèmes quantiques intégrables. Les travaux de Bertini \emph{et al.}~\cite{Bertini2016} s’appuient en partie sur ceux de Bonnes \emph{et al.}~\cite{Bonnes2014}, où la formule de la vitesse effective~\eqref{eq:47} avait été présentée pour la première fois dans le cadre d’un système quantique intégrable.

Dans le cadre général de l’hydrodynamique généralisée (GHD), l’équation~(48) s’interprète comme une extension naturelle du résultat classique obtenu pour le gaz de tiges dures. La distinction essentielle réside dans le fait que le décalage de diffusion \(\Delta(\theta - \theta_0)\) dépend désormais explicitement de la rapidité relative entre les quasi-particules, alors que, dans le cas du gaz de tiges dures, \(\Delta\) est une constante égale à l’opposé du diamètre des particules.

Sur le plan cinématique, on peut décrire la situation de la manière suivante~: un quasi-particule \emph{traceur} de rapidité \(\theta\) --- c’est-à-dire de moment asymptotique \(\theta\) en l’absence d’interactions --- se déplacerait, dans le vide, à vitesse constante \( \theta \). En présence d’une densité finie \(\rho(\theta_0)\) de quasi-particules de rapidité \(\theta_0\), cette vitesse est modifiée par les processus de diffusion à deux corps.

Pendant un intervalle de temps infinitésimal \([t,\, t + \delta t]\), le traceur subit en moyenne
\[
\delta t \times \left| v_{\mathrm{eff}}[\rho](\theta) - v_{\mathrm{eff}}[\rho](\theta_0) \right| \, \rho(\theta_0)
\]
collisions avec des quasi-particules de rapidité \(\theta_0\). Chaque interaction provoque un décalage spatial \(\Delta(\theta - \theta_0)\) vers l’arrière. L’équation~(48) formalise précisément cet effet cumulatif, résultant de l’intégration des contributions de toutes les collisions binaires sur l’espace des rapidités.

Cette analyse microscopique s’étend naturellement aux modèles à \(N\) corps, où les processus de diffusion se combinent et interagissent de manière non triviale, la fonction \(\Delta(\theta - \theta_0)\) encapsulant alors l’intégralité de la structure intégrable du système.

\subsection{Modèle de Lieb–Liniger}

Les informations relatives aux interactions entre particules sont contenues dans la définition du crochet de Poisson \eqref{chap:GHD:eq.chochet.bonnemain.1}, associée à l’opérateur de \emph{dressing} spécifique au modèle de Lieb–Liniger, défini en \eqref{eq:dessing}.  
L’Hamiltonien $H = \mathcal{Q}[h]$ \eqref{chap:GHD:eq.ham.1} s’écrit ici :
\begin{equation}\label{chap:GHD:eq.ham.2}
	h(x , \theta ) = \varepsilon(\theta) + V(x),
\end{equation}
où l’énergie cinétique est $\varepsilon(\theta) = \theta^2 / 2$ et $V(x)$ représente le potentiel extérieur.

\medskip

Dans ce modèle, la vitesse effective et l’accélération effective de \eqref{chap:GHD:eq.nu.v.a.1} se réécrivent :
\begin{equation}
	v^{\mathrm{eff}} = \frac{(\mathrm{id})^{\mathrm{dr}}_{[\nu]}}{1^{\mathrm{dr}}_{[\nu]}}, 
	\quad a^{\mathrm{eff}} = - V'(x).
\end{equation}

Avec l'equation \eqref{chap:GHD:veff.4} la vitesse effectiff dans le modèle de LL s'écrit 
\begin{equation}
	v^{\mathrm{eff}} = \theta +	\int d \theta' \, \Delta(\theta - \theta') \rho ( \theta') ( v^{\mathrm{eff}} ( \theta' ) - v^{\mathrm{eff}} ( \theta )  ) 
\end{equation}


Ainsi, les termes de force dans \eqref{chap:GHD:eq.conserv.2} et \eqref{chap:GHD:eq.conserv.2.1} prennent la forme :
\begin{equation}
	F_{[f]} = -V'(x) \int_{\mathbb{R}} d\theta \, f'(\theta) \, \rho(x, \theta).
\end{equation}

L’équation GHD \eqref{chap:GHD:eq.conserv.3} devient alors :
\begin{equation}\label{chap:GHD:eq.conserv.3.1}
	\partial_t \rho + \partial_x\!\left(v^{\mathrm{eff}} \rho\right) - V'(x) \, \partial_\theta \rho = 0.
\end{equation}


\subsection{Diagonalisation et invariants de Riemann dans la GHD spatiale étendue}

En dérivant la définition de l'application \emph{dressing} \eqref{eq:dessing}, on obtient la relation suivante :
\begin{equation}\label{chap:GHD:d.dressing}
	\partial_X(f^{\mathrm{dr}}) = \left (\partial_X f \right )^{\mathrm{dr}} + \frac{\Delta}{2\pi} \star ( f^{\mathrm{dr}} \partial_X \nu ), 	
\end{equation}
où les variables \(X = t, x, \theta\).

\medskip

En injectant les définitions \eqref{chap:GHD:eq.nu.v.a.1} dans l'équation GHD \eqref{chap:GHD:eq.conserv.3} puis en appliquant les dérivées à l'application \emph{dressing} conformément à \eqref{chap:GHD:d.dressing}, on obtient :  
\begin{eqnarray}
	\begin{array}{c}\left ( \left(\partial_t 1 \right)^{\mathrm{dr}} + \left(\partial_x  \partial_\theta h  \right)^{\mathrm{dr}} - \left(\partial_\theta  \partial_x h  \right)^{\mathrm{dr}} \right ) \nu + \left ( 1 + \nu \,  \frac{\Delta}{2 \pi } \star  \right ) \left ( 1^{\mathrm{dr}} \partial_t v +  \left ( \partial_\theta h \right )^{\mathrm{dr}} \partial_x \nu -  \left ( \partial_x h \right )^{\mathrm{dr}} \partial_\theta \nu\right ) = 0  \end{array}. 
\end{eqnarray}
Or, on a \(\partial_t 1 = 0\) et \(\partial_x \partial_\theta h = \partial_\theta \partial_x h\). Il en résulte donc l'équation locale de conservation :  
\begin{equation}
	\partial_t \nu + v^{\mathrm{eff}}\partial_x \nu
	+ a^{\mathrm{eff}} \partial_\theta \nu = 0.
\end{equation}  

\medskip

Dans les systèmes hyperboliques, la \emph{diagonalisation} d'une équation consiste à trouver une transformation des variables qui permet de décomposer le système couplé en un ensemble de modes indépendants, appelés \emph{invariants de Riemann} ou \emph{modes normaux}. 

\medskip

Dans le cadre de la GHD spatiale étendue, l'équation d'évolution de la densité \(\rho(x,\theta,t)\) est couplée de manière non triviale en \((x,\theta)\) par la vitesse effective \(v^{\mathrm{eff}}\) et l'accélération effective \(a^{\mathrm{eff}}\). La fonction d'occupation \(\nu(x,\theta,t)\) est définie par une transformation non locale dite \emph{dressing} qui incorpore les interactions du système.

\medskip

Grâce à cela, on peut affirmer que la fonction $\nu (x , \theta)$  s’interprète comme un continuum d’{\bf invariants de Riemann}, c’est-à-dire des variables normales qui restent constantes le long des caractéristiques du système.

\medskip

Cette diagonalisation est essentielle pour comprendre la structure hamiltonienne du système et simplifier l'analyse de sa dynamique, notamment dans le cadre spatialement étendu avec un dressing dépendant de la position. 





\section{Interpretation}

 






%\section{Equation Hydrodynamique Généralisé}
%
%\subsection{Description classique sans interaction}
%Considérons une distribution classique de particules dans l’espace des phases, notée $\varphi(x, p, t)$, représentant la densité de particules autour du point $(x, p)$ à l’instant $t$. En l’absence de phénomènes dissipatifs, cette densité est conservée le long du flot hamiltonien, c’est-à-dire \(
%\frac{d\varphi}{dt} = \frac{\partial \varphi}{\partial t} + \{ \varphi , H \} = 0,
%\)
%où $\{ \cdot , \cdot \}$ désigne le crochet de Poisson canonique :
%\begin{equation}
%\{ \varphi , H \} = \frac{\partial \varphi}{\partial x} \frac{\partial H}{\partial p} - \frac{\partial \varphi}{\partial p} \frac{\partial H}{\partial x}.
%\end{equation}
%Pour $d \varphi /dt = 0 $ , 
%\begin{equation}
%	\frac{\partial \varphi}{\partial t} + \{ \varphi , H \} = 0	
%\end{equation}
%
%
%Ce résultat exprime que la distribution $\varphi$ est constante le long des trajectoires dans l’espace des phases générées par le hamiltonien $H$. Sous cette hypothèse, on peut réécrire l’équation de conservation sous forme différentielle :
%
%\begin{equation}
%\partial_t \varphi + \partial_x ( \dot{x} \varphi ) + \partial_p ( \dot{p} \varphi ) = 0,
%\end{equation}
%
%où les équations du mouvement hamiltonien sont :
%\(
%\dot{x} = \frac{\partial H}{\partial p}, \qquad \dot{p} = -\frac{\partial H}{\partial x}.
%\)
%
%Cette équation prend alors la forme d’une équation de continuité dans l’espace des phases :
%
%\begin{equation}
%\partial_t \varphi + \partial_x j_x + \partial_p j_p = 0,
%\end{equation}
%
%où les densités de courant sont données par :
%\(
%j_x = \dot{x} \varphi, \qquad j_p = \dot{p} \varphi.
%\)
%
%\paragraph{Exemple : particules libres dans un potentiel externe}
%Prenons pour Hamiltonien :
%
%\begin{equation}
%H = \varepsilon(p) + V(x), \qquad \text{où } \varepsilon(p) = \frac{p^2}{2m},
%\end{equation}
%
%correspondant à un système de particules classiques de masse $m$ soumises à un potentiel externe $V(x)$, sans interaction entre particules.
%
%L'équation de conservation s’écrit alors :
%
%\begin{equation}
%\partial_t \varphi + v(p) , \partial_x \varphi - \partial_x V(x) , \partial_p \varphi = 0,
%\end{equation}
%
%où $v(p) = \partial_p \varepsilon(p) = p/m$ est la vitesse du flot hamiltonien dans l’espace des phases.
%
%\paragraph{Charges locales conservées et équations hydrodynamiques}
%On définit une observable locale (ou charge locale) $q[f](x, t)$ associée à une fonction test $f(p)$ par :
%
%\begin{equation}
%q[f](x, t) = \frac{1}{m} \int_{\mathbb{R}} dp \, f(p) \, \varphi(x, p, t).
%\end{equation}
%
%Cette quantité représente la moyenne locale de $f(p)$ pondérée par la distribution $\varphi$. En particulier : la densité de particules : $n(x, t) = q[1]$,l’impulsion moyenne locale : $u(x, t) = \frac{q[p]}{n m}$, la pression cinétique : $\mathcal{P}(x, t) = \frac{1}{m} \left( q[p^2] - \frac{q[p]^2}{q[1]} \right)$.
%
%Les courants associés à ces charges s’écrivent :
%\begin{equation}
%j[f](x, t) = \frac{1}{m} \int dp \, f(p) , \partial_p H(x, p) \, \varphi(x, p, t).
%\end{equation}
%
%En prenant la dérivée temporelle de $q[f]$ et en utilisant l’équation de Liouville, on obtient une équation de conservation de la forme :
%
%\begin{equation}
%\partial_t q[f] + \partial_x j[f] = \frac{1}{m} \int dp \, f(p) , \partial_p \left( \partial_x V(x) \, \varphi \right),
%\end{equation}
%
%qui ne s’annule en général que si $V(x)$ est constant. Toutefois, dans le régime dit hydrodynamique, où $\varphi(x,p,t)$ varie lentement en espace, cette équation devient fermée sur les seules densités $q[f]$, en négligeant les dérivées spatiales d'ordre élevé.
%
%Dans ce cadre, et en ne retenant que les premières charges conservées associées à $f(p) = 1$, $p$, $p^2$, on retrouve les équations d’Euler classiques :
%
%\begin{eqnarray*}
%	\partial_t n + \partial_x (n u) &=& 0, \\
%	\partial_t (m n u) + \partial_x (m n u^2 + \mathcal{P}) &=& -n \, \partial_x V(x), \\
%	\partial_t \mathcal{E} + \partial_x j[\varepsilon(p)] &=& -\partial_x V(x) \cdot q[p],
%\end{eqnarray*}
%
%où $\mathcal{E} = q[\varepsilon(p)]$ est la densité d'énergie, et $j[\varepsilon(p)]$ le courant d'énergie.
%
%\paragraph{Remarques sur les charges globales}
%En l’absence de potentiel externe ($V = 0$), le système conserve certaines charges globales. Dans un système classique non intégrable, seules ces quelques charges sont conservées. Par exemple dans un système de Gibbs sont conservé
%\(
%	Q[1] = \int dx \, q[1] \quad \text{(nombre total de particules)}, 
%	Q[p] = \int dx \, q[p] \quad \text{(quantité de mouvement totale)}, 
%	Q\left[\frac{p^2}{2m}\right] = \int dx \, q\left[\frac{p^2}{2m}\right] \text{(énergie cinétique totale)}.
%\)
%En revanche, dans un système intégrable, une infinité de charges sont conservées. En particulier, pour tout $p \in \mathbb{R}$:
%
%\begin{equation}
%Q[\delta(\cdot - p)] = \frac{1}{m} \int dx \, \varphi(x, p, t),
%\end{equation}
%
%%et les charges associées à une observable quelconque $f(p)$ s’écrivent :
%%
%%\begin{equation}
%%Q[f] = \int dp , f(p) , \rho(p, t).
%%\end{equation}
%%
%%Cette structure est l’analogue classique de la description en termes de "rapidity distribution function" dans le cadre quantique. Elle constitue le point de départ naturel pour développer une description hydrodynamique généralisée (GHD) dans le cas intégrable.
%%
%%
%
%\subsection{Description classique avec interactions}
%
%%Dans la formulation hamiltonienne de la GHD, le champ dynamique est la densité fluide à deux variables  . 
%On définit un crochet de Poisson fonctionnel agissant sur les fonctionnelles $F[\rho]$ et $G[\rho]$  de la distribution de rapidité , avec intéraction. Conformément à Bonnemain et al.\cite{bonnemain2024hamiltonian}  :
%\begin{equation}
%	\{F,G\}\;=\;\iint dx\,d\theta\;\frac{\nu(\theta)}{2\pi}\,\Bigl[\partial_x\frac{\delta F}{\delta \rho(x,\theta)}\,\left(\partial_\theta \left ( \frac{\delta G}{\delta \rho(x,\theta)} \right ) \right)^{\mathrm{dr}} -\partial_x\frac{\delta G}{\delta \rho (x,\theta)}\,\left( \partial_\theta \left ( \frac{\delta F}{\delta \rho (x,\theta)} \right )\right)^{\mathrm{dr}} \Bigr]\,,	
%\end{equation}
%où $\nu$ est la fonction d’occupation.
%%\cite{bonnemain2024hamiltonian,doyon2020lecture}
%
%Pour toute fonction réelle et régulière \( f(x, \theta) \) définie sur \( \mathbb{R}^2 \), on associe le fonctionnel linéaire suivant :
%\begin{equation}
%	Q[f] = \int_{\mathbb{R}^2} dx\, d\theta\, f(x, \theta)\, \rho(x, \theta).
%\end{equation}
%Il s'agit de la charge totale associée à une quantité prenant la valeur \( f(x, \theta) \) pour chaque quasi-particule.  Le crochet de Poisson entre deux charges \( Q[f] \) et \( Q[g]\) s’écrit :
%\begin{equation}
%	\{ Q[f] , Q[g] \} = \int_{\mathbb{R}^2} \frac{dx\, d\theta}{2\pi} \nu  \left( \partial_x f  (\partial_\theta g )^{\mathrm{dr}}  - \partial_x g (\partial_\theta f)^{\mathrm{dr}}  \right),
%\end{equation}
%or l'application dressing satisfait la relation de symétrie \cite{doyon2020lecture}:
%\begin{equation}
%	\int_{\mathbb{R}^2}	 dx\, d\theta \, \nu f g^{\mathrm{dr}} = \int_{\mathbb{R}^2}	 dx\, d\theta \, \nu f^{\mathrm{dr}} g,
%\end{equation}
%soit avec une integration part partie, on réécrit le crochet 
%\begin{equation}
%	\{ Q[f] , Q[g]\} = \int_{\mathbb{R}^2} \frac{dx\, d\theta}{2\pi} f  \left( \partial_\theta ( \nu   (\partial_x g )^{\mathrm{dr}} )   - \partial_x ( \nu   (\partial_\theta g )^{\mathrm{dr}} )  \right).
%\end{equation}
%
%La distribution de rapidité $\rho( x , \theta )  = Q[\delta( \cdot - x )\delta( \cdot - \theta  )  ]$ et pour un hamiltinien $H = Q[h]$ avec $h(x , \theta ) = \varepsilon(\theta) + V(x)$ avec $\varepsilon(\theta) = m \theta^2/2$.
%
%\begin{equation}
%	\{ \rho(x, \theta), Q[h] \} + \partial_x (v^{\mathrm{eff}} \rho) + \partial_\theta (a^{\mathrm{eff}} \rho) = 0.
%\end{equation}
%
%Nous avons ici utilisé les identités (2.29), ainsi que la définition de la fonction d’occupation (rappelée pour commodité) :
%
%\begin{equation}
%	v^{\mathrm{eff}} = \frac{\varepsilon'^{\mathrm{dr}}}{1^{\mathrm{dr}}}, 
%	\quad 
%	a^{\mathrm{eff}} = -V, 
%	\quad 
%	\nu = \frac{\rho}{\rho_s}.
%\end{equation}
%
%Ainsi, en posant \( \partial_t \rho(x, \theta) = \{ \rho(x, \theta), Q[h] \} \), on retrouve bien les équations de la GHD sous forme hamiltonienne étendue à l’espace :
%
%\begin{equation}
%	\partial_t \rho(x, \theta) = -\partial_x (v^{\mathrm{eff}} \rho) - \partial_\theta (a^{\mathrm{eff}} \rho).
%\end{equation}
%
%











%\input{chapters/97_GHD}
\input{chapters/04_GGE_Fluctuation}
\chapter{Dispositif expérimental et méthodes d’analyse}
\label{chap:disp.exp}
\minitoc

%\section{Présentation de l’expérience}
%\section*{Introduction}
%
%\section{Refroidissement}
%
%\section{Imagerie}
%\subsection{Prubleme d'iamgerie et idée numerique}
%
%\section{Confinement transverse}
%
%\section{Confinement longitudinale}
%
%\subsection{Evolution logitudinale}
%
%\section{Outil de sélection spatial}
%
%\subsection{Mesure de distribution de rapidités locales $\rho(x , \theta ) $  pour des systèmes en équilibre}
%
%%\subsection{Piégeage transverses et longitudinale}
%%\section{Outil de sélection spatial}
%%%\section{Mesure de $\rho(x , \theta ) $ }
%
%%\section{Mesure de distribution de rapidités locales $\rho(x , \theta ) $  pour des systèmes en équilibre}

\section*{Introduction}

%\begin{itemize}
%	\item Objectif du chapitre : présentation synthétique de l’expérience
%	\item Distinction claire des contributions : mise en place initiale (précédents doctorants), développement (travail de Léa Dubois), contribution personnelle (prise de données, analyses spécifiques, participation à certaines manipulations)
%	\item Rôle de l’expérience dans l’étude de la dynamique des gaz de Bose 1D
%\end{itemize}

Ce chapitre présente l’expérience utilisée pour étudier les gaz unidimensionnels de rubidium ultra-froids. Nous décrivons l’architecture du dispositif, les méthodes d’imagerie et d’analyse, ainsi que les protocoles expérimentaux auxquels j’ai participé. Le développement initial du refroidissement et du piégeage avant la puce a été réalisé par d’anciens doctorants. La mise en place du piégeage sur la puce et du système de sélection spatiale à l’aide d’un DMD a été initiée par Léa Dubois, alors en première année de doctorat à mon arrivée. Mon travail s’est concentré principalement sur la prise de données, l’analyse et la participation à certaines expériences spécifiques telles que l’expansion longitudinale et la mesure locale de la distribution de rapidité.


\paragraph{Objectif du chapitre}  
Ce chapitre a pour objectif de fournir une présentation synthétique et structurée du dispositif expérimental utilisé pour étudier la dynamique de gaz de Bose unidimensionnels ultra-froids. Il constitue un socle indispensable pour comprendre les protocoles expérimentaux développés au cours de ma thèse et les analyses présentées dans les chapitres suivants.

\paragraph{Architecture générale}  
Nous présentons d'abord l’architecture complète de l’expérience, depuis la production des atomes jusqu’à leur imagerie, en passant par les étapes de refroidissement, de piégeage magnétique sur puce, de manipulation optique, et de génération de potentiels. Cette description s’accompagne d’une mise en contexte des contributions historiques au dispositif.

\paragraph{Contributions successives et personnelles}  
Une attention particulière est portée à la répartition chronologique des contributions. Les étapes initiales (source atomique, MOT, piège DC) ont été développées par d’anciens doctorants. La mise en place du piégeage 1D sur puce ainsi que l’utilisation du DMD pour la sélection spatiale ont été réalisées au cours de la thèse de Léa Dubois. Mon travail s’inscrit dans cette continuité et concerne principalement la prise de données, l’analyse de protocoles dynamiques, ainsi que la participation à certaines opérations de maintenance et d’optimisation du système.

\paragraph{Rôle du dispositif dans la thèse}  
Ce dispositif permet d’explorer des phénomènes hors équilibre dans des gaz quantiques 1D. Il constitue une plateforme particulièrement adaptée à l’étude de protocoles d’expansion, de sondes locales, ou de dynamiques guidées par la théorie hydrodynamique généralisée (GHD), qui sont au cœur de cette thèse.




%\section{Présentation générale de l’expérience}
%\subsection{Vue d’ensemble du dispositif}
%\begin{itemize}
%    \item Architecture générale : production, piégeage, manipulation et imagerie.
%    \item Systèmes étudiés : gaz de rubidium 87 dans des pièges 1D.
%    \item Objectifs : exploration de dynamiques hors équilibre.
%\end{itemize}
%
%\subsection{Historique et contributions successives}
%\begin{itemize}
%    \item Étapes de refroidissement et piégeage initial : travaux antérieurs (voir thèses citées).
%    \item Développement du piégeage 1D sur puce et du DMD : thèse de Léa Dubois.
%    \item Contributions personnelles : prise de données, protocoles dynamiques, analyse.
%\end{itemize}

\section{Le dispositif expérimental}
\subsection{Système laser et contrôle de fréquence}
\label{sec:systeme_laser}

%\paragraph{Laser maître 1 : référence de fréquence}
%La référence principale de fréquence pour l'ensemble des faisceaux utilisés dans l'expérience est fournie par un laser à cavité étendue, développé au SYRTE. Ce laser est asservi par spectroscopie d’absorption saturée sur la transition D2 du $^{87}$Rb, au croisement des transitions $|F=2\rangle \rightarrow |F'=2,3\rangle$. Ce signal de référence est utilisé pour verrouiller les autres sources laser par battement optique.

\paragraph{Laser maître 1 : référence de fréquence}
La stabilité en fréquence de l’ensemble des faisceaux employés dans l’expérience est assurée par un laser à cavité étendue conçu au SYRTE. Ce laser est verrouillé par spectroscopie d’absorption saturée sur la raie D2 du $^{87}$Rb, en ciblant le croisement des transitions $|F=2\rangle \rightarrow |F'=2,3\rangle$. Ce verrouillage fournit la référence absolue de fréquence à partir de laquelle les autres sources laser sont synchronisées par battement optique.

%\paragraph{Laser repompeur}
%Un laser DFB (Distributed Feedback Diode) est utilisé pour produire le faisceau repompeur, permettant de transférer les atomes retombés dans l’état $|F=1\rangle$ vers l’état $|F=2\rangle$. Ce laser est asservi à une fréquence distante de 6\,GHz de celle du maître 1, en utilisant un montage de battement optique et mélange avec un oscillateur à 6.6\,GHz. Une diode Fabry-Perot injectée par la DFB permet d’amplifier la puissance au-delà de 100\,mW.
%
%\paragraph{Laser repompeur}
%Le faisceau de repompage, qui permet de transférer les atomes piégés dans l’état $|F=1\rangle$ vers l’état $|F=2\rangle$, est généré par une diode DFB (Distributed Feedback). Sa fréquence est décalée de 6,GHz par rapport au maître 1 grâce à un système de battement optique combiné à un mélange avec un oscillateur micro-onde à 6.6,GHz. Une diode Fabry–Perot, injectée par la DFB, permet d’augmenter la puissance de sortie au-delà de 100,mW.

\paragraph{Laser repompeur}
Le faisceau de repompage, qui transfère les atomes tombé  dans l’état $|F=1\rangle$ vers l’état $|F=2\rangle$, est produit par une diode DFB (Distributed Feedback). Sa fréquence est décalée de 6 GHz par rapport au maître 1 par battement optique et mélange avec un oscillateur à micro-ondes de 6.6 GHz. Une diode Fabry–Perot, injectée par la DFB, élève la puissance de sortie au-delà de 100 mW.

%\paragraph{Laser maître 2 : laser principal de manipulation}
%Un second laser à cavité étendue, identique au maître 1, est asservi par battement optique à la fréquence du maître 1. Il est amplifié par un amplificateur à semi-conducteur évasé (Tapered Amplifier), permettant d’atteindre une puissance de sortie supérieure à 1\,W. Ce faisceau est ensuite divisé en plusieurs branches pour alimenter :
%\begin{itemize}
%    \item le Piège Magnéto-Optique (PMO),
%    \item la mélasse optique,
%    \item le pompage optique,
%    \item l’imagerie par absorption,
%    \item le faisceau de sélection.
%\end{itemize}

\paragraph{Laser maître 2 : source principale de manipulation}
Un second laser à cavité étendue, est verrouillé par battement optique sur la fréquence du maître 1. L’émission est amplifiée au moyen d’un amplificateur à semi-conducteur évasé (Tapered Amplifier), fournissant plus de 1\,W en sortie. Le faisceau ainsi produit est distribué vers différentes parties de l’installation expérimentale : alimentation du piège magnéto-optique (PMO), formation de la mélasse optique, réalisation du pompage optique, imagerie par absorption,génération du faisceau de sélection.


%\paragraph{Contrôle de fréquence et polarisation}
%Les fréquences des différents faisceaux sont ajustées via des Modulateurs Acousto-Optiques (AOM), tandis que leur polarisation et leur intensité sont contrôlées à l’aide de cubes PBS en combinaison avec des lames demi-onde motorisées ou fixes. Cette configuration assure une grande flexibilité dans la mise en œuvre des différentes phases expérimentales.

\paragraph{Gestion des fréquences et polarisations}
%Les ajustements de fréquence des divers faisceaux sont réalisés à l’aide de modulateurs acousto-optiques (AOM).
Les faisceaux peuvent être interrompus soit à l’aide d’obturateurs mécaniques, soit via des modulateurs acousto-optiques (AOM). Ces derniers offrent un temps de commutation beaucoup plus court que les systèmes mécaniques, car ils permettent de sélectionner uniquement un ordre de diffraction non nul et d’éteindre instantanément le faisceau en interrompant l’alimentation radiofréquence. L’intensité et la polarisation sont réglées via des cubes séparateurs PBS associés à des lames demi-onde, fixes ou motorisées. Ce dispositif offre une grande souplesse pour adapter la configuration optique aux différentes étapes de l’expérience.

%\paragraph{Remarque}
%Une description plus détaillée du montage laser et de son verrouillage peut être trouvée dans la thèse de A.~Johnson~\cite{Johnson2016}. L’ensemble a été maintenu et utilisé sans modifications majeures au cours de ma thèse.

\paragraph{Note}
Une présentation plus exhaustive du montage laser et de son système de verrouillage est disponible dans la thèse de A.Johnson\cite{Johnson2016}. Le dispositif a été conservé dans son architecture d’origine tout au long de mes travaux, avec seulement un entretien régulier.


\subsection{Production et refroidissement des atomes (non détaillé ici, renvoi à d'autres travaux)}
{\color{blue}
\begin{itemize}
    \item Source chaude de rubidium, MOT, molasses optique.
    \item Refroidissement à des températures sub-$\mu~K$ Refroidissement sub-Doppler (détails renvoyés aux travaux précédents).
\end{itemize}
}
%Le dispositif expérimental permet de produire des gaz de rubidium ultra-froids, avec pour objectif final l’obtention de gaz unidimensionnels dans le régime quantique dégénéré. La production suit une séquence expérimentale déjà bien établie, initialement développée par d’anciens doctorants (voir par exemple la thèse d’A. Johnson~\cite{Johnson2016}), puis réoptimisée au début de la thèse de Léa-Dubois ~\cite{L.Dubois2024} sous la supervision d’I. Bouchoule.

Le dispositif expérimental permet de produire des gaz ultra-froids de rubidium, en vue d’obtenir des gaz unidimensionnels dans le régime quantique dégénéré. La séquence expérimentale suit un protocole établi, initialement développé par d’anciens doctorants (voir par exemple la thèse d’A. Johnson~\cite{Johnson2016}) et réoptimisé au début de la thèse de Léa. Dubois~\cite{L.Dubois2024} sous la supervision d’I. Bouchoule.

%\paragraph{Libération des atomes de rubidium}
%Les atomes de $^{87}$Rb sont libérés à partir d’un \emph{dispenser}, placé directement dans l’enceinte à vide, sur le côté de la monture de la puce atomique. Ce composant, parcouru par un courant de \( 4.5\,\mathrm{A} \) pendant environ \( 5\,\mathrm{s} \), émet un flux d’atomes thermiques dans la chambre à vide.

\paragraph{Libération des atomes de rubidium}
Les atomes de $^{87}$Rb sont émis à partir d’un \emph{dispenser} placé directement dans l’enceinte à vide, à proximité de la monture de la puce atomique. Un courant de \( 4.5\,\mathrm{A} \)  est appliqué pendant environ \( 5\,\mathrm{s} \), générant un flux d’atomes thermiques dans la chambre à vide.

%
%\paragraph{Capture par piège magnéto-optique (PMO)}
%Les atomes thermiques sont ralentis et piégés à l’aide d’un piège magnéto-optique. Celui-ci utilise quatre faisceaux laser (dont deux sont réfléchis par la puce) et un champ quadrupolaire magnétique généré par des bobines. Le nuage ainsi formé se situe à quelques millimètres de la surface de la puce.

\paragraph{Capture par le piège magnéto-optique (PMO)}
Les atomes thermiques sont ralentis et confinés dans un piège magnéto-optique. Quatre faisceaux laser (dont deux réfléchis par la puce) combinés à un champ quadrupolaire magnétique produit par des bobines permettent de former un nuage atomique situé à quelques millimètres de la surface de la puce.

%\paragraph{Rapprochement vers la puce}
%Pour rapprocher les atomes de la puce, on transfère le champ quadrupolaire depuis les bobines vers un champ généré par le fil en forme de U de la puce (fil bleu dans la Fig.~\ref{fig:puce}). Ce fil est parcouru par un courant variant de \( 3.6\,\mathrm{A} \) à \( 1.5\,\mathrm{A} \), ce qui rapproche le nuage à quelques centaines de micromètres de la surface.

\paragraph{Rapprochement vers la puce}
Le nuage est rapproché de la surface de la puce en transférant le champ quadrupolaire depuis les bobines vers le champ produit par le fil en forme de U de la puce (fil bleu, Fig.~\ref{fig:puce}). Le courant dans ce fil est ajusté lentement de \( 3.6\,\mathrm{A} \) à \( 1.5\,\mathrm{A} \), ce qui positionne le nuage à quelques centaines de micromètres de la surface.

%\paragraph{Mélasse optique}
%Une phase de mélasse optique permet un refroidissement sub-Doppler des atomes capturés. Un système d’imagerie provisoire est utilisé à cette étape pour visualiser le nuage atomique, dont la taille dépasse le champ d’observation du système d’imagerie final.

%\paragraph{Mélasse optique}
%Une étape de mélasse optique est ensuite appliquée pour atteindre un refroidissement sub-Doppler des atomes capturés. %Un système d’imagerie provisoire permet de visualiser le nuage, dont la taille dépasse le champ d’observation du dispositif final.

%\paragraph{Pompage optique}
%Afin de polariser les atomes dans l’état magnétique \( |F=2,\,m_F=2\rangle \), un pompage optique est effectué avec un faisceau circulairement polarisé \( \sigma^+ \), résonant sur la transition \( |F=2\rangle \rightarrow |F'=2\rangle \).

\paragraph{Pompage optique}
Enfin, les atomes sont préparés dans l’état magnétique \( |F=2,\,m_F=2\rangle \) par pompage optique. Un faisceau circulairement polarisé \( \sigma^+ \), résonant sur la transition \( |F=2\rangle \rightarrow |F'=2\rangle \), assure la polarisation du nuage.

\paragraph{Mélasse optique}
Après la capture dans le PMO, une étape de mélasse optique est appliquée pour refroidir davantage le nuage atomique, au-delà de la limite de Doppler. La mélasse optique repose sur l’utilisation de faisceaux laser légèrement désaccordés en fréquence et polarisés de manière appropriée, qui interagissent avec les atomes selon le mécanisme de refroidissement sub-Doppler.

Le principe physique est le suivant : les atomes en mouvement voient les faisceaux laser avec un décalage Doppler, ce qui modifie la probabilité d’absorption selon leur vitesse et leur position. Combiné avec les effets de polarisation (notamment les forces de type Sisyphus dans un champ de polarisation variable), cela crée un potentiel de friction optique qui ralentit les atomes. Contrairement au refroidissement Doppler standard, la mélasse optique permet de réduire l’énergie cinétique des atomes en dessous de la limite Doppler, atteignant des températures beaucoup plus basses.

Ainsi, cette étape permet d’obtenir un nuage plus dense et plus froid, condition essentielle pour les manipulations ultérieures et la formation de gaz unidimensionnels dans le régime quantique dégénéré.






\subsection{Piégeage magnétique sur puce}
%{\color{blue}
%\begin{itemize}
%    \item Présentation de la puce atomique.
%    \item Confinement transverse et longitudinal.
%    \item Régime 1D : conditions d’accès (\(\hbar \omega_\perp \gg k_B T\)).
%    \item Problèmes de rugosité, stabilité magnétique.
%\end{itemize}
%}

\subsubsection{Piégeage magnétique sur puce}
\label{sec:piegeage_puce}

%On peut utiliser des piégeage optique pour produire des stracture atomique longitudinale alongé. Certaines groupe de recherche utilise un redeau optique 2D pour obtenir un réseau 2D de tube longitudinaaux \cite{Kinoshita2004,LaburtheTolra2004,Paredes2004,Moritz2003}. Ce raseaux 2D produit un grand nombre de systéme atomique propise è l'étude de de gase 1D. Avec ce genre de dispositif on peux etudier des gas 1D peut dense car les densité peut etre moyenné sur tous les tudes. Mais avec ce genre de dispositif on ne peut pas étudier experimentalement les fluctudation dans le systéme. Nous pour gièger les atomes on utilise une puce atomique.
%
%\paragraph{Principe général}
%Les atomes de rubidium sont piégés grâce à une puce atomique intégrée dans l’enceinte à vide. Une puce atomique est un circuit microfabriqué contenant des micro-fils dans lesquels circulent des courants permettant de générer des champs magnétiques à géométrie contrôlée. Ce dispositif, développé dans les années 1990 \cite{Denschlag1999,Fortagh1998}, permet une miniaturisation du système de piégeage \cite{Folman2000,Reichel1999}, les premiers condensats sur puce ont été obtenus en 2001 \cite{Haensel2001,Ott2001} et la premièref fois aux laboratoir Charles Fabry (LCF) \cite{Aussibal2003} et un accès à des confinements forts, particulièrement adaptés à l'étude de gaz de Bose unidimensionnels \cite{Schumm2005,Trebbia2006}.

-------

On peut créer des structures atomiques allongées en utilisant des techniques de piégeage optique. Par exemple, plusieurs groupes de recherche ont recours à des réseaux optiques bidimensionnels (2D) pour former un ensemble de tubes atomiques longitudinaux \cite{Kinoshita2004,LaburtheTolra2004,Paredes2004,Moritz2003}. Ces réseaux 2D permettent de produire un grand nombre de systèmes atomiques quasi-unidimensionnels, offrant ainsi une plateforme idéale pour l’étude des gaz 1D. Ce type de dispositif est particulièrement adapté à l’étude de gaz faiblement denses, car les densités peuvent être moyennées sur l’ensemble des tubes. Cependant, l’étude expérimentale des fluctuations locales dans chaque tube reste difficile avec ce genre de configuration. Pour surmonter cette limitation, on utilise le piégeage à l’aide de puces atomiques.

\paragraph{Principe général}
Les atomes de rubidium sont confinés par une puce atomique intégrée dans l’enceinte à vide. Une puce atomique est un circuit microfabriqué comportant de fins micro-fils parcourus par des courants électriques, ce qui permet de générer des champs magnétiques à géométrie contrôlée. Cette technologie, développée dans les années 1990 \cite{Denschlag1999,Fortagh1998}, offre une miniaturisation significative des dispositifs de piégeage \cite{Folman2000,Reichel1999}. Les premiers condensats de Bose–Einstein sur puce ont été réalisés en 2001 \cite{Haensel2001,Ott2001}, puis ultérieurement au Laboratoire Charles Fabry \cite{Aussibal2003}. Les puces atomiques permettent d’accéder à des confinements très forts, particulièrement adaptés à l’étude des gaz de Bose unidimensionnels et à l’exploration de leurs propriétés quantiques locales \cite{Schumm2005,Trebbia2006}.


-----
Des structures atomiques allongées peuvent être réalisées par piégeage optique. Dans ce cadre, des réseaux optiques bidimensionnels (2D) permettent de créer un ensemble de tubes atomiques quasi-unidimensionnels \cite{Kinoshita2004,LaburtheTolra2004,Paredes2004,Moritz2003}. Ces réseaux offrent un grand nombre de systèmes atomiques identiques, facilitant l’étude statistique de gaz 1D faiblement dense. Toutefois, l’accès expérimental aux fluctuations locales dans chaque tube reste limité.

Pour contourner cette contrainte, les puces atomiques offrent une solution efficace. Ces dispositifs microfabriqués intègrent de fins micro-fils parcourus par des courants, générant des champs magnétiques de géométrie contrôlée et permettant des confinements très forts \cite{Denschlag1999,Fortagh1998,Folman2000,Reichel1999}. La miniaturisation ainsi obtenue a permis l’obtention des premiers condensats de Bose–Einstein sur puce dès 2001 \cite{Haensel2001,Ott2001}, et dés 2003 au Laboratoire Charles Fabry \cite{Aussibal2003}. Grâce à ces confinements, il devient possible d’étudier expérimentalement les propriétés de gaz de Bose unidimensionnels et leurs fluctuations locales \cite{Schumm2005,Trebbia2006}.

-------

\paragraph{Structure de la puce utilisée}
La puce utilisée au cours de cette expérience a été conçue en collaboration avec S.~Bouchoule, A.~Durnez et A.~Harouri (C2N). Elle repose sur un substrat de carbure de silicium sur lequel est déposé le circuit électrique. Ce dernier est recouvert d’une couche de résine BCB, aplanie par des cycles d’enduction et d’attaque plasma. Une fine couche d’or (\(\sim200\,\mathrm{nm}\)) est finalement évaporée afin de permettre l’utilisation de la puce comme miroir pour l’imagerie à \(780\,\mathrm{nm}\). La puce est soudée à l’indium sur une monture en cuivre inclinée à \(45^\circ\) par rapport à l’axe optique.

%\paragraph{Fils de piégeage et géométrie des champs}
%Plusieurs fils sont intégrés à la puce pour assurer les différentes étapes du piégeage et du transport des atomes : un fil en forme de Z est utilisé pour le piégeage initial (DC), tandis que trois micro-fils (symétriques et parallèles) sont utilisés pour former un guide unidimensionnel par courants alternatifs (AC). La géométrie des fils a été optimisée pour minimiser la dissipation de chaleur, limiter les couplages parasites et améliorer la symétrie du piège. Dans la zone d’intérêt, les atomes sont piégés à environ \(15\,\mu\mathrm{m}\) au-dessus des fils, soit à \(8\,\mu\mathrm{m}\) au-dessus de la surface de la puce.

%\paragraph{Fils de piégeage et géométrie des champs}
%La puce atomique comporte plusieurs ensembles de fils, chacun jouant un rôle précis dans les différentes étapes de la capture, du transport et du confinement des atomes.
\paragraph{Fils de piégeage et géométrie des champs}
La puce atomique intègre plusieurs ensembles de conducteurs, chacun conçu pour une étape spécifique de la capture, du transport et du confinement des atomes. L’ensemble de la séquence de transfert, depuis le piège magnéto-optique (PMO) jusqu’au guide unidimensionnel, repose sur une succession de configurations magnétiques générées par ces différents fils.

%\medskip
%\subparagraph{Fil en forme de U .}
%Après la phase de pré-refroidissement, le nuage est initialement capturé dans un piège magnéto-optique (PMO) situé au-dessus de la puce. Il est ensuite approché de la surface en transférant progressivement le champ quadrupolaire des bobines externes vers celui produit par un fil en forme de U intégré à la puce (phase \textit{U} : transfert du PMO vers la puce + mélace optique + ponpage optique). 

\subparagraph{Phase U : approche de la surface}
Après la phase de pré-refroidissement, le nuage est initialement capturé dans un PMO situé au-dessus de la puce. Il est ensuite rapproché de la surface en transférant progressivement le champ quadrupolaire des bobines externes vers celui produit par un fil en forme de U intégré à la puce (fils bleus dans la Fig.~\ref{fig:puce}). Cette étape (\textit{phase U}) est accompagnée d’un mélange optique et d’un pompage optique afin de préparer les atomes pour le piégeage magnétique.


%\medskip
%\subparagraph{Fil en forme de Z : Chargement dans le piège DC .}
%Après le pompage optique, les atomes sont transférés dans un piège magnétique combinant un courant continu circulant dans le fil en forme de Z de la puce (fil orange dans la Fig.~\ref{fig:puce}) et un champ magnétique externe. Ce piège, noté \emph{piège DC}, permet un confinement transverse important. Un refroidissement par évaporation radiofréquence est alors réalisé pendant environ \( 2.3\,\mathrm{s} \), ce qui abaisse la température du nuage à environ \( 1\,\mu\mathrm{K} \), pour un nombre d’atomes typiquement autour de \( 2.5 \times 10^5 \).

\subparagraph{Phase Z : piège DC et refroidissement}
À l’issue du pompage optique, les atomes sont transférés dans un piège magnétique combinant un courant continu circulant dans un fil en forme de Z (fil orange) et un champ magnétique externe. Ce \emph{piège DC} assure un confinement transverse fort. Un refroidissement par évaporation radiofréquence, d’une durée d’environ \(2.3\,\mathrm{s}\), abaisse la température du nuage à environ \(1\,\mu\mathrm{K}\), pour un nombre typique d’atomes de l’ordre de \(2.5\times 10^5\).
%\medskip
%Une fois chargé dans ce piège intermédiaire, le nuage est transporté vers la zone expérimentale. Dans cette région, trois micro-fils parallèles et symétriques (jaune), parcourus par des courants alternatifs (AC), créent un guide magnétique unidimensionnel assurant le confinement transversal des atomes. Le confinement longitudinal est obtenu grâce à deux paires de fils : d/d′ (rose) et D/D′ (vert).

\subparagraph{Transfert vers le guide unidimensionnel}
Une fois refroidi, le nuage est acheminé vers la zone expérimentale où trois micro-fils parallèles et symétriques (fils jaunes) parcourus par des courants alternatifs (AC) génèrent un guide magnétique unidimensionnel assurant le confinement transverse. Le confinement longitudinal est fourni par deux paires de fils : $d/d'$ (rose) et $D/D'$ (vert).

Le passage du piège DC au guide 1D est réalisé de manière adiabatique grâce à cinq rampes linéaires de courant d’une durée comprise entre \(50\) et \(60\,\mathrm{ms}\) chacune. Durant cette opération :  
(i) le courant dans le fil Z est progressivement réduit,  
(ii) le courant dans les micro-fils du guide est augmenté jusqu’à environ \(50\,\mathrm{mA}\),  
(iii) un courant initial de \(0.5\,\mathrm{A}\) est appliqué dans les fils $D$ et $D'$, puis ajusté pour maintenir fixe la position du centre de masse du nuage.  

Ce protocole minimise les oscillations résiduelles dans le guide et assure un découplage efficace entre la dynamique longitudinale et le confinement transverse. Ce dispositif a été développé au cours de la thèse de Léa Dubois~\cite{TheseLea} et a été utilisé dans le cadre de mes protocoles expérimentaux sur l’expansion longitudinale et les sondes locales de distribution de rapidité.

\subparagraph{Optimisation géométrique}
La géométrie des conducteurs de la puce a été conçue pour réduire la dissipation thermique, limiter les couplages parasites et garantir une bonne symétrie des champs magnétiques. Dans la zone expérimentale, les atomes sont piégés à environ \(15\,\mu\mathrm{m}\) au-dessus des fils, soit \(8\,\mu\mathrm{m}\) au-dessus de la surface de la puce.


\paragraph{Refroidissement final et accès au régime unidimensionnel}
Une dernière phase de refroidissement par évaporation radiofréquence est effectuée directement dans le guide AC. Grâce à l’anisotropie marquée du piège, le confinement transverse atteint une fréquence \(\omega_\perp\) telle que l’énergie quantique \(\hbar \omega_\perp\) dépasse largement les énergies thermique et chimique du système. On atteint ainsi le régime unidimensionnel, caractérisé par la hiérarchie d’énergies :
\[
k_B T, \mu \ll \hbar \omega_\perp,
\]
où \(\mu\) désigne le potentiel chimique et \(T\) la température du gaz.

Dans ce régime, le confinement transverse est assuré principalement par la géométrie des micro-fils et la présence de champs magnétiques externes, tandis que le confinement longitudinal, plus faible, est ajustable via une combinaison de champs magnétiques externes et de courants circulant dans des fils additionnels ($d/d'$ et $D/D'$). 

Les gaz obtenus contiennent typiquement entre \(3\times 10^3\) et \(1.5\times 10^4\) atomes, pour des températures de l’ordre de \(50\) à \(200\,\mathrm{nK}\). La Fig.~\ref{fig:gaz1D} illustre un exemple de nuage dans ce régime, observé avec le système d’imagerie final.



%\paragraph{Confinement transverse et longitudinal}
%Le confinement transverse est assuré principalement par la géométrie des fils et la présence de champs magnétiques externes. Sa fréquence élevée permet d’atteindre des énergies de confinement \(\hbar \omega_\perp\) bien supérieures aux énergies thermiques et chimiques du système, condition nécessaire à l’accès au régime 1D :
%\[
%k_B T, \mu \ll \hbar \omega_\perp.
%\]
%Le confinement longitudinal, plus faible, est modulable par combinaison de champs magnétiques externes et courants dans les fils additionnels.

\paragraph{Avantages du piégeage sur puce}
Comparé aux systèmes utilisant des réseaux optiques 2D, le piégeage sur puce ne fournit qu’un seul tube, ce qui permet un meilleur accès aux fluctuations locales de densité et aux observables résolues spatialement. Ce type de dispositif est ainsi particulièrement adapté à l'étude de la thermodynamique et de la dynamique de gaz 1D isolés.

\paragraph{Limitations et effets parasites}
Parmi les limitations spécifiques au piégeage sur puce figurent la rugosité des potentiels magnétiques due aux imperfections des fils, qui peut induire des modulations parasites du confinement longitudinal. De plus, la stabilité du dispositif est sensible aux champs parasites magnétiques externes ainsi qu’aux échauffements dus aux courants continus.





\paragraph{Imagerie finale}
À l’issue de ce refroidissement, les atomes sont observés avec le système d’imagerie final (voir Fig.~\ref{fig:imagerieFinale}), adapté aux tailles caractéristiques du gaz dans le piège. Une image typique de ce nuage est présentée en Fig.~\ref{fig:nuageDC}.



%\paragraph{Refroidissement final et accès au régime unidimensionnel}
%Une dernière phase de refroidissement par évaporation radiofréquence est ensuite réalisée dans le guide AC. Ce refroidissement, mené dans le piège à forte anisotropie, permet d’atteindre le régime unidimensionnel, caractérisé par la hiérarchie d’énergies :
%\[
%k_B T, \mu \ll \hbar \omega_\perp
%\]
%où \( \omega_\perp \) est la fréquence de confinement transverse, \( \mu \) le potentiel chimique et \( T \) la température du gaz.
%
%Les gaz obtenus contiennent typiquement entre \( 3 \times 10^3 \) et \( 1.5 \times 10^4 \) atomes, pour des températures de l’ordre de \( 50 \text{ à } 200\,\mathrm{nK} \). La Fig.~\ref{fig:gaz1D} montre un exemple de tel gaz observé avec le système d’imagerie final.


\paragraph{Remarques expérimentales}
Lorsque j’ai rejoint l’équipe, la première année thèse de Léa Dubois touchait à sa fin et le dispositif expérimental était en fonctionnement stable. Les différentes étapes du cycle (dispenser, PMO, mélasse, pompage optique, piège DC, transfert vers le guide, évaporation finale) avaient été mises en place et optimisées pendant les premières années de sa thèse, sous la supervision d’I. Bouchoule.Le cycle expérimental complet dure environ 15 secondes. Une description plus détaillée peut être trouvée dans la thèse d’A. Johnson~\cite{Johnson2016}.


Pendant ma première année, j’ai principalement participé à la prise de données en collaboration avec Léa. Grâce à la qualité de son travail, le dispositif était globalement très fiable, ce qui a permis de mener des campagnes expérimentales riches sans intervention lourde. Néanmoins, cette stabilité avait pour contrepartie que je n’ai pas été directement impliqué dans la résolution des pannes complexes ou dans le reconditionnement complet de la manipulation, ce qui a limité ma formation sur les aspects de maintenance approfondie du dispositif.

En revanche, peu avant la fin de la thèse de Léa et au début de ma troisième année, nous avons observé une chute significative du nombre d’atomes capturés. Sous la supervision d’I. Bouchoule, une intervention lourde a alors été décidée : nous avons cassé le vide pour diagnostiquer le problème. Il s’est avéré que les connecteurs du dispenser étaient endommagés. L’opération a été mise à profit pour installer un nouveau dispenser et remplacer la puce atomique.

Cette opération a mobilisé plusieurs personnes du laboratoire et de ses partenaires : S. Bouchoule (C2N) et Anne [Nom complet à préciser] ont participé à la manipulation et à la pose de la puce, tandis que j’ai pu assister à l’étuvage de l’enceinte à vide avec F. Nogrette. Après cette intervention, j’ai suivi avec I. Bouchoule le réajustement progressif de la séquence de refroidissement : alignement des faisceaux, réglages de la mélasse, optimisation du chargement dans le piège DC, puis dans le guide.

Cet épisode m’a permis de me confronter plus directement aux paramètres critiques du cycle d’évaporation et à la reprise d’une séquence complète. Toutefois, le départ de Léa, qui maîtrisait tous les aspects de la manipulation, a marqué une rupture importante dans la continuité des savoir-faire pratiques liés à cette expérience.


\begin{center}
	({fig:puce} — Schéma de la puce atomique avec fils U, Z, AC, D et D'.)
\end{center}
\begin{center}
	({fig:imagerieFinale} — Schéma optique du système d’imagerie final)
\end{center}
\begin{center}
	[{fig:nuageDC} — Image du gaz dans le piège DC après évaporation]
\end{center}
\begin{center}
	[{fig:gaz1D} — Image typique d’un gaz dans le régime 1D]
\end{center}



\subsection{Génération de potentiels modulés}
%\begin{itemize}
%    \item Courants modulés pour créer des pièges harmoniques ou quartiques.
%    \item Découplage transverse/longitudinal.
%\end{itemize}

\paragraph{Champ des micro-fils.}
Puisque que $m_F = 2 $, (état assuré par pompage optique), le potentiel magnétique $-\vec{\mu} \vec{B}(\vec{r}) $ (avec moment dipolaire magnétique alors $\vec{\mu}$ et le champs magnetque totale que resente les atomes$\vec{B}(\vec{r})$) est proportionnel à $\vert \vec{B}(\vec{r}) \vert$  de sorte que les atomes, en état low-field seeking, sont attirés vers les régions de champ magnétique minimal. Les micro-fils, alignés selon l’axe horizontal $\vec{e}_x$, sont parcourus par des courants alternatifs $\pm I$ (déphasés) produisant le champ magnétique de confinement : un fil central parcouru par un courant \( I \), et deux fils latéraux par des courants opposés \(-I\). 

\paragraph{Champ de biais.}
Un champ de biais transverse $\vec{B}_{\mathrm{biais}} = {B}_{\mathrm{biais}} \, \vec{e}_y$ , avec l'axe verticale par $\vec{e}_y$ , est appliqué afin de régler la distance des atomes par rapport aux micro-fils. En notant $\vec{e}_z$ l’axe horizontal perpendiculaire à $\vec{e}_x$ et $\vec{e}_y$ l’annulation du champ total a lieu en
%Dans cette configuration, un champ de biais transverse est appliqué pour ajuster la distance des atomes au-dessus des micro-fils. Pour y avoir une idée notons l'axe verticale par $\vec{e}_y$, et $\vec{B}_{biais} = {B}_{biais} \, \vec{e}_y$. Alors en notan $\vec{e}_z$ l'axe hortisontale perpetdiculaire à $\vec{e}_x$ et $\vec{e}_y$, le champs totale s'anume en ​
  %permet ainsi de positionner précisément le minimum du potentiel à une hauteur
$z_0 = \mu_0 I / (2 \pi {B}_{\mathrm{biais}} ) $ avec $\mu_0$ la perméabilité du vide . La modulation de ${B}_{\mathrm{biais}}$ permet de déplacer le point où le champ total s’annule, ce qui permet de positionner précisément le minimum du potentiel à une distance $d$ du plan des fils. 

\paragraph{Champ d’Ioffe.}
Afin d’éviter les pertes de Majorana liées à la présence d’un champ nul, un champ longitudinal $B_0 \, \vec{e}_x$ est ajouté, garantissant que le minimum de champ reste non nul.%.Un champ longitudinal (selon $\vec{e}_x$) $B_0$ est ajouté afin que ce minimum ne corresponde pas à un champ nul, ce qui supprime les pertes de Majorana dues aux inversions de spin au voisinage d’un zéro de champ. 

%L’intérêt de ces pièges est que les atomes peuvent être confinés très près des micro-fils — ici à $ d = 15\, \mu m$ , soit l’espacement entre deux fils — ce qui maximise le gradient de champ et donc la fréquence de piégeage transverse

\paragraph{Fréquence de piégeage transverse.}
Dans la configuration étudiée, les atomes sont confinés à $ d = 15\, \mu m$ au-dessus de la puce, soit l’espacement entre deux micro-fils. Cette faible distance maximise le gradient de champ et donc la fréquence de piégeage transverse, qui s’écrit
\begin{eqnarray*}
	\omega_\perp^{(0)} =  \sqrt{\frac{\mu_B}{mB_0}} \frac{\mu_0 I }{2\pi d^2} 
\end{eqnarray*}
avec $\mu_B$ le magnéton de Bohr, $m$ la masse atomique et $\mu_0$ la perméabilité du vide.
%Pour éviter que les atomes ne perçoivent les rugosités magnétiques dues aux défauts des conducteurs, on fait circuler dans les fils un courant alternatif à haute fréquence ($\sim 400\,KHz$) : le potentiel est alors moyenné temporellement, produisant un confinement plus lisse. À $15\, \mu m$ au-dessus de la puce, le profil de champ est localement harmonique, et la fréquence de piégeage transverse devient

\paragraph{Rugosité et suppression par modulation}
Les imperfections géométriques des micro-fils engendrent des fluctuations parasites du champ magnétique le long du guide, créant une rugosité du potentiel. Pour la supprimer, les courants sont modulés à haute fréquence ($\sim 400\,KHz$), bien au-delà des fréquences de piégeage. Dans ce régime, les atomes ne perçoivent que le potentiel moyenné temporellement, où la composante parasite longitudinale est fortement réduite. Le confinement effectif reste harmonique, avec une fréquence transverse donnée par
\begin{eqnarray*}
	\omega_\perp = \frac{\omega_\perp^{(0)}}{\sqrt{2}}.		
\end{eqnarray*}




%\paragraph{Découplage des confinements transverses et longitudinaux.}
%Les courants qui parcourent les fils D, D', d, d' sons selon $\vec{e}_u$ donc les chanps induit sont selon $\vec{e}_x$ noté $B_\parallel^x$ et $\vec{e}_v$ (axex normale à la puce) , noté $B_\parallel^v$. Si les champs selon $\vec{e}_x$ est négligeable devant $B_0$ alors la moyenne de pottenstelle presente une partie transverce et longitudinale decouplés : $\braket{V} = V_\perp ( y , z ) + V_\parallel(x) $.
\paragraph{Découplage des confinements transverses et longitudinaux.}
Les courants qui parcourent les fils $D$, $D'$, $d$ et $d'$ sont orientés selon $\vec{e}_u$. 
Les champs magnétiques induits possèdent alors une composante selon $\vec{e}_x$, notée $B_\parallel^x$, et une composante selon $\vec{e}_v$ (axe normal à la puce), notée $B_\parallel^v$. 
Si le champ selon $\vec{e}_x$ est négligeable devant $B_0$, alors le potentiel moyen se sépare en une partie transverse et une partie longitudinale découplées : 
\(
\braket{V} = V_\perp(y,z) + V_\parallel(x) .
\)


\paragraph{Potentiel longitudinal harmonique.}
Dans la configuration où seuls les fils $D$ et $D'$ sont utilisés, le potentiel longitudinal peut, à l’ordre 2 en $x$, être considéré comme harmonique :
\begin{eqnarray*}
	V_\parallel (x) = V_0 + \frac{1}{2} m \omega_\parallel^2 x^2 ,
\end{eqnarray*}
On note  $2L=1.89 \,mm$ est la distance séparant les fils $D$ et $D'$. Les courants circulant dans ces deux fils sont identiques et notés $I_D = I_{D'}$. Si la condition $B_0 \gg \mu_0 I_D d /(\pi L)^2 $ est vérifiée, alors le terme constant du potentiel vaut approximativement $V_0 \simeq \mu_B B_0$.

\medskip

La pulsation longitudinale totale $\omega_\parallel$ se décompose en deux contributions : (i) une pulsation $\omega_\parallel^x = \sqrt{\frac{6\, d \, \mu_B \, \mu_0 \,I_D }{\pi \, L^4 \, m}}$ induite par le champ longitudinal $B_\parallel^x$ et (ii) une pulsation $\omega_\parallel^v = \sqrt{\frac{\mu_B }{m \, B_0}}\frac{\mu_0 \, I_D }{\pi \, L^2}$ liée au champ  $B_\parallel^v$. Pour des courants $I>1A$ , on a $\omega_\parallel^v \gg \omega_\parallel^x$, et ainsi :  
\begin{eqnarray*}
	\omega_\parallel \propto \frac{I_D}{\sqrt{B_0} L^2}.
\end{eqnarray*} 
La fréquence longitudinale est donc réglée expérimentalement en ajustant $I_D$.

\medskip

Avec les dimensions caractéristiques de la puce et des fils, il est possible d’atteindre des confinements longitudinaux de fréquence $f_\parallel = \omega_\parallel/ 2 \pi$ allant jusqu’à $\sim 150 \, H_z$, la limite étant imposée par le chauffage des fils pour $I_D \leq =4 \, A$.
 
 \medskip
 
 \subparagraph{Mesure de la fréquence transverse et longitudinale}
Pour la caractérisation, la pulsation transverse $\omega_\perp$ a été mesurée par la méthode du mode de respiration transverse \cite{Kagan1996}, tandis que $\omega_\parallel$ a été obtenue à partir des oscillations dipolaires longitudinales. Les détails expérimentaux de ces méthodes figurent dans le manuscrit de thèse de Léa Dubois \cite{L.Dubois2024}, p. 73 et p. 78.

\medskip

 \paragraph{Potentiel longitudinal quartic.}
 Si on ajoute du courand  dans les fils $d$ et $d'$.  Alors on peux avoir un potentiel non gégligeable à l'ordre 4 . Pour simmplifier, les courants dans ces fils $I_d$ et $I_{d'}$ sont identique. et le potentiel s'écrit : 
 \paragraph{Potentiel longitudinal quartique.}
Si l’on ajoute un courant dans les fils $d$ et $d'$, on peut générer un potentiel longitudinal comportant un terme significatif à l’ordre 4 en $x$ . Pour simplifier, on suppose $I_d=I_{d'}$. On obtient alors : 
 \begin{eqnarray*}
 	V_\parallel(x) \, = \, \mu_B B_0  & + & 	 \frac{\mu_B \, \mu_0}{\pi} d  \left [ \frac{I_D}{L^2} + \frac{I_d}{l^2} + 3 \left ( \frac{I_D}{L^4} + \frac{I_d}{l^4} \right ) x^2  +  5 \left ( \frac{I_D}{L^6} + \frac{I_d}{l^6} \right ) x^4 \right ] \\
 	& + & \frac{\mu_B}{B_0} \left ( \frac{\mu_0}{\pi} \right )^2  \left [ \left ( \frac{I_D}{L^2} + \frac{I_d}{l^2} \right ) x^2  + 2 \left ( \frac{I_D}{L^2} + \frac{I_d}{l^2} \right )\left ( \frac{I_D}{L^4} + \frac{I_d}{l^4} \right ) x^4 \right ].
 \end{eqnarray*}
 
 En ajustant $I_D$ et $i_d$, on peut réaliser par exemple un double puits \cite{Schemmer2019}, ou bien supprimer le terme quadratique $x^2$ afin d’obtenir un potentiel quartique pur :
\begin{eqnarray*}
	V_\parallel(x) = a_0 + a_4 x^4 	
\end{eqnarray*}
comme on le fais dans \cite{Dubois2025}.

En pratique, la puce présente des dimensions finies et n’est pas parfaitement symétrique. Un calcul plus précis, prenant en compte la géométrie exacte (disposition et épaisseur des fils), est présenté en annexe de la thèse de Thibault Jacqmin \cite{???}, p. 151. Cela impose un ajustement fin et asymétrique des courants $I_D$, $I_{D'}$, $I_d$ et $I_{d'}$.


On ajuste les courant $I_D$ et $i_d$ pour par exemple fais des douple puit \cite{Schemmer2019} ou en supriment le terme en $x^2$  d’obtenir un potentiel longitudinal quartique de la forme $V_\parallel(x) = a_0 + a_4 x^4$ \cite{Dubois2025}.\\

En réalité la puce presente des dimention finie, Un calcul plus précis prenant en compte la géométrie exacte des fils (disposition sur la
puce, épaisseur finie) se trouve en appendice de la thèse de Thibault Jacqmin [112] , page 151. De plus la pude n'est pas pardetement symetrique donc on doit ajuster les courant $I_D$, $I_{D'}$, $I_d$ et $I_{d'}$.


\paragraph{Caractérisation des potentiels longitudinal et transverse.}
Pour atteindre le régime unidimensionnel, les confinements doivent être fortement anisotropes : un piégeage transverse très fort et un piégeage longitudinal faible. La condition \(\mu, k_B T \ll \hbar \omega_\perp\) garantit le gel des degrés de liberté transverses.

\medskip

Cette configuration est particulièrement adaptée pour obtenir des profils de densité homogènes, nécessaires à certaines expériences de transport. Le transfert des atomes du piège harmonique vers le piège quartique est réalisé de manière \emph{diabatique} (changement rapide du potentiel), car un transfert adiabatique entraîne des pertes importantes.

\paragraph{Caractérisation des potentiels longitudinal et transverse.}
Pour atteindre le régime unidimensionnel, les potentiels de piégeage doivent être très asymétriques : un confinement transverse fort et un confinement longitudinal faible. La fréquence transverse \(\omega_\perp\) doit être suffisamment élevée pour geler les degrés de liberté dans cette direction, avec la condition \(\mu, k_B T \ll \hbar \omega_\perp\).

%\paragraph{Potentiel longitudinal}
%
%Le confinement longitudinal est produit par des courants continus ou modulés dans certains fils. Dans certains protocoles spécifiques, on utilise un potentiel quartique \( V_\parallel(x) = c_4 x^4 \). Le système reste dans le régime 1D tant que la longueur caractéristique longitudinale reste beaucoup plus grande que la transverse.
%
%\paragraph{Potentiel transverse}
%
%Le confinement transverse est réalisé à l’aide de trois micro-fils parallèles situés sur la puce : un fil central parcouru par un courant \( I \), et deux fils latéraux par des courants opposés \(-I\). Cette configuration crée un piège transverse harmonique avec une fréquence \(\omega_\perp\) contrôlable par la valeur du champ \( B_0 \) et le courant. Les atomes sont piégés à environ \( d = 15~\mu\text{m} \) au-dessus de la puce. La fréquence maximale accessible expérimentalement est de l’ordre de \( \sim 100~\text{kHz} \).
%
%\paragraph{Effet de rugosité et suppression par modulation}
%
%La rugosité des micro-fils induit des fluctuations parasites du champ magnétique le long du guide. Pour supprimer cet effet, les courants sont modulés à haute fréquence (environ 400~kHz). Grâce à cette modulation rapide, les atomes ne ressentent que le potentiel moyen, dans lequel la composante parasite longitudinale du champ s’annule. Ce procédé permet d’obtenir un potentiel transverse régulier et stable, avec une fréquence efficace \[ f_\perp = \frac{f_\perp^{(0)}}{\sqrt{2}}. \]
%
%\paragraph{Découplage des confinements transverse et longitudinal.}
%Dans notre dispositif, le confinement transverse est assuré par les micro-fils modulés, tandis que le confinement longitudinal est généré par quatre fils extérieurs (D, D', d, d'). L’analyse du potentiel magnétique moyen montre que, sous l’hypothèse d’un champ de bobine homogène et dominant, les contributions transverse et longitudinale du potentiel sont découplées. Cette propriété est cruciale pour nos expériences : elle permet de modifier la géométrie du potentiel longitudinal sans perturber le confinement transverse, facilitant ainsi l’exploration de différentes configurations dynamiques.
%
%\paragraph{Piégeage longitudinal harmonique.}
%Un piège longitudinal harmonique est réalisé en appliquant des courants égaux dans les fils D et D', disposés de manière symétrique. Le champ magnétique longitudinal produit conduit à un potentiel quadratique local :
%\[
%V_\parallel(x) = V_0 + \frac{1}{2} m \omega_\parallel^2 x^2,
%\]
%avec une fréquence $\omega_\parallel$ contrôlée par le courant et la géométrie de la puce. En pratique, des fréquences jusqu’à 150 Hz sont atteintes pour des courants de 4 A. Une correction peut être nécessaire pour prendre en compte un champ magnétique résiduel $B_{0v}$, responsable d’un déplacement du centre du nuage atomique.
%
%\paragraph{Piégeage longitudinal quartique.}
%L’ajout de deux fils supplémentaires (d et d') permet de modifier la forme du potentiel longitudinal jusqu’à l’ordre 4. En ajustant les courants dans les quatre fils, on peut annuler le terme quadratique et obtenir un potentiel quartique :
%\[
%V_\parallel(x) = a_0 + a_4 x^4.
%\]
%Cette configuration est particulièrement adaptée pour générer des profils de densité homogènes, comme requis dans certaines expériences de transport. Le transfert des atomes du piège harmonique vers le piège quartique est réalisé de manière diabatique (changement rapide du potentiel), car un transfert adiabatique entraînait des pertes importantes.



\section{Sélection spatiale avec DMD}
\subsection{Motivation et principe}
{\color{blue}
\begin{itemize}
    \item Besoin de préparer des tranches homogènes.
    \item Intérêt dans les protocoles hors équilibre.
\end{itemize}
}

\paragraph{Objectif du dispositif de sélection}

L’outil de sélection spatiale a été conçu pour permettre une action locale sur le gaz atomique. Il présente deux objectifs principaux. D’une part, il permet de mesurer la distribution de rapidité localement résolue, en sélectionnant une tranche du gaz avant de la libérer et de suivre son expansion. D’autre part, il offre la possibilité de créer des situations hors équilibre en retirant une partie du gaz à l’équilibre, ce qui perturbe la configuration initiale et initie une dynamique.

\paragraph{Intérêt pour les protocoles hors équilibre}

Ce dispositif permet ainsi de générer des protocoles analogues à des configurations classiques comme le pendule de Newton, ou de sonder directement la dynamique d’un gaz de Lieb-Liniger dans des conditions contrôlées. Il constitue une brique essentielle pour les expériences de dynamique et de transport quantique.


\subsection{Mise en place technique (initiée par Léa Dubois)}

{\color{blue}
\begin{itemize}
    \item Dispositif optique de projection.
    \item Contrôle numérique des motifs.
    \item Calibration et stabilité.
\end{itemize}
}

\paragraph{Principe de sélection par pression de radiation}

La sélection repose sur l’illumination d’une zone définie du gaz avec un faisceau quasi-résonant avec la transition cyclique \( F=2 \rightarrow F'=3 \) de la ligne D2 du rubidium. Les atomes subissent une pression de radiation due aux cycles absorption/émission spontanée, ce qui les pousse hors du piège ou les amène dans un état non piégé.

\paragraph{Façonnage spatial du faisceau}

La sélection doit être spatialement résolue. Le profil d’intensité dans le plan des atomes est de type binaire :
\[
I(x) = 
\begin{cases}
0 & \text{si } x \in [x_1, x_2] \\
I_0 & \text{sinon}
\end{cases}
\]
ce qui permet de préserver ou d’éjecter les atomes selon leur position longitudinale.

\paragraph{Utilisation du DMD}

Pour générer ce profil, un DMD (Digital Micromirror Device) est utilisé. Il s’agit d’une matrice de \(1024 \times 768\) micro-miroirs orientables individuellement (±12°). En inclinant ces miroirs, on contrôle localement la réflexion de la lumière. L’image du DMD est projetée directement sur le plan des atomes, en imagerie directe.

\paragraph{Avantages du DMD}

Le DMD permet une reconfiguration rapide et programmable du motif de lumière. Cette technologie est largement utilisée dans les expériences d’atomes froids pour produire des potentiels structurés, homogénéiser un faisceau ou adresser localement les atomes.

\paragraph{Alternatives possibles}

Il est possible, en théorie, d’atteindre un effet similaire par un transfert cohérent des atomes vers un état anti-piégé via un pulse micro-onde ou une transition Raman. Cependant, la méthode par pression de radiation est plus simple à mettre en œuvre et adaptée à nos objectifs expérimentaux.

\paragraph{Principe de l’expulsion par pression de radiation}

Un atome illuminé par un faisceau proche de la résonance peut être expulsé du piège soit par transition vers un état anti-piégé, soit par effet de pression de radiation. Cette dernière génère une accélération suffisante pour fournir une énergie cinétique supérieure à la profondeur du puits magnétique. Le nombre de photons diffusés nécessaire peut être estimé à partir de la conservation de l’impulsion : une vingtaine de photons suffisent typiquement à extraire un atome du piège dans nos conditions.

\paragraph{Modèle de diffusion et estimation du seuil}

Le taux de diffusion de photons est modélisé à l’aide d’un taux \(\Gamma_{\mathrm{sc}}\), dépendant de l’intensité \(I\), de l’intensité de saturation \(I_{\mathrm{sat}}\), d’un paramètre \(\alpha\) (lié à la polarisation et au champ magnétique) et du désaccord \(\delta\). À résonance, et pour un temps d’illumination \(\tau_p\), on peut estimer le nombre total de photons diffusés par atome par \(N_{\mathrm{sc}} = \tau_p \Gamma_{\mathrm{sc}}\).

\paragraph{Mesures expérimentales de la puissance nécessaire}

La puissance minimale nécessaire pour éjecter tous les atomes d’une zone illuminée est déterminée en fixant un temps d’illumination donné, puis en variant l’intensité du faisceau. L’analyse est réalisée après un délai d’attente de \(\sim 10\) ms, pour s’assurer que seuls les atomes encore piégés soient détectés. Il est observé que 99$\%$ des atomes sont retirés à partir d’un rapport \(I/I_{\mathrm{sat}} \simeq 0.12\).

\paragraph{Mesures de photons diffusés par fluorescence}

La quantité de photons diffusés est également mesurée par l’analyse du signal de fluorescence capté par la caméra. En calibrant le rapport entre photons détectés et photons diffusés (en tenant compte de l’efficacité optique du système), le nombre moyen de photons nécessaires pour éjecter un atome est confirmé expérimentalement autour de 20. Un ajustement du modèle de diffusion permet d’estimer le paramètre \(\alpha \simeq 0.4\).

\paragraph{Saturation et effets Doppler}

À fort temps d’illumination (\(\tau_p > 150\,\mu\)s), une saturation du nombre de photons diffusés est observée, interprétée comme un effet géométrique : les atomes accélérés atteignent physiquement la puce atomique et cessent de contribuer au signal. Une correction Doppler peut être introduite dans le modèle, mais reste négligeable (\(< 5\%\)) dans les régimes expérimentaux utilisés.

\paragraph{Limitations expérimentales de la sélection}

Plusieurs effets peuvent limiter l'efficacité ou la propreté de la sélection :
\begin{itemize}
    \item La diffraction liée à la taille finie de l’objectif entraîne un flou de l’ordre de \(1{-}2\,\mu\)m au bord des zones éclairées.
    \item Une diffusion parasite par la puce peut se produire à forte intensité si tout le DMD est illuminé ; cela est évité en réduisant la taille transverse du faisceau à quelques micro-miroirs seulement.
    \item Des inhomogénéités d’éclairement dues à la gaussienne du faisceau et au speckle peuvent conduire à une sur-illumination de certaines zones. Un effort a été fait pour homogénéiser l’intensité en sortie de fibre.
    \item La réabsorption des photons diffusés pourrait entraîner un échauffement du gaz restant. Un désaccord en fréquence de 15 MHz a été testé pour éviter ce phénomène, sans effet visible sur la température du gaz.
\end{itemize}

\paragraph{Mesures de l’impact sur le gaz restant}

La température du gaz sélectionné est comparée avant et après sélection via l’analyse des fluctuations de densité après temps de vol. Aucun changement significatif de température ni d’élargissement n’a été observé. Ces résultats suggèrent que, dans les conditions expérimentales utilisées, la sélection ne perturbe pas significativement les atomes restants.




\subsection{Utilisation dans les protocoles}

{\color{blue}
\begin{itemize}
    \item Formes utilisées : boîtes, barrières, coupures.
    \item Préparation initiale contrôlée du gaz.
    \item Exemples de protocoles expérimentaux utilisant le DMD
\end{itemize}
}

\paragraph{Sélection locale et mesure de rapidité}

En sélectionnant une tranche du gaz, on peut ensuite couper le confinement longitudinal et laisser cette tranche s’étendre. Le profil de densité asymptotique obtenu après un long temps d’expansion est proportionnel à la distribution de rapidité locale du gaz initial. Ce protocole permet ainsi une mesure résolue de \(\rho(x,t \to \infty) \sim \rho(v)\).

\paragraph{Génération d’états hors équilibre}

La sélection permet également de créer des discontinuités dans le profil de densité, et donc d’initier une dynamique hors équilibre. Par exemple, on peut ne conserver que deux paquets séparés de gaz, qui vont alors osciller l’un vers l’autre. Cette configuration est analogue à un pendule de Newton quantique.

\paragraph{Formes utilisées}

Les motifs projetés par le DMD peuvent prendre différentes formes : boîtes, barrières, coupures, etc. Cette flexibilité rend l’outil extrêmement précieux pour explorer diverses configurations initiales et protocoles dynamiques.

\paragraph{Contrôle logiciel du DMD}

Le pilotage du DMD repose sur l’utilisation d’un module intégré fourni par Vialux (V7001-SuperSpeed), qui comprend les bibliothèques logicielles ALP-4. Plusieurs configurations du DMD peuvent être chargées en mémoire au début de chaque cycle expérimental, puis sélectionnées en cours de séquence à l’aide d’un signal digital. Le temps de commutation des miroirs est inférieur à \(30\,\mu\mathrm{s}\), ce qui est compatible avec les protocoles étudiés.

\paragraph{Partage du faisceau avec la voie d’imagerie}

Le faisceau utilisé pour la sélection spatiale est prélevé à partir du faisceau sonde déjà accordé sur la transition \(F=2 \rightarrow F'=3\) de la raie D2. Le partage est réalisé à l’aide d’un cube séparateur de polarisation placé en aval d’une lame demi-onde, permettant de contrôler la puissance injectée dans la fibre optique. Ce choix simplifie la mise en œuvre en évitant d’ajouter une source laser supplémentaire.

\paragraph{Blocage du faisceau de sélection}

Deux systèmes permettent de couper le faisceau de sélection pendant le cycle expérimental :
\begin{itemize}
    \item un cache mécanique (type électro-aimant), utilisé pour un blocage longue durée ;
    \item un modulateur acousto-optique (AOM), permettant de produire des impulsions brèves de quelques dizaines de \(\mu\mathrm{s}\), en amont du séparateur.
\end{itemize}
Pour garantir que le faisceau ne perturbe pas l’imagerie, le cache mécanique reste fermé pendant l’utilisation du faisceau sonde.

\paragraph{Montage optique de projection}

Le faisceau façonné par le DMD est projeté dans le plan des atomes à l’aide d’un système optique permettant de sélectionner l’ordre 0 de diffraction. L’ensemble des optiques est dimensionné (diamètre \(50\,\mathrm{mm}\)) pour limiter la diffraction. L’alignement est effectué en superposant le faisceau de sélection à la voie d’imagerie.

\paragraph{Grandissement et champ couvert}

Le montage permet de couvrir une zone de l’ordre de \(600\,\mu\mathrm{m}\) dans le plan des atomes, soit plus que la longueur typique d’un nuage (\(\sim 400\,\mu\mathrm{m}\) pour \(f_{\parallel}=5\,\mathrm{Hz}\)). Le grandissement est déterminé par les focales utilisées : une focale \(f_1 = 750\,\mathrm{mm}\) du côté du DMD, et \(f = 32\,\mathrm{mm}\) pour l’objectif côté atomes, donnant \(G = f/f_1 \approx 0.043\).

\paragraph{Visualisation et interface}

Le contrôle du DMD s’effectue via une interface graphique permettant de prévisualiser les configurations de miroirs. Une capture d’écran de cette interface est présentée dans la Fig.~\ref{fig:dmd_interface}, où la zone active réfléchie est visualisée en rouge. Cette interface est pilotée de manière automatisée pendant le déroulement de la séquence expérimentale.


\section{Techniques d’imagerie et d’analyse}
\subsection{Imagerie par absorption}
{\color{blue}
\begin{itemize}
    \item Imagerie \textit{in situ} et après temps de vol.
    \item Résolution, limites instrumentales.
\end{itemize}
}

\paragraph{Système d’imagerie par absorption}

L’imagerie est réalisée à l’aide d’une caméra CCD à déplétion profonde, optimisée pour une grande efficacité quantique à la longueur d’onde de 780 nm. On utilise des techniques d’imagerie par absorption permettant d’extraire la densité optique \( D(x, z) \), elle-même reliée à la densité atomique 3D via la loi de Beer-Lambert. Le profil de densité linéaire \( n(x) \) est obtenu par intégration sur les directions transverses.

\paragraph{Imagerie après temps de vol}

En appliquant un champ magnétique vertical (\( B = 8\,\mathrm{G} \)), la polarisation du faisceau peut être rendue circulaire (\( \sigma^+ \)) pour adresser la transition fermée \( |F=2, m_F=2\rangle \rightarrow |F'=3, m_F'=3\rangle \). Cette configuration assure une meilleure définition de la section efficace d’absorption. Un temps de vol de quelques ms est utilisé avant l’imagerie, permettant également de décomprimer le nuage.

\paragraph{Imagerie in situ}

Sans champ magnétique, les atomes sont imagés à $7~\mu m$ de la puce, ce qui implique une double absorption du faisceau incident et réfléchi. Dans ce cas, la transition n’est pas fermée, ce qui nécessite une calibration du facteur de conversion entre la densité mesurée et la densité réelle. Un ajustement linéaire permet de relier les profils in situ aux profils obtenus après temps de vol.

\paragraph{Choix des paramètres d’imagerie}

L’intensité du faisceau sonde est choisie typiquement à \( I_0/I_{\mathrm{sat}} \approx 0.3 \) pour optimiser le rapport signal sur bruit tout en restant dans une zone de linéarité acceptable. Dans ces conditions, le nombre de photons diffusés est de l’ordre de \( N_{\mathrm{sc}} \approx 230 \) et le rayon de diffusion reste comparable à la résolution du système d’imagerie (\( \sim 2.6\,\mu \mathrm{m} \)).

\paragraph{Limites du modèle de Beer-Lambert}

La validité de la loi de Beer-Lambert repose sur une approximation à une particule. Dans le cas des gaz fortement denses ou quasi 1D, les effets collectifs, les réabsorptions et les couplages dipolaires peuvent invalider ce modèle. Pour cette raison, même pour l’imagerie in situ, un temps de vol court (\( \sim 1\,\mathrm{ms} \)) est souvent appliqué afin de diluer le gaz transversalement.

\paragraph{Défauts et instabilités expérimentales}

Plusieurs limitations instrumentales ont été identifiées :
\begin{itemize}
    \item La caméra initialement utilisée montrait des motifs parasites aléatoires ainsi qu’un offset variant au cours du temps. Le remplacement de la caméra a permis de résoudre ces problèmes.
    \item Des franges d’interférences apparaissaient lors de la division des images d’absorption, probablement dues à des effets Fabry-Pérot dans les optiques. Le désaxage du faisceau d’imagerie a permis d’en limiter l’impact.
    \item Des photons résiduels, même en l’absence de faisceau sonde, ont été détectés. Ces derniers proviennent vraisemblablement de diffusions multiples dans le système optique.
\end{itemize}

\paragraph{Conclusion}

La combinaison de l’imagerie in situ et après temps de vol, ainsi qu’une calibration soigneuse des paramètres optiques et expérimentaux, permettent d’accéder à des profils de densité fiables malgré les limites intrinsèques du système d’imagerie. Une attention particulière a été portée à la réduction des artefacts expérimentaux afin de garantir la précision des mesures.


\subsection{Analyse des profils}

{\color{blue}
\begin{itemize}
    \item Extraction des densités, tailles, températures.
    \item Distribution longitudinale.
    \item Estimation de la température par ajustement Yang-Yang (optionnel si pertinent).
\end{itemize}
}


\section{Expériences et protocoles étudiés}
Cette section peut être la plus personnelle, en précisant ton rôle à chaque fois.
\subsection{Expansion longitudinale}
\begin{itemize}
    \item Protocole d’expansion (libération longitudinale, maintien du confinement transverse).
    \item Suivi de l’évolution du profil.
    \item Analyse à différents temps d’expansion
    \item Comparaison aux modèles analytiques : solutions homothétiques, GP, asymptotiques.
\end{itemize}

\subsection{Motivation et protocole expérimental d’expansion longitudinale}

\paragraph{Motivation.}
Une partie essentielle de mon travail de thèse a consisté à sonder la distribution de rapidités résolue spatialement, ce qui constitue une information clé pour comprendre la dynamique hors équilibre d’un gaz quantique unidimensionnel. Pour accéder à cette observable, il est nécessaire de réaliser un protocole qui relie la distribution de rapidités à des profils de densité mesurables expérimentalement. L’expansion longitudinale dans le guide 1D s’impose alors comme un outil naturel : en laissant le nuage se dilater librement dans la direction longitudinale, on convertit en partie l’information contenue dans les phases et les excitations collectives du système en une dynamique de densité directement accessible par imagerie. Ce protocole permet ainsi de comparer les prédictions issues des équations effectives, comme l’équation de Gross–Pitaevskii dans différents régimes de confinement, avec des mesures expérimentales résolues spatialement.

\paragraph{Considérations physiques.}
Au-delà de son intérêt pratique, l’expansion longitudinale offre une fenêtre unique sur la physique des gaz bosoniques 1D. Elle permet d’étudier comment un système initialement confiné évolue vers un état dilué, révélant à la fois l’impact du régime transverse (TF 3D vs TF 1D) et l’influence des fluctuations de phase. Dans le régime TF 1D, ces fluctuations deviennent dominantes et se traduisent par des ondulations de densité mesurables. Leur analyse expérimentale, via le spectre de puissance, fournit un accès direct aux corrélations de phase et à la thermodynamique effective du gaz.

\paragraph{Protocole expérimental.}
Concrètement, l’expansion longitudinale est réalisée selon la séquence illustrée en Fig.~??? :
\begin{itemize}
  \item Le nuage est initialement piégé dans un potentiel magnétique caractérisé par une fréquence longitudinale $f_{\parallel} = 5.0$ ou $9.4\,\mathrm{Hz}$ selon les jeux de données, et une fréquence transverse $f_{\perp} = 2.56\,\mathrm{kHz}$.
  \item À $t=0$, le confinement longitudinal est éteint en annulant les courants $I_D=I_{D'}=0$. La coupure est réalisée sur un temps fini $t_{\parallel} = 70\,\mu\mathrm{s} \ll 1/f_{\parallel}$, ce qui évite un pic de courant parasite tout en préservant la dynamique du gaz.
  \item Le nuage se dilate librement dans la direction longitudinale pendant une durée $\tau$. Ensuite, le confinement transverse est relâché en annulant $I_{\perp}$, avec un temps de coupure $t_{\perp} = 5\,\mu\mathrm{s} \ll 1/f_{\perp}$.
  \item Une image par absorption est enfin prise après un temps de vol $t_v$. Pour l’étude des profils de densité, on utilise typiquement $t_v = 1\,\mathrm{ms}$.
\end{itemize}

%\paragraph{Découplage des confinements.}
%Comme discuté en Section~\ref{chap:...}, l’architecture expérimentale rend ce protocole particulièrement simple à mettre en œuvre. La modulation des courants transverses $I_{\perp}$ garantit que le potentiel longitudinal est découplé de celui transverse, ce qui permet un contrôle précis et indépendant des deux confinements.

\paragraph{Perspective.}
La mise en œuvre de ce protocole d’expansion longitudinale ne répond donc pas seulement à un besoin technique de mesure, mais s’inscrit dans une stratégie plus générale : relier les prédictions théoriques de la GHD et des modèles effectifs à des observables accessibles, et sonder directement l’évolution des fluctuations et des corrélations dans un système quantique 1D.

\paragraph{Équations Gross-Pitaevskii dépendantes du temps.}
La dynamique du système étudié est décrite par l’équation de Gross-Pitaevskii (GP) \eqref{chap.1:eq.GP.1} :
\begin{eqnarray*}
	i \partial_\tau\phi = \left \{ - \frac{1}{2}\Delta_{\vec{r}} + V(\vec{r}) + g_{\mathrm{3D}} N \vert \phi \vert^2 \right \} \phi,
\end{eqnarray*}
avec $g_{\mathrm{3D}} = 4 \pi a_{\mathrm{3D}}$ et en présence d’un potentiel externe (voir \eqref{} et \eqref{}) :
\begin{eqnarray*}
	V(\vec{r}) = V_\perp(\vec{r}_\perp) + V_\parallel(x), 
	\qquad 
	V_\perp(\vec{r}_\perp) = \tfrac{1}{2} \, \omega_\perp^2 \, \vec{r}_\perp^2, 
	\qquad 
	V_\parallel(x) = \tfrac{1}{2} \, \omega_\parallel^2 \, x^2.
\end{eqnarray*} 


\paragraph{Séparation des degrés de liberté.}
Dans un piège de type cigare, caractérisé par $\omega_\perp \gg \omega_\parallel$, la dynamique transverse se déroule sur des temps caractéristiques beaucoup plus courts que la dynamique longitudinale. On fait alors l’hypothèse d’un \emph{suivi adiabatique transverse} : l’état reste en permanence dans son état fondamental transverse. Ainsi, les degrés de liberté transverses et longitudinaux se découplent et la fonction d’onde peut se factoriser sous la forme
\begin{equation}
    \phi(r,\tau) = \psi(x,\tau)\,\Phi\!\left(\vec{r}_\perp, n(x,\tau)\right),
\end{equation}
où $\psi(x,\tau)$ décrit la dynamique longitudinale et $\Phi$ est la fonction d’onde transverse dépendant paramétriquement de la densité linéaire $n(x,\tau)$. La condition de normalisation 
\(
\int d \vec{r}_\perp \, \big|\Phi\!\left(\vec{r}_\perp, n\right)\big|^2 = 1
\)
permet de réécrire la densité linéaire définie par 
\(
n \doteq N \int d \vec{r}_\perp \, |\phi|^2
\)
sous la forme
\begin{eqnarray*}
	n(x,\tau) = N \, |\psi(x,\tau)|^2.
\end{eqnarray*}
L’équation de Gross-Pitaevskii se réécrit alors
\begin{eqnarray}
	\left( i \partial_\tau + \tfrac{1}{2} \partial_x^2 - V_\parallel(x) - \mu(n) \right) \psi = 0, 
	\qquad 
	\mu(n)\,\Phi = \left( - \tfrac{1}{2} \Delta_{\vec{r}_\perp} + V_\perp + g_{\mathrm{3D}} \, n \, \big|\Phi(\vec{r}_\perp,n)\big|^2 \right)\Phi.
\end{eqnarray}

\paragraph{Équations hydrodynamiques.}
En utilisant la transformation de Madelung 
\(
\psi(x,\tau) = \sqrt{n(x,\tau)} \, e^{i \vartheta(x,\tau)},
\)
et en introduisant la vitesse $u = \partial_x \vartheta$, on obtient les équations hydrodynamiques associées :
\begin{eqnarray}\label{chap:5:eq.hydro.1}
	\left\{
	\begin{array}{rcl}
		\partial_\tau n + \partial_x ( n u )	 & = & 0, \\[0.3em]
		\partial_\tau u + \partial_x \left( \tfrac{u^2}{2} + V_\parallel(x) + \mu(n) + Q(n) \right) & = & 0,
	\end{array} 
	\right.
\end{eqnarray}
où le terme de pression quantique est donné par
\(
Q(n) = - \frac{1}{2} \, \frac{\partial_x^2 \sqrt{n}}{\sqrt{n}}.
\)
Ces équations sont équivalentes aux deux premières de \eqref{chap:3:eq:hydro.1}, en tenant compte de la relation thermodynamique $dP = n \, d\mu$ et en négligeant le terme de pression quantique $Q(n)$.

\medskip

Pour notre protocole, pour $\tau < 0$ le système est à l’équilibre, avec la condition
\(
\mu(n) + V_\parallel(x) = \mu\bigl(n(x=0)\bigr).
\)
Pour $\tau \geq 0$, le potentiel longitudinal est éteint : $V_\parallel(x) = 0$.

\medskip

\paragraph{Solutions analytiques homothétique.}
Si $n$ est solution des equation hydrodynamique \eqref{chap:5:eq.hydro.1} , pour $\tau \geq 0$. On fais l'hypothèse que la densité linéaire suit une forme homothétique
\begin{eqnarray}
	n(x,\tau) = \frac{1}{\lambda(\tau)} n_0 \left ( \frac{x}{\lambda(\tau)} \right ) ,	
\end{eqnarray}
avec $n_0$ le profil de densité à $\tau = 0 $ et $\lambda(\tau)$ le facter d'echelle à une temps d'expension $\tau$. Avec les containtes $\lambda(0) = 1$ et $\lambda'(0) = 0$ et $N = \int dx \, n(x , \tau ) $. En injectant dans \eqref{chap:5:eq.hydro.1} il vient que 
\begin{eqnarray}\label{chap:5:eq.hydro.2}
	\left\{
	\begin{array}{rcl}
		u(x, \tau ) & = & \displaystyle \frac{\dot\lambda(\tau)}{\lambda(\tau)} x , \\[0.3em]
		\partial_x \mu ( n ( x , \tau ))  & = & - \displaystyle \frac{\ddot\lambda(\tau)}{\lambda(\tau)} x,
	\end{array} 
	\right.
\end{eqnarray}
(car \(\partial_\tau u=(\ddot\lambda/\lambda-\dot\lambda^2/\lambda^2)x\) et \(v\partial_x v=(\dot\lambda/\lambda)^2 x\), leur somme donne \((\ddot\lambda/\lambda)x\)) et initialement $\mu( n_0 ( x ) ) = \mu( n_0 ( x = 0  ) ) - \frac{1}{2} \omega_\parallel^2 x^2 $.

\medskip

Calculons maintenant \(\partial_x\mu(n(x))\). D'abord
\[
\partial_x n(x)=\frac{1}{\lambda^2}\,n_0'\!\Big(\frac{x}{\lambda}\Big).
\]
À l'équilibre \(\mu\big(n_0(y)\big)=\mu_0-\tfrac12 \omega_\parallel^2 y^2\), d'où
\[
\mu'(n_0(y))\,n_0'(y)=-\omega_\parallel^2 y
\quad\Rightarrow\quad
n_0'(y)=-\frac{\omega_\parallel^2\,y}{\mu'(n_0(y))}.
\]
En prenant \(y=x/\lambda\) on obtient
\[
n_0'\!\Big(\frac{x}{\lambda}\Big)
= -\frac{\omega_\parallel^2}{\lambda}\,\frac{x}{\mu'\big(n_0(x/\lambda)\big)}.
\]
Donc
\[
\partial_x n(x) = -\frac{\omega_\parallel^2\,x}{\lambda^3}\;
\frac{1}{\mu'\big(n_0(x/\lambda)\big)}.
\]
Puis
\[
\partial_x\mu(n(x))=\mu'\big(n(x)\big)\,\partial_x n(x)
= -\frac{m\omega_\parallel^2\,x}{\lambda^3}\;
\frac{\mu'\big(n(x)\big)}{\mu'\big(n_0(x/\lambda)\big)}.
\]
Or \(n_0(x/\lambda)=\lambda\,n(x)\), donc on définit
\[
f(\lambda)\equiv\frac{\mu'(n)}{\mu'(\lambda n)}.
\]
On obtient finalement
\[
\partial_x\mu(n(x)) = -\frac{\omega_\parallel^2}{\lambda^3}\,f(\lambda)\,x.
\]

%On souhaite calculer $\partial_x \mu\bigl(n(x,\tau)\bigr)$ en utilisant la règle de la chaîne et la forme homothétique. On a
%\[
%	\partial_x \mu\bigl(n(x)\bigr) 
%	= \mu'(n(x)) \, \partial_x n(x) 
%	= \frac{1}{\lambda^2} \, \mu'(n(x)) \, \partial_x n_0\!\left(\tfrac{x}{\lambda}\right),
%\]
%où le dernier terme s’écrit
%\[
%	\left. \frac{\partial n_0}{\partial x} \right|_{x/\lambda} 
%	= \left. \frac{\partial n_0}{\partial \mu} \right|_{\mu(n_0(x/\lambda))} 
%	\left. \frac{\partial \mu}{\partial x} \right|_{n_0(x/\lambda)} .
%\]
%On utilise alors 
%\[
%	\left. \frac{\partial \mu}{\partial x} \right|_{n_0(x/\lambda)} = - \frac{\omega_\parallel^2}{\lambda^2} \, x,
%	\qquad 
%	\left. \frac{\partial n_0}{\partial \mu} \right|_{\mu(n_0(x/\lambda))} 
%	= \left. \frac{\partial n}{\partial \mu} \right|_{\mu(\lambda n)} .
%\]
%Il vient donc, en utilisant de plus la deuxième équation de 
En remplaçant dans la deuxième d'Euler \eqref{chap:5:eq.hydro.2} et en simplifiant  \(x\),
\begin{eqnarray}\label{chap:5:eq.hydro.3}
	\frac{\ddot\lambda}{\lambda}  
	 =  \frac{\omega_\parallel^2}{\lambda^3} \, f(\lambda)  .
\end{eqnarray}

\paragraph{Proposition.}
Si le facteur
\(
f(\lambda)
\)
est bien défini indépendamment de \(n>0\) (ce qui est le cas pour les solutions homothétiques),
alors \(f\) est une loi de puissance.

\paragraph{Preuve.}
Posons \(g(n) = \mu'(n)>0\) ou \(<0\) (\ie $\mu$ strictement monotone). La définition de \(f\) équivaut à l’existence d’une fonction
\(\chi(\lambda)=1/f(\lambda)\) telle que
\[
g(\lambda n)=\chi(\lambda)\,g(n)\qquad(\forall\,\lambda,n>0).
\]
En prenant \(n=1\), on a \(\chi(\lambda)=g(\lambda)/g(1)\).
Donc, pour tous \(a,b>0\),
\[
\chi(ab)=\frac{g(ab)}{g(1)}=\frac{\chi(a)\,g(b)}{g(1)}=g(a)\,g(b),
\]
c’est-à-dire que \(\chi\) est \emph{multiplicative}. Sous une hypothèse physique très faible
(continuité/mesurabilité ou simple localement bornée), toute fonction multiplicative sur
\(\mathbb{R}_+^\ast\) est de la forme
\[
\chi(\lambda)=\lambda^{\alpha-1}
\quad\Rightarrow\quad
f(\lambda)=\lambda^{1-\alpha}.
\]
%En réintégrant \(g=\mu'\propto n^{\alpha-1}\), on retrouve \(\mu(n)\propto n^\alpha\) pour \(\alpha\neq 0\)
%(et \(\mu(n)\propto \ln n\) pour \(\alpha=0\)).
\qed


% --- Démonstration que f(λ)=λ^{1-\alpha} et réciproque ---
\paragraph{Proposition.}
$f(\lambda) = \lambda^{1-\alpha}$ et $\mu (n) \propto n^\alpha $ sont equivalents.
%
%On définit
%\[
%f(\lambda)=\frac{\mu'(n)}{\mu'(\lambda n)},
%\]
%en supposant \(\mu\in C^1\) et \(\mu'(n)>0\) pour \(n>0\).

\paragraph{1. Si \(\mu(n)=C\,n^\alpha\) (avec \(C\neq0\)) :}
Alors \(\mu'(n)=C\alpha\,n^{\alpha-1}\). Par conséquent
\[
f(\lambda)=\frac{C\alpha\,n^{\alpha-1}}{C\alpha\,(\lambda n)^{\alpha-1}}
=\lambda^{1-\alpha}.
\]

\paragraph{2. Réciproque : si \(f(\lambda)=\lambda^{1-\alpha}\) pour tout \(\lambda>0\) (et tout \(n>0\)) :}
Posons \(g(n)=\mu'(n)\). L'hypothèse s'écrit
\[
\frac{g(n)}{g(\lambda n)}=\lambda^{1-\alpha}
\quad\Longleftrightarrow\quad
g(\lambda n)=\lambda^{\alpha-1}\,g(n),
\]
pour tout \(n>0\) et tout \(\lambda>0\).

Fixons \(n_0>0\) et définissons \(\varphi(\lambda)\equiv g(\lambda n_0)\). La relation ci-dessus donne
\[
\varphi(\lambda)=\lambda^{\alpha-1}\,\varphi(1).
\]
Autrement dit \(\varphi(\lambda)=C_1\,\lambda^{\alpha-1}\) pour une constante \(C_1=\varphi(1)=g(n_0)\). En remplaçant \(\lambda=x/n_0\) on obtient pour tout \(x>0\)
\[
g(x)=C_1\,x^{\alpha-1}.
\]
Ainsi \(g(n)=\mu'(n)=C\,n^{\alpha-1}\) avec \(C\) constant.

En intégrant (en supposant \(\alpha\neq 0\)),
%\[
%\mu(n)=\int \mu'(n)\,dn = \int C\,n^{\alpha-1}\,dn = \frac{C}{\alpha}\,n^\alpha + \text{const},
%\]
%donc 
\(\mu(n)\propto n^\alpha\). (Pour \(\alpha=0\) on obtient \(\mu'(n)=C\,n^{-1}\) et \(\mu(n)=C\ln n+\text{const}\).)

\paragraph{Remarque sur les hypothèses.}
La démonstration utilise la propriété fonctionnelle multiplicative
\(g(\lambda n)=\lambda^{\alpha-1}g(n)\). Sous une hypothèse faible de continuité (ou dérivabilité) en \(n\) cette équation force la forme de puissance \(g(n)\propto n^{\alpha-1}\). Sans régularité, des solutions pathologiques peuvent exister mais ne sont pas physiquement pertinentes dans le contexte thermodynamique.

\qed

%% --- Bref argument (à insérer) ---
%En posant \(g(n)=\mu'(n)\) et \(f(\lambda)=\dfrac{g(n)}{g(\lambda n)}=\lambda^{1-\alpha}\), on obtient
%\[
%g(\lambda n)=\lambda^{\alpha-1}g(n)\quad\forall\,\lambda,n>0,
%\]
%d'où \(g(n)=C\,n^{\alpha-1}\) et, pour \(\alpha\neq0\), \(\mu(n)=\dfrac{C}{\alpha}n^\alpha+\mathrm{const}\), i.e. \(\mu\propto n^\alpha\).
%
%
%% --- Encadré pour le cas alpha = 0 ---
%\medskip
%\noindent\textbf{Remarque (cas \(\alpha=0\)).} Si \(\alpha=0\) alors la relation fonctionnelle donne \(g(n)=\mu'(n)=C\,n^{-1}\). En intégrant on obtient
%\[
%\mu(n)=C\ln n + \mathrm{const}.
%\]
%Ce cas correspond physiquement, par exemple, au gaz isotherme idéal en 1D (ou plus généralement à une dépendance logarithmique du potentiel chimique), où la compressibilité \(\mu'(n)\propto 1/n\).
%
%--------------------------------------------
%
%avec $f(\lambda) = \frac{\mu'(n)}{\mu'(\lambda n )}$ avec  $\mu(n)$ est continue et strictement monitone (donc inversible). Puisque $f(1)= 1$ et $f(\lambda_1 \lambda_2) =  f(\lambda_1) f( \lambda_2)$ alors $f$ est une fonction de puissace $f(\lambda) = \lambda^{1-\beta}$. Ainssi une solution de l'éqtation hydrondynamique homothètique donne  
%\begin{eqnarray}
%	\mu(n) = a n^\beta + b,  	
%\end{eqnarray}
%avec $a, b$ et $\beta$ des réelles. 

\paragraph{Cas particulier.}
Dans le régime quasi-1D on utilise l'expression d'interpolation (cf. Salasnich et al.)
\[
\mu(n)=\hbar\omega_\perp\Big(\sqrt{1+4\,a_{\mathrm{3D}}\,n}-1\Big),
\]
où \(n\) est la densité linéique et \(a_{\mathrm{3D}}\) le scattering length. De cette formule on obtient deux limites asymptotiques :

\begin{itemize}
\item \emph{Régime transverse Thomas--Fermi (TF), \(4a_{\mathrm{3D}}n\gg1\).} 
Alors \(\sqrt{1+4a_{\mathrm{3D}}n}\simeq 2\sqrt{a_{\mathrm{3D}}n}\) et
\[
\mu(n)\simeq 2\hbar\omega_\perp\sqrt{a_{\mathrm{3D}}\,n},
\]
ce qui correspond à \(\mu\propto n^{1/2}\) (donc \(\alpha=\tfrac12\)). Ce régime décrit la situation où \(\mu\gg\hbar\omega_\perp\) et de nombreux niveaux transverses sont excités.
\item \emph{Régime quasi-1D (transverse fondamental), \(4a_{\mathrm{3D}}n\ll1\).} 
Alors \(\sqrt{1+4a_{\mathrm{3D}}n}\simeq 1+2a_{\mathrm{3D}}n\) et
\[
\mu(n)\simeq 2\hbar\omega_\perp\,a_{\mathrm{3D}}\,n \equiv g\,n,
\]
avec \(g=2\hbar\omega_\perp a_{\mathrm{3D}}\). Ici \(\mu\propto n\) (donc \(\alpha=1\)) ; on est proche de l'état fondamental transverse (gaussien).
\end{itemize}

Les deux formes ci-dessus sont bien les limites asymptotiques de l'expression d'interpolation donnée plus haut.

\medskip

Enfin, l'équation d'évolution du facteur d'échelle obtenue précédemment s'écrit correctement
\[
\boxed{\qquad \ddot\lambda\,\lambda^{\alpha+1}=\omega_\parallel^2 \qquad}
\]

% --- Première intégration de l'équation ---
On part de
\[
\ddot\lambda\,\lambda^{\alpha+1}=\omega_\parallel^2,
\]
et on pose \(v=\dot\lambda\). Comme \(\ddot\lambda=\dot\lambda\frac{d\dot\lambda}{d\lambda}\), on obtient
\[
\dot\lambda\frac{d\dot\lambda}{d\lambda}=\omega_\parallel^2\,\lambda^{-(\alpha+1)}.
\]

\paragraph{Cas \(\alpha\neq0\).}
Intégration par rapport à \(\lambda\) :
\[
\frac{1}{2}\dot\lambda^2
= \omega_\parallel^2\int \lambda^{-(\alpha+1)}\,d\lambda
= -\frac{\omega_\parallel^2}{\alpha}\,\lambda^{-\alpha} + C,
\]
où \(C\) est une constante d'intégration déterminée par les conditions initiales \(\lambda(0)=\lambda_0\), \(\dot\lambda(0)=\dot\lambda_0\) :
\[
C=\frac{1}{2}\dot\lambda_0^2+\frac{\omega_\parallel^2}{\alpha}\,\lambda_0^{-\alpha}.
\]
On a donc la première intégrale
\[
\boxed{\; \dot\lambda^2
= \dot\lambda_0^2 + \frac{2\omega_\parallel^2}{\alpha}\big(\lambda_0^{-\alpha}-\lambda^{-\alpha}\big)\; }.
\]
%La solution en quadrature s'écrit alors
%\[
%t-t_0=\int_{\lambda_0}^{\lambda(t)}\frac{d\lambda}{\sqrt{\,v_0^2 + \dfrac{2\omega_\parallel^2}{\alpha}\big(\lambda_0^{-\alpha}-\lambda^{-\alpha}\big)\,}}.
%\]

\paragraph{Cas \(\alpha=0\).}
L'équation devient \(\ddot\lambda\,\lambda=\omega_\parallel^2\). On obtient
%\[
%\frac{1}{2}v^2=\omega_\parallel^2\ln\lambda + C,
%\]
%avec \(C=\tfrac12 v_0^2-\omega_\parallel^2\ln\lambda_0\). D'où
\[
\boxed{\; \dot\lambda^2 = \dot\lambda_0^2 + 2\omega_\parallel^2\ln\!\big(\tfrac{\lambda}{\lambda_0}\big)\; }.
\]

%La quadrature est
%\[
%t-t_0=\int_{\lambda_0}^{\lambda(t)}\frac{d\lambda}{\sqrt{\,v_0^2 + 2\omega_\parallel^2\ln(\lambda/\lambda_0)\,}}.
%\]

%\paragraph{Remarques.}
%\begin{itemize}
%\item Ces intégrales donnent la solution implicite \(t(\lambda)\). En général on ne dispose pas d'une primitive élémentaire fermée pour \(\lambda(t)\) (sauf cas particuliers de choix des conditions initiales), mais la première intégrale ci-dessus est très utile pour l'analyse qualitative (points de retournement, énergie effective, petites oscillations).
%\item Pour les petites oscillations autour de \(\lambda=1\) on peut linéariser et retrouver la fréquence \(\omega_{\rm breath}=\sqrt{4+\beta}\,\omega_\parallel=\sqrt{3+\alpha}\,\omega_\parallel\) (avec \(\beta=\alpha-1\)).
%\end{itemize}


%\subsubsection{Comportement asymptotique du facteur d'échelle}



%\paragraph{Régime à temps longs.} 
%On considère la condition initiale
%\(
%\lambda(0)=1,\,\dot\lambda(0)=0.
%\) 
%et pour $\alpha > 0$ 
%Pour \(\tau\) très grand, on a \(\lambda^{-\alpha}\ll 1\). On a 
%\[
%\lambda(\tau) \simeq \frac{2}{\alpha}\,\omega_\parallel \tau.
%\]
%En particulier :  
%\begin{itemize}
%\item TF 1D (\(\alpha=1\)) : \(\lambda(\tau)\simeq \sqrt{2}\,\omega_\parallel \tau\),  
%\item TF 3D (\(\alpha=1/2\)) : \(\lambda(\tau)\simeq 2\,\omega_\parallel \tau\).  
%\end{itemize}
%%Ces comportements sont observés sur la Fig.~7.4(b).
%
%\paragraph{Régime à temps courts.}  
%À temps courts, on peut approximer \(\mu(x,\tau)\simeq \mu_0(x)=\mu_p-\frac{1}{2}m\omega_\parallel^2 x^2\). Le comportement initial est alors indépendant de l'équation d'état \(\mu(n)\). L'équation d'Euler sans potentiel extérieur donne
%\[
%\frac{d^2 x}{d\tau^2} \simeq \omega_\parallel^2 x \quad\Rightarrow\quad v(x,\tau)\simeq \omega_\parallel^2 x \,\tau \quad (\tau\to 0).
%\]
%En réinjectant ce profil dans l'équation de continuité et en intégrant, on obtient
%\[
%\frac{1}{\lambda(\tau)} \simeq 1 - \frac{\omega_\parallel^2 \tau^2}{2} + \mathcal{O}(\tau^4) \quad\Rightarrow\quad
%\lambda(\tau)\simeq 1 + \frac{\omega_\parallel^2 \tau^2}{2} + \mathcal{O}(\tau^4),
%\]
%ce qui correspond au comportement observé à temps courts sur la Fig.~7.4(a), identique pour les régimes TF 1D et TF 3D.
%
%
%------------------
%
%\paragraph{Régime à temps courts.}  
%Pour \(\tau \to 0\), on linéarise le facteur d'échelle autour de l'équilibre \(\lambda=1\) en posant
%\(\lambda(\tau) = 1 + \epsilon(\tau)\) avec \(|\epsilon|\ll 1\). L'équation de mouvement devient alors
%\[
%\ddot \epsilon + (1+\alpha)\,\omega_\parallel^2 \epsilon - \omega_\parallel^2 = 0,
%\]
%équivalente à un oscillateur harmonique forcé. La solution pour des conditions initiales
%\(\epsilon(0)=0\), \(\dot\epsilon(0)=0\) est
%\[
%\epsilon(\tau) \simeq \frac{\omega_\parallel^2}{1+\alpha}\left[1 - \cos\left(\sqrt{1+\alpha}\,\omega_\parallel \tau\right)\right].
%\]
%Ainsi, à temps très courts \(\tau\ll 1/\omega_\parallel\), on retrouve
%\[
%\epsilon(\tau) \simeq \frac{1}{2}\,\omega_\parallel^2 \tau^2 + \mathcal{O}(\tau^4),
%\]
%et donc
%\[
%\lambda(\tau) \simeq 1 + \frac{\omega_\parallel^2 \tau^2}{2} + \mathcal{O}(\tau^4),
%\]
%ce qui coïncide avec le comportement universel observé à temps courts pour tous les régimes TF, indépendamment de \(\alpha\) et de l'équation d'état \(\mu(n)\).
%
%----------------------------

On impose les conditions initiales
\[
\lambda(0)=1, \qquad \dot\lambda(0)=0,
\]
et l'on considère le cas \(\alpha>0\).

\paragraph{Régime à temps courts (\(\tau \ll 1/\omega_\parallel\)).}  
On linéarise autour de l'équilibre \(\lambda=1\) en posant \(\lambda(\tau)=1+\epsilon(\tau)\) avec \(|\epsilon|\ll 1\). L'équation de mouvement devient un oscillateur harmonique forcé :
\[
\ddot \epsilon + (1+\alpha)\,\omega_\parallel^2 \epsilon - \omega_\parallel^2 = 0.
\]
Pour les conditions initiales choisies, la solution à petits temps est
\[
\epsilon(\tau) \simeq \frac{1}{2}\,\omega_\parallel^2 \tau^2 \quad\Rightarrow\quad
\lambda(\tau) \simeq 1 + \frac{\omega_\parallel^2 \tau^2}{2},
\]
indépendamment de l'équation d'état \(\mu(n)\). Ce comportement correspond au profil universel observé à temps courts (Fig.~7.4(a)) pour tous les régimes TF.

\paragraph{Régime à temps longs (\(\tau \gg 1/\omega_\parallel\)).}  
Pour \(\lambda^{-\alpha}\ll 1\), l'équation intégrée donne
\[
\dot\lambda \simeq \sqrt{\frac{2\omega_\parallel^2}{\alpha}} \quad\Rightarrow\quad
\lambda(\tau) \simeq \frac{2}{\alpha}\,\omega_\parallel \tau.
\]
En particulier :
\begin{itemize}
\item TF 1D (\(\alpha=1\)) : \(\lambda(\tau)\simeq \sqrt{2}\,\omega_\parallel \tau\),  
\item TF 3D (\(\alpha=1/2\)) : \(\lambda(\tau)\simeq 2\,\omega_\parallel \tau\).  
\end{itemize}
Ces comportements sont bien observés sur la Fig.~7.4(b) et correspondent à l’expansion asymptotique du gaz.





% --- Table révisée (petites corrections typographiques) ---
\begin{table}[h]
\centering
\begin{tabular}{l c c c}
\hline
Système & loi pour $\mu(n)$ & $\beta$ (avec $f(\lambda)=\lambda^{-\beta}$) & $\displaystyle \omega_{\rm breath}/\omega_\parallel$ \\
\hline
Gaz classique isotherme (1D, $\mu\propto\ln n$) 
& $\mu'(n)\propto 1/n$ 
& $-1$ 
& $\sqrt{3}\approx1.732$ \\[4pt]

Gaz de Bose 1D en régime moyen (GP, $\mu\propto n$) 
& $\alpha=1$ 
& $0$ 
& $2$ \\[4pt]

Tonks--Girardeau (1D, $\mu\propto n^2$) 
& $\alpha=2$ 
& $1$ 
& $\sqrt{5}\approx2.236$ \\[4pt]

Gaz de Fermi unitaire (ex. 3D, $\mu\propto n^{2/3}$) 
& $\alpha=\tfrac{2}{3}$ 
& $-\tfrac{1}{3}$ 
& $\sqrt{3+\tfrac{2}{3}}\approx1.915$ \\[4pt]

Cas général (loi de puissance) 
& $\mu\propto n^\alpha$ 
& $\beta=\alpha-1$ 
& $\displaystyle \sqrt{3+\alpha}$ \\
\hline
\end{tabular}
\caption{Valeurs de $\beta$ et fréquences du mode de souffle pour quelques régimes usuels.}
\label{tab:breathing}
\end{table}
 



\subsection{Sonde locale de distribution de rapidité}
\begin{itemize}
    \item Principe de la mesure : coupure d’une tranche puis expansion.
    \item Rôle du DMD dans la sélection.
    \item Accès à la distribution de vitesse locale.
    \item Comparaison avec les prédictions GHD.
    \item Limites et incertitudes
\end{itemize}

\subsubsection{Distribution de rapidités locale dans les gaz 1D}

\paragraph{Motivation.}  
La compréhension des gaz de bosons 1D avec interactions de contact répulsives repose sur la notion de distribution de rapidités \(\rho(\theta)\). Chaque état propre du système peut être paramétré par un ensemble de rapidités \(\{\theta_i\}\) (Ansatz de Bethe), ou interprété comme les vitesses de quasi-particules à durée de vie infinie. Ici, on utilise la définition pratique issue des expansions 1D : les rapidités correspondent aux vitesses asymptotiques des atomes après une expansion, avec \(x_j \simeq \tau \theta_j\) pour un temps \(\tau\) long. Cette définition est directement applicable à des mesures expérimentales de distribution de rapidités locales.

\paragraph{Distribution locale et LDA.}  
Pour un nuage atomique piégé dans un potentiel longitudinal variant lentement, on peut appliquer l’Approximation de Densité Locale (LDA). Le gaz est alors vu comme un fluide décomposé en cellules mésoscopiques de densité homogène et relaxée. Dans chaque cellule, l’état d’équilibre est décrit par un Ensemble de Gibbs Généralisé (GGE), ou équivalemment par une distribution de rapidités locale \(\rho(x,\theta)\). Cette description permet d’étudier non seulement l’équilibre, mais aussi la dynamique hors équilibre à grandes échelles spatiales et temporelles, via la théorie Hydrodynamique Généralisée (GHD).

\paragraph{Protocole expérimental.}  
Pour mesurer \(\rho(x,\theta)\) localement :  
\begin{enumerate}
    \item Une zone du nuage atomique de taille \(\ell\) centrée en \(x_0\) est sélectionnée à l’aide d’un dispositif de micromiroirs digitaux (DMD). La pression de radiation supprime instantanément les atomes en dehors de la zone, laissant uniquement ceux de la cellule.
    \item Après la sélection, le confinement longitudinal est relâché, tandis que le confinement transverse reste actif. Les atomes réalisent une expansion 1D pendant un temps \(\tau\), puis le profil de densité est imagé (typiquement pour \(\tau\sim 1\) ms).
    \item Le protocole est répété pour plusieurs positions \(x_0\), permettant d’obtenir la distribution de rapidités locale sur l’ensemble du nuage.
\end{enumerate}

\paragraph{Mesures à l’équilibre.}  
Pour un gaz initialement à l’équilibre dans un piège harmonique, le profil de densité de chaque zone sélectionnée est analysé via la thermodynamique Yang-Yang et la LDA, donnant température \(T_{\rm YY}\) et potentiel chimique \(\mu_{\rm YY}\). Après un temps d’expansion long, le profil devient homothétique à la distribution de rapidités locale \(\rho(x,\theta)\). La comparaison avec les prédictions numériques montre une bonne cohérence, confirmant que le protocole permet de sonder efficacement \(\rho(x,\theta)\).

\paragraph{Résumé.}  
— Une sonde locale de distribution de rapidités a été mise en place grâce au DMD.  
— Les atomes sélectionnés réalisent une expansion dans le guide 1D.  
— Après un temps long, le profil de densité reflète la distribution de rapidités locale.  
— Ce protocole a été appliqué avec succès sur un nuage atomique initialement à l’équilibre.


\section{Discussion sur les limites et les perspectives}
\begin{itemize}
    \item Contraintes techniques (bruit, alignement, stabilité de la puce…).
    \item Améliorations potentielles (résolution, contrôle du potentiel, automatisation).
    \item Perspectives pour d’autres types d’expériences (étude de chocs, turbulence quantique, etc.)
\end{itemize}

\section*{Conclusion}
\begin{itemize}
    \item Résumé de l’architecture du dispositif).
    \item Méthodes d’analyse utilisées et robustesse.
    \item Importance de l’expérience dans le contexte de l’étude des gaz quantiques unidimensionnels
\end{itemize}
Ce chapitre a présenté les éléments essentiels du dispositif expérimental, les méthodes d’imagerie, ainsi que les expériences auxquelles j’ai participé. L’ensemble constitue une plateforme performante pour l’étude de la dynamique de gaz 1D hors équilibre.

\paragraph{Résumé de l’architecture expérimentale}  
Nous avons décrit les éléments clés du dispositif utilisé : un système de refroidissement laser basé sur trois sources couplées, un piégeage magnétique sur puce optimisé pour réaliser des géométries unidimensionnelles, une plateforme de modulation de potentiel via un DMD, et un système d’imagerie haute résolution. L’ensemble permet une manipulation fine des nuages atomiques dans un cadre reproductible et stable.

\paragraph{Méthodes d’analyse et robustesse}  
L’imagerie par absorption, couplée à une analyse rigoureuse des profils atomiques, fournit des outils fiables pour extraire les grandeurs pertinentes : densités, tailles, températures, distributions de vitesses. Ces méthodes ont permis de confronter les résultats expérimentaux à des prédictions théoriques de type GHD ou Yang-Yang.

\paragraph{Importance du dispositif pour la thèse}  
Ce dispositif a été essentiel pour mener à bien les expériences présentées dans cette thèse. Il offre à la fois un contrôle local (grâce au DMD), un bon confinement transverse (grâce à la puce) et une imagerie précise. La plateforme est ainsi bien adaptée pour étudier des systèmes 1D fortement corrélés hors équilibre, et pour tester les prédictions de la physique statistique intégrable.

\paragraph{Perspectives}  
Malgré ses atouts, le dispositif présente des limitations techniques (rugosité magnétique, sensibilité à l’alignement, etc.) qui laissent entrevoir des pistes d’amélioration. Des développements futurs pourraient notamment viser à augmenter la résolution spatiale, automatiser davantage les séquences, ou explorer d'autres régimes dynamiques comme la turbulence ou les collisions de chocs quantiques.



%\appendix
\section*{Annexes}
\begin{itemize}
    \item Schémas techniques (puce, DMD, optique).
    \item Tableaux de paramètres expérimentaux.
    \item Exemples de motifs DMD utilisés.
\end{itemize}
\input{chapters/06_Bipart}
\input{chapters/07_Dipolaire}

%\chapter*{Conclusion}
\addcontentsline{toc}{chapter}{Conclusion}

Conclusion de la thèse.


%\appendix
%\chapter{Annexes}

Informations complémentaires.



\bibliographystyle{abbrv}
\bibliography{thesis}

%\printbibliography

\end{document}

%| Style     | Description                                                             |
%| --------- | ----------------------------------------------------------------------- |
%| `plain`   | Tri alphabétique, numérotation croissante                               |
%| `unsrt`   | Même que `plain` mais sans tri, respecte l’ordre d’apparition           |
%| `abbrv`   | Comme `plain` mais avec prénoms et noms abrégés                         |
%| `alpha`   | Les références sont étiquetées par une combinaison du nom et de l’année |
%| `apalike` | Style APA simplifié                                                     |
%| `ieeetr`  | Style IEEE, tri par ordre d’apparition                                  |
%| `siam`    | Style SIAM (mathématiques appliquées)                                   |
%| `acm`     | Style ACM (informatique)                                                |
%



\section{Rôle des charges conservées extensives et quasi-locales}
%Dans les systèmes intégrables, l’état stationnaire atteint après une évolution hors d’équilibre n’est généralement pas décrit par un état de Gibbs classique, mais par un ensemble généralisé de Gibbs (GGE). Celui-ci est construit à partir de toutes les charges conservées du système

\paragraph{Écriture des observables thermodynamiques comme sommes sur les rapidités.}

%Dans le cas thermique, les valeurs moyennes des observables classiques telles que le nombre de particules et l'énergie peuvent s'exprimer comme des sommes de puissances des rapidités :
Dans un système à $N$ particules caractérisé par des rapidités $\{ \theta_a \}_{a = 1}^N$, les charges conservées classiques — telles que le nombre de particules, l’impulsion ou l’énergie — s’écrivent comme des sommes de puissances des rapidités :
\(
	\langle \operator{Q} \rangle_{\{ \theta_a\} } \propto \sum_{a = 1}^N \theta_a^0 , \,  \langle \operator{P} \rangle_{\{ \theta_a\} } \propto \sum_{a = 1}^N \theta_a^1  ,\,  \mbox{et} \langle \operator{K} \rangle_{\{ \theta_a\} } \propto \sum_{a = 1}^N \theta_a^2 .	
\)
(cf. équations \eqref{chap.2.gge.1})
Dans ce paragraphe précédent, nous avons sous-entendu — sans l’expliciter — qu’il est montré que l’ensemble des charges locales conservées forme une famille donnée par :
\begin{eqnarray}
	\operator{Q}_i^{(\mathcal{S})} \ket{\{\theta_a\} } & \propto & \sum_a \theta_a^i \ket{\{\theta_a\} }.
\end{eqnarray}
Ces charges agissent donc de manière diagonale sur les états de Bethe, avec des valeurs propres correspondant aux moments des rapidités.
%%%%%%%%%%%%%%%%%%%%%%%%%%%%%%%%%%%%%%%%%%%%%%%%%%
\paragraph{Charges locales conservées .\label{sec:charges-gen}}

%Les états propres du Hamiltonien de Lieb–Liniger~\eqref{eq:LL} sont les états de Bethe
%\(
%  \ket{\boldsymbol{\theta}}
%  =\ket{\theta_1,\dots,\theta_N}\!,
%\)
%déterminés par leurs rapidités \(\boldsymbol{\theta}\).

À toute fonction régulière
\(
  f:\mathbb R\!\to\!\mathbb R
\)
on associe un opérateur-charge loclal :
\begin{eqnarray}\label{chap.2.charge.f.1}
	\operator{\mathcal{Q}}^{(\mathcal{S})}[f] & = &  L \int_0^L d\theta \, f(\theta) \operator{\rho}^{(\mathcal{S})}(\theta).	
\end{eqnarray}
où $\operator{\rho}(\theta)$ agit sur une état de Bethe comme 
\begin{eqnarray}\label{chap.2.rho.1}
	 \operator{\rho}(\theta) \ket{ \{ \theta_a \} } &=& \frac{1}{L} \sum_{a = 1 }^N  \delta ( \theta - \theta_a ) \ket{ \{ \theta_a \} }.	
\end{eqnarray}
De sorte que $\operator{\mathcal{Q}}^{(\mathcal{S})}[f]$ agit sur une état de Bethe comme
\begin{eqnarray}\label{chap.2.charge.1}
	\operator{\mathcal{Q}}^{(\mathcal{S})}[f]\,\ket{\{\theta_a\} } =  \sum_{a=1}^{N}f(\theta_a)\,\ket{\{\theta_a\} } \quad \mbox{de sorte que} \quad \braket{\operator{\mathcal{Q}}^{(\mathcal{S})}[f]}_{\{\theta_a\}} = \sum_{a=1}^N f(\theta_a)
\end{eqnarray}
Les choix particuliers
\(
  f_0(\theta)=1
\)
,
\(
  f_1(\theta)=\theta
\)
et
\(
  f_2(\theta)=\theta^{2}/2
\)
redonnent respectivement l'opérateur nombre \(\operator{Q}=\operator{Q}_0^{(\mathcal{S})} = \operator{\mathcal{Q}}^{(\mathcal{S})}[1]\) , impulsion \(\operator{P}=\operator{Q}_1^{(\mathcal{S})} = \operator{\mathcal{Q}}^{(\mathcal{S})}[\theta]\) et énergie cinétique
\(\operator{K}=\operator{Q}_2^{(\mathcal{S})} = \operator{\mathcal{Q}}^{(\mathcal{S})}[\theta^2/2]\). Et dans le cadre des (GGE), pour tous les ordres $i$ on note :
\begin{eqnarray}\label{chap.2.charge.ordre.i.1}
	\operator{Q}^{(\mathcal{S})}_i = \operator{\mathcal{Q}}^{(\mathcal{S})}[f_i]	, \quad \mbox{de sorte que} \quad \braket{\operator{Q}^{(\mathcal{S})}_i}_{\{\theta_a\}} = \sum_{a=1}^N f_i(\theta_a)  
\end{eqnarray}
avec les densités spectrales $f_i(\theta) \propto \theta^i$ . 

Ces charges sont extensives : leur densité locale $\operator{q}^{(\mathcal{S})}_{[f]}$ permet d’écrire
\(
  \operator{\mathcal{Q}}^{(\mathcal{S})}[f]=\int_0^{L}\!dx\;\operator{q}^{(\mathcal{S})}_{[f]}(x).
\)

\paragraph{Charges conservées généralisée.\label{sec:charges-gen}}
Les fonction $f_i$ étant fixées, on note la fonction régulière
\(
  w:\mathbb R\!\to\!\mathbb R
\)
–– dorénavant appelée \emph{poids spectral}, ou \emph{potentiel spectral} ––
\begin{eqnarray}
	w = \sum_i \beta_i f_i \label{chap.2.w.1},	
\end{eqnarray}
on associe un opérateur-charge généralisé $\operator{\mathcal{Q}}^{(\mathcal{S})}[w]$ :
\begin{eqnarray}\label{chap.2.charge.gen.1}
	\operator{\mathcal{Q}}^{(\mathcal{S})}[w]\,\ket{\{\theta_a\} } =  \sum_{a=1}^{N}w(\theta_a)\,\ket{\{\theta_a\} } \quad \mbox{de sorte que} \quad \braket{\operator{\mathcal{Q}}^{(\mathcal{S})}[w]}_{\{\theta_a\}} = \sum_{i} \beta_i  \braket{\operator{Q}^{(\mathcal{S})}_i}_{\{\theta_a\}}
\end{eqnarray}

%%%%%%%%%%%%%%%%%%%%%%%%%%%%%%%%%%%%%%%%%
\paragraph{Expression de la matrice densité généralisée.}
La matrice densité  s’écrit sous la forme :
L’ensemble général défini par $\operator{\varrho}^{(\mathcal{S})}[w]$ 
\begin{eqnarray}\label{chap.2.densite.1}
	\operator{\varrho}^{(\mathcal{S})}[w]  =  \frac{e^{-\operator{\mathcal{Q}}^{(\mathcal{S})}[w]}}{Z^{(\mathcal{S})}[w]}, \, \mbox{avec} \quad e^{-\operator{\mathcal{Q}}^{(\mathcal{S})}[w]}  = 	\sum_{\{\theta_a \}} e^{- \sum_{a = 1}^N w(\theta_a) } \vert \{ \theta_a\} \rangle \langle  \{ \theta_a\}  \vert, 
\end{eqnarray}	
	%pour une certaine fonction $w$ relié à la charge% $\operator{\mathcal{Q}} [w]  = \sum_{\{\theta_a \}} \left ( \sum_{a = 1}^N w ( \theta_a )  \right ) \vert \{ \theta_a \} \rangle \langle \{ \theta_a \} \vert $.
%où l'opérateur de charge associé à $w$ s’écrit :
%\begin{eqnarray}
%	\operator{\mathcal{Q}} [w]   & = &  \sum_{\{\theta_a \}} \left ( \sum_{a = 1}^N w ( \theta_a )  \right ) \vert \{ \theta_a \} \rangle \langle \{ \theta_a \} \vert,	
%\end{eqnarray}
et la fonction de partition \eqref{chap.TBA.op.Z.S} s'écrit $Z^{(\mathcal{S})}[w]\doteq \bm{\mathrm{Tr}}\left [ e^{-\operator{\mathcal{Q}}^{(\mathcal{S})}[w]}\right ] $ vaux :
\begin{eqnarray}
	Z^{(\mathcal{S})}[w]   =  \sum_{\{\theta_a \}} e^{-\sum_{a = 1}^N w(\theta_a)},\label{chap.TBA.op.Z.S.1}	
\end{eqnarray}
devient un Generalized Gibbs Ensemble (GGE), $\operator{\rho}^{(\mathcal{S})}_{\mathrm{GGE}}$ (de l'équation \eqref{chap.TBA.op.rho.S})	 dès lors que $w(\theta) = \sum_i \beta_i f_i(\theta)$ (de l'équation \eqref{chap.2.w.1}) où $f_i$ sont les densités spectrales associées aux charges locales conservées (de l'équation \eqref{chap.2.charge.ordre.i.1}).


%%%%%%%%%%%%%%%%%%%%%%%%%%%%%%%%%%
\paragraph{Probabilité associée à une configuration de rapidités.}
	%Et on peut réecrire la probabilité de la configuration $\{\theta_a\}$ :% $ P_{\{ \theta_a \}} = \langle \{ \theta_a \}\vert \operator{\rho}_{GGE}[w] \vert  \{ \theta_a \} \rangle = e^{-\sum_{a = 1}^N w(\theta_a)} / Z $ avec $Z = \sum_{\{\theta_a \}} e^{-\sum_{a = 1}^N w(\theta_a)}$.\\
	%La probabilité d’occuper un état à $N$ particules caractérisé par les rapidités ${\theta_a}$ est alors :
Dans ce formalisme, la probabilité d’occuper l’état $\ket{\{\theta \}}$ \eqref{chap.TBA.P.1} est donc
\begin{eqnarray}
	\mathbb{P}^{(\mathcal{S})}_{\{ \theta_a \}} & = &  Z^{(\mathcal{S})}[w]^{-1}e^{-\sum_{a = 1}^N w(\theta_a)}\label{chap.TBA.P.w.2}. 		
\end{eqnarray}
%Cela montre que le poids statistique d’une configuration factorise naturellement sur les pseudo-moments, avec un poids spectrale / energie génralisé $w(\theta)$ attribué à chaque particule.
On voit ainsi que le poids statistique factorise naturellement sur les
pseudo‑moments, chaque particule étant pondérée par $w(\theta_a)$.

%avec 
%\begin{eqnarray}
%	Z  & = & \sum_{\{\theta_a \}} e^{-\sum_{a = 1}^N w(\theta_a)}.		
%\end{eqnarray}


%%%%%%%%%%%%%%%%%%%%%%%%
\paragraph{Moyennes d'observables dans le GGE.}
%La valeur moyenne d’un observable locale $\operator{\mathcal{O}}$ dans l’ensemble généralisé s’écrit :
Pour tout opérateur local $\operator{\mathcal{O}}$ diagonal dans la base de Bethe,
la moyenne généralisée vaut
\begin{eqnarray}\label{chap.2.moyenne.1}
	\langle \operator{\mathcal{O}}\rangle_{\operator{\varrho}^{(\mathcal{S})}[w]} & = & \displaystyle   \frac{\sum_{\{\theta_a \}} \braket{ \operator{\mathcal{O}}}_{\{ \theta_a\}} e^{- \sum_{a = 1}^N w(\theta_a) }  }{\sum_{\{\theta_a  \}} e^{- \sum_{a = 1}^N  w(\theta_a) } }
\end{eqnarray}
%Cette expression formelle montre que la connaissance de $w(\theta)$ suffit à déterminer les propriétés statistiques de toutes les observables diagonales dans cette base, incluant les charges conservées elles-mêmes.
Ainsi, la connaissance de la fonction $w(\theta)$ suffit à déterminer
les propriétés statistiques de toute observable diagonale,
y compris les charges conservées elles‑mêmes.	
	% Nous aimerions calculer les valeurs d'attente par rapport à cette matrice de densité, par exemple
	%La moyenne GGE d'un observable s'écrit ,
	%\begin{aff}
	%\begin{eqnarray}
	%	\langle \operator{\mathcal{O}} \rangle_{GGE} & \doteq & \displaystyle  \text{Tr} (\operator{\mathcal{O}}\operator{\rho}[w]) = \frac{\text{Tr} (\operator{\mathcal{O}}e^{-\operator{\mathcal{Q}}[w]})}{\text{Tr} (e^{-\operator{\mathcal{Q}}[w]})}	 = \frac{\sum_{\{\theta_a \}} \langle  \{ \theta_a\}  \vert   \operator{\mathcal{O}} \vert \{ \theta_a\} \rangle e^{- \sum_{a = 1}^N w(\theta_a) }  }{\sum_{\{\theta_a  \}} e^{- \sum_{a = 1}^N  f(\theta_a) } }
		%& =  & \frac{ \sum_{\pi} \sum_{\vert \{\theta_a \}\rangle \vert \Pi } \langle  \{ \theta_a\}  \vert   \operator{\mathcal{O}} \vert \{ \theta_a\} \rangle e^{- \sum_{a = 1}^N f(\theta_a) }  }{\sum_{\pi} \sum_{\vert \{\theta_a \}\rangle \vert \Pi }  e^{- \sum_{a = 1}^N  f(\theta_a) } }
	%\end{eqnarray}
	%pour une certaine observable $\operator{\mathcal{O}}$.\\
	%\end{aff}
	

\paragraph{Conclusion de la section : vers la thermodynamique de Bethe.}

Nous avons vu que, dans un système intégrable, la description correcte de l’équilibre stationnaire requiert l’introduction d’une \emph{famille infinie de charges conservées}, comprenant à la fois des charges strictement locales et des charges quasi‑locales.
Toutes ces charges se réunissent dans l’opérateur fonctionnel
\(
\operator{\mathcal{Q}}^{(\mathcal{S})}[w]
\)
, défini par un \emph{poids spectral}  $w(\theta)$ (cf. équations~\eqref{chap.2.charge.1}).
Cette construction conduit naturellement à la matrice densité généralisée
\(
\operator{\rho}^{(\mathcal{S})}_{\mathrm{GGE}}  \propto  e^{-\operator{\mathcal{Q}}^{(\mathcal{S})}[w]}
\) 
(cf. équations~\eqref{chap.2.densite.1}), et à la moyenne d’un opérateur local $\operator{\mathcal{O}}$ donnée par
\(
\langle \operator{\mathcal{O}}\rangle_{\operator{\rho}^{(\mathcal{S})}_{\mathrm{GGE}}}  =  \displaystyle  \text{Tr} (\operator{\mathcal{O}}\operator{\varrho}^{(\mathcal{S})}[w])
\)
(cf. équations~\eqref{chap.2.moyenne.1}).
La connaissance de $w(\theta)$ suffit donc pour prédire les valeurs moyennes de toutes les observables diagonales, y compris celles des charges elles‑mêmes ; c’est le cœur du {\bf Ensemble de Gibbs Généralisé (GGE pour Generalized Gibbs Ensemble)} .

\medskip
Cette base est désormais posée : dans la section suivante, nous passerons au \emph{thermodynamique de Bethe}.
Nous verrons comment, dans la limite thermodynamique, les sommes sur les configurations de rapidités se transforment en intégrales sur des densités continues, comment apparaît l’entropie de Yang–Yang, et comment les moyennes de l’ensemble généralisé se réexpriment à l’aide de ces densités macroscopiques.
C’est ce formalisme qui permettra d’analyser finement la relaxation post‑quench et de relier microscopie intégrable et hydrodynamique généralisée.



%% !TEX encoding = IsoLatin

%\documentclass[11pt,a4paper]{report}
%\documentclass[11pt,a4paper]{book}
\documentclass[10pt, titlepage]{book} % Taille de base des caractères (12pt recommandée pour lecture)


% -------------------------------------
% Encodage et langue
% -------------------------------------
\usepackage[utf8]{inputenc}
\usepackage[T1]{fontenc}
\usepackage[french]{babel}

% -------------------------------------
% Marges et dimensions
% -------------------------------------
\usepackage[a4paper, top=1.0cm, bottom=1.0cm, left=1cm, right=1cm]{geometry} 
% Ajuste ici les marges selon tes préférences

% -------------------------------------
% Interligne
% -------------------------------------
\usepackage{setspace}
%\onehalfspacing  % Interligne 1.5 
%\doublespacing %(utilise \doublespacing pour double interligne)

% -------------------------------------
% Police (facultatif)
% -------------------------------------
%\usepackage{mathptmx} % Police Times (ancienne)
%\usepackage{libertine} % Police élégante

\usepackage{newtxtext,newtxmath} % Times moderne pour texte et maths

% -------------------------------------
% Paquets utiles
% -------------------------------------
\let\Bbbk\relax
\let\openbox\relax
\usepackage{amsmath, amssymb, amsthm}
\usepackage{graphicx}
\usepackage{hyperref}
\usepackage{xcolor}
\usepackage{braket}
\usepackage{tikz}
\usepackage{pgfplots}
\usepackage{float}
\usepackage{enumitem}
\usepackage{caption}
\usepackage{subcaption}
\usepackage{algorithm2e}
\usepackage{cancel}
\usepackage{bm}
\usepackage{listings}
\usepackage{pdfpages}
\usepackage{mdframed}
\usepackage{braket}
\usepackage{stmaryrd} 
\usetikzlibrary {datavisualization}
\usetikzlibrary {arrows.meta,bending,positioning}
\usetikzlibrary {datavisualization.formats.functions}
%PREAMBULE pour schÃéma
\usepackage{pgfplots}
\usepackage{tikz}
\usepackage[european resistor, european voltage, european current]{circuitikz}
\usetikzlibrary{arrows,shapes,positioning}
\usetikzlibrary{decorations.markings,decorations.pathmorphing,
decorations.pathreplacing}
\usetikzlibrary{calc,patterns,shapes.geometric}
\usepackage{anyfontsize}


% -------------------------------------
% Pour les chapitres
% -------------------------------------
\usepackage[Glenn]{fncychap} % Style de chapitres


% -------------------------------------
% Largeur du texte (évite de le redéfinir si tu utilises geometry)
% -------------------------------------
%\setlength\textwidth{20.5cm}
%\setlength\textheight{22cm}

% -------------------------------------
% Optionnel : si tu veux jouer avec les marges manuellement
% -------------------------------------
% \setlength\topmargin{-1cm}
% \setlength\evensidemargin{-2cm}
% \setlength\oddsidemargin{\evensidemargin}

\usepackage{mdframed}

\usepackage{scalerel}
\usepackage{xcolor}
\usepackage{stackengine}
\usepgflibrary {shadings}


\usetikzlibrary {decorations.pathmorphing}

\usepackage{tikz}

\usepackage{marvosym}
\usepackage{changepage}

\usepackage{minitoc}
\usepackage{tocloft}
%\renewcommand{\cfttoctitle}{\hspace{-2em}}
% Nastaveni obsahu
% Nastaveni obsahu

\usepackage{imakeidx}
\usepackage{fancyhdr}

%\usepackage{makeidx}
\makeindex[intoc=true]
\makeindex[name=pers, title=Index of person names, intoc=true]

\usepackage{xcolor}

\usepackage{hyperref}

%%%%%%%%%%%%%%%%%%%%%
%\definecolor{linkcolor}{RGB}{0,0,180}
\usepackage{titlesec}

\usepackage{tocloft}
\usepackage{datetime} % Pour une date personnalisée
\usepackage[useregional]{datetime2}

\usepackage{mathrsfs}

% -------------------------------------
% Pour les mini-tables des matières
% -------------------------------------
\usepackage{minitoc}
\dominitoc

%\usepackage[most]{tcolorbox}

%%%%%%%%%%%%%%%%%%%%%%%%%%%%%
%\usepackage[utf8]{inputenc}
%\usepackage[T1]{fontenc}
%\usepackage[french]{babel}
%\usepackage{amsmath, amssymb}
%\usepackage{graphicx}
%\usepackage{hyperref}
%\usepackage{tikz}
%\usepackage{physics}
%\usepackage{float}


%\newcommand{\ket}[1]{\left|#1\right\rangle}
%\newcommand{\bra}[1]{\left\langle#1\right|}
%\newcommand{\mean}[1]{\left\langle#1\right\rangle}
%\newcommand{\dd}{\mathrm{d}}

% Activer \frontmatter, \mainmatter et \appendix pour la classe report
%\newcommand{\frontmatter}{%
%  \pagenumbering{roman}%
%  \setcounter{page}{1}%
%  \renewcommand{\chaptermark}[1]{\markboth{##1}{}}
%  \renewcommand{\sectionmark}[1]{\markright{##1}}
%}
%
%\newcommand{\mainmatter}{%
%  \pagenumbering{arabic}%
%  \setcounter{page}{1}
%}


% \appendix est déjà défini dans report, inutile de le redéfinir


% Figures flottantes:
% fraction maximale d'une page pouvant etre occupe par une figure:
\renewcommand{\topfraction}{0.8}
% fraction minimale d'une page reservee pour le texte:
\renewcommand{\textfraction}{0.2}
% fraction minimale d'occupation de la page par une figure pleine page:
\renewcommand{\floatpagefraction}{0.7}

%%%%%%%%%%%%%%%%%%%%%%%%%%%%%%%%%%%%%%%%
%         D\'ecoupage des mots           %
%%%%%%%%%%%%%%%%%%%%%%%%%%%%%%%%%%%%%%%%
\hyphenation{}

%%%%%%%%%%%%%%%%%%%%%%%%%%%%%%%%%%%%%%%%
%%%%  Th\'eor\`emes, d\'efinitions, etc.
%%%%%%%%%%%%%%%%%%%%%%%%%%%%%%%%%%%%%%%%


% Il y a diffÃérents types d'ÃénoncÃés qui mÃéritent un environnement spÃécifique, voici une liste assez exhaustive.
\theoremstyle{plain}
    \newtheorem{Theo}{Th\'eor\`eme}[section] %compteur commençant par le numÃéro de la section (on pourrait aussi faire commencer par le numÃéro de la sous-section - remplacer "section" par "subsection")
    \newtheorem{Prop}[Theo]{Proposition}        %mÃême compteur que pour les thÃéorÃèmes
    \newtheorem{Prob}[Theo]{Probl\`eme}        %idem
    \newtheorem{Lemm}[Theo]{Lemme}            %etc...
    \newtheorem{Coro}[Theo]{Corollaire}
    \newtheorem{Propr}[Theo]{Propri\'et\'e}
    \newtheorem{Conj}[Theo]{ Conjecture}
    \newtheorem{Aff}[Theo]{Affirmation}

    \newtheorem{TheoPrinc}{Th\'eor\`eme}     %compteur spÃécifique pour les thÃéorÃèmes les plus importants du papier
        
\theoremstyle{definition}
    \newtheorem{Defi}[Theo]{D\'efinition}
    \newtheorem{Exem}[Theo]{Exemple}
    \newtheorem{Nota}[Theo]{\Large Notation}

\theoremstyle{remark}
    \newtheorem{Rema}[Theo]{Remarque}
    \newtheorem{NB}[Theo]{N.B.}
    \newtheorem{Comm}[Theo]{Commentaire}
    \newtheorem{question}[Theo]{$\ast$ Question}
    \newtheorem{exer}[Theo]{Exercice}
    \newtheorem{Consequence}[Theo]{Conséquence}
    \newtheorem{Rap}[Theo]{Rappel}
    \newtheorem*{Merci}{Remerciements}

\mdfdefinestyle{propstyle}{%
linecolor=black,linewidth=2pt,%
hidealllines=true,
frametitlerule=true,%
frametitlebackgroundcolor=gray!20,
backgroundcolor=gray!10!white,
roundcorner=5pt,
innertopmargin=\topskip,
}

%\mdtheorem[style=propstyle]{prop}{Property}[chapter]
\mdtheorem[style=propstyle]{lemma}[prop]{Lemma}
\mdtheorem[style=propstyle]{TheoPrinc}{Th\'eor\`eme}[chapter]

% Définition d'un style personnalisé pour les Affirmations
\mdfdefinestyle{affirmestyle}{%
    linecolor=gray, % Couleur de la bordure
    linewidth=1pt, % Épaisseur de la bordure
    backgroundcolor=gray!10, % Couleur de fond (gris clair)
    roundcorner=5pt, % Coins arrondis
    innertopmargin=0pt, % Marge intérieure au-dessus du cadre
    innerbottommargin=10pt, % Marge intérieure en-dessous du cadre
    innerleftmargin=10pt, % Marge intérieure à gauche
    innerrightmargin=10pt, % Marge intérieure à droite
    skipabove=10pt, % Espace au-dessus du cadre
    skipbelow=10pt % Espace en-dessous du cadre
}

% Définition de l'environnement Affirmation
\theoremstyle{definition} % Style de théorème pour les affirmations
\newmdtheoremenv[style=affirmestyle]{aff}{Point clé n$^{\circ}$} % Environnement Affirmation avec le style personnalisé
    
\newcommand\dangersign[1]{%
    \renewcommand\stacktype{L}%
    \scaleto{\stackon[1.3pt]{\color{red}$\triangle$}{\tiny !}}{#1}%
}

\tikzset{every picture/.style={execute at begin picture={\shorthandoff{:;!?};}}}
\tikzstyle{every picture}+=[remember picture]
\tikzstyle{na} = [shape=rectangle,inner sep=0pt]

% Commandes pour les flèches textuelles
\newcommand{\ptFleche}[2]{        % Déclaration d'une extrémité de flèche
    \tikz[baseline=(#1.base)]\node[na](#1){#2};
  }
%\newcommand{\Fleche}[5][thick]{    % Dessin de la flèche
%    \begin{tikzpicture}[overlay]
%        \path[->,#1](#2) edge [out=#4, in=#5] (#3);
%    \end{tikzpicture}
%  }
  
% \newcommand{\Flecheprim}[5][thick]{    % Dessin de la flèche
%    \begin{tikzpicture}[overlay]
%        \path[->,#1](#2) edge [out=#4, in=#5] (#3);
%    \end{tikzpicture}
%  }
%



\definecolor{linkcolor}{RGB}{0,0,180}
\PassOptionsToPackage{
    colorlinks=true,
    linkcolor=linkcolor,
    citecolor=linkcolor,
    urlcolor=linkcolor
}{hyperref}

% Appliquer la couleur à tous les niveaux de titre
\titleformat{\section}{\normalfont\color{colorSix!90!black}\Large\bfseries}{\thesection}{1em}{}
\titleformat{\subsection}{\normalfont\color{colorSix!70!black}\large\bfseries}{\thesubsection}{1em}{}
\titleformat{\subsubsection}{\normalfont\color{colorSix!50!black}\normalsize\bfseries}{\thesubsubsection}{1em}{}
\titleformat{\paragraph}[runin]{\normalfont\color{colorOne!30!black}\bfseries}{\theparagraph}{1em}{}
\titleformat{\subparagraph}[runin]{\normalfont\color{colorOne!10!black}\itshape}{\thesubparagraph}{1em}{}
%%%%%%%%%%%%%%%%%%%%%%%

%%Couleurs dans la table des matières

% Modifier la couleur des entrées de la TOC
\renewcommand{\cftsecfont}{\color{linkcolor!90!black}}
\renewcommand{\cftsubsecfont}{\color{linkcolor!70!black}}
\renewcommand{\cftsubsubsecfont}{\color{linkcolor!50!black}}
\renewcommand{\cftparafont}{\color{linkcolor!30!black}}
\renewcommand{\cftsubparafont}{\color{linkcolor!10!black}}
%%%%%%%%%%%%%%%%%%%%%%%%%%%%%
% Reglages:
%
%\pagestyle{fancyplain}
%\addtolength{\headwidth}{\marginparsep}
%\addtolength{\headwidth}{\marginparwidth}
%\renewcommand{\chaptermark}[1]{\markboth{#1}{}}
%\renewcommand{\sectionmark}[1]{\markright{\thesection\ #1}}
%\lhead[\fancyplain{}{\bfseries\thepage}]{}
%\rhead[]{\fancyplain{}{\bfseries\thepage}}
%\chead[\fancyplain{}{\bfseries\leftmark}]{\fancyplain{}{\bfseries\rightmark}}
%\cfoot{}
%

%usepackage{titlesec}
% Changer la couleur des paragraphes en rouge par exemple :
%\titleformat{\paragraph}[runin] % ou [block] selon ce que tu veux
%  {\normalfont\color{red}\bfseries}
%  {\theparagraph}{1em}{}

% Définition des couleurs avec les codes HTML
\definecolor{colorOne}{HTML}{443E46}
\definecolor{colorTwo}{HTML}{F6DEB8}
\definecolor{colorThree}{HTML}{908CA4}
\definecolor{colorFour}{HTML}{57659E}
\definecolor{colorFive}{HTML}{C57284}
\definecolor{colorSix}{HTML}{FF5B69}

% Raccourcis pour les couleurs
\def\colorOne{colorOne}
\def\colorTwo{colorTwo}
\def\colorThree{colorThree}
\def\colorFour{colorFour}
\def\colorFive{colorFive}
\def\colorSix{colorSix}

%%% ===== Index principal + index secondaire (noms propres) =====
\makeindex[intoc=true]
\makeindex[name=pers, title=Index des noms propres, intoc=true]

%%% ===== Couleur des liens =====
\definecolor{linkcolor}{RGB}{0,0,180}
\PassOptionsToPackage{
    colorlinks=true,
    linkcolor=linkcolor,
    citecolor=linkcolor,
    urlcolor=linkcolor
}{hyperref}
\usepackage{hyperref}

%%% ===== Réglages hyperref =====
\hypersetup{
  pdftitle={Étude de la dynamique hors équilibre de bosons unidimensionnels},
  pdfsubject={Quantum Physics},
  pdfauthor={Guillaume THEMEZE <guillaume.themeze@gmail.fr>},
  pdfkeywords={LaTeX, quantum, bosons, dynamique},
  colorlinks=true
}

%%% ===== Style des titres (colorés) =====
\titleformat{\chapter}[display]{\normalfont\sffamily\huge\bfseries\color{colorSix}}{\chaptertitlename\ \thechapter}{20pt}{\Huge}
\titleformat{\section}{\normalfont\color{colorSix!90!colorFour}\Large\bfseries}{\thesection}{1em}{}
\titleformat{\subsection}{\normalfont\color{colorSix!70!colorFour}\large\bfseries}{\thesubsection}{1em}{}
\titleformat{\subsubsection}{\normalfont\color{colorSix!50!colorFour}\normalsize\bfseries}{\thesubsubsection}{1em}{}
\titleformat{\paragraph}[runin]{\normalfont\color{colorSix!30!colorFour}\bfseries}{\theparagraph}{1em}{}
\titleformat{\subparagraph}[runin]{\normalfont\color{colorSix!10!colorFour}\itshape}{\thesubparagraph}{1em}{}

%%% ===== Couleurs de la table des matières =====
\renewcommand{\cftsecfont}{\color{linkcolor!90!black}}
\renewcommand{\cftsubsecfont}{\color{linkcolor!70!black}}
\renewcommand{\cftsubsubsecfont}{\color{linkcolor!50!black}}
\renewcommand{\cftparafont}{\color{linkcolor!30!black}}
\renewcommand{\cftsubparafont}{\color{linkcolor!10!black}}

%%% ===== En-têtes et pieds de page =====
\pagestyle{fancy}
\fancyhf{}
\setlength{\headheight}{14pt}

\fancyhead[RO,LE]{\thepage}
\fancyhead[LO]{\scshape \nouppercase{\rightmark}}  % Section
\fancyhead[RE]{\scshape \nouppercase{\leftmark}}  % Chapitre
\renewcommand{\headrulewidth}{.4pt}


\newdateformat{mydate}{\THEDAY~\monthname[\THEMONTH]~\THEYEAR}
\newdateformat{mydatetime}{\THEDAY~\monthname[\THEMONTH]~\THEYEAR~à~\currenttime}

%\DTMsetstyle{french} % ou autre style
%\DTMsetup{showtimezone=false}

\fancyfoot[L]{Thèse}
%\fancyfoot[R]{Paris, \mydatetime\today{} -- Période 2022--2025}
\fancyfoot[R]{Paris, \DTMnow -- Période 2022--2025}
%\fancyfoot[R]{Paris, le \DTMdate\today{} à \DTMcurrenttime -- Période 2022--2025}
\renewcommand{\footrulewidth}{.4pt}

% Supprimer les numéros sur la première page de chaque chapitre
\makeatletter
\let\ps@plain=\ps@empty
\makeatother

%%% ===== Réglages des titres de sections dans les en-têtes =====
\renewcommand{\chaptermark}[1]{\markboth{#1}{}}
\renewcommand{\sectionmark}[1]{\markright{\thesection\ #1}}

%%% ===== Notes de bas de page à la française =====
\usepackage[french]{babel}
%\usepackage[frenchfootnotes]{french}
%\FrenchFootnotes
%\AddThinSpaceBeforeFootnotes

%%%%%%%%%%%%%%%%%%%%%%%%%%%%%%%%%%
\newcommand{\operatorvec}[1]{\vec{{\bm{#1}}}} % pour les operateur
\newcommand{\operator}[1]{\hat{\bm{#1}}} % pour les operaeur vecteur
\newcommand{\operatormat}[1]{\operatorname{#1}} % pour les operaeur vecteur
\newcommand{\operatortilde}[1]{\tilde{\bm{#1}}} % pour les opetateur avec un tilde
\newcommand{\operatortildevec}[1]{\tilde{\bm{#1}}}% pour les opetateur avec un tilde et vecteur
\newcommand{\dfonc}[1]{\mathscr{D}_{[#1]}}
%%%%%%%%%%%%%%%%%%%%%%%%%%%%%%%%%%

%🔤 2. Abréviations classiques
% Mathématiques générales
\newcommand{\dd}{\mathrm{d}}           % différentielle droite
\newcommand{\ii}{\mathrm{i}}           % unité imaginaire
\newcommand{\ee}{\mathrm{e}}           % exponentielle

% Pour les ensembles usuels
\newcommand{\R}{\mathbb{R}}            % réels
\newcommand{\C}{\mathbb{C}}            % complexes
\newcommand{\Z}{\mathbb{Z}}            % entiers
\newcommand{\N}{\mathbb{N}}            % naturels

% Délimiteurs automatiques
%\newcommand{\abs}[1]{\left|#1\right|}
%\newcommand{\norm}[1]{\left\lVert#1\right\rVert}
%\newcommand{\paren}[1]{\left(#1\right)}
%\newcommand{\bracket}[1]{\left[#1\right]}
%\newcommand{\set}[1]{\left\{#1\right\}}

%⚛️ 3. Physique quantique
% Bra-ket
%\newcommand{\ketbra}[2]{\ket{#1}\!\bra{#2}}
%\newcommand{\braket}[2]{\left\langle #1 \middle| #2 \right\rangle}
%\newcommand{\ketproj}[1]{\ket{#1}\!\bra{#1}}

% Hamiltonien, opérateurs
\newcommand{\Ham}{\mathcal{H}}
\newcommand{\Op}[1]{\hat{#1}}
\newcommand{\Tr}{\mathrm{Tr}}

% Commutateurs et anticommutateurs
\newcommand{\comm}[2]{\left[#1, #2\right]}
\newcommand{\acomm}[2]{\left\{#1, #2\right\}}

%🌡️ 4. GHD ou dynamique intégrable
\newcommand{\rhoP}{\rho_{\mathrm{p}}}       % densité de particules
\newcommand{\rhoT}{\rho_{\mathrm{t}}}       % densité totale
\newcommand{\veff}{v^{\mathrm{eff}}}        % vitesse efficace
\newcommand{\dr}{\partial}                  % dérivée
\newcommand{\nustar}{\nu^\ast}              % solution auto-similaire

%✍️ 5. Utilisation typographique
\newcommand{\eg}{\emph{e.g.}\xspace}
\newcommand{\ie}{\emph{i.e.}\xspace}
\newcommand{\etal}{\emph{et al.}\xspace}



% Commandes spécifiques ou pour la mise en forme

\makeatletter
\newcommand\xleftrightarrow[2][]{%
  \ext@arrow 9999{\longleftrightarrowfill@}{#1}{#2}}
\newcommand\longleftrightarrowfill@{%
  \arrowfill@\leftarrow\relbar\rightarrow}
\makeatother


\newacronym{LL}{LL}{Lieb-Liniger}
\newacronym{NS}{NS}{Schrödinger non linéaire}
\newacronym{GP}{GP}{Gross–Pitaevskii}
\newacronym{GGE}{GGE}{Generalized Gibbs Ensemble}

\title{Titre de la thèse}
\author{Prénom NOM}
\date{\today}



\begin{document}

\frontmatter
%\chapter*{Introduction}
\addcontentsline{toc}{chapter}{Introduction}
\minitoc

Ceci est l’introduction de la thèse.


\tableofcontents
\mainmatter

\chapter{Modèle de Lieb-Liniger et approche Bethe Ansatz}
\minitoc

\section*{Introduction}

Dans ce chapitre, nous introduisons progressivement le modèle de Lieb-Liniger et l'Ansatz de Bethe, outils fondamentaux pour décrire un gaz de bosons unidimensionnel avec interactions delta. L'objectif est d'accompagner pas à pas le lecteur depuis la formulation du problème quantique en champ de bosons jusqu'aux solutions exactes obtenues par l'Ansatz de Bethe.

Nous commençons par écrire l'équation du champ de bosons, exprimée à l’aide des opérateurs de création et d’annihilation en représentation de position. Pour des raisons pédagogiques, nous abordons d’abord le cas d’une seule particule, sans interaction. Cela permet d’introduire naturellement les états de position et leur évolution sous l’action du Hamiltonien libre.

Ensuite, nous étudions le cas de deux particules, cette fois en tenant compte de l’interaction locale. Cela nous amène à considérer les états de position dans le cas général, y compris lorsque les deux particules peuvent occuper la même position. Cette situation, bien plus subtile qu’il n’y paraît, met en évidence la complexité introduite par l’interaction, et justifie que l’on commence par analyser les configurations où les particules sont à des positions distinctes.

Dans le référentiel du centre de masse, le problème à deux corps avec interaction devient équivalent à un problème à une seule particule en interaction avec une barrière delta au centre. Cette reformulation permet d’interpréter l’effet de l’interaction comme une condition de raccord sur la fonction d’onde, tout en respectant la symétrie bosonique.

Nous revenons ensuite aux coordonnées du laboratoire afin d’introduire naturellement la forme des solutions imposée par l’Ansatz de Bethe. Cela nous conduit aux équations dites de Bethe, qui relient les quasimoments des particules à travers des conditions de périodicité modifiées par l’interaction.

Une fois les notations bien établies, nous généralisons le raisonnement au cas de \(N\) particules, pour obtenir l’Hamiltonien de Lieb-Liniger complet ainsi que la forme générale de l’Ansatz de Bethe. Les solutions ainsi construites permettent non seulement de déterminer le spectre de l’Hamiltonien, mais aussi de calculer des observables physiques importantes, telles que l’impulsion totale ou le nombre de particules.

Enfin, nous introduisons la notion de distribution de rapidité, outil essentiel dans l’étude des états d’énergie minimale (états fondamentaux) et dans la description thermodynamique du système. Ce cadre servira de base aux développements ultérieurs sur les gaz intégrables à température finie et les états stationnaires après quench quantique.

\section{Description du modèle de Lieb-Liniger}

\subsection{Introduction au modèle de gaz de Bose unidimensionnel et Hamiltonien du modèle}

\subsubsection{De la première à la seconde quantification}

\paragraph{Introduction.}

La mécanique quantique se développe historiquement en deux grandes étapes : la \emph{première quantification}, aussi appelée quantification canonique, et la \emph{seconde quantification}. Comprendre ces deux cadres est essentiel pour aborder les systèmes quantiques complexes, en particulier ceux où le nombre de particules peut varier.

%La mécanique quantique s’est historiquement développée en deux étapes : la \emph{première quantification}, aussi appelée quantification canonique, puis la \emph{seconde quantification}. Comprendre ces deux cadres est essentiel pour aborder les systèmes à nombre de particules variable.


%\vspace{0.5cm}

\paragraph{Première quantification (quantification canonique, particule unique).}

La première quantification est la mécanique quantique standard, celle que vous avez rencontrée dès vos premiers cours. Elle consiste à quantifier un système classique décrit par des variables dynamiques telles que la position $x$ et la quantité de mouvement $p$. On procède en remplaçant ces variables par des {\bf opérateurs hermitiens} $\operator{x}$ et %$\operator{p}$
\begin{eqnarray}
	\operator{p} \doteq -i\hbar \operator{\partial}_x,	\label{chap.1.rapel.1}
\end{eqnarray}
où $\hbar$ est la constante de Planck réduite, satisfaisant la {\bf relation de commutation canonique} fondamentale $[\operator{x}, \operator{p}] = i\hbar$. L’état du système est alors décrit par une {\bf fonction d’onde} $\psi(x,t)$, solution de {\bf l’équation de  Schrödinger} indépendante du nombre de particules :
\begin{eqnarray}
\quad i \hbar \frac{\partial \psi }{\partial t}  &= \operator{\mathcal{H}} \psi,\label{chap.1.rapel.2}
\end{eqnarray}

avec $\operator{\mathcal{H}}$ l’opérateur hamiltonien. 

\begin{mdframed}[
	linewidth=0.5pt, 
	backgroundcolor=gray!5, 
	roundcorner=50pt,	
	innerleftmargin=5pt,
    innerrightmargin=5pt,
    innertopmargin=-10pt,
    innerbottommargin=2pt,
    leftmargin=2pt,
    rightmargin=2pt
	]
\subparagraph{Exemple : particule libre en une boite à une dimension.} 
	{~}\\
	
	Dans le cas d’une particule libre de masse $m$ se déplaçant en une dimension, l’Hamiltonien est constitué uniquement du terme cinétique $\operator{\mathcal{H}} = \operator{p}^2 / 2m$. En représentation position, où l’opérateur quantité de mouvement s’écrit comme dans l’équation \eqref{chap.1.rapel.1}, l’Hamiltonien prend alors la forme différentielle :
	\begin{eqnarray}
		\operator{\mathcal{H}} = -\frac{\hbar^2}{2m} \partial_x^2.
	\end{eqnarray}
	Les états propres stationnaires de \eqref{chap.1.rapel.2} dépendant du temps sont de la forme $\psi_k(x,t) = \varphi_k(x)\,e^{-i\varepsilon(k)t/\hbar}$ où $\varphi_k(x)$ est une fonction propre de l’hamiltonien,  soit de  l’équation stationnaire  $\operator{\mathcal{H}}\varphi_k = \varepsilon(k)\varphi_k$ \ie pour une particule libre:
	\begin{eqnarray}
		\frac{\hbar^2}{2m} \partial_x^2 \varphi_k = \varepsilon(k) \varphi_k,
	\end{eqnarray}
	avec $\varepsilon(k)$ l’énergie associée à une onde plane de nombre d’onde $k$
	\begin{eqnarray}
		\varepsilon(k) = \frac{\hbar^2 k^2 }{2 m}.
	\end{eqnarray}
	Les fonctions propres spatiales $\varphi_k(x)$ de l’hamiltonien libre s’écrivent comme des combinaisons linéaires d’ondes planes  
	\begin{eqnarray}
		\varphi_k(x) = a e^{-i k x} + b e^{i k x},~~ \mbox{avec}\quad  (a,b) \in \mathbb{C}^2.
	\end{eqnarray}
	Si la particule est confinée dans une boîte de longueur $L$ avec des conditions aux limites périodiques (ie $\varphi_k(x+L) = \varphi_k(x)$), alors le spectre de $k$ est quantifié : 
	\begin{eqnarray}
		kL \in 2\pi\mathbb{Z}.
	\end{eqnarray}
	Le problème est équivalent à celui d’une particule libre sur un cercle de périmètre $L$.\\
	
	\medskip
	
	Les solutions générales de l’équation de Schrödinger s’écrivent alors comme une superposition d’états propres  $\psi = \sum_k c_k \psi_k $.  

%On résume :
%\begin{eqnarray}
%	,~~ , ~~\varphi_k(x) = a e^{-i k x} + b e^{i k x},~~ kL \in 2\pi\mathbb{Z}.\label{chap.1.recap}
%\end{eqnarray}
\end{mdframed}

Ces solutions correspondent à des {\bf états non liés} (ou états de diffusion) : la particule est délocalisée sur tout l’espace (le cercle), sans structure particulière.

La fonction $\varphi_k(x)$ est supposée normalisée dans l’espace des états quantifiés (boîte finie) :
\(
\int_0^L dx \, \varphi_{k'}^\ast(x)\, \varphi_k(x) = \delta_{k,\pm k'}.
\)
avec  $ \vert a \vert^2 + \vert b \vert^2 = L^{-1}$.
Et dans le sous espace engendré pas $x \mapsto e^{-ikx}$ et $x \mapsto e^{ikx}$ (de deux dimension si $k \neq 0$ et une dimension si $k$ =0), $x \mapsto \pm ( b^\ast e^{-ikx} - a^\ast e^{ikx} )$ est orthogonale à  $\varphi_k$ pour $k \neq 0$.
\begin{mdframed}[
	linewidth=0.5pt, 
	backgroundcolor=gray!5, 
	roundcorner=50pt,	
	innerleftmargin=5pt,
    innerrightmargin=5pt,
    innertopmargin=-10pt,
    innerbottommargin=2pt,
    leftmargin=2pt,
    rightmargin=2pt
	]
\subparagraph{Remarque.} On choisie  \( a = \frac{1}{\sqrt{L}} \) et \( b = 0 \)), alors
\(
\varphi_k(x) = \frac{1}{\sqrt{L}}\, e^{i k x}
\)
est une onde plane. 

\end{mdframed}

Avec le formalisme de Dirac, la fonction d’onde $\varphi_k$ est représentée par le ket $\ket{k}$ normé (i.e. $\langle k' \vert k \rangle = \delta_{k, k'}$, où $\delta_{p,q}$ est le symbole de Kronecker)
, et l’équation de Schrödinger s’écrit :
\(
\operator{\mathcal{H}}_1 \ket{k} = \varepsilon(k) \ket{k}.
\)
En appliquant le bra $\bra{x}$ de part et d’autre, on obtient :
\(
\bra{x} \operator{\mathcal{H}}_1 \ket{k} = \varepsilon(k) \langle x \vert k \rangle,
\)
où $\varphi_k(x) = \langle x \vert k \rangle$ est la représentation positionnelle de l’état $\ket{k}$.



%\begin{mdframed}[
%	linewidth=0.5pt, 
%	backgroundcolor=gray!5, 
%	roundcorner=50pt,	
%	innerleftmargin=5pt,
%    innerrightmargin=5pt,
%    innertopmargin=-10pt,
%    innerbottommargin=2pt,
%    leftmargin=2pt,
%    rightmargin=2pt
%	]
%\subparagraph{Remarque.} Si l’on choisit une base orthonormée d’états propres (par exemple en fixant \( a = \frac{1}{\sqrt{L}}, b = 0 \)), alors
%\(
%\varphi_k(x) = \frac{1}{\sqrt{L}}\, e^{i k x}, \quad \text{et donc} \quad \langle k \vert x \rangle = \varphi_k^\ast(x) = \frac{1}{\sqrt{L}}\, e^{-i k x},
%\)
%ce qui est bien une onde plane. 
%En revanche, dans l’écriture générale \( \varphi_k(x) = a\, e^{i k x} + b\, e^{-i k x} \), la fonction \( \langle k \vert x \rangle = \varphi_k^\ast(x) \) n’est \emph{pas nécessairement} une onde plane, car \( \varphi_k(x) \) n’est pas normalisée.
%\end{mdframed}

\begin{mdframed}[
	linewidth=0.5pt, 
	backgroundcolor=gray!5, 
	roundcorner=50pt,	
	innerleftmargin=5pt,
    innerrightmargin=5pt,
    innertopmargin=1pt,
    innerbottommargin=2pt,
    leftmargin=2pt,
    rightmargin=2pt
	]
La base $\{\ket{x}\}$ étant continue, et les états $\{\ket{k}\}$ quantifiés (par exemple dans une boîte de taille finie avec conditions aux limites périodiques), les relations de changement de base s’écrivent :
\begin{eqnarray}
	\ket{k} = \int dx \, \varphi_k(x) \ket{x}, \qquad   
	\ket{x} = \sum_k \varphi_k^\ast(x) \ket{k},
\end{eqnarray}
avec $\varphi_k^\ast(x) = \langle k \vert x \rangle$. L’état $\ket{x}$ est relié aux états $\ket{k}$ par une transformation de Fourier discrète. Ces formules montrent que les états $\ket{k}$ sont les composantes de Fourier de l’état $\ket{x}$.
\end{mdframed}






\subparagraph{De la particule unique aux systèmes à $N$ particules.}

Pour un système composé de $N$ particules identiques, une approche naturelle consiste à introduire une fonction d’onde $\varphi(x_1, \dots, x_N)$ dépendant de $N$ variables , symétrique pour des bosons ou antisymétrique pour des fermions sous l’échange de deux coordonnées $x_i \leftrightarrow x_j$, solution de l’équation de Schrödinger à $N$ corps. %Dans le cas bosonique, des interactions à courte portée peuvent être modélisées par un potentiel de type Dirac :

%\begin{equation}
%V_{\text{int}}(x_1, \dots, x_N) = g \sum_{i<j} \delta(x_i - x_j),
%\end{equation}

%où $g$ est un paramètre d’interaction contrôlant l’intensité des collisions. 
Toutefois, cette description devient rapidement inextricable lorsque le nombre de particules augmente, ou lorsque le système permet la création et l’annihilation de particules, comme dans un milieu ouvert ou en contact avec un bain thermique.





%{\color{blue} \paragraph{Seconde quantification.}
%
%%Dans ce formalisme, l’espace des états est une **somme directe d’espaces à $N$ particules**, et chaque état est décrit par son occupation des modes quantiques. Les opérateurs $\hat{a}_k^\dagger$ et $\hat{a}_k$ créent et détruisent une particule dans l’état d’onde plane de moment $k$, satisfaisant les relations de commutation (bosons) ou d’anticommutation (fermions) :
%%\begin{equation}
%%[\hat{a}_k, \hat{a}_{k'}^\dagger] = \delta_{k,k'} \quad \text{(bosons)}.
%%\end{equation}
%
%%L’hamiltonien d’un gaz de particules libres s’écrit alors simplement :
%%\begin{equation}
%%\hat{\mathcal{H}} = \sum_k \varepsilon(k) \, \hat{a}_k^\dagger \hat{a}_k,
%%\end{equation}
%%avec $\varepsilon(k) = \frac{\hbar^2 k^2}{2m}$ comme dans la quantification canonique.
%
%\paragraph{Vers le Bethe ansatz.}
%
%Ce formalisme devient particulièrement utile lorsque des interactions entre particules sont introduites. Dans le cas d’un **gaz de bosons en une dimension avec interactions de contact**, le système reste exactement soluble : la solution repose sur une **superposition cohérente d’ondes planes symétrisées**, ajustées par les conditions d’interaction.
%
%C’est l’idée fondamentale du **Bethe ansatz**, qui généralise la solution d’une particule libre sur un cercle à $N$ particules **avec collisions élastiques**. On y retrouve des relations de quantification du type :
%\begin{equation}
%k_j L + \sum_{\substack{\ell=1 \\ \ell \neq j}}^N \theta(k_j - k_\ell) = 2\pi n_j,
%\end{equation}
%où $\theta$ est une phase de diffusion et les $n_j \in \mathbb{Z}$.
%
%On passe ainsi des conditions de périodicité simples à des **conditions de type Bethe**, qui encodent les effets des interactions sous forme de **conditions de compatibilité entre les moments**.
%
%}

\subsubsection{Seconde quantification}

Pour dépasser ces limitations, on adopte le \textbf{formalisme de la seconde quantification}, dans lequel l’état du système est décrit non plus par une fonction d’onde mais par un vecteur dans un espace de Fock. Les opérateurs de création et d’annihilation remplacent alors les variables dynamiques classiques et permettent une description unifiée et élégante des systèmes à nombre variable de particules.

%\paragraph{Structure de l’espace des états de Fock.}
%Dans ce formalisme, l’espace des états est une {\bf somme directe d’espaces à $N$ particules}, et chaque état est décrit par l’occupation des différents modes quantiques. Les opérateurs $\operator{a}_k^\dagger$ et $\operator{a}_k^{}$ créent et annihilent une particule dans l’état d’onde plane de moment $k$ :
%\begin{eqnarray*}
%	\ket{k} & = & \operator{a}_k^\dagger \ket{\emptyset} ~=~ \text{état avec une particule dans le mode } k,	
%\end{eqnarray*}
%où \(\ket{\emptyset}\) est le vide quantique de Fock, défini par :
%\begin{eqnarray}
%	\forall k \in \mathbb{R}\colon \qquad \operator{a}_k \ket{\emptyset} = 0 ,\quad  \langle \emptyset\vert \emptyset \rangle = 1, \label{chap:eq.vide.fock.k}
%\end{eqnarray}
%où \( \operator{a}_\lambda \) peut ici estre une notation générique désignant \( \operator{b}_\lambda \) pour les bosons, ou \( \operator{c}_\lambda \) pour les fermions,
%et satisfaisant les relations de commutation (pour les bosons) ou d’anticommutation (pour les fermions). Dans ce qui suit, nous nous restreignons au cas bosonique. \subparagraph{Relations de commutation bosoniques.} Les relations de commutation bosoniques fondamentales sont alors :
%\begin{eqnarray}
%[\operator{a}_k^{}, \operator{a}_{k'}^{}]= [\operator{a}_k^\dagger, \operator{a}_{k'}^\dagger]= 0 ,\qquad [\operator{a}_k^{}, \operator{a}_{k'}^\dagger] = \operator{\delta}_{k,k'},\label{chap:1:com.1.k}
%\end{eqnarray}
%où $\operator{\delta}_{k,k'}$ est le symbole de Kronecker, qui vaut $1$ si $k = k'$ et $0$ sinon.\\

%%%%%%%%%%%%%%%%%%%%%%%%
\paragraph{Structure de l’espace des états de Fock.}
Dans ce formalisme, l’espace des états est une {\bf somme directe d’espaces à $N$ particules}, et chaque état est décrit par l’occupation des différents modes quantiques. Les opérateurs $\operator{a}_k^\dagger$ et $\operator{a}_k$ créent et annihilent une particule dans l’état d’onde plane de moment $k$ :
\begin{eqnarray*}
	\ket{k} & = & \operator{a}_k^\dagger \ket{\emptyset} ~=~ \text{état avec une particule dans le mode } k,	
\end{eqnarray*}
où \(\ket{\emptyset}\) désigne le vide quantique de Fock, défini par :
\begin{eqnarray}
	\forall k \in \mathbb{R}\colon \qquad \operator{a}_k \ket{\emptyset} = 0 ,\quad  \langle \emptyset \vert \emptyset \rangle = 1. \label{chap:eq.vide.fock.k}
\end{eqnarray}
Le symbole \( \operator{a}_\lambda \) représente ici de manière générique soit l’opérateur \( \operator{b}_\lambda \) pour les bosons, soit \( \operator{c}_\lambda \) pour les fermions, et satisfait respectivement les relations de commutation (pour les bosons) ou d’anticommutation (pour les fermions). Dans ce qui suit, nous nous restreignons au cas bosonique.

\subparagraph{Relations de commutation bosoniques.} Les relations de commutation fondamentales pour les bosons sont :
\begin{eqnarray}
	[\operator{b}_k, \operator{b}_{k'}] = [\operator{b}_k^\dagger, \operator{b}_{k'}^\dagger] = 0 ,\qquad [\operator{b}_k, \operator{b}_{k'}^\dagger] = \operator{\delta}_{k,k'}, \label{chap:1:com.1.k}
\end{eqnarray}
où $\operator{\delta}_{k,k'}$ est le symbole de Kronecker, valant $1$ si $k = k'$ et $0$ sinon.
%%%%%%%%%%%%%%%%%%%%%%%%%%%%%%%%%%%%%%%%

%\vspace{1em}
\paragraph{Nature du champ quantique.}
La seconde quantification généralise ce cadre en permettant de traiter des systèmes où le nombre de particules n’est pas fixé, ce qui est fréquent en physique des particules, des champs quantiques, ou des gaz quantiques.

L’idée principale est de ne plus quantifier directement les particules, mais le \emph{champ quantique} associé. Les états d’une particule unique deviennent alors des états d’occupation dans un espace de Fock, qui décrit l’ensemble des configurations possibles avec zéro, une, ou plusieurs particules.



\subparagraph{Champs de Bose.}
Le gaz de Bose unidimensionnel est décrit dans le cadre de la théorie quantique des champs par un champ bosonique canonique \( \operator{\Psi}(x) \), qui agit sur l’espace de Fock des états du système. Ce champ quantique encode l’annihilation d’une particule en \( x \), et son adjoint \( \operator{\Psi}^\dag(x) \) correspond à la création d’une particule en ce point. 
\begin{eqnarray}
	\vert x \rangle  & = & \operator{\Psi}^\dag (x)\ket{\emptyset} ~=~ \text{état avec une particule en } x,
\end{eqnarray}
et \(\ket{\emptyset}\) est le vide quantique de Fock défini par :
\begin{eqnarray}
	\forall x \in \mathbb{R}, \qquad \operator{\Psi}(x) \ket{\emptyset} = 0. \label{chap:eq.vide.fock}
\end{eqnarray}

\subparagraph{Relations de commutation bosoniques.}
Ces champs satisfont les relations de commutation canoniques à temps égal :
%\begin{eqnarray}
%	\left . \begin{array}{rcl}
%		[ \operator{\Psi}(x),  \operator{\Psi}^\dagger(y) ]  &=&  \operator{\delta}(x - y) \\
%		\left [ \operator{\Psi}(x),  \operator{\Psi}(y) \right ]   =  [ \operator{\Psi}^\dag(x),  \operator{\Psi}^\dag(y) ]  &=&  0 
%	\end{array} \right . \label{chap:1:com.1}
%\end{eqnarray}
\begin{eqnarray}
	 [ \operator{\Psi}(x),  \operator{\Psi}(y)  ]   =  [ \operator{\Psi}^\dag(x),  \operator{\Psi}^\dag(y) ]  =  0,   & & [ \operator{\Psi}(x),  \operator{\Psi}^\dagger(y) ]  =  \operator{\delta}(x - y) ,\label{chap:1:com.1}
\end{eqnarray}
où $\operator{\delta}(x - y)$ est la fonction delta de Dirac.  
Ces relations expriment le caractère bosonique des excitations du champ.

\paragraph{État à $N$ particules.} Soient $N$ bosons dans les états $\{ k_1 , \cdots , k_N \}$ (un boson dans l’état $k_1$, un autre dans $k_2$, etc.) et aux positions $\{ x_1 , \cdots , x_N \}$ (un boson en $x_1$, un autre en $x_2$, etc.). Leurs états s’écrivent alors :
\begin{eqnarray}
	\ket{ \{ k_1 , \cdots , k_N \}} = \frac{1}{\sqrt{N!}} \operator{b}_{k_1}^\dag\, \cdots \, \operator{b}_{k_N}^\dag \ket{\emptyset}, \quad \ket{\{x_1 , \cdots , x_N\}} = \frac{1}{\sqrt{N!}} \operator{\Psi}^\dag(x_1)\, \cdots \, \operator{\Psi}^\dag(x_N) \ket{\emptyset}	, \label{eq.chap.1.ket.N}
\end{eqnarray}
où le facteur \( 1/\sqrt{N!} \) traduit le caractère d’indiscernabilité des bosons et garantit la symétrisation correcte de l’état.

\subparagraph{Changement de base.}
On peut relier les opérateurs de création/annihilation dans la base des ondes planes aux opérateurs de champ via :
\begin{eqnarray}
	\operator{b}_k^\dagger = \int dx \, \varphi_k(x) \operator{\Psi}^\dagger(x), \qquad 
	\operator{\Psi}^\dagger(x) = \sum_k \varphi_k^\ast(x)\operator{b}_k^\dagger.\label{eq.chap.1.TF.1}
\end{eqnarray}
Le champ quantique $\operator{\Psi}(x)$ est relié aux opérateurs de moment $\operator{b}_k$ par une transformation de Fourier. Ces formules montrent que les opérateurs $\operator{b}_{k}$ sont les composantes de Fourier du champ $\operator{\Psi}(x)$.

%où $\varphi_k(x)$ est la fonction d’onde d’un état d’énergie bien définie \( \ket{k} \) dans la représentation positionnelle.
Ainsi, un état à \(N\) bosons dans la base \( \ket{k}^{\otimes N} \) peut s’écrire :
\begin{eqnarray}
	\ket{\{k_1 , \cdots , k_N\}} = \frac{1}{\sqrt{N!}} \int dx_1 \cdots dx_N \, \varphi_{\{k_a\}} ( x_1 , \cdots , x_N ) \, \hat{\Psi}^\dag(x_1) \cdots \hat{\Psi}^\dag(x_N) \ket{\emptyset},
\end{eqnarray}
où \( \{k_a\} \equiv \{k_1, \dots, k_N\} \), et la fonction d’onde symétrisée s’écrit :
\(
	\varphi_{\{k_a\}} ( x_1 , \cdots , x_N ) = \frac{1}{\sqrt{N!}} \sum_{\sigma \in \operator{S}_N } \prod_{i=1}^N \varphi_{k_{\sigma(i)}}(x_i),
\) 
avec $\operator{S}_N $  le groupe symétrique d'ordre $N$ mais aussi :
\begin{eqnarray}
	\varphi_{\{k_a\}} ( x_1 , \cdots , x_N ) = \frac{1}{\sqrt{N!}} \bra{\emptyset} \hat{\Psi}(x_1) \cdots \hat{\Psi}(x_N) \ket{\{k_1, \cdots , k_N\}}.
\end{eqnarray}



\subsubsection{Operateur. }


\paragraph{Opérateur à un corps.}

Soit \( \operator{f} \) un opérateur à une particule, dont les éléments de matrice dans une base orthonormée \( \{ \ket{k} \} \) sont donnés par \( f_{\lambda\nu} = \langle \lambda \vert \operator{f} \vert \nu \rangle \). Un opérateur symétrique à \( N \) particules correspondant à la somme des actions de \( \operator{f} \) sur chacune des particules s’écrit en première configuration  :
\(
	\operator{F} = \sum_{i=1}^N \operator{f}^{(i)},
\)
où \( \operator{f}^{(i)} \) désigne l’action de \( \operator{f} \) sur la $i^\text{e}$ particule uniquement. En base de Dirac, cela donne :
\(
	\operator{f}^{(i)} = \sum_{\lambda, \nu} f_{\lambda\nu} \, \ket{i\!:\!\lambda} \bra{i\!:\!\nu},
\)
où \( \ket{i\!:\!\lambda} \) représente un état où seule la $i^\text{e}$ particule est dans l’état \( \lambda \). (Par construction, l’opérateur \( \operator{F} \) commute avec les projecteurs de symétrisation \( \operator{S}_N \) (bosons) et d’antisymétrisation \( \operator{A}_N \) (fermions).)
On peut montrer que la somme des projecteurs agissant sur chaque particule s’identifie à une combinaison d’opérateurs de création et d’annihilation :
\(
	\sum_{i=1}^N \ket{i\!:\!\lambda} \bra{i\!:\!\nu} = \operator{a}^\dagger_\lambda \operator{a}_\nu^{},
\)
(où \( \operator{a}_\lambda \) peut ici estre une notation générique désignant \( \operator{b}_\lambda \) pour les bosons, ou \( \operator{c}_\lambda \) pour les fermions).

On en déduit que l’opérateur à un corps \( \operator{F} \) peut se réécrire dans le formalisme de la seconde quantification comme :
\begin{eqnarray}
	\operator{F} = \sum_{\lambda, \nu} f_{\lambda\nu} \, \operator{a}^\dagger_\lambda \operator{a}_\nu^{}.
\end{eqnarray}


\subparagraph{Exemples d’opérateurs à un corps.}

Si l’on sait diagonaliser l’opérateur \( \operator{f} \), c’est-à-dire si l’on peut écrire :
\(
	\operator{f} = \sum_k f_k \ket{k} \bra{k},
\)
alors l’opérateur à $N$ corps associé s’écrit :
\(
	\operator{F} = \sum_k f_k \, \operator{a}^\dagger_k \operator{a}_k^{} = \sum_k f_k \, \operator{n}_k,
\)
où \( \operator{n}_k = \operator{a}^\dagger_k \operator{a}_k \) est l’opérateur nombre de particules dans le mode \( k \). On obtient ainsi une forme diagonale de \( \operator{F} \) en seconde quantification.
\begin{mdframed}[linewidth=0.5pt, backgroundcolor=gray!5, roundcorner=5pt]
Un exemple immédiat est celui des particules libres. Si l’on diagonalise le problème à une particule selon :
\(
	\operator{\mathcal{H}}_1 \ket{k} = \varepsilon(k) \ket{k},
\)
alors l’énergie totale du système correspond ici uniquement à son énergie cinétique, et s’écrit :
\begin{equation}
	\operator{K} = \sum_{k} \varepsilon(k) \, \operator{a}^\dagger_k \operator{a}_k^{}.\label{eq.chap.1.cinietique.1}
\end{equation}

Et pour $N$ particules, en écrivant l’état sous la forme~\eqref{eq.chap.1.ket.N}, en utilisant les relations de commutation~\eqref{chap:1:com.1.k} et la définition de l’état de Fock~\eqref{chap:eq.vide.fock.k}, on trouve que $\ket{\{k_1, \cdots, k_N\}}$ est un état propre de $\operator{K}$ associé à l'énergie $\left( \sum_{i = 1}^N \varepsilon(k_i) \right)$, c’est-à-dire :
\begin{eqnarray}
	\operator{K} \ket{\{k_1, \cdots, k_N\}} = \left( \sum_{i = 1}^N \varepsilon(k_i) \right) \ket{\{k_1, \cdots, k_N\}}.\label{eq.chap.1.cinietique.2}
\end{eqnarray}
\end{mdframed}

\paragraph{Forme champ des opérateurs à un corps.}

Les opérateurs à plusieurs corps peuvent être exprimés de manière remarquable à l’aide des opérateurs de champ, d’une façon physiquement transparente qui rappelle les formules bien connues du cas à une particule.

La forme générale d’un opérateur à un corps s’écrit :
\begin{eqnarray}
\operator{F} = \int dx \, dx' \, \operator{\Psi}^\dagger(x) \, \bra{ x} \operator{f} \ket{x'} \, \operator{\Psi}(x').
\end{eqnarray}%où \( \hat{f} \) est l’opérateur à un corps exprimé dans la base position, et \( \hat{\psi}^\dagger(\vec{r}) \), \( \hat{\psi}(\vec{r}) \) sont les opérateurs de création et d’annihilation d’une particule au point \( \vec{r} \).
\begin{mdframed}[
	linewidth=0.5pt, 
	backgroundcolor=gray!5, 
	roundcorner=50pt,	
	innerleftmargin=5pt,
    innerrightmargin=5pt,
    innertopmargin=-10pt,
    innerbottommargin=2pt,
    leftmargin=2pt,
    rightmargin=2pt
]
\subparagraph{Énergie cinétique totale.}

Pour des particules non relativistes, l’énergie cinétique élémentaire s’écrit $\operator{f} = \frac{\hbar^2 \operator{p}^2}{2m}$. À l’échelle du champ quantique, l’énergie cinétique totale prend la forme opératorielle :
\begin{eqnarray}
\operator{K} =  -\frac{\hbar^2}{2m} \int dx \, \operator{\Psi}^\dagger(x) \, \operator{\partial}_x^2 \operator{\Psi}(x)
= \frac{\hbar^2}{2m} \int dx \, \operator{\partial}_x \operator{\Psi}^\dagger(x) \cdot \operator{\partial}_x \operator{\Psi}(x). \label{eq.chap.1.cinietique.3}
\end{eqnarray}

Le champ quantique $\operator{\Psi}(x)$ est relié aux opérateurs de moment $\operator{b}_k$ par une transformation de Fourier. En injectant l'expression \eqref{eq.chap.1.TF.1} dans \eqref{eq.chap.1.cinietique.3}, on retrouve la forme discrète \eqref{eq.chap.1.cinietique.1}, cette fois exprimée en termes des opérateurs $\operator{b}_k$.

Lorsque cet Hamiltonien agit sur l’état de Fock à $N$ particules $\ket{\{k_1 , \cdots , k_N\}}$, les règles de commutation (\ref{chap:1:com.1}) ainsi que la définition des états de Fock (\ref{chap:eq.vide.fock}) impliquent (cf. Annexe \ref{annex:N.part}) :
\begin{eqnarray}
\operator{K}\ket{k_1 , \cdots , k_N } =  \int d^N z \, \operator{\mathcal{K}}_N \, \varphi_{\{k_a\}}(z_1 , \cdots , z_N ) \operator{\Psi}(z_1) \cdots \operator{\Psi}^\dag(z_N) \ket{\emptyset}
\end{eqnarray}
avec :
\[
	\operator{\mathcal{K}}_N = \sum_{i=1}^N \frac{\operator{p}_i^2}{2m},
\]
où \( \operator{p}_i \) désigne l’opérateur impulsion de la \( i^\text{ème} \) particule.
\end{mdframed}




\paragraph{Opérateurs à deux corps}

Nous considérons à présent les termes d’interaction impliquant deux particules , $\operator{v}$ , dont les éléments de matrices sont donnés par $v_{\alpha \beta \gamma \delta} = \bra{ 1 : \alpha; 2 : \beta } \operator{v}\ket{ 1 : \gamma; 2 : \delta }$ , où $\ket{ i : \gamma; j : \delta }$ représente l'état où la $i^\text{e}$  particules est dans l'état $\gamma$ et la $j^\text{e}$ dans l'état $\delta$  . Ceux-ci correspondent à des opérateurs de la forme :
\(
    \operator{V} = \sum_{j < i} \operator{v}^{(i, j)} = \frac{1}{2} \sum_{i, j \ne i} \operator{v}^{(i, j)}
    \label{eq:V2corps}.
\)
avec $\operator{v}^{(i, j)}$ désigne l’interaction à deux corps entre les $i^\text{e}$ et $j^\text{e}$ particules , exprimés dans la base à deux états :
\(
	\operator{v}^{(i, j)} = \sum_{\alpha,\beta,\delta,\gamma} \ket{i : \alpha; j : \beta }v_{\alpha \beta \gamma \delta} \bra{ i : \gamma; j : \delta }.
    %v_{\alpha \beta \gamma \delta} = \bra{ i : \alpha; j : \beta } \operator{v}^{(i,j)} \ket{ i : \gamma; j : \delta }.
    \label{eq:matriceV}
\)
On peut réécrire l’opérateur \( \operator{V} \) en termes d’opérateurs de création et d’annihilation comme suit :
\begin{equation}
    \operator{V} = \frac{1}{2} \sum_{\alpha, \beta, \gamma, \delta} v_{\alpha \beta \gamma \delta} \, \operator{a}^\dagger_\alpha \operator{a}^\dagger_\beta \operator{a}_\delta^{} \operator{a}_\gamma^{}.
    \label{eq:Vcreation}
\end{equation}

Cette forme est particulièrement utile pour le traitement des interactions dans l’espace de Fock, notamment en théorie des champs et en physique des gaz quantiques.

\subparagraph{Expression générale d’un terme à deux corps. }

Un terme d’interaction à deux corps général peut s’écrire :
\begin{equation}
    \operator{V} = \frac{1}{2} \int dx_1^{} \, dx_2^{} \, dx_1' \, dx_2' \; 
    \bra{ 1 : x_1^{}, 2 : x_2^{} } \operator{v} \ket{ 1 : x_1', 2 : x_2' } \,
    \operator{\Psi}^\dagger(x_1^{}) \, \operator{\Psi}^\dagger(x_2^{}) \, 
    \operator{\Psi}(x_2') \, \operator{\Psi}(x_1')
    \label{eq:V_general}
\end{equation}

\begin{mdframed}[
	linewidth=0.5pt, 
	backgroundcolor=gray!5, 
	roundcorner=50pt,	
	innerleftmargin=5pt,
    innerrightmargin=5pt,
    innertopmargin=-10pt,
    innerbottommargin=2pt,
    leftmargin=2pt,
    rightmargin=2pt
	]
\subparagraph{Interactions ponctuelles.} 
Dans le cas d’une interaction ne dépendant que de la distance relative entre deux particules, cette expression se simplifie :
\(
     \operator{V} = \frac{1}{2} \sum_{i, j \ne i}  \operator{v}(x_i^{} - x_j^{}) = 
    \frac{1}{2} \int dx_1^{} \, dx_2^{} \; v(x_1^{} - x_2^{}) \,
    \operator{\Psi}^\dagger(x_1^{}) \, \operator{\Psi}^\dagger(x_2^{}) \, 
    \operator{\Psi}(x_2^{}) \, \operator{\Psi}(x_1^{})
    \label{eq:V_local}
\) soit pour des interactions ponctuelles :	
\begin{eqnarray}
	\quad \operator{V}  = \frac{g}{2} \int dx \,
    \operator{\Psi}^\dagger(x) \, \operator{\Psi}^\dagger(x) \, 
    \operator{\Psi}(x) \, \operator{\Psi}(x)  		
\end{eqnarray}
et quand on l'applique à l'état $\ket{\{k_1 , \cdots , k_N\}} $ , les règles de commutations (\ref{chap:1:com.1}) et la définition d'état de Fock (\ref{chap:eq.vide.fock}) impliquent que (cf Annex \ref{annex:N.part})
\begin{eqnarray}
\operator{V}\ket{\{k_1 , \cdots , k_N\}} =  \int d^Nz \, \operator{\mathcal{V}}_N \varphi_{\{k_a\}}(z_1 , \cdots , z_N )\operator{\Psi}(z_1)\cdots \operator{\Psi}^\dag(z_N) \ket{\emptyset} 
\end{eqnarray}
avec 
\(
	\operator{\mathcal{V}}_N 	
 = g\sum_{1\leq i < j \leq N } \operator{\delta}(x_i - x_j)	
\)
où \( g \) est la constante de couplage.
\end{mdframed}


%Le hamiltonien général décrivant des particules identiques en interaction s’écrit alors :
%\begin{equation}
%    \hat{H} = \int d\vec{r} \; \hat{\psi}^\dagger(\vec{r}) 
%    \left( -\frac{\hbar^2}{2m} \Delta + u(\vec{r}) - \mu \right) 
%    \hat{\psi}(\vec{r})
%    + \frac{1}{2} \int d\vec{r} \, d\vec{r}' \; v(\vec{r} - \vec{r}') \,
%    \hat{\psi}^\dagger(\vec{r}') \, \hat{\psi}^\dagger(\vec{r}) \,
%    \hat{\psi}(\vec{r}) \, \hat{\psi}(\vec{r}')
%    \label{eq:H_general}
%\end{equation}

%\noindent
%Bien que cette expression ait une interprétation physique très claire, il est important de garder à l'esprit que \( \hat{H} \) et \( \hat{\psi} \) sont des objets du formalisme à plusieurs corps.



%%%%%%%%%%%%%%%%
%........................

%\subsubsection{Seconde quantification}



%\paragraph{Hamiltoniens en seconde quantification.}
%\subparagraph{Terme à un corps.}
%Un hamiltonien à un corps, correspondant à une énergie cinétique ou un potentiel externe, s’écrit :
%\[
%\hat{\mathcal{H}}_1 = \int dx\, \operator{\Psi}^\dagger(x) \hat{h}(x) \operator{\Psi}(x),
%\]
%où \( \hat{h}(x) \) est l’opérateur d’un corps (ex. : \( -\frac{\hbar^2}{2m} \partial_x^2 + V(x) \)).

%\subparagraph{Terme à deux corps.}
%Les interactions entre particules, modélisées par une interaction à deux corps \( V(x - y) \), s’expriment comme :
%\[
%\hat{\mathcal{H}}_2 = \frac{1}{2} \int dx\,dy\, \operator{\Psi}^\dagger(x) \operator{\Psi}^\dagger(y) V(x - y) \operator{\Psi}(y) \operator{\Psi}(x).
%\]


%.......................


\paragraph{Expression de l’Hamiltonien. }
L’hamiltonien dans ce formalisme s’écrit en termes des opérateurs de champ, par exemple pour l’énergie cinétique et les interactions ponctuelles avec $\hbar = m = 1 $  :

%Le Hamiltonien du modèle est donné par

%\begin{eqnarray}
%	\operator{H} & = & \int dx \, \left [ \operator{\partial}_x \operator{\Psi}^\dag (x)\operator{\partial}_x \operator{\Psi}(x) + c \operator{\Psi}^\dag (x) \operator{\Psi}^\dag (x) \operator{\Psi} (x) \operator{\Psi} (x) \right ] \label{chap:1:ham.mod}
%\end{eqnarray}

\begin{eqnarray}
	\operator{H} & = & \int dx \, \operator{\Psi}^\dag (x)\left [-\frac{1}{2}\operator{\partial}_x^2 + \frac{g}{2}  \operator{\Psi}^\dag (x) \operator{\Psi} (x) \right ] \operator{\Psi} (x) \label{chap:1:ham.mod}.
\end{eqnarray}

Quand on l'applique à l'état $\ket{\{\theta_1 , \cdots , \theta_N \}} $, avec $\theta_i$ homogène à des nombres d'onde ou à des vitesse , il vient que %, les règles de commutations (\ref{chap:1:com.1}) et la définition d'état de Fock (\ref{chap:eq.vide.fock}) impliquent que (cf Annex \ref{annex:N.part})
\begin{eqnarray}
\operator{H}\ket{\{\theta_1 , \cdots , \theta_N\}} =  \int d^Nz \, \operator{\mathcal{H}}_N \varphi_{\{\theta_a\}}(z_1 , \cdots , z_N )\operator{\Psi}(z_1)\cdots \operator{\Psi}^\dag(z_N) \ket{\emptyset} 
\end{eqnarray}
avec 
\(
	\operator{\mathcal{H}}_N 	
 =  \operator{\mathcal{K}}_N  +  \operator{\mathcal{V}}_N .	
\)


%où \( g \) est la constante de couplage. %Dans ce chapitre, nous considérons uniquement les propriétés du système à un instant donné, de sorte que la dépendance temporelle des champs est omise pour alléger l’écriture.

Ce formalisme est ainsi adapté pour décrire des condensats de Bose, des gaz quantiques, ou la création/annihilation de particules dans les champs quantiques.

\paragraph{Équation du mouvement associée.}

L’équation du mouvement du champ \( \Psi(x) \) est obtenue à partir de l’équation de Heisenberg :

\begin{eqnarray}
	i\operator{\partial}_t	\operator{\Psi} & = & [ \operator{\Psi} , \operator{H} ]
\end{eqnarray}

ce qui, après évaluation explicite du commutateur (\ref{chap:1:com.1}), conduit à :


%\begin{eqnarray}
%	i \operator{\partial}_t \operator{\Psi}	 & = & - \operator{\partial}_x^2 \operator{\Psi} + 2c \operator{\Psi}^\dag\operator{\Psi} \operator{\Psi}
%\end{eqnarray}

\begin{eqnarray}
	i \operator{\partial}_t \operator{\Psi}	 & = & - \frac{1}{2}\operator{\partial}_x^2 \operator{\Psi} + g \operator{\Psi}^\dag\operator{\Psi} \operator{\Psi}
\end{eqnarray}

est appelée l'équation de \textbf{Schrödinger non linéaire (NS)}.

Pour $g > 0$, l'état fondamental à température nulle est une sphère de Fermi. Seul ce cas sera considéré par la suite.

%\vspace{0.5cm}

\subsubsection*{Conclusion}

La première quantification est la base indispensable qui permet de comprendre le comportement quantique d’un nombre fixé de particules. La seconde quantification en est une extension naturelle, nécessaire pour décrire des systèmes plus complexes où le nombre de particules peut varier. Elle repose sur la quantification des champs, et l’introduction d’opérateurs créant ou détruisant ces particules, ouvrant ainsi la voie à la physique quantique des champs et à de nombreuses applications modernes.


\subsection{Opérateurs nombre de particules et moment dans la formulation quantique du gaz de Lieb-Liniger}

Dans cette section, nous nous intéressons aux opérateurs fondamentaux que sont le {\em nombre total de particules} $\operator{Q}$ et le {\em moment total} $\operator{P}$, dans le cadre du gaz de bosons unidimensionnel décrit par l’Hamiltonien de Lieb-Liniger. Après avoir introduit ces opérateurs dans le langage de la seconde quantification, nous montrons qu’ils sont {\em conservés} par la dynamique, et qu’ils admettent les {\em mêmes états propres} que l’Hamiltonien. Nous donnons ensuite leur expression dans la représentation à  $N$ particules, ainsi que la forme explicite de leurs valeurs propres en fonction des {\em rapidités} $\theta_a$ , illustrant la structure polynomiale typique des intégrales du mouvement dans les systèmes intégrables.

\subsubsection{Définition en seconde quantification}

Les opérateurs du nombre total de particules $\operator{Q}$ et du moment total $\operator{P}$ s’écrivent en seconde quantification comme suit :
\begin{eqnarray}
	\operator{Q}  =  \int \operator{\Psi}^\dag (x) \operator{\Psi} (x) \, d x, \quad 
	\operator{P}  =  - \frac{i}2 \int \left \{  \operator{\Psi}^\dag(x) \operator{\partial}_x \operator{\Psi}(x) - \left [ \operator{\partial}_x \operator{\Psi}^\dag(x)\right ] \operator{\Psi}(x)\right \} dx \label{eq.1.7}
\end{eqnarray}
Ces deux opérateurs sont {\em hermitiens}, et représentent des observables physiques fondamentales : le nombre de particules et la quantité de mouvement du système.

\subsubsection{Conservation et commutation}
Ces opérateurs commutent avec l’Hamiltonien $\operator{H}$ du modèle de Lieb-Liniger :
\begin{eqnarray}
[ \operator{H} , \operator{Q} ] = 0, \quad [ \operator{H} , \operator{P} ] = 0.
\end{eqnarray}
Ils constituent ainsi des intégrales du mouvement. Cette propriété est une manifestation de la symétrie translationnelle du système (pour $\operator{P}$) et de la conservation du nombre total de particules (pour $\operator{Q}$).

\begin{mdframed}[
	linewidth=0.5pt, 
	backgroundcolor=gray!5, 
	roundcorner=50pt,	
	innerleftmargin=5pt,
    innerrightmargin=5pt,
    innertopmargin=5pt,
    innerbottommargin=2pt,
    leftmargin=2pt,
    rightmargin=2pt
	]
	Nous verrons au chapitre 2 que cette situation s’étend à une {\bf \em infinité d’intégrales du mouvement} dans les systèmes intégrables, ce qui permettra de construire l’ensemble de Gibbs généralisé (GGE).
\end{mdframed}

\subsubsection{États propres et valeurs propres}
Les états propres $\ket{\{\theta_a\}}$, construits dans le cadre de la seconde quantification à partir de la solution du modèle de Lieb-Liniger, sont simultanément fonctions propres des opérateurs $\operator{Q}$, $\operator{P}$ et $\operator{H}$ :
\begin{eqnarray}
\operator{Q} \ket{\{\theta_a\}} = N \ket{\{\theta_a\}}, \quad
\operator{P} \ket{\{\theta_a\}} = \left( \sum_{a=1}^N \theta_a \right) \ket{\{\theta_a\}}, \
\operator{H} \ket{\{\theta_a\}} = \left( \frac{1}{2} \sum_{a=1}^N \theta_a^2 \right) \ket{\{\theta_a\}}.
\end{eqnarray}
Autrement dit, les valeurs propres associées à ces trois opérateurs sont données par :
\begin{eqnarray}
N = \sum_{a = 1}^N \theta_a^0, \quad p = \sum_{a = 1}^N \theta_a, \quad e = \frac{1}{2} \sum_{a = 1}^N \theta_a^2.
\end{eqnarray}
Cela illustre que les trois premières intégrales du mouvement du système — nombre, moment, énergie — peuvent être exprimées comme des {\bf \em moments successifs} des rapidités.	

\subsubsection{Forme en première quantification}
En utilisant la représentation en espace de configuration $\{z_a\} \equiv \{z_1 , \cdots , z_N \}$, les opérateurs $\operator{Q}$ et $\operator{P}$ agissent comme suit sur les fonctions d’onde $\varphi_{\{\theta_a\}}(\{z_a\})$ :
\begin{eqnarray}
	\operator{Q}\ket{\{\theta_a\}} =  \sqrt{N!}\int d^Nz \, \operator{\mathcal{N}} \varphi_{\{\theta_a\}}(\{z_a\} )\ket{\{z_a\}}, \, \operator{P}\ket{\{\theta_a\}} =  \sqrt{N!}\int d^Nz \, \operator{\mathcal{P}}_N \varphi_{\{\theta_a\}}(\{z_a\} )\ket{\{z_a\}} 
\end{eqnarray}
où les opérateurs associés agissant sur les fonctions d’onde à $N$ particules sont :
\begin{eqnarray}
	\operator{ \mathcal{N}}  =  \sum_{k = 1}^N 1 = N ,~\operator{ \mathcal{P}}_N  = -i \sum_{k = 1}^N k =- i\sum_{k = 1}^N \operator{\partial}_{z_k}	
\end{eqnarray}

Ces formes découlent directement des règles de commutation canonique (\ref{chap:1:com.1}) et de la définition des opérateurs en seconde quantification (\ref{chap:eq.vide.fock}) (cf. annexes \ref{annex:N.part}).

\subsubsection{Conclusion}
Ainsi, les opérateurs $\operator{Q}$ , $\operator{P}$ et $\operator{H}$ possèdent une structure diagonale commune dans la base des états propres $\ket{\{\theta_a\}}$, révélant la nature intégrable du modèle de Lieb-Liniger. Leurs valeurs propres sont respectivement les 0e, 1er et 2e moments des rapidités. Cette structure permet de généraliser la construction à une hiérarchie complète d’observables conservées, qui seront présentées au chapitre suivant.


\subsection{Fonction d’onde et Hamiltonien et moment à 2 corps}

%Nous considérons à présent le cas de deux bosons quantiques dans la même boîte unidimensionnelle de longueur \(L\), avec des conditions aux limites périodiques. Contrairement au cas à une particule, le terme d’interaction à contact intervient dans la dynamique. L'hamiltonien à 2 particule s'écrit :
%En première quantification, en utilisant les coordonnées du centre de masse et relatives $Z = (z_1 + z_2)/2$ et $Y = z_1 - z_2$, il vient que
%l'hamiltonien (\ref{chap:1:hal.mod.2.part.3}) se divise en une somme de deux problèmes indépendants à une seule particule.
%Les états propres de l'hamiltonien du centre de masse de masse $\overline{m}= 2$, $-\frac{1}{4} \partial_Z^2$, sont des ondes planes, et l'hamiltonien pour la coordonnée relative $Y$ correspond à celui d'une particule de masse réduite $\tilde{m} = 1/2$ en présence d'un potentiel delta en $Y = 0$. 
%\paragraph{Introduction au système à deux bosons avec interaction de contact.}
%Nous considérons à présent le cas de deux bosons quantiques dans une même boîte unidimensionnelle de longueur \(L\), avec des conditions aux limites périodiques. Contrairement au cas à une particule, un terme d’interaction de contact intervient ici dans la dynamique. L’Hamiltonien à deux particules s’écrit :
%\begin{eqnarray}
%	\operator{\mathcal{H}}_2  =  \operator{\mathcal{K}}_2 +\operator{\mathcal{V}}_2  & avec & \operator{\mathcal{K}}_2 =   - \frac{1}{2} \partial_{z_1}^2 - \frac{1}{2} \partial_{z_2}^2,  \quad \mbox{et} \quad  \operator{\mathcal{V}}_2  =  	g  \delta(z_1 - z_2). \label{chap:1:hal.mod.2.part.3} 		
%\end{eqnarray}
%On rappelle que l'énergies propres de  $\operator{\mathcal{K}}_2$ associées aux fonction d'ondes $\varphi_{\{ \theta_1 , \theta_2 \}}$ , la masse des particule étant égale à 1 (ie $\hbar= m=1$) s'écrit 
%\begin{eqnarray}
%	\varepsilon(\theta_1) + 	\varepsilon(\theta_2) & = & \frac{\theta_1^2}{2} + \frac{\theta_2^2}{2} 
%\end{eqnarray}
%On vas travailler dans le centre de masse.

%\paragraph{Changement de variables : coordonnées du centre de masse et relatives.}
 
%En première quantification, en introduisant les coordonnées du centre de masse \(Z = \frac{z_1 + z_2}{2}\) et relative \(Y = z_1 - z_2\), on obtient :
%\(
%	\partial_{z_1}^2 + \partial_{z_2}^2 = \frac{1}{2} \partial_Z^2 + 	2\partial_Y^2.  
%\)
%L’Hamiltonien~\eqref{chap:1:hal.mod.2.part.3} se décompose alors en une somme de deux problèmes indépendants à une seule variable :

%\begin{eqnarray}\label{chap:1:hal.mod.2.part.4}
%	\operator{\mathcal{H}}_2  =  	-\frac{1}{4} \partial_Z^2 + \operator{\mathcal{H}}_{rel} , \quad \mbox{avec}\quad  \operator{\mathcal{H}}_{rel} =  - 	\partial_Y^2 + g \delta ( Y ). 
%\end{eqnarray}

%\paragraph{Résolution du problème de centre de masse et de coordonnée relative.}

%Les états propres de l’Hamiltonien associé au centre de masse, \(-\frac{1}{4} \partial_Z^2\), correspondant à une particule de masse totale \(\bar{m} = 2\), sont des ondes planes associés à l'énergie $\overline{\theta}^2$ avec $\overline{\theta} = \frac{ \theta_1 + \theta_2}{2}$. L’Hamiltonien, $\operator{\mathcal{H}}_{rel}$, associé à la coordonnée relative \(Y\) correspond quant à lui à celui d’une particule de masse réduite \(\tilde{m} = \frac{1}{2}\), soumise à un potentiel delta en \(Y = 0\) :
%\begin{eqnarray}\label{chap:1:hal.mod.2.part.5}
%	- 	\partial_Y^2 \tilde{\varphi}(Y) + g \delta ( Y )\tilde{\varphi}(Y) & = & \tilde{\varepsilon}\,\tilde{\varphi}(Y),
%\end{eqnarray}
%où $\tilde{\varepsilon}$ est l’énergie propre du problème relatif.

%%%%%%
\paragraph{Introduction au système de deux bosons avec interaction de contact.}

Considérons maintenant un système de deux bosons quantiques confinés dans une boîte unidimensionnelle de longueur \(L\), avec des conditions aux limites périodiques. Contrairement au cas à une seule particule, une interaction de contact intervient ici dans la dynamique. L’Hamiltonien à deux particules s’écrit :
\begin{eqnarray}
	\operator{\mathcal{H}}_2 = \operator{\mathcal{K}}_2 + \operator{\mathcal{V}}_2, \quad \text{avec} \quad \operator{\mathcal{K}}_2 = - \frac{1}{2} \partial_{z_1}^2 - \frac{1}{2} \partial_{z_2}^2, \quad \text{et} \quad \operator{\mathcal{V}}_2 = g \, \delta(z_1 - z_2). \label{chap:1:hal.mod.2.part.3}
\end{eqnarray}

On rappelle que, pour des particules de masse unitaire (i.e., \(\hbar = m = 1\)), les énergies propres de l’opérateur cinétique \(\operator{\mathcal{K}}_2\), associées aux fonctions d’onde symétrisées \(\varphi_{\{ \theta_1 , \theta_2 \}}\), sont données par :
\begin{eqnarray}
	\varepsilon(\theta_1) + \varepsilon(\theta_2) = \frac{\theta_1^2}{2} + \frac{\theta_2^2}{2}.
\end{eqnarray}

Afin de simplifier le problème, nous nous plaçons dans le référentiel du centre de masse.

\paragraph{Changement de variables : coordonnées du centre de masse et relative.}

En première quantification, on introduit les nouvelles variables :
\(
Z = \frac{z_1 + z_2}{2} \, \text{(centre de masse)}, \qquad Y = z_1 - z_2 \, \text{(coordonnée relative)}.
\)
Dans ce changement de variables, l’opérateur laplacien total devient :
\(
\partial_{z_1}^2 + \partial_{z_2}^2 = \frac{1}{2} \partial_Z^2 + 2 \, \partial_Y^2.
\)
L’Hamiltonien~\eqref{chap:1:hal.mod.2.part.3} se décompose alors en la somme de deux Hamiltoniens agissant respectivement sur \(Z\) et \(Y\) :
\begin{eqnarray}\label{chap:1:hal.mod.2.part.4}
	\operator{\mathcal{H}}_2 = -\frac{1}{4} \partial_Z^2 + \operator{\mathcal{H}}_{\text{rel}}, \qquad \text{avec} \quad \operator{\mathcal{H}}_{\text{rel}} = - \partial_Y^2 + g \, \delta(Y).
\end{eqnarray}

\paragraph{Résolution du problème du centre de masse et de la coordonnée relative.}

L’Hamiltonien du centre de masse, \(-\frac{1}{4} \partial_Z^2\), décrit une particule de masse totale \(\bar{m} = 2\). Ses états propres sont des ondes planes associées à une énergie \(\overline{\theta}^2\), avec :
\(
\overline{\theta} = \frac{\theta_1 + \theta_2}{2},
\)
jouant ici un rôle analogue à celui d’un pseudo-moment associé dans le référentielle de laboratoire.
Le Hamiltonien relatif, \(\operator{\mathcal{H}}_{\text{rel}}\), correspond quant à lui à une particule de masse réduite \(\tilde{m} = \frac{1}{2}\) soumise à un potentiel delta centré en \(Y = 0\). Son équation propre s’écrit :
\begin{eqnarray}\label{chap:1:hal.mod.2.part.5}
	- \partial_Y^2 \, \tilde{\varphi}(Y) + g \, \delta(Y) \, \tilde{\varphi}(Y) = \tilde{\varepsilon} \, \tilde{\varphi}(Y),
\end{eqnarray}
où \(\tilde{\varepsilon}\) désigne l’énergie associée au mouvement relatif.
%%%%%%%%%%%%%%

\paragraph{Forme symétrique de la fonction d'onde pour bosons.}
Dans le référentiel du centre de masse. Le système est le même que que celuis d'un particules de masse $\tilde{m}= \frac{1}{2}$.
Le système étant composé de particules bosoniques, on cherche une solution symétrique que l’on écrit sous la forme  :
\begin{eqnarray}
	\tilde{\varphi}(Y) ~=~a~e^{i\frac{1}{2} \tilde{\theta} \vert Y \vert } + b~e^{-i\frac{1}{2} \tilde{\theta}\vert Y \vert } ~\propto~  \sin\left( \frac{1}{2} (\tilde{\theta} |Y| + \Phi ) \right). \label{eq:ansatz.boson}
\end{eqnarray}
Le paramètre \( \tilde{\theta} = \theta_1 - \theta_2 \) joue ici un rôle analogue à celui d’un pseudo-moment associé à la coordonnée relative,
est  la phase s'écrit
\begin{eqnarray}
	\Phi(\tilde{\theta}) &=& 2 \arctan\left (\frac{1}{i} \frac{a+b}{a-b}\right),	\label{chap:1:dif.mod.2.part.1} 
\end{eqnarray}
car \( a\exp(ix)+b\exp(-ix) = 2\sqrt{ab}\sin\left(x+\arctan\left(-i\, \frac{a+b}{a-b}\right)\right) \). Pour $\tilde{\theta}<0$, les termes exponentiels \( \exp(i\tilde{\theta} \vert Y \vert/2 ) \) et \( \exp(-i\tilde{\theta} \vert Y \vert/2 ) \) correspondent aux paires de particules entrantes et sortantes d’un processus de diffusion à deux corps.


%En réinjectant l'équation \eqref{eq:ansatz.boson} dans l’équation \eqref{chap:1:hal.mod.2.part.5}, on obtient l’énergie propre du problème réduit $\tilde{\varepsilon}$ associé à l’état lié. Celle-ci prend la forme classique de l’énergie cinétique d’une particule, \( \frac{1}{2} \times \text{masse} \times \text{vitesse}^2 \), la masse réduite du problème étant ici \( \tilde{m} = \frac{1}{2} \), et où \( \tilde{\theta} \) joue un rôle analogue à celui d’une vitesse. On en déduit :
%\begin{eqnarray}\tilde{\varepsilon}(\tilde{\theta})  & = &  \frac{1}{2} \cdot \tilde{m} \cdot \tilde{\theta}^2 = \frac{1}{2} \cdot \frac{1}{2} \cdot \tilde{\theta}^2 = \frac{\tilde{\theta}^2}{4}.\end{eqnarray}
%\begin{eqnarray}
%	\tilde{\varepsilon}(\tilde{\theta})  & = &  \frac{\tilde{\theta}^2}{4}.
%\end{eqnarray}
% Il encode la décroissance exponentielle de la fonction d’onde liée dans l’espace relatif, et sa valeur est directement reliée à la profondeur de l’état lié. Une valeur plus grande de \( \tilde{\theta} \) correspond à un état plus fortement lié, c’est-à-dire plus localisé autour de \( Y = 0 \), ce qui reflète une interaction plus attractive entre les deux particules. $\overline{\theta}^2 +  \tilde{\varepsilon}(\tilde{\theta}) = \varepsilon{\theta_1} + \varepsilon{\theta_2}$.
En réinjectant l’ansatz~\eqref{eq:ansatz.boson} dans l’équation relative
\eqref{chap:1:hal.mod.2.part.5}, on obtient l’énergie propre
\(\tilde{\varepsilon}\) du problème réduit.  
Elle prend la forme cinétique usuelle
\(\tfrac{1}{2}\times\text{masse}\times\text{vitesse}^{2}\).  
La masse réduite vaut ici \(\tilde{m}= \frac{1}{2}\) et le paramètre
\(\tilde{\theta}\) joue le rôle d’une impulsion ; ainsi
\begin{equation}
   \tilde{\varepsilon}(\tilde{\theta})
   \;=\;
   \frac{1}{2}\,\tilde{m}\,\tilde{\theta}^{2}
   \;=\;
   \frac{1}{2}\times\frac{1}{2}\,\tilde{\theta}^{2}
   \;=\;
   \frac{\tilde{\theta}^{2}}{4}.
   \label{eq:energie_relative}
\end{equation}

Cette énergie gouverne la décroissance exponentielle de la fonction
d’onde dans la coordonnée relative : plus \(\tilde{\theta}\) est grand,
plus l’état est localisé autour de \(Y=0\), signe d’une interaction
attractive plus forte entre les deux bosons.

L’énergie totale se décompose enfin en la somme du mouvement du centre
de masse et du mouvement relatif :
\(
   \overline{\theta}^{2}
   \;+\;
   \tilde{\varepsilon}(\tilde{\theta})
   \;=\;
   \varepsilon(\theta_{1})
   \;+\;
   \varepsilon(\theta_{2}),
\)
où \(\overline{\theta}= \tfrac{\theta_{1}+\theta_{2}}{2}\) et
\(\varepsilon(\theta)=\theta^{2}/2\).






%%%%%%%%%%%%%%%%%%%%%%%%%%%
\paragraph{Condition de discontinuité à cause du potentiel delta.}
En raison de la présence du potentiel delta centré en $Y = 0$, la dérivée première de la fonction d’onde $\tilde{\varphi}(Y)$ présente une discontinuité en ce point. En effet, le potentiel étant infini en $Y = 0$, la phase $\Phi$ du régime symétrique est déterminée en intégrant l’équation du mouvement autour de la singularité. En intégrant entre $- \epsilon$ et $+ \epsilon$ et en faisant tendre $\epsilon \to 0$, on obtient la condition de saut de la dérivée :

%avec $\Phi$ une phase à déterminer. %\begin{equation}
%	E = \frac{\tilde{m} \theta^2}{2}.
%\end{equation}

%La dérivée de la fonction d’onde n’est pas continue en $Y = 0$. Le potentiel étant infini en $Y = 0$, la phase $\Phi$ est obtenue en intégrant l’équation du mouvement entre $- \epsilon$ et $+ \epsilon$ et en faisant tendre $\epsilon$ vers zéro :


%En raison de ce potentiel delta, la dérivée première de la fonction d'onde $\varphi(Y)$ doit avoir une discontinuité en $Y = 0$ : 

%{\color{lightgray} 
%\begin{eqnarray*}
%	\underset{ \epsilon \to 0 }{\lim} \int_{-\epsilon}^{+\epsilon}  	-\underbrace{\cancel{\frac{1}{4} \partial_Z^2\varphi(Y)}}_{0} - 	\partial_Y^2\varphi(Y) + c \delta ( Y )\varphi(Y) \, dY  & = & \underset{ \epsilon \to 0 }{\lim}  \int_{-\epsilon}^{+\epsilon}  E d Y , \\
%	\underset{ \epsilon \to 0 }{\lim}  \left [ \varphi'(\epsilon) - \varphi'(-\epsilon) \right ] - c \varphi (  0 ) & =  &  -\underset{ \epsilon \to 0 }{\lim}  \int_{-\epsilon}^{+\epsilon}  E d Y,\\
%	 \varphi'(0^+) - \varphi'(0^-) - c \varphi (  0 ) & = & 0 .
%\end{eqnarray*}


%}

\begin{eqnarray*}
	\underset{ \epsilon \to 0 }{\lim} \int_{-\epsilon}^{+\epsilon}  - 	\partial_Y^2\tilde{\varphi}(Y) + g \delta ( Y )\tilde{\varphi}(Y) \, dY  & = & \underset{ \epsilon \to 0 }{\lim}  \int_{-\epsilon}^{+\epsilon}  \tilde{\varepsilon}(\tilde{\theta})d Y ,\\
	\\
	\tilde{\varphi}'(0^+) - \tilde{\varphi}'(0^-) - g \tilde{\varphi} (  0 ) & = & 0 .
\end{eqnarray*}


%soit $\tilde{\varphi}'(0^+) - \tilde{\varphi}'(0^-) - c \tilde{\varphi} (  0 )  =  0 $ .

%%%%%%%%%%%%%%%
\paragraph{Détermination de la phase $\Phi$.}
Et en évaluant la discontinuité de sa dérivée au point $Y = 0$, on trouve que la phase $\Phi$ satisfait la condition :

%\begin{equation}
%	\tan\left( \frac{\Phi}{2} \right) = \frac{\tilde{\theta}}{c}.
%\end{equation}

\begin{eqnarray}\label{chap:1:dif.mod.2.part.2}
	\Phi(\tilde{\theta}) & = & 2 \arctan (\tilde{\theta}/g) \in [ - \pi , +\pi ].
\end{eqnarray}

%{\color{red}( à revoir)} Cette relation exprime l’impact de l’interaction delta sur le déphasage de la solution liée. On en déduit que plus le couplage $g$ est fort ($g \to \infty$), plus la phase $\Phi$ se rapproche de $0$, ce qui correspond à une fonction d’onde présentant s'annulant en $Y = 0$. En revanche, dans la limite d’interaction faible ($g \to 0$), la phase $\Phi$ tend vers $\pm \pi$ et la discontinué de la dérivé de la fonction d'onde devient négligeable.
%Cette relation exprime l’impact de l’interaction de type delta sur le déphasage de la fonction d’onde liée.On en déduit que plus le couplage $g$ est fort ($g \to \infty$), la phase $\Phi$ se rapproche de $0$, ce qui correspond à une fonction d’onde présentant s'annulant en $Y = 0$, à l’image du régime d’imperméabilité totale.
%À l’inverse, dans la limite d’interaction faible (\( g \to 0 \)), si bien que \( \Phi \) tend vers $\pi$ (ou \( -\pi \), selon le signe de \( \tilde{\theta} \)). Dans ce cas, la discontinuité de la dérivée de la fonction d’onde au point \( Y = 0 \) devient négligeable, ce qui traduit un couplage quasi inexistant entre les deux particules.
%Cette relation exprime l’impact de l’interaction de type delta sur le déphasage de la fonction d’onde liée. Lorsque le couplage \( g \) devient très fort (\( g \to \infty \)), la fraction \( \tilde{\theta}/g \to 0 \), et la phase \( \Phi(\tilde{\theta}) \to 0 \). Cela correspond à une situation dans laquelle la fonction d’onde est fortement contrainte à s’annuler en \( Y = 0 \), à l’image du régime d’imperméabilité totale.
%À l’inverse, dans la limite d’interaction faible (\( g \to 0 \)), la fraction \( \tilde{\theta}/g \to \infty \), si bien que \( \Phi(\tilde{\theta}) \to \pi \) (ou \( -\pi \), selon le signe de \( \tilde{\theta} \)). Dans ce cas, la discontinuité de la dérivée de la fonction d’onde au point \( Y = 0 \) devient négligeable, ce qui traduit un couplage quasi inexistant entre les deux particules.

Cette relation exprime l’impact de l’interaction de type delta sur le déphasage de la fonction d’onde liée. On en déduit que plus le couplage \( g \) est fort (\( g \to \infty \)), plus la phase \( \Phi \) se rapproche de zéro. Cela correspond à une fonction d’onde qui s’annule en \( Y = 0 \), caractéristique d’un régime d’imperméabilité totale.

À l’inverse, dans la limite d’une interaction faible (\( g \to 0 \)), la phase \( \Phi \) tend vers \( \pi \) (ou \( -\pi \), selon le signe de \( \tilde{\theta} \)). Dans ce cas, la discontinuité de la dérivée de la fonction d’onde au point \( Y = 0 \) devient négligeable, ce qui traduit une interaction presque absente entre les deux particules.


%%%%%%%%%%%%%%%%%%%%%%%%%%%%%%%%%%%
%\paragraph{Phase de diffusion à un corp.}
%Les équations \eqref{chap:1:dif.mod.2.part.1} et \eqref{chap:1:dif.mod.2.part.2}  et en remarquant que pour $z \in \mathbb{C} \backslash \{ \pm i \} 2\artan(z) = i \ln \left( \frac{ 1 - i z }{1+iz} \right ) $ soit $\exp(2i\arctan(x)) = (1 + ix)/(1 - ix)$ et $\Phi(\tilde{\theta}) = i \ln ( - b/a ) $  donne rapport entre les amplitudes $a$ et $b$ de la fonction d'onde \eqref{eq:ansatz.boson} définit la phase de diffusion / {\em matrice diffusion} $S( \tilde{\theta}) \doteq e^{i\Phi ( \tilde{\theta}  ) }$  :

%\begin{eqnarray}
%	e^{i\Phi ( \tilde{\theta}  ) } &=& -\frac{a}{b} ~=~\frac{1 +i\tilde{\theta}/g} { 1 - i\tilde{\theta}/g} .\label{chap:1:dif.mod.2.part.3}
%\end{eqnarray}

\paragraph{Phase de diffusion à deux corps.}

En combinant les équations~\eqref{chap:1:dif.mod.2.part.1} et~\eqref{chap:1:dif.mod.2.part.2} avec l’identité analytique valable pour tout
\(z \in \mathbb{C}\setminus\{\pm i\}\),
\(
2\arctan(z)=i\ln\!\left(\frac{1-iz}{1+iz}\right)
\Leftrightarrow
e^{2i\arctan(z)}=\frac{1+iz}{1-iz},
\)
on obtient que le rapport des amplitudes \(a\) et \(b\) de la fonction
d’onde relative~\eqref{eq:ansatz.boson} définit la {\em phase de diffusion }
\(
\Phi(\tilde{\theta}) = i\ln\!\left(-\frac{b}{a}\right).
\)
On introduit alors la {\em matrice de diffusion} (ou facteur de diffusion)
\begin{eqnarray}
	S(\tilde{\theta}) \;\doteq\; e^{i\Phi(\tilde{\theta})}= -\frac{a}{b}= \frac{1 + i\,\tilde{\theta}/g}{1 - i\,\tilde{\theta}/g}.%\tag{\ref{chap:1:dif.mod.2.part.3}}
\end{eqnarray}
%où \(g\) est le paramètre d’interaction et
%\(\tilde{\theta} = \theta_1 - \theta_2\) le pseudo‑moment relatif.  
Cette expression, unitaire et analytique, caractérise entièrement la diffusion élastique à deux corps dans le modèle considéré.



\paragraph{Lien entre phase de diffusion et décalage temporel : interprétation semi-classique. {\color{red}(à revoir)}}

Il a été souligné par {\color{black}Eisenbud (1948)} et {\color{black}Wigner (1955)} que la phase de diffusion peut être interprétée, de manière semi-classique, comme un {\em décalage temporel}. Esquissons brièvement l'argument de {\color{black}Wigner (1955)}.Tout d'abord, notons que, pour une particule unique, une approximation simple d’un paquet d’ondes peut être obtenue en superposant deux ondes planes avec des moments $\tilde{\theta}/2$ et $\tilde{\theta}/2 + \delta \tilde{\theta}$, respectivement :
\begin{eqnarray}
	\tilde{\varphi}_{\text{inc}}(Y) & \propto & e^{i\frac{1}{2}\tilde{\theta} \vert Y\vert} + e^{i\frac{1}{2}\left(\tilde{\theta} + 2\delta \tilde{\theta} \right) \vert Y\vert}.
\end{eqnarray}
Cette superposition évolue dans le temps comme :
\begin{eqnarray}
\tilde{\varphi}_{\text{inc}}(Y, t) &\propto &  e^{i\left( \frac{1}{2} \tilde{\theta}\vert Y\vert - t\,\tilde{\varepsilon}(\tilde{\theta}) \right)} + e^{i\left( \frac{1}{2}\left(  \tilde{\theta} + 2\delta \tilde{\theta} \right) \vert Y\vert - t\,\tilde{\varepsilon}(\tilde{\theta} + 2\delta \tilde{\theta}) \right)}.
\end{eqnarray}
%où l'on a utilisé l'expression de l'énergie réduite : $\tilde{\varepsilon}(\tilde{\theta}) = \tilde{\theta}^2 / 4$.
Le centre de ce 'paquet d'ondes' se situe à la position où les phases des deux termes coïncident, c'est-à-dire au point où $\vert Y\vert\delta \tilde{\theta}  - t[\tilde{\varepsilon}(\tilde{\theta} + 2\delta \tilde{\theta} ) - \tilde{\varepsilon}(\tilde{\theta})] = 0$, ce qui donne $\vert Y\vert \simeq \tilde{\theta} t$ avec la vitesse réduite $\tilde{\theta} = 1/2 \varepsilon'(\tilde{\theta}) $. %Ainsi, il s'agit effectivement d'un 'paquet d'ondes' se déplaçant à la vitesse $\theta$. Ensuite, considérons deux particules entrantes dans un état tel que le centre de masse $Z = (z_1 + z_2)/2$ ait une impulsion $\theta_1 - \theta_2$, tandis que la coordonnée relative $Y = z_1 - z_2$ se trouve dans un 'paquet d'ondes' se déplaçant à la vitesse $ (\theta_1 - \theta_2)/2$,
Selon les équations (\ref{eq:ansatz.boson}) et (\ref{chap:1:dif.mod.2.part.3}), l'état sortant de la diffusion correspondant serait :
\begin{eqnarray}
	\tilde{\varphi}_{outc} ( Y, t ) & \propto & -e^{i\Phi(\tilde{\theta})}e^{-i\frac{1}{2}\tilde{\theta} \vert Y\vert} - e^{i\Phi(\tilde{\theta} + 2 \delta \tilde{\theta} )}e^{-i\frac{1}{2}\left(\tilde{\theta} + 2\delta \tilde{\theta} \right) \vert Y\vert}. %\tag{2}
\end{eqnarray}
En répétant l'argument précédent de la stationnarité de phase, on trouve que la coordonnée relative est à la position $\vert Y \vert  \simeq \tilde{\theta} t - 2\Phi'( \tilde{\theta})$ au moment $t$. %Étant donné que le centre de masse n'est pas affecté par la collision et se déplace à la vitesse de groupe $\tilde{\theta} =(\theta_1 + \theta_2)/2$, nous constatons que la position des deux particules semiclassiques après la collision sera
\begin{eqnarray}
	\vert Y \vert & \simeq & 	\tilde{\theta} t  - 2 \Delta (\tilde{\theta} )
\end{eqnarray}
où le déplacement de diffusion $\Delta (\theta)$ est donné par la dérivée de la phase de diffusion,
\begin{eqnarray}\label{eq:I-1-16}
	\Delta ( \theta ) & \doteq & \frac{ d \Phi }{ d \theta } ( \theta )= \frac{ 2 g }{ g^2 + \theta^2} . 	
\end{eqnarray}


%\paragraph{Retour aux coordonnées du laboratoire.}
%En revenant aux coordonnées d'origine (celles du laboratoire), on constate que la fonction d'onde à deux corps 
%\(
%	\varphi_{\{\theta_1 , \theta_2\}} (z_1, z_2) = \langle \emptyset \vert \operator{\Psi} (z_1)\operator{\Psi} (z_2) \vert \{\theta_1, \theta_2\} \rangle,
%\)
%avec \(z_1 < z_2\) , (ie $Y>0$) . Et le centre de masse sur le mouvement
%\(
%	Z  =  \overline{\theta} t.	
%\)
%avec,  on rappelle , $\overline{\theta}$ la vitesse de groupe dans le référentielle de laboratoire.\\
%Nous constatons que la position des deux particules semiclassiques après la collision sera
%\begin{eqnarray}
%	z_1 ~=~ Z + \frac{Y}2 ~\simeq ~ \theta_1 t - \Delta(\theta_1 - \theta_2), & & 	z_2 ~=~ Z - \frac{Y}2 ~\simeq ~ \theta_2t + \Delta(\theta_1 - \theta_2),
%\end{eqnarray}

%avec  $\theta_1$ et $\theta_2$ on rappelle définie tel que 
%\(
%	\tilde{\theta} ~=~\theta_1 - \theta_2 , \,	\overline{\theta}~=~\frac{\theta_1 + \theta_2}{2}.	
%\)
%On remarquant que 
%\begin{eqnarray*}
%	z_1 \theta_1  + z_2  \theta_2 ~=~ 2Z\overline{\theta} + \frac{1}{2}Y\tilde{\theta}, & & z_1 \theta_2  + z_2  \theta_1 ~=~ 2Z\overline{\theta} - \frac{1}{2}Y\tilde{\theta}. 
%\end{eqnarray*}
%Ce qui est en accod avec la masse total $\overline{m} = 2$ et la masse résuite $\tilde{m} = \frac{1}{2}$ \\
%Ce qui nous motive à multiplier la fonction d'onde dans le référentiel du centre de masse \eqref{eq:ansatz.boson} par $\exp(2iZ\overline{\theta})$ pour obtenir 

%\begin{eqnarray}\label{eq:I-1-10}
%	\varphi_{\{\theta_1 , \theta_2\}}(z_1 , z_2) & \propto &  \left \{ \begin{array} { c cl} ( \theta_2 - \theta_1 - ic) e^{ i z_1 \theta_1 + iz_2 \theta_2 } - ( \theta_1 - \theta_2 - ic) e^{ i z_1 \theta_2 + iz_2 \theta_1} & \mbox{si} & z_1 < z_2 \\ (z_1 \leftrightarrow z_2) & \mbox{si} & z_1 > z_2 \end{array} \right.
%\end{eqnarray}

%correspondant aux valeurs propres

%\begin{eqnarray}
%	\varepsilon(\theta_1 , \theta_2) ~=~ \overbrace{ \overline{\theta}^2}^{\overline{\varepsilon}(\overline{\theta})}	 + \overbrace{\frac{1}{4} \tilde{\theta}^2}^{\tilde{\varepsilon}(\tilde{\theta})} ~=~ \frac{\theta_1}{2} + \frac{\theta_2}{2}.	
%\end{eqnarray}

%Pour $\theta_1 > \theta_2$, les deux termes $e^{iz_1 \theta_1 + iz_2 \theta_2 }$ et $e^{iz_1 \theta_2 + iz_2 \theta_1 }$ correspondent aux paires de particules entrantes et sortantes dans un processus de diffusion à deux corps. Le rapport de leurs amplitudes est la phase de diffusion à deux corps \eqref{chap:1:dif.mod.2.part.3} reste inchangé

%\begin{eqnarray}\label{chap:1:dif.mod.2.part.4}
%	e^{i\Phi ( \theta_1 - \theta_2  ) }~=~ -\frac{a}{b} ~=~\frac{\theta_1 - \theta_2  -ic} { \theta_2 - \theta_1  - ic}. 
%\end{eqnarray}


%%%%%%%%%%%%%%%%%%%%%%%%%%
\paragraph{Retour aux coordonnées du laboratoire.}

En revenant aux coordonnées du laboratoire, la fonction d’onde à deux corps s’écrit
\(
	\varphi_{\{\theta_1 , \theta_2\}} (z_1, z_2) 
	= \langle \emptyset \vert \operator{\Psi} (z_1)\operator{\Psi} (z_2) \vert \{\theta_1, \theta_2\} \rangle/\sqrt{2},
\)
dans le cas \(z_1 < z_2\), c’est-à-dire pour une séparation relative \(Y = z_1 - z_2 < 0\) (on pourra symétriser ultérieurement).  
Dans le référentiel du laboratoire, le centre de masse évolue selon
\(
	Z = \frac{z_1 + z_2}{2} = \overline{\theta}\,t.
\)
%où l’on rappelle que \(\overline{\theta} = \frac{\theta_1 + \theta_2}{2}\) est la vitesse de groupe du système dans le référentiel laboratoire.
Ainsi, la position semi-classique des deux particules après la collision s’écrit
\begin{eqnarray}
	z_1 = Z + \frac{Y}{2} \;\simeq\; \theta_1 t - \Delta(\theta_1 - \theta_2),\quad
	z_2 = Z - \frac{Y}{2} \;\simeq\; \theta_2 t + \Delta(\theta_1 - \theta_2),
\end{eqnarray}
%où \(\Delta(\theta_1 - \theta_2)\) représente le décalage dû à l’interaction entre les deux particules.
%On rappelle les définitions :
%\[
%	\tilde{\theta} = \theta_1 - \theta_2, 
%	\quad
%	\overline{\theta} = \frac{\theta_1 + \theta_2}{2}.
%\]
On peut vérifier les identités utiles suivantes :
\begin{eqnarray*}
	z_1 \theta_1 + z_2 \theta_2 = 2Z \overline{\theta} + \frac{1}{2} Y \tilde{\theta}, \quad
	z_1 \theta_2 + z_2 \theta_1 &=& 2Z \overline{\theta} - \frac{1}{2} Y \tilde{\theta},
\end{eqnarray*}
ce qui est en accord avec les masses associées : masse totale \(\overline{m} = 2\), masse réduite \(\tilde{m} = \frac{1}{2}\).

Cela nous motive à multiplier l’ansatz dans le référentiel du centre de masse (équation~\eqref{eq:ansatz.boson}) par un facteur de phase globale \(\exp(2iZ\overline{\theta})\) pour revenir à la représentation dans le laboratoire. On obtient alors l’expression de la fonction d’onde :
\begin{eqnarray}\label{eq:I-1-10}
	\varphi_{\{\theta_1 , \theta_2\}}(z_1 , z_2) & \propto &  \left \{ \begin{array} { c cl} ( \theta_2 - \theta_1 - ig) e^{ i z_1 \theta_1 + iz_2 \theta_2 } - ( \theta_1 - \theta_2 - ig) e^{ i z_1 \theta_2 + iz_2 \theta_1} & \mbox{si} & z_1 < z_2 \\ (z_1 \leftrightarrow z_2) & \mbox{si} & z_1 > z_2 \end{array} \right.
\end{eqnarray}

%Cette fonction d’onde correspond à une valeur propre d’énergie donnée par la somme des énergies associées aux deux degrés de liberté :

%\begin{equation}
%	\varepsilon(\theta_1 , \theta_2) 
%	= \underbrace{\overline{\theta}^2}_{\overline{\varepsilon}(\overline{\theta})}
%	+ \underbrace{\frac{1}{4} \tilde{\theta}^2}_{\tilde{\varepsilon}(\tilde{\theta})}
%	= \frac{\theta_1^2}{2} + \frac{\theta_2^2}{2}.
%\end{equation}

Pour \(\theta_1 > \theta_2\), les deux termes exponentiels 
\(e^{i z_1 \theta_1 + i z_2 \theta_2}\) et \(e^{i z_1 \theta_2 + i z_2 \theta_1}\)
correspondent respectivement aux ondes entrantes et sortantes dans le canal de diffusion à deux corps.  
Le rapport de leurs amplitudes définit la phase de diffusion / matrice diffusion $e^{i\Phi ( \tilde{\theta}  ) }$  à deux corps \eqref{chap:1:dif.mod.2.part.3} , reste inchangé :

\begin{equation}\label{chap:1:dif.mod.2.part.4}
	S(\theta_1- \theta_2) \doteq e^{i\Phi(\theta_1 - \theta_2)} 
	= \frac{\theta_1 - \theta_2 - ig}{\theta_2 - \theta_1 - ig}.
\end{equation}

Cette phase caractérise entièrement le processus de diffusion dans le modèle de Lieb-Liniger à deux particules.

\paragraph{Conditions périodiques et équations de Bethe pour deux bosons {\color{red}(à révoir)}.}

%La fonction d’onde obtenue par Bethe ansatz (voir
%\eqref{eq:I-1-10}) est, pour $z_{1}<z_{2}$,
%\[
%	\varphi_{\{\theta_{1},\theta_{2}\}}(z_{1},z_{2})
%		= a\,e^{i\theta_{1}z_{1}+i\theta_{2}z_{2}}
%		+b\,e^{i\theta_{2}z_{1}+i\theta_{1}z_{2}},
%	\quad
%	a=\theta_{2}-\theta_{1}-ic,\;
%	b=-(\theta_{1}-\theta_{2}-ic).
%\]

%\medskip
%\subparagraph{Périodicité sur $z_{2}$.}  
%On impose à la fonction d’onde obtenue par Bethe ansatz (voir
%\eqref{eq:I-1-10})
%\(
%	\varphi_{\{\theta_{1},\theta_{2}\}}(z_{1},z_{2}\!=\!L)
%	=
%	\varphi_{\{\theta_{1},\theta_{2}\}}(z_{1},z_{2}\!=\!0)
%\)
%avec $0<z_{1}<z_{2}=L$.  
%Au point $z_{2}=L$ on reste dans le secteur $z_{1}<z_{2}$, tandis qu’au point $z_{2}=0$ le domaine pertinent devient $z_{2}<z_{1}$;  la fonction d’onde y est obtenue en échangeant $z_{1}\leftrightarrow z_{2}$ , soit 
%\(
%	\varphi_{\{\theta_{1},\theta_{2}\}}(z_{1},\!L)
%	=
%	\varphi_{\{\theta_{1},\theta_{2}\}}(0 , z_{1})
%\)
%.  
%On obtient ainsi
%\begin{eqnarray*}
%	a\,e^{i\theta_{1}z_{1}+i\theta_{2}L}+b\,e^{i\theta_{2}z_{1}+i\theta_{1}L} & = &
%	a\,e^{i\theta_{2}z_{1}}\,e^{i\theta_{1}\! \cdot 0} + b \,e^{i\theta_{1}z_{1}}\,e^{i\theta_{2}\! \cdot 0},	
%\end{eqnarray*}
%avec la condition $z_1< z_2$, avec le rapport $a$ et $b$ vérifiant \eqref{chap:1:dif.mod.2.part.4} de la sorte $-b/a = e^{i\Phi(\theta_1 - \theta_2)}$ .

%%%%%%%%%%%%%%%%

\subparagraph{Périodicité en \( z_2 \).}  
On impose une condition de périodicité sur la fonction d’onde obtenue par ansatz de Bethe (voir équation~\eqref{eq:I-1-10}) :
\(
	\varphi_{\{\theta_1,\theta_2\}}(z_1, z_2 = L) = \varphi_{\{\theta_1,\theta_2\}}(z_1, z_2 = 0),
\)
avec \( 0 < z_1 < z_2 = L \).  
Au point \( z_2 = L \), la configuration reste dans le secteur \( z_1 < z_2 \), tandis qu’à \( z_2 = 0 \), on entre dans le secteur \( z_2 < z_1 \). La continuité de la fonction d’onde impose alors d’échanger les coordonnées \( z_1 \leftrightarrow z_2 \) :
\(
	\varphi_{\{\theta_1,\theta_2\}}(z_1, L) = \varphi_{\{\theta_1,\theta_2\}}(0, z_1).
\)
En utilisant l’expression explicite de l’ansatz dans les deux secteurs, on obtient l’égalité suivante :
\begin{eqnarray*}
	a\,e^{i\theta_1 z_1 + i\theta_2 L} + b\,e^{i\theta_2 z_1 + i\theta_1 L}
	&=& a\,e^{i\theta_2 z_1} + b\,e^{i\theta_1 z_1}.
\end{eqnarray*}
%où le second membre correspond à la fonction d’onde dans le secteur \( z_2 < z_1 \), évaluée en \( z_2 = 0 \) et \( z_1 = z_1 \).  
%La condition de périodicité impose donc :
%\[
%	a\,e^{i\theta_1 z_1 + i\theta_2 L} + b\,e^{i\theta_2 z_1 + i\theta_1 L}
%	= a\,e^{i\theta_2 z_1} + b\,e^{i\theta_1 z_1}.
%\]
Cette relation, valable pour tout \( z_1 \in (0,L) \), fixe une contrainte sur le rapport \( b/a \). En utilisant l’expression de la phase de diffusion introduite en \eqref{chap:1:dif.mod.2.part.4} pour $z_1<z_2$ :
\begin{eqnarray*}
	-\frac{b}{a} = e^{i\Phi(\theta_1 - \theta_2)},
\end{eqnarray*}
on obtient une condition quantique sur les phases \( \theta_1 \) et \( \theta_2 \), cœur de la quantification imposée par le formalisme de Bethe.

%\[
%	( \theta_2 - \theta_1 - ig)\,e^{i\theta_{1}z_{1}+i\theta_{2}L}
%	- ( \theta_1 - \theta_2 - ig)\,e^{i\theta_{2}z_{1}+i\theta_{1}L}
%	=
%	( \theta_2 - \theta_1 - ig)\,e^{i\theta_{2}z_{1}}\,e^{i\theta_{1}\! \cdot 0}
%	- ( \theta_1 - \theta_2 - ig)\,e^{i\theta_{1}z_{1}}\,e^{i\theta_{2}\! \cdot 0}.
%\]
En identifiant les coefficients de $e^{i\theta_{1}z_{1}}$ et
$e^{i\theta_{2}z_{1}}$ indépendamment, on obtient
\(
	e^{i\theta_{2}L}\;a = b, 
	\,
	e^{i\theta_{1}L}\;b = a,
\)
c’est‑à‑dire l'équations de Bethe
%\begin{equation}\label{eq:PC2}
%	e^{i\theta_{2}L} = \frac{b}{a}
%	= \frac{\theta_{1}-\theta_{2}+ic}{\theta_{2}-\theta_{1}+ic},
%\quad
%	e^{i\theta_{1}L} = \frac{a}{b}
%	= \frac{\theta_{2}-\theta_{1}+ic}{\theta_{1}-\theta_{2}+ic}.
%\end{equation}
\begin{eqnarray*}\label{eq:PC2}
	e^{i\theta_{1}L}\,e^{i\Phi(\theta_{1}-\theta_{2})} = -1,
	\qquad
	e^{i\theta_{2}L}\,e^{i\Phi(\theta_{2}-\theta_{1})} = -1.	
\end{eqnarray*}
En prenant le logarithme on obtient les \emph{équations de Bethe à deux
particules} :
\begin{equation}\label{eq:Bethe2}
	\theta_{1}L + \Phi(\theta_{1}-\theta_{2}) = 2\pi I_{1}, 
	\qquad
	\theta_{2}L + \Phi(\theta_{2}-\theta_{1}) = 2\pi I_{2},
\end{equation}
où $I_{1},I_{2}\in\mathbb{Z}$ sont les nombres quantiques entiers
(caractère bosonique). 

\subparagraph{Périodicité sur $z_{1}$.}  Le raisonnement symétrique conduit exactement aux mêmes égalités \eqref{eq:PC2}.  
\bigskip
Les équations \eqref{eq:Bethe2} constituent la quantification complète
du gaz de Lieb–Liniger à deux bosons sur un cercle de longueur $L$ et
seront le point de départ pour l’étude de l’état fondamental et des
excitations.



\begin{figure}[H]
	\centering
  %\includegraphics[width=0.5\textwidth]{}
  %\caption{Gauche : La fonction d'onde (\ref{eq:I-1-10}) sur la ligne infinie correspond à un processus de diffusion à deux corps. Semiclassiquement, la phase de diffusion dans ce processus à deux corps se reflète dans le décalage de diffusion (\ref{eq:I-1-16}) : après la collision, la position de la particule a été déplacée d'une distance $\Delta ( \theta_1 - \theta_2 )$ . Droite : La fonction d'onde de Bethe (\ref{eq:I-2-17}) sur la ligne infinie correspond à un processus de diffusion à $N$-corps qui se factorise en des processus à deux corps (le décalage de diffusion $\Delta$ est également présent ici, mais il n'est pas représenté dans la caricature). Dans ce processus à $N$-corps, les rapidités $\theta_j$ sont les moments asymptotiques des bosons.}
  \label{}	
\end{figure}



\section{Équation de Bethe et distribution de rapidité}

\subsection{Fonction d’onde dans le secteur ordonné et représentation de Gaudin}

Dans le domaine $z_1 < z_2 < \cdots < z_N$, la fonction d’onde pour un état de Bethe à $N$ particules s’écrit ({\color{blue}Gaudin 2014}, {\color{blue}Korepin et al. 1997}, {\color{black}Lieb et Liniger 1963}) :
\begin{eqnarray}
	\varphi_{\{\theta_a\}} ( z_1 , \cdots , z_N ) & = &  \frac{1}{\sqrt{N!}}\langle \emptyset \vert \operator{\Psi} ( z_1 ) \cdots \operator{\Psi} (z_N ) \vert \{ \theta_a \} \rangle \notag\\
	& \propto & \sum_\sigma (-1)^{|\sigma|} \left( \prod_{1 \leq a < b \leq N} (\theta_{\sigma(b)} - \theta_{\sigma(a)} - i g) \right) e^{i \sum_{j=1}^{N} z_j \theta_{\sigma(j)}},\label{eq:I-2-17}
\end{eqnarray}
où la somme s'étend sur toutes les permutations $\sigma$ de $\{1,\dots,N\}$. Le facteur $(-1)^{|\sigma|}$ est la signature de la permutation, et les amplitudes dépendent des différences de quasi-moments $\theta_j$ ainsi que du couplage $c$.
Cette fonction d’onde est ensuite étendue par symétrie aux autres domaines du type $z_{\pi(1)} < z_{\pi(2)} < \cdots < z_{\pi(N)}$ via des propriétés d’échange symétriques.

\vspace{1em}

\subsection{Conditions aux bords périodiques}

Les équations précédentes ont été établies pour un système défini sur la droite réelle. Cependant, dans une perspective thermodynamique, il est essentiel de considérer une densité finie $ N/L$. Cela peut être obtenu en compactifiant l’espace sur un cercle de longueur $L$, i.e. en imposant les {\em conditions aux bords périodiques}.

Concrètement, cela consiste à identifier $x = 0$ et $x = L$ et à exiger que la fonction d’onde soit périodique lorsqu’une particule fait le tour du système :
\begin{equation}\label{eq:periodic}
\varphi_{\{\theta_a\}}(x_1, \dots, x_{N-1}, L) = \varphi_{\{\theta_a\}}(0, x_1, \dots, x_{N-1}).
\end{equation}
Cette condition doit être satisfaite pour chaque particule. Or, déplacer la $j$-ième particule de $x_j$ à $x_j + L$ revient à la faire passer devant toutes les autres : cela introduit un facteur de diffusion à chaque croisement.

%\vspace{1em}

\subsection{Équations de Bethe exponentielles}

En imposant les conditions de périodicité sur la fonction d’onde de type Bethe~\eqref{eq:I-2-17}, on obtient que chaque moment $\theta_a$ doit satisfaire l’équation :
\begin{equation}
	e^{i \theta_a L} \prod_{b \ne a} S(\theta_a - \theta_b) = (-1)^{N-1}, \quad a = 1, \dots, N,
	\label{eq:bethe_exp}
\end{equation}
où la matrice diffusion $S(\theta) = \frac{\theta - i g}{-\theta - i g} = e^{i \Phi(\theta)}$ est l’amplitude de diffusion à deux corps, et $\Phi(\theta) = 2 \arctan\left( \frac{\theta}{c} \right)$ est la phase associée~\eqref{chap:1:eq:Phi}. Le signe $(-1)^{N-1}$ vient du fait que chaque permutation change la signature du déterminant dans la représentation de Gaudin.
%\vspace{1em}

\subsection{Équations de Bethe logarithmiques}

En prenant le logarithme du membre gauche et du membre droit de l’équation~\eqref{eq:bethe_exp}, on obtient :
\begin{equation}\label{chap:1:eq:EBA}
	L \theta_a + \sum_{b=1}^N \Phi(\theta_a - \theta_b) = 2\pi I_a, \qquad a = 1, \dots, N,
\end{equation}
où les $I_a \in \mathbb{Z}$ (ou $\mathbb{Z} + \tfrac{1}{2}$) sont des nombres quantiques entiers (ou demis entiers) . Dans la configuration d’état fondamental (ou de type “mer de Fermi”), ces nombres sont pris de manière symétrique autour de zéro :
\[
I_a = a - \frac{N+1}{2}, \quad \text{pour } a \in \llbracket 1 , N \rrbracket.
\]
Ce choix garantit une distribution uniforme des $\theta_a$ à l’état fondamental.
%\vspace{1em}

\subsection{Interprétation physique}

Les équations de Bethe~\eqref{chap:1:eq:EBA} représentent une {\em quantification des pseudo‑impulsions $\theta_a$} des particules en interaction, résultant d’un {\em interféromètre multi‑corps sur le cercle} : chaque particule accumule une phase $e^{i \theta_a L}$ due au mouvement libre, ainsi que des phases de diffusion lorsqu’elle croise les autres.

Ce système d'équations détermine les états propres du système de Lieb–Liniger en volume fini, et joue un rôle fondamental dans la description exacte de ses propriétés thermodynamiques et dynamiques.


\subsection{Thermodynamique du gaz de Lieb–Liniger à température nulle}

Dans la limite thermodynamique, le nombre de particules \( N \) et la longueur \( L \) du système tendent vers l'infini de telle sorte que leur rapport reste fini :
\begin{eqnarray*}
	\lim_{N,\, L \to \infty} \frac{N}{L} = D < \infty,
\end{eqnarray*}
où \( D \) désigne la densité linéique de particules.

Considérons désormais le système à température nulle. L’état fondamental dans le secteur à nombre de particules fixé correspond à la configuration d’énergie minimale parmi les solutions des équations de Bethe \eqref{chap:1:eq:EBA}.

Dans la limite thermodynamique, les valeurs de \( \theta_a \) deviennent quasi-continues, avec un espacement \( \theta_{a+1} - \theta_a = \mathcal{O}(1/L) \), et se condensent dans un intervalle symétrique autour de zéro :
\[
\theta_a \in [-K, K],
\]
où \( K \) est le paramètre de Fermi (ou rapidité maximale), défini par \( K = \theta_N \). En supposant l'ordre \( I_a \geq I_b \Rightarrow \theta_a \geq \theta_b \), cet intervalle constitue ce qu'on appelle la {\em mer de Dirac} (ou sphère de Fermi en dimension un).

Nous introduisons la densité d’états \( \rho_s(\theta) \), définie par
\begin{eqnarray*}
	2\pi \rho_s(\theta_a) &=& \frac{2\pi}{L} \lim_{\text{therm}} \frac{|I_{a+1} - I_a|}{|\theta_{a+1} - \theta_a|} = \frac{2\pi}{L} \frac{\partial I}{\partial \theta}(\theta_a),
\end{eqnarray*}
où \( I(\theta_a) = I_a \). L’application des équations de Bethe sous forme logarithmique conduit alors à
\begin{eqnarray*}
	2\pi \rho_s(\theta_a) = 1 + \frac{1}{L} \sum_{b = 1}^N \Delta(\theta_a - \theta_b),
\end{eqnarray*}
ce qui relie \( \rho_s \) à la fonction d’interaction \( \Delta \) entre les rapidités.

Intéressons-nous maintenant à la {\em densité de particules dans l’espace des moments}, notée \( \rho(\theta) \), définie par
\begin{eqnarray*}
	\rho(\theta_a) = \lim_{L \to \infty} \frac{1}{L} \cdot \frac{1}{\theta_{a+1} - \theta_a} > 0.
\end{eqnarray*}
Dans l’état fondamental, toutes les positions disponibles dans l’intervalle \( [-K, K] \) sont occupées. On a donc :
\begin{eqnarray}\label{chap.1.rho.2}
	\rho(\theta) = \rho_s(\theta).
\end{eqnarray}

La quantité \( L \rho(\theta) d\theta \) représente le nombre de rapidités dans la cellule infinitésimale \( [\theta, \theta + d\theta] \), tandis que
\(
	N = L \int_{-K}^{K} \rho(\theta)\, d\theta
\)
donne le nombre total de particules dans le système. Le passage de la somme discrète à l'intégrale dans le second membre de l'équation de Bethe permet d’écrire :
\begin{eqnarray*}
	\frac{1}{L} \sum_{b = 1}^N \Delta(\theta_a - \theta_b) \longrightarrow \int_{-K}^{K} \Delta(\theta_a - \theta)\, \rho(\theta)\, d\theta.
\end{eqnarray*}
Ainsi, l'équation pour la densité d'états devient :
\begin{eqnarray}\label{chap.1.rho.s.2}
	2\pi \rho_s(\theta) = 1 + \int_{-K}^{K} \Delta(\theta - \theta')\, \rho(\theta')\, d\theta',
\end{eqnarray}
et, comme \( \rho = \rho_s \), on obtient l’équation linéaire intégrale satisfaite par la densité de rapidités :
\begin{eqnarray}\label{chap.1.rho.3}
	\rho(\theta) - \int_{-K}^{K} \frac{\Delta(\theta - \theta')}{2\pi} \rho(\theta')\, d\theta' = \frac{1}{2\pi}.
\end{eqnarray}


\subsection{Excitations élémentaires à température nulle}




\chapter{Relaxation et Équilibre dans les Systèmes Quantiques Intégrables : Une Approche par la Thermodynamique de Bethe}\label{chap:relaxation}
\minitoc

%------------------------------------------------------------------
\section*{Introduction générale}

Dans les modèles quantiques intégrables, l’évolution vers l’équilibre, à partir d’un état initial arbitraire (et typiquement hors d’équilibre), ne conduit pas à une thermique de Gibbs classique.  
En effet, du fait de l’existence d’une infinité de charges conservées en involution, les systèmes intégrables n’explorent qu’une sous-partie contrainte de l’espace des états accessibles.  
Ils relaxent alors vers un état stationnaire décrit par une \emph{Ensemble Thermodynamique Généralisé} (GGE), qui encode la conservation de toutes ces quantités.

Cette section pose les fondations nécessaires à la description de ces états stationnaires dans le cadre de la \textbf{thermodynamique de Bethe} (TBA), qui généralise l’analyse au-delà de l’état fondamental.  
Nous considérons ici un régime macroscopique à température (ou entropie) finie, correspondant à des états hautement excités du spectre, mais toujours décrits dans le formalisme intégrable exact.

Notre point de départ est la relation constitutive entre la \emph{densité de quasi-particules} (ou \emph{rapidités}) $\rho(\theta)$ et la \emph{densité d’états} disponibles $\rho_s(\theta)$, qui encode le spectre accessible en présence d’interactions.  
Nous introduisons ensuite une opération clé de la TBA, appelée \emph{habillage} (\emph{dressing}), qui intervient systématiquement dans le calcul des observables physiques et permet de prendre en compte de manière non perturbative les effets des interactions.  
Cette construction sera illustrée dans le cadre du modèle intégrable de Lieb–Liniger, qui décrit un gaz unidimensionnel de bosons avec interaction delta répulsive.

Les outils développés ici seront fondamentaux pour formuler dans la section suivante le concept d’ensemble généralisé (GGE), et pour décrire la dynamique de relaxation des systèmes intégrables.



\section{Notion d’état d’équilibre généralisé (GGE)}

\paragraph{Introduction.}


\paragraph{Configuration des états.}\label{sec:config-etats}.
On désigne par $\boldsymbol{\{ \theta_a \}}\equiv \{ \theta_1 , \cdots , \theta_{N} \}$ la \emph{configuration de rapidités} caractérisant un état propre à $N\!\equiv\!N(\{ \theta_a \})$ particules – le nombre de particules n’est donc pas fixé \emph{a priori} mais dépend de la configuration.  
L’état propre correspondant est noté $\ket{\{ \theta_a \}}\;=\;\ket{\{\theta_1,\dots,\theta_N \}}$.

%%%%%%%%%%%%%%%%%%%%%%%%%%%%%%%%%%%%%%%%%%%%%%%%%%
\paragraph{Observables diagonales dans la base des états propres.}
Dans le chapitre précédent (\ref{chap:LL-BA}), on a vu que l'état $\ket{\{ \theta_a \}}$ associé à cette configuration est une état propre des observables nombre et quantité de mouvement  et  énergie cinétique \eqref{chap1:eq.Q.P.K.theta.1}. Ces observables sont diagonales dans la base des états propres :
\begin{eqnarray}
	\operator{Q}  =  \sum_{ \{\theta_a\} } \left ( \sum_{a = 1}^{N}  1 \right )  \vert \{ \theta_a\}\rangle	\langle \{ \theta_a \}\vert, \, 
	\operator{P}  =  \sum_{\{ \theta_a\}}\left( \sum_{a = 1}^{N}  \theta_a \right )   \vert \{ \theta_a\}\rangle	\langle \{ \theta_a \}\vert,\,\operator{K}  =  \sum_{\{ \theta_a\}}\left ( \sum_{a = 1}^{N} \frac{\theta_a^2}{2} \right )   \vert \{ \theta_a\}\rangle	\langle \{ \theta_a \}\vert.\label{chap.2.gge.1}		
\end{eqnarray}
avec $ \sum_{\{ \theta_a\}}$ une somme sur tous les configurations.\\
%\begin{eqnarray}
%	\operator{Q} \ket{\{ \theta_a\}}  =  \sum_{ \{\theta_a\} } \left ( \sum_{a = 1}^{N}  1 \right ) \ket{\{ \theta_a\}}, \, 
%	\operator{P} \ket{\{ \theta_a\}}  =  \sum_{\{ \theta_a\}}\left( \sum_{a = 1}^{N}  \theta_a \right ) \ket{\{ \theta_a\}},\,\operator{H} \ket{\{ \theta_a\}}  =  \sum_{\{ \theta_a\}}\left ( \sum_{a = 1}^{N} \frac{\theta_a^2}{2} \right )   \ket{\{ \theta_a\}}.		
%\end{eqnarray}

Nous avons introduit ces observables en injectant des opérateurs $\operator{f}$ proportionnels à des puissances de la quantité de mouvement d’une particule $\operator{p}$, respectivement $\propto \operator{p}^0$, $\propto \operator{p}^1$ et $\propto \operator{p}^2$, dans l’opérateur à un corps $\operator{F}$ défini dans l’équation \eqref{chap.1:eq.rapel.opp.1.second.2}. Écrit de cette manière, nous avons vu dans l’équation \eqref{chap.1:eq.rapel.opp.1.second.3} que pour $\operator{f} = \operator{p}^q$ avec $q$ entier, l’état de Bethe $\ket{\{ \theta_a \} }$ est un état propre de $\operator{F}$ :
\begin{eqnarray}\label{chap.2:eq.rapel.opp.1.second.1}
	 \operator{F} \ket{\{\theta_a\}} =   \sum_{ \{\theta_a\} }\left( \sum_{a = 1}^N \theta_a^q \right) \ket{\{\theta_a\}},
\end{eqnarray}
avec des valeurs propres données par des puissances de $\theta$. Cela motive l’étude d’états d’équilibre statistique au-delà de l’équilibre thermique, c’est-à-dire au-delà de l’ensemble de Gibbs.
   




%%%%%%%%%%%%%%%%%%%%%%%%%%%%%%%%%%%%%%%%%%%%
\paragraph{Contexte et GGE dans les systèmes intégrables.}

Dans un système quantique {\bf intégrable}, il existe une infinité de charges conservées locales $\operator{Q}_i$ commutant entre elles et avec l’Hamiltonien $\operator{H}$ ([Rigol et al. 2007] ) \cite{??}. Concrètement, chaque charge se présente sous la forme $\operator{Q}_i = \int dx \,\operator{q}_i(x)$, où $\operator{q}_i(x)$ est une densité d’observable locale à support borné. L’intégrabilité implique ainsi une caractérisation complète des états propres par un ensemble de paramètres (rapidités $\{\theta_j\}$ dans le modèle de Lieb-Liniger) \cite{??}. En particulier, contrairement aux systèmes génériques, un système intégrable ne thermalise pas au sens canonique classique, car la présence de toutes ces contraintes empêche l’oubli complet des conditions initiales. Les points clés sont alors :

\begin{itemize}[label = $\bullet$]
	\item {\bf Charges conservées} : infinité de locales $\operator{Q}_i$ satisfaisant et $[\operator{Q}_i , \operator{H} ] = 0$ et $[\operator{Q}_i , \operator{Q}_j ] = 0$.
	\item {\bf Densités locales} : chaque $\operator{Q}_i$ s’écrit $\operator{Q}_i = \int_\mathbb{R} dx \, \operator{q}_i(x)$ avec $\operator{q}_i(x)$ à support fini.
	\item {\bf Relaxation non canonique} : après un {\em quench} (changement brutal de paramètre), le système évolue vers un état stationnaire qui n’est pas décrit par l’ensemble canonique habituel.
\end{itemize}

Pour décrire cet état, on introduit l’{\bf ensemble de Gibbs généralisé (GGE)}. Rigol et al. ont montré qu’une « extension naturelle de l’ensemble de Gibbs aux systèmes intégrables » prédit correctement les valeurs moyennes des observables après relaxation \cite{??}.  Formellement, pour une région finie du système $\mathcal{S} \subset \mathbb{R}$, on définit la matrice densité locale :
\begin{eqnarray}
	\operator{\rho}^{(\mathcal{S})}_{\mathrm{GGE}} = \frac{1}{Z^{(\mathcal{S})}}\exp \left ( - \sum_i \beta_i \operator{Q}_i^{(\mathcal{S})} \right), \quad \operator{Q}_i^{(\mathcal{S})} = \int_\mathcal{S} dx \, \operator{q}_i(x), \label{chap.TBA.op.rho.S}	
\end{eqnarray}

où $\beta_i \in \mathbb{R}$ sont les multiplicateurs de Lagrange (ou « températures généralisées ») associés aux charges locales conservées $\{\operator{Q}_i\}$. La fonction de partition 
\begin{eqnarray}
	Z^{(\mathcal{S})} = \bm{\mathrm{Tr}}\left [\exp \left( - \sum_i \beta_i \operator{Q}_i^{(\mathcal{S})} \right ) \right ]  \label{chap.TBA.op.Z.S}	
\end{eqnarray}
 assure la normalisation. L’{\bf état GGE} ainsi défini est le seul permettant de prédire de manière cohérente les observables locales de $\mathcal{S}$ à long temps \cite{??}. Autrement dit, l’équilibre local après quench est un état stationnaire faisant perdurer la mémoire de chaque charge conservée, ce qui conduit à un nombre macroscopique de paramètres $\beta_i$ thermodynamiques (une « température » par charge) \cite{??}.

 \subparagraph{Interprétation des multiplicateurs de Lagrange.}
Les multiplicateurs de Lagranges $\beta_i$ apparaissent naturellement lors de l'optimisation sous contraintes, par exemple dans le formalisme de l'{\bf ensemble de Gibbs généralisé (GGE)}, oû il imposent la conservation des valeurs moyennes des charges $\langle \operator{Q}_i^{(\mathcal{S})} \rangle_{\operator{\rho}^{(\mathcal{S})}_{\mathrm{GGE}}} = \bm{\mathrm{Tr}}[\operator{\rho}^{(\mathcal{S})}_{\mathrm{GGE}} \operator{Q}_i^{(\mathcal{S})}]   $.\\

En résumé, la GGE généralise les ensembles canoniques standard : au lieu de retenir uniquement l’énergie, on impose la conservation de l’ensemble complet $\{\operator{Q}_i \}$. Cette construction rend compte du fait que, dans un système intégrable, les observables locaux convergent vers les valeurs moyennes de $\operator{\rho}^{(\mathcal{S})}_{\mathrm{GGE}}$ , et non vers celles d’un Gibbs thermique ordinaire \cite{??}\cite{??}. On comprend ainsi pourquoi la {\em thermalisation habituelle} (canonique ou microcanonique) échoue : seul l’ensemble de Gibbs généralisé peut intégrer toutes les contraintes locales.

\paragraph{Rappel sur le modèle de Lieb-Liniger et distribution de rapidités.}
Comme rappelé au chapitre précédent, {\bf le modèle de  Lieb-Liniger} (gaz bosonique 1D à interactions de contact) est un exemple paradigmatique d’un système intégrable \cite{??}. Ses états propres sont caractérisés par un ensemble de $N$  rapidités $\{ \theta_a \}$ , qui jouent le rôle de quasi-momenta ({\bf Bethe ansatz}). Dans ce contexte, l’état macroscopique du gaz après relaxation unitaire est entièrement déterminé par la {\bf distribution des rapidités}. Formellement, on définit $\rho(\theta)$ la distribution intensive des rapidités telle que $\rho(\theta) d \theta$ donne la fraction de particules par unité de longueur ayant une rapidité dans la cellule $[\theta , \theta + d \theta ] $.\\

Cette « distribution de rapidités » est d’autant plus pertinente qu’elle est {\em accessible expérimentalement}. En effet, lorsque le gaz bosonique 1D est libéré et laissé s’étendre, la distribution asymptotique des vitesses des atomes coïncide avec la distribution initiale des rapidités \cite{??} . Autrement dit, la GGE prédit un profil de vitesses observables en laboratoire. Léa Dubois souligne dans sa thèse que " la distribution de rapidités est la distribution asymptotique des vitesses des atomes après une expansion dans le guide 1D ", et qu’elle peut être extraite par l’hydrodynamique généralisée \cite{??}. \\

Dans la GGE, cette distribution macroscopique $\rho(\theta)$ est fixée par l’ensemble des charges conservées. Par exemple, on ajuste les $\beta_i$ de sorte que les valeurs moyennes $\langle \operator{Q}_i \rangle_{\operator{\rho}^{(\mathcal{S})}_{\mathrm{GGE}}}$ correspondent aux valeurs initiales. Ce processus détermine donc la fonction $\rho(\theta)$ décrivant l’état d’équilibre local. Les observables locaux du gaz (densité, corrélations, etc.) en découlent alors via les équations de Bethe ansatz. 


\paragraph{Convention pour les moyennes d'observables.}
Dans la suite du chapitre, nous noterons la moyenne d’une observable $\operator{\mathcal{O}}$ dans un état décrit par une matrice densité (ici noté) $\operator{\rho}$ par :
\begin{eqnarray}\label{chap.TBA.moy.dens}	
	\braket{\operator{\mathcal{O}}}_{\operator{\rho}} \doteq \bm{\mathrm{Tr}}[\operator{\rho} \, \operator{\mathcal{O}}],
\end{eqnarray}
En particulier, si la matrice densité est un projecteur, comme $\ket{\{\theta_a \}}\!\bra{\{\theta_a \}}$, $\bm{\mathrm{Tr}}[\ket{\{\theta_a \}}\!\bra{\{\theta_a \}} \operator{\mathcal{O}}] =  \bra{\{\theta_a \}}\operator{\mathcal{O}}\ket{\{\theta_a \}}$. dans ce cas on notera la moyenne :
\begin{eqnarray}\label{chap.TBA.moy.dens.pur}
	\braket{\operator{\mathcal{O}}}_{\{\theta_a \}} = \bra{\{\theta_a \}} \operator{\mathcal{O}} \ket{\{\theta_a \}},
\end{eqnarray}
où l’on note simplement l’ensemble des rapidité ${\theta_a}$ pour désigner l’état pur.

%%%%%%%%%%%%%%%%%%%%%%%%%%%%%%%%%%%%%%%%%%%%%%%%%%
\paragraph{Charges conservées locales diagonales dans la base des états propres.}
Les charges conservées locales $\operator{Q}_i^{(\mathcal{S})}$ est diagonale dans la base des  états propres $\ket{ \{ \theta_a \}}$ , avec pour valeurs propres $\langle \operator{Q}_i^{(\mathcal{S})} \rangle_{\{\theta_a \}} $ 	 :
%\begin{eqnarray}
%	\operator{Q}_i^{(\mathcal{S})} & = & \sum_{ \{\theta_a\} } \langle \operator{Q}_i^{(\mathcal{S})} \rangle_{\{\theta_a \}}  \ket{\{\theta_a \}}\!\bra{\{\theta_a \}}.		
%\end{eqnarray}
\begin{eqnarray}\label{chap.TBA.Qi.diag}
	\operator{Q}_i^{(\mathcal{S})}\ket{\{\theta_a \}} & = &  \langle \operator{Q}_i^{(\mathcal{S})} \rangle_{\{\theta_a \}}  \ket{\{\theta_a \}}.		
\end{eqnarray}
%%%%%%%%%%%%%%%%%%%%%%%%%%%%%%%%%%%%%%%%
\paragraph{Probabilité d’un état à rapidités fixées.}
On peut alors définir la probabilité d’occurrence d’un état $\ket{\{ \theta_a \} }$ comme la moyenne de la matrice densité locale $\operator{\rho}^{(\mathcal{S})}_{\mathrm{GGE}}$ définie dans \eqref{chap.TBA.op.rho.S}:
\begin{eqnarray}
	\mathbb{P}^{(\mathcal{S})}_{\{ \theta_a \}}  & \equiv &  \langle \operator{\rho}^{(\mathcal{S})}_{\mathrm{GGE}} \rangle_{\{\theta_a \}}, \label{chap.TBA.P.1}\\
	& = & 
	\frac{1}{Z^{(\mathcal{S})}} \exp \left (- \sum_i \beta_i \langle \operator{Q}_i^{(\mathcal{S})} \rangle_{\{\theta_a \}} \right ) \label{chap.TBA.P.2}.
\end{eqnarray}

%%%%%%%%%%%%%%%%%%%%%%%%%%%
\paragraph{Moyenne d’un charges conservées locales et dérivées de $Z^{(\mathcal{S})}$.} Les charges locales $\operator{Q}_i^{(\mathcal{S})}$ sont diagonale dans la bases \( \{ \ket{\{\theta_a \}} \}  \) [cf eq~ ~\eqref{chap.TBA.Qi.diag}]. 
On peut donc  écrire la moyenne d’une observable comme une somme pondérée par cette probabilité [cf eqs ~\eqref{chap.TBA.P.1}-\eqref{chap.TBA.P.2}] , ou encore comme une dérivée de la fonction de partition définie dans l'équation \eqref{chap.TBA.op.Z.S} :
\begin{eqnarray}
	\langle \operator{Q}_i^{(\mathcal{S})} \rangle_{\operator{\rho}^{(\mathcal{S})}_{\mathrm{GGE}}} &= & \sum_{\{ \theta_a\}} \langle \operator{Q}_i^{(\mathcal{S})} \rangle_{\{\theta_a \}} \mathbb{P}^{(\mathcal{S})}_{\{ \theta_a \}} \label{chap.TBA.moy.1}\\
	 & = &  \left. \frac{1}{Z^{(\mathcal{S})}} \frac{\partial Z^{(\mathcal{S})}}{\partial \beta_i} \right )_{\beta_{j \neq i }}	 \label{chap.TBA.moy.2}
\end{eqnarray}

Par le même raisonnement le moment non centré s'écrit :
\begin{eqnarray}
	\braket{ \operator{Q}_{i_1}^{(\mathcal{S})} \, \operator{Q}_{i_2}^{(\mathcal{S})} \cdots \operator{Q}_{i_q}^{(\mathcal{S})} }_{\operator{\rho}^{(\mathcal{S})}_{\mathrm{GGE}}} &= &  (-1)^q \frac{1}{Z^{(\mathcal{S})}} \left.\frac{\partial}{\partial \beta_{i_1}} \right )_{\beta_{j \neq i_1 }} \left.\frac{\partial}{\partial \beta_{i_2}} \right )_{\beta_{j \neq i_2 }} \cdots \left.\frac{\partial}{\partial \beta_{i_q}} \right )_{\beta_{j \neq i_q }} Z^{(\mathcal{S})} \label{chap.TBA.mom.1}.	
\end{eqnarray}

%%%%%%%%%%%%%%%%%%%%%%%%%%%%%%%
\paragraph{Moments d’ordre supérieur et fluctuations.} On s'avance sur le chapitre (\ref{chap:Fluctu}).
Le premier et second moments permettent d’accéder à la variance 
\begin{eqnarray}
	 \left \langle \left (\operator{Q}_i^{(\mathcal{S})} - \langle\operator{Q}_i^{(\mathcal{S})} \rangle_{\operator{\rho}^{(\mathcal{S})}_{\mathrm{GGE}}} \right )^2  \right \rangle_{\operator{\rho}^{(\mathcal{S})}_{\mathrm{GGE}}} = \langle(\operator{Q}_i^{(\mathcal{S})})^2 \rangle_{\operator{\rho}^{(\mathcal{S})}_{\mathrm{GGE}}}  -  \langle\operator{Q}_i^{(\mathcal{S})} \rangle_{\operator{\rho}^{(\mathcal{S})}_{\mathrm{GGE}}}^2	
\end{eqnarray}
de le charge locale $\operator{Q}_i^{(\mathcal{S})}$, en injectant \eqref{chap.TBA.moy.2} et \eqref{chap.TBA.mom.1} et en utilisant $\frac{1}{f} \partial_x^2 f - ( \frac{1}{f} \partial_x f ) = \partial_x^2 \ln f  $:
\begin{eqnarray}
	\left \langle \left (\operator{Q}_i^{(\mathcal{S})} - \langle\operator{Q}_i^{(\mathcal{S})} \rangle_{\operator{\rho}^{(\mathcal{S})}_{\mathrm{GGE}}} \right )^2  \right \rangle_{\operator{\rho}^{(\mathcal{S})}_{\mathrm{GGE}}}  &=&	  \left . \frac{\partial^2 \ln Z^{(\mathcal{S})}  }{{\partial \beta_i}^2 }  \right )_{\beta_{j\neq i}},\\
	& = &  - \left . 	\frac{\partial \langle\operator{Q}_i^{(\mathcal{S})} \rangle_{\operator{\rho}^{(\mathcal{S})}_{\mathrm{GGE}}} }{\partial \beta_i } \right )_{\beta_{j\neq i}}.	
\end{eqnarray}

%%%%%%%%%%%%%%%%%%%%%%%%%%%%%%
\paragraph{Cas particulier de l’équilibre thermique.}

Dans le cas particulier de l’équilibre thermique standard (\ie Gibbsien), le système est décrit par une seule contrainte d’énergie (ou d’énergie et de particule, dans le cas d’un grand canonique). Les multiplicateurs de Lagrange associés aux charges conservées peuvent alors être identifiés à des grandeurs thermodynamiques classiques.

\begin{itemize}[label=$\bullet$]
	\item Si la seule charge conservée est le nombre de particules $\operator{Q}_0^{(\mathcal{S})} = \operator{Q}$, le multiplicateur associé est $\beta_0 = -\beta \mu$, où $\mu$ est le potentiel chimique et $\beta = T^{-1}$ l’inverse de la température (avec $k_B = 1$).
	
	\item Si la charge conservée est $\operator{Q}_2^{(\mathcal{S})}-\mu\operator{Q}_0^{(\mathcal{S})}  = \operator{K} - \mu \operator{Q} $ (ensemble grand canonique), alors le multiplicateur est simplement $ \beta$.
\end{itemize}

Dans le cadre de l’équilibre thermique , les moyennes et les fluctuations thermodynamiques usuelles s’expriment naturellement comme dérivées du logarithme de la fonction de partition $Z^{(\mathcal{S})}$ :
\begin{eqnarray}
	\langle \operator{Q} \rangle_{\operator{\rho}^{(\mathcal{S})}_{\mathrm{GGE}}}  = \left .\frac{1}{\beta} \frac{ \partial \ln Z^{(\mathcal{S})}}{\partial \mu } \right )_{T},  & &  \left . \frac{1}{\beta} \frac{ \partial \langle \operator{Q} \rangle_{\operator{\rho}^{(\mathcal{S})}_{\mathrm{GGE}}}}{\partial \mu } \right )_{T} =  \left . \frac{1}{\beta^2} \frac{ \partial^2 \ln Z^{(\mathcal{S})}}{{\partial \mu}^2 } \right )_{T} \\
	\langle \operator{H} - \mu\operator{Q}  \rangle_{\operator{\rho}^{(\mathcal{S})}_{\mathrm{GGE}}}  = -\left . \frac{ \partial \ln Z^{(\mathcal{S})}}{\partial \beta } \right )_{\mu} ,  & & -\left .  \frac{ \partial \langle \operator{H} - \mu\operator{Q} \rangle_{\operator{\rho}^{(\mathcal{S})}_{\mathrm{GGE}}}}{\partial \beta } \right )_{\mu } = \left .  \frac{ \partial^2 \ln Z^{(\mathcal{S})}}{{\partial \beta}^2 } \right )_{\mu}   .		
\end{eqnarray}
En combinant ces relations, on peut également exprimer l’énergie moyenne et ses fluctuations comme :
\begin{eqnarray}
	\langle \operator{H} \rangle_{\operator{\rho}^{(\mathcal{S})}_{\mathrm{GGE}}}  = \left [ \left .\frac{\mu}{\beta} \frac{ \partial}{\partial \mu } \right )_{T} -\left . \frac{ \partial }{\partial \beta } \right )_{\mu}   \right ]\ln Z^{(\mathcal{S})},  \quad  -\left .  \frac{ \partial \langle \operator{H} \rangle_{\operator{\rho}^{(\mathcal{S})}_{\mathrm{GGE}}}}{\partial \beta } \right )_{-\mu \beta } = \left [ \left .\frac{\mu}{\beta} \frac{ \partial}{\partial \mu } \right )_{T} -\left . \frac{ \partial }{\partial \beta } \right )_{\mu}  \right ]^2\ln Z^{(\mathcal{S})}.		
\end{eqnarray}

%%%%%%%%%%%%%

\section{Remarques sur le formalisme}




%\input{preamble}

\begin{document}

\frontmatter
%\input{chapters/00_intro}
\tableofcontents
\mainmatter

\input{chapters/01_LL_BA}
\input{chapters/02_GGE_TBA}
\input{chapters/03_GHD}
%\input{chapters/97_GHD}
\input{chapters/04_GGE_Fluctuation}
\input{chapters/05_Disp_Exp}
\input{chapters/06_Bipart}
\input{chapters/07_Dipolaire}

%\input{chapters/08_conclusion}
%\appendix
%\input{chapters/99_annexes}

\bibliographystyle{abbrv}
\bibliography{thesis}

%\printbibliography

\end{document}

%| Style     | Description                                                             |
%| --------- | ----------------------------------------------------------------------- |
%| `plain`   | Tri alphabétique, numérotation croissante                               |
%| `unsrt`   | Même que `plain` mais sans tri, respecte l’ordre d’apparition           |
%| `abbrv`   | Comme `plain` mais avec prénoms et noms abrégés                         |
%| `alpha`   | Les références sont étiquetées par une combinaison du nom et de l’année |
%| `apalike` | Style APA simplifié                                                     |
%| `ieeetr`  | Style IEEE, tri par ordre d’apparition                                  |
%| `siam`    | Style SIAM (mathématiques appliquées)                                   |
%| `acm`     | Style ACM (informatique)                                                |
%



\section{Rôle des charges conservées extensives et quasi-locales}
%Dans les systèmes intégrables, l’état stationnaire atteint après une évolution hors d’équilibre n’est généralement pas décrit par un état de Gibbs classique, mais par un ensemble généralisé de Gibbs (GGE). Celui-ci est construit à partir de toutes les charges conservées du système

\paragraph{Écriture des observables thermodynamiques comme sommes sur les rapidités.}

%Dans le cas thermique, les valeurs moyennes des observables classiques telles que le nombre de particules et l'énergie peuvent s'exprimer comme des sommes de puissances des rapidités :
Dans un système à $N$ particules caractérisé par des rapidités $\{ \theta_a \}_{a = 1}^N$, les charges conservées classiques — telles que le nombre de particules, l’impulsion ou l’énergie — s’écrivent comme des sommes de puissances des rapidités :
\(
	\langle \operator{Q} \rangle_{\{ \theta_a\} } \propto \sum_{a = 1}^N \theta_a^0 , \,  \langle \operator{P} \rangle_{\{ \theta_a\} } \propto \sum_{a = 1}^N \theta_a^1  ,\,  \mbox{et} \langle \operator{K} \rangle_{\{ \theta_a\} } \propto \sum_{a = 1}^N \theta_a^2 .	
\)
(cf. équations \eqref{chap.2.gge.1})
Dans ce paragraphe précédent, nous avons sous-entendu — sans l’expliciter — qu’il est montré que l’ensemble des charges locales conservées forme une famille donnée par :
\begin{eqnarray}
	\operator{Q}_i^{(\mathcal{S})} \ket{\{\theta_a\} } & \propto & \sum_a \theta_a^i \ket{\{\theta_a\} }.
\end{eqnarray}
Ces charges agissent donc de manière diagonale sur les états de Bethe, avec des valeurs propres correspondant aux moments des rapidités.
%%%%%%%%%%%%%%%%%%%%%%%%%%%%%%%%%%%%%%%%%%%%%%%%%%
\paragraph{Charges locales conservées .\label{sec:charges-gen}}

%Les états propres du Hamiltonien de Lieb–Liniger~\eqref{eq:LL} sont les états de Bethe
%\(
%  \ket{\boldsymbol{\theta}}
%  =\ket{\theta_1,\dots,\theta_N}\!,
%\)
%déterminés par leurs rapidités \(\boldsymbol{\theta}\).

À toute fonction régulière
\(
  f:\mathbb R\!\to\!\mathbb R
\)
on associe un opérateur-charge loclal :
\begin{eqnarray}\label{chap.2.charge.f.1}
	\operator{\mathcal{Q}}^{(\mathcal{S})}[f] & = &  L \int_0^L d\theta \, f(\theta) \operator{\rho}^{(\mathcal{S})}(\theta).	
\end{eqnarray}
où $\operator{\rho}(\theta)$ agit sur une état de Bethe comme 
\begin{eqnarray}\label{chap.2.rho.1}
	 \operator{\rho}(\theta) \ket{ \{ \theta_a \} } &=& \frac{1}{L} \sum_{a = 1 }^N  \delta ( \theta - \theta_a ) \ket{ \{ \theta_a \} }.	
\end{eqnarray}
De sorte que $\operator{\mathcal{Q}}^{(\mathcal{S})}[f]$ agit sur une état de Bethe comme
\begin{eqnarray}\label{chap.2.charge.1}
	\operator{\mathcal{Q}}^{(\mathcal{S})}[f]\,\ket{\{\theta_a\} } =  \sum_{a=1}^{N}f(\theta_a)\,\ket{\{\theta_a\} } \quad \mbox{de sorte que} \quad \braket{\operator{\mathcal{Q}}^{(\mathcal{S})}[f]}_{\{\theta_a\}} = \sum_{a=1}^N f(\theta_a)
\end{eqnarray}
Les choix particuliers
\(
  f_0(\theta)=1
\)
,
\(
  f_1(\theta)=\theta
\)
et
\(
  f_2(\theta)=\theta^{2}/2
\)
redonnent respectivement l'opérateur nombre \(\operator{Q}=\operator{Q}_0^{(\mathcal{S})} = \operator{\mathcal{Q}}^{(\mathcal{S})}[1]\) , impulsion \(\operator{P}=\operator{Q}_1^{(\mathcal{S})} = \operator{\mathcal{Q}}^{(\mathcal{S})}[\theta]\) et énergie cinétique
\(\operator{K}=\operator{Q}_2^{(\mathcal{S})} = \operator{\mathcal{Q}}^{(\mathcal{S})}[\theta^2/2]\). Et dans le cadre des (GGE), pour tous les ordres $i$ on note :
\begin{eqnarray}\label{chap.2.charge.ordre.i.1}
	\operator{Q}^{(\mathcal{S})}_i = \operator{\mathcal{Q}}^{(\mathcal{S})}[f_i]	, \quad \mbox{de sorte que} \quad \braket{\operator{Q}^{(\mathcal{S})}_i}_{\{\theta_a\}} = \sum_{a=1}^N f_i(\theta_a)  
\end{eqnarray}
avec les densités spectrales $f_i(\theta) \propto \theta^i$ . 

Ces charges sont extensives : leur densité locale $\operator{q}^{(\mathcal{S})}_{[f]}$ permet d’écrire
\(
  \operator{\mathcal{Q}}^{(\mathcal{S})}[f]=\int_0^{L}\!dx\;\operator{q}^{(\mathcal{S})}_{[f]}(x).
\)

\paragraph{Charges conservées généralisée.\label{sec:charges-gen}}
Les fonction $f_i$ étant fixées, on note la fonction régulière
\(
  w:\mathbb R\!\to\!\mathbb R
\)
–– dorénavant appelée \emph{poids spectral}, ou \emph{potentiel spectral} ––
\begin{eqnarray}
	w = \sum_i \beta_i f_i \label{chap.2.w.1},	
\end{eqnarray}
on associe un opérateur-charge généralisé $\operator{\mathcal{Q}}^{(\mathcal{S})}[w]$ :
\begin{eqnarray}\label{chap.2.charge.gen.1}
	\operator{\mathcal{Q}}^{(\mathcal{S})}[w]\,\ket{\{\theta_a\} } =  \sum_{a=1}^{N}w(\theta_a)\,\ket{\{\theta_a\} } \quad \mbox{de sorte que} \quad \braket{\operator{\mathcal{Q}}^{(\mathcal{S})}[w]}_{\{\theta_a\}} = \sum_{i} \beta_i  \braket{\operator{Q}^{(\mathcal{S})}_i}_{\{\theta_a\}}
\end{eqnarray}

%%%%%%%%%%%%%%%%%%%%%%%%%%%%%%%%%%%%%%%%%
\paragraph{Expression de la matrice densité généralisée.}
La matrice densité  s’écrit sous la forme :
L’ensemble général défini par $\operator{\varrho}^{(\mathcal{S})}[w]$ 
\begin{eqnarray}\label{chap.2.densite.1}
	\operator{\varrho}^{(\mathcal{S})}[w]  =  \frac{e^{-\operator{\mathcal{Q}}^{(\mathcal{S})}[w]}}{Z^{(\mathcal{S})}[w]}, \, \mbox{avec} \quad e^{-\operator{\mathcal{Q}}^{(\mathcal{S})}[w]}  = 	\sum_{\{\theta_a \}} e^{- \sum_{a = 1}^N w(\theta_a) } \vert \{ \theta_a\} \rangle \langle  \{ \theta_a\}  \vert, 
\end{eqnarray}	
	%pour une certaine fonction $w$ relié à la charge% $\operator{\mathcal{Q}} [w]  = \sum_{\{\theta_a \}} \left ( \sum_{a = 1}^N w ( \theta_a )  \right ) \vert \{ \theta_a \} \rangle \langle \{ \theta_a \} \vert $.
%où l'opérateur de charge associé à $w$ s’écrit :
%\begin{eqnarray}
%	\operator{\mathcal{Q}} [w]   & = &  \sum_{\{\theta_a \}} \left ( \sum_{a = 1}^N w ( \theta_a )  \right ) \vert \{ \theta_a \} \rangle \langle \{ \theta_a \} \vert,	
%\end{eqnarray}
et la fonction de partition \eqref{chap.TBA.op.Z.S} s'écrit $Z^{(\mathcal{S})}[w]\doteq \bm{\mathrm{Tr}}\left [ e^{-\operator{\mathcal{Q}}^{(\mathcal{S})}[w]}\right ] $ vaux :
\begin{eqnarray}
	Z^{(\mathcal{S})}[w]   =  \sum_{\{\theta_a \}} e^{-\sum_{a = 1}^N w(\theta_a)},\label{chap.TBA.op.Z.S.1}	
\end{eqnarray}
devient un Generalized Gibbs Ensemble (GGE), $\operator{\rho}^{(\mathcal{S})}_{\mathrm{GGE}}$ (de l'équation \eqref{chap.TBA.op.rho.S})	 dès lors que $w(\theta) = \sum_i \beta_i f_i(\theta)$ (de l'équation \eqref{chap.2.w.1}) où $f_i$ sont les densités spectrales associées aux charges locales conservées (de l'équation \eqref{chap.2.charge.ordre.i.1}).


%%%%%%%%%%%%%%%%%%%%%%%%%%%%%%%%%%
\paragraph{Probabilité associée à une configuration de rapidités.}
	%Et on peut réecrire la probabilité de la configuration $\{\theta_a\}$ :% $ P_{\{ \theta_a \}} = \langle \{ \theta_a \}\vert \operator{\rho}_{GGE}[w] \vert  \{ \theta_a \} \rangle = e^{-\sum_{a = 1}^N w(\theta_a)} / Z $ avec $Z = \sum_{\{\theta_a \}} e^{-\sum_{a = 1}^N w(\theta_a)}$.\\
	%La probabilité d’occuper un état à $N$ particules caractérisé par les rapidités ${\theta_a}$ est alors :
Dans ce formalisme, la probabilité d’occuper l’état $\ket{\{\theta \}}$ \eqref{chap.TBA.P.1} est donc
\begin{eqnarray}
	\mathbb{P}^{(\mathcal{S})}_{\{ \theta_a \}} & = &  Z^{(\mathcal{S})}[w]^{-1}e^{-\sum_{a = 1}^N w(\theta_a)}\label{chap.TBA.P.w.2}. 		
\end{eqnarray}
%Cela montre que le poids statistique d’une configuration factorise naturellement sur les pseudo-moments, avec un poids spectrale / energie génralisé $w(\theta)$ attribué à chaque particule.
On voit ainsi que le poids statistique factorise naturellement sur les
pseudo‑moments, chaque particule étant pondérée par $w(\theta_a)$.

%avec 
%\begin{eqnarray}
%	Z  & = & \sum_{\{\theta_a \}} e^{-\sum_{a = 1}^N w(\theta_a)}.		
%\end{eqnarray}


%%%%%%%%%%%%%%%%%%%%%%%%
\paragraph{Moyennes d'observables dans le GGE.}
%La valeur moyenne d’un observable locale $\operator{\mathcal{O}}$ dans l’ensemble généralisé s’écrit :
Pour tout opérateur local $\operator{\mathcal{O}}$ diagonal dans la base de Bethe,
la moyenne généralisée vaut
\begin{eqnarray}\label{chap.2.moyenne.1}
	\langle \operator{\mathcal{O}}\rangle_{\operator{\varrho}^{(\mathcal{S})}[w]} & = & \displaystyle   \frac{\sum_{\{\theta_a \}} \braket{ \operator{\mathcal{O}}}_{\{ \theta_a\}} e^{- \sum_{a = 1}^N w(\theta_a) }  }{\sum_{\{\theta_a  \}} e^{- \sum_{a = 1}^N  w(\theta_a) } }
\end{eqnarray}
%Cette expression formelle montre que la connaissance de $w(\theta)$ suffit à déterminer les propriétés statistiques de toutes les observables diagonales dans cette base, incluant les charges conservées elles-mêmes.
Ainsi, la connaissance de la fonction $w(\theta)$ suffit à déterminer
les propriétés statistiques de toute observable diagonale,
y compris les charges conservées elles‑mêmes.	
	% Nous aimerions calculer les valeurs d'attente par rapport à cette matrice de densité, par exemple
	%La moyenne GGE d'un observable s'écrit ,
	%\begin{aff}
	%\begin{eqnarray}
	%	\langle \operator{\mathcal{O}} \rangle_{GGE} & \doteq & \displaystyle  \text{Tr} (\operator{\mathcal{O}}\operator{\rho}[w]) = \frac{\text{Tr} (\operator{\mathcal{O}}e^{-\operator{\mathcal{Q}}[w]})}{\text{Tr} (e^{-\operator{\mathcal{Q}}[w]})}	 = \frac{\sum_{\{\theta_a \}} \langle  \{ \theta_a\}  \vert   \operator{\mathcal{O}} \vert \{ \theta_a\} \rangle e^{- \sum_{a = 1}^N w(\theta_a) }  }{\sum_{\{\theta_a  \}} e^{- \sum_{a = 1}^N  f(\theta_a) } }
		%& =  & \frac{ \sum_{\pi} \sum_{\vert \{\theta_a \}\rangle \vert \Pi } \langle  \{ \theta_a\}  \vert   \operator{\mathcal{O}} \vert \{ \theta_a\} \rangle e^{- \sum_{a = 1}^N f(\theta_a) }  }{\sum_{\pi} \sum_{\vert \{\theta_a \}\rangle \vert \Pi }  e^{- \sum_{a = 1}^N  f(\theta_a) } }
	%\end{eqnarray}
	%pour une certaine observable $\operator{\mathcal{O}}$.\\
	%\end{aff}
	

\paragraph{Conclusion de la section : vers la thermodynamique de Bethe.}

Nous avons vu que, dans un système intégrable, la description correcte de l’équilibre stationnaire requiert l’introduction d’une \emph{famille infinie de charges conservées}, comprenant à la fois des charges strictement locales et des charges quasi‑locales.
Toutes ces charges se réunissent dans l’opérateur fonctionnel
\(
\operator{\mathcal{Q}}^{(\mathcal{S})}[w]
\)
, défini par un \emph{poids spectral}  $w(\theta)$ (cf. équations~\eqref{chap.2.charge.1}).
Cette construction conduit naturellement à la matrice densité généralisée
\(
\operator{\rho}^{(\mathcal{S})}_{\mathrm{GGE}}  \propto  e^{-\operator{\mathcal{Q}}^{(\mathcal{S})}[w]}
\) 
(cf. équations~\eqref{chap.2.densite.1}), et à la moyenne d’un opérateur local $\operator{\mathcal{O}}$ donnée par
\(
\langle \operator{\mathcal{O}}\rangle_{\operator{\rho}^{(\mathcal{S})}_{\mathrm{GGE}}}  =  \displaystyle  \text{Tr} (\operator{\mathcal{O}}\operator{\varrho}^{(\mathcal{S})}[w])
\)
(cf. équations~\eqref{chap.2.moyenne.1}).
La connaissance de $w(\theta)$ suffit donc pour prédire les valeurs moyennes de toutes les observables diagonales, y compris celles des charges elles‑mêmes ; c’est le cœur du {\bf Ensemble de Gibbs Généralisé (GGE pour Generalized Gibbs Ensemble)} .

\medskip
Cette base est désormais posée : dans la section suivante, nous passerons au \emph{thermodynamique de Bethe}.
Nous verrons comment, dans la limite thermodynamique, les sommes sur les configurations de rapidités se transforment en intégrales sur des densités continues, comment apparaît l’entropie de Yang–Yang, et comment les moyennes de l’ensemble généralisé se réexpriment à l’aide de ces densités macroscopiques.
C’est ce formalisme qui permettra d’analyser finement la relaxation post‑quench et de relier microscopie intégrable et hydrodynamique généralisée.



%\input{preamble}

\begin{document}

\frontmatter
%\input{chapters/00_intro}
\tableofcontents
\mainmatter

\input{chapters/01_LL_BA}
\input{chapters/02_GGE_TBA}
\input{chapters/03_GHD}
%\input{chapters/97_GHD}
\input{chapters/04_GGE_Fluctuation}
\input{chapters/05_Disp_Exp}
\input{chapters/06_Bipart}
\input{chapters/07_Dipolaire}

%\input{chapters/08_conclusion}
%\appendix
%\input{chapters/99_annexes}

\bibliographystyle{abbrv}
\bibliography{thesis}

%\printbibliography

\end{document}

%| Style     | Description                                                             |
%| --------- | ----------------------------------------------------------------------- |
%| `plain`   | Tri alphabétique, numérotation croissante                               |
%| `unsrt`   | Même que `plain` mais sans tri, respecte l’ordre d’apparition           |
%| `abbrv`   | Comme `plain` mais avec prénoms et noms abrégés                         |
%| `alpha`   | Les références sont étiquetées par une combinaison du nom et de l’année |
%| `apalike` | Style APA simplifié                                                     |
%| `ieeetr`  | Style IEEE, tri par ordre d’apparition                                  |
%| `siam`    | Style SIAM (mathématiques appliquées)                                   |
%| `acm`     | Style ACM (informatique)                                                |
%




\section{Thermodynamique de Bethe et relaxation}

%------------------------------------------------------------------
\subsection{Limite thermodynamique}

\paragraph{Observables locales dans la limite thermodynamique.}
%Lorsque l'observable $\operator{\mathcal{O}}$ est suffisamment local, on croit que la valeur d'attente $\langle  \{ \theta_a\}  \vert   \mathcal{O} \vert \{ \theta_a\} \rangle$ ne dépend pas de l'état microscopique spécifique du système, de sorte qu'elle devient une fonctionnelle de $\Pi$ dans la limite thermodynamique.
Dans la suite de ce chapitre, nous omettrons l’exposant $(\mathcal{S})$.
\vspace{0.2em}
Dans la base des états de Bethe \( \{ \ket{\{ \theta_a \}} \} \), l’opérateur \( \hat{\rho}(\theta) \) défini en \eqref{chap.2.rho.1} est diagonal, et agit comme un projecteur sur les valeurs de rapidité.

\vspace{0.5em}

Dans la limite thermodynamique, différentes configurations microscopiques \( \{ \theta_a \} \) peuvent correspondre à la même distribution de rapidité macroscopique \( \rho(\theta) \). Autrement dit, plusieurs états \( \ket{\{ \theta_a \}} \) partagent la même valeur propre \( \rho(\theta) \) de l’opérateur \( \operator{\rho}(\theta) \). Cela reflète une {\em dégénérescence macroscopique} induite par le passage à la limite thermodynamique (\( N, L \to \infty \) avec \( N/L \to \text{const} \)).

\vspace{0.5em}

Si l’observable $\mathcal{O}$ est suffisamment locale, sa valeur d’attente dans un état propre ne dépend pas des détails microscopiques, mais uniquement de la distribution de rapidité. On écrit alors :
\begin{eqnarray}
	\underset{\mbox{\tiny therm.}}{\lim} \braket{  \operator{\mathcal{O}} }_{\{ \theta_a\}}  & = & \langle \operator{\mathcal{O}}\rangle_{[\rho]},
\end{eqnarray}
où $\underset{\mbox{\tiny therm.}}{\lim}$ est la limite thermodynamique ($N,L \to \infty$ avec $N/L \to $ const) et où \( \langle \mathcal{O} \rangle_{[\rho]} \) désigne la valeur d’attente de \( \mathcal{O} \) dans un état macroscopique caractérisé par la distribution de rapidité \( \rho(\theta) \).


\medskip
Dans un ensemble général (GGE), la valeur moyenne de l’observable \eqref{chap.2.moyenne.1} devient alors :		
\begin{eqnarray}\label{chap.2.moyenne.2}
	\underset{\mbox{\tiny therm.}}{\lim} \langle \operator{\mathcal{O}} \rangle_{\operator{\varrho}[w]} & =  & \frac{  \displaystyle \sum_{\rho }  \langle \operator{\mathcal{O}}\rangle_{[\rho]} \Omega[\rho] e^{- \sum_{a = 1}^N  w(\theta_a)    }}{ \displaystyle \sum_{\rho}   \Omega[\rho]\,e^{- \sum_{a = 1}^N  w(\theta_a) } } ,
\end{eqnarray}
où $\sum_{\rho }$ est une somme sus tous les distribution de rapidité $\rho$ et 
où $\Omega[\rho]$ désigne le nombre de micro-états compatibles avec la distribution de rapidité $\rho$.

%où $\# \mbox{micro-états.}$ est les nombre de micro état associée àa la distribution de rapidité $\rho$.
%Avant de se plonger sur $\# \mbox{micro-états.}$, regardons le changement des équation de Bethes. 

\medskip
Pour établir la fonction $\Omega[\rho]$, reppelons-nons de la transformation des équations de Bethe dans dans la limite thermodynamique, hors état fondamentale \eqref{eq:TBA-nu} et \eqref{eq:TBA-rhos-2}.
\begin{equation}
	\nu = \frac{\rho}{\rho_s} \, , \qquad 2\pi \rho_s = 1^{\mathrm{dr}}_{[\nu]} 
\label{chap.2:eq:TBA-rhos}
\end{equation}
où $f^{\mathrm{dr}}_{[\nu]}$ est définie en \eqref{eq:dessing}.

\medskip

Cette formalisation constitue la brique de base de la \textbf{hydrodynamique généralisée} et, dans la section suivante, permet de définir rigoureusement l’\textbf{entropie de Yang–Yang}, indispensable pour décrire la relaxation hors d’équilibre des systèmes intégrables.

%\vspace{1ex}
%La formalisation ci‑dessus fournit la brique de base pour la
%\textbf{hydrodynamique généralisée} et, dans la section suivante, pour la
%définition précise de l’\textbf{entropie de Yang-Yang}
%assurant la relaxation des systèmes intégrables hors‑équilibre.

%\input{preamble}

\begin{document}

\frontmatter
%\input{chapters/00_intro}
\tableofcontents
\mainmatter

\input{chapters/01_LL_BA}
\input{chapters/02_GGE_TBA}
\input{chapters/03_GHD}
%\input{chapters/97_GHD}
\input{chapters/04_GGE_Fluctuation}
\input{chapters/05_Disp_Exp}
\input{chapters/06_Bipart}
\input{chapters/07_Dipolaire}

%\input{chapters/08_conclusion}
%\appendix
%\input{chapters/99_annexes}

\bibliographystyle{abbrv}
\bibliography{thesis}

%\printbibliography

\end{document}

%| Style     | Description                                                             |
%| --------- | ----------------------------------------------------------------------- |
%| `plain`   | Tri alphabétique, numérotation croissante                               |
%| `unsrt`   | Même que `plain` mais sans tri, respecte l’ordre d’apparition           |
%| `abbrv`   | Comme `plain` mais avec prénoms et noms abrégés                         |
%| `alpha`   | Les références sont étiquetées par une combinaison du nom et de l’année |
%| `apalike` | Style APA simplifié                                                     |
%| `ieeetr`  | Style IEEE, tri par ordre d’apparition                                  |
%| `siam`    | Style SIAM (mathématiques appliquées)                                   |
%| `acm`     | Style ACM (informatique)                                                |
%






\subsection{Statistique des macro-états : entropie de Yang-Yang}

%\paragraph{Macro-états et entropie dans la TBA.}

%Dans la limite thermodynamique, dans le modèle statistique (GGE) , les moyenne, observables physiques deviennent des fonctionnelles de la {\bf distribution de rapidité}  $\rho(\theta)$ et du {\bf poing spectrale} $w(\theta)$ . Cette description est efficace car elle permet d’échapper au détail de chaque état propre. 
%Toutefois, cette simplification laisse en suspens une question cruciale : 
%Mais dans ce modelle qui simplifie on veux {\bf la distribution de rapidité d’un système à l'équilibre thermique à température finie} que l'on notera $\langle \rho \rangle$ pour dire la dansité moyenne. Et les lien entre  $w$ et $\langle \rho \rangle$.  Le problème est étudier par par Yang et Yang en 1969. Pour saisir l'enssentielle, nous devons comprendre la {\bf structure statistique des états propres} associés à une même distribution $\rho(\theta)$. Nous nous interrensons comme promis plus haut : à $\Omega(\theta)$ dans l'équation de moyenne \eqref{chap.2.moyenne.2}  ,  {\bf  nombre états propres microscopiques correspondent à une même distribution $\rho(\theta)$}.
%{\bf quelle est la distribution de rapidité d’un système à l'équilibre thermique à température finie ?}. 
%La question a été répondue dans les travaux pionniers de Yang et Yang (1969), que nous allons maintenant examiner brièvement. Pour répondre à cette question, nous devons comprendre la {\bf structure statistique des états propres} associés à une même distribution $\rho(\theta)$.

\paragraph{Motivation.}

Dans la limite thermodynamique, une observable locale dans un \textit{Generalized Gibbs Ensemble} (GGE) dépend uniquement de deux objets continus :  (i)  la \textbf{distribution de rapidité} $\rho(\theta)$, (ii) le \textbf{poids spectral} $w(\theta)$, c.-à-d.\ la " température généralisée " assignée à chaque quasi‑particule.
Cette reformulation est puissante car elle fait disparaître les détails d’un état propre individuel.  

\medskip
Cependant, pour décrire un \emph{vrai} équilibre à température finie, il faut la distribution à l'équilibre :
\begin{eqnarray}\label{chap.2:eq.rho.eq.1}
	\rho_{\mathrm{eq}}(\theta)\;\doteq\;\braket{\operator{\rho}(\theta)}_{\operator{\varrho}[w]}	,  
\end{eqnarray}
donc le lien entre $\rho_{\mathrm{eq}}$ et $w$.
La réponse fut donnée dans les travaux pionniers de \textsc{Yang \& Yang} (1969).  
Leur approche repose sur l’analyse de la \textbf{structure statistique des états propres} partageant la même distribution $\rho(\theta)$.

% : combien d’états microscopiquement distincts correspondent à ce même « macro‑état » ?

\paragraph{Distribution de rapidité comme macro-état.}

Chaque distribution de rapidité $\rho(\theta)$ ne correspond pas à un état propre unique, mais à un grand {\bf ensemble de micro-états} : différents choix des ensembles de quasi-moments $(\{\theta_a\}_{a \in \llbracket 1 , N \rrbracket })_{N \in \mathbb{Z}} $ peuvent conduire à la même densité de distribution à l’échelle macroscopique. Ainsi, $\rho(\theta)$ doit être interprétée comme un {\bf macro-état}, qui agrège un très grand nombre d’états propres microscopiques.

La question thermodynamique devient alors : {\bf Combien de micro-états microscopiquement distincts sont compatibles avec un même macro-état $\rho(\theta)$ ?} 

\medskip
Plus précisément, dans l’expression de moyenne des operateurs locaux \eqref{chap.2.moyenne.2}, apparaît le facteur
\(
\Omega[\rho]
\),
qui compte ces états propres.  
La détermination de $\Omega[\rho]$ (ou équivalemment de l’entropie de Yang–Yang $\mathcal{S}_{YY}[\rho]$ car 
\(
\Omega[\rho] = e^{L\mathcal{S}_{YY}[\rho]}
\)
avec $L$ la taille du système
) est donc la clé pour relier \emph{(i)} le poids spectral $w(\theta)$ imposé dans le GGE et \emph{(ii)} la distribution de rapidité moyenne $\rho_{\mathrm{eq}}(\theta)$ observée à l’équilibre.

\paragraph{Dénombrement local des configurations microcanoniques.}
Pour répondre à cette question, on subdivise l’axe des rapidités en petites tranches ou cellules de largeur $\delta \theta$, chacune centrée en un point $\theta_a$. Dans une tranche $[\theta_a, \theta_a + \delta\theta]$, on suppose que la densité $\rho(\theta)$ est à peu près constante. Le nombre de quasi-particules dans cette tranche est alors approximativement :
\begin{eqnarray*}
	N_a = L\rho(\theta_a) \delta \theta,
\end{eqnarray*}
et le nombre total d'états disponibles (\ie, le nombre d’états possibles si toutes les positions en moment étaient disponibles) est donné par la densité totale de niveaux 
\begin{eqnarray*}
	M_a = L\rho_s(\theta_a) \delta \theta.
\end{eqnarray*}
%La densité de niveaux $\rho_s(\theta)$ tient compte du fait que les moments sont quantifiés de manière discrète, en raison des équations de Bethe (voir équation (??)).

Les particules occupent ces niveaux de manière analogue à des fermions libres (principe d’exclusion de Pauli), le nombre de manières différentes de choisir $N_a$ niveaux parmi $M_a$ est donné par :
	
	
	\begin{figure}[H]
		\centering 
		\begin{tikzpicture}
			%\input{figures/04_GGE_Fluctuation/Occupation_code}	
			\begin{scope}[transform canvas={scale=0.6}]
			\input{figures/04_GGE_Fluctuation/Occupation_theta_code}	
			\end{scope}
			
			\draw[color = red , scale = 0.5 , draw = none ] (-13.5 , -1) rectangle (13 , 10) ; 
				
			
		\end{tikzpicture}	
		\captionsetup{skip=10pt} % Ajoute de l’espace après la légende
	\end{figure}
	
	
\begin{eqnarray}
	\Omega(\theta_a) & \approx  & \binom{M_a}{N_a} ~= ~   \frac{[ L\rho_s ( \theta ) \delta \theta ] ! }{ [ L\rho ( \theta ) \delta \theta ] ! [( L\rho_s ( \theta ) - L\rho ( \theta ) )  \delta \theta ] ! }. 	
\end{eqnarray}

\paragraph{Estimation asymptotique à l’aide de Stirling.}

En utilisant la formule de Stirling :
\begin{eqnarray}
	n! & \underset{n \to \infty}{\sim} &  n^n e^{-n} \sqrt{2\pi n}.,
\end{eqnarray}	
composé du fonction logarithmique, il vient cette équivalence : 
\begin{eqnarray}
	\ln n! & \underset{n \to \infty}{\rightarrow} & n \ln n \underbrace{- n + \ln \sqrt{2 \pi n }}_{o \left ( n \ln n \right ) } ,\\
	&  \underset{n \to \infty}{\sim} & n \ln n  
\end{eqnarray}
	
$\# \mbox{conf.}$ est jamais null donc on peut approximer, pour de grandes valeurs de $L$ et de $\delta\theta$  : 
\begin{eqnarray}
    \ln \Omega(\theta) & \underset{\underset{\rho (\theta )\leq  \rho_s (\theta )}{\rho \delta \theta  \to \infty}}{\sim}   & L [ \rho_s\ln \rho_s - \rho \ln \rho - (\rho_s - \rho ) \ln ( \rho_s - \rho) ] (\theta )\delta \theta .
\end{eqnarray}

Cette expression donne la contribution par unité de $\theta$ à l’{\bf entropie}  associée à la cellule autour de $\theta_a$.

\paragraph{Entropie de Yang-Yang : définition .}
%L'entropie totale du macro-état $\rho(\theta)$, notée $\mathcal{S}_{YY}[\rho]$, est obtenue en sommant sur toutes les tranches. Pour alléger la notation, nous écrivons cette somme comme :
%Le nombre total de micro-états est le produit de toutes ces configurations pour toutes les cellules de rapidité $[\theta, \theta + \delta \theta]$. %En prenant le logarithme et en remplaçant la somme par une intégrale sur $ \theta$, nous obtenons l'entropie de Yang-Yang :

%L'entropie totale du macro-état $\rho(\theta)$, notée $\mathcal{S}_{YY}[\rho]$, est obtenue en sommant sur toutes les tranches. Pour alléger la notation, nous écrivons cette somme comme :
%Le nombre total de micro-états compatibles avec une distribution macroscopique $\rho(\theta)$ est donné par le produit des nombres de configurations pour chaque cellule de rapidité $[\theta, \theta + \delta \theta]$.

%En prenant la sum le logarithme des $\Omega(\theta)$ , on obtient l'entropie totale de Yang-Yang. Pour alléger la notation, cette somme sur les tranches est notée :

Le nombre total de micro-états compatibles avec une distribution macroscopique donnée $\rho(\theta)$ est obtenu en prenant le produit des nombres de configurations pour chaque cellule de rapidité $[\theta_a, \theta_a + \delta \theta]$ : $ \Omega(\theta_a)$ .
En prenant le logarithme de ce produit, on accède à l'entropie totale. Pour alléger la notation, cette somme sur les cellules est notée
\(
	\sum_a^{\theta-\mbox{\tiny cellules}}	
\)
où chaque $a$ indexe une cellule de rapidité $[\theta_a, \theta_a + \delta\theta]$.
On écrit alors :
\begin{eqnarray}
    \ln \Omega[\rho] & = & \sum_a^{\theta-\mbox{\tiny cellules}} \ln \Omega(\theta_a), \\
    & \approx &   L\mathcal{S}_{YY} [ \rho ] , 	
\end{eqnarray}
où l’on définit l’\textbf{entropie de Yang–Yang} par la formule discrétisée :
\begin{eqnarray}
    \mathcal{S}_{YY}[\rho] & \doteq & \sum_a^{\theta-\mbox{\tiny cellules}} \, [ \rho_s\ln \rho_s - \rho \ln \rho - ( \rho_s - \rho ) \ln ( \rho_s - \rho ) ] (\theta_a) \delta \theta .
\end{eqnarray}

%\paragraph{Énergie généralisée.}	
%Les variations de $w(\theta)$ étant négligeables sur chaque tranche de largeur $\delta\theta$, on peut approximer l’énergie généralisée comme :%  $\sum_{a = 1}^N  f(\theta_a) = \sum_{a \vert tranche } f(\theta_a) \Pi( \theta_a)\delta \theta$.

%\begin{eqnarray}
%	 \mathcal{W} & = & \sum_{a = 1}^N  w(\theta_a)	 ~ \sim ~ L\mathcal{W}[\rho] ~=~ L \sum_a^{\theta-\mbox{\tiny tranches}}	 w(\theta_a) \rho(\theta_a) \delta \theta.
%\end{eqnarray}

\paragraph{Énergie généralisée par unité de longueur : définition.}

Dans le cadre du Generalized Gibbs Ensemble (GGE), l’\textbf{énergie généralisée} associée à une distribution de rapidité $\rho(\theta)$ et à un poids spectral $w(\theta)$ est définie comme la somme des poids assignés à chaque quasi-particule. 
Dans la limite thermodynamique, en supposant que $w(\theta)$ varie lentement sur chaque tranche $[\theta_a, \theta_a + \delta\theta]$ ,  cette somme soit l’\textbf{énergie généralisée par unité de longueur} $\mathcal{W}$ se se définit par :
\begin{eqnarray}
	L \mathcal{W}(\{\theta_a\}) \doteq  \sum_{a = 1}^N w(\theta_a) 
	 \underset{\mbox{\tiny therm .}}{\sim}  L \mathcal{W}[\rho]  \doteq  L \sum_a^{\theta\text{-cellules}} w(\theta_a) \rho(\theta_a)\, \delta\theta. 
\end{eqnarray} 
%La fonctionnelle
%\(
%\mathcal{W}[\rho] = \int d\theta\, w(\theta)\, \rho(\theta)
%\)
%représente donc l’énergie généralisée par unité de longueur, dans l’état macroscopique défini par la distribution $\rho$.


\paragraph{Moyenne des Observables locales dans la limite thermodynamique.}

Dans un ensemble général (GGE), la valeur moyenne de l’observable \eqref{chap.2.moyenne.2} devient :	
	
\begin{eqnarray}\label{chap.2.moyenne.3}
	\underset{\mbox{\tiny therm.}}{\lim} \langle \operator{\mathcal{O}} \rangle_{\operator{\varrho}[w]} &  \approx &  ~ \frac{  \displaystyle \sum_{\rho }  \langle \operator{\mathcal{O}}\rangle_{[\rho]}  e^{L(\mathcal{S}_{YY}[\rho] -  \mathcal{W}[\rho]) }}{ \displaystyle \sum_{\rho } e^{L(\mathcal{S}_{YY}[\rho] -  \mathcal{W}[\rho]) } },
\end{eqnarray}
où la somme $\sum\rho$ porte sur toutes les distributions possibles de rapidité $\rho$

%%%%%%%%%%%%%%%%%%%%%%%%%%%%%%%%%%%%%%%%%
\paragraph{Passage à la limite continue.}
%En faisant tandre $\delta \theta \to 0 $ , les somme devienen des integrales 
En faisant tendre $\delta\theta \to 0$, les sommes deviennent des intégrales 
%\(
%\sum_a^{\theta-\mbox{\tiny tranches}}\delta \theta   \underset{\delta \theta \to 0 }{\rightarrow}  \int d \theta ,	
%\)
et l'entropie de Yang-Yang ainsi que l’énergie généralisée par unité de longueur prennent la forme :
\begin{eqnarray}
	\mathcal{S}_{YY}[\rho] & = & \int d \theta  \, [ \rho_s\ln \rho_s - \rho \ln \rho - ( \rho_s - \rho ) \ln ( \rho_s - \rho ) ] (\theta) , \label{chap.2.entropi.int}\\
	\mathcal{W}[\rho] & = & \int   w(\theta) \rho(\theta) \, d \theta \label{chap.2.W.int}		
\end{eqnarray}

%%%%%%%%%%%%%%%%%%%%%%%%%%%%%%%%%%%%%%%%%%
\paragraph{Formule fonctionnelle pour les moyennes.}

%et la valeur moyenne des opservables $\langle \operator{\mathcal{O}} \rangle$ s'écrit commes une intégrale de chemin/formelle
Dans la limite thermodynamique $L \to \infty$, la somme sur les distributions de rapidité $\rho$ admissibles peut être approximée par une intégrale fonctionnelle sur l’espace des densités de rapidité continues, munie d’une mesure fonctionnelle $\mathcal{D}\rho$ : 
\(
\sum_{\rho } \sim \int \mathcal{D} \rho .
\)
Cette correspondance repose sur l’idée que les macro-états admissibles deviennent denses dans l’espace fonctionnel, et que le poids statistique associé à chaque configuration est donné par l’entropie de Yang–Yang.
La mesure fonctionnelle $\mathcal{D}\rho$ parcourt l’espace des densités
$\rho(\theta)$ continues, \emph{chaque configuration étant pondérée par le
facteur exponentiel}
\(
e^{\,L(\mathcal{S}_{YY}[\rho]-\mathcal{W}[\rho])}.
\)
Finalement, la moyenne d'une observable dans le GGE \eqref{chap.2.moyenne.3} s’écrit comme une intégrale fonctionnelle/de chemin :
\begin{eqnarray}
	\underset{\mbox{\tiny therm.}}{\lim} \langle \operator{\mathcal{O}} \rangle_{\operator{\varrho}[w]} & = & \frac{\int \mathcal{D} \rho \; e^{L (\mathcal{S}_{YY}[\rho] - \mathcal{W}[\rho])} \, \langle\operator{\mathcal{O}}\rangle_{[\rho]}}{\int \mathcal{D} \rho \; e^{L (\mathcal{S}_{YY}[\rho] - \mathcal{W}[\rho])}}. \label{chap:TBA:eq:ensemble_average}
\end{eqnarray}


%----------------------
%------------------------------------------------------------------
%\paragraph{Passage de la somme discrète à l’intégrale fonctionnelle.}

%Dans la limite thermodynamique $L\to\infty$, l’ensemble (discret) des
%distributions de rapidité admissibles devient dense dans l’espace
%fonctionnel ; la somme correspondante peut donc s’approximer par une
%intégrale fonctionnelle :
%\[
%\sum_{\rho}\; \longrightarrow\; \int\! \mathcal{D}\rho .
%\]
%La mesure fonctionnelle $\mathcal{D}\rho$ parcourt l’espace des densités
%$\rho(\theta)$ continues, \emph{chaque configuration étant pondérée par le
%facteur exponentiel}
%\(
%e^{\,L\bigl[\mathcal{S}_{YY}[\rho]-\mathcal{W}[\rho]\bigr]},
%\)
%qui combine
%\begin{itemize}
%\item l’\textbf{entropie de Yang–Yang}
%      $\displaystyle
%        \mathcal{S}_{YY}[\rho]
%        =\!\int d\theta\,
%          \bigl[
%            \rho_s\ln\rho_s
%            -\rho\ln\rho
%            -(\rho_s-\rho)\ln(\rho_s-\rho)
%          \bigr]$,
%\item le \textbf{coût énergétique généralisé}
%      $\displaystyle
%        \mathcal{W}[\rho]
%        =\!\int d\theta\, w(\theta)\,\rho(\theta)$,
%\end{itemize}
%où $w(\theta)$ est le \emph{poids spectral} fixé par le GGE.

%------------------------------------------------------------------
%\paragraph{Moyenne d’une observable dans le GGE.}

%On obtient alors la formule de champ moyen
%\begin{equation}\label{eq:GGE-functional-average}
%\bigl\langle\mathcal{O}\bigr\rangle_{\!{\rm GGE}}
%=
%\frac{\displaystyle
%      \int \mathcal{D}\rho\;
%      e^{L\bigl[\mathcal{S}_{YY}[\rho]-\mathcal{W}[\rho]\bigr]}\,
%      \langle\mathcal{O}\rangle_{[\rho]}}
%     {\displaystyle
%      \int \mathcal{D}\rho\;
%      e^{L\bigl[\mathcal{S}_{YY}[\rho]-\mathcal{W}[\rho]\bigr]}}.
%\end{equation}

%------------------------------------------------------------------
\paragraph{Interprétation thermodynamique.}

\begin{itemize}[label = $\bullet$] 
\item $\mathcal{S}_{YY}[\rho]$ \emph{compte} le logarithme du nombre de
      micro-états réalisant la distribution $\rho(\theta)$ :
      c’est l’\textbf{entropie combinatoire}.
\item $\mathcal{W}[\rho]$ mesure le \emph{coût énergétique généralisé}
      associé à cette distribution, dicté par le poids spectral $w(\theta)$.
\end{itemize}

Leur différence
\[
(\mathcal{S}_{YY}-\mathcal{W})[\rho]
\]
joue donc le rôle d’une \emph{fonction thermodynamique effective}
(analogue à une entropie libre).  
L’exposant $e^{L(\mathcal{S}_{YY}-\mathcal{W})[\rho]}$ fixe la \textbf{probabilité relative} d’un
macro-état $\rho(\theta)$ dans le GGE : le terme entropique favorise la
multiplicité des états, tandis que le terme énergétique pénalise les
configurations coûteuses — d’où la compétition caractéristique de
l’équilibre statistique.





%avec $\mathcal{O}[\rho]$ la valeur de l’observable dans un état propre caractérisé par la distribution de rapidité $\rho$.	
%où $\mathcal{O}[\rho]$ est la valeur de l’observable dans un état propre caractérisé par la distribution $\rho$.

%\input{preamble}

\begin{document}

\frontmatter
%\input{chapters/00_intro}
\tableofcontents
\mainmatter

\input{chapters/01_LL_BA}
\input{chapters/02_GGE_TBA}
\input{chapters/03_GHD}
%\input{chapters/97_GHD}
\input{chapters/04_GGE_Fluctuation}
\input{chapters/05_Disp_Exp}
\input{chapters/06_Bipart}
\input{chapters/07_Dipolaire}

%\input{chapters/08_conclusion}
%\appendix
%\input{chapters/99_annexes}

\bibliographystyle{abbrv}
\bibliography{thesis}

%\printbibliography

\end{document}

%| Style     | Description                                                             |
%| --------- | ----------------------------------------------------------------------- |
%| `plain`   | Tri alphabétique, numérotation croissante                               |
%| `unsrt`   | Même que `plain` mais sans tri, respecte l’ordre d’apparition           |
%| `abbrv`   | Comme `plain` mais avec prénoms et noms abrégés                         |
%| `alpha`   | Les références sont étiquetées par une combinaison du nom et de l’année |
%| `apalike` | Style APA simplifié                                                     |
%| `ieeetr`  | Style IEEE, tri par ordre d’apparition                                  |
%| `siam`    | Style SIAM (mathématiques appliquées)                                   |
%| `acm`     | Style ACM (informatique)                                                |
%



\subsection{Équations intégrales de la TBA}

\paragraph{Moyenne des observables dans l’ensemble généralisé de Gibbs.}

\paragraph{Approximation au point selle («\,méthode de la selle statique\,»)}

Dans la limite thermodynamique \( L \to \infty \), cette intégrale est dominée par la configuration \( \rho_{eq} \) qui maximise le poids exponentiel $e^{L(\mathcal{S}_{YY}-\mathcal{W})[\rho]}$  dans l'expression \eqref{chap:TBA:eq:ensemble_average}. Il s’agit de la densité de rapidité la plus probable, solution d’un problème de maximisation. On obtient à l’ordre principal
\begin{eqnarray}
	\underset{\mbox{\tiny therm.}}{\lim} \langle \operator{\mathcal{O}} \rangle_{\operator{\varrho}[w]} & \approx &  \langle\operator{\mathcal{O}}\rangle_{[\rho_{eq} ]},	
	\label{chap:TBA:eq:ensemble_average:approx}
\end{eqnarray}
où $\rho_{eq}$ est la distribution de rapidité à l'équilibre \eqref{chap.2:eq.rho.eq.1}.
Cette approximation correspond à une méthode de \textit{selle statique}, où l’on développe la \emph{fonction thermodynamique effective}, $\mathcal{S}_{YY}-\mathcal{W}$  au voisinage de la distribution dominante.


\paragraph{Développement fonctionnel au premier ordre.}

%On effectue un développement de Taylor fonctionnel de l'action à l’ordre linéaire en $\rho = \rho_{eq} + \delta \rho$ :
Écrivons
\(
\rho=\rho_{\text{eq}}+\delta\rho
\)
et développons $(\mathcal{S}_{YY}-\mathcal{W})[\rho]$ à l’ordre linéaire :
\begin{eqnarray*}
	\mathcal{S}_{YY}[\rho] - \mathcal{W}[\rho] & \approx & \mathcal{S}_{YY}[ \rho_{eq}] - \mathcal{W}[ \rho_{eq}] +  \left. \frac{\delta (\mathcal{S}_{YY}[\rho] - \mathcal{W}[\rho]) }{\delta \rho} \right|_{\rho = \rho_{eq} }	(\delta \rho) + \mathcal{O}(\delta \rho^2 ) ,
	\label{chap:TBA:eq:action}	
\end{eqnarray*}	
La condition de stationnarité au point selle impose :
\(
	\left. \frac{\delta (\mathcal{S}_{YY}[\rho] - \mathcal{W}[\rho]) }{\delta \rho} \right|_{\rho = \rho_{eq} }	  =  0  	
\)
soit 
\begin{equation}
\left. \frac{\delta \mathcal{S}_{YY}}{\delta \rho} \right|_{\rho = \rho_{eq}} = \left. \frac{\delta \mathcal{W}}{\delta \rho} \right|_{\rho = \rho_{eq}}. \label{chap:TBA:eq:stationnarite}
\end{equation}

%%%%%%%%%%%%%%%%
%-----------------------------------------------------

%------------------------------------------------------------------
%\subsection{Équations intégrales de la TBA}

%\paragraph{Moyenne des observables dans le Generalized Gibbs Ensemble.}

%Dans la limite thermodynamique, la moyenne d’une observable locale
%s’écrit formellement comme une intégrale fonctionnelle sur les densités de
%rapidité\,\footnote{%
%La mesure fonctionnelle $\mathcal{D}\rho$ est la limite continue de la
%somme discrète sur les macro-états admissibles, chacun étant pondéré par
%le facteur combinatoire $e^{L\mathcal{S}_{YY}[\rho]}$.}
%
%\begin{equation}\label{eq:TBA:ensemble_average}
%\left\langle \mathcal{O} \right\rangle_{\!\text{GGE}}
%=\frac{\displaystyle
%      \int\!\mathcal{D}\rho\;
%      e^{L\bigl[\mathcal{S}_{YY}[\rho]-\mathcal{W}[\rho]\bigr]}\;
%      \langle\mathcal{O}\rangle_{[\rho]}}
%     {\displaystyle
%      \int\!\mathcal{D}\rho\;
%      e^{L\bigl[\mathcal{S}_{YY}[\rho]-\mathcal{W}[\rho]\bigr]}} .
%\end{equation}

%------------------------------------------------------------------
%\paragraph{Approximation au point selle («\,méthode de la selle statique\,»).}

%Lorsque $L\to\infty$, les intégrales \eqref{eq:TBA:ensemble_average}
%sont dominées par la distribution
%$\rho_{\text{eq}}$ qui \emph{maximise} l’exposant
%\(
%\Phi[\rho]=\mathcal{S}_{YY}[\rho]-\mathcal{W}[\rho].
%\)
%On obtient à l’ordre principal
%\begin{equation}
%\left\langle \mathcal{O} \right\rangle_{\!\text{GGE}}
%\;\simeq\;
%\langle \mathcal{O} \rangle_{[\rho_{\text{eq}}]} .
%\label{eq:TBA:saddle_average}
%\end{equation}

%------------------------------------------------------------------
%\paragraph{Condition de stationnarité et équation variationnelle.}

%Écrivons
%\(
%\rho=\rho_{\text{eq}}+\delta\rho
%\)
%et développons $\Phi[\rho]$ à l’ordre linéaire :
%\[
%\Phi[\rho]\;=\;
%\Phi[\rho_{\text{eq}}]
%+
%\int d\theta\,
%\left.
%\frac{\delta\Phi}{\delta\rho(\theta)}
%\right|_{\rho_{\text{eq}}}
%\delta\rho(\theta)
%+O(\delta\rho^{2}).
%\]
%La stationnarité impose
%\(
%\dfrac{\delta\Phi}{\delta\rho(\theta)}\bigl|_{\rho_{\text{eq}}}=0,
%\)
%soit
%\begin{equation}
%\left.
%\frac{\delta\mathcal{S}_{YY}}{\delta\rho(\theta)}
%\right|_{\rho_{\text{eq}}}
%=
%\left.
%\frac{\delta\mathcal{W}}{\delta\rho(\theta)}
%\right|_{\rho_{\text{eq}}}.
%\label{eq:TBA:variational_condition}
%\end{equation}

%------------------------------------------------------------------
%\paragraph{Forme explicite : introduction de la pseudo-énergie.}

%Pour le modèle de Lieb–Liniger (et, plus généralement, pour un modèle
%intégrable à noyau $\Delta$), on introduit la \emph{pseudo-énergie}
%\[
%\varepsilon(\theta)
%\;=\;
%w(\theta)
%\;+\;\Bigl[\Delta\star\ln\!\bigl(1+e^{-\varepsilon}\bigr)\Bigr](\theta),
%\]
%obtenue en réécrivant \eqref{eq:TBA:variational_condition}.
%Le \emph{facteur d’occupation}
%\(
%\nu(\theta)=\rho(\theta)/\rho_s(\theta)
%\)
%se donne alors par la statistique de type Fermi-Dirac
%\[
%\nu(\theta)=\frac1{1+e^{\varepsilon(\theta)}}.
%\]

%Les équations intégrales complètes de la \textbf{Thermodynamique de Bethe}
%(TBA) sont donc
%\begin{align}
%2\pi\rho_s(\theta) &= 1 + \bigl[\Delta \star \rho\bigr](\theta),
%\label{eq:TBA:rho_s}\\[4pt]
%\rho(\theta) &= \frac{\rho_s(\theta)}{1+e^{\varepsilon(\theta)}},
%\qquad
%\varepsilon(\theta)=w(\theta)+\bigl[\Delta\star\ln(1+e^{-\varepsilon})\bigr](\theta).
%\label{eq:TBA:epsilon}
%\end{align}
%Elles déterminent sans ambiguïté la distribution d’équilibre
%$\rho_{\text{eq}}(\theta)$ en fonction du poids spectral $w(\theta)$.

%\medskip
%Ainsi, la méthode du point selle relie le \emph{poids spectral}
%(caractéristique du GGE) à la distribution de rapidité la plus probable,
%et permet d’évaluer les observables par la formule
%\label{chap:TBA:eq:ensemble_average:approx}.


%-----------------------------------------------------
%%%%%%%%%%%%%%%%

%\paragraph{Équation intégrale de la TBA.}

%Cette égalité donne naissance à une équation intégrale pour le poids spectral \( w \), défini comme la dérivée fonctionnelle de l'énergie généralisée pris en $\rho_{eq}$ :
%\(
%w ~=~ \left. \frac{\delta \mathcal{W}[\rho]}{\delta \rho} \right|_{\rho =  \rho_{eq} }
%\)
%qui par stationnarité (cf équation \eqref{chap:TBA:eq:stationnarite}) est égale à la dérivée fonctionnelle de l'entropie de Yang-Yang pris en $\rho_{eq}$ :
%\(
%\left. \frac{\delta \mathcal{S}_{YY}[\rho]}{\delta \rho} \right|_{\rho = \rho_{eq} }
%\) 
%qui lui vaux 
%\(
%\ln ( \nu_{eq}^{-1}  - 1 ) - \frac{\Delta}{2\pi} \star \ln ( 1 -  \nu_{eq })
%\)
%avec le facteur d'ocupation à l'équilibre $\nu_{eq} = \rho_{eq}/{\rho_{eq}}_s$. Ainci on peux s'arreter sur l'équation 
%\begin{eqnarray}
%	w & = & \ln ( \nu_{eq}^{-1}  - 1 ) - \frac{\Delta}{2\pi} \star \ln ( 1 -  \nu_{eq }).\label{chap:TBA:eq:w}
%\end{eqnarray}

%\medskip
%Ainsi, la méthode du point selle relie le \emph{poids spectral}
%(caractéristique du GGE) à la distribution de rapidité la plus probable,
%et permet d’évaluer les observables par la formule
%\eqref{chap:TBA:eq:ensemble_average:approx}.\\

%\paragraph{Forme explicite : introduction de la pseudo-énergie.}

%Le \emph{facteur d’occupation}
%\(
%\nu_{eq}
%\)
%se donne alors par la statistique de type Fermi-Dirac
%\begin{eqnarray}
%	\nu_{eq}=\frac1{1+e^{\epsilon}},\label{chap:TBA:eq:nu_eq}
%\end{eqnarray}
%où \emph{pseudo-énergie} 
%\(
%\epsilon
%\)
%se définie en intectant \eqref{chap:TBA:eq:nu_eq} dans \eqref{chap:TBA:eq:w} : 
%\begin{eqnarray}
%	\epsilon & = & w + \frac{\Delta}{2\pi} \star \ln ( 1  + e^{-\epsilon}).\label{chap:TBA:eq:e}	
%\end{eqnarray}


%---------------------------------
%------------------------------------------------------------------
\paragraph{Équation intégrale de la TBA.}

La condition de stationnarité au point selle \(\rho=\rho_{\mathrm{eq}}\) \eqref{chap:TBA:eq:stationnarite} implique :
\begin{eqnarray}
	\left.\frac{\delta\mathcal{S}_{YY}}{\delta\rho(\theta)}\right|_{\rho_{\mathrm{eq}}} = \left.\frac{\delta\mathcal{W}}{\delta\rho(\theta)}\right|_{\rho_{\mathrm{eq}}}\;\doteq\;w(\theta),
\end{eqnarray}
En utilisant l’expression explicite de l’entropie de Yang–Yang \eqref{chap.2.entropi.int}, on obtient l’identité fonctionnelle
\begin{eqnarray}
	w & = & \ln ( \nu_{\!eq}^{-1}  - 1 ) - \frac{\Delta}{2\pi} \star \ln ( 1 -  \nu_{\!eq}).\label{chap:TBA:eq:w}
\end{eqnarray}
où
\(
\nu_{\!eq}=\rho_{\!eq}/\rho_{s,\!eq}
\)
est le \textbf{facteur d’occupation} à l’équilibre.
%------------------------------------------------------------------
\paragraph{Forme pseudo-énergie.}
La \textbf{pseudo-énergie} $\epsilon$ se donne alors par la statistique de type Fermi-Dirac
\begin{eqnarray}
	\epsilon =\ln(\nu^{-1}_{\!eq}-1),\qquad\nu_{\!eq}=\frac{1}{1+e^{\epsilon}}.\label{chap:TBA:eq:nu}%\tag{\text{TBA--$\nu$}} 
\end{eqnarray}
En réinjectant \eqref{chap:TBA:eq:nu} dans \eqref{chap:TBA:eq:w} on obtient
l’équation intégrale canonique de la thermodynamique de Bethe :
\begin{eqnarray}
	\epsilon & = & w - \frac{\Delta}{2\pi} \star \ln ( 1  + e^{-\epsilon}).\label{chap:TBA:eq:e}%\tag{\text{TBA–-$\varepsilon$}}	
\end{eqnarray}
%\[
%\boxed{\;
%\varepsilon(\theta)
%=
%w(\theta)
%+\frac{\Delta}{2\pi}\star\ln\!\bigl[1+e^{-\varepsilon(\theta)}\bigr]
%\;}
%\tag{TBA–$\varepsilon$}\label{eq:TBA:eq:e}
%\]

Les relations \eqref{chap:TBA:eq:nu}–\eqref{chap:TBA:eq:e} déterminent de façon univoque la distribution de rapidité d’équilibre \(\rho_{\!eq}\) à partir du poids spectral \(w\), caractéristique du GGE.

\medskip
Ainsi, la méthode du point selle relie \emph{explicitement} le {\em poids spectral}, $w$  (caractéristique du GGE) au \emph{macro-état le plus probable}, $\rho_{eq}$ , et permet d’évaluer les observables par la formule d’ensemble \eqref{chap:TBA:eq:ensemble_average:approx}.


\paragraph{Résolution numérique de l’équation TBA.}\label{para-algho-TBA}

Prenons un poids spectrale quelconque, par exemple : 
\begin{equation}
  w(\theta)= \theta^2 .\label{eq:TBA:w:quadra}
\end{equation} 
En injectant $w$ dans l’équation intégrale pour lapseudo-énergie \eqref{chap:TBA:eq:e}, on obtient l’équation non linéaire.
Cette équation définit un opérateur contractant sur l’espace des fonctions
\( \epsilon(\theta) \) ; son Jacobien a une norme strictement
inférieure à 1, garantissant la convergence de l’itération de Picard.

\medskip
\subparagraph{Algorithme d’itération.}  
La structure contractante de l’équation garantit l’absence de cycles ou de points fixes multiples, assurant la convergence de l’itération vers l’unique solution admissible.
L’équation \eqref{eq:num:TBA} est non linéaire ; pour la résoudre numériquement, on utilise une méthode itérative de type Picard. On initialise
\(
  \epsilon_0 = w ,
\)
puis on construit une suite de fonctions \(\varepsilon_n\) définie par
\begin{eqnarray*}
	\epsilon_{n+1} & = & \epsilon_0 -   \frac{\Delta}{2\pi} \star \ln \left( 1 + e^{-\epsilon_n} \right) ,\quad n\ge0
\end{eqnarray*}
L’itération est poursuivie jusqu’à convergence, que l’on peut tester via le critère numérique
\(
  \beta \left\| \varepsilon_{n+1} - \varepsilon_n \right\|_\infty < 10^{-12},
\)
où \(\|\cdot\|_\infty\) désigne la norme \(L^\infty\) (ou un maximum discret après discrétisation).


\medskip
\subparagraph{Facteur d’occupation et densités.}  
Une fois la pseudo-énergie \( \epsilon(\theta) \) convergée, le facteur d’occupation  à l'équilibre est obtenu en injectant $\epsilon$ dans l’équation \eqref{chap:TBA:eq:nu}, ce qui donne  $\nu_{\!eq}$.
 
On en déduit ensuite la densité d'état à l'équilibre $\rho_{s,eq}$ via le {\bf dressing}  de la fonction constante $f(\theta) = 1$, selon \eqref{eq:TBA-rhos-2}, rappelée ici pour mémoire : $ 2\pi \rho_{s,eq}  =  1^{\mathrm{dr}}_{[\nu_{\! eq}]}$.\\

L’opérateur de dressing \eqref{eq:dressing} étant linéaire, il se résout numériquement sous la forme :
\begin{eqnarray*}
	\left\{ \mathrm{id} - \frac{\Delta}{2\pi} \star ( \nu \ast \cdot ) \right\} f^{\mathrm{dr}}_{[\nu]} & = & f,\label{eq:TBA:rho_s:num}
\end{eqnarray*}
où $\mathrm{id} \colon f \mapsto f$ est l’identité fonctionnelle, et $\ast$ désigne la multiplication.
Après discrétisation de la variable $\theta$, cette équation devient un système linéaire de type $Ax=b$ , facilement résoluble numériquement.

La distribution de rapidité est alors obtenue par $\rho_{\!\mathrm{eq}} = \nu_{\!\mathrm{eq}} \ast \rho_{\! s,\mathrm{eq}}$.\\

\medskip
Ainsi en fixant le poids spectral $w(\theta)$, l’algorithme fournit la pseudo-énergie \( \epsilon \), le facteur d’occupation \( \nu_{\mathrm{eq}} \) et la distribution de rapidité \( \rho_{\!\mathrm{eq}} \).

\medskip
\subparagraph{À l'équilibre thermique.} 
Si on se place à l’équilibre canonique, caractérisé par la température \( T \) et le potentiel chimique \( \mu \).  Dans ce cadre, le poids spectral vaut
\begin{equation}
  w(\theta)=\beta\bigl[\varepsilon(\theta)-\mu\bigr],\qquad\beta=\tfrac1T\; (k_B = 1 ),\quad\varepsilon(\theta)=\tfrac{\theta^{2}}{2}\;(m=1).\label{eq:TBA:w:canonical}
\end{equation}
%En injectant \eqref{eq:TBA:w:canonical} dans l’équation intégrale pour lapseudo-énergie \eqref{chap:TBA:eq:e}, on obtient l’équation non linéaire :
%\begin{eqnarray*}
%	\epsilon & = & \beta(\varepsilon - \mu)  -  \frac{\Delta}{2\pi} \star \ln \left( 1 + e^{-\epsilon} \right) ,\label{eq:num:TBA}
%\end{eqnarray*}
%Ainsi, pour tout couple \((T,\mu)\), l’algorithme fournit la pseudo-énergie \( \epsilon \), le facteur d’occupation \( \nu_{\mathrm{eq}} \) et la distribution de rapidité \( \rho_{\mathrm{eq}} \) à l’équilibre thermique, prêts à être utilisés pour le calcul des observables.
%
%\medskip
%Pour $w$ quelconque , l'algorythme est identique.




		


\chapter{Dynamique hors-équilibre et hydrodynamique généralisée}
\label{chap:GHD}
\minitoc

%\chapter{Hydrodynamique généralisée (GHD)}

\section*{Introduction}


\paragraph{De l’état stationnaire à la dynamique}  
Après avoir étudié les propriétés stationnaires des gaz de bosons unidimensionnels, nous nous tournons désormais vers leur évolution temporelle. Ce chapitre s’appuie sur une approche hydrodynamique adaptée aux systèmes intégrables : la théorie dite d’Hydrodynamique Généralisée (GHD). Celle-ci est largement documentée dans la littérature (voir par exemple [50, 24, 51, 52]) et nous en présentons ici les concepts essentiels.

\paragraph{Principe général d’une approche hydrodynamique}  
De manière générale, l’hydrodynamique vise à décrire la dynamique à grande échelle (\emph{coarse grained dynamics}) d’un système, également appelée « échelle d’Euler ». L’idée consiste à découper l’espace-temps d’un système de taille $L$ en cellules de dimensions $\ell \times \tau$, comme illustré en Fig.~???.  
La longueur $\ell$ est choisie de sorte que $L \gg \ell \gg \ell_c$, où $\ell_c$ désigne une longueur microscopique caractéristique, par exemple la distance inter-particule. On peut alors considérer que la densité est uniforme à l’intérieur de chaque cellule, ce qui correspond à l’Approximation de Densité Locale.

\paragraph{Choix des échelles spatio-temporelles}  
Le temps $\tau$ est fixé pour être beaucoup plus grand que le temps caractéristique de relaxation. Ainsi, chaque cellule de l’espace-temps est supposée décrire un état localement relaxé. La notion de relaxation occupe donc une place centrale dans la construction des approches hydrodynamiques.

\paragraph{Particularités pour les systèmes quantiques isolés}  
Dans le cadre de systèmes quantiques isolés, la relaxation n’est pas un concept trivial, qu’il s’agisse de systèmes chaotiques ou intégrables. La section suivante s’attache à définir plus précisément cette notion, avant de présenter les approches hydrodynamiques adaptées à chaque cas. Pour les systèmes intégrables, une attention particulière est portée à la formulation et aux implications de l’Hydrodynamique Généralisée.

\paragraph{Équations hydrodynamiques de type Euler}  
Les équations hydrodynamiques de type Euler sont des équations hyperboliques qui décrivent la dynamique émergente des systèmes à plusieurs corps à grandes échelles d’espace et de temps~\cite{ref1}. Elles rendent compte de la propagation de la relaxation locale, c’est-à-dire la séparation entre une dynamique lente, émergente, et la projection rapide des observables locales sur les quantités conservées. En une dimension d’espace, elles prennent la forme locale de conservation
\begin{equation}\label{chap:GHD:eq.conserv.1}
	\partial_t q_i + \partial_x j_i = F_i,	
\end{equation}
où l’indice $i$ énumère les lois de conservation locales admises, et où $F_i$ représente les contributions provenant de champs de force externes, qui rompent en général la conservation stricte.

\paragraph{Relations constitutives et exemples}  
Les flux $j_i$ et les termes de force $F_i$ dépendent uniquement des densités conservées $q_i$ (équations d’état), et sont déterminés à partir de considérations thermodynamiques, telles que la maximisation de l’entropie. Les équations d’Euler pour un fluide galiléen, ou encore l’hydrodynamique relativiste, constituent des exemples classiques de ce type d’équations.

\paragraph{Cas intégrable et hydrodynamique généralisée}  
En dimension un, de nombreux systèmes à plusieurs corps présentent une propriété d’intégrabilité~\cite{ref2,ref3}. Dans ce contexte, il existe une infinité de lois de conservation, et la théorie universelle qui décrit leur hydrodynamique à l’échelle d’Euler est l’Hydrodynamique Généralisée (GHD)~\cite{ref4,ref5}. Cette approche englobe les équations connues pour les bâtons durs~\cite{ref1,ref6} et les gaz de solitons~\cite{ref7,ref8,ref9}, tout en s’appliquant plus largement, aussi bien à des systèmes classiques que quantiques : particules en interaction, chaînes de spins ou théories des champs quantiques (voir~\cite{ref10} pour des revues).

\paragraph{Paramétrisation spectrale et densité conservée}  
La GHD reformule l’infinité de lois de conservation (éventuellement rompues) en une famille indexée par un paramètre spectral continu $\theta$, plutôt que par un indice discret $i$. On note $\rho(x,\theta,t)$ la densité conservée en espace réel, espace spectral et temps. Le paramètre spectral énumère les objets asymptotiques issus de la théorie de diffusion correspondante (particules, solitons, etc.), incluant leur quantité de mouvement et leurs éventuels degrés internes. Dans de nombreux cas simples, $\theta$ appartient à un sous-ensemble de $\mathbb{R}$, représentant les moments asymptotiques, et les coordonnées $(x,\theta)$ forment un « espace des phases spectral » sur lequel $\rho$ joue le rôle de densité.

\paragraph{Prise en compte des champs de force}  
L’inclusion de champs de force externes couplés aux densités conservées a été introduite dans~\cite{ref11}, où il est montré que la GHD s’écrit
\begin{equation}\label{chap:GHD:eq.GHD.1}
	\partial_t \rho + \partial_x(v^{\text{eff}} \rho) + \partial_\theta(a^{\text{eff}} \rho) = 0.
\end{equation}
Ici, $v^{\text{eff}}$ et $a^{\text{eff}}$ sont des fonctionnels appropriés de $\rho(x,\cdot,t)$, et le dernier terme représente la contribution des champs de force. D’autres types de forces ont été étudiés~\cite{Bastianello2019a,Bastianello2019b}, mais ne seront pas considérés ici.

%\section{Formulation hamiltonienne de la GHD}
%
%\subsection{Crochet de Poisson fonctionnel}
%
%\paragraph{Définition générale}
%Bonnemain \emph{et al.}~\cite{bonnemain2024hamiltonian} définissent un crochet de Poisson fonctionnel agissant sur les fonctionnelles $F$ et $G$ de la distribution de rapidité, avec interactions :
%\begin{equation}\label{chap:GHD:eq.chochet.bonnemain.1}
%	\{F,G\}=\iint dx\,d\theta\;\frac{\nu}{2\pi}\,\left[\partial_x \left ( \frac{\delta F}{\delta \rho(x,\theta)} \right )\,\left(\partial_\theta \left ( \frac{\delta G}{\delta \rho(x,\theta)} \right ) \right)^{\mathrm{dr}}_{[\nu]} -\partial_x \left ( \frac{\delta G}{\delta \rho (x,\theta)} \right ) \,\left( \partial_\theta \left ( \frac{\delta F}{\delta \rho (x,\theta)} \right )\right)^{\mathrm{dr}}_{[\nu]} \right],
%\end{equation}
%où $\nu$ est la fonction d’occupation. L'application de l’opérateur de \emph{dressing} dans ce crochet traduit les interactions entre particules.
%
%\paragraph{Cas des charges globales}
%Les charges locales conservées ont été définies en \eqref{chap.2.charge.f.1}.  
%Avec le même formalisme, les charges globales conservées se définissent comme fonctionnelles linéaires d’une fonction réelle et régulière \( f(x, \theta) \) définie sur \( \mathbb{R}^2 \) :
%\begin{equation}\label{chap:GHD:eq.charge.global.1}
%	\mathcal{Q}[f] = \int_{\mathbb{R}^2} dx\, d\theta\, f(x, \theta)\, \rho(x, \theta),
%\end{equation}
%qui représente la charge totale associée à une quantité prenant la valeur \( f(x, \theta) \) pour chaque quasi-particule.
%
%Dans notre étude de la dynamique, nous n’avons pas besoin de l’information sur le poids spectral.  
%On notera donc, dans la limite thermodynamique, les moyennes d’opérateurs simplement en retirant leur chapeau :
%\[
%\underset{\mathrm{therm}}{\lim} \braket{\mathcal{O}}_{\varrho[w]} \equiv \mathcal{O}.
%\]
%Ainsi, dans cette limite, la charge globale \eqref{chap:GHD:eq.charge.global.1} s’écrit directement comme ci-dessus.
%
%Le crochet de Poisson \eqref{chap:GHD:eq.chochet.bonnemain.1} appliqué à deux charges globales \( \mathcal{Q}[f] \) et \( \mathcal{Q}[g]\) s’écrit :
%\begin{equation}\label{chap:GHD:eq.chochet.bonnemain.2}
%	\{\mathcal{Q}[f], \mathcal{Q}[g]\} = \int_{\mathbb{R}^2} dx\, d\theta \frac{\nu}{2\pi}  \left( \partial_x f  (\partial_\theta g )^{\mathrm{dr}}_{[\nu]}  - \partial_x g (\partial_\theta f)^{\mathrm{dr}}_{[\nu]}  \right).
%\end{equation}
%L’application du dressing satisfait la symétrie~\cite{doyon2020lecture} :
%\begin{equation}\label{chap:GHD:eq.sym.dr.1}
%	\int_{\mathbb{R}^2}	 dx\, d\theta \, \nu f g^{\mathrm{dr}}_{[\nu]} = \int_{\mathbb{R}^2}	 dx\, d\theta \, \nu f^{\mathrm{dr}}_{[\nu]} g.
%\end{equation}
%Par intégration par parties, le crochet \eqref{chap:GHD:eq.chochet.bonnemain.2} devient :
%\begin{equation}\label{chap:GHD:eq.chochet.bonnemain.3}
%	\{ \mathcal{Q}[f] , \mathcal{Q}[g]\} = \int_{\mathbb{R}^2} dx\, d\theta \,   f  \left( \partial_\theta \left ( \frac{\nu }{2\pi}  (\partial_x g )^{\mathrm{dr}}_{[\nu]} \right )   - \partial_x  \left ( \frac{\nu}{2\pi}  (\partial_\theta g )^{\mathrm{dr}}_{[\nu]} \right )  \right).
%\end{equation}
%
%\subsection{Crochet avec l’Hamiltonien}
%
%\paragraph{Densité hamiltonienne et grandeurs effectives}
%On note $h(x,\theta)$ la densité associée à la moyenne de l’Hamiltonien :
%\begin{equation}\label{chap:GHD:eq.ham.1}
%	H = \mathcal{Q}[h].
%\end{equation}
%La fonction d’occupation $\nu$, la vitesse effective $v^{\mathrm{eff}}$ et l’accélération effective $a^{\mathrm{eff}}$ sont définies par :
%\begin{equation}\label{chap:GHD:eq.nu.v.a.1}
%	\nu = 2\pi \frac{\rho}{1^{\mathrm{dr}}_{[\nu]}}, \quad  
%	v^{\mathrm{eff}} = \frac{(\partial_\theta h )^{\mathrm{dr}}_{[\nu]}}{1^{\mathrm{dr}}_{[\nu]}}, \quad  
%	a^{\mathrm{eff}} = -\frac{(\partial_x h )^{\mathrm{dr}}_{[\nu]}}{1^{\mathrm{dr}}_{[\nu]}},
%\end{equation}
%fonctions de $\rho(x,\theta,t)$.
%
%Le crochet \eqref{chap:GHD:eq.chochet.bonnemain.3} appliqué à $(f,h)$ devient :
%\begin{equation}\label{chap:GHD:eq.chochet.bonnemain.4}
%	\{\mathcal{Q}[f] , \mathcal{Q}[h]\} = -\int_{\mathbb{R}^2} dx\, d\theta \,   f  \left[ \partial_x \left ( \rho  v^{\mathrm{eff}} \right )   +  \partial_\theta   \left ( \rho  a^{\mathrm{eff}} \right )  \right].
%\end{equation}
%
%\paragraph{Forme locale : densités conservées}
%En choisissant $f(x,\theta) \mapsto \delta(\cdot - x)f(\theta)$ dans \eqref{chap:GHD:eq.charge.global.1}, on obtient la densité conservée :
%\[
%q_{[f]}(x) = \mathcal{Q}[(x,\theta) \mapsto \delta(\cdot - x) f(\theta)].
%\]
%Appliquée à \eqref{chap:GHD:eq.chochet.bonnemain.4}, cette prescription donne :
%\begin{equation}\label{chap:GHD:eq.chochet.bonnemain.5}
%	\{ q_{[f]}(x) , \mathcal{Q}[h]\} = - \partial_x \left ( \int_{\mathbb{R}} d\theta \,   f  \,  \rho  \,  v^{\mathrm{eff}} \right ) + \int_{\mathbb{R}} d\theta \, f' \,    \rho \, a^{\mathrm{eff}}.
%\end{equation}
%En utilisant l’équation de Liouville \eqref{chap:GHD:eq.Liouv.1}, on retrouve la forme de convection :
%\begin{equation}\label{chap:GHD:eq.conserv.2}
%	\partial_t q_{[f]} + \partial_x j_{[f]} = F_{[f]},
%\end{equation}
%avec
%\begin{equation}\label{chap:GHD:eq.conserv.2.1}
%	j_{[f]} = \int_{\mathbb{R}} d\theta \,v^{\mathrm{eff}} \, f \, \rho, 
%	\quad F_{[f]} = \int_{\mathbb{R}} d\theta \,  a^{\mathrm{eff}} \, f' \, \rho.
%\end{equation}
%
%\paragraph{Forme locale : équation sur \texorpdfstring{$\rho$}{rho}}
%En prenant $\rho(x,\theta) = \mathcal{Q}[\delta(\cdot - x)\delta(\cdot - \theta)]$ et en l’appliquant à \eqref{chap:GHD:eq.chochet.bonnemain.4}, on obtient :
%\begin{equation}\label{chap:GHD:eq.chochet.bonnemain.6}
%	\{ \rho ( x , \theta ) , \mathcal{Q}[h]\} = - \partial_x \left (  v^{\mathrm{eff}} \,  \rho   \right ) - \partial_\theta \left (  a^{\mathrm{eff}}  \,  \rho  \right).
%\end{equation}
%En appliquant l’équation de Liouville \eqref{chap:GHD:eq.Liouv.1}, on retrouve l’équation GHD :
%\begin{equation}\label{chap:GHD:eq.conserv.3}
%	\partial_t \rho + \partial_x(v^{\mathrm{eff}} \rho) + \partial_\theta(a^{\mathrm{eff}} \rho) = 0.
%\end{equation}
%
%---------------------
\section{Formulation hamiltonienne de la GHD}

\subsection{Crochet de Poisson fonctionnel}

\paragraph{Interprétation et limite non-interactive}  
À ce niveau de généralité, l'équation de l’Hydrodynamique Généralisée (GHD) \eqref{chap:GHD:eq.GHD.1} peut être interprétée comme la dynamique hydrodynamique d’un fluide bidimensionnel dont la densité est conservée dans l’espace des phases spectral.  
Les effets d’interaction se traduisent par un couplage non local dans la direction des rapidités $\theta$, reflétant les processus de diffusion élastique entre quasi-particules possédant des paramètres spectraux distincts.

\medskip

Dans le cas limite d’un système \emph{sans interactions}, l’espace spectral coïncide avec l’espace des phases classique, et l’équation de GHD se réduit alors à l’équation de Liouville (ou, de façon équivalente, à l’équation de Boltzmann sans terme de collisions) issue de la théorie cinétique élémentaire.

\medskip

En l’absence de phénomènes dissipatifs, la densité de distribution $\rho$ est conservée le long du flot hamiltonien associé à l’énergie $H$, ce qui s’exprime par
\begin{equation}\label{chap:GHD:eq.Liouv.1}
	\frac{d \rho}{dt} 
	= \frac{\partial \rho}{\partial t } + \{ \rho , H \} = 0,
\end{equation}
où $\{\cdot , \cdot\}$ désigne le crochet de Poisson canonique dans l’espace des phases.  
Dans cette perspective, l’Hydrodynamique Généralisée apparaît comme une extension naturelle de l’équation de Liouville aux systèmes intégrables, incorporant les effets collectifs induits par les interactions tout en préservant une description exacte à grande échelle.


\paragraph{Structure hamiltonienne et crochet de Poisson fonctionnel}  
Bonnemain \emph{et al.} \cite{bonnemain2024hamiltonian} introduisent un crochet de Poisson fonctionnel agissant sur des fonctionnelles $F$ et $G$ de la distribution de rapidité $\rho(x,\theta)$ en présence d’interactions. Celui-ci s’écrit
\begin{equation}\label{chap:GHD:eq.chochet.bonnemain.1}
	\{F,G\}
	=\iint dx\,d\theta\;\frac{\nu}{2\pi}\,
	\left[
		\partial_x \left( \frac{\delta F}{\delta \rho(x,\theta)} \right)
		\left( \partial_\theta \left( \frac{\delta G}{\delta \rho(x,\theta)} \right) \right)^{\mathrm{dr}}_{[\nu]}
		-
		\partial_x \left( \frac{\delta G}{\delta \rho(x,\theta)} \right)
		\left( \partial_\theta \left( \frac{\delta F}{\delta \rho(x,\theta)} \right) \right)^{\mathrm{dr}}_{[\nu]}
	\right],
\end{equation}
où $\nu$ désigne la fonction d’occupation. Dans ce crochet l'application de l’opérateur de \emph{dressing} $(\cdot)^{\mathrm{dr}}_{[\nu]}$ (introduit dans \eqref{eq:dessing})  traduit les interactions entre particules.

%L’opérateur de \emph{dressing} $(\cdot)^{\mathrm{dr}}_{[\nu]}$ agit ici sur les dérivées fonctionnelles dans la variable spectrale $\theta$ ; il encode les effets des interactions à longue portée dans l’espace des rapidités. Cette structure hamiltonienne permet de reformuler la GHD comme une équation de type Liouville sur l’espace fonctionnel des distributions $\rho$, mais avec un crochet de Poisson modifié par le \emph{dressing}, traduisant la nature intégrable et non-locale des interactions.

\medskip

\paragraph{Charges globales conservées}  
Les charges locales conservées ont été définies dans les équations~\eqref{chap.2.charge.f.1}.  
Dans le même formalisme, on définit les \emph{charges globales conservées} comme des fonctionnelles linéaires agissant sur une fonction réelle et régulière $f(x,\theta)$ définie sur $\mathbb{R}^2$, selon
\begin{equation}\label{chap:GHD:eq.charge.global.0}
	\operator{\mathcal{Q}}[f] 
	= \int_{\mathbb{R}^2} dx\, d\theta\, f(x, \theta)\, \operator{\rho}(x, \theta),
\end{equation}
où $\operator{\rho}(x,\theta)$ est l'opérateur distribution de rapidité.  
Cette quantité correspond à la charge totale associée à une observable prenant la valeur $f(x,\theta)$ pour chaque quasi-particule.

\medskip

La valeur moyenne $\langle \operator{\mathcal{Q}}[f] \rangle_{\operator{\varrho}[w]}$ a été définie en~\eqref{chap.TBA.moy.dens}.  
La matrice densité locale $\operator{\varrho}^{(\mathcal{S})}[w]$ a été introduite en~\eqref{chap.2.densite.1}.  
De manière analogue, la \emph{matrice densité globale} $\operator{\varrho}[w]$ s’écrit
\begin{equation}\label{chap:GHD:eq.charge.global.2}
	\operator{\varrho}[w] 
	= \frac{1}{Z[w]}\, e^{-\operator{\mathcal{Q}}[w]}, 
	\qquad  
	Z[w] = \mathrm{Tr} \left[ e^{-\operator{\mathcal{Q}}[w]} \right],
\end{equation}
où la charge globale $\operator{\mathcal{Q}}[w]$ est définie par~\eqref{chap:GHD:eq.charge.global.0}, et $w$ désigne le poids spectral.  

%Cette formulation met en évidence le lien entre la description statistique du système et la conservation des charges globales, en généralisant le principe de Gibbs aux systèmes intégrables par l’introduction de l’ensemble d’observables $\operator{\mathcal{Q}}[f]$ sur l’espace spectral.

\medskip

\paragraph{Crochet de Poisson entre charges globales}  
Dans notre étude de la dynamique, nous n’avons pas besoin de l’information détaillée sur le poids spectral $w$.  
Nous noterons donc, dans ce chapitre, et dans la limite thermodynamique, les moyennes des opérateurs en supprimant leur chapeau, \emph{i.e.}
\begin{equation}
\underset{\mathrm{therm}}{\lim} \, \langle \operator{\mathcal{O}} \rangle_{\varrho[w]} \; \equiv \; \mathcal{O},
\end{equation}
de sorte que, dans cette limite, la moyenne de la charge globale s’écrit
\begin{equation}\label{chap:GHD:eq.charge.global.1}
	\mathcal{Q}[f] 
	= \int_{\mathbb{R}^2} dx\, d\theta\, f(x, \theta)\, \rho(x, \theta),
\end{equation}
où $f$ est une fonction régulière sur $\mathbb{R}^2$.

\medskip

Le crochet de Poisson (défini en~\eqref{chap:GHD:eq.chochet.bonnemain.1}) entre deux charges $\mathcal{Q}[f]$ et $\mathcal{Q}[g]$ prend la forme
\begin{equation}\label{chap:GHD:eq.chochet.bonnemain.2}
	\{\mathcal{Q}[f], \mathcal{Q}[g]\}
	= \int_{\mathbb{R}^2} dx\, d\theta\, \frac{\nu}{2\pi} 
	\left[ \partial_x f \, (\partial_\theta g)^{\mathrm{dr}}_{[\nu]} 
	     - \partial_x g \, (\partial_\theta f)^{\mathrm{dr}}_{[\nu]} \right].
\end{equation}
%où $\nu$ est la fonction d’occupation et $(\cdot)^{\mathrm{dr}}_{[\nu]}$ désigne l’application de \emph{dressing} associée à $\nu$.

\medskip

Cette application de \emph{dressing} satisfait la relation de symétrie~\cite{doyon2020lecture} :
\begin{equation}\label{chap:GHD:eq.sym.dr.1}
	\int_{\mathbb{R}^2} dx\, d\theta \; \nu \, f \, g^{\mathrm{dr}}_{[\nu]} 
	= \int_{\mathbb{R}^2} dx\, d\theta \; \nu \, f^{\mathrm{dr}}_{[\nu]} \, g.
\end{equation}

Pour appliquer la relation de symétrie~\eqref{chap:GHD:eq.sym.dr.1} au crochet~\eqref{chap:GHD:eq.chochet.bonnemain.2}, il est nécessaire de vérifier que les fonctions impliquées satisfont les conditions requises sur leurs types tensoriels.
\footnote{
La relation de symétrie~\eqref{chap:GHD:eq.sym.dr.1} est valable lorsque la somme des types tensoriels de $f$ et $g$ est $(1,1)$ dans le sens de~\cite{doyon2020lecture}. Dans ce formalisme, le type $(a,b)$ caractérise la transformation d'un objet vis-à-vis de $x$ (première entrée) et de $\theta$ (seconde entrée). Si $f$ est de type $(p,q)$ et $g$ de type $(r,s)$, alors leur somme est $(p+r,q+s)$. La condition $(1,1)$ garantit que l'intégrande $\nu\, f\, g^{\mathrm{dr}}$ est un scalaire invariant, rendant l'intégrale bien définie. Dans~\eqref{chap:GHD:eq.chochet.bonnemain.2}, $\partial_x f$ est de type $(1,0)$ et $\partial_\theta g$ de type $(0,1)$, ce qui satisfait cette condition et permet l'utilisation de~\eqref{chap:GHD:eq.sym.dr.1}.
}

En utilisant cette symétrie ainsi qu’une intégration par parties, le crochet~\eqref{chap:GHD:eq.chochet.bonnemain.2} se réécrit
\begin{equation}\label{chap:GHD:eq.chochet.bonnemain.3}
	\{\mathcal{Q}[f], \mathcal{Q}[g]\}
	= \int_{\mathbb{R}^2} dx\, d\theta \; f \,
	\left[
		\partial_\theta \left( \frac{\nu}{2\pi} \, (\partial_x g)^{\mathrm{dr}}_{[\nu]} \right)
		- \partial_x \left( \frac{\nu}{2\pi} \, (\partial_\theta g)^{\mathrm{dr}}_{[\nu]} \right)
	\right].
\end{equation}

\medskip

\subsection{Crochet avec l’Hamiltonien}

\paragraph{Densité hamiltonienne et grandeurs effectives} 
On note $h(x,\theta)$ la densité associée à la moyenne de l’Hamiltonien, telle que
\begin{equation}\label{chap:GHD:eq.ham.1}
	H = \mathcal{Q}[h].
\end{equation}

La fonction d’occupation $\nu$, la vitesse effective $v^{\mathrm{eff}}$ et l’accélération effective $a^{\mathrm{eff}}$ sont définies par
%\begin{equation}\label{chap:GHD:eq.nu.v.a.1}
%	\nu = 2\pi \frac{\rho}{1^{\mathrm{dr}}_{[\nu]}}, 
%	\quad v^{\mathrm{eff}} = \frac{(\partial_\theta h )^{\mathrm{dr}}_{[\nu]}}{1^{\mathrm{dr}}_{[\nu]}}, 
%	\quad a^{\mathrm{eff}} = -\frac{(\partial_x h )^{\mathrm{dr}}_{[\nu]}}{1^{\mathrm{dr}}_{[\nu]}},
%\end{equation}
\begin{equation}\label{chap:GHD:eq.nu.v.a.1}
	2 \pi \rho =  1^{\mathrm{dr}}_{[\nu]} \, \nu , 
	\quad 2 \pi \, v^{\mathrm{eff}} \, \rho  =(\partial_\theta h )^{\mathrm{dr}}_{[\nu]} \, \nu , 
	\quad 2 \pi \, a^{\mathrm{eff}} \, \rho  = -(\partial_x h )^{\mathrm{dr}}_{[\nu]}\, \nu ,
\end{equation}
toutes trois étant des fonctions de $\rho(\cdot,\cdot,t)$. Ces quantités interviennent dans les équations de mouvement
\begin{equation}
	\dot{x} = v^{\mathrm{eff}}, \qquad \dot{\theta} = a^{\mathrm{eff}},
\end{equation}
montrant que les dérivées $\partial_x$ et $\partial_\theta$ présentes dans le crochet de Poisson correspondent respectivement à l'action de l'accélération effective sur $\theta$ et de la vitesse effective sur $x$.

\medskip 

%Avec ces définitions, le crochet~\eqref{chap:GHD:eq.chochet.bonnemain.3} s’écrit
Le crochet \eqref{chap:GHD:eq.chochet.bonnemain.3} appliqué à $(f,h)$ devient :
\begin{equation}\label{chap:GHD:eq.chochet.bonnemain.4}
	\{\mathcal{Q}[f], \mathcal{Q}[h]\} 
	= - \int_{\mathbb{R}^2} dx\, d\theta \; f \left[ \partial_x \big( \rho \, v^{\mathrm{eff}} \big) 
	+ \partial_\theta \big( \rho \, a^{\mathrm{eff}} \big) \right].
\end{equation}

\paragraph{Forme locale : densités conservées .} 
En choisissant $(x,\theta) \mapsto \delta(\cdot - x)f(\theta)$ dans \eqref{chap:GHD:eq.charge.global.1}, on obtient la moyenne de la densité conservée \ie
%On remarque que les moyennes des densités conservées $q_{[f]}(x)$ s’obtiennent en appliquant la prescription
%\[
%(x,\theta) \mapsto \delta(\cdot - x) \, f(\theta)
%\]
%dans~\eqref{chap:GHD:eq.charge.global.1}, \emph{i.e.}
\begin{equation}
	q_{[f]}(x) = \mathcal{Q} \big[ (x,\theta) \mapsto \delta(\cdot - x) \, f(\theta) \big].
\end{equation}

Appliqué à~\eqref{chap:GHD:eq.chochet.bonnemain.4}, on obtient
\begin{equation}\label{chap:GHD:eq.chochet.bonnemain.5}
	\{q_{[f]}(x), \mathcal{Q}[h]\} 
	= - \partial_x \left[ \int_{\mathbb{R}} d\theta \; f \, \rho \, v^{\mathrm{eff}} \right]
	+ \int_{\mathbb{R}} d\theta \; f' \, \rho \, a^{\mathrm{eff}}.
\end{equation}

En appliquant l’équation de Liouville~\eqref{chap:GHD:eq.Liouv.1}, on retrouve la forme de convection~\eqref{chap:GHD:eq.conserv.1} :
\begin{equation}\label{chap:GHD:eq.conserv.2}
	\partial_t q_{[f]} + \partial_x j_{[f]} = F_{[f]},
\end{equation}
où le flux $j_{[f]}$ et le terme de force $F_{[f]}$ sont donnés par
\begin{equation}\label{chap:GHD:eq.conserv.2.1}
	j_{[f]} = \int_{\mathbb{R}} d\theta \; v^{\mathrm{eff}} \, f \, \rho,
	\quad F_{[f]} = \int_{\mathbb{R}} d\theta \; a^{\mathrm{eff}} \, f' \, \rho.
\end{equation}

\paragraph{Forme locale : équation sur \texorpdfstring{$\rho$}{rho}} 
De manière analogue, pour la distribution de rapidité à l’équilibre thermodynamique, on note
\begin{equation}
	\rho(x,\theta) = \mathcal{Q}\big[ \delta(\cdot - x) \, \delta(\cdot - \theta) \big].
\end{equation}
Appliqué à~\eqref{chap:GHD:eq.chochet.bonnemain.4}, on obtient
\begin{equation}\label{chap:GHD:eq.chochet.bonnemain.6}
	\{\rho(x,\theta), \mathcal{Q}[h]\} 
	= - \partial_x \big( v^{\mathrm{eff}} \, \rho \big)
	  - \partial_\theta \big( a^{\mathrm{eff}} \, \rho \big).
\end{equation}

En appliquant l’équation de Liouville~\eqref{chap:GHD:eq.Liouv.1}, on retrouve l’équation GHD~\eqref{chap:GHD:eq.GHD.1} :
\begin{equation}\label{chap:GHD:eq.conserv.3}
	\partial_t \rho + \partial_x \big( v^{\mathrm{eff}} \rho \big)
	+ \partial_\theta \big( a^{\mathrm{eff}} \rho \big) = 0.
\end{equation}

Le résultat remarquable \eqref{chap:GHD:eq.conserv.3} a été obtenu pour la première fois par Bertini et al. (2016) et Castro-Alvaredo et al. (2016). Cette observation clé a déclenché l’ensemble des développements ultérieurs de la dynamique hydrodynamique généralisée (GHD) dans les systèmes quantiques intégrables. Les travaux de Bertini et al. (2016) s’appuient en partie sur ceux de Bonnes et al. (2014), où la formule donnant la vitesse effective \eqref{chap:GHD:eq.nu.v.a.1} était apparue pour la première fois dans le contexte d’un système quantique intégrable.\\

Les équation \eqref{chap:GHD:eq.Liouv.1},  \eqref{chap:GHD:eq.conserv.2} et \eqref{chap:GHD:eq.conserv.3} décrivent la dynamique au régime d’Euler. En dehors de cette approximation, il est nécessaire de prendre en compte des contributions supplémentaires liées aux effets diffusifs \cite{DeNardis2018}.




\section{Cas particuliers et interpretations}


\subsection{Cas sans interaction}

En l’absence d’interaction, l’opérateur de \emph{dressing} se réduit à l’identité.  
Dans ce cas, la fonction d’occupation \eqref{chap:GHD:eq.nu.v.a.1} devient :
\begin{equation}
	\nu = 2\pi \rho,
\end{equation}
et le crochet \eqref{chap:GHD:eq.chochet.bonnemain.1} se simplifie en :
\begin{equation}
	\{F,G\} = \iint dx\,d\theta\;\rho \,\left[\partial_x \!\left( \frac{\delta F}{\delta \rho (x , \theta)} \right)\, \partial_\theta \!\left( \frac{\delta G}{\delta \rho (x , \theta)} \right) - \partial_x \!\left( \frac{\delta G}{\delta \rho (x , \theta)} \right) \, \partial_\theta \!\left( \frac{\delta F}{\delta \rho(x , \theta) } \right) \right].
\end{equation}

Les flux et termes de force \eqref{chap:GHD:eq.conserv.2.1} s’expriment alors en remplaçant la vitesse effective $v^{\mathrm{eff}}$ et l’accélération effective $a^{\mathrm{eff}}$ (de \eqref{chap:GHD:eq.nu.v.a.1}) par leurs expressions issues de la dynamique hamiltonienne libre :
\begin{equation}
	v^{\mathrm{eff}} \to \partial_\theta h, 
	\quad a^{\mathrm{eff}} \to  -\partial_x h.
\end{equation}

Dans le cadre de \eqref{chap:GHD:eq.ham.2} \(\partial_\theta h = \theta\) et   \(\partial_x h  = V' \). De plus en ne considérant que les premières charges conservées associées à $f(\theta) = 1$, $\theta$ et $\theta^2/2$ dans \eqref{chap:GHD:eq.conserv.2} et \eqref{chap:GHD:eq.conserv.2.1}, on retrouve les équations d’Euler classiques :
\begin{eqnarray}\label{chap:3:eq:hydro.1}
	\left\{
	\begin{array}{rcl}
	\partial_t n + \partial_x (n u) &=& 0, \\
	\partial_t (n u) + \partial_x (n u^2 + \mathcal{P}) &=& -n \, \partial_x V(x), \\
	\partial_t E + \partial_x (u(E+\mathcal{P})) &=& 0,
	\end{array}
	\right .  
\end{eqnarray}
avec la densité de particule $n(x, t) = q_{[1]}$, la vitesse moyenne du fluide $u = \frac{q_{[\theta]}}{n}$ , la pression cinétique du fluide $\mathcal{P}(x, t) = \left( q_{[\theta^2]} - \frac{q_{[\theta]}^2}{n} \right)$ , l'énergie totale $E = nu^2/2 + nV + ne$ où $e(x,t)$ est l'énergie interne d'une particule.



\paragraph{Remarques sur les charges globales}
En l’absence de potentiel externe ($V = 0$), le système conserve certaines charges globales. Dans un système classique non intégrable, seules ces quelques charges sont conservées. Par exemple dans un système de Gibbs sont conservé , nombre total de particules , quantité de mouvement totale , énergie cinétique totale soit respectivement $Q[1]  =  \int dx \, q[1]$ , $Q[\theta] = \int dx \, q[\theta$ et $Q\left[\frac{\theta^2}{2}\right] = \int dx \, q\left[\frac{\theta^2}{2}\right] $. 
%\begin{equation}
%	Q[1]  &= & \int dx \, q[1] \, \text{(nombre total de particules)},\\ 
%	Q[\theta] = \int dx \, q[\theta] \, \text{(quantité de mouvement totale)},\\ 
%	Q\left[\frac{\theta^2}{2}\right] = \int dx \, q\left[\frac{\theta^2}{2}\right] \text{(énergie cinétique totale)}.
%\end{equation}

\medskip

En revanche, dans un système intégrable, une infinité de charges sont conservées. En particulier, pour tout $\theta \in \mathbb{R}$:
\(
\rho(\theta , t ) = Q[\delta(\cdot - \theta)] = \int dx \, \rho(x, \theta, t),
\)
et les charges associées à une observable quelconque $f(\theta)$ s’écrivent :
\(
Q[f] = \int d\theta \, f(\theta) \, \rho(\theta, t).
\)

\medskip

Cette structure est l’analogue classique de la description en termes de {\bf distribution de rapidité} dans le cadre intégrable. Elle constitue le point de départ naturel pour développer une description hydrodynamique généralisée (GHD) dans le cas intégrable.


\subsection{la vitesse effectif}

En partant de la définition de la vitesse effectif en \eqref{chap:GHD:eq.nu.v.a.1} et en utilisant la définition de l'appliocation \emph{dressing} \eqref{eq:dessing}, il vient que 
\begin{eqnarray}\label{chap:GHD:veff.1}
	2 \pi \, v^{\mathrm{eff}} \,   \rho   = \nu  \,  \partial_\theta h   + \nu  \,  \left ( \Delta \star ( \rho \,  v^{\mathrm{eff}} )  \right ) ,
\end{eqnarray}
où $\Delta$ désigne le décalage en diffusion défini dans le modèle LL  en Eq.\eqref{eq:I-1-16}
et en soustraiant $v^{\mathrm{eff}} ( \theta ) \nu (\theta) \left ( \Delta \star  \rho  \right )( \theta )$ des deux cotés et on obtiens 

\begin{eqnarray}\label{chap:GHD:veff.2}
	v^{\mathrm{eff}} ( \theta )  \left (  2\pi \, \rho ( \theta ) - \nu (\theta) \left ( \Delta \star  \rho  \right )( \theta ) \right )    = \nu( \theta )  \,  \left (  \partial_\theta h ( \theta )  +  \int d \theta' \, \Delta(\theta - \theta') \rho ( \theta') ( v^{\mathrm{eff}} ( \theta' ) - v^{\mathrm{eff}} ( \theta )  )   \right )  ,
\end{eqnarray}

En partant de la l'écriture de la fonction d'ocumation en \eqref{chap:GHD:eq.nu.v.a.1} et en utilisant la définition de l'appliocation \emph{dressing} \eqref{eq:dessing}, il vient que 
\begin{eqnarray}\label{chap:GHD:veff.3}
	2 \pi \rho - \nu \, 	\Delta \star  \rho = \nu, 
\end{eqnarray}
On obtient 
\begin{eqnarray}\label{chap:GHD:veff.4}
	v^{\mathrm{eff}} ( \theta )      =  \partial_\theta h ( \theta )   +  \int d \theta' \, \Delta(\theta - \theta') \rho ( \theta') ( v^{\mathrm{eff}} ( \theta' ) - v^{\mathrm{eff}} ( \theta )  )   ,
\end{eqnarray}
et dans le modèle de Lieb-Liniger $\partial_\theta h ( \theta )   = \theta$.
Sur le plan physique, le premier terme peut être interprété comme un décalage spatial induit par un processus de diffusion à deux corps. Le second terme quantifie le taux de ces processus de diffusion par unité de temps. L’équation ainsi obtenue correspond à l’équation hydrodynamique généralisée (GHD), formulée pour la première fois en 2016\cite{Bertini2016,CastroAlvaredo2016}.

Le résultat remarquable~\eqref{eq:46} a été obtenu pour la première fois par Bertini \emph{et al.}~\cite{Bertini2016} et Castro-Alvaredo \emph{et al.}~\cite{CastroAlvaredo2016}. Cette observation a constitué le point de départ des développements ultérieurs de l’hydrodynamique généralisée (GHD) dans les systèmes quantiques intégrables. Les travaux de Bertini \emph{et al.}~\cite{Bertini2016} s’appuient en partie sur ceux de Bonnes \emph{et al.}~\cite{Bonnes2014}, où la formule de la vitesse effective~\eqref{eq:47} avait été présentée pour la première fois dans le cadre d’un système quantique intégrable.

Dans le cadre général de l’hydrodynamique généralisée (GHD), l’équation~(48) s’interprète comme une extension naturelle du résultat classique obtenu pour le gaz de tiges dures. La distinction essentielle réside dans le fait que le décalage de diffusion \(\Delta(\theta - \theta_0)\) dépend désormais explicitement de la rapidité relative entre les quasi-particules, alors que, dans le cas du gaz de tiges dures, \(\Delta\) est une constante égale à l’opposé du diamètre des particules.

Sur le plan cinématique, on peut décrire la situation de la manière suivante~: un quasi-particule \emph{traceur} de rapidité \(\theta\) --- c’est-à-dire de moment asymptotique \(\theta\) en l’absence d’interactions --- se déplacerait, dans le vide, à vitesse constante \( \theta \). En présence d’une densité finie \(\rho(\theta_0)\) de quasi-particules de rapidité \(\theta_0\), cette vitesse est modifiée par les processus de diffusion à deux corps.

Pendant un intervalle de temps infinitésimal \([t,\, t + \delta t]\), le traceur subit en moyenne
\[
\delta t \times \left| v_{\mathrm{eff}}[\rho](\theta) - v_{\mathrm{eff}}[\rho](\theta_0) \right| \, \rho(\theta_0)
\]
collisions avec des quasi-particules de rapidité \(\theta_0\). Chaque interaction provoque un décalage spatial \(\Delta(\theta - \theta_0)\) vers l’arrière. L’équation~(48) formalise précisément cet effet cumulatif, résultant de l’intégration des contributions de toutes les collisions binaires sur l’espace des rapidités.

Cette analyse microscopique s’étend naturellement aux modèles à \(N\) corps, où les processus de diffusion se combinent et interagissent de manière non triviale, la fonction \(\Delta(\theta - \theta_0)\) encapsulant alors l’intégralité de la structure intégrable du système.

\subsection{Modèle de Lieb–Liniger}

Les informations relatives aux interactions entre particules sont contenues dans la définition du crochet de Poisson \eqref{chap:GHD:eq.chochet.bonnemain.1}, associée à l’opérateur de \emph{dressing} spécifique au modèle de Lieb–Liniger, défini en \eqref{eq:dessing}.  
L’Hamiltonien $H = \mathcal{Q}[h]$ \eqref{chap:GHD:eq.ham.1} s’écrit ici :
\begin{equation}\label{chap:GHD:eq.ham.2}
	h(x , \theta ) = \varepsilon(\theta) + V(x),
\end{equation}
où l’énergie cinétique est $\varepsilon(\theta) = \theta^2 / 2$ et $V(x)$ représente le potentiel extérieur.

\medskip

Dans ce modèle, la vitesse effective et l’accélération effective de \eqref{chap:GHD:eq.nu.v.a.1} se réécrivent :
\begin{equation}
	v^{\mathrm{eff}} = \frac{(\mathrm{id})^{\mathrm{dr}}_{[\nu]}}{1^{\mathrm{dr}}_{[\nu]}}, 
	\quad a^{\mathrm{eff}} = - V'(x).
\end{equation}

Avec l'equation \eqref{chap:GHD:veff.4} la vitesse effectiff dans le modèle de LL s'écrit 
\begin{equation}
	v^{\mathrm{eff}} = \theta +	\int d \theta' \, \Delta(\theta - \theta') \rho ( \theta') ( v^{\mathrm{eff}} ( \theta' ) - v^{\mathrm{eff}} ( \theta )  ) 
\end{equation}


Ainsi, les termes de force dans \eqref{chap:GHD:eq.conserv.2} et \eqref{chap:GHD:eq.conserv.2.1} prennent la forme :
\begin{equation}
	F_{[f]} = -V'(x) \int_{\mathbb{R}} d\theta \, f'(\theta) \, \rho(x, \theta).
\end{equation}

L’équation GHD \eqref{chap:GHD:eq.conserv.3} devient alors :
\begin{equation}\label{chap:GHD:eq.conserv.3.1}
	\partial_t \rho + \partial_x\!\left(v^{\mathrm{eff}} \rho\right) - V'(x) \, \partial_\theta \rho = 0.
\end{equation}


\subsection{Diagonalisation et invariants de Riemann dans la GHD spatiale étendue}

En dérivant la définition de l'application \emph{dressing} \eqref{eq:dessing}, on obtient la relation suivante :
\begin{equation}\label{chap:GHD:d.dressing}
	\partial_X(f^{\mathrm{dr}}) = \left (\partial_X f \right )^{\mathrm{dr}} + \frac{\Delta}{2\pi} \star ( f^{\mathrm{dr}} \partial_X \nu ), 	
\end{equation}
où les variables \(X = t, x, \theta\).

\medskip

En injectant les définitions \eqref{chap:GHD:eq.nu.v.a.1} dans l'équation GHD \eqref{chap:GHD:eq.conserv.3} puis en appliquant les dérivées à l'application \emph{dressing} conformément à \eqref{chap:GHD:d.dressing}, on obtient :  
\begin{eqnarray}
	\begin{array}{c}\left ( \left(\partial_t 1 \right)^{\mathrm{dr}} + \left(\partial_x  \partial_\theta h  \right)^{\mathrm{dr}} - \left(\partial_\theta  \partial_x h  \right)^{\mathrm{dr}} \right ) \nu + \left ( 1 + \nu \,  \frac{\Delta}{2 \pi } \star  \right ) \left ( 1^{\mathrm{dr}} \partial_t v +  \left ( \partial_\theta h \right )^{\mathrm{dr}} \partial_x \nu -  \left ( \partial_x h \right )^{\mathrm{dr}} \partial_\theta \nu\right ) = 0  \end{array}. 
\end{eqnarray}
Or, on a \(\partial_t 1 = 0\) et \(\partial_x \partial_\theta h = \partial_\theta \partial_x h\). Il en résulte donc l'équation locale de conservation :  
\begin{equation}
	\partial_t \nu + v^{\mathrm{eff}}\partial_x \nu
	+ a^{\mathrm{eff}} \partial_\theta \nu = 0.
\end{equation}  

\medskip

Dans les systèmes hyperboliques, la \emph{diagonalisation} d'une équation consiste à trouver une transformation des variables qui permet de décomposer le système couplé en un ensemble de modes indépendants, appelés \emph{invariants de Riemann} ou \emph{modes normaux}. 

\medskip

Dans le cadre de la GHD spatiale étendue, l'équation d'évolution de la densité \(\rho(x,\theta,t)\) est couplée de manière non triviale en \((x,\theta)\) par la vitesse effective \(v^{\mathrm{eff}}\) et l'accélération effective \(a^{\mathrm{eff}}\). La fonction d'occupation \(\nu(x,\theta,t)\) est définie par une transformation non locale dite \emph{dressing} qui incorpore les interactions du système.

\medskip

Grâce à cela, on peut affirmer que la fonction $\nu (x , \theta)$  s’interprète comme un continuum d’{\bf invariants de Riemann}, c’est-à-dire des variables normales qui restent constantes le long des caractéristiques du système.

\medskip

Cette diagonalisation est essentielle pour comprendre la structure hamiltonienne du système et simplifier l'analyse de sa dynamique, notamment dans le cadre spatialement étendu avec un dressing dépendant de la position. 





\section{Interpretation}

 






%\section{Equation Hydrodynamique Généralisé}
%
%\subsection{Description classique sans interaction}
%Considérons une distribution classique de particules dans l’espace des phases, notée $\varphi(x, p, t)$, représentant la densité de particules autour du point $(x, p)$ à l’instant $t$. En l’absence de phénomènes dissipatifs, cette densité est conservée le long du flot hamiltonien, c’est-à-dire \(
%\frac{d\varphi}{dt} = \frac{\partial \varphi}{\partial t} + \{ \varphi , H \} = 0,
%\)
%où $\{ \cdot , \cdot \}$ désigne le crochet de Poisson canonique :
%\begin{equation}
%\{ \varphi , H \} = \frac{\partial \varphi}{\partial x} \frac{\partial H}{\partial p} - \frac{\partial \varphi}{\partial p} \frac{\partial H}{\partial x}.
%\end{equation}
%Pour $d \varphi /dt = 0 $ , 
%\begin{equation}
%	\frac{\partial \varphi}{\partial t} + \{ \varphi , H \} = 0	
%\end{equation}
%
%
%Ce résultat exprime que la distribution $\varphi$ est constante le long des trajectoires dans l’espace des phases générées par le hamiltonien $H$. Sous cette hypothèse, on peut réécrire l’équation de conservation sous forme différentielle :
%
%\begin{equation}
%\partial_t \varphi + \partial_x ( \dot{x} \varphi ) + \partial_p ( \dot{p} \varphi ) = 0,
%\end{equation}
%
%où les équations du mouvement hamiltonien sont :
%\(
%\dot{x} = \frac{\partial H}{\partial p}, \qquad \dot{p} = -\frac{\partial H}{\partial x}.
%\)
%
%Cette équation prend alors la forme d’une équation de continuité dans l’espace des phases :
%
%\begin{equation}
%\partial_t \varphi + \partial_x j_x + \partial_p j_p = 0,
%\end{equation}
%
%où les densités de courant sont données par :
%\(
%j_x = \dot{x} \varphi, \qquad j_p = \dot{p} \varphi.
%\)
%
%\paragraph{Exemple : particules libres dans un potentiel externe}
%Prenons pour Hamiltonien :
%
%\begin{equation}
%H = \varepsilon(p) + V(x), \qquad \text{où } \varepsilon(p) = \frac{p^2}{2m},
%\end{equation}
%
%correspondant à un système de particules classiques de masse $m$ soumises à un potentiel externe $V(x)$, sans interaction entre particules.
%
%L'équation de conservation s’écrit alors :
%
%\begin{equation}
%\partial_t \varphi + v(p) , \partial_x \varphi - \partial_x V(x) , \partial_p \varphi = 0,
%\end{equation}
%
%où $v(p) = \partial_p \varepsilon(p) = p/m$ est la vitesse du flot hamiltonien dans l’espace des phases.
%
%\paragraph{Charges locales conservées et équations hydrodynamiques}
%On définit une observable locale (ou charge locale) $q[f](x, t)$ associée à une fonction test $f(p)$ par :
%
%\begin{equation}
%q[f](x, t) = \frac{1}{m} \int_{\mathbb{R}} dp \, f(p) \, \varphi(x, p, t).
%\end{equation}
%
%Cette quantité représente la moyenne locale de $f(p)$ pondérée par la distribution $\varphi$. En particulier : la densité de particules : $n(x, t) = q[1]$,l’impulsion moyenne locale : $u(x, t) = \frac{q[p]}{n m}$, la pression cinétique : $\mathcal{P}(x, t) = \frac{1}{m} \left( q[p^2] - \frac{q[p]^2}{q[1]} \right)$.
%
%Les courants associés à ces charges s’écrivent :
%\begin{equation}
%j[f](x, t) = \frac{1}{m} \int dp \, f(p) , \partial_p H(x, p) \, \varphi(x, p, t).
%\end{equation}
%
%En prenant la dérivée temporelle de $q[f]$ et en utilisant l’équation de Liouville, on obtient une équation de conservation de la forme :
%
%\begin{equation}
%\partial_t q[f] + \partial_x j[f] = \frac{1}{m} \int dp \, f(p) , \partial_p \left( \partial_x V(x) \, \varphi \right),
%\end{equation}
%
%qui ne s’annule en général que si $V(x)$ est constant. Toutefois, dans le régime dit hydrodynamique, où $\varphi(x,p,t)$ varie lentement en espace, cette équation devient fermée sur les seules densités $q[f]$, en négligeant les dérivées spatiales d'ordre élevé.
%
%Dans ce cadre, et en ne retenant que les premières charges conservées associées à $f(p) = 1$, $p$, $p^2$, on retrouve les équations d’Euler classiques :
%
%\begin{eqnarray*}
%	\partial_t n + \partial_x (n u) &=& 0, \\
%	\partial_t (m n u) + \partial_x (m n u^2 + \mathcal{P}) &=& -n \, \partial_x V(x), \\
%	\partial_t \mathcal{E} + \partial_x j[\varepsilon(p)] &=& -\partial_x V(x) \cdot q[p],
%\end{eqnarray*}
%
%où $\mathcal{E} = q[\varepsilon(p)]$ est la densité d'énergie, et $j[\varepsilon(p)]$ le courant d'énergie.
%
%\paragraph{Remarques sur les charges globales}
%En l’absence de potentiel externe ($V = 0$), le système conserve certaines charges globales. Dans un système classique non intégrable, seules ces quelques charges sont conservées. Par exemple dans un système de Gibbs sont conservé
%\(
%	Q[1] = \int dx \, q[1] \quad \text{(nombre total de particules)}, 
%	Q[p] = \int dx \, q[p] \quad \text{(quantité de mouvement totale)}, 
%	Q\left[\frac{p^2}{2m}\right] = \int dx \, q\left[\frac{p^2}{2m}\right] \text{(énergie cinétique totale)}.
%\)
%En revanche, dans un système intégrable, une infinité de charges sont conservées. En particulier, pour tout $p \in \mathbb{R}$:
%
%\begin{equation}
%Q[\delta(\cdot - p)] = \frac{1}{m} \int dx \, \varphi(x, p, t),
%\end{equation}
%
%%et les charges associées à une observable quelconque $f(p)$ s’écrivent :
%%
%%\begin{equation}
%%Q[f] = \int dp , f(p) , \rho(p, t).
%%\end{equation}
%%
%%Cette structure est l’analogue classique de la description en termes de "rapidity distribution function" dans le cadre quantique. Elle constitue le point de départ naturel pour développer une description hydrodynamique généralisée (GHD) dans le cas intégrable.
%%
%%
%
%\subsection{Description classique avec interactions}
%
%%Dans la formulation hamiltonienne de la GHD, le champ dynamique est la densité fluide à deux variables  . 
%On définit un crochet de Poisson fonctionnel agissant sur les fonctionnelles $F[\rho]$ et $G[\rho]$  de la distribution de rapidité , avec intéraction. Conformément à Bonnemain et al.\cite{bonnemain2024hamiltonian}  :
%\begin{equation}
%	\{F,G\}\;=\;\iint dx\,d\theta\;\frac{\nu(\theta)}{2\pi}\,\Bigl[\partial_x\frac{\delta F}{\delta \rho(x,\theta)}\,\left(\partial_\theta \left ( \frac{\delta G}{\delta \rho(x,\theta)} \right ) \right)^{\mathrm{dr}} -\partial_x\frac{\delta G}{\delta \rho (x,\theta)}\,\left( \partial_\theta \left ( \frac{\delta F}{\delta \rho (x,\theta)} \right )\right)^{\mathrm{dr}} \Bigr]\,,	
%\end{equation}
%où $\nu$ est la fonction d’occupation.
%%\cite{bonnemain2024hamiltonian,doyon2020lecture}
%
%Pour toute fonction réelle et régulière \( f(x, \theta) \) définie sur \( \mathbb{R}^2 \), on associe le fonctionnel linéaire suivant :
%\begin{equation}
%	Q[f] = \int_{\mathbb{R}^2} dx\, d\theta\, f(x, \theta)\, \rho(x, \theta).
%\end{equation}
%Il s'agit de la charge totale associée à une quantité prenant la valeur \( f(x, \theta) \) pour chaque quasi-particule.  Le crochet de Poisson entre deux charges \( Q[f] \) et \( Q[g]\) s’écrit :
%\begin{equation}
%	\{ Q[f] , Q[g] \} = \int_{\mathbb{R}^2} \frac{dx\, d\theta}{2\pi} \nu  \left( \partial_x f  (\partial_\theta g )^{\mathrm{dr}}  - \partial_x g (\partial_\theta f)^{\mathrm{dr}}  \right),
%\end{equation}
%or l'application dressing satisfait la relation de symétrie \cite{doyon2020lecture}:
%\begin{equation}
%	\int_{\mathbb{R}^2}	 dx\, d\theta \, \nu f g^{\mathrm{dr}} = \int_{\mathbb{R}^2}	 dx\, d\theta \, \nu f^{\mathrm{dr}} g,
%\end{equation}
%soit avec une integration part partie, on réécrit le crochet 
%\begin{equation}
%	\{ Q[f] , Q[g]\} = \int_{\mathbb{R}^2} \frac{dx\, d\theta}{2\pi} f  \left( \partial_\theta ( \nu   (\partial_x g )^{\mathrm{dr}} )   - \partial_x ( \nu   (\partial_\theta g )^{\mathrm{dr}} )  \right).
%\end{equation}
%
%La distribution de rapidité $\rho( x , \theta )  = Q[\delta( \cdot - x )\delta( \cdot - \theta  )  ]$ et pour un hamiltinien $H = Q[h]$ avec $h(x , \theta ) = \varepsilon(\theta) + V(x)$ avec $\varepsilon(\theta) = m \theta^2/2$.
%
%\begin{equation}
%	\{ \rho(x, \theta), Q[h] \} + \partial_x (v^{\mathrm{eff}} \rho) + \partial_\theta (a^{\mathrm{eff}} \rho) = 0.
%\end{equation}
%
%Nous avons ici utilisé les identités (2.29), ainsi que la définition de la fonction d’occupation (rappelée pour commodité) :
%
%\begin{equation}
%	v^{\mathrm{eff}} = \frac{\varepsilon'^{\mathrm{dr}}}{1^{\mathrm{dr}}}, 
%	\quad 
%	a^{\mathrm{eff}} = -V, 
%	\quad 
%	\nu = \frac{\rho}{\rho_s}.
%\end{equation}
%
%Ainsi, en posant \( \partial_t \rho(x, \theta) = \{ \rho(x, \theta), Q[h] \} \), on retrouve bien les équations de la GHD sous forme hamiltonienne étendue à l’espace :
%
%\begin{equation}
%	\partial_t \rho(x, \theta) = -\partial_x (v^{\mathrm{eff}} \rho) - \partial_\theta (a^{\mathrm{eff}} \rho).
%\end{equation}
%
%











%\input{chapters/97_GHD}
\input{chapters/04_GGE_Fluctuation}
\chapter{Dispositif expérimental et méthodes d’analyse}
\label{chap:disp.exp}
\minitoc

%\section{Présentation de l’expérience}
%\section*{Introduction}
%
%\section{Refroidissement}
%
%\section{Imagerie}
%\subsection{Prubleme d'iamgerie et idée numerique}
%
%\section{Confinement transverse}
%
%\section{Confinement longitudinale}
%
%\subsection{Evolution logitudinale}
%
%\section{Outil de sélection spatial}
%
%\subsection{Mesure de distribution de rapidités locales $\rho(x , \theta ) $  pour des systèmes en équilibre}
%
%%\subsection{Piégeage transverses et longitudinale}
%%\section{Outil de sélection spatial}
%%%\section{Mesure de $\rho(x , \theta ) $ }
%
%%\section{Mesure de distribution de rapidités locales $\rho(x , \theta ) $  pour des systèmes en équilibre}

\section*{Introduction}

%\begin{itemize}
%	\item Objectif du chapitre : présentation synthétique de l’expérience
%	\item Distinction claire des contributions : mise en place initiale (précédents doctorants), développement (travail de Léa Dubois), contribution personnelle (prise de données, analyses spécifiques, participation à certaines manipulations)
%	\item Rôle de l’expérience dans l’étude de la dynamique des gaz de Bose 1D
%\end{itemize}

Ce chapitre présente l’expérience utilisée pour étudier les gaz unidimensionnels de rubidium ultra-froids. Nous décrivons l’architecture du dispositif, les méthodes d’imagerie et d’analyse, ainsi que les protocoles expérimentaux auxquels j’ai participé. Le développement initial du refroidissement et du piégeage avant la puce a été réalisé par d’anciens doctorants. La mise en place du piégeage sur la puce et du système de sélection spatiale à l’aide d’un DMD a été initiée par Léa Dubois, alors en première année de doctorat à mon arrivée. Mon travail s’est concentré principalement sur la prise de données, l’analyse et la participation à certaines expériences spécifiques telles que l’expansion longitudinale et la mesure locale de la distribution de rapidité.


\paragraph{Objectif du chapitre}  
Ce chapitre a pour objectif de fournir une présentation synthétique et structurée du dispositif expérimental utilisé pour étudier la dynamique de gaz de Bose unidimensionnels ultra-froids. Il constitue un socle indispensable pour comprendre les protocoles expérimentaux développés au cours de ma thèse et les analyses présentées dans les chapitres suivants.

\paragraph{Architecture générale}  
Nous présentons d'abord l’architecture complète de l’expérience, depuis la production des atomes jusqu’à leur imagerie, en passant par les étapes de refroidissement, de piégeage magnétique sur puce, de manipulation optique, et de génération de potentiels. Cette description s’accompagne d’une mise en contexte des contributions historiques au dispositif.

\paragraph{Contributions successives et personnelles}  
Une attention particulière est portée à la répartition chronologique des contributions. Les étapes initiales (source atomique, MOT, piège DC) ont été développées par d’anciens doctorants. La mise en place du piégeage 1D sur puce ainsi que l’utilisation du DMD pour la sélection spatiale ont été réalisées au cours de la thèse de Léa Dubois. Mon travail s’inscrit dans cette continuité et concerne principalement la prise de données, l’analyse de protocoles dynamiques, ainsi que la participation à certaines opérations de maintenance et d’optimisation du système.

\paragraph{Rôle du dispositif dans la thèse}  
Ce dispositif permet d’explorer des phénomènes hors équilibre dans des gaz quantiques 1D. Il constitue une plateforme particulièrement adaptée à l’étude de protocoles d’expansion, de sondes locales, ou de dynamiques guidées par la théorie hydrodynamique généralisée (GHD), qui sont au cœur de cette thèse.




%\section{Présentation générale de l’expérience}
%\subsection{Vue d’ensemble du dispositif}
%\begin{itemize}
%    \item Architecture générale : production, piégeage, manipulation et imagerie.
%    \item Systèmes étudiés : gaz de rubidium 87 dans des pièges 1D.
%    \item Objectifs : exploration de dynamiques hors équilibre.
%\end{itemize}
%
%\subsection{Historique et contributions successives}
%\begin{itemize}
%    \item Étapes de refroidissement et piégeage initial : travaux antérieurs (voir thèses citées).
%    \item Développement du piégeage 1D sur puce et du DMD : thèse de Léa Dubois.
%    \item Contributions personnelles : prise de données, protocoles dynamiques, analyse.
%\end{itemize}

\section{Le dispositif expérimental}
\subsection{Système laser et contrôle de fréquence}
\label{sec:systeme_laser}

%\paragraph{Laser maître 1 : référence de fréquence}
%La référence principale de fréquence pour l'ensemble des faisceaux utilisés dans l'expérience est fournie par un laser à cavité étendue, développé au SYRTE. Ce laser est asservi par spectroscopie d’absorption saturée sur la transition D2 du $^{87}$Rb, au croisement des transitions $|F=2\rangle \rightarrow |F'=2,3\rangle$. Ce signal de référence est utilisé pour verrouiller les autres sources laser par battement optique.

\paragraph{Laser maître 1 : référence de fréquence}
La stabilité en fréquence de l’ensemble des faisceaux employés dans l’expérience est assurée par un laser à cavité étendue conçu au SYRTE. Ce laser est verrouillé par spectroscopie d’absorption saturée sur la raie D2 du $^{87}$Rb, en ciblant le croisement des transitions $|F=2\rangle \rightarrow |F'=2,3\rangle$. Ce verrouillage fournit la référence absolue de fréquence à partir de laquelle les autres sources laser sont synchronisées par battement optique.

%\paragraph{Laser repompeur}
%Un laser DFB (Distributed Feedback Diode) est utilisé pour produire le faisceau repompeur, permettant de transférer les atomes retombés dans l’état $|F=1\rangle$ vers l’état $|F=2\rangle$. Ce laser est asservi à une fréquence distante de 6\,GHz de celle du maître 1, en utilisant un montage de battement optique et mélange avec un oscillateur à 6.6\,GHz. Une diode Fabry-Perot injectée par la DFB permet d’amplifier la puissance au-delà de 100\,mW.
%
%\paragraph{Laser repompeur}
%Le faisceau de repompage, qui permet de transférer les atomes piégés dans l’état $|F=1\rangle$ vers l’état $|F=2\rangle$, est généré par une diode DFB (Distributed Feedback). Sa fréquence est décalée de 6,GHz par rapport au maître 1 grâce à un système de battement optique combiné à un mélange avec un oscillateur micro-onde à 6.6,GHz. Une diode Fabry–Perot, injectée par la DFB, permet d’augmenter la puissance de sortie au-delà de 100,mW.

\paragraph{Laser repompeur}
Le faisceau de repompage, qui transfère les atomes tombé  dans l’état $|F=1\rangle$ vers l’état $|F=2\rangle$, est produit par une diode DFB (Distributed Feedback). Sa fréquence est décalée de 6 GHz par rapport au maître 1 par battement optique et mélange avec un oscillateur à micro-ondes de 6.6 GHz. Une diode Fabry–Perot, injectée par la DFB, élève la puissance de sortie au-delà de 100 mW.

%\paragraph{Laser maître 2 : laser principal de manipulation}
%Un second laser à cavité étendue, identique au maître 1, est asservi par battement optique à la fréquence du maître 1. Il est amplifié par un amplificateur à semi-conducteur évasé (Tapered Amplifier), permettant d’atteindre une puissance de sortie supérieure à 1\,W. Ce faisceau est ensuite divisé en plusieurs branches pour alimenter :
%\begin{itemize}
%    \item le Piège Magnéto-Optique (PMO),
%    \item la mélasse optique,
%    \item le pompage optique,
%    \item l’imagerie par absorption,
%    \item le faisceau de sélection.
%\end{itemize}

\paragraph{Laser maître 2 : source principale de manipulation}
Un second laser à cavité étendue, est verrouillé par battement optique sur la fréquence du maître 1. L’émission est amplifiée au moyen d’un amplificateur à semi-conducteur évasé (Tapered Amplifier), fournissant plus de 1\,W en sortie. Le faisceau ainsi produit est distribué vers différentes parties de l’installation expérimentale : alimentation du piège magnéto-optique (PMO), formation de la mélasse optique, réalisation du pompage optique, imagerie par absorption,génération du faisceau de sélection.


%\paragraph{Contrôle de fréquence et polarisation}
%Les fréquences des différents faisceaux sont ajustées via des Modulateurs Acousto-Optiques (AOM), tandis que leur polarisation et leur intensité sont contrôlées à l’aide de cubes PBS en combinaison avec des lames demi-onde motorisées ou fixes. Cette configuration assure une grande flexibilité dans la mise en œuvre des différentes phases expérimentales.

\paragraph{Gestion des fréquences et polarisations}
%Les ajustements de fréquence des divers faisceaux sont réalisés à l’aide de modulateurs acousto-optiques (AOM).
Les faisceaux peuvent être interrompus soit à l’aide d’obturateurs mécaniques, soit via des modulateurs acousto-optiques (AOM). Ces derniers offrent un temps de commutation beaucoup plus court que les systèmes mécaniques, car ils permettent de sélectionner uniquement un ordre de diffraction non nul et d’éteindre instantanément le faisceau en interrompant l’alimentation radiofréquence. L’intensité et la polarisation sont réglées via des cubes séparateurs PBS associés à des lames demi-onde, fixes ou motorisées. Ce dispositif offre une grande souplesse pour adapter la configuration optique aux différentes étapes de l’expérience.

%\paragraph{Remarque}
%Une description plus détaillée du montage laser et de son verrouillage peut être trouvée dans la thèse de A.~Johnson~\cite{Johnson2016}. L’ensemble a été maintenu et utilisé sans modifications majeures au cours de ma thèse.

\paragraph{Note}
Une présentation plus exhaustive du montage laser et de son système de verrouillage est disponible dans la thèse de A.Johnson\cite{Johnson2016}. Le dispositif a été conservé dans son architecture d’origine tout au long de mes travaux, avec seulement un entretien régulier.


\subsection{Production et refroidissement des atomes (non détaillé ici, renvoi à d'autres travaux)}
{\color{blue}
\begin{itemize}
    \item Source chaude de rubidium, MOT, molasses optique.
    \item Refroidissement à des températures sub-$\mu~K$ Refroidissement sub-Doppler (détails renvoyés aux travaux précédents).
\end{itemize}
}
%Le dispositif expérimental permet de produire des gaz de rubidium ultra-froids, avec pour objectif final l’obtention de gaz unidimensionnels dans le régime quantique dégénéré. La production suit une séquence expérimentale déjà bien établie, initialement développée par d’anciens doctorants (voir par exemple la thèse d’A. Johnson~\cite{Johnson2016}), puis réoptimisée au début de la thèse de Léa-Dubois ~\cite{L.Dubois2024} sous la supervision d’I. Bouchoule.

Le dispositif expérimental permet de produire des gaz ultra-froids de rubidium, en vue d’obtenir des gaz unidimensionnels dans le régime quantique dégénéré. La séquence expérimentale suit un protocole établi, initialement développé par d’anciens doctorants (voir par exemple la thèse d’A. Johnson~\cite{Johnson2016}) et réoptimisé au début de la thèse de Léa. Dubois~\cite{L.Dubois2024} sous la supervision d’I. Bouchoule.

%\paragraph{Libération des atomes de rubidium}
%Les atomes de $^{87}$Rb sont libérés à partir d’un \emph{dispenser}, placé directement dans l’enceinte à vide, sur le côté de la monture de la puce atomique. Ce composant, parcouru par un courant de \( 4.5\,\mathrm{A} \) pendant environ \( 5\,\mathrm{s} \), émet un flux d’atomes thermiques dans la chambre à vide.

\paragraph{Libération des atomes de rubidium}
Les atomes de $^{87}$Rb sont émis à partir d’un \emph{dispenser} placé directement dans l’enceinte à vide, à proximité de la monture de la puce atomique. Un courant de \( 4.5\,\mathrm{A} \)  est appliqué pendant environ \( 5\,\mathrm{s} \), générant un flux d’atomes thermiques dans la chambre à vide.

%
%\paragraph{Capture par piège magnéto-optique (PMO)}
%Les atomes thermiques sont ralentis et piégés à l’aide d’un piège magnéto-optique. Celui-ci utilise quatre faisceaux laser (dont deux sont réfléchis par la puce) et un champ quadrupolaire magnétique généré par des bobines. Le nuage ainsi formé se situe à quelques millimètres de la surface de la puce.

\paragraph{Capture par le piège magnéto-optique (PMO)}
Les atomes thermiques sont ralentis et confinés dans un piège magnéto-optique. Quatre faisceaux laser (dont deux réfléchis par la puce) combinés à un champ quadrupolaire magnétique produit par des bobines permettent de former un nuage atomique situé à quelques millimètres de la surface de la puce.

%\paragraph{Rapprochement vers la puce}
%Pour rapprocher les atomes de la puce, on transfère le champ quadrupolaire depuis les bobines vers un champ généré par le fil en forme de U de la puce (fil bleu dans la Fig.~\ref{fig:puce}). Ce fil est parcouru par un courant variant de \( 3.6\,\mathrm{A} \) à \( 1.5\,\mathrm{A} \), ce qui rapproche le nuage à quelques centaines de micromètres de la surface.

\paragraph{Rapprochement vers la puce}
Le nuage est rapproché de la surface de la puce en transférant le champ quadrupolaire depuis les bobines vers le champ produit par le fil en forme de U de la puce (fil bleu, Fig.~\ref{fig:puce}). Le courant dans ce fil est ajusté lentement de \( 3.6\,\mathrm{A} \) à \( 1.5\,\mathrm{A} \), ce qui positionne le nuage à quelques centaines de micromètres de la surface.

%\paragraph{Mélasse optique}
%Une phase de mélasse optique permet un refroidissement sub-Doppler des atomes capturés. Un système d’imagerie provisoire est utilisé à cette étape pour visualiser le nuage atomique, dont la taille dépasse le champ d’observation du système d’imagerie final.

%\paragraph{Mélasse optique}
%Une étape de mélasse optique est ensuite appliquée pour atteindre un refroidissement sub-Doppler des atomes capturés. %Un système d’imagerie provisoire permet de visualiser le nuage, dont la taille dépasse le champ d’observation du dispositif final.

%\paragraph{Pompage optique}
%Afin de polariser les atomes dans l’état magnétique \( |F=2,\,m_F=2\rangle \), un pompage optique est effectué avec un faisceau circulairement polarisé \( \sigma^+ \), résonant sur la transition \( |F=2\rangle \rightarrow |F'=2\rangle \).

\paragraph{Pompage optique}
Enfin, les atomes sont préparés dans l’état magnétique \( |F=2,\,m_F=2\rangle \) par pompage optique. Un faisceau circulairement polarisé \( \sigma^+ \), résonant sur la transition \( |F=2\rangle \rightarrow |F'=2\rangle \), assure la polarisation du nuage.

\paragraph{Mélasse optique}
Après la capture dans le PMO, une étape de mélasse optique est appliquée pour refroidir davantage le nuage atomique, au-delà de la limite de Doppler. La mélasse optique repose sur l’utilisation de faisceaux laser légèrement désaccordés en fréquence et polarisés de manière appropriée, qui interagissent avec les atomes selon le mécanisme de refroidissement sub-Doppler.

Le principe physique est le suivant : les atomes en mouvement voient les faisceaux laser avec un décalage Doppler, ce qui modifie la probabilité d’absorption selon leur vitesse et leur position. Combiné avec les effets de polarisation (notamment les forces de type Sisyphus dans un champ de polarisation variable), cela crée un potentiel de friction optique qui ralentit les atomes. Contrairement au refroidissement Doppler standard, la mélasse optique permet de réduire l’énergie cinétique des atomes en dessous de la limite Doppler, atteignant des températures beaucoup plus basses.

Ainsi, cette étape permet d’obtenir un nuage plus dense et plus froid, condition essentielle pour les manipulations ultérieures et la formation de gaz unidimensionnels dans le régime quantique dégénéré.






\subsection{Piégeage magnétique sur puce}
%{\color{blue}
%\begin{itemize}
%    \item Présentation de la puce atomique.
%    \item Confinement transverse et longitudinal.
%    \item Régime 1D : conditions d’accès (\(\hbar \omega_\perp \gg k_B T\)).
%    \item Problèmes de rugosité, stabilité magnétique.
%\end{itemize}
%}

\subsubsection{Piégeage magnétique sur puce}
\label{sec:piegeage_puce}

%On peut utiliser des piégeage optique pour produire des stracture atomique longitudinale alongé. Certaines groupe de recherche utilise un redeau optique 2D pour obtenir un réseau 2D de tube longitudinaaux \cite{Kinoshita2004,LaburtheTolra2004,Paredes2004,Moritz2003}. Ce raseaux 2D produit un grand nombre de systéme atomique propise è l'étude de de gase 1D. Avec ce genre de dispositif on peux etudier des gas 1D peut dense car les densité peut etre moyenné sur tous les tudes. Mais avec ce genre de dispositif on ne peut pas étudier experimentalement les fluctudation dans le systéme. Nous pour gièger les atomes on utilise une puce atomique.
%
%\paragraph{Principe général}
%Les atomes de rubidium sont piégés grâce à une puce atomique intégrée dans l’enceinte à vide. Une puce atomique est un circuit microfabriqué contenant des micro-fils dans lesquels circulent des courants permettant de générer des champs magnétiques à géométrie contrôlée. Ce dispositif, développé dans les années 1990 \cite{Denschlag1999,Fortagh1998}, permet une miniaturisation du système de piégeage \cite{Folman2000,Reichel1999}, les premiers condensats sur puce ont été obtenus en 2001 \cite{Haensel2001,Ott2001} et la premièref fois aux laboratoir Charles Fabry (LCF) \cite{Aussibal2003} et un accès à des confinements forts, particulièrement adaptés à l'étude de gaz de Bose unidimensionnels \cite{Schumm2005,Trebbia2006}.

-------

On peut créer des structures atomiques allongées en utilisant des techniques de piégeage optique. Par exemple, plusieurs groupes de recherche ont recours à des réseaux optiques bidimensionnels (2D) pour former un ensemble de tubes atomiques longitudinaux \cite{Kinoshita2004,LaburtheTolra2004,Paredes2004,Moritz2003}. Ces réseaux 2D permettent de produire un grand nombre de systèmes atomiques quasi-unidimensionnels, offrant ainsi une plateforme idéale pour l’étude des gaz 1D. Ce type de dispositif est particulièrement adapté à l’étude de gaz faiblement denses, car les densités peuvent être moyennées sur l’ensemble des tubes. Cependant, l’étude expérimentale des fluctuations locales dans chaque tube reste difficile avec ce genre de configuration. Pour surmonter cette limitation, on utilise le piégeage à l’aide de puces atomiques.

\paragraph{Principe général}
Les atomes de rubidium sont confinés par une puce atomique intégrée dans l’enceinte à vide. Une puce atomique est un circuit microfabriqué comportant de fins micro-fils parcourus par des courants électriques, ce qui permet de générer des champs magnétiques à géométrie contrôlée. Cette technologie, développée dans les années 1990 \cite{Denschlag1999,Fortagh1998}, offre une miniaturisation significative des dispositifs de piégeage \cite{Folman2000,Reichel1999}. Les premiers condensats de Bose–Einstein sur puce ont été réalisés en 2001 \cite{Haensel2001,Ott2001}, puis ultérieurement au Laboratoire Charles Fabry \cite{Aussibal2003}. Les puces atomiques permettent d’accéder à des confinements très forts, particulièrement adaptés à l’étude des gaz de Bose unidimensionnels et à l’exploration de leurs propriétés quantiques locales \cite{Schumm2005,Trebbia2006}.


-----
Des structures atomiques allongées peuvent être réalisées par piégeage optique. Dans ce cadre, des réseaux optiques bidimensionnels (2D) permettent de créer un ensemble de tubes atomiques quasi-unidimensionnels \cite{Kinoshita2004,LaburtheTolra2004,Paredes2004,Moritz2003}. Ces réseaux offrent un grand nombre de systèmes atomiques identiques, facilitant l’étude statistique de gaz 1D faiblement dense. Toutefois, l’accès expérimental aux fluctuations locales dans chaque tube reste limité.

Pour contourner cette contrainte, les puces atomiques offrent une solution efficace. Ces dispositifs microfabriqués intègrent de fins micro-fils parcourus par des courants, générant des champs magnétiques de géométrie contrôlée et permettant des confinements très forts \cite{Denschlag1999,Fortagh1998,Folman2000,Reichel1999}. La miniaturisation ainsi obtenue a permis l’obtention des premiers condensats de Bose–Einstein sur puce dès 2001 \cite{Haensel2001,Ott2001}, et dés 2003 au Laboratoire Charles Fabry \cite{Aussibal2003}. Grâce à ces confinements, il devient possible d’étudier expérimentalement les propriétés de gaz de Bose unidimensionnels et leurs fluctuations locales \cite{Schumm2005,Trebbia2006}.

-------

\paragraph{Structure de la puce utilisée}
La puce utilisée au cours de cette expérience a été conçue en collaboration avec S.~Bouchoule, A.~Durnez et A.~Harouri (C2N). Elle repose sur un substrat de carbure de silicium sur lequel est déposé le circuit électrique. Ce dernier est recouvert d’une couche de résine BCB, aplanie par des cycles d’enduction et d’attaque plasma. Une fine couche d’or (\(\sim200\,\mathrm{nm}\)) est finalement évaporée afin de permettre l’utilisation de la puce comme miroir pour l’imagerie à \(780\,\mathrm{nm}\). La puce est soudée à l’indium sur une monture en cuivre inclinée à \(45^\circ\) par rapport à l’axe optique.

%\paragraph{Fils de piégeage et géométrie des champs}
%Plusieurs fils sont intégrés à la puce pour assurer les différentes étapes du piégeage et du transport des atomes : un fil en forme de Z est utilisé pour le piégeage initial (DC), tandis que trois micro-fils (symétriques et parallèles) sont utilisés pour former un guide unidimensionnel par courants alternatifs (AC). La géométrie des fils a été optimisée pour minimiser la dissipation de chaleur, limiter les couplages parasites et améliorer la symétrie du piège. Dans la zone d’intérêt, les atomes sont piégés à environ \(15\,\mu\mathrm{m}\) au-dessus des fils, soit à \(8\,\mu\mathrm{m}\) au-dessus de la surface de la puce.

%\paragraph{Fils de piégeage et géométrie des champs}
%La puce atomique comporte plusieurs ensembles de fils, chacun jouant un rôle précis dans les différentes étapes de la capture, du transport et du confinement des atomes.
\paragraph{Fils de piégeage et géométrie des champs}
La puce atomique intègre plusieurs ensembles de conducteurs, chacun conçu pour une étape spécifique de la capture, du transport et du confinement des atomes. L’ensemble de la séquence de transfert, depuis le piège magnéto-optique (PMO) jusqu’au guide unidimensionnel, repose sur une succession de configurations magnétiques générées par ces différents fils.

%\medskip
%\subparagraph{Fil en forme de U .}
%Après la phase de pré-refroidissement, le nuage est initialement capturé dans un piège magnéto-optique (PMO) situé au-dessus de la puce. Il est ensuite approché de la surface en transférant progressivement le champ quadrupolaire des bobines externes vers celui produit par un fil en forme de U intégré à la puce (phase \textit{U} : transfert du PMO vers la puce + mélace optique + ponpage optique). 

\subparagraph{Phase U : approche de la surface}
Après la phase de pré-refroidissement, le nuage est initialement capturé dans un PMO situé au-dessus de la puce. Il est ensuite rapproché de la surface en transférant progressivement le champ quadrupolaire des bobines externes vers celui produit par un fil en forme de U intégré à la puce (fils bleus dans la Fig.~\ref{fig:puce}). Cette étape (\textit{phase U}) est accompagnée d’un mélange optique et d’un pompage optique afin de préparer les atomes pour le piégeage magnétique.


%\medskip
%\subparagraph{Fil en forme de Z : Chargement dans le piège DC .}
%Après le pompage optique, les atomes sont transférés dans un piège magnétique combinant un courant continu circulant dans le fil en forme de Z de la puce (fil orange dans la Fig.~\ref{fig:puce}) et un champ magnétique externe. Ce piège, noté \emph{piège DC}, permet un confinement transverse important. Un refroidissement par évaporation radiofréquence est alors réalisé pendant environ \( 2.3\,\mathrm{s} \), ce qui abaisse la température du nuage à environ \( 1\,\mu\mathrm{K} \), pour un nombre d’atomes typiquement autour de \( 2.5 \times 10^5 \).

\subparagraph{Phase Z : piège DC et refroidissement}
À l’issue du pompage optique, les atomes sont transférés dans un piège magnétique combinant un courant continu circulant dans un fil en forme de Z (fil orange) et un champ magnétique externe. Ce \emph{piège DC} assure un confinement transverse fort. Un refroidissement par évaporation radiofréquence, d’une durée d’environ \(2.3\,\mathrm{s}\), abaisse la température du nuage à environ \(1\,\mu\mathrm{K}\), pour un nombre typique d’atomes de l’ordre de \(2.5\times 10^5\).
%\medskip
%Une fois chargé dans ce piège intermédiaire, le nuage est transporté vers la zone expérimentale. Dans cette région, trois micro-fils parallèles et symétriques (jaune), parcourus par des courants alternatifs (AC), créent un guide magnétique unidimensionnel assurant le confinement transversal des atomes. Le confinement longitudinal est obtenu grâce à deux paires de fils : d/d′ (rose) et D/D′ (vert).

\subparagraph{Transfert vers le guide unidimensionnel}
Une fois refroidi, le nuage est acheminé vers la zone expérimentale où trois micro-fils parallèles et symétriques (fils jaunes) parcourus par des courants alternatifs (AC) génèrent un guide magnétique unidimensionnel assurant le confinement transverse. Le confinement longitudinal est fourni par deux paires de fils : $d/d'$ (rose) et $D/D'$ (vert).

Le passage du piège DC au guide 1D est réalisé de manière adiabatique grâce à cinq rampes linéaires de courant d’une durée comprise entre \(50\) et \(60\,\mathrm{ms}\) chacune. Durant cette opération :  
(i) le courant dans le fil Z est progressivement réduit,  
(ii) le courant dans les micro-fils du guide est augmenté jusqu’à environ \(50\,\mathrm{mA}\),  
(iii) un courant initial de \(0.5\,\mathrm{A}\) est appliqué dans les fils $D$ et $D'$, puis ajusté pour maintenir fixe la position du centre de masse du nuage.  

Ce protocole minimise les oscillations résiduelles dans le guide et assure un découplage efficace entre la dynamique longitudinale et le confinement transverse. Ce dispositif a été développé au cours de la thèse de Léa Dubois~\cite{TheseLea} et a été utilisé dans le cadre de mes protocoles expérimentaux sur l’expansion longitudinale et les sondes locales de distribution de rapidité.

\subparagraph{Optimisation géométrique}
La géométrie des conducteurs de la puce a été conçue pour réduire la dissipation thermique, limiter les couplages parasites et garantir une bonne symétrie des champs magnétiques. Dans la zone expérimentale, les atomes sont piégés à environ \(15\,\mu\mathrm{m}\) au-dessus des fils, soit \(8\,\mu\mathrm{m}\) au-dessus de la surface de la puce.


\paragraph{Refroidissement final et accès au régime unidimensionnel}
Une dernière phase de refroidissement par évaporation radiofréquence est effectuée directement dans le guide AC. Grâce à l’anisotropie marquée du piège, le confinement transverse atteint une fréquence \(\omega_\perp\) telle que l’énergie quantique \(\hbar \omega_\perp\) dépasse largement les énergies thermique et chimique du système. On atteint ainsi le régime unidimensionnel, caractérisé par la hiérarchie d’énergies :
\[
k_B T, \mu \ll \hbar \omega_\perp,
\]
où \(\mu\) désigne le potentiel chimique et \(T\) la température du gaz.

Dans ce régime, le confinement transverse est assuré principalement par la géométrie des micro-fils et la présence de champs magnétiques externes, tandis que le confinement longitudinal, plus faible, est ajustable via une combinaison de champs magnétiques externes et de courants circulant dans des fils additionnels ($d/d'$ et $D/D'$). 

Les gaz obtenus contiennent typiquement entre \(3\times 10^3\) et \(1.5\times 10^4\) atomes, pour des températures de l’ordre de \(50\) à \(200\,\mathrm{nK}\). La Fig.~\ref{fig:gaz1D} illustre un exemple de nuage dans ce régime, observé avec le système d’imagerie final.



%\paragraph{Confinement transverse et longitudinal}
%Le confinement transverse est assuré principalement par la géométrie des fils et la présence de champs magnétiques externes. Sa fréquence élevée permet d’atteindre des énergies de confinement \(\hbar \omega_\perp\) bien supérieures aux énergies thermiques et chimiques du système, condition nécessaire à l’accès au régime 1D :
%\[
%k_B T, \mu \ll \hbar \omega_\perp.
%\]
%Le confinement longitudinal, plus faible, est modulable par combinaison de champs magnétiques externes et courants dans les fils additionnels.

\paragraph{Avantages du piégeage sur puce}
Comparé aux systèmes utilisant des réseaux optiques 2D, le piégeage sur puce ne fournit qu’un seul tube, ce qui permet un meilleur accès aux fluctuations locales de densité et aux observables résolues spatialement. Ce type de dispositif est ainsi particulièrement adapté à l'étude de la thermodynamique et de la dynamique de gaz 1D isolés.

\paragraph{Limitations et effets parasites}
Parmi les limitations spécifiques au piégeage sur puce figurent la rugosité des potentiels magnétiques due aux imperfections des fils, qui peut induire des modulations parasites du confinement longitudinal. De plus, la stabilité du dispositif est sensible aux champs parasites magnétiques externes ainsi qu’aux échauffements dus aux courants continus.





\paragraph{Imagerie finale}
À l’issue de ce refroidissement, les atomes sont observés avec le système d’imagerie final (voir Fig.~\ref{fig:imagerieFinale}), adapté aux tailles caractéristiques du gaz dans le piège. Une image typique de ce nuage est présentée en Fig.~\ref{fig:nuageDC}.



%\paragraph{Refroidissement final et accès au régime unidimensionnel}
%Une dernière phase de refroidissement par évaporation radiofréquence est ensuite réalisée dans le guide AC. Ce refroidissement, mené dans le piège à forte anisotropie, permet d’atteindre le régime unidimensionnel, caractérisé par la hiérarchie d’énergies :
%\[
%k_B T, \mu \ll \hbar \omega_\perp
%\]
%où \( \omega_\perp \) est la fréquence de confinement transverse, \( \mu \) le potentiel chimique et \( T \) la température du gaz.
%
%Les gaz obtenus contiennent typiquement entre \( 3 \times 10^3 \) et \( 1.5 \times 10^4 \) atomes, pour des températures de l’ordre de \( 50 \text{ à } 200\,\mathrm{nK} \). La Fig.~\ref{fig:gaz1D} montre un exemple de tel gaz observé avec le système d’imagerie final.


\paragraph{Remarques expérimentales}
Lorsque j’ai rejoint l’équipe, la première année thèse de Léa Dubois touchait à sa fin et le dispositif expérimental était en fonctionnement stable. Les différentes étapes du cycle (dispenser, PMO, mélasse, pompage optique, piège DC, transfert vers le guide, évaporation finale) avaient été mises en place et optimisées pendant les premières années de sa thèse, sous la supervision d’I. Bouchoule.Le cycle expérimental complet dure environ 15 secondes. Une description plus détaillée peut être trouvée dans la thèse d’A. Johnson~\cite{Johnson2016}.


Pendant ma première année, j’ai principalement participé à la prise de données en collaboration avec Léa. Grâce à la qualité de son travail, le dispositif était globalement très fiable, ce qui a permis de mener des campagnes expérimentales riches sans intervention lourde. Néanmoins, cette stabilité avait pour contrepartie que je n’ai pas été directement impliqué dans la résolution des pannes complexes ou dans le reconditionnement complet de la manipulation, ce qui a limité ma formation sur les aspects de maintenance approfondie du dispositif.

En revanche, peu avant la fin de la thèse de Léa et au début de ma troisième année, nous avons observé une chute significative du nombre d’atomes capturés. Sous la supervision d’I. Bouchoule, une intervention lourde a alors été décidée : nous avons cassé le vide pour diagnostiquer le problème. Il s’est avéré que les connecteurs du dispenser étaient endommagés. L’opération a été mise à profit pour installer un nouveau dispenser et remplacer la puce atomique.

Cette opération a mobilisé plusieurs personnes du laboratoire et de ses partenaires : S. Bouchoule (C2N) et Anne [Nom complet à préciser] ont participé à la manipulation et à la pose de la puce, tandis que j’ai pu assister à l’étuvage de l’enceinte à vide avec F. Nogrette. Après cette intervention, j’ai suivi avec I. Bouchoule le réajustement progressif de la séquence de refroidissement : alignement des faisceaux, réglages de la mélasse, optimisation du chargement dans le piège DC, puis dans le guide.

Cet épisode m’a permis de me confronter plus directement aux paramètres critiques du cycle d’évaporation et à la reprise d’une séquence complète. Toutefois, le départ de Léa, qui maîtrisait tous les aspects de la manipulation, a marqué une rupture importante dans la continuité des savoir-faire pratiques liés à cette expérience.


\begin{center}
	({fig:puce} — Schéma de la puce atomique avec fils U, Z, AC, D et D'.)
\end{center}
\begin{center}
	({fig:imagerieFinale} — Schéma optique du système d’imagerie final)
\end{center}
\begin{center}
	[{fig:nuageDC} — Image du gaz dans le piège DC après évaporation]
\end{center}
\begin{center}
	[{fig:gaz1D} — Image typique d’un gaz dans le régime 1D]
\end{center}



\subsection{Génération de potentiels modulés}
%\begin{itemize}
%    \item Courants modulés pour créer des pièges harmoniques ou quartiques.
%    \item Découplage transverse/longitudinal.
%\end{itemize}

\paragraph{Champ des micro-fils.}
Puisque que $m_F = 2 $, (état assuré par pompage optique), le potentiel magnétique $-\vec{\mu} \vec{B}(\vec{r}) $ (avec moment dipolaire magnétique alors $\vec{\mu}$ et le champs magnetque totale que resente les atomes$\vec{B}(\vec{r})$) est proportionnel à $\vert \vec{B}(\vec{r}) \vert$  de sorte que les atomes, en état low-field seeking, sont attirés vers les régions de champ magnétique minimal. Les micro-fils, alignés selon l’axe horizontal $\vec{e}_x$, sont parcourus par des courants alternatifs $\pm I$ (déphasés) produisant le champ magnétique de confinement : un fil central parcouru par un courant \( I \), et deux fils latéraux par des courants opposés \(-I\). 

\paragraph{Champ de biais.}
Un champ de biais transverse $\vec{B}_{\mathrm{biais}} = {B}_{\mathrm{biais}} \, \vec{e}_y$ , avec l'axe verticale par $\vec{e}_y$ , est appliqué afin de régler la distance des atomes par rapport aux micro-fils. En notant $\vec{e}_z$ l’axe horizontal perpendiculaire à $\vec{e}_x$ et $\vec{e}_y$ l’annulation du champ total a lieu en
%Dans cette configuration, un champ de biais transverse est appliqué pour ajuster la distance des atomes au-dessus des micro-fils. Pour y avoir une idée notons l'axe verticale par $\vec{e}_y$, et $\vec{B}_{biais} = {B}_{biais} \, \vec{e}_y$. Alors en notan $\vec{e}_z$ l'axe hortisontale perpetdiculaire à $\vec{e}_x$ et $\vec{e}_y$, le champs totale s'anume en ​
  %permet ainsi de positionner précisément le minimum du potentiel à une hauteur
$z_0 = \mu_0 I / (2 \pi {B}_{\mathrm{biais}} ) $ avec $\mu_0$ la perméabilité du vide . La modulation de ${B}_{\mathrm{biais}}$ permet de déplacer le point où le champ total s’annule, ce qui permet de positionner précisément le minimum du potentiel à une distance $d$ du plan des fils. 

\paragraph{Champ d’Ioffe.}
Afin d’éviter les pertes de Majorana liées à la présence d’un champ nul, un champ longitudinal $B_0 \, \vec{e}_x$ est ajouté, garantissant que le minimum de champ reste non nul.%.Un champ longitudinal (selon $\vec{e}_x$) $B_0$ est ajouté afin que ce minimum ne corresponde pas à un champ nul, ce qui supprime les pertes de Majorana dues aux inversions de spin au voisinage d’un zéro de champ. 

%L’intérêt de ces pièges est que les atomes peuvent être confinés très près des micro-fils — ici à $ d = 15\, \mu m$ , soit l’espacement entre deux fils — ce qui maximise le gradient de champ et donc la fréquence de piégeage transverse

\paragraph{Fréquence de piégeage transverse.}
Dans la configuration étudiée, les atomes sont confinés à $ d = 15\, \mu m$ au-dessus de la puce, soit l’espacement entre deux micro-fils. Cette faible distance maximise le gradient de champ et donc la fréquence de piégeage transverse, qui s’écrit
\begin{eqnarray*}
	\omega_\perp^{(0)} =  \sqrt{\frac{\mu_B}{mB_0}} \frac{\mu_0 I }{2\pi d^2} 
\end{eqnarray*}
avec $\mu_B$ le magnéton de Bohr, $m$ la masse atomique et $\mu_0$ la perméabilité du vide.
%Pour éviter que les atomes ne perçoivent les rugosités magnétiques dues aux défauts des conducteurs, on fait circuler dans les fils un courant alternatif à haute fréquence ($\sim 400\,KHz$) : le potentiel est alors moyenné temporellement, produisant un confinement plus lisse. À $15\, \mu m$ au-dessus de la puce, le profil de champ est localement harmonique, et la fréquence de piégeage transverse devient

\paragraph{Rugosité et suppression par modulation}
Les imperfections géométriques des micro-fils engendrent des fluctuations parasites du champ magnétique le long du guide, créant une rugosité du potentiel. Pour la supprimer, les courants sont modulés à haute fréquence ($\sim 400\,KHz$), bien au-delà des fréquences de piégeage. Dans ce régime, les atomes ne perçoivent que le potentiel moyenné temporellement, où la composante parasite longitudinale est fortement réduite. Le confinement effectif reste harmonique, avec une fréquence transverse donnée par
\begin{eqnarray*}
	\omega_\perp = \frac{\omega_\perp^{(0)}}{\sqrt{2}}.		
\end{eqnarray*}




%\paragraph{Découplage des confinements transverses et longitudinaux.}
%Les courants qui parcourent les fils D, D', d, d' sons selon $\vec{e}_u$ donc les chanps induit sont selon $\vec{e}_x$ noté $B_\parallel^x$ et $\vec{e}_v$ (axex normale à la puce) , noté $B_\parallel^v$. Si les champs selon $\vec{e}_x$ est négligeable devant $B_0$ alors la moyenne de pottenstelle presente une partie transverce et longitudinale decouplés : $\braket{V} = V_\perp ( y , z ) + V_\parallel(x) $.
\paragraph{Découplage des confinements transverses et longitudinaux.}
Les courants qui parcourent les fils $D$, $D'$, $d$ et $d'$ sont orientés selon $\vec{e}_u$. 
Les champs magnétiques induits possèdent alors une composante selon $\vec{e}_x$, notée $B_\parallel^x$, et une composante selon $\vec{e}_v$ (axe normal à la puce), notée $B_\parallel^v$. 
Si le champ selon $\vec{e}_x$ est négligeable devant $B_0$, alors le potentiel moyen se sépare en une partie transverse et une partie longitudinale découplées : 
\(
\braket{V} = V_\perp(y,z) + V_\parallel(x) .
\)


\paragraph{Potentiel longitudinal harmonique.}
Dans la configuration où seuls les fils $D$ et $D'$ sont utilisés, le potentiel longitudinal peut, à l’ordre 2 en $x$, être considéré comme harmonique :
\begin{eqnarray*}
	V_\parallel (x) = V_0 + \frac{1}{2} m \omega_\parallel^2 x^2 ,
\end{eqnarray*}
On note  $2L=1.89 \,mm$ est la distance séparant les fils $D$ et $D'$. Les courants circulant dans ces deux fils sont identiques et notés $I_D = I_{D'}$. Si la condition $B_0 \gg \mu_0 I_D d /(\pi L)^2 $ est vérifiée, alors le terme constant du potentiel vaut approximativement $V_0 \simeq \mu_B B_0$.

\medskip

La pulsation longitudinale totale $\omega_\parallel$ se décompose en deux contributions : (i) une pulsation $\omega_\parallel^x = \sqrt{\frac{6\, d \, \mu_B \, \mu_0 \,I_D }{\pi \, L^4 \, m}}$ induite par le champ longitudinal $B_\parallel^x$ et (ii) une pulsation $\omega_\parallel^v = \sqrt{\frac{\mu_B }{m \, B_0}}\frac{\mu_0 \, I_D }{\pi \, L^2}$ liée au champ  $B_\parallel^v$. Pour des courants $I>1A$ , on a $\omega_\parallel^v \gg \omega_\parallel^x$, et ainsi :  
\begin{eqnarray*}
	\omega_\parallel \propto \frac{I_D}{\sqrt{B_0} L^2}.
\end{eqnarray*} 
La fréquence longitudinale est donc réglée expérimentalement en ajustant $I_D$.

\medskip

Avec les dimensions caractéristiques de la puce et des fils, il est possible d’atteindre des confinements longitudinaux de fréquence $f_\parallel = \omega_\parallel/ 2 \pi$ allant jusqu’à $\sim 150 \, H_z$, la limite étant imposée par le chauffage des fils pour $I_D \leq =4 \, A$.
 
 \medskip
 
 \subparagraph{Mesure de la fréquence transverse et longitudinale}
Pour la caractérisation, la pulsation transverse $\omega_\perp$ a été mesurée par la méthode du mode de respiration transverse \cite{Kagan1996}, tandis que $\omega_\parallel$ a été obtenue à partir des oscillations dipolaires longitudinales. Les détails expérimentaux de ces méthodes figurent dans le manuscrit de thèse de Léa Dubois \cite{L.Dubois2024}, p. 73 et p. 78.

\medskip

 \paragraph{Potentiel longitudinal quartic.}
 Si on ajoute du courand  dans les fils $d$ et $d'$.  Alors on peux avoir un potentiel non gégligeable à l'ordre 4 . Pour simmplifier, les courants dans ces fils $I_d$ et $I_{d'}$ sont identique. et le potentiel s'écrit : 
 \paragraph{Potentiel longitudinal quartique.}
Si l’on ajoute un courant dans les fils $d$ et $d'$, on peut générer un potentiel longitudinal comportant un terme significatif à l’ordre 4 en $x$ . Pour simplifier, on suppose $I_d=I_{d'}$. On obtient alors : 
 \begin{eqnarray*}
 	V_\parallel(x) \, = \, \mu_B B_0  & + & 	 \frac{\mu_B \, \mu_0}{\pi} d  \left [ \frac{I_D}{L^2} + \frac{I_d}{l^2} + 3 \left ( \frac{I_D}{L^4} + \frac{I_d}{l^4} \right ) x^2  +  5 \left ( \frac{I_D}{L^6} + \frac{I_d}{l^6} \right ) x^4 \right ] \\
 	& + & \frac{\mu_B}{B_0} \left ( \frac{\mu_0}{\pi} \right )^2  \left [ \left ( \frac{I_D}{L^2} + \frac{I_d}{l^2} \right ) x^2  + 2 \left ( \frac{I_D}{L^2} + \frac{I_d}{l^2} \right )\left ( \frac{I_D}{L^4} + \frac{I_d}{l^4} \right ) x^4 \right ].
 \end{eqnarray*}
 
 En ajustant $I_D$ et $i_d$, on peut réaliser par exemple un double puits \cite{Schemmer2019}, ou bien supprimer le terme quadratique $x^2$ afin d’obtenir un potentiel quartique pur :
\begin{eqnarray*}
	V_\parallel(x) = a_0 + a_4 x^4 	
\end{eqnarray*}
comme on le fais dans \cite{Dubois2025}.

En pratique, la puce présente des dimensions finies et n’est pas parfaitement symétrique. Un calcul plus précis, prenant en compte la géométrie exacte (disposition et épaisseur des fils), est présenté en annexe de la thèse de Thibault Jacqmin \cite{???}, p. 151. Cela impose un ajustement fin et asymétrique des courants $I_D$, $I_{D'}$, $I_d$ et $I_{d'}$.


On ajuste les courant $I_D$ et $i_d$ pour par exemple fais des douple puit \cite{Schemmer2019} ou en supriment le terme en $x^2$  d’obtenir un potentiel longitudinal quartique de la forme $V_\parallel(x) = a_0 + a_4 x^4$ \cite{Dubois2025}.\\

En réalité la puce presente des dimention finie, Un calcul plus précis prenant en compte la géométrie exacte des fils (disposition sur la
puce, épaisseur finie) se trouve en appendice de la thèse de Thibault Jacqmin [112] , page 151. De plus la pude n'est pas pardetement symetrique donc on doit ajuster les courant $I_D$, $I_{D'}$, $I_d$ et $I_{d'}$.


\paragraph{Caractérisation des potentiels longitudinal et transverse.}
Pour atteindre le régime unidimensionnel, les confinements doivent être fortement anisotropes : un piégeage transverse très fort et un piégeage longitudinal faible. La condition \(\mu, k_B T \ll \hbar \omega_\perp\) garantit le gel des degrés de liberté transverses.

\medskip

Cette configuration est particulièrement adaptée pour obtenir des profils de densité homogènes, nécessaires à certaines expériences de transport. Le transfert des atomes du piège harmonique vers le piège quartique est réalisé de manière \emph{diabatique} (changement rapide du potentiel), car un transfert adiabatique entraîne des pertes importantes.

\paragraph{Caractérisation des potentiels longitudinal et transverse.}
Pour atteindre le régime unidimensionnel, les potentiels de piégeage doivent être très asymétriques : un confinement transverse fort et un confinement longitudinal faible. La fréquence transverse \(\omega_\perp\) doit être suffisamment élevée pour geler les degrés de liberté dans cette direction, avec la condition \(\mu, k_B T \ll \hbar \omega_\perp\).

%\paragraph{Potentiel longitudinal}
%
%Le confinement longitudinal est produit par des courants continus ou modulés dans certains fils. Dans certains protocoles spécifiques, on utilise un potentiel quartique \( V_\parallel(x) = c_4 x^4 \). Le système reste dans le régime 1D tant que la longueur caractéristique longitudinale reste beaucoup plus grande que la transverse.
%
%\paragraph{Potentiel transverse}
%
%Le confinement transverse est réalisé à l’aide de trois micro-fils parallèles situés sur la puce : un fil central parcouru par un courant \( I \), et deux fils latéraux par des courants opposés \(-I\). Cette configuration crée un piège transverse harmonique avec une fréquence \(\omega_\perp\) contrôlable par la valeur du champ \( B_0 \) et le courant. Les atomes sont piégés à environ \( d = 15~\mu\text{m} \) au-dessus de la puce. La fréquence maximale accessible expérimentalement est de l’ordre de \( \sim 100~\text{kHz} \).
%
%\paragraph{Effet de rugosité et suppression par modulation}
%
%La rugosité des micro-fils induit des fluctuations parasites du champ magnétique le long du guide. Pour supprimer cet effet, les courants sont modulés à haute fréquence (environ 400~kHz). Grâce à cette modulation rapide, les atomes ne ressentent que le potentiel moyen, dans lequel la composante parasite longitudinale du champ s’annule. Ce procédé permet d’obtenir un potentiel transverse régulier et stable, avec une fréquence efficace \[ f_\perp = \frac{f_\perp^{(0)}}{\sqrt{2}}. \]
%
%\paragraph{Découplage des confinements transverse et longitudinal.}
%Dans notre dispositif, le confinement transverse est assuré par les micro-fils modulés, tandis que le confinement longitudinal est généré par quatre fils extérieurs (D, D', d, d'). L’analyse du potentiel magnétique moyen montre que, sous l’hypothèse d’un champ de bobine homogène et dominant, les contributions transverse et longitudinale du potentiel sont découplées. Cette propriété est cruciale pour nos expériences : elle permet de modifier la géométrie du potentiel longitudinal sans perturber le confinement transverse, facilitant ainsi l’exploration de différentes configurations dynamiques.
%
%\paragraph{Piégeage longitudinal harmonique.}
%Un piège longitudinal harmonique est réalisé en appliquant des courants égaux dans les fils D et D', disposés de manière symétrique. Le champ magnétique longitudinal produit conduit à un potentiel quadratique local :
%\[
%V_\parallel(x) = V_0 + \frac{1}{2} m \omega_\parallel^2 x^2,
%\]
%avec une fréquence $\omega_\parallel$ contrôlée par le courant et la géométrie de la puce. En pratique, des fréquences jusqu’à 150 Hz sont atteintes pour des courants de 4 A. Une correction peut être nécessaire pour prendre en compte un champ magnétique résiduel $B_{0v}$, responsable d’un déplacement du centre du nuage atomique.
%
%\paragraph{Piégeage longitudinal quartique.}
%L’ajout de deux fils supplémentaires (d et d') permet de modifier la forme du potentiel longitudinal jusqu’à l’ordre 4. En ajustant les courants dans les quatre fils, on peut annuler le terme quadratique et obtenir un potentiel quartique :
%\[
%V_\parallel(x) = a_0 + a_4 x^4.
%\]
%Cette configuration est particulièrement adaptée pour générer des profils de densité homogènes, comme requis dans certaines expériences de transport. Le transfert des atomes du piège harmonique vers le piège quartique est réalisé de manière diabatique (changement rapide du potentiel), car un transfert adiabatique entraînait des pertes importantes.



\section{Sélection spatiale avec DMD}
\subsection{Motivation et principe}
{\color{blue}
\begin{itemize}
    \item Besoin de préparer des tranches homogènes.
    \item Intérêt dans les protocoles hors équilibre.
\end{itemize}
}

\paragraph{Objectif du dispositif de sélection}

L’outil de sélection spatiale a été conçu pour permettre une action locale sur le gaz atomique. Il présente deux objectifs principaux. D’une part, il permet de mesurer la distribution de rapidité localement résolue, en sélectionnant une tranche du gaz avant de la libérer et de suivre son expansion. D’autre part, il offre la possibilité de créer des situations hors équilibre en retirant une partie du gaz à l’équilibre, ce qui perturbe la configuration initiale et initie une dynamique.

\paragraph{Intérêt pour les protocoles hors équilibre}

Ce dispositif permet ainsi de générer des protocoles analogues à des configurations classiques comme le pendule de Newton, ou de sonder directement la dynamique d’un gaz de Lieb-Liniger dans des conditions contrôlées. Il constitue une brique essentielle pour les expériences de dynamique et de transport quantique.


\subsection{Mise en place technique (initiée par Léa Dubois)}

{\color{blue}
\begin{itemize}
    \item Dispositif optique de projection.
    \item Contrôle numérique des motifs.
    \item Calibration et stabilité.
\end{itemize}
}

\paragraph{Principe de sélection par pression de radiation}

La sélection repose sur l’illumination d’une zone définie du gaz avec un faisceau quasi-résonant avec la transition cyclique \( F=2 \rightarrow F'=3 \) de la ligne D2 du rubidium. Les atomes subissent une pression de radiation due aux cycles absorption/émission spontanée, ce qui les pousse hors du piège ou les amène dans un état non piégé.

\paragraph{Façonnage spatial du faisceau}

La sélection doit être spatialement résolue. Le profil d’intensité dans le plan des atomes est de type binaire :
\[
I(x) = 
\begin{cases}
0 & \text{si } x \in [x_1, x_2] \\
I_0 & \text{sinon}
\end{cases}
\]
ce qui permet de préserver ou d’éjecter les atomes selon leur position longitudinale.

\paragraph{Utilisation du DMD}

Pour générer ce profil, un DMD (Digital Micromirror Device) est utilisé. Il s’agit d’une matrice de \(1024 \times 768\) micro-miroirs orientables individuellement (±12°). En inclinant ces miroirs, on contrôle localement la réflexion de la lumière. L’image du DMD est projetée directement sur le plan des atomes, en imagerie directe.

\paragraph{Avantages du DMD}

Le DMD permet une reconfiguration rapide et programmable du motif de lumière. Cette technologie est largement utilisée dans les expériences d’atomes froids pour produire des potentiels structurés, homogénéiser un faisceau ou adresser localement les atomes.

\paragraph{Alternatives possibles}

Il est possible, en théorie, d’atteindre un effet similaire par un transfert cohérent des atomes vers un état anti-piégé via un pulse micro-onde ou une transition Raman. Cependant, la méthode par pression de radiation est plus simple à mettre en œuvre et adaptée à nos objectifs expérimentaux.

\paragraph{Principe de l’expulsion par pression de radiation}

Un atome illuminé par un faisceau proche de la résonance peut être expulsé du piège soit par transition vers un état anti-piégé, soit par effet de pression de radiation. Cette dernière génère une accélération suffisante pour fournir une énergie cinétique supérieure à la profondeur du puits magnétique. Le nombre de photons diffusés nécessaire peut être estimé à partir de la conservation de l’impulsion : une vingtaine de photons suffisent typiquement à extraire un atome du piège dans nos conditions.

\paragraph{Modèle de diffusion et estimation du seuil}

Le taux de diffusion de photons est modélisé à l’aide d’un taux \(\Gamma_{\mathrm{sc}}\), dépendant de l’intensité \(I\), de l’intensité de saturation \(I_{\mathrm{sat}}\), d’un paramètre \(\alpha\) (lié à la polarisation et au champ magnétique) et du désaccord \(\delta\). À résonance, et pour un temps d’illumination \(\tau_p\), on peut estimer le nombre total de photons diffusés par atome par \(N_{\mathrm{sc}} = \tau_p \Gamma_{\mathrm{sc}}\).

\paragraph{Mesures expérimentales de la puissance nécessaire}

La puissance minimale nécessaire pour éjecter tous les atomes d’une zone illuminée est déterminée en fixant un temps d’illumination donné, puis en variant l’intensité du faisceau. L’analyse est réalisée après un délai d’attente de \(\sim 10\) ms, pour s’assurer que seuls les atomes encore piégés soient détectés. Il est observé que 99$\%$ des atomes sont retirés à partir d’un rapport \(I/I_{\mathrm{sat}} \simeq 0.12\).

\paragraph{Mesures de photons diffusés par fluorescence}

La quantité de photons diffusés est également mesurée par l’analyse du signal de fluorescence capté par la caméra. En calibrant le rapport entre photons détectés et photons diffusés (en tenant compte de l’efficacité optique du système), le nombre moyen de photons nécessaires pour éjecter un atome est confirmé expérimentalement autour de 20. Un ajustement du modèle de diffusion permet d’estimer le paramètre \(\alpha \simeq 0.4\).

\paragraph{Saturation et effets Doppler}

À fort temps d’illumination (\(\tau_p > 150\,\mu\)s), une saturation du nombre de photons diffusés est observée, interprétée comme un effet géométrique : les atomes accélérés atteignent physiquement la puce atomique et cessent de contribuer au signal. Une correction Doppler peut être introduite dans le modèle, mais reste négligeable (\(< 5\%\)) dans les régimes expérimentaux utilisés.

\paragraph{Limitations expérimentales de la sélection}

Plusieurs effets peuvent limiter l'efficacité ou la propreté de la sélection :
\begin{itemize}
    \item La diffraction liée à la taille finie de l’objectif entraîne un flou de l’ordre de \(1{-}2\,\mu\)m au bord des zones éclairées.
    \item Une diffusion parasite par la puce peut se produire à forte intensité si tout le DMD est illuminé ; cela est évité en réduisant la taille transverse du faisceau à quelques micro-miroirs seulement.
    \item Des inhomogénéités d’éclairement dues à la gaussienne du faisceau et au speckle peuvent conduire à une sur-illumination de certaines zones. Un effort a été fait pour homogénéiser l’intensité en sortie de fibre.
    \item La réabsorption des photons diffusés pourrait entraîner un échauffement du gaz restant. Un désaccord en fréquence de 15 MHz a été testé pour éviter ce phénomène, sans effet visible sur la température du gaz.
\end{itemize}

\paragraph{Mesures de l’impact sur le gaz restant}

La température du gaz sélectionné est comparée avant et après sélection via l’analyse des fluctuations de densité après temps de vol. Aucun changement significatif de température ni d’élargissement n’a été observé. Ces résultats suggèrent que, dans les conditions expérimentales utilisées, la sélection ne perturbe pas significativement les atomes restants.




\subsection{Utilisation dans les protocoles}

{\color{blue}
\begin{itemize}
    \item Formes utilisées : boîtes, barrières, coupures.
    \item Préparation initiale contrôlée du gaz.
    \item Exemples de protocoles expérimentaux utilisant le DMD
\end{itemize}
}

\paragraph{Sélection locale et mesure de rapidité}

En sélectionnant une tranche du gaz, on peut ensuite couper le confinement longitudinal et laisser cette tranche s’étendre. Le profil de densité asymptotique obtenu après un long temps d’expansion est proportionnel à la distribution de rapidité locale du gaz initial. Ce protocole permet ainsi une mesure résolue de \(\rho(x,t \to \infty) \sim \rho(v)\).

\paragraph{Génération d’états hors équilibre}

La sélection permet également de créer des discontinuités dans le profil de densité, et donc d’initier une dynamique hors équilibre. Par exemple, on peut ne conserver que deux paquets séparés de gaz, qui vont alors osciller l’un vers l’autre. Cette configuration est analogue à un pendule de Newton quantique.

\paragraph{Formes utilisées}

Les motifs projetés par le DMD peuvent prendre différentes formes : boîtes, barrières, coupures, etc. Cette flexibilité rend l’outil extrêmement précieux pour explorer diverses configurations initiales et protocoles dynamiques.

\paragraph{Contrôle logiciel du DMD}

Le pilotage du DMD repose sur l’utilisation d’un module intégré fourni par Vialux (V7001-SuperSpeed), qui comprend les bibliothèques logicielles ALP-4. Plusieurs configurations du DMD peuvent être chargées en mémoire au début de chaque cycle expérimental, puis sélectionnées en cours de séquence à l’aide d’un signal digital. Le temps de commutation des miroirs est inférieur à \(30\,\mu\mathrm{s}\), ce qui est compatible avec les protocoles étudiés.

\paragraph{Partage du faisceau avec la voie d’imagerie}

Le faisceau utilisé pour la sélection spatiale est prélevé à partir du faisceau sonde déjà accordé sur la transition \(F=2 \rightarrow F'=3\) de la raie D2. Le partage est réalisé à l’aide d’un cube séparateur de polarisation placé en aval d’une lame demi-onde, permettant de contrôler la puissance injectée dans la fibre optique. Ce choix simplifie la mise en œuvre en évitant d’ajouter une source laser supplémentaire.

\paragraph{Blocage du faisceau de sélection}

Deux systèmes permettent de couper le faisceau de sélection pendant le cycle expérimental :
\begin{itemize}
    \item un cache mécanique (type électro-aimant), utilisé pour un blocage longue durée ;
    \item un modulateur acousto-optique (AOM), permettant de produire des impulsions brèves de quelques dizaines de \(\mu\mathrm{s}\), en amont du séparateur.
\end{itemize}
Pour garantir que le faisceau ne perturbe pas l’imagerie, le cache mécanique reste fermé pendant l’utilisation du faisceau sonde.

\paragraph{Montage optique de projection}

Le faisceau façonné par le DMD est projeté dans le plan des atomes à l’aide d’un système optique permettant de sélectionner l’ordre 0 de diffraction. L’ensemble des optiques est dimensionné (diamètre \(50\,\mathrm{mm}\)) pour limiter la diffraction. L’alignement est effectué en superposant le faisceau de sélection à la voie d’imagerie.

\paragraph{Grandissement et champ couvert}

Le montage permet de couvrir une zone de l’ordre de \(600\,\mu\mathrm{m}\) dans le plan des atomes, soit plus que la longueur typique d’un nuage (\(\sim 400\,\mu\mathrm{m}\) pour \(f_{\parallel}=5\,\mathrm{Hz}\)). Le grandissement est déterminé par les focales utilisées : une focale \(f_1 = 750\,\mathrm{mm}\) du côté du DMD, et \(f = 32\,\mathrm{mm}\) pour l’objectif côté atomes, donnant \(G = f/f_1 \approx 0.043\).

\paragraph{Visualisation et interface}

Le contrôle du DMD s’effectue via une interface graphique permettant de prévisualiser les configurations de miroirs. Une capture d’écran de cette interface est présentée dans la Fig.~\ref{fig:dmd_interface}, où la zone active réfléchie est visualisée en rouge. Cette interface est pilotée de manière automatisée pendant le déroulement de la séquence expérimentale.


\section{Techniques d’imagerie et d’analyse}
\subsection{Imagerie par absorption}
{\color{blue}
\begin{itemize}
    \item Imagerie \textit{in situ} et après temps de vol.
    \item Résolution, limites instrumentales.
\end{itemize}
}

\paragraph{Système d’imagerie par absorption}

L’imagerie est réalisée à l’aide d’une caméra CCD à déplétion profonde, optimisée pour une grande efficacité quantique à la longueur d’onde de 780 nm. On utilise des techniques d’imagerie par absorption permettant d’extraire la densité optique \( D(x, z) \), elle-même reliée à la densité atomique 3D via la loi de Beer-Lambert. Le profil de densité linéaire \( n(x) \) est obtenu par intégration sur les directions transverses.

\paragraph{Imagerie après temps de vol}

En appliquant un champ magnétique vertical (\( B = 8\,\mathrm{G} \)), la polarisation du faisceau peut être rendue circulaire (\( \sigma^+ \)) pour adresser la transition fermée \( |F=2, m_F=2\rangle \rightarrow |F'=3, m_F'=3\rangle \). Cette configuration assure une meilleure définition de la section efficace d’absorption. Un temps de vol de quelques ms est utilisé avant l’imagerie, permettant également de décomprimer le nuage.

\paragraph{Imagerie in situ}

Sans champ magnétique, les atomes sont imagés à $7~\mu m$ de la puce, ce qui implique une double absorption du faisceau incident et réfléchi. Dans ce cas, la transition n’est pas fermée, ce qui nécessite une calibration du facteur de conversion entre la densité mesurée et la densité réelle. Un ajustement linéaire permet de relier les profils in situ aux profils obtenus après temps de vol.

\paragraph{Choix des paramètres d’imagerie}

L’intensité du faisceau sonde est choisie typiquement à \( I_0/I_{\mathrm{sat}} \approx 0.3 \) pour optimiser le rapport signal sur bruit tout en restant dans une zone de linéarité acceptable. Dans ces conditions, le nombre de photons diffusés est de l’ordre de \( N_{\mathrm{sc}} \approx 230 \) et le rayon de diffusion reste comparable à la résolution du système d’imagerie (\( \sim 2.6\,\mu \mathrm{m} \)).

\paragraph{Limites du modèle de Beer-Lambert}

La validité de la loi de Beer-Lambert repose sur une approximation à une particule. Dans le cas des gaz fortement denses ou quasi 1D, les effets collectifs, les réabsorptions et les couplages dipolaires peuvent invalider ce modèle. Pour cette raison, même pour l’imagerie in situ, un temps de vol court (\( \sim 1\,\mathrm{ms} \)) est souvent appliqué afin de diluer le gaz transversalement.

\paragraph{Défauts et instabilités expérimentales}

Plusieurs limitations instrumentales ont été identifiées :
\begin{itemize}
    \item La caméra initialement utilisée montrait des motifs parasites aléatoires ainsi qu’un offset variant au cours du temps. Le remplacement de la caméra a permis de résoudre ces problèmes.
    \item Des franges d’interférences apparaissaient lors de la division des images d’absorption, probablement dues à des effets Fabry-Pérot dans les optiques. Le désaxage du faisceau d’imagerie a permis d’en limiter l’impact.
    \item Des photons résiduels, même en l’absence de faisceau sonde, ont été détectés. Ces derniers proviennent vraisemblablement de diffusions multiples dans le système optique.
\end{itemize}

\paragraph{Conclusion}

La combinaison de l’imagerie in situ et après temps de vol, ainsi qu’une calibration soigneuse des paramètres optiques et expérimentaux, permettent d’accéder à des profils de densité fiables malgré les limites intrinsèques du système d’imagerie. Une attention particulière a été portée à la réduction des artefacts expérimentaux afin de garantir la précision des mesures.


\subsection{Analyse des profils}

{\color{blue}
\begin{itemize}
    \item Extraction des densités, tailles, températures.
    \item Distribution longitudinale.
    \item Estimation de la température par ajustement Yang-Yang (optionnel si pertinent).
\end{itemize}
}


\section{Expériences et protocoles étudiés}
Cette section peut être la plus personnelle, en précisant ton rôle à chaque fois.
\subsection{Expansion longitudinale}
\begin{itemize}
    \item Protocole d’expansion (libération longitudinale, maintien du confinement transverse).
    \item Suivi de l’évolution du profil.
    \item Analyse à différents temps d’expansion
    \item Comparaison aux modèles analytiques : solutions homothétiques, GP, asymptotiques.
\end{itemize}

\subsection{Motivation et protocole expérimental d’expansion longitudinale}

\paragraph{Motivation.}
Une partie essentielle de mon travail de thèse a consisté à sonder la distribution de rapidités résolue spatialement, ce qui constitue une information clé pour comprendre la dynamique hors équilibre d’un gaz quantique unidimensionnel. Pour accéder à cette observable, il est nécessaire de réaliser un protocole qui relie la distribution de rapidités à des profils de densité mesurables expérimentalement. L’expansion longitudinale dans le guide 1D s’impose alors comme un outil naturel : en laissant le nuage se dilater librement dans la direction longitudinale, on convertit en partie l’information contenue dans les phases et les excitations collectives du système en une dynamique de densité directement accessible par imagerie. Ce protocole permet ainsi de comparer les prédictions issues des équations effectives, comme l’équation de Gross–Pitaevskii dans différents régimes de confinement, avec des mesures expérimentales résolues spatialement.

\paragraph{Considérations physiques.}
Au-delà de son intérêt pratique, l’expansion longitudinale offre une fenêtre unique sur la physique des gaz bosoniques 1D. Elle permet d’étudier comment un système initialement confiné évolue vers un état dilué, révélant à la fois l’impact du régime transverse (TF 3D vs TF 1D) et l’influence des fluctuations de phase. Dans le régime TF 1D, ces fluctuations deviennent dominantes et se traduisent par des ondulations de densité mesurables. Leur analyse expérimentale, via le spectre de puissance, fournit un accès direct aux corrélations de phase et à la thermodynamique effective du gaz.

\paragraph{Protocole expérimental.}
Concrètement, l’expansion longitudinale est réalisée selon la séquence illustrée en Fig.~??? :
\begin{itemize}
  \item Le nuage est initialement piégé dans un potentiel magnétique caractérisé par une fréquence longitudinale $f_{\parallel} = 5.0$ ou $9.4\,\mathrm{Hz}$ selon les jeux de données, et une fréquence transverse $f_{\perp} = 2.56\,\mathrm{kHz}$.
  \item À $t=0$, le confinement longitudinal est éteint en annulant les courants $I_D=I_{D'}=0$. La coupure est réalisée sur un temps fini $t_{\parallel} = 70\,\mu\mathrm{s} \ll 1/f_{\parallel}$, ce qui évite un pic de courant parasite tout en préservant la dynamique du gaz.
  \item Le nuage se dilate librement dans la direction longitudinale pendant une durée $\tau$. Ensuite, le confinement transverse est relâché en annulant $I_{\perp}$, avec un temps de coupure $t_{\perp} = 5\,\mu\mathrm{s} \ll 1/f_{\perp}$.
  \item Une image par absorption est enfin prise après un temps de vol $t_v$. Pour l’étude des profils de densité, on utilise typiquement $t_v = 1\,\mathrm{ms}$.
\end{itemize}

%\paragraph{Découplage des confinements.}
%Comme discuté en Section~\ref{chap:...}, l’architecture expérimentale rend ce protocole particulièrement simple à mettre en œuvre. La modulation des courants transverses $I_{\perp}$ garantit que le potentiel longitudinal est découplé de celui transverse, ce qui permet un contrôle précis et indépendant des deux confinements.

\paragraph{Perspective.}
La mise en œuvre de ce protocole d’expansion longitudinale ne répond donc pas seulement à un besoin technique de mesure, mais s’inscrit dans une stratégie plus générale : relier les prédictions théoriques de la GHD et des modèles effectifs à des observables accessibles, et sonder directement l’évolution des fluctuations et des corrélations dans un système quantique 1D.

\paragraph{Équations Gross-Pitaevskii dépendantes du temps.}
La dynamique du système étudié est décrite par l’équation de Gross-Pitaevskii (GP) \eqref{chap.1:eq.GP.1} :
\begin{eqnarray*}
	i \partial_\tau\phi = \left \{ - \frac{1}{2}\Delta_{\vec{r}} + V(\vec{r}) + g_{\mathrm{3D}} N \vert \phi \vert^2 \right \} \phi,
\end{eqnarray*}
avec $g_{\mathrm{3D}} = 4 \pi a_{\mathrm{3D}}$ et en présence d’un potentiel externe (voir \eqref{} et \eqref{}) :
\begin{eqnarray*}
	V(\vec{r}) = V_\perp(\vec{r}_\perp) + V_\parallel(x), 
	\qquad 
	V_\perp(\vec{r}_\perp) = \tfrac{1}{2} \, \omega_\perp^2 \, \vec{r}_\perp^2, 
	\qquad 
	V_\parallel(x) = \tfrac{1}{2} \, \omega_\parallel^2 \, x^2.
\end{eqnarray*} 


\paragraph{Séparation des degrés de liberté.}
Dans un piège de type cigare, caractérisé par $\omega_\perp \gg \omega_\parallel$, la dynamique transverse se déroule sur des temps caractéristiques beaucoup plus courts que la dynamique longitudinale. On fait alors l’hypothèse d’un \emph{suivi adiabatique transverse} : l’état reste en permanence dans son état fondamental transverse. Ainsi, les degrés de liberté transverses et longitudinaux se découplent et la fonction d’onde peut se factoriser sous la forme
\begin{equation}
    \phi(r,\tau) = \psi(x,\tau)\,\Phi\!\left(\vec{r}_\perp, n(x,\tau)\right),
\end{equation}
où $\psi(x,\tau)$ décrit la dynamique longitudinale et $\Phi$ est la fonction d’onde transverse dépendant paramétriquement de la densité linéaire $n(x,\tau)$. La condition de normalisation 
\(
\int d \vec{r}_\perp \, \big|\Phi\!\left(\vec{r}_\perp, n\right)\big|^2 = 1
\)
permet de réécrire la densité linéaire définie par 
\(
n \doteq N \int d \vec{r}_\perp \, |\phi|^2
\)
sous la forme
\begin{eqnarray*}
	n(x,\tau) = N \, |\psi(x,\tau)|^2.
\end{eqnarray*}
L’équation de Gross-Pitaevskii se réécrit alors
\begin{eqnarray}
	\left( i \partial_\tau + \tfrac{1}{2} \partial_x^2 - V_\parallel(x) - \mu(n) \right) \psi = 0, 
	\qquad 
	\mu(n)\,\Phi = \left( - \tfrac{1}{2} \Delta_{\vec{r}_\perp} + V_\perp + g_{\mathrm{3D}} \, n \, \big|\Phi(\vec{r}_\perp,n)\big|^2 \right)\Phi.
\end{eqnarray}

\paragraph{Équations hydrodynamiques.}
En utilisant la transformation de Madelung 
\(
\psi(x,\tau) = \sqrt{n(x,\tau)} \, e^{i \vartheta(x,\tau)},
\)
et en introduisant la vitesse $u = \partial_x \vartheta$, on obtient les équations hydrodynamiques associées :
\begin{eqnarray}\label{chap:5:eq.hydro.1}
	\left\{
	\begin{array}{rcl}
		\partial_\tau n + \partial_x ( n u )	 & = & 0, \\[0.3em]
		\partial_\tau u + \partial_x \left( \tfrac{u^2}{2} + V_\parallel(x) + \mu(n) + Q(n) \right) & = & 0,
	\end{array} 
	\right.
\end{eqnarray}
où le terme de pression quantique est donné par
\(
Q(n) = - \frac{1}{2} \, \frac{\partial_x^2 \sqrt{n}}{\sqrt{n}}.
\)
Ces équations sont équivalentes aux deux premières de \eqref{chap:3:eq:hydro.1}, en tenant compte de la relation thermodynamique $dP = n \, d\mu$ et en négligeant le terme de pression quantique $Q(n)$.

\medskip

Pour notre protocole, pour $\tau < 0$ le système est à l’équilibre, avec la condition
\(
\mu(n) + V_\parallel(x) = \mu\bigl(n(x=0)\bigr).
\)
Pour $\tau \geq 0$, le potentiel longitudinal est éteint : $V_\parallel(x) = 0$.

\medskip

\paragraph{Solutions analytiques homothétique.}
Si $n$ est solution des equation hydrodynamique \eqref{chap:5:eq.hydro.1} , pour $\tau \geq 0$. On fais l'hypothèse que la densité linéaire suit une forme homothétique
\begin{eqnarray}
	n(x,\tau) = \frac{1}{\lambda(\tau)} n_0 \left ( \frac{x}{\lambda(\tau)} \right ) ,	
\end{eqnarray}
avec $n_0$ le profil de densité à $\tau = 0 $ et $\lambda(\tau)$ le facter d'echelle à une temps d'expension $\tau$. Avec les containtes $\lambda(0) = 1$ et $\lambda'(0) = 0$ et $N = \int dx \, n(x , \tau ) $. En injectant dans \eqref{chap:5:eq.hydro.1} il vient que 
\begin{eqnarray}\label{chap:5:eq.hydro.2}
	\left\{
	\begin{array}{rcl}
		u(x, \tau ) & = & \displaystyle \frac{\dot\lambda(\tau)}{\lambda(\tau)} x , \\[0.3em]
		\partial_x \mu ( n ( x , \tau ))  & = & - \displaystyle \frac{\ddot\lambda(\tau)}{\lambda(\tau)} x,
	\end{array} 
	\right.
\end{eqnarray}
(car \(\partial_\tau u=(\ddot\lambda/\lambda-\dot\lambda^2/\lambda^2)x\) et \(v\partial_x v=(\dot\lambda/\lambda)^2 x\), leur somme donne \((\ddot\lambda/\lambda)x\)) et initialement $\mu( n_0 ( x ) ) = \mu( n_0 ( x = 0  ) ) - \frac{1}{2} \omega_\parallel^2 x^2 $.

\medskip

Calculons maintenant \(\partial_x\mu(n(x))\). D'abord
\[
\partial_x n(x)=\frac{1}{\lambda^2}\,n_0'\!\Big(\frac{x}{\lambda}\Big).
\]
À l'équilibre \(\mu\big(n_0(y)\big)=\mu_0-\tfrac12 \omega_\parallel^2 y^2\), d'où
\[
\mu'(n_0(y))\,n_0'(y)=-\omega_\parallel^2 y
\quad\Rightarrow\quad
n_0'(y)=-\frac{\omega_\parallel^2\,y}{\mu'(n_0(y))}.
\]
En prenant \(y=x/\lambda\) on obtient
\[
n_0'\!\Big(\frac{x}{\lambda}\Big)
= -\frac{\omega_\parallel^2}{\lambda}\,\frac{x}{\mu'\big(n_0(x/\lambda)\big)}.
\]
Donc
\[
\partial_x n(x) = -\frac{\omega_\parallel^2\,x}{\lambda^3}\;
\frac{1}{\mu'\big(n_0(x/\lambda)\big)}.
\]
Puis
\[
\partial_x\mu(n(x))=\mu'\big(n(x)\big)\,\partial_x n(x)
= -\frac{m\omega_\parallel^2\,x}{\lambda^3}\;
\frac{\mu'\big(n(x)\big)}{\mu'\big(n_0(x/\lambda)\big)}.
\]
Or \(n_0(x/\lambda)=\lambda\,n(x)\), donc on définit
\[
f(\lambda)\equiv\frac{\mu'(n)}{\mu'(\lambda n)}.
\]
On obtient finalement
\[
\partial_x\mu(n(x)) = -\frac{\omega_\parallel^2}{\lambda^3}\,f(\lambda)\,x.
\]

%On souhaite calculer $\partial_x \mu\bigl(n(x,\tau)\bigr)$ en utilisant la règle de la chaîne et la forme homothétique. On a
%\[
%	\partial_x \mu\bigl(n(x)\bigr) 
%	= \mu'(n(x)) \, \partial_x n(x) 
%	= \frac{1}{\lambda^2} \, \mu'(n(x)) \, \partial_x n_0\!\left(\tfrac{x}{\lambda}\right),
%\]
%où le dernier terme s’écrit
%\[
%	\left. \frac{\partial n_0}{\partial x} \right|_{x/\lambda} 
%	= \left. \frac{\partial n_0}{\partial \mu} \right|_{\mu(n_0(x/\lambda))} 
%	\left. \frac{\partial \mu}{\partial x} \right|_{n_0(x/\lambda)} .
%\]
%On utilise alors 
%\[
%	\left. \frac{\partial \mu}{\partial x} \right|_{n_0(x/\lambda)} = - \frac{\omega_\parallel^2}{\lambda^2} \, x,
%	\qquad 
%	\left. \frac{\partial n_0}{\partial \mu} \right|_{\mu(n_0(x/\lambda))} 
%	= \left. \frac{\partial n}{\partial \mu} \right|_{\mu(\lambda n)} .
%\]
%Il vient donc, en utilisant de plus la deuxième équation de 
En remplaçant dans la deuxième d'Euler \eqref{chap:5:eq.hydro.2} et en simplifiant  \(x\),
\begin{eqnarray}\label{chap:5:eq.hydro.3}
	\frac{\ddot\lambda}{\lambda}  
	 =  \frac{\omega_\parallel^2}{\lambda^3} \, f(\lambda)  .
\end{eqnarray}

\paragraph{Proposition.}
Si le facteur
\(
f(\lambda)
\)
est bien défini indépendamment de \(n>0\) (ce qui est le cas pour les solutions homothétiques),
alors \(f\) est une loi de puissance.

\paragraph{Preuve.}
Posons \(g(n) = \mu'(n)>0\) ou \(<0\) (\ie $\mu$ strictement monotone). La définition de \(f\) équivaut à l’existence d’une fonction
\(\chi(\lambda)=1/f(\lambda)\) telle que
\[
g(\lambda n)=\chi(\lambda)\,g(n)\qquad(\forall\,\lambda,n>0).
\]
En prenant \(n=1\), on a \(\chi(\lambda)=g(\lambda)/g(1)\).
Donc, pour tous \(a,b>0\),
\[
\chi(ab)=\frac{g(ab)}{g(1)}=\frac{\chi(a)\,g(b)}{g(1)}=g(a)\,g(b),
\]
c’est-à-dire que \(\chi\) est \emph{multiplicative}. Sous une hypothèse physique très faible
(continuité/mesurabilité ou simple localement bornée), toute fonction multiplicative sur
\(\mathbb{R}_+^\ast\) est de la forme
\[
\chi(\lambda)=\lambda^{\alpha-1}
\quad\Rightarrow\quad
f(\lambda)=\lambda^{1-\alpha}.
\]
%En réintégrant \(g=\mu'\propto n^{\alpha-1}\), on retrouve \(\mu(n)\propto n^\alpha\) pour \(\alpha\neq 0\)
%(et \(\mu(n)\propto \ln n\) pour \(\alpha=0\)).
\qed


% --- Démonstration que f(λ)=λ^{1-\alpha} et réciproque ---
\paragraph{Proposition.}
$f(\lambda) = \lambda^{1-\alpha}$ et $\mu (n) \propto n^\alpha $ sont equivalents.
%
%On définit
%\[
%f(\lambda)=\frac{\mu'(n)}{\mu'(\lambda n)},
%\]
%en supposant \(\mu\in C^1\) et \(\mu'(n)>0\) pour \(n>0\).

\paragraph{1. Si \(\mu(n)=C\,n^\alpha\) (avec \(C\neq0\)) :}
Alors \(\mu'(n)=C\alpha\,n^{\alpha-1}\). Par conséquent
\[
f(\lambda)=\frac{C\alpha\,n^{\alpha-1}}{C\alpha\,(\lambda n)^{\alpha-1}}
=\lambda^{1-\alpha}.
\]

\paragraph{2. Réciproque : si \(f(\lambda)=\lambda^{1-\alpha}\) pour tout \(\lambda>0\) (et tout \(n>0\)) :}
Posons \(g(n)=\mu'(n)\). L'hypothèse s'écrit
\[
\frac{g(n)}{g(\lambda n)}=\lambda^{1-\alpha}
\quad\Longleftrightarrow\quad
g(\lambda n)=\lambda^{\alpha-1}\,g(n),
\]
pour tout \(n>0\) et tout \(\lambda>0\).

Fixons \(n_0>0\) et définissons \(\varphi(\lambda)\equiv g(\lambda n_0)\). La relation ci-dessus donne
\[
\varphi(\lambda)=\lambda^{\alpha-1}\,\varphi(1).
\]
Autrement dit \(\varphi(\lambda)=C_1\,\lambda^{\alpha-1}\) pour une constante \(C_1=\varphi(1)=g(n_0)\). En remplaçant \(\lambda=x/n_0\) on obtient pour tout \(x>0\)
\[
g(x)=C_1\,x^{\alpha-1}.
\]
Ainsi \(g(n)=\mu'(n)=C\,n^{\alpha-1}\) avec \(C\) constant.

En intégrant (en supposant \(\alpha\neq 0\)),
%\[
%\mu(n)=\int \mu'(n)\,dn = \int C\,n^{\alpha-1}\,dn = \frac{C}{\alpha}\,n^\alpha + \text{const},
%\]
%donc 
\(\mu(n)\propto n^\alpha\). (Pour \(\alpha=0\) on obtient \(\mu'(n)=C\,n^{-1}\) et \(\mu(n)=C\ln n+\text{const}\).)

\paragraph{Remarque sur les hypothèses.}
La démonstration utilise la propriété fonctionnelle multiplicative
\(g(\lambda n)=\lambda^{\alpha-1}g(n)\). Sous une hypothèse faible de continuité (ou dérivabilité) en \(n\) cette équation force la forme de puissance \(g(n)\propto n^{\alpha-1}\). Sans régularité, des solutions pathologiques peuvent exister mais ne sont pas physiquement pertinentes dans le contexte thermodynamique.

\qed

%% --- Bref argument (à insérer) ---
%En posant \(g(n)=\mu'(n)\) et \(f(\lambda)=\dfrac{g(n)}{g(\lambda n)}=\lambda^{1-\alpha}\), on obtient
%\[
%g(\lambda n)=\lambda^{\alpha-1}g(n)\quad\forall\,\lambda,n>0,
%\]
%d'où \(g(n)=C\,n^{\alpha-1}\) et, pour \(\alpha\neq0\), \(\mu(n)=\dfrac{C}{\alpha}n^\alpha+\mathrm{const}\), i.e. \(\mu\propto n^\alpha\).
%
%
%% --- Encadré pour le cas alpha = 0 ---
%\medskip
%\noindent\textbf{Remarque (cas \(\alpha=0\)).} Si \(\alpha=0\) alors la relation fonctionnelle donne \(g(n)=\mu'(n)=C\,n^{-1}\). En intégrant on obtient
%\[
%\mu(n)=C\ln n + \mathrm{const}.
%\]
%Ce cas correspond physiquement, par exemple, au gaz isotherme idéal en 1D (ou plus généralement à une dépendance logarithmique du potentiel chimique), où la compressibilité \(\mu'(n)\propto 1/n\).
%
%--------------------------------------------
%
%avec $f(\lambda) = \frac{\mu'(n)}{\mu'(\lambda n )}$ avec  $\mu(n)$ est continue et strictement monitone (donc inversible). Puisque $f(1)= 1$ et $f(\lambda_1 \lambda_2) =  f(\lambda_1) f( \lambda_2)$ alors $f$ est une fonction de puissace $f(\lambda) = \lambda^{1-\beta}$. Ainssi une solution de l'éqtation hydrondynamique homothètique donne  
%\begin{eqnarray}
%	\mu(n) = a n^\beta + b,  	
%\end{eqnarray}
%avec $a, b$ et $\beta$ des réelles. 

\paragraph{Cas particulier.}
Dans le régime quasi-1D on utilise l'expression d'interpolation (cf. Salasnich et al.)
\[
\mu(n)=\hbar\omega_\perp\Big(\sqrt{1+4\,a_{\mathrm{3D}}\,n}-1\Big),
\]
où \(n\) est la densité linéique et \(a_{\mathrm{3D}}\) le scattering length. De cette formule on obtient deux limites asymptotiques :

\begin{itemize}
\item \emph{Régime transverse Thomas--Fermi (TF), \(4a_{\mathrm{3D}}n\gg1\).} 
Alors \(\sqrt{1+4a_{\mathrm{3D}}n}\simeq 2\sqrt{a_{\mathrm{3D}}n}\) et
\[
\mu(n)\simeq 2\hbar\omega_\perp\sqrt{a_{\mathrm{3D}}\,n},
\]
ce qui correspond à \(\mu\propto n^{1/2}\) (donc \(\alpha=\tfrac12\)). Ce régime décrit la situation où \(\mu\gg\hbar\omega_\perp\) et de nombreux niveaux transverses sont excités.
\item \emph{Régime quasi-1D (transverse fondamental), \(4a_{\mathrm{3D}}n\ll1\).} 
Alors \(\sqrt{1+4a_{\mathrm{3D}}n}\simeq 1+2a_{\mathrm{3D}}n\) et
\[
\mu(n)\simeq 2\hbar\omega_\perp\,a_{\mathrm{3D}}\,n \equiv g\,n,
\]
avec \(g=2\hbar\omega_\perp a_{\mathrm{3D}}\). Ici \(\mu\propto n\) (donc \(\alpha=1\)) ; on est proche de l'état fondamental transverse (gaussien).
\end{itemize}

Les deux formes ci-dessus sont bien les limites asymptotiques de l'expression d'interpolation donnée plus haut.

\medskip

Enfin, l'équation d'évolution du facteur d'échelle obtenue précédemment s'écrit correctement
\[
\boxed{\qquad \ddot\lambda\,\lambda^{\alpha+1}=\omega_\parallel^2 \qquad}
\]

% --- Première intégration de l'équation ---
On part de
\[
\ddot\lambda\,\lambda^{\alpha+1}=\omega_\parallel^2,
\]
et on pose \(v=\dot\lambda\). Comme \(\ddot\lambda=\dot\lambda\frac{d\dot\lambda}{d\lambda}\), on obtient
\[
\dot\lambda\frac{d\dot\lambda}{d\lambda}=\omega_\parallel^2\,\lambda^{-(\alpha+1)}.
\]

\paragraph{Cas \(\alpha\neq0\).}
Intégration par rapport à \(\lambda\) :
\[
\frac{1}{2}\dot\lambda^2
= \omega_\parallel^2\int \lambda^{-(\alpha+1)}\,d\lambda
= -\frac{\omega_\parallel^2}{\alpha}\,\lambda^{-\alpha} + C,
\]
où \(C\) est une constante d'intégration déterminée par les conditions initiales \(\lambda(0)=\lambda_0\), \(\dot\lambda(0)=\dot\lambda_0\) :
\[
C=\frac{1}{2}\dot\lambda_0^2+\frac{\omega_\parallel^2}{\alpha}\,\lambda_0^{-\alpha}.
\]
On a donc la première intégrale
\[
\boxed{\; \dot\lambda^2
= \dot\lambda_0^2 + \frac{2\omega_\parallel^2}{\alpha}\big(\lambda_0^{-\alpha}-\lambda^{-\alpha}\big)\; }.
\]
%La solution en quadrature s'écrit alors
%\[
%t-t_0=\int_{\lambda_0}^{\lambda(t)}\frac{d\lambda}{\sqrt{\,v_0^2 + \dfrac{2\omega_\parallel^2}{\alpha}\big(\lambda_0^{-\alpha}-\lambda^{-\alpha}\big)\,}}.
%\]

\paragraph{Cas \(\alpha=0\).}
L'équation devient \(\ddot\lambda\,\lambda=\omega_\parallel^2\). On obtient
%\[
%\frac{1}{2}v^2=\omega_\parallel^2\ln\lambda + C,
%\]
%avec \(C=\tfrac12 v_0^2-\omega_\parallel^2\ln\lambda_0\). D'où
\[
\boxed{\; \dot\lambda^2 = \dot\lambda_0^2 + 2\omega_\parallel^2\ln\!\big(\tfrac{\lambda}{\lambda_0}\big)\; }.
\]

%La quadrature est
%\[
%t-t_0=\int_{\lambda_0}^{\lambda(t)}\frac{d\lambda}{\sqrt{\,v_0^2 + 2\omega_\parallel^2\ln(\lambda/\lambda_0)\,}}.
%\]

%\paragraph{Remarques.}
%\begin{itemize}
%\item Ces intégrales donnent la solution implicite \(t(\lambda)\). En général on ne dispose pas d'une primitive élémentaire fermée pour \(\lambda(t)\) (sauf cas particuliers de choix des conditions initiales), mais la première intégrale ci-dessus est très utile pour l'analyse qualitative (points de retournement, énergie effective, petites oscillations).
%\item Pour les petites oscillations autour de \(\lambda=1\) on peut linéariser et retrouver la fréquence \(\omega_{\rm breath}=\sqrt{4+\beta}\,\omega_\parallel=\sqrt{3+\alpha}\,\omega_\parallel\) (avec \(\beta=\alpha-1\)).
%\end{itemize}


%\subsubsection{Comportement asymptotique du facteur d'échelle}



%\paragraph{Régime à temps longs.} 
%On considère la condition initiale
%\(
%\lambda(0)=1,\,\dot\lambda(0)=0.
%\) 
%et pour $\alpha > 0$ 
%Pour \(\tau\) très grand, on a \(\lambda^{-\alpha}\ll 1\). On a 
%\[
%\lambda(\tau) \simeq \frac{2}{\alpha}\,\omega_\parallel \tau.
%\]
%En particulier :  
%\begin{itemize}
%\item TF 1D (\(\alpha=1\)) : \(\lambda(\tau)\simeq \sqrt{2}\,\omega_\parallel \tau\),  
%\item TF 3D (\(\alpha=1/2\)) : \(\lambda(\tau)\simeq 2\,\omega_\parallel \tau\).  
%\end{itemize}
%%Ces comportements sont observés sur la Fig.~7.4(b).
%
%\paragraph{Régime à temps courts.}  
%À temps courts, on peut approximer \(\mu(x,\tau)\simeq \mu_0(x)=\mu_p-\frac{1}{2}m\omega_\parallel^2 x^2\). Le comportement initial est alors indépendant de l'équation d'état \(\mu(n)\). L'équation d'Euler sans potentiel extérieur donne
%\[
%\frac{d^2 x}{d\tau^2} \simeq \omega_\parallel^2 x \quad\Rightarrow\quad v(x,\tau)\simeq \omega_\parallel^2 x \,\tau \quad (\tau\to 0).
%\]
%En réinjectant ce profil dans l'équation de continuité et en intégrant, on obtient
%\[
%\frac{1}{\lambda(\tau)} \simeq 1 - \frac{\omega_\parallel^2 \tau^2}{2} + \mathcal{O}(\tau^4) \quad\Rightarrow\quad
%\lambda(\tau)\simeq 1 + \frac{\omega_\parallel^2 \tau^2}{2} + \mathcal{O}(\tau^4),
%\]
%ce qui correspond au comportement observé à temps courts sur la Fig.~7.4(a), identique pour les régimes TF 1D et TF 3D.
%
%
%------------------
%
%\paragraph{Régime à temps courts.}  
%Pour \(\tau \to 0\), on linéarise le facteur d'échelle autour de l'équilibre \(\lambda=1\) en posant
%\(\lambda(\tau) = 1 + \epsilon(\tau)\) avec \(|\epsilon|\ll 1\). L'équation de mouvement devient alors
%\[
%\ddot \epsilon + (1+\alpha)\,\omega_\parallel^2 \epsilon - \omega_\parallel^2 = 0,
%\]
%équivalente à un oscillateur harmonique forcé. La solution pour des conditions initiales
%\(\epsilon(0)=0\), \(\dot\epsilon(0)=0\) est
%\[
%\epsilon(\tau) \simeq \frac{\omega_\parallel^2}{1+\alpha}\left[1 - \cos\left(\sqrt{1+\alpha}\,\omega_\parallel \tau\right)\right].
%\]
%Ainsi, à temps très courts \(\tau\ll 1/\omega_\parallel\), on retrouve
%\[
%\epsilon(\tau) \simeq \frac{1}{2}\,\omega_\parallel^2 \tau^2 + \mathcal{O}(\tau^4),
%\]
%et donc
%\[
%\lambda(\tau) \simeq 1 + \frac{\omega_\parallel^2 \tau^2}{2} + \mathcal{O}(\tau^4),
%\]
%ce qui coïncide avec le comportement universel observé à temps courts pour tous les régimes TF, indépendamment de \(\alpha\) et de l'équation d'état \(\mu(n)\).
%
%----------------------------

On impose les conditions initiales
\[
\lambda(0)=1, \qquad \dot\lambda(0)=0,
\]
et l'on considère le cas \(\alpha>0\).

\paragraph{Régime à temps courts (\(\tau \ll 1/\omega_\parallel\)).}  
On linéarise autour de l'équilibre \(\lambda=1\) en posant \(\lambda(\tau)=1+\epsilon(\tau)\) avec \(|\epsilon|\ll 1\). L'équation de mouvement devient un oscillateur harmonique forcé :
\[
\ddot \epsilon + (1+\alpha)\,\omega_\parallel^2 \epsilon - \omega_\parallel^2 = 0.
\]
Pour les conditions initiales choisies, la solution à petits temps est
\[
\epsilon(\tau) \simeq \frac{1}{2}\,\omega_\parallel^2 \tau^2 \quad\Rightarrow\quad
\lambda(\tau) \simeq 1 + \frac{\omega_\parallel^2 \tau^2}{2},
\]
indépendamment de l'équation d'état \(\mu(n)\). Ce comportement correspond au profil universel observé à temps courts (Fig.~7.4(a)) pour tous les régimes TF.

\paragraph{Régime à temps longs (\(\tau \gg 1/\omega_\parallel\)).}  
Pour \(\lambda^{-\alpha}\ll 1\), l'équation intégrée donne
\[
\dot\lambda \simeq \sqrt{\frac{2\omega_\parallel^2}{\alpha}} \quad\Rightarrow\quad
\lambda(\tau) \simeq \frac{2}{\alpha}\,\omega_\parallel \tau.
\]
En particulier :
\begin{itemize}
\item TF 1D (\(\alpha=1\)) : \(\lambda(\tau)\simeq \sqrt{2}\,\omega_\parallel \tau\),  
\item TF 3D (\(\alpha=1/2\)) : \(\lambda(\tau)\simeq 2\,\omega_\parallel \tau\).  
\end{itemize}
Ces comportements sont bien observés sur la Fig.~7.4(b) et correspondent à l’expansion asymptotique du gaz.





% --- Table révisée (petites corrections typographiques) ---
\begin{table}[h]
\centering
\begin{tabular}{l c c c}
\hline
Système & loi pour $\mu(n)$ & $\beta$ (avec $f(\lambda)=\lambda^{-\beta}$) & $\displaystyle \omega_{\rm breath}/\omega_\parallel$ \\
\hline
Gaz classique isotherme (1D, $\mu\propto\ln n$) 
& $\mu'(n)\propto 1/n$ 
& $-1$ 
& $\sqrt{3}\approx1.732$ \\[4pt]

Gaz de Bose 1D en régime moyen (GP, $\mu\propto n$) 
& $\alpha=1$ 
& $0$ 
& $2$ \\[4pt]

Tonks--Girardeau (1D, $\mu\propto n^2$) 
& $\alpha=2$ 
& $1$ 
& $\sqrt{5}\approx2.236$ \\[4pt]

Gaz de Fermi unitaire (ex. 3D, $\mu\propto n^{2/3}$) 
& $\alpha=\tfrac{2}{3}$ 
& $-\tfrac{1}{3}$ 
& $\sqrt{3+\tfrac{2}{3}}\approx1.915$ \\[4pt]

Cas général (loi de puissance) 
& $\mu\propto n^\alpha$ 
& $\beta=\alpha-1$ 
& $\displaystyle \sqrt{3+\alpha}$ \\
\hline
\end{tabular}
\caption{Valeurs de $\beta$ et fréquences du mode de souffle pour quelques régimes usuels.}
\label{tab:breathing}
\end{table}
 



\subsection{Sonde locale de distribution de rapidité}
\begin{itemize}
    \item Principe de la mesure : coupure d’une tranche puis expansion.
    \item Rôle du DMD dans la sélection.
    \item Accès à la distribution de vitesse locale.
    \item Comparaison avec les prédictions GHD.
    \item Limites et incertitudes
\end{itemize}

\subsubsection{Distribution de rapidités locale dans les gaz 1D}

\paragraph{Motivation.}  
La compréhension des gaz de bosons 1D avec interactions de contact répulsives repose sur la notion de distribution de rapidités \(\rho(\theta)\). Chaque état propre du système peut être paramétré par un ensemble de rapidités \(\{\theta_i\}\) (Ansatz de Bethe), ou interprété comme les vitesses de quasi-particules à durée de vie infinie. Ici, on utilise la définition pratique issue des expansions 1D : les rapidités correspondent aux vitesses asymptotiques des atomes après une expansion, avec \(x_j \simeq \tau \theta_j\) pour un temps \(\tau\) long. Cette définition est directement applicable à des mesures expérimentales de distribution de rapidités locales.

\paragraph{Distribution locale et LDA.}  
Pour un nuage atomique piégé dans un potentiel longitudinal variant lentement, on peut appliquer l’Approximation de Densité Locale (LDA). Le gaz est alors vu comme un fluide décomposé en cellules mésoscopiques de densité homogène et relaxée. Dans chaque cellule, l’état d’équilibre est décrit par un Ensemble de Gibbs Généralisé (GGE), ou équivalemment par une distribution de rapidités locale \(\rho(x,\theta)\). Cette description permet d’étudier non seulement l’équilibre, mais aussi la dynamique hors équilibre à grandes échelles spatiales et temporelles, via la théorie Hydrodynamique Généralisée (GHD).

\paragraph{Protocole expérimental.}  
Pour mesurer \(\rho(x,\theta)\) localement :  
\begin{enumerate}
    \item Une zone du nuage atomique de taille \(\ell\) centrée en \(x_0\) est sélectionnée à l’aide d’un dispositif de micromiroirs digitaux (DMD). La pression de radiation supprime instantanément les atomes en dehors de la zone, laissant uniquement ceux de la cellule.
    \item Après la sélection, le confinement longitudinal est relâché, tandis que le confinement transverse reste actif. Les atomes réalisent une expansion 1D pendant un temps \(\tau\), puis le profil de densité est imagé (typiquement pour \(\tau\sim 1\) ms).
    \item Le protocole est répété pour plusieurs positions \(x_0\), permettant d’obtenir la distribution de rapidités locale sur l’ensemble du nuage.
\end{enumerate}

\paragraph{Mesures à l’équilibre.}  
Pour un gaz initialement à l’équilibre dans un piège harmonique, le profil de densité de chaque zone sélectionnée est analysé via la thermodynamique Yang-Yang et la LDA, donnant température \(T_{\rm YY}\) et potentiel chimique \(\mu_{\rm YY}\). Après un temps d’expansion long, le profil devient homothétique à la distribution de rapidités locale \(\rho(x,\theta)\). La comparaison avec les prédictions numériques montre une bonne cohérence, confirmant que le protocole permet de sonder efficacement \(\rho(x,\theta)\).

\paragraph{Résumé.}  
— Une sonde locale de distribution de rapidités a été mise en place grâce au DMD.  
— Les atomes sélectionnés réalisent une expansion dans le guide 1D.  
— Après un temps long, le profil de densité reflète la distribution de rapidités locale.  
— Ce protocole a été appliqué avec succès sur un nuage atomique initialement à l’équilibre.


\section{Discussion sur les limites et les perspectives}
\begin{itemize}
    \item Contraintes techniques (bruit, alignement, stabilité de la puce…).
    \item Améliorations potentielles (résolution, contrôle du potentiel, automatisation).
    \item Perspectives pour d’autres types d’expériences (étude de chocs, turbulence quantique, etc.)
\end{itemize}

\section*{Conclusion}
\begin{itemize}
    \item Résumé de l’architecture du dispositif).
    \item Méthodes d’analyse utilisées et robustesse.
    \item Importance de l’expérience dans le contexte de l’étude des gaz quantiques unidimensionnels
\end{itemize}
Ce chapitre a présenté les éléments essentiels du dispositif expérimental, les méthodes d’imagerie, ainsi que les expériences auxquelles j’ai participé. L’ensemble constitue une plateforme performante pour l’étude de la dynamique de gaz 1D hors équilibre.

\paragraph{Résumé de l’architecture expérimentale}  
Nous avons décrit les éléments clés du dispositif utilisé : un système de refroidissement laser basé sur trois sources couplées, un piégeage magnétique sur puce optimisé pour réaliser des géométries unidimensionnelles, une plateforme de modulation de potentiel via un DMD, et un système d’imagerie haute résolution. L’ensemble permet une manipulation fine des nuages atomiques dans un cadre reproductible et stable.

\paragraph{Méthodes d’analyse et robustesse}  
L’imagerie par absorption, couplée à une analyse rigoureuse des profils atomiques, fournit des outils fiables pour extraire les grandeurs pertinentes : densités, tailles, températures, distributions de vitesses. Ces méthodes ont permis de confronter les résultats expérimentaux à des prédictions théoriques de type GHD ou Yang-Yang.

\paragraph{Importance du dispositif pour la thèse}  
Ce dispositif a été essentiel pour mener à bien les expériences présentées dans cette thèse. Il offre à la fois un contrôle local (grâce au DMD), un bon confinement transverse (grâce à la puce) et une imagerie précise. La plateforme est ainsi bien adaptée pour étudier des systèmes 1D fortement corrélés hors équilibre, et pour tester les prédictions de la physique statistique intégrable.

\paragraph{Perspectives}  
Malgré ses atouts, le dispositif présente des limitations techniques (rugosité magnétique, sensibilité à l’alignement, etc.) qui laissent entrevoir des pistes d’amélioration. Des développements futurs pourraient notamment viser à augmenter la résolution spatiale, automatiser davantage les séquences, ou explorer d'autres régimes dynamiques comme la turbulence ou les collisions de chocs quantiques.



%\appendix
\section*{Annexes}
\begin{itemize}
    \item Schémas techniques (puce, DMD, optique).
    \item Tableaux de paramètres expérimentaux.
    \item Exemples de motifs DMD utilisés.
\end{itemize}
\input{chapters/06_Bipart}
\input{chapters/07_Dipolaire}

%\chapter*{Conclusion}
\addcontentsline{toc}{chapter}{Conclusion}

Conclusion de la thèse.


%\appendix
%\chapter{Annexes}

Informations complémentaires.



\bibliographystyle{abbrv}
\bibliography{thesis}

%\printbibliography

\end{document}

%| Style     | Description                                                             |
%| --------- | ----------------------------------------------------------------------- |
%| `plain`   | Tri alphabétique, numérotation croissante                               |
%| `unsrt`   | Même que `plain` mais sans tri, respecte l’ordre d’apparition           |
%| `abbrv`   | Comme `plain` mais avec prénoms et noms abrégés                         |
%| `alpha`   | Les références sont étiquetées par une combinaison du nom et de l’année |
%| `apalike` | Style APA simplifié                                                     |
%| `ieeetr`  | Style IEEE, tri par ordre d’apparition                                  |
%| `siam`    | Style SIAM (mathématiques appliquées)                                   |
%| `acm`     | Style ACM (informatique)                                                |
%




\section{Thermodynamique de Bethe et relaxation}

%------------------------------------------------------------------
\subsection{Limite thermodynamique}

\paragraph{Observables locales dans la limite thermodynamique.}
%Lorsque l'observable $\operator{\mathcal{O}}$ est suffisamment local, on croit que la valeur d'attente $\langle  \{ \theta_a\}  \vert   \mathcal{O} \vert \{ \theta_a\} \rangle$ ne dépend pas de l'état microscopique spécifique du système, de sorte qu'elle devient une fonctionnelle de $\Pi$ dans la limite thermodynamique.
Dans la suite de ce chapitre, nous omettrons l’exposant $(\mathcal{S})$.
\vspace{0.2em}
Dans la base des états de Bethe \( \{ \ket{\{ \theta_a \}} \} \), l’opérateur \( \hat{\rho}(\theta) \) défini en \eqref{chap.2.rho.1} est diagonal, et agit comme un projecteur sur les valeurs de rapidité.

\vspace{0.5em}

Dans la limite thermodynamique, différentes configurations microscopiques \( \{ \theta_a \} \) peuvent correspondre à la même distribution de rapidité macroscopique \( \rho(\theta) \). Autrement dit, plusieurs états \( \ket{\{ \theta_a \}} \) partagent la même valeur propre \( \rho(\theta) \) de l’opérateur \( \operator{\rho}(\theta) \). Cela reflète une {\em dégénérescence macroscopique} induite par le passage à la limite thermodynamique (\( N, L \to \infty \) avec \( N/L \to \text{const} \)).

\vspace{0.5em}

Si l’observable $\mathcal{O}$ est suffisamment locale, sa valeur d’attente dans un état propre ne dépend pas des détails microscopiques, mais uniquement de la distribution de rapidité. On écrit alors :
\begin{eqnarray}
	\underset{\mbox{\tiny therm.}}{\lim} \braket{  \operator{\mathcal{O}} }_{\{ \theta_a\}}  & = & \langle \operator{\mathcal{O}}\rangle_{[\rho]},
\end{eqnarray}
où $\underset{\mbox{\tiny therm.}}{\lim}$ est la limite thermodynamique ($N,L \to \infty$ avec $N/L \to $ const) et où \( \langle \mathcal{O} \rangle_{[\rho]} \) désigne la valeur d’attente de \( \mathcal{O} \) dans un état macroscopique caractérisé par la distribution de rapidité \( \rho(\theta) \).


\medskip
Dans un ensemble général (GGE), la valeur moyenne de l’observable \eqref{chap.2.moyenne.1} devient alors :		
\begin{eqnarray}\label{chap.2.moyenne.2}
	\underset{\mbox{\tiny therm.}}{\lim} \langle \operator{\mathcal{O}} \rangle_{\operator{\varrho}[w]} & =  & \frac{  \displaystyle \sum_{\rho }  \langle \operator{\mathcal{O}}\rangle_{[\rho]} \Omega[\rho] e^{- \sum_{a = 1}^N  w(\theta_a)    }}{ \displaystyle \sum_{\rho}   \Omega[\rho]\,e^{- \sum_{a = 1}^N  w(\theta_a) } } ,
\end{eqnarray}
où $\sum_{\rho }$ est une somme sus tous les distribution de rapidité $\rho$ et 
où $\Omega[\rho]$ désigne le nombre de micro-états compatibles avec la distribution de rapidité $\rho$.

%où $\# \mbox{micro-états.}$ est les nombre de micro état associée àa la distribution de rapidité $\rho$.
%Avant de se plonger sur $\# \mbox{micro-états.}$, regardons le changement des équation de Bethes. 

\medskip
Pour établir la fonction $\Omega[\rho]$, reppelons-nons de la transformation des équations de Bethe dans dans la limite thermodynamique, hors état fondamentale \eqref{eq:TBA-nu} et \eqref{eq:TBA-rhos-2}.
\begin{equation}
	\nu = \frac{\rho}{\rho_s} \, , \qquad 2\pi \rho_s = 1^{\mathrm{dr}}_{[\nu]} 
\label{chap.2:eq:TBA-rhos}
\end{equation}
où $f^{\mathrm{dr}}_{[\nu]}$ est définie en \eqref{eq:dessing}.

\medskip

Cette formalisation constitue la brique de base de la \textbf{hydrodynamique généralisée} et, dans la section suivante, permet de définir rigoureusement l’\textbf{entropie de Yang–Yang}, indispensable pour décrire la relaxation hors d’équilibre des systèmes intégrables.

%\vspace{1ex}
%La formalisation ci‑dessus fournit la brique de base pour la
%\textbf{hydrodynamique généralisée} et, dans la section suivante, pour la
%définition précise de l’\textbf{entropie de Yang-Yang}
%assurant la relaxation des systèmes intégrables hors‑équilibre.

%% !TEX encoding = IsoLatin

%\documentclass[11pt,a4paper]{report}
%\documentclass[11pt,a4paper]{book}
\documentclass[10pt, titlepage]{book} % Taille de base des caractères (12pt recommandée pour lecture)


% -------------------------------------
% Encodage et langue
% -------------------------------------
\usepackage[utf8]{inputenc}
\usepackage[T1]{fontenc}
\usepackage[french]{babel}

% -------------------------------------
% Marges et dimensions
% -------------------------------------
\usepackage[a4paper, top=1.0cm, bottom=1.0cm, left=1cm, right=1cm]{geometry} 
% Ajuste ici les marges selon tes préférences

% -------------------------------------
% Interligne
% -------------------------------------
\usepackage{setspace}
%\onehalfspacing  % Interligne 1.5 
%\doublespacing %(utilise \doublespacing pour double interligne)

% -------------------------------------
% Police (facultatif)
% -------------------------------------
%\usepackage{mathptmx} % Police Times (ancienne)
%\usepackage{libertine} % Police élégante

\usepackage{newtxtext,newtxmath} % Times moderne pour texte et maths

% -------------------------------------
% Paquets utiles
% -------------------------------------
\let\Bbbk\relax
\let\openbox\relax
\usepackage{amsmath, amssymb, amsthm}
\usepackage{graphicx}
\usepackage{hyperref}
\usepackage{xcolor}
\usepackage{braket}
\usepackage{tikz}
\usepackage{pgfplots}
\usepackage{float}
\usepackage{enumitem}
\usepackage{caption}
\usepackage{subcaption}
\usepackage{algorithm2e}
\usepackage{cancel}
\usepackage{bm}
\usepackage{listings}
\usepackage{pdfpages}
\usepackage{mdframed}
\usepackage{braket}
\usepackage{stmaryrd} 
\usetikzlibrary {datavisualization}
\usetikzlibrary {arrows.meta,bending,positioning}
\usetikzlibrary {datavisualization.formats.functions}
%PREAMBULE pour schÃéma
\usepackage{pgfplots}
\usepackage{tikz}
\usepackage[european resistor, european voltage, european current]{circuitikz}
\usetikzlibrary{arrows,shapes,positioning}
\usetikzlibrary{decorations.markings,decorations.pathmorphing,
decorations.pathreplacing}
\usetikzlibrary{calc,patterns,shapes.geometric}
\usepackage{anyfontsize}


% -------------------------------------
% Pour les chapitres
% -------------------------------------
\usepackage[Glenn]{fncychap} % Style de chapitres


% -------------------------------------
% Largeur du texte (évite de le redéfinir si tu utilises geometry)
% -------------------------------------
%\setlength\textwidth{20.5cm}
%\setlength\textheight{22cm}

% -------------------------------------
% Optionnel : si tu veux jouer avec les marges manuellement
% -------------------------------------
% \setlength\topmargin{-1cm}
% \setlength\evensidemargin{-2cm}
% \setlength\oddsidemargin{\evensidemargin}

\usepackage{mdframed}

\usepackage{scalerel}
\usepackage{xcolor}
\usepackage{stackengine}
\usepgflibrary {shadings}


\usetikzlibrary {decorations.pathmorphing}

\usepackage{tikz}

\usepackage{marvosym}
\usepackage{changepage}

\usepackage{minitoc}
\usepackage{tocloft}
%\renewcommand{\cfttoctitle}{\hspace{-2em}}
% Nastaveni obsahu
% Nastaveni obsahu

\usepackage{imakeidx}
\usepackage{fancyhdr}

%\usepackage{makeidx}
\makeindex[intoc=true]
\makeindex[name=pers, title=Index of person names, intoc=true]

\usepackage{xcolor}

\usepackage{hyperref}

%%%%%%%%%%%%%%%%%%%%%
%\definecolor{linkcolor}{RGB}{0,0,180}
\usepackage{titlesec}

\usepackage{tocloft}
\usepackage{datetime} % Pour une date personnalisée
\usepackage[useregional]{datetime2}

\usepackage{mathrsfs}

% -------------------------------------
% Pour les mini-tables des matières
% -------------------------------------
\usepackage{minitoc}
\dominitoc

%\usepackage[most]{tcolorbox}

%%%%%%%%%%%%%%%%%%%%%%%%%%%%%
%\usepackage[utf8]{inputenc}
%\usepackage[T1]{fontenc}
%\usepackage[french]{babel}
%\usepackage{amsmath, amssymb}
%\usepackage{graphicx}
%\usepackage{hyperref}
%\usepackage{tikz}
%\usepackage{physics}
%\usepackage{float}


%\newcommand{\ket}[1]{\left|#1\right\rangle}
%\newcommand{\bra}[1]{\left\langle#1\right|}
%\newcommand{\mean}[1]{\left\langle#1\right\rangle}
%\newcommand{\dd}{\mathrm{d}}

% Activer \frontmatter, \mainmatter et \appendix pour la classe report
%\newcommand{\frontmatter}{%
%  \pagenumbering{roman}%
%  \setcounter{page}{1}%
%  \renewcommand{\chaptermark}[1]{\markboth{##1}{}}
%  \renewcommand{\sectionmark}[1]{\markright{##1}}
%}
%
%\newcommand{\mainmatter}{%
%  \pagenumbering{arabic}%
%  \setcounter{page}{1}
%}


% \appendix est déjà défini dans report, inutile de le redéfinir


% Figures flottantes:
% fraction maximale d'une page pouvant etre occupe par une figure:
\renewcommand{\topfraction}{0.8}
% fraction minimale d'une page reservee pour le texte:
\renewcommand{\textfraction}{0.2}
% fraction minimale d'occupation de la page par une figure pleine page:
\renewcommand{\floatpagefraction}{0.7}

%%%%%%%%%%%%%%%%%%%%%%%%%%%%%%%%%%%%%%%%
%         D\'ecoupage des mots           %
%%%%%%%%%%%%%%%%%%%%%%%%%%%%%%%%%%%%%%%%
\hyphenation{}

%%%%%%%%%%%%%%%%%%%%%%%%%%%%%%%%%%%%%%%%
%%%%  Th\'eor\`emes, d\'efinitions, etc.
%%%%%%%%%%%%%%%%%%%%%%%%%%%%%%%%%%%%%%%%


% Il y a diffÃérents types d'ÃénoncÃés qui mÃéritent un environnement spÃécifique, voici une liste assez exhaustive.
\theoremstyle{plain}
    \newtheorem{Theo}{Th\'eor\`eme}[section] %compteur commençant par le numÃéro de la section (on pourrait aussi faire commencer par le numÃéro de la sous-section - remplacer "section" par "subsection")
    \newtheorem{Prop}[Theo]{Proposition}        %mÃême compteur que pour les thÃéorÃèmes
    \newtheorem{Prob}[Theo]{Probl\`eme}        %idem
    \newtheorem{Lemm}[Theo]{Lemme}            %etc...
    \newtheorem{Coro}[Theo]{Corollaire}
    \newtheorem{Propr}[Theo]{Propri\'et\'e}
    \newtheorem{Conj}[Theo]{ Conjecture}
    \newtheorem{Aff}[Theo]{Affirmation}

    \newtheorem{TheoPrinc}{Th\'eor\`eme}     %compteur spÃécifique pour les thÃéorÃèmes les plus importants du papier
        
\theoremstyle{definition}
    \newtheorem{Defi}[Theo]{D\'efinition}
    \newtheorem{Exem}[Theo]{Exemple}
    \newtheorem{Nota}[Theo]{\Large Notation}

\theoremstyle{remark}
    \newtheorem{Rema}[Theo]{Remarque}
    \newtheorem{NB}[Theo]{N.B.}
    \newtheorem{Comm}[Theo]{Commentaire}
    \newtheorem{question}[Theo]{$\ast$ Question}
    \newtheorem{exer}[Theo]{Exercice}
    \newtheorem{Consequence}[Theo]{Conséquence}
    \newtheorem{Rap}[Theo]{Rappel}
    \newtheorem*{Merci}{Remerciements}

\mdfdefinestyle{propstyle}{%
linecolor=black,linewidth=2pt,%
hidealllines=true,
frametitlerule=true,%
frametitlebackgroundcolor=gray!20,
backgroundcolor=gray!10!white,
roundcorner=5pt,
innertopmargin=\topskip,
}

%\mdtheorem[style=propstyle]{prop}{Property}[chapter]
\mdtheorem[style=propstyle]{lemma}[prop]{Lemma}
\mdtheorem[style=propstyle]{TheoPrinc}{Th\'eor\`eme}[chapter]

% Définition d'un style personnalisé pour les Affirmations
\mdfdefinestyle{affirmestyle}{%
    linecolor=gray, % Couleur de la bordure
    linewidth=1pt, % Épaisseur de la bordure
    backgroundcolor=gray!10, % Couleur de fond (gris clair)
    roundcorner=5pt, % Coins arrondis
    innertopmargin=0pt, % Marge intérieure au-dessus du cadre
    innerbottommargin=10pt, % Marge intérieure en-dessous du cadre
    innerleftmargin=10pt, % Marge intérieure à gauche
    innerrightmargin=10pt, % Marge intérieure à droite
    skipabove=10pt, % Espace au-dessus du cadre
    skipbelow=10pt % Espace en-dessous du cadre
}

% Définition de l'environnement Affirmation
\theoremstyle{definition} % Style de théorème pour les affirmations
\newmdtheoremenv[style=affirmestyle]{aff}{Point clé n$^{\circ}$} % Environnement Affirmation avec le style personnalisé
    
\newcommand\dangersign[1]{%
    \renewcommand\stacktype{L}%
    \scaleto{\stackon[1.3pt]{\color{red}$\triangle$}{\tiny !}}{#1}%
}

\tikzset{every picture/.style={execute at begin picture={\shorthandoff{:;!?};}}}
\tikzstyle{every picture}+=[remember picture]
\tikzstyle{na} = [shape=rectangle,inner sep=0pt]

% Commandes pour les flèches textuelles
\newcommand{\ptFleche}[2]{        % Déclaration d'une extrémité de flèche
    \tikz[baseline=(#1.base)]\node[na](#1){#2};
  }
%\newcommand{\Fleche}[5][thick]{    % Dessin de la flèche
%    \begin{tikzpicture}[overlay]
%        \path[->,#1](#2) edge [out=#4, in=#5] (#3);
%    \end{tikzpicture}
%  }
  
% \newcommand{\Flecheprim}[5][thick]{    % Dessin de la flèche
%    \begin{tikzpicture}[overlay]
%        \path[->,#1](#2) edge [out=#4, in=#5] (#3);
%    \end{tikzpicture}
%  }
%



\definecolor{linkcolor}{RGB}{0,0,180}
\PassOptionsToPackage{
    colorlinks=true,
    linkcolor=linkcolor,
    citecolor=linkcolor,
    urlcolor=linkcolor
}{hyperref}

% Appliquer la couleur à tous les niveaux de titre
\titleformat{\section}{\normalfont\color{colorSix!90!black}\Large\bfseries}{\thesection}{1em}{}
\titleformat{\subsection}{\normalfont\color{colorSix!70!black}\large\bfseries}{\thesubsection}{1em}{}
\titleformat{\subsubsection}{\normalfont\color{colorSix!50!black}\normalsize\bfseries}{\thesubsubsection}{1em}{}
\titleformat{\paragraph}[runin]{\normalfont\color{colorOne!30!black}\bfseries}{\theparagraph}{1em}{}
\titleformat{\subparagraph}[runin]{\normalfont\color{colorOne!10!black}\itshape}{\thesubparagraph}{1em}{}
%%%%%%%%%%%%%%%%%%%%%%%

%%Couleurs dans la table des matières

% Modifier la couleur des entrées de la TOC
\renewcommand{\cftsecfont}{\color{linkcolor!90!black}}
\renewcommand{\cftsubsecfont}{\color{linkcolor!70!black}}
\renewcommand{\cftsubsubsecfont}{\color{linkcolor!50!black}}
\renewcommand{\cftparafont}{\color{linkcolor!30!black}}
\renewcommand{\cftsubparafont}{\color{linkcolor!10!black}}
%%%%%%%%%%%%%%%%%%%%%%%%%%%%%
% Reglages:
%
%\pagestyle{fancyplain}
%\addtolength{\headwidth}{\marginparsep}
%\addtolength{\headwidth}{\marginparwidth}
%\renewcommand{\chaptermark}[1]{\markboth{#1}{}}
%\renewcommand{\sectionmark}[1]{\markright{\thesection\ #1}}
%\lhead[\fancyplain{}{\bfseries\thepage}]{}
%\rhead[]{\fancyplain{}{\bfseries\thepage}}
%\chead[\fancyplain{}{\bfseries\leftmark}]{\fancyplain{}{\bfseries\rightmark}}
%\cfoot{}
%

%usepackage{titlesec}
% Changer la couleur des paragraphes en rouge par exemple :
%\titleformat{\paragraph}[runin] % ou [block] selon ce que tu veux
%  {\normalfont\color{red}\bfseries}
%  {\theparagraph}{1em}{}

% Définition des couleurs avec les codes HTML
\definecolor{colorOne}{HTML}{443E46}
\definecolor{colorTwo}{HTML}{F6DEB8}
\definecolor{colorThree}{HTML}{908CA4}
\definecolor{colorFour}{HTML}{57659E}
\definecolor{colorFive}{HTML}{C57284}
\definecolor{colorSix}{HTML}{FF5B69}

% Raccourcis pour les couleurs
\def\colorOne{colorOne}
\def\colorTwo{colorTwo}
\def\colorThree{colorThree}
\def\colorFour{colorFour}
\def\colorFive{colorFive}
\def\colorSix{colorSix}

%%% ===== Index principal + index secondaire (noms propres) =====
\makeindex[intoc=true]
\makeindex[name=pers, title=Index des noms propres, intoc=true]

%%% ===== Couleur des liens =====
\definecolor{linkcolor}{RGB}{0,0,180}
\PassOptionsToPackage{
    colorlinks=true,
    linkcolor=linkcolor,
    citecolor=linkcolor,
    urlcolor=linkcolor
}{hyperref}
\usepackage{hyperref}

%%% ===== Réglages hyperref =====
\hypersetup{
  pdftitle={Étude de la dynamique hors équilibre de bosons unidimensionnels},
  pdfsubject={Quantum Physics},
  pdfauthor={Guillaume THEMEZE <guillaume.themeze@gmail.fr>},
  pdfkeywords={LaTeX, quantum, bosons, dynamique},
  colorlinks=true
}

%%% ===== Style des titres (colorés) =====
\titleformat{\chapter}[display]{\normalfont\sffamily\huge\bfseries\color{colorSix}}{\chaptertitlename\ \thechapter}{20pt}{\Huge}
\titleformat{\section}{\normalfont\color{colorSix!90!colorFour}\Large\bfseries}{\thesection}{1em}{}
\titleformat{\subsection}{\normalfont\color{colorSix!70!colorFour}\large\bfseries}{\thesubsection}{1em}{}
\titleformat{\subsubsection}{\normalfont\color{colorSix!50!colorFour}\normalsize\bfseries}{\thesubsubsection}{1em}{}
\titleformat{\paragraph}[runin]{\normalfont\color{colorSix!30!colorFour}\bfseries}{\theparagraph}{1em}{}
\titleformat{\subparagraph}[runin]{\normalfont\color{colorSix!10!colorFour}\itshape}{\thesubparagraph}{1em}{}

%%% ===== Couleurs de la table des matières =====
\renewcommand{\cftsecfont}{\color{linkcolor!90!black}}
\renewcommand{\cftsubsecfont}{\color{linkcolor!70!black}}
\renewcommand{\cftsubsubsecfont}{\color{linkcolor!50!black}}
\renewcommand{\cftparafont}{\color{linkcolor!30!black}}
\renewcommand{\cftsubparafont}{\color{linkcolor!10!black}}

%%% ===== En-têtes et pieds de page =====
\pagestyle{fancy}
\fancyhf{}
\setlength{\headheight}{14pt}

\fancyhead[RO,LE]{\thepage}
\fancyhead[LO]{\scshape \nouppercase{\rightmark}}  % Section
\fancyhead[RE]{\scshape \nouppercase{\leftmark}}  % Chapitre
\renewcommand{\headrulewidth}{.4pt}


\newdateformat{mydate}{\THEDAY~\monthname[\THEMONTH]~\THEYEAR}
\newdateformat{mydatetime}{\THEDAY~\monthname[\THEMONTH]~\THEYEAR~à~\currenttime}

%\DTMsetstyle{french} % ou autre style
%\DTMsetup{showtimezone=false}

\fancyfoot[L]{Thèse}
%\fancyfoot[R]{Paris, \mydatetime\today{} -- Période 2022--2025}
\fancyfoot[R]{Paris, \DTMnow -- Période 2022--2025}
%\fancyfoot[R]{Paris, le \DTMdate\today{} à \DTMcurrenttime -- Période 2022--2025}
\renewcommand{\footrulewidth}{.4pt}

% Supprimer les numéros sur la première page de chaque chapitre
\makeatletter
\let\ps@plain=\ps@empty
\makeatother

%%% ===== Réglages des titres de sections dans les en-têtes =====
\renewcommand{\chaptermark}[1]{\markboth{#1}{}}
\renewcommand{\sectionmark}[1]{\markright{\thesection\ #1}}

%%% ===== Notes de bas de page à la française =====
\usepackage[french]{babel}
%\usepackage[frenchfootnotes]{french}
%\FrenchFootnotes
%\AddThinSpaceBeforeFootnotes

%%%%%%%%%%%%%%%%%%%%%%%%%%%%%%%%%%
\newcommand{\operatorvec}[1]{\vec{{\bm{#1}}}} % pour les operateur
\newcommand{\operator}[1]{\hat{\bm{#1}}} % pour les operaeur vecteur
\newcommand{\operatormat}[1]{\operatorname{#1}} % pour les operaeur vecteur
\newcommand{\operatortilde}[1]{\tilde{\bm{#1}}} % pour les opetateur avec un tilde
\newcommand{\operatortildevec}[1]{\tilde{\bm{#1}}}% pour les opetateur avec un tilde et vecteur
\newcommand{\dfonc}[1]{\mathscr{D}_{[#1]}}
%%%%%%%%%%%%%%%%%%%%%%%%%%%%%%%%%%

%🔤 2. Abréviations classiques
% Mathématiques générales
\newcommand{\dd}{\mathrm{d}}           % différentielle droite
\newcommand{\ii}{\mathrm{i}}           % unité imaginaire
\newcommand{\ee}{\mathrm{e}}           % exponentielle

% Pour les ensembles usuels
\newcommand{\R}{\mathbb{R}}            % réels
\newcommand{\C}{\mathbb{C}}            % complexes
\newcommand{\Z}{\mathbb{Z}}            % entiers
\newcommand{\N}{\mathbb{N}}            % naturels

% Délimiteurs automatiques
%\newcommand{\abs}[1]{\left|#1\right|}
%\newcommand{\norm}[1]{\left\lVert#1\right\rVert}
%\newcommand{\paren}[1]{\left(#1\right)}
%\newcommand{\bracket}[1]{\left[#1\right]}
%\newcommand{\set}[1]{\left\{#1\right\}}

%⚛️ 3. Physique quantique
% Bra-ket
%\newcommand{\ketbra}[2]{\ket{#1}\!\bra{#2}}
%\newcommand{\braket}[2]{\left\langle #1 \middle| #2 \right\rangle}
%\newcommand{\ketproj}[1]{\ket{#1}\!\bra{#1}}

% Hamiltonien, opérateurs
\newcommand{\Ham}{\mathcal{H}}
\newcommand{\Op}[1]{\hat{#1}}
\newcommand{\Tr}{\mathrm{Tr}}

% Commutateurs et anticommutateurs
\newcommand{\comm}[2]{\left[#1, #2\right]}
\newcommand{\acomm}[2]{\left\{#1, #2\right\}}

%🌡️ 4. GHD ou dynamique intégrable
\newcommand{\rhoP}{\rho_{\mathrm{p}}}       % densité de particules
\newcommand{\rhoT}{\rho_{\mathrm{t}}}       % densité totale
\newcommand{\veff}{v^{\mathrm{eff}}}        % vitesse efficace
\newcommand{\dr}{\partial}                  % dérivée
\newcommand{\nustar}{\nu^\ast}              % solution auto-similaire

%✍️ 5. Utilisation typographique
\newcommand{\eg}{\emph{e.g.}\xspace}
\newcommand{\ie}{\emph{i.e.}\xspace}
\newcommand{\etal}{\emph{et al.}\xspace}



% Commandes spécifiques ou pour la mise en forme

\makeatletter
\newcommand\xleftrightarrow[2][]{%
  \ext@arrow 9999{\longleftrightarrowfill@}{#1}{#2}}
\newcommand\longleftrightarrowfill@{%
  \arrowfill@\leftarrow\relbar\rightarrow}
\makeatother


\newacronym{LL}{LL}{Lieb-Liniger}
\newacronym{NS}{NS}{Schrödinger non linéaire}
\newacronym{GP}{GP}{Gross–Pitaevskii}
\newacronym{GGE}{GGE}{Generalized Gibbs Ensemble}

\title{Titre de la thèse}
\author{Prénom NOM}
\date{\today}



\begin{document}

\frontmatter
%\chapter*{Introduction}
\addcontentsline{toc}{chapter}{Introduction}
\minitoc

Ceci est l’introduction de la thèse.


\tableofcontents
\mainmatter

\chapter{Modèle de Lieb-Liniger et approche Bethe Ansatz}
\minitoc

\section*{Introduction}

Dans ce chapitre, nous introduisons progressivement le modèle de Lieb-Liniger et l'Ansatz de Bethe, outils fondamentaux pour décrire un gaz de bosons unidimensionnel avec interactions delta. L'objectif est d'accompagner pas à pas le lecteur depuis la formulation du problème quantique en champ de bosons jusqu'aux solutions exactes obtenues par l'Ansatz de Bethe.

Nous commençons par écrire l'équation du champ de bosons, exprimée à l’aide des opérateurs de création et d’annihilation en représentation de position. Pour des raisons pédagogiques, nous abordons d’abord le cas d’une seule particule, sans interaction. Cela permet d’introduire naturellement les états de position et leur évolution sous l’action du Hamiltonien libre.

Ensuite, nous étudions le cas de deux particules, cette fois en tenant compte de l’interaction locale. Cela nous amène à considérer les états de position dans le cas général, y compris lorsque les deux particules peuvent occuper la même position. Cette situation, bien plus subtile qu’il n’y paraît, met en évidence la complexité introduite par l’interaction, et justifie que l’on commence par analyser les configurations où les particules sont à des positions distinctes.

Dans le référentiel du centre de masse, le problème à deux corps avec interaction devient équivalent à un problème à une seule particule en interaction avec une barrière delta au centre. Cette reformulation permet d’interpréter l’effet de l’interaction comme une condition de raccord sur la fonction d’onde, tout en respectant la symétrie bosonique.

Nous revenons ensuite aux coordonnées du laboratoire afin d’introduire naturellement la forme des solutions imposée par l’Ansatz de Bethe. Cela nous conduit aux équations dites de Bethe, qui relient les quasimoments des particules à travers des conditions de périodicité modifiées par l’interaction.

Une fois les notations bien établies, nous généralisons le raisonnement au cas de \(N\) particules, pour obtenir l’Hamiltonien de Lieb-Liniger complet ainsi que la forme générale de l’Ansatz de Bethe. Les solutions ainsi construites permettent non seulement de déterminer le spectre de l’Hamiltonien, mais aussi de calculer des observables physiques importantes, telles que l’impulsion totale ou le nombre de particules.

Enfin, nous introduisons la notion de distribution de rapidité, outil essentiel dans l’étude des états d’énergie minimale (états fondamentaux) et dans la description thermodynamique du système. Ce cadre servira de base aux développements ultérieurs sur les gaz intégrables à température finie et les états stationnaires après quench quantique.

\section{Description du modèle de Lieb-Liniger}

\subsection{Introduction au modèle de gaz de Bose unidimensionnel et Hamiltonien du modèle}

\subsubsection{De la première à la seconde quantification}

\paragraph{Introduction.}

La mécanique quantique se développe historiquement en deux grandes étapes : la \emph{première quantification}, aussi appelée quantification canonique, et la \emph{seconde quantification}. Comprendre ces deux cadres est essentiel pour aborder les systèmes quantiques complexes, en particulier ceux où le nombre de particules peut varier.

%La mécanique quantique s’est historiquement développée en deux étapes : la \emph{première quantification}, aussi appelée quantification canonique, puis la \emph{seconde quantification}. Comprendre ces deux cadres est essentiel pour aborder les systèmes à nombre de particules variable.


%\vspace{0.5cm}

\paragraph{Première quantification (quantification canonique, particule unique).}

La première quantification est la mécanique quantique standard, celle que vous avez rencontrée dès vos premiers cours. Elle consiste à quantifier un système classique décrit par des variables dynamiques telles que la position $x$ et la quantité de mouvement $p$. On procède en remplaçant ces variables par des {\bf opérateurs hermitiens} $\operator{x}$ et %$\operator{p}$
\begin{eqnarray}
	\operator{p} \doteq -i\hbar \operator{\partial}_x,	\label{chap.1.rapel.1}
\end{eqnarray}
où $\hbar$ est la constante de Planck réduite, satisfaisant la {\bf relation de commutation canonique} fondamentale $[\operator{x}, \operator{p}] = i\hbar$. L’état du système est alors décrit par une {\bf fonction d’onde} $\psi(x,t)$, solution de {\bf l’équation de  Schrödinger} indépendante du nombre de particules :
\begin{eqnarray}
\quad i \hbar \frac{\partial \psi }{\partial t}  &= \operator{\mathcal{H}} \psi,\label{chap.1.rapel.2}
\end{eqnarray}

avec $\operator{\mathcal{H}}$ l’opérateur hamiltonien. 

\begin{mdframed}[
	linewidth=0.5pt, 
	backgroundcolor=gray!5, 
	roundcorner=50pt,	
	innerleftmargin=5pt,
    innerrightmargin=5pt,
    innertopmargin=-10pt,
    innerbottommargin=2pt,
    leftmargin=2pt,
    rightmargin=2pt
	]
\subparagraph{Exemple : particule libre en une boite à une dimension.} 
	{~}\\
	
	Dans le cas d’une particule libre de masse $m$ se déplaçant en une dimension, l’Hamiltonien est constitué uniquement du terme cinétique $\operator{\mathcal{H}} = \operator{p}^2 / 2m$. En représentation position, où l’opérateur quantité de mouvement s’écrit comme dans l’équation \eqref{chap.1.rapel.1}, l’Hamiltonien prend alors la forme différentielle :
	\begin{eqnarray}
		\operator{\mathcal{H}} = -\frac{\hbar^2}{2m} \partial_x^2.
	\end{eqnarray}
	Les états propres stationnaires de \eqref{chap.1.rapel.2} dépendant du temps sont de la forme $\psi_k(x,t) = \varphi_k(x)\,e^{-i\varepsilon(k)t/\hbar}$ où $\varphi_k(x)$ est une fonction propre de l’hamiltonien,  soit de  l’équation stationnaire  $\operator{\mathcal{H}}\varphi_k = \varepsilon(k)\varphi_k$ \ie pour une particule libre:
	\begin{eqnarray}
		\frac{\hbar^2}{2m} \partial_x^2 \varphi_k = \varepsilon(k) \varphi_k,
	\end{eqnarray}
	avec $\varepsilon(k)$ l’énergie associée à une onde plane de nombre d’onde $k$
	\begin{eqnarray}
		\varepsilon(k) = \frac{\hbar^2 k^2 }{2 m}.
	\end{eqnarray}
	Les fonctions propres spatiales $\varphi_k(x)$ de l’hamiltonien libre s’écrivent comme des combinaisons linéaires d’ondes planes  
	\begin{eqnarray}
		\varphi_k(x) = a e^{-i k x} + b e^{i k x},~~ \mbox{avec}\quad  (a,b) \in \mathbb{C}^2.
	\end{eqnarray}
	Si la particule est confinée dans une boîte de longueur $L$ avec des conditions aux limites périodiques (ie $\varphi_k(x+L) = \varphi_k(x)$), alors le spectre de $k$ est quantifié : 
	\begin{eqnarray}
		kL \in 2\pi\mathbb{Z}.
	\end{eqnarray}
	Le problème est équivalent à celui d’une particule libre sur un cercle de périmètre $L$.\\
	
	\medskip
	
	Les solutions générales de l’équation de Schrödinger s’écrivent alors comme une superposition d’états propres  $\psi = \sum_k c_k \psi_k $.  

%On résume :
%\begin{eqnarray}
%	,~~ , ~~\varphi_k(x) = a e^{-i k x} + b e^{i k x},~~ kL \in 2\pi\mathbb{Z}.\label{chap.1.recap}
%\end{eqnarray}
\end{mdframed}

Ces solutions correspondent à des {\bf états non liés} (ou états de diffusion) : la particule est délocalisée sur tout l’espace (le cercle), sans structure particulière.

La fonction $\varphi_k(x)$ est supposée normalisée dans l’espace des états quantifiés (boîte finie) :
\(
\int_0^L dx \, \varphi_{k'}^\ast(x)\, \varphi_k(x) = \delta_{k,\pm k'}.
\)
avec  $ \vert a \vert^2 + \vert b \vert^2 = L^{-1}$.
Et dans le sous espace engendré pas $x \mapsto e^{-ikx}$ et $x \mapsto e^{ikx}$ (de deux dimension si $k \neq 0$ et une dimension si $k$ =0), $x \mapsto \pm ( b^\ast e^{-ikx} - a^\ast e^{ikx} )$ est orthogonale à  $\varphi_k$ pour $k \neq 0$.
\begin{mdframed}[
	linewidth=0.5pt, 
	backgroundcolor=gray!5, 
	roundcorner=50pt,	
	innerleftmargin=5pt,
    innerrightmargin=5pt,
    innertopmargin=-10pt,
    innerbottommargin=2pt,
    leftmargin=2pt,
    rightmargin=2pt
	]
\subparagraph{Remarque.} On choisie  \( a = \frac{1}{\sqrt{L}} \) et \( b = 0 \)), alors
\(
\varphi_k(x) = \frac{1}{\sqrt{L}}\, e^{i k x}
\)
est une onde plane. 

\end{mdframed}

Avec le formalisme de Dirac, la fonction d’onde $\varphi_k$ est représentée par le ket $\ket{k}$ normé (i.e. $\langle k' \vert k \rangle = \delta_{k, k'}$, où $\delta_{p,q}$ est le symbole de Kronecker)
, et l’équation de Schrödinger s’écrit :
\(
\operator{\mathcal{H}}_1 \ket{k} = \varepsilon(k) \ket{k}.
\)
En appliquant le bra $\bra{x}$ de part et d’autre, on obtient :
\(
\bra{x} \operator{\mathcal{H}}_1 \ket{k} = \varepsilon(k) \langle x \vert k \rangle,
\)
où $\varphi_k(x) = \langle x \vert k \rangle$ est la représentation positionnelle de l’état $\ket{k}$.



%\begin{mdframed}[
%	linewidth=0.5pt, 
%	backgroundcolor=gray!5, 
%	roundcorner=50pt,	
%	innerleftmargin=5pt,
%    innerrightmargin=5pt,
%    innertopmargin=-10pt,
%    innerbottommargin=2pt,
%    leftmargin=2pt,
%    rightmargin=2pt
%	]
%\subparagraph{Remarque.} Si l’on choisit une base orthonormée d’états propres (par exemple en fixant \( a = \frac{1}{\sqrt{L}}, b = 0 \)), alors
%\(
%\varphi_k(x) = \frac{1}{\sqrt{L}}\, e^{i k x}, \quad \text{et donc} \quad \langle k \vert x \rangle = \varphi_k^\ast(x) = \frac{1}{\sqrt{L}}\, e^{-i k x},
%\)
%ce qui est bien une onde plane. 
%En revanche, dans l’écriture générale \( \varphi_k(x) = a\, e^{i k x} + b\, e^{-i k x} \), la fonction \( \langle k \vert x \rangle = \varphi_k^\ast(x) \) n’est \emph{pas nécessairement} une onde plane, car \( \varphi_k(x) \) n’est pas normalisée.
%\end{mdframed}

\begin{mdframed}[
	linewidth=0.5pt, 
	backgroundcolor=gray!5, 
	roundcorner=50pt,	
	innerleftmargin=5pt,
    innerrightmargin=5pt,
    innertopmargin=1pt,
    innerbottommargin=2pt,
    leftmargin=2pt,
    rightmargin=2pt
	]
La base $\{\ket{x}\}$ étant continue, et les états $\{\ket{k}\}$ quantifiés (par exemple dans une boîte de taille finie avec conditions aux limites périodiques), les relations de changement de base s’écrivent :
\begin{eqnarray}
	\ket{k} = \int dx \, \varphi_k(x) \ket{x}, \qquad   
	\ket{x} = \sum_k \varphi_k^\ast(x) \ket{k},
\end{eqnarray}
avec $\varphi_k^\ast(x) = \langle k \vert x \rangle$. L’état $\ket{x}$ est relié aux états $\ket{k}$ par une transformation de Fourier discrète. Ces formules montrent que les états $\ket{k}$ sont les composantes de Fourier de l’état $\ket{x}$.
\end{mdframed}






\subparagraph{De la particule unique aux systèmes à $N$ particules.}

Pour un système composé de $N$ particules identiques, une approche naturelle consiste à introduire une fonction d’onde $\varphi(x_1, \dots, x_N)$ dépendant de $N$ variables , symétrique pour des bosons ou antisymétrique pour des fermions sous l’échange de deux coordonnées $x_i \leftrightarrow x_j$, solution de l’équation de Schrödinger à $N$ corps. %Dans le cas bosonique, des interactions à courte portée peuvent être modélisées par un potentiel de type Dirac :

%\begin{equation}
%V_{\text{int}}(x_1, \dots, x_N) = g \sum_{i<j} \delta(x_i - x_j),
%\end{equation}

%où $g$ est un paramètre d’interaction contrôlant l’intensité des collisions. 
Toutefois, cette description devient rapidement inextricable lorsque le nombre de particules augmente, ou lorsque le système permet la création et l’annihilation de particules, comme dans un milieu ouvert ou en contact avec un bain thermique.





%{\color{blue} \paragraph{Seconde quantification.}
%
%%Dans ce formalisme, l’espace des états est une **somme directe d’espaces à $N$ particules**, et chaque état est décrit par son occupation des modes quantiques. Les opérateurs $\hat{a}_k^\dagger$ et $\hat{a}_k$ créent et détruisent une particule dans l’état d’onde plane de moment $k$, satisfaisant les relations de commutation (bosons) ou d’anticommutation (fermions) :
%%\begin{equation}
%%[\hat{a}_k, \hat{a}_{k'}^\dagger] = \delta_{k,k'} \quad \text{(bosons)}.
%%\end{equation}
%
%%L’hamiltonien d’un gaz de particules libres s’écrit alors simplement :
%%\begin{equation}
%%\hat{\mathcal{H}} = \sum_k \varepsilon(k) \, \hat{a}_k^\dagger \hat{a}_k,
%%\end{equation}
%%avec $\varepsilon(k) = \frac{\hbar^2 k^2}{2m}$ comme dans la quantification canonique.
%
%\paragraph{Vers le Bethe ansatz.}
%
%Ce formalisme devient particulièrement utile lorsque des interactions entre particules sont introduites. Dans le cas d’un **gaz de bosons en une dimension avec interactions de contact**, le système reste exactement soluble : la solution repose sur une **superposition cohérente d’ondes planes symétrisées**, ajustées par les conditions d’interaction.
%
%C’est l’idée fondamentale du **Bethe ansatz**, qui généralise la solution d’une particule libre sur un cercle à $N$ particules **avec collisions élastiques**. On y retrouve des relations de quantification du type :
%\begin{equation}
%k_j L + \sum_{\substack{\ell=1 \\ \ell \neq j}}^N \theta(k_j - k_\ell) = 2\pi n_j,
%\end{equation}
%où $\theta$ est une phase de diffusion et les $n_j \in \mathbb{Z}$.
%
%On passe ainsi des conditions de périodicité simples à des **conditions de type Bethe**, qui encodent les effets des interactions sous forme de **conditions de compatibilité entre les moments**.
%
%}

\subsubsection{Seconde quantification}

Pour dépasser ces limitations, on adopte le \textbf{formalisme de la seconde quantification}, dans lequel l’état du système est décrit non plus par une fonction d’onde mais par un vecteur dans un espace de Fock. Les opérateurs de création et d’annihilation remplacent alors les variables dynamiques classiques et permettent une description unifiée et élégante des systèmes à nombre variable de particules.

%\paragraph{Structure de l’espace des états de Fock.}
%Dans ce formalisme, l’espace des états est une {\bf somme directe d’espaces à $N$ particules}, et chaque état est décrit par l’occupation des différents modes quantiques. Les opérateurs $\operator{a}_k^\dagger$ et $\operator{a}_k^{}$ créent et annihilent une particule dans l’état d’onde plane de moment $k$ :
%\begin{eqnarray*}
%	\ket{k} & = & \operator{a}_k^\dagger \ket{\emptyset} ~=~ \text{état avec une particule dans le mode } k,	
%\end{eqnarray*}
%où \(\ket{\emptyset}\) est le vide quantique de Fock, défini par :
%\begin{eqnarray}
%	\forall k \in \mathbb{R}\colon \qquad \operator{a}_k \ket{\emptyset} = 0 ,\quad  \langle \emptyset\vert \emptyset \rangle = 1, \label{chap:eq.vide.fock.k}
%\end{eqnarray}
%où \( \operator{a}_\lambda \) peut ici estre une notation générique désignant \( \operator{b}_\lambda \) pour les bosons, ou \( \operator{c}_\lambda \) pour les fermions,
%et satisfaisant les relations de commutation (pour les bosons) ou d’anticommutation (pour les fermions). Dans ce qui suit, nous nous restreignons au cas bosonique. \subparagraph{Relations de commutation bosoniques.} Les relations de commutation bosoniques fondamentales sont alors :
%\begin{eqnarray}
%[\operator{a}_k^{}, \operator{a}_{k'}^{}]= [\operator{a}_k^\dagger, \operator{a}_{k'}^\dagger]= 0 ,\qquad [\operator{a}_k^{}, \operator{a}_{k'}^\dagger] = \operator{\delta}_{k,k'},\label{chap:1:com.1.k}
%\end{eqnarray}
%où $\operator{\delta}_{k,k'}$ est le symbole de Kronecker, qui vaut $1$ si $k = k'$ et $0$ sinon.\\

%%%%%%%%%%%%%%%%%%%%%%%%
\paragraph{Structure de l’espace des états de Fock.}
Dans ce formalisme, l’espace des états est une {\bf somme directe d’espaces à $N$ particules}, et chaque état est décrit par l’occupation des différents modes quantiques. Les opérateurs $\operator{a}_k^\dagger$ et $\operator{a}_k$ créent et annihilent une particule dans l’état d’onde plane de moment $k$ :
\begin{eqnarray*}
	\ket{k} & = & \operator{a}_k^\dagger \ket{\emptyset} ~=~ \text{état avec une particule dans le mode } k,	
\end{eqnarray*}
où \(\ket{\emptyset}\) désigne le vide quantique de Fock, défini par :
\begin{eqnarray}
	\forall k \in \mathbb{R}\colon \qquad \operator{a}_k \ket{\emptyset} = 0 ,\quad  \langle \emptyset \vert \emptyset \rangle = 1. \label{chap:eq.vide.fock.k}
\end{eqnarray}
Le symbole \( \operator{a}_\lambda \) représente ici de manière générique soit l’opérateur \( \operator{b}_\lambda \) pour les bosons, soit \( \operator{c}_\lambda \) pour les fermions, et satisfait respectivement les relations de commutation (pour les bosons) ou d’anticommutation (pour les fermions). Dans ce qui suit, nous nous restreignons au cas bosonique.

\subparagraph{Relations de commutation bosoniques.} Les relations de commutation fondamentales pour les bosons sont :
\begin{eqnarray}
	[\operator{b}_k, \operator{b}_{k'}] = [\operator{b}_k^\dagger, \operator{b}_{k'}^\dagger] = 0 ,\qquad [\operator{b}_k, \operator{b}_{k'}^\dagger] = \operator{\delta}_{k,k'}, \label{chap:1:com.1.k}
\end{eqnarray}
où $\operator{\delta}_{k,k'}$ est le symbole de Kronecker, valant $1$ si $k = k'$ et $0$ sinon.
%%%%%%%%%%%%%%%%%%%%%%%%%%%%%%%%%%%%%%%%

%\vspace{1em}
\paragraph{Nature du champ quantique.}
La seconde quantification généralise ce cadre en permettant de traiter des systèmes où le nombre de particules n’est pas fixé, ce qui est fréquent en physique des particules, des champs quantiques, ou des gaz quantiques.

L’idée principale est de ne plus quantifier directement les particules, mais le \emph{champ quantique} associé. Les états d’une particule unique deviennent alors des états d’occupation dans un espace de Fock, qui décrit l’ensemble des configurations possibles avec zéro, une, ou plusieurs particules.



\subparagraph{Champs de Bose.}
Le gaz de Bose unidimensionnel est décrit dans le cadre de la théorie quantique des champs par un champ bosonique canonique \( \operator{\Psi}(x) \), qui agit sur l’espace de Fock des états du système. Ce champ quantique encode l’annihilation d’une particule en \( x \), et son adjoint \( \operator{\Psi}^\dag(x) \) correspond à la création d’une particule en ce point. 
\begin{eqnarray}
	\vert x \rangle  & = & \operator{\Psi}^\dag (x)\ket{\emptyset} ~=~ \text{état avec une particule en } x,
\end{eqnarray}
et \(\ket{\emptyset}\) est le vide quantique de Fock défini par :
\begin{eqnarray}
	\forall x \in \mathbb{R}, \qquad \operator{\Psi}(x) \ket{\emptyset} = 0. \label{chap:eq.vide.fock}
\end{eqnarray}

\subparagraph{Relations de commutation bosoniques.}
Ces champs satisfont les relations de commutation canoniques à temps égal :
%\begin{eqnarray}
%	\left . \begin{array}{rcl}
%		[ \operator{\Psi}(x),  \operator{\Psi}^\dagger(y) ]  &=&  \operator{\delta}(x - y) \\
%		\left [ \operator{\Psi}(x),  \operator{\Psi}(y) \right ]   =  [ \operator{\Psi}^\dag(x),  \operator{\Psi}^\dag(y) ]  &=&  0 
%	\end{array} \right . \label{chap:1:com.1}
%\end{eqnarray}
\begin{eqnarray}
	 [ \operator{\Psi}(x),  \operator{\Psi}(y)  ]   =  [ \operator{\Psi}^\dag(x),  \operator{\Psi}^\dag(y) ]  =  0,   & & [ \operator{\Psi}(x),  \operator{\Psi}^\dagger(y) ]  =  \operator{\delta}(x - y) ,\label{chap:1:com.1}
\end{eqnarray}
où $\operator{\delta}(x - y)$ est la fonction delta de Dirac.  
Ces relations expriment le caractère bosonique des excitations du champ.

\paragraph{État à $N$ particules.} Soient $N$ bosons dans les états $\{ k_1 , \cdots , k_N \}$ (un boson dans l’état $k_1$, un autre dans $k_2$, etc.) et aux positions $\{ x_1 , \cdots , x_N \}$ (un boson en $x_1$, un autre en $x_2$, etc.). Leurs états s’écrivent alors :
\begin{eqnarray}
	\ket{ \{ k_1 , \cdots , k_N \}} = \frac{1}{\sqrt{N!}} \operator{b}_{k_1}^\dag\, \cdots \, \operator{b}_{k_N}^\dag \ket{\emptyset}, \quad \ket{\{x_1 , \cdots , x_N\}} = \frac{1}{\sqrt{N!}} \operator{\Psi}^\dag(x_1)\, \cdots \, \operator{\Psi}^\dag(x_N) \ket{\emptyset}	, \label{eq.chap.1.ket.N}
\end{eqnarray}
où le facteur \( 1/\sqrt{N!} \) traduit le caractère d’indiscernabilité des bosons et garantit la symétrisation correcte de l’état.

\subparagraph{Changement de base.}
On peut relier les opérateurs de création/annihilation dans la base des ondes planes aux opérateurs de champ via :
\begin{eqnarray}
	\operator{b}_k^\dagger = \int dx \, \varphi_k(x) \operator{\Psi}^\dagger(x), \qquad 
	\operator{\Psi}^\dagger(x) = \sum_k \varphi_k^\ast(x)\operator{b}_k^\dagger.\label{eq.chap.1.TF.1}
\end{eqnarray}
Le champ quantique $\operator{\Psi}(x)$ est relié aux opérateurs de moment $\operator{b}_k$ par une transformation de Fourier. Ces formules montrent que les opérateurs $\operator{b}_{k}$ sont les composantes de Fourier du champ $\operator{\Psi}(x)$.

%où $\varphi_k(x)$ est la fonction d’onde d’un état d’énergie bien définie \( \ket{k} \) dans la représentation positionnelle.
Ainsi, un état à \(N\) bosons dans la base \( \ket{k}^{\otimes N} \) peut s’écrire :
\begin{eqnarray}
	\ket{\{k_1 , \cdots , k_N\}} = \frac{1}{\sqrt{N!}} \int dx_1 \cdots dx_N \, \varphi_{\{k_a\}} ( x_1 , \cdots , x_N ) \, \hat{\Psi}^\dag(x_1) \cdots \hat{\Psi}^\dag(x_N) \ket{\emptyset},
\end{eqnarray}
où \( \{k_a\} \equiv \{k_1, \dots, k_N\} \), et la fonction d’onde symétrisée s’écrit :
\(
	\varphi_{\{k_a\}} ( x_1 , \cdots , x_N ) = \frac{1}{\sqrt{N!}} \sum_{\sigma \in \operator{S}_N } \prod_{i=1}^N \varphi_{k_{\sigma(i)}}(x_i),
\) 
avec $\operator{S}_N $  le groupe symétrique d'ordre $N$ mais aussi :
\begin{eqnarray}
	\varphi_{\{k_a\}} ( x_1 , \cdots , x_N ) = \frac{1}{\sqrt{N!}} \bra{\emptyset} \hat{\Psi}(x_1) \cdots \hat{\Psi}(x_N) \ket{\{k_1, \cdots , k_N\}}.
\end{eqnarray}



\subsubsection{Operateur. }


\paragraph{Opérateur à un corps.}

Soit \( \operator{f} \) un opérateur à une particule, dont les éléments de matrice dans une base orthonormée \( \{ \ket{k} \} \) sont donnés par \( f_{\lambda\nu} = \langle \lambda \vert \operator{f} \vert \nu \rangle \). Un opérateur symétrique à \( N \) particules correspondant à la somme des actions de \( \operator{f} \) sur chacune des particules s’écrit en première configuration  :
\(
	\operator{F} = \sum_{i=1}^N \operator{f}^{(i)},
\)
où \( \operator{f}^{(i)} \) désigne l’action de \( \operator{f} \) sur la $i^\text{e}$ particule uniquement. En base de Dirac, cela donne :
\(
	\operator{f}^{(i)} = \sum_{\lambda, \nu} f_{\lambda\nu} \, \ket{i\!:\!\lambda} \bra{i\!:\!\nu},
\)
où \( \ket{i\!:\!\lambda} \) représente un état où seule la $i^\text{e}$ particule est dans l’état \( \lambda \). (Par construction, l’opérateur \( \operator{F} \) commute avec les projecteurs de symétrisation \( \operator{S}_N \) (bosons) et d’antisymétrisation \( \operator{A}_N \) (fermions).)
On peut montrer que la somme des projecteurs agissant sur chaque particule s’identifie à une combinaison d’opérateurs de création et d’annihilation :
\(
	\sum_{i=1}^N \ket{i\!:\!\lambda} \bra{i\!:\!\nu} = \operator{a}^\dagger_\lambda \operator{a}_\nu^{},
\)
(où \( \operator{a}_\lambda \) peut ici estre une notation générique désignant \( \operator{b}_\lambda \) pour les bosons, ou \( \operator{c}_\lambda \) pour les fermions).

On en déduit que l’opérateur à un corps \( \operator{F} \) peut se réécrire dans le formalisme de la seconde quantification comme :
\begin{eqnarray}
	\operator{F} = \sum_{\lambda, \nu} f_{\lambda\nu} \, \operator{a}^\dagger_\lambda \operator{a}_\nu^{}.
\end{eqnarray}


\subparagraph{Exemples d’opérateurs à un corps.}

Si l’on sait diagonaliser l’opérateur \( \operator{f} \), c’est-à-dire si l’on peut écrire :
\(
	\operator{f} = \sum_k f_k \ket{k} \bra{k},
\)
alors l’opérateur à $N$ corps associé s’écrit :
\(
	\operator{F} = \sum_k f_k \, \operator{a}^\dagger_k \operator{a}_k^{} = \sum_k f_k \, \operator{n}_k,
\)
où \( \operator{n}_k = \operator{a}^\dagger_k \operator{a}_k \) est l’opérateur nombre de particules dans le mode \( k \). On obtient ainsi une forme diagonale de \( \operator{F} \) en seconde quantification.
\begin{mdframed}[linewidth=0.5pt, backgroundcolor=gray!5, roundcorner=5pt]
Un exemple immédiat est celui des particules libres. Si l’on diagonalise le problème à une particule selon :
\(
	\operator{\mathcal{H}}_1 \ket{k} = \varepsilon(k) \ket{k},
\)
alors l’énergie totale du système correspond ici uniquement à son énergie cinétique, et s’écrit :
\begin{equation}
	\operator{K} = \sum_{k} \varepsilon(k) \, \operator{a}^\dagger_k \operator{a}_k^{}.\label{eq.chap.1.cinietique.1}
\end{equation}

Et pour $N$ particules, en écrivant l’état sous la forme~\eqref{eq.chap.1.ket.N}, en utilisant les relations de commutation~\eqref{chap:1:com.1.k} et la définition de l’état de Fock~\eqref{chap:eq.vide.fock.k}, on trouve que $\ket{\{k_1, \cdots, k_N\}}$ est un état propre de $\operator{K}$ associé à l'énergie $\left( \sum_{i = 1}^N \varepsilon(k_i) \right)$, c’est-à-dire :
\begin{eqnarray}
	\operator{K} \ket{\{k_1, \cdots, k_N\}} = \left( \sum_{i = 1}^N \varepsilon(k_i) \right) \ket{\{k_1, \cdots, k_N\}}.\label{eq.chap.1.cinietique.2}
\end{eqnarray}
\end{mdframed}

\paragraph{Forme champ des opérateurs à un corps.}

Les opérateurs à plusieurs corps peuvent être exprimés de manière remarquable à l’aide des opérateurs de champ, d’une façon physiquement transparente qui rappelle les formules bien connues du cas à une particule.

La forme générale d’un opérateur à un corps s’écrit :
\begin{eqnarray}
\operator{F} = \int dx \, dx' \, \operator{\Psi}^\dagger(x) \, \bra{ x} \operator{f} \ket{x'} \, \operator{\Psi}(x').
\end{eqnarray}%où \( \hat{f} \) est l’opérateur à un corps exprimé dans la base position, et \( \hat{\psi}^\dagger(\vec{r}) \), \( \hat{\psi}(\vec{r}) \) sont les opérateurs de création et d’annihilation d’une particule au point \( \vec{r} \).
\begin{mdframed}[
	linewidth=0.5pt, 
	backgroundcolor=gray!5, 
	roundcorner=50pt,	
	innerleftmargin=5pt,
    innerrightmargin=5pt,
    innertopmargin=-10pt,
    innerbottommargin=2pt,
    leftmargin=2pt,
    rightmargin=2pt
]
\subparagraph{Énergie cinétique totale.}

Pour des particules non relativistes, l’énergie cinétique élémentaire s’écrit $\operator{f} = \frac{\hbar^2 \operator{p}^2}{2m}$. À l’échelle du champ quantique, l’énergie cinétique totale prend la forme opératorielle :
\begin{eqnarray}
\operator{K} =  -\frac{\hbar^2}{2m} \int dx \, \operator{\Psi}^\dagger(x) \, \operator{\partial}_x^2 \operator{\Psi}(x)
= \frac{\hbar^2}{2m} \int dx \, \operator{\partial}_x \operator{\Psi}^\dagger(x) \cdot \operator{\partial}_x \operator{\Psi}(x). \label{eq.chap.1.cinietique.3}
\end{eqnarray}

Le champ quantique $\operator{\Psi}(x)$ est relié aux opérateurs de moment $\operator{b}_k$ par une transformation de Fourier. En injectant l'expression \eqref{eq.chap.1.TF.1} dans \eqref{eq.chap.1.cinietique.3}, on retrouve la forme discrète \eqref{eq.chap.1.cinietique.1}, cette fois exprimée en termes des opérateurs $\operator{b}_k$.

Lorsque cet Hamiltonien agit sur l’état de Fock à $N$ particules $\ket{\{k_1 , \cdots , k_N\}}$, les règles de commutation (\ref{chap:1:com.1}) ainsi que la définition des états de Fock (\ref{chap:eq.vide.fock}) impliquent (cf. Annexe \ref{annex:N.part}) :
\begin{eqnarray}
\operator{K}\ket{k_1 , \cdots , k_N } =  \int d^N z \, \operator{\mathcal{K}}_N \, \varphi_{\{k_a\}}(z_1 , \cdots , z_N ) \operator{\Psi}(z_1) \cdots \operator{\Psi}^\dag(z_N) \ket{\emptyset}
\end{eqnarray}
avec :
\[
	\operator{\mathcal{K}}_N = \sum_{i=1}^N \frac{\operator{p}_i^2}{2m},
\]
où \( \operator{p}_i \) désigne l’opérateur impulsion de la \( i^\text{ème} \) particule.
\end{mdframed}




\paragraph{Opérateurs à deux corps}

Nous considérons à présent les termes d’interaction impliquant deux particules , $\operator{v}$ , dont les éléments de matrices sont donnés par $v_{\alpha \beta \gamma \delta} = \bra{ 1 : \alpha; 2 : \beta } \operator{v}\ket{ 1 : \gamma; 2 : \delta }$ , où $\ket{ i : \gamma; j : \delta }$ représente l'état où la $i^\text{e}$  particules est dans l'état $\gamma$ et la $j^\text{e}$ dans l'état $\delta$  . Ceux-ci correspondent à des opérateurs de la forme :
\(
    \operator{V} = \sum_{j < i} \operator{v}^{(i, j)} = \frac{1}{2} \sum_{i, j \ne i} \operator{v}^{(i, j)}
    \label{eq:V2corps}.
\)
avec $\operator{v}^{(i, j)}$ désigne l’interaction à deux corps entre les $i^\text{e}$ et $j^\text{e}$ particules , exprimés dans la base à deux états :
\(
	\operator{v}^{(i, j)} = \sum_{\alpha,\beta,\delta,\gamma} \ket{i : \alpha; j : \beta }v_{\alpha \beta \gamma \delta} \bra{ i : \gamma; j : \delta }.
    %v_{\alpha \beta \gamma \delta} = \bra{ i : \alpha; j : \beta } \operator{v}^{(i,j)} \ket{ i : \gamma; j : \delta }.
    \label{eq:matriceV}
\)
On peut réécrire l’opérateur \( \operator{V} \) en termes d’opérateurs de création et d’annihilation comme suit :
\begin{equation}
    \operator{V} = \frac{1}{2} \sum_{\alpha, \beta, \gamma, \delta} v_{\alpha \beta \gamma \delta} \, \operator{a}^\dagger_\alpha \operator{a}^\dagger_\beta \operator{a}_\delta^{} \operator{a}_\gamma^{}.
    \label{eq:Vcreation}
\end{equation}

Cette forme est particulièrement utile pour le traitement des interactions dans l’espace de Fock, notamment en théorie des champs et en physique des gaz quantiques.

\subparagraph{Expression générale d’un terme à deux corps. }

Un terme d’interaction à deux corps général peut s’écrire :
\begin{equation}
    \operator{V} = \frac{1}{2} \int dx_1^{} \, dx_2^{} \, dx_1' \, dx_2' \; 
    \bra{ 1 : x_1^{}, 2 : x_2^{} } \operator{v} \ket{ 1 : x_1', 2 : x_2' } \,
    \operator{\Psi}^\dagger(x_1^{}) \, \operator{\Psi}^\dagger(x_2^{}) \, 
    \operator{\Psi}(x_2') \, \operator{\Psi}(x_1')
    \label{eq:V_general}
\end{equation}

\begin{mdframed}[
	linewidth=0.5pt, 
	backgroundcolor=gray!5, 
	roundcorner=50pt,	
	innerleftmargin=5pt,
    innerrightmargin=5pt,
    innertopmargin=-10pt,
    innerbottommargin=2pt,
    leftmargin=2pt,
    rightmargin=2pt
	]
\subparagraph{Interactions ponctuelles.} 
Dans le cas d’une interaction ne dépendant que de la distance relative entre deux particules, cette expression se simplifie :
\(
     \operator{V} = \frac{1}{2} \sum_{i, j \ne i}  \operator{v}(x_i^{} - x_j^{}) = 
    \frac{1}{2} \int dx_1^{} \, dx_2^{} \; v(x_1^{} - x_2^{}) \,
    \operator{\Psi}^\dagger(x_1^{}) \, \operator{\Psi}^\dagger(x_2^{}) \, 
    \operator{\Psi}(x_2^{}) \, \operator{\Psi}(x_1^{})
    \label{eq:V_local}
\) soit pour des interactions ponctuelles :	
\begin{eqnarray}
	\quad \operator{V}  = \frac{g}{2} \int dx \,
    \operator{\Psi}^\dagger(x) \, \operator{\Psi}^\dagger(x) \, 
    \operator{\Psi}(x) \, \operator{\Psi}(x)  		
\end{eqnarray}
et quand on l'applique à l'état $\ket{\{k_1 , \cdots , k_N\}} $ , les règles de commutations (\ref{chap:1:com.1}) et la définition d'état de Fock (\ref{chap:eq.vide.fock}) impliquent que (cf Annex \ref{annex:N.part})
\begin{eqnarray}
\operator{V}\ket{\{k_1 , \cdots , k_N\}} =  \int d^Nz \, \operator{\mathcal{V}}_N \varphi_{\{k_a\}}(z_1 , \cdots , z_N )\operator{\Psi}(z_1)\cdots \operator{\Psi}^\dag(z_N) \ket{\emptyset} 
\end{eqnarray}
avec 
\(
	\operator{\mathcal{V}}_N 	
 = g\sum_{1\leq i < j \leq N } \operator{\delta}(x_i - x_j)	
\)
où \( g \) est la constante de couplage.
\end{mdframed}


%Le hamiltonien général décrivant des particules identiques en interaction s’écrit alors :
%\begin{equation}
%    \hat{H} = \int d\vec{r} \; \hat{\psi}^\dagger(\vec{r}) 
%    \left( -\frac{\hbar^2}{2m} \Delta + u(\vec{r}) - \mu \right) 
%    \hat{\psi}(\vec{r})
%    + \frac{1}{2} \int d\vec{r} \, d\vec{r}' \; v(\vec{r} - \vec{r}') \,
%    \hat{\psi}^\dagger(\vec{r}') \, \hat{\psi}^\dagger(\vec{r}) \,
%    \hat{\psi}(\vec{r}) \, \hat{\psi}(\vec{r}')
%    \label{eq:H_general}
%\end{equation}

%\noindent
%Bien que cette expression ait une interprétation physique très claire, il est important de garder à l'esprit que \( \hat{H} \) et \( \hat{\psi} \) sont des objets du formalisme à plusieurs corps.



%%%%%%%%%%%%%%%%
%........................

%\subsubsection{Seconde quantification}



%\paragraph{Hamiltoniens en seconde quantification.}
%\subparagraph{Terme à un corps.}
%Un hamiltonien à un corps, correspondant à une énergie cinétique ou un potentiel externe, s’écrit :
%\[
%\hat{\mathcal{H}}_1 = \int dx\, \operator{\Psi}^\dagger(x) \hat{h}(x) \operator{\Psi}(x),
%\]
%où \( \hat{h}(x) \) est l’opérateur d’un corps (ex. : \( -\frac{\hbar^2}{2m} \partial_x^2 + V(x) \)).

%\subparagraph{Terme à deux corps.}
%Les interactions entre particules, modélisées par une interaction à deux corps \( V(x - y) \), s’expriment comme :
%\[
%\hat{\mathcal{H}}_2 = \frac{1}{2} \int dx\,dy\, \operator{\Psi}^\dagger(x) \operator{\Psi}^\dagger(y) V(x - y) \operator{\Psi}(y) \operator{\Psi}(x).
%\]


%.......................


\paragraph{Expression de l’Hamiltonien. }
L’hamiltonien dans ce formalisme s’écrit en termes des opérateurs de champ, par exemple pour l’énergie cinétique et les interactions ponctuelles avec $\hbar = m = 1 $  :

%Le Hamiltonien du modèle est donné par

%\begin{eqnarray}
%	\operator{H} & = & \int dx \, \left [ \operator{\partial}_x \operator{\Psi}^\dag (x)\operator{\partial}_x \operator{\Psi}(x) + c \operator{\Psi}^\dag (x) \operator{\Psi}^\dag (x) \operator{\Psi} (x) \operator{\Psi} (x) \right ] \label{chap:1:ham.mod}
%\end{eqnarray}

\begin{eqnarray}
	\operator{H} & = & \int dx \, \operator{\Psi}^\dag (x)\left [-\frac{1}{2}\operator{\partial}_x^2 + \frac{g}{2}  \operator{\Psi}^\dag (x) \operator{\Psi} (x) \right ] \operator{\Psi} (x) \label{chap:1:ham.mod}.
\end{eqnarray}

Quand on l'applique à l'état $\ket{\{\theta_1 , \cdots , \theta_N \}} $, avec $\theta_i$ homogène à des nombres d'onde ou à des vitesse , il vient que %, les règles de commutations (\ref{chap:1:com.1}) et la définition d'état de Fock (\ref{chap:eq.vide.fock}) impliquent que (cf Annex \ref{annex:N.part})
\begin{eqnarray}
\operator{H}\ket{\{\theta_1 , \cdots , \theta_N\}} =  \int d^Nz \, \operator{\mathcal{H}}_N \varphi_{\{\theta_a\}}(z_1 , \cdots , z_N )\operator{\Psi}(z_1)\cdots \operator{\Psi}^\dag(z_N) \ket{\emptyset} 
\end{eqnarray}
avec 
\(
	\operator{\mathcal{H}}_N 	
 =  \operator{\mathcal{K}}_N  +  \operator{\mathcal{V}}_N .	
\)


%où \( g \) est la constante de couplage. %Dans ce chapitre, nous considérons uniquement les propriétés du système à un instant donné, de sorte que la dépendance temporelle des champs est omise pour alléger l’écriture.

Ce formalisme est ainsi adapté pour décrire des condensats de Bose, des gaz quantiques, ou la création/annihilation de particules dans les champs quantiques.

\paragraph{Équation du mouvement associée.}

L’équation du mouvement du champ \( \Psi(x) \) est obtenue à partir de l’équation de Heisenberg :

\begin{eqnarray}
	i\operator{\partial}_t	\operator{\Psi} & = & [ \operator{\Psi} , \operator{H} ]
\end{eqnarray}

ce qui, après évaluation explicite du commutateur (\ref{chap:1:com.1}), conduit à :


%\begin{eqnarray}
%	i \operator{\partial}_t \operator{\Psi}	 & = & - \operator{\partial}_x^2 \operator{\Psi} + 2c \operator{\Psi}^\dag\operator{\Psi} \operator{\Psi}
%\end{eqnarray}

\begin{eqnarray}
	i \operator{\partial}_t \operator{\Psi}	 & = & - \frac{1}{2}\operator{\partial}_x^2 \operator{\Psi} + g \operator{\Psi}^\dag\operator{\Psi} \operator{\Psi}
\end{eqnarray}

est appelée l'équation de \textbf{Schrödinger non linéaire (NS)}.

Pour $g > 0$, l'état fondamental à température nulle est une sphère de Fermi. Seul ce cas sera considéré par la suite.

%\vspace{0.5cm}

\subsubsection*{Conclusion}

La première quantification est la base indispensable qui permet de comprendre le comportement quantique d’un nombre fixé de particules. La seconde quantification en est une extension naturelle, nécessaire pour décrire des systèmes plus complexes où le nombre de particules peut varier. Elle repose sur la quantification des champs, et l’introduction d’opérateurs créant ou détruisant ces particules, ouvrant ainsi la voie à la physique quantique des champs et à de nombreuses applications modernes.


\subsection{Opérateurs nombre de particules et moment dans la formulation quantique du gaz de Lieb-Liniger}

Dans cette section, nous nous intéressons aux opérateurs fondamentaux que sont le {\em nombre total de particules} $\operator{Q}$ et le {\em moment total} $\operator{P}$, dans le cadre du gaz de bosons unidimensionnel décrit par l’Hamiltonien de Lieb-Liniger. Après avoir introduit ces opérateurs dans le langage de la seconde quantification, nous montrons qu’ils sont {\em conservés} par la dynamique, et qu’ils admettent les {\em mêmes états propres} que l’Hamiltonien. Nous donnons ensuite leur expression dans la représentation à  $N$ particules, ainsi que la forme explicite de leurs valeurs propres en fonction des {\em rapidités} $\theta_a$ , illustrant la structure polynomiale typique des intégrales du mouvement dans les systèmes intégrables.

\subsubsection{Définition en seconde quantification}

Les opérateurs du nombre total de particules $\operator{Q}$ et du moment total $\operator{P}$ s’écrivent en seconde quantification comme suit :
\begin{eqnarray}
	\operator{Q}  =  \int \operator{\Psi}^\dag (x) \operator{\Psi} (x) \, d x, \quad 
	\operator{P}  =  - \frac{i}2 \int \left \{  \operator{\Psi}^\dag(x) \operator{\partial}_x \operator{\Psi}(x) - \left [ \operator{\partial}_x \operator{\Psi}^\dag(x)\right ] \operator{\Psi}(x)\right \} dx \label{eq.1.7}
\end{eqnarray}
Ces deux opérateurs sont {\em hermitiens}, et représentent des observables physiques fondamentales : le nombre de particules et la quantité de mouvement du système.

\subsubsection{Conservation et commutation}
Ces opérateurs commutent avec l’Hamiltonien $\operator{H}$ du modèle de Lieb-Liniger :
\begin{eqnarray}
[ \operator{H} , \operator{Q} ] = 0, \quad [ \operator{H} , \operator{P} ] = 0.
\end{eqnarray}
Ils constituent ainsi des intégrales du mouvement. Cette propriété est une manifestation de la symétrie translationnelle du système (pour $\operator{P}$) et de la conservation du nombre total de particules (pour $\operator{Q}$).

\begin{mdframed}[
	linewidth=0.5pt, 
	backgroundcolor=gray!5, 
	roundcorner=50pt,	
	innerleftmargin=5pt,
    innerrightmargin=5pt,
    innertopmargin=5pt,
    innerbottommargin=2pt,
    leftmargin=2pt,
    rightmargin=2pt
	]
	Nous verrons au chapitre 2 que cette situation s’étend à une {\bf \em infinité d’intégrales du mouvement} dans les systèmes intégrables, ce qui permettra de construire l’ensemble de Gibbs généralisé (GGE).
\end{mdframed}

\subsubsection{États propres et valeurs propres}
Les états propres $\ket{\{\theta_a\}}$, construits dans le cadre de la seconde quantification à partir de la solution du modèle de Lieb-Liniger, sont simultanément fonctions propres des opérateurs $\operator{Q}$, $\operator{P}$ et $\operator{H}$ :
\begin{eqnarray}
\operator{Q} \ket{\{\theta_a\}} = N \ket{\{\theta_a\}}, \quad
\operator{P} \ket{\{\theta_a\}} = \left( \sum_{a=1}^N \theta_a \right) \ket{\{\theta_a\}}, \
\operator{H} \ket{\{\theta_a\}} = \left( \frac{1}{2} \sum_{a=1}^N \theta_a^2 \right) \ket{\{\theta_a\}}.
\end{eqnarray}
Autrement dit, les valeurs propres associées à ces trois opérateurs sont données par :
\begin{eqnarray}
N = \sum_{a = 1}^N \theta_a^0, \quad p = \sum_{a = 1}^N \theta_a, \quad e = \frac{1}{2} \sum_{a = 1}^N \theta_a^2.
\end{eqnarray}
Cela illustre que les trois premières intégrales du mouvement du système — nombre, moment, énergie — peuvent être exprimées comme des {\bf \em moments successifs} des rapidités.	

\subsubsection{Forme en première quantification}
En utilisant la représentation en espace de configuration $\{z_a\} \equiv \{z_1 , \cdots , z_N \}$, les opérateurs $\operator{Q}$ et $\operator{P}$ agissent comme suit sur les fonctions d’onde $\varphi_{\{\theta_a\}}(\{z_a\})$ :
\begin{eqnarray}
	\operator{Q}\ket{\{\theta_a\}} =  \sqrt{N!}\int d^Nz \, \operator{\mathcal{N}} \varphi_{\{\theta_a\}}(\{z_a\} )\ket{\{z_a\}}, \, \operator{P}\ket{\{\theta_a\}} =  \sqrt{N!}\int d^Nz \, \operator{\mathcal{P}}_N \varphi_{\{\theta_a\}}(\{z_a\} )\ket{\{z_a\}} 
\end{eqnarray}
où les opérateurs associés agissant sur les fonctions d’onde à $N$ particules sont :
\begin{eqnarray}
	\operator{ \mathcal{N}}  =  \sum_{k = 1}^N 1 = N ,~\operator{ \mathcal{P}}_N  = -i \sum_{k = 1}^N k =- i\sum_{k = 1}^N \operator{\partial}_{z_k}	
\end{eqnarray}

Ces formes découlent directement des règles de commutation canonique (\ref{chap:1:com.1}) et de la définition des opérateurs en seconde quantification (\ref{chap:eq.vide.fock}) (cf. annexes \ref{annex:N.part}).

\subsubsection{Conclusion}
Ainsi, les opérateurs $\operator{Q}$ , $\operator{P}$ et $\operator{H}$ possèdent une structure diagonale commune dans la base des états propres $\ket{\{\theta_a\}}$, révélant la nature intégrable du modèle de Lieb-Liniger. Leurs valeurs propres sont respectivement les 0e, 1er et 2e moments des rapidités. Cette structure permet de généraliser la construction à une hiérarchie complète d’observables conservées, qui seront présentées au chapitre suivant.


\subsection{Fonction d’onde et Hamiltonien et moment à 2 corps}

%Nous considérons à présent le cas de deux bosons quantiques dans la même boîte unidimensionnelle de longueur \(L\), avec des conditions aux limites périodiques. Contrairement au cas à une particule, le terme d’interaction à contact intervient dans la dynamique. L'hamiltonien à 2 particule s'écrit :
%En première quantification, en utilisant les coordonnées du centre de masse et relatives $Z = (z_1 + z_2)/2$ et $Y = z_1 - z_2$, il vient que
%l'hamiltonien (\ref{chap:1:hal.mod.2.part.3}) se divise en une somme de deux problèmes indépendants à une seule particule.
%Les états propres de l'hamiltonien du centre de masse de masse $\overline{m}= 2$, $-\frac{1}{4} \partial_Z^2$, sont des ondes planes, et l'hamiltonien pour la coordonnée relative $Y$ correspond à celui d'une particule de masse réduite $\tilde{m} = 1/2$ en présence d'un potentiel delta en $Y = 0$. 
%\paragraph{Introduction au système à deux bosons avec interaction de contact.}
%Nous considérons à présent le cas de deux bosons quantiques dans une même boîte unidimensionnelle de longueur \(L\), avec des conditions aux limites périodiques. Contrairement au cas à une particule, un terme d’interaction de contact intervient ici dans la dynamique. L’Hamiltonien à deux particules s’écrit :
%\begin{eqnarray}
%	\operator{\mathcal{H}}_2  =  \operator{\mathcal{K}}_2 +\operator{\mathcal{V}}_2  & avec & \operator{\mathcal{K}}_2 =   - \frac{1}{2} \partial_{z_1}^2 - \frac{1}{2} \partial_{z_2}^2,  \quad \mbox{et} \quad  \operator{\mathcal{V}}_2  =  	g  \delta(z_1 - z_2). \label{chap:1:hal.mod.2.part.3} 		
%\end{eqnarray}
%On rappelle que l'énergies propres de  $\operator{\mathcal{K}}_2$ associées aux fonction d'ondes $\varphi_{\{ \theta_1 , \theta_2 \}}$ , la masse des particule étant égale à 1 (ie $\hbar= m=1$) s'écrit 
%\begin{eqnarray}
%	\varepsilon(\theta_1) + 	\varepsilon(\theta_2) & = & \frac{\theta_1^2}{2} + \frac{\theta_2^2}{2} 
%\end{eqnarray}
%On vas travailler dans le centre de masse.

%\paragraph{Changement de variables : coordonnées du centre de masse et relatives.}
 
%En première quantification, en introduisant les coordonnées du centre de masse \(Z = \frac{z_1 + z_2}{2}\) et relative \(Y = z_1 - z_2\), on obtient :
%\(
%	\partial_{z_1}^2 + \partial_{z_2}^2 = \frac{1}{2} \partial_Z^2 + 	2\partial_Y^2.  
%\)
%L’Hamiltonien~\eqref{chap:1:hal.mod.2.part.3} se décompose alors en une somme de deux problèmes indépendants à une seule variable :

%\begin{eqnarray}\label{chap:1:hal.mod.2.part.4}
%	\operator{\mathcal{H}}_2  =  	-\frac{1}{4} \partial_Z^2 + \operator{\mathcal{H}}_{rel} , \quad \mbox{avec}\quad  \operator{\mathcal{H}}_{rel} =  - 	\partial_Y^2 + g \delta ( Y ). 
%\end{eqnarray}

%\paragraph{Résolution du problème de centre de masse et de coordonnée relative.}

%Les états propres de l’Hamiltonien associé au centre de masse, \(-\frac{1}{4} \partial_Z^2\), correspondant à une particule de masse totale \(\bar{m} = 2\), sont des ondes planes associés à l'énergie $\overline{\theta}^2$ avec $\overline{\theta} = \frac{ \theta_1 + \theta_2}{2}$. L’Hamiltonien, $\operator{\mathcal{H}}_{rel}$, associé à la coordonnée relative \(Y\) correspond quant à lui à celui d’une particule de masse réduite \(\tilde{m} = \frac{1}{2}\), soumise à un potentiel delta en \(Y = 0\) :
%\begin{eqnarray}\label{chap:1:hal.mod.2.part.5}
%	- 	\partial_Y^2 \tilde{\varphi}(Y) + g \delta ( Y )\tilde{\varphi}(Y) & = & \tilde{\varepsilon}\,\tilde{\varphi}(Y),
%\end{eqnarray}
%où $\tilde{\varepsilon}$ est l’énergie propre du problème relatif.

%%%%%%
\paragraph{Introduction au système de deux bosons avec interaction de contact.}

Considérons maintenant un système de deux bosons quantiques confinés dans une boîte unidimensionnelle de longueur \(L\), avec des conditions aux limites périodiques. Contrairement au cas à une seule particule, une interaction de contact intervient ici dans la dynamique. L’Hamiltonien à deux particules s’écrit :
\begin{eqnarray}
	\operator{\mathcal{H}}_2 = \operator{\mathcal{K}}_2 + \operator{\mathcal{V}}_2, \quad \text{avec} \quad \operator{\mathcal{K}}_2 = - \frac{1}{2} \partial_{z_1}^2 - \frac{1}{2} \partial_{z_2}^2, \quad \text{et} \quad \operator{\mathcal{V}}_2 = g \, \delta(z_1 - z_2). \label{chap:1:hal.mod.2.part.3}
\end{eqnarray}

On rappelle que, pour des particules de masse unitaire (i.e., \(\hbar = m = 1\)), les énergies propres de l’opérateur cinétique \(\operator{\mathcal{K}}_2\), associées aux fonctions d’onde symétrisées \(\varphi_{\{ \theta_1 , \theta_2 \}}\), sont données par :
\begin{eqnarray}
	\varepsilon(\theta_1) + \varepsilon(\theta_2) = \frac{\theta_1^2}{2} + \frac{\theta_2^2}{2}.
\end{eqnarray}

Afin de simplifier le problème, nous nous plaçons dans le référentiel du centre de masse.

\paragraph{Changement de variables : coordonnées du centre de masse et relative.}

En première quantification, on introduit les nouvelles variables :
\(
Z = \frac{z_1 + z_2}{2} \, \text{(centre de masse)}, \qquad Y = z_1 - z_2 \, \text{(coordonnée relative)}.
\)
Dans ce changement de variables, l’opérateur laplacien total devient :
\(
\partial_{z_1}^2 + \partial_{z_2}^2 = \frac{1}{2} \partial_Z^2 + 2 \, \partial_Y^2.
\)
L’Hamiltonien~\eqref{chap:1:hal.mod.2.part.3} se décompose alors en la somme de deux Hamiltoniens agissant respectivement sur \(Z\) et \(Y\) :
\begin{eqnarray}\label{chap:1:hal.mod.2.part.4}
	\operator{\mathcal{H}}_2 = -\frac{1}{4} \partial_Z^2 + \operator{\mathcal{H}}_{\text{rel}}, \qquad \text{avec} \quad \operator{\mathcal{H}}_{\text{rel}} = - \partial_Y^2 + g \, \delta(Y).
\end{eqnarray}

\paragraph{Résolution du problème du centre de masse et de la coordonnée relative.}

L’Hamiltonien du centre de masse, \(-\frac{1}{4} \partial_Z^2\), décrit une particule de masse totale \(\bar{m} = 2\). Ses états propres sont des ondes planes associées à une énergie \(\overline{\theta}^2\), avec :
\(
\overline{\theta} = \frac{\theta_1 + \theta_2}{2},
\)
jouant ici un rôle analogue à celui d’un pseudo-moment associé dans le référentielle de laboratoire.
Le Hamiltonien relatif, \(\operator{\mathcal{H}}_{\text{rel}}\), correspond quant à lui à une particule de masse réduite \(\tilde{m} = \frac{1}{2}\) soumise à un potentiel delta centré en \(Y = 0\). Son équation propre s’écrit :
\begin{eqnarray}\label{chap:1:hal.mod.2.part.5}
	- \partial_Y^2 \, \tilde{\varphi}(Y) + g \, \delta(Y) \, \tilde{\varphi}(Y) = \tilde{\varepsilon} \, \tilde{\varphi}(Y),
\end{eqnarray}
où \(\tilde{\varepsilon}\) désigne l’énergie associée au mouvement relatif.
%%%%%%%%%%%%%%

\paragraph{Forme symétrique de la fonction d'onde pour bosons.}
Dans le référentiel du centre de masse. Le système est le même que que celuis d'un particules de masse $\tilde{m}= \frac{1}{2}$.
Le système étant composé de particules bosoniques, on cherche une solution symétrique que l’on écrit sous la forme  :
\begin{eqnarray}
	\tilde{\varphi}(Y) ~=~a~e^{i\frac{1}{2} \tilde{\theta} \vert Y \vert } + b~e^{-i\frac{1}{2} \tilde{\theta}\vert Y \vert } ~\propto~  \sin\left( \frac{1}{2} (\tilde{\theta} |Y| + \Phi ) \right). \label{eq:ansatz.boson}
\end{eqnarray}
Le paramètre \( \tilde{\theta} = \theta_1 - \theta_2 \) joue ici un rôle analogue à celui d’un pseudo-moment associé à la coordonnée relative,
est  la phase s'écrit
\begin{eqnarray}
	\Phi(\tilde{\theta}) &=& 2 \arctan\left (\frac{1}{i} \frac{a+b}{a-b}\right),	\label{chap:1:dif.mod.2.part.1} 
\end{eqnarray}
car \( a\exp(ix)+b\exp(-ix) = 2\sqrt{ab}\sin\left(x+\arctan\left(-i\, \frac{a+b}{a-b}\right)\right) \). Pour $\tilde{\theta}<0$, les termes exponentiels \( \exp(i\tilde{\theta} \vert Y \vert/2 ) \) et \( \exp(-i\tilde{\theta} \vert Y \vert/2 ) \) correspondent aux paires de particules entrantes et sortantes d’un processus de diffusion à deux corps.


%En réinjectant l'équation \eqref{eq:ansatz.boson} dans l’équation \eqref{chap:1:hal.mod.2.part.5}, on obtient l’énergie propre du problème réduit $\tilde{\varepsilon}$ associé à l’état lié. Celle-ci prend la forme classique de l’énergie cinétique d’une particule, \( \frac{1}{2} \times \text{masse} \times \text{vitesse}^2 \), la masse réduite du problème étant ici \( \tilde{m} = \frac{1}{2} \), et où \( \tilde{\theta} \) joue un rôle analogue à celui d’une vitesse. On en déduit :
%\begin{eqnarray}\tilde{\varepsilon}(\tilde{\theta})  & = &  \frac{1}{2} \cdot \tilde{m} \cdot \tilde{\theta}^2 = \frac{1}{2} \cdot \frac{1}{2} \cdot \tilde{\theta}^2 = \frac{\tilde{\theta}^2}{4}.\end{eqnarray}
%\begin{eqnarray}
%	\tilde{\varepsilon}(\tilde{\theta})  & = &  \frac{\tilde{\theta}^2}{4}.
%\end{eqnarray}
% Il encode la décroissance exponentielle de la fonction d’onde liée dans l’espace relatif, et sa valeur est directement reliée à la profondeur de l’état lié. Une valeur plus grande de \( \tilde{\theta} \) correspond à un état plus fortement lié, c’est-à-dire plus localisé autour de \( Y = 0 \), ce qui reflète une interaction plus attractive entre les deux particules. $\overline{\theta}^2 +  \tilde{\varepsilon}(\tilde{\theta}) = \varepsilon{\theta_1} + \varepsilon{\theta_2}$.
En réinjectant l’ansatz~\eqref{eq:ansatz.boson} dans l’équation relative
\eqref{chap:1:hal.mod.2.part.5}, on obtient l’énergie propre
\(\tilde{\varepsilon}\) du problème réduit.  
Elle prend la forme cinétique usuelle
\(\tfrac{1}{2}\times\text{masse}\times\text{vitesse}^{2}\).  
La masse réduite vaut ici \(\tilde{m}= \frac{1}{2}\) et le paramètre
\(\tilde{\theta}\) joue le rôle d’une impulsion ; ainsi
\begin{equation}
   \tilde{\varepsilon}(\tilde{\theta})
   \;=\;
   \frac{1}{2}\,\tilde{m}\,\tilde{\theta}^{2}
   \;=\;
   \frac{1}{2}\times\frac{1}{2}\,\tilde{\theta}^{2}
   \;=\;
   \frac{\tilde{\theta}^{2}}{4}.
   \label{eq:energie_relative}
\end{equation}

Cette énergie gouverne la décroissance exponentielle de la fonction
d’onde dans la coordonnée relative : plus \(\tilde{\theta}\) est grand,
plus l’état est localisé autour de \(Y=0\), signe d’une interaction
attractive plus forte entre les deux bosons.

L’énergie totale se décompose enfin en la somme du mouvement du centre
de masse et du mouvement relatif :
\(
   \overline{\theta}^{2}
   \;+\;
   \tilde{\varepsilon}(\tilde{\theta})
   \;=\;
   \varepsilon(\theta_{1})
   \;+\;
   \varepsilon(\theta_{2}),
\)
où \(\overline{\theta}= \tfrac{\theta_{1}+\theta_{2}}{2}\) et
\(\varepsilon(\theta)=\theta^{2}/2\).






%%%%%%%%%%%%%%%%%%%%%%%%%%%
\paragraph{Condition de discontinuité à cause du potentiel delta.}
En raison de la présence du potentiel delta centré en $Y = 0$, la dérivée première de la fonction d’onde $\tilde{\varphi}(Y)$ présente une discontinuité en ce point. En effet, le potentiel étant infini en $Y = 0$, la phase $\Phi$ du régime symétrique est déterminée en intégrant l’équation du mouvement autour de la singularité. En intégrant entre $- \epsilon$ et $+ \epsilon$ et en faisant tendre $\epsilon \to 0$, on obtient la condition de saut de la dérivée :

%avec $\Phi$ une phase à déterminer. %\begin{equation}
%	E = \frac{\tilde{m} \theta^2}{2}.
%\end{equation}

%La dérivée de la fonction d’onde n’est pas continue en $Y = 0$. Le potentiel étant infini en $Y = 0$, la phase $\Phi$ est obtenue en intégrant l’équation du mouvement entre $- \epsilon$ et $+ \epsilon$ et en faisant tendre $\epsilon$ vers zéro :


%En raison de ce potentiel delta, la dérivée première de la fonction d'onde $\varphi(Y)$ doit avoir une discontinuité en $Y = 0$ : 

%{\color{lightgray} 
%\begin{eqnarray*}
%	\underset{ \epsilon \to 0 }{\lim} \int_{-\epsilon}^{+\epsilon}  	-\underbrace{\cancel{\frac{1}{4} \partial_Z^2\varphi(Y)}}_{0} - 	\partial_Y^2\varphi(Y) + c \delta ( Y )\varphi(Y) \, dY  & = & \underset{ \epsilon \to 0 }{\lim}  \int_{-\epsilon}^{+\epsilon}  E d Y , \\
%	\underset{ \epsilon \to 0 }{\lim}  \left [ \varphi'(\epsilon) - \varphi'(-\epsilon) \right ] - c \varphi (  0 ) & =  &  -\underset{ \epsilon \to 0 }{\lim}  \int_{-\epsilon}^{+\epsilon}  E d Y,\\
%	 \varphi'(0^+) - \varphi'(0^-) - c \varphi (  0 ) & = & 0 .
%\end{eqnarray*}


%}

\begin{eqnarray*}
	\underset{ \epsilon \to 0 }{\lim} \int_{-\epsilon}^{+\epsilon}  - 	\partial_Y^2\tilde{\varphi}(Y) + g \delta ( Y )\tilde{\varphi}(Y) \, dY  & = & \underset{ \epsilon \to 0 }{\lim}  \int_{-\epsilon}^{+\epsilon}  \tilde{\varepsilon}(\tilde{\theta})d Y ,\\
	\\
	\tilde{\varphi}'(0^+) - \tilde{\varphi}'(0^-) - g \tilde{\varphi} (  0 ) & = & 0 .
\end{eqnarray*}


%soit $\tilde{\varphi}'(0^+) - \tilde{\varphi}'(0^-) - c \tilde{\varphi} (  0 )  =  0 $ .

%%%%%%%%%%%%%%%
\paragraph{Détermination de la phase $\Phi$.}
Et en évaluant la discontinuité de sa dérivée au point $Y = 0$, on trouve que la phase $\Phi$ satisfait la condition :

%\begin{equation}
%	\tan\left( \frac{\Phi}{2} \right) = \frac{\tilde{\theta}}{c}.
%\end{equation}

\begin{eqnarray}\label{chap:1:dif.mod.2.part.2}
	\Phi(\tilde{\theta}) & = & 2 \arctan (\tilde{\theta}/g) \in [ - \pi , +\pi ].
\end{eqnarray}

%{\color{red}( à revoir)} Cette relation exprime l’impact de l’interaction delta sur le déphasage de la solution liée. On en déduit que plus le couplage $g$ est fort ($g \to \infty$), plus la phase $\Phi$ se rapproche de $0$, ce qui correspond à une fonction d’onde présentant s'annulant en $Y = 0$. En revanche, dans la limite d’interaction faible ($g \to 0$), la phase $\Phi$ tend vers $\pm \pi$ et la discontinué de la dérivé de la fonction d'onde devient négligeable.
%Cette relation exprime l’impact de l’interaction de type delta sur le déphasage de la fonction d’onde liée.On en déduit que plus le couplage $g$ est fort ($g \to \infty$), la phase $\Phi$ se rapproche de $0$, ce qui correspond à une fonction d’onde présentant s'annulant en $Y = 0$, à l’image du régime d’imperméabilité totale.
%À l’inverse, dans la limite d’interaction faible (\( g \to 0 \)), si bien que \( \Phi \) tend vers $\pi$ (ou \( -\pi \), selon le signe de \( \tilde{\theta} \)). Dans ce cas, la discontinuité de la dérivée de la fonction d’onde au point \( Y = 0 \) devient négligeable, ce qui traduit un couplage quasi inexistant entre les deux particules.
%Cette relation exprime l’impact de l’interaction de type delta sur le déphasage de la fonction d’onde liée. Lorsque le couplage \( g \) devient très fort (\( g \to \infty \)), la fraction \( \tilde{\theta}/g \to 0 \), et la phase \( \Phi(\tilde{\theta}) \to 0 \). Cela correspond à une situation dans laquelle la fonction d’onde est fortement contrainte à s’annuler en \( Y = 0 \), à l’image du régime d’imperméabilité totale.
%À l’inverse, dans la limite d’interaction faible (\( g \to 0 \)), la fraction \( \tilde{\theta}/g \to \infty \), si bien que \( \Phi(\tilde{\theta}) \to \pi \) (ou \( -\pi \), selon le signe de \( \tilde{\theta} \)). Dans ce cas, la discontinuité de la dérivée de la fonction d’onde au point \( Y = 0 \) devient négligeable, ce qui traduit un couplage quasi inexistant entre les deux particules.

Cette relation exprime l’impact de l’interaction de type delta sur le déphasage de la fonction d’onde liée. On en déduit que plus le couplage \( g \) est fort (\( g \to \infty \)), plus la phase \( \Phi \) se rapproche de zéro. Cela correspond à une fonction d’onde qui s’annule en \( Y = 0 \), caractéristique d’un régime d’imperméabilité totale.

À l’inverse, dans la limite d’une interaction faible (\( g \to 0 \)), la phase \( \Phi \) tend vers \( \pi \) (ou \( -\pi \), selon le signe de \( \tilde{\theta} \)). Dans ce cas, la discontinuité de la dérivée de la fonction d’onde au point \( Y = 0 \) devient négligeable, ce qui traduit une interaction presque absente entre les deux particules.


%%%%%%%%%%%%%%%%%%%%%%%%%%%%%%%%%%%
%\paragraph{Phase de diffusion à un corp.}
%Les équations \eqref{chap:1:dif.mod.2.part.1} et \eqref{chap:1:dif.mod.2.part.2}  et en remarquant que pour $z \in \mathbb{C} \backslash \{ \pm i \} 2\artan(z) = i \ln \left( \frac{ 1 - i z }{1+iz} \right ) $ soit $\exp(2i\arctan(x)) = (1 + ix)/(1 - ix)$ et $\Phi(\tilde{\theta}) = i \ln ( - b/a ) $  donne rapport entre les amplitudes $a$ et $b$ de la fonction d'onde \eqref{eq:ansatz.boson} définit la phase de diffusion / {\em matrice diffusion} $S( \tilde{\theta}) \doteq e^{i\Phi ( \tilde{\theta}  ) }$  :

%\begin{eqnarray}
%	e^{i\Phi ( \tilde{\theta}  ) } &=& -\frac{a}{b} ~=~\frac{1 +i\tilde{\theta}/g} { 1 - i\tilde{\theta}/g} .\label{chap:1:dif.mod.2.part.3}
%\end{eqnarray}

\paragraph{Phase de diffusion à deux corps.}

En combinant les équations~\eqref{chap:1:dif.mod.2.part.1} et~\eqref{chap:1:dif.mod.2.part.2} avec l’identité analytique valable pour tout
\(z \in \mathbb{C}\setminus\{\pm i\}\),
\(
2\arctan(z)=i\ln\!\left(\frac{1-iz}{1+iz}\right)
\Leftrightarrow
e^{2i\arctan(z)}=\frac{1+iz}{1-iz},
\)
on obtient que le rapport des amplitudes \(a\) et \(b\) de la fonction
d’onde relative~\eqref{eq:ansatz.boson} définit la {\em phase de diffusion }
\(
\Phi(\tilde{\theta}) = i\ln\!\left(-\frac{b}{a}\right).
\)
On introduit alors la {\em matrice de diffusion} (ou facteur de diffusion)
\begin{eqnarray}
	S(\tilde{\theta}) \;\doteq\; e^{i\Phi(\tilde{\theta})}= -\frac{a}{b}= \frac{1 + i\,\tilde{\theta}/g}{1 - i\,\tilde{\theta}/g}.%\tag{\ref{chap:1:dif.mod.2.part.3}}
\end{eqnarray}
%où \(g\) est le paramètre d’interaction et
%\(\tilde{\theta} = \theta_1 - \theta_2\) le pseudo‑moment relatif.  
Cette expression, unitaire et analytique, caractérise entièrement la diffusion élastique à deux corps dans le modèle considéré.



\paragraph{Lien entre phase de diffusion et décalage temporel : interprétation semi-classique. {\color{red}(à revoir)}}

Il a été souligné par {\color{black}Eisenbud (1948)} et {\color{black}Wigner (1955)} que la phase de diffusion peut être interprétée, de manière semi-classique, comme un {\em décalage temporel}. Esquissons brièvement l'argument de {\color{black}Wigner (1955)}.Tout d'abord, notons que, pour une particule unique, une approximation simple d’un paquet d’ondes peut être obtenue en superposant deux ondes planes avec des moments $\tilde{\theta}/2$ et $\tilde{\theta}/2 + \delta \tilde{\theta}$, respectivement :
\begin{eqnarray}
	\tilde{\varphi}_{\text{inc}}(Y) & \propto & e^{i\frac{1}{2}\tilde{\theta} \vert Y\vert} + e^{i\frac{1}{2}\left(\tilde{\theta} + 2\delta \tilde{\theta} \right) \vert Y\vert}.
\end{eqnarray}
Cette superposition évolue dans le temps comme :
\begin{eqnarray}
\tilde{\varphi}_{\text{inc}}(Y, t) &\propto &  e^{i\left( \frac{1}{2} \tilde{\theta}\vert Y\vert - t\,\tilde{\varepsilon}(\tilde{\theta}) \right)} + e^{i\left( \frac{1}{2}\left(  \tilde{\theta} + 2\delta \tilde{\theta} \right) \vert Y\vert - t\,\tilde{\varepsilon}(\tilde{\theta} + 2\delta \tilde{\theta}) \right)}.
\end{eqnarray}
%où l'on a utilisé l'expression de l'énergie réduite : $\tilde{\varepsilon}(\tilde{\theta}) = \tilde{\theta}^2 / 4$.
Le centre de ce 'paquet d'ondes' se situe à la position où les phases des deux termes coïncident, c'est-à-dire au point où $\vert Y\vert\delta \tilde{\theta}  - t[\tilde{\varepsilon}(\tilde{\theta} + 2\delta \tilde{\theta} ) - \tilde{\varepsilon}(\tilde{\theta})] = 0$, ce qui donne $\vert Y\vert \simeq \tilde{\theta} t$ avec la vitesse réduite $\tilde{\theta} = 1/2 \varepsilon'(\tilde{\theta}) $. %Ainsi, il s'agit effectivement d'un 'paquet d'ondes' se déplaçant à la vitesse $\theta$. Ensuite, considérons deux particules entrantes dans un état tel que le centre de masse $Z = (z_1 + z_2)/2$ ait une impulsion $\theta_1 - \theta_2$, tandis que la coordonnée relative $Y = z_1 - z_2$ se trouve dans un 'paquet d'ondes' se déplaçant à la vitesse $ (\theta_1 - \theta_2)/2$,
Selon les équations (\ref{eq:ansatz.boson}) et (\ref{chap:1:dif.mod.2.part.3}), l'état sortant de la diffusion correspondant serait :
\begin{eqnarray}
	\tilde{\varphi}_{outc} ( Y, t ) & \propto & -e^{i\Phi(\tilde{\theta})}e^{-i\frac{1}{2}\tilde{\theta} \vert Y\vert} - e^{i\Phi(\tilde{\theta} + 2 \delta \tilde{\theta} )}e^{-i\frac{1}{2}\left(\tilde{\theta} + 2\delta \tilde{\theta} \right) \vert Y\vert}. %\tag{2}
\end{eqnarray}
En répétant l'argument précédent de la stationnarité de phase, on trouve que la coordonnée relative est à la position $\vert Y \vert  \simeq \tilde{\theta} t - 2\Phi'( \tilde{\theta})$ au moment $t$. %Étant donné que le centre de masse n'est pas affecté par la collision et se déplace à la vitesse de groupe $\tilde{\theta} =(\theta_1 + \theta_2)/2$, nous constatons que la position des deux particules semiclassiques après la collision sera
\begin{eqnarray}
	\vert Y \vert & \simeq & 	\tilde{\theta} t  - 2 \Delta (\tilde{\theta} )
\end{eqnarray}
où le déplacement de diffusion $\Delta (\theta)$ est donné par la dérivée de la phase de diffusion,
\begin{eqnarray}\label{eq:I-1-16}
	\Delta ( \theta ) & \doteq & \frac{ d \Phi }{ d \theta } ( \theta )= \frac{ 2 g }{ g^2 + \theta^2} . 	
\end{eqnarray}


%\paragraph{Retour aux coordonnées du laboratoire.}
%En revenant aux coordonnées d'origine (celles du laboratoire), on constate que la fonction d'onde à deux corps 
%\(
%	\varphi_{\{\theta_1 , \theta_2\}} (z_1, z_2) = \langle \emptyset \vert \operator{\Psi} (z_1)\operator{\Psi} (z_2) \vert \{\theta_1, \theta_2\} \rangle,
%\)
%avec \(z_1 < z_2\) , (ie $Y>0$) . Et le centre de masse sur le mouvement
%\(
%	Z  =  \overline{\theta} t.	
%\)
%avec,  on rappelle , $\overline{\theta}$ la vitesse de groupe dans le référentielle de laboratoire.\\
%Nous constatons que la position des deux particules semiclassiques après la collision sera
%\begin{eqnarray}
%	z_1 ~=~ Z + \frac{Y}2 ~\simeq ~ \theta_1 t - \Delta(\theta_1 - \theta_2), & & 	z_2 ~=~ Z - \frac{Y}2 ~\simeq ~ \theta_2t + \Delta(\theta_1 - \theta_2),
%\end{eqnarray}

%avec  $\theta_1$ et $\theta_2$ on rappelle définie tel que 
%\(
%	\tilde{\theta} ~=~\theta_1 - \theta_2 , \,	\overline{\theta}~=~\frac{\theta_1 + \theta_2}{2}.	
%\)
%On remarquant que 
%\begin{eqnarray*}
%	z_1 \theta_1  + z_2  \theta_2 ~=~ 2Z\overline{\theta} + \frac{1}{2}Y\tilde{\theta}, & & z_1 \theta_2  + z_2  \theta_1 ~=~ 2Z\overline{\theta} - \frac{1}{2}Y\tilde{\theta}. 
%\end{eqnarray*}
%Ce qui est en accod avec la masse total $\overline{m} = 2$ et la masse résuite $\tilde{m} = \frac{1}{2}$ \\
%Ce qui nous motive à multiplier la fonction d'onde dans le référentiel du centre de masse \eqref{eq:ansatz.boson} par $\exp(2iZ\overline{\theta})$ pour obtenir 

%\begin{eqnarray}\label{eq:I-1-10}
%	\varphi_{\{\theta_1 , \theta_2\}}(z_1 , z_2) & \propto &  \left \{ \begin{array} { c cl} ( \theta_2 - \theta_1 - ic) e^{ i z_1 \theta_1 + iz_2 \theta_2 } - ( \theta_1 - \theta_2 - ic) e^{ i z_1 \theta_2 + iz_2 \theta_1} & \mbox{si} & z_1 < z_2 \\ (z_1 \leftrightarrow z_2) & \mbox{si} & z_1 > z_2 \end{array} \right.
%\end{eqnarray}

%correspondant aux valeurs propres

%\begin{eqnarray}
%	\varepsilon(\theta_1 , \theta_2) ~=~ \overbrace{ \overline{\theta}^2}^{\overline{\varepsilon}(\overline{\theta})}	 + \overbrace{\frac{1}{4} \tilde{\theta}^2}^{\tilde{\varepsilon}(\tilde{\theta})} ~=~ \frac{\theta_1}{2} + \frac{\theta_2}{2}.	
%\end{eqnarray}

%Pour $\theta_1 > \theta_2$, les deux termes $e^{iz_1 \theta_1 + iz_2 \theta_2 }$ et $e^{iz_1 \theta_2 + iz_2 \theta_1 }$ correspondent aux paires de particules entrantes et sortantes dans un processus de diffusion à deux corps. Le rapport de leurs amplitudes est la phase de diffusion à deux corps \eqref{chap:1:dif.mod.2.part.3} reste inchangé

%\begin{eqnarray}\label{chap:1:dif.mod.2.part.4}
%	e^{i\Phi ( \theta_1 - \theta_2  ) }~=~ -\frac{a}{b} ~=~\frac{\theta_1 - \theta_2  -ic} { \theta_2 - \theta_1  - ic}. 
%\end{eqnarray}


%%%%%%%%%%%%%%%%%%%%%%%%%%
\paragraph{Retour aux coordonnées du laboratoire.}

En revenant aux coordonnées du laboratoire, la fonction d’onde à deux corps s’écrit
\(
	\varphi_{\{\theta_1 , \theta_2\}} (z_1, z_2) 
	= \langle \emptyset \vert \operator{\Psi} (z_1)\operator{\Psi} (z_2) \vert \{\theta_1, \theta_2\} \rangle/\sqrt{2},
\)
dans le cas \(z_1 < z_2\), c’est-à-dire pour une séparation relative \(Y = z_1 - z_2 < 0\) (on pourra symétriser ultérieurement).  
Dans le référentiel du laboratoire, le centre de masse évolue selon
\(
	Z = \frac{z_1 + z_2}{2} = \overline{\theta}\,t.
\)
%où l’on rappelle que \(\overline{\theta} = \frac{\theta_1 + \theta_2}{2}\) est la vitesse de groupe du système dans le référentiel laboratoire.
Ainsi, la position semi-classique des deux particules après la collision s’écrit
\begin{eqnarray}
	z_1 = Z + \frac{Y}{2} \;\simeq\; \theta_1 t - \Delta(\theta_1 - \theta_2),\quad
	z_2 = Z - \frac{Y}{2} \;\simeq\; \theta_2 t + \Delta(\theta_1 - \theta_2),
\end{eqnarray}
%où \(\Delta(\theta_1 - \theta_2)\) représente le décalage dû à l’interaction entre les deux particules.
%On rappelle les définitions :
%\[
%	\tilde{\theta} = \theta_1 - \theta_2, 
%	\quad
%	\overline{\theta} = \frac{\theta_1 + \theta_2}{2}.
%\]
On peut vérifier les identités utiles suivantes :
\begin{eqnarray*}
	z_1 \theta_1 + z_2 \theta_2 = 2Z \overline{\theta} + \frac{1}{2} Y \tilde{\theta}, \quad
	z_1 \theta_2 + z_2 \theta_1 &=& 2Z \overline{\theta} - \frac{1}{2} Y \tilde{\theta},
\end{eqnarray*}
ce qui est en accord avec les masses associées : masse totale \(\overline{m} = 2\), masse réduite \(\tilde{m} = \frac{1}{2}\).

Cela nous motive à multiplier l’ansatz dans le référentiel du centre de masse (équation~\eqref{eq:ansatz.boson}) par un facteur de phase globale \(\exp(2iZ\overline{\theta})\) pour revenir à la représentation dans le laboratoire. On obtient alors l’expression de la fonction d’onde :
\begin{eqnarray}\label{eq:I-1-10}
	\varphi_{\{\theta_1 , \theta_2\}}(z_1 , z_2) & \propto &  \left \{ \begin{array} { c cl} ( \theta_2 - \theta_1 - ig) e^{ i z_1 \theta_1 + iz_2 \theta_2 } - ( \theta_1 - \theta_2 - ig) e^{ i z_1 \theta_2 + iz_2 \theta_1} & \mbox{si} & z_1 < z_2 \\ (z_1 \leftrightarrow z_2) & \mbox{si} & z_1 > z_2 \end{array} \right.
\end{eqnarray}

%Cette fonction d’onde correspond à une valeur propre d’énergie donnée par la somme des énergies associées aux deux degrés de liberté :

%\begin{equation}
%	\varepsilon(\theta_1 , \theta_2) 
%	= \underbrace{\overline{\theta}^2}_{\overline{\varepsilon}(\overline{\theta})}
%	+ \underbrace{\frac{1}{4} \tilde{\theta}^2}_{\tilde{\varepsilon}(\tilde{\theta})}
%	= \frac{\theta_1^2}{2} + \frac{\theta_2^2}{2}.
%\end{equation}

Pour \(\theta_1 > \theta_2\), les deux termes exponentiels 
\(e^{i z_1 \theta_1 + i z_2 \theta_2}\) et \(e^{i z_1 \theta_2 + i z_2 \theta_1}\)
correspondent respectivement aux ondes entrantes et sortantes dans le canal de diffusion à deux corps.  
Le rapport de leurs amplitudes définit la phase de diffusion / matrice diffusion $e^{i\Phi ( \tilde{\theta}  ) }$  à deux corps \eqref{chap:1:dif.mod.2.part.3} , reste inchangé :

\begin{equation}\label{chap:1:dif.mod.2.part.4}
	S(\theta_1- \theta_2) \doteq e^{i\Phi(\theta_1 - \theta_2)} 
	= \frac{\theta_1 - \theta_2 - ig}{\theta_2 - \theta_1 - ig}.
\end{equation}

Cette phase caractérise entièrement le processus de diffusion dans le modèle de Lieb-Liniger à deux particules.

\paragraph{Conditions périodiques et équations de Bethe pour deux bosons {\color{red}(à révoir)}.}

%La fonction d’onde obtenue par Bethe ansatz (voir
%\eqref{eq:I-1-10}) est, pour $z_{1}<z_{2}$,
%\[
%	\varphi_{\{\theta_{1},\theta_{2}\}}(z_{1},z_{2})
%		= a\,e^{i\theta_{1}z_{1}+i\theta_{2}z_{2}}
%		+b\,e^{i\theta_{2}z_{1}+i\theta_{1}z_{2}},
%	\quad
%	a=\theta_{2}-\theta_{1}-ic,\;
%	b=-(\theta_{1}-\theta_{2}-ic).
%\]

%\medskip
%\subparagraph{Périodicité sur $z_{2}$.}  
%On impose à la fonction d’onde obtenue par Bethe ansatz (voir
%\eqref{eq:I-1-10})
%\(
%	\varphi_{\{\theta_{1},\theta_{2}\}}(z_{1},z_{2}\!=\!L)
%	=
%	\varphi_{\{\theta_{1},\theta_{2}\}}(z_{1},z_{2}\!=\!0)
%\)
%avec $0<z_{1}<z_{2}=L$.  
%Au point $z_{2}=L$ on reste dans le secteur $z_{1}<z_{2}$, tandis qu’au point $z_{2}=0$ le domaine pertinent devient $z_{2}<z_{1}$;  la fonction d’onde y est obtenue en échangeant $z_{1}\leftrightarrow z_{2}$ , soit 
%\(
%	\varphi_{\{\theta_{1},\theta_{2}\}}(z_{1},\!L)
%	=
%	\varphi_{\{\theta_{1},\theta_{2}\}}(0 , z_{1})
%\)
%.  
%On obtient ainsi
%\begin{eqnarray*}
%	a\,e^{i\theta_{1}z_{1}+i\theta_{2}L}+b\,e^{i\theta_{2}z_{1}+i\theta_{1}L} & = &
%	a\,e^{i\theta_{2}z_{1}}\,e^{i\theta_{1}\! \cdot 0} + b \,e^{i\theta_{1}z_{1}}\,e^{i\theta_{2}\! \cdot 0},	
%\end{eqnarray*}
%avec la condition $z_1< z_2$, avec le rapport $a$ et $b$ vérifiant \eqref{chap:1:dif.mod.2.part.4} de la sorte $-b/a = e^{i\Phi(\theta_1 - \theta_2)}$ .

%%%%%%%%%%%%%%%%

\subparagraph{Périodicité en \( z_2 \).}  
On impose une condition de périodicité sur la fonction d’onde obtenue par ansatz de Bethe (voir équation~\eqref{eq:I-1-10}) :
\(
	\varphi_{\{\theta_1,\theta_2\}}(z_1, z_2 = L) = \varphi_{\{\theta_1,\theta_2\}}(z_1, z_2 = 0),
\)
avec \( 0 < z_1 < z_2 = L \).  
Au point \( z_2 = L \), la configuration reste dans le secteur \( z_1 < z_2 \), tandis qu’à \( z_2 = 0 \), on entre dans le secteur \( z_2 < z_1 \). La continuité de la fonction d’onde impose alors d’échanger les coordonnées \( z_1 \leftrightarrow z_2 \) :
\(
	\varphi_{\{\theta_1,\theta_2\}}(z_1, L) = \varphi_{\{\theta_1,\theta_2\}}(0, z_1).
\)
En utilisant l’expression explicite de l’ansatz dans les deux secteurs, on obtient l’égalité suivante :
\begin{eqnarray*}
	a\,e^{i\theta_1 z_1 + i\theta_2 L} + b\,e^{i\theta_2 z_1 + i\theta_1 L}
	&=& a\,e^{i\theta_2 z_1} + b\,e^{i\theta_1 z_1}.
\end{eqnarray*}
%où le second membre correspond à la fonction d’onde dans le secteur \( z_2 < z_1 \), évaluée en \( z_2 = 0 \) et \( z_1 = z_1 \).  
%La condition de périodicité impose donc :
%\[
%	a\,e^{i\theta_1 z_1 + i\theta_2 L} + b\,e^{i\theta_2 z_1 + i\theta_1 L}
%	= a\,e^{i\theta_2 z_1} + b\,e^{i\theta_1 z_1}.
%\]
Cette relation, valable pour tout \( z_1 \in (0,L) \), fixe une contrainte sur le rapport \( b/a \). En utilisant l’expression de la phase de diffusion introduite en \eqref{chap:1:dif.mod.2.part.4} pour $z_1<z_2$ :
\begin{eqnarray*}
	-\frac{b}{a} = e^{i\Phi(\theta_1 - \theta_2)},
\end{eqnarray*}
on obtient une condition quantique sur les phases \( \theta_1 \) et \( \theta_2 \), cœur de la quantification imposée par le formalisme de Bethe.

%\[
%	( \theta_2 - \theta_1 - ig)\,e^{i\theta_{1}z_{1}+i\theta_{2}L}
%	- ( \theta_1 - \theta_2 - ig)\,e^{i\theta_{2}z_{1}+i\theta_{1}L}
%	=
%	( \theta_2 - \theta_1 - ig)\,e^{i\theta_{2}z_{1}}\,e^{i\theta_{1}\! \cdot 0}
%	- ( \theta_1 - \theta_2 - ig)\,e^{i\theta_{1}z_{1}}\,e^{i\theta_{2}\! \cdot 0}.
%\]
En identifiant les coefficients de $e^{i\theta_{1}z_{1}}$ et
$e^{i\theta_{2}z_{1}}$ indépendamment, on obtient
\(
	e^{i\theta_{2}L}\;a = b, 
	\,
	e^{i\theta_{1}L}\;b = a,
\)
c’est‑à‑dire l'équations de Bethe
%\begin{equation}\label{eq:PC2}
%	e^{i\theta_{2}L} = \frac{b}{a}
%	= \frac{\theta_{1}-\theta_{2}+ic}{\theta_{2}-\theta_{1}+ic},
%\quad
%	e^{i\theta_{1}L} = \frac{a}{b}
%	= \frac{\theta_{2}-\theta_{1}+ic}{\theta_{1}-\theta_{2}+ic}.
%\end{equation}
\begin{eqnarray*}\label{eq:PC2}
	e^{i\theta_{1}L}\,e^{i\Phi(\theta_{1}-\theta_{2})} = -1,
	\qquad
	e^{i\theta_{2}L}\,e^{i\Phi(\theta_{2}-\theta_{1})} = -1.	
\end{eqnarray*}
En prenant le logarithme on obtient les \emph{équations de Bethe à deux
particules} :
\begin{equation}\label{eq:Bethe2}
	\theta_{1}L + \Phi(\theta_{1}-\theta_{2}) = 2\pi I_{1}, 
	\qquad
	\theta_{2}L + \Phi(\theta_{2}-\theta_{1}) = 2\pi I_{2},
\end{equation}
où $I_{1},I_{2}\in\mathbb{Z}$ sont les nombres quantiques entiers
(caractère bosonique). 

\subparagraph{Périodicité sur $z_{1}$.}  Le raisonnement symétrique conduit exactement aux mêmes égalités \eqref{eq:PC2}.  
\bigskip
Les équations \eqref{eq:Bethe2} constituent la quantification complète
du gaz de Lieb–Liniger à deux bosons sur un cercle de longueur $L$ et
seront le point de départ pour l’étude de l’état fondamental et des
excitations.



\begin{figure}[H]
	\centering
  %\includegraphics[width=0.5\textwidth]{}
  %\caption{Gauche : La fonction d'onde (\ref{eq:I-1-10}) sur la ligne infinie correspond à un processus de diffusion à deux corps. Semiclassiquement, la phase de diffusion dans ce processus à deux corps se reflète dans le décalage de diffusion (\ref{eq:I-1-16}) : après la collision, la position de la particule a été déplacée d'une distance $\Delta ( \theta_1 - \theta_2 )$ . Droite : La fonction d'onde de Bethe (\ref{eq:I-2-17}) sur la ligne infinie correspond à un processus de diffusion à $N$-corps qui se factorise en des processus à deux corps (le décalage de diffusion $\Delta$ est également présent ici, mais il n'est pas représenté dans la caricature). Dans ce processus à $N$-corps, les rapidités $\theta_j$ sont les moments asymptotiques des bosons.}
  \label{}	
\end{figure}



\section{Équation de Bethe et distribution de rapidité}

\subsection{Fonction d’onde dans le secteur ordonné et représentation de Gaudin}

Dans le domaine $z_1 < z_2 < \cdots < z_N$, la fonction d’onde pour un état de Bethe à $N$ particules s’écrit ({\color{blue}Gaudin 2014}, {\color{blue}Korepin et al. 1997}, {\color{black}Lieb et Liniger 1963}) :
\begin{eqnarray}
	\varphi_{\{\theta_a\}} ( z_1 , \cdots , z_N ) & = &  \frac{1}{\sqrt{N!}}\langle \emptyset \vert \operator{\Psi} ( z_1 ) \cdots \operator{\Psi} (z_N ) \vert \{ \theta_a \} \rangle \notag\\
	& \propto & \sum_\sigma (-1)^{|\sigma|} \left( \prod_{1 \leq a < b \leq N} (\theta_{\sigma(b)} - \theta_{\sigma(a)} - i g) \right) e^{i \sum_{j=1}^{N} z_j \theta_{\sigma(j)}},\label{eq:I-2-17}
\end{eqnarray}
où la somme s'étend sur toutes les permutations $\sigma$ de $\{1,\dots,N\}$. Le facteur $(-1)^{|\sigma|}$ est la signature de la permutation, et les amplitudes dépendent des différences de quasi-moments $\theta_j$ ainsi que du couplage $c$.
Cette fonction d’onde est ensuite étendue par symétrie aux autres domaines du type $z_{\pi(1)} < z_{\pi(2)} < \cdots < z_{\pi(N)}$ via des propriétés d’échange symétriques.

\vspace{1em}

\subsection{Conditions aux bords périodiques}

Les équations précédentes ont été établies pour un système défini sur la droite réelle. Cependant, dans une perspective thermodynamique, il est essentiel de considérer une densité finie $ N/L$. Cela peut être obtenu en compactifiant l’espace sur un cercle de longueur $L$, i.e. en imposant les {\em conditions aux bords périodiques}.

Concrètement, cela consiste à identifier $x = 0$ et $x = L$ et à exiger que la fonction d’onde soit périodique lorsqu’une particule fait le tour du système :
\begin{equation}\label{eq:periodic}
\varphi_{\{\theta_a\}}(x_1, \dots, x_{N-1}, L) = \varphi_{\{\theta_a\}}(0, x_1, \dots, x_{N-1}).
\end{equation}
Cette condition doit être satisfaite pour chaque particule. Or, déplacer la $j$-ième particule de $x_j$ à $x_j + L$ revient à la faire passer devant toutes les autres : cela introduit un facteur de diffusion à chaque croisement.

%\vspace{1em}

\subsection{Équations de Bethe exponentielles}

En imposant les conditions de périodicité sur la fonction d’onde de type Bethe~\eqref{eq:I-2-17}, on obtient que chaque moment $\theta_a$ doit satisfaire l’équation :
\begin{equation}
	e^{i \theta_a L} \prod_{b \ne a} S(\theta_a - \theta_b) = (-1)^{N-1}, \quad a = 1, \dots, N,
	\label{eq:bethe_exp}
\end{equation}
où la matrice diffusion $S(\theta) = \frac{\theta - i g}{-\theta - i g} = e^{i \Phi(\theta)}$ est l’amplitude de diffusion à deux corps, et $\Phi(\theta) = 2 \arctan\left( \frac{\theta}{c} \right)$ est la phase associée~\eqref{chap:1:eq:Phi}. Le signe $(-1)^{N-1}$ vient du fait que chaque permutation change la signature du déterminant dans la représentation de Gaudin.
%\vspace{1em}

\subsection{Équations de Bethe logarithmiques}

En prenant le logarithme du membre gauche et du membre droit de l’équation~\eqref{eq:bethe_exp}, on obtient :
\begin{equation}\label{chap:1:eq:EBA}
	L \theta_a + \sum_{b=1}^N \Phi(\theta_a - \theta_b) = 2\pi I_a, \qquad a = 1, \dots, N,
\end{equation}
où les $I_a \in \mathbb{Z}$ (ou $\mathbb{Z} + \tfrac{1}{2}$) sont des nombres quantiques entiers (ou demis entiers) . Dans la configuration d’état fondamental (ou de type “mer de Fermi”), ces nombres sont pris de manière symétrique autour de zéro :
\[
I_a = a - \frac{N+1}{2}, \quad \text{pour } a \in \llbracket 1 , N \rrbracket.
\]
Ce choix garantit une distribution uniforme des $\theta_a$ à l’état fondamental.
%\vspace{1em}

\subsection{Interprétation physique}

Les équations de Bethe~\eqref{chap:1:eq:EBA} représentent une {\em quantification des pseudo‑impulsions $\theta_a$} des particules en interaction, résultant d’un {\em interféromètre multi‑corps sur le cercle} : chaque particule accumule une phase $e^{i \theta_a L}$ due au mouvement libre, ainsi que des phases de diffusion lorsqu’elle croise les autres.

Ce système d'équations détermine les états propres du système de Lieb–Liniger en volume fini, et joue un rôle fondamental dans la description exacte de ses propriétés thermodynamiques et dynamiques.


\subsection{Thermodynamique du gaz de Lieb–Liniger à température nulle}

Dans la limite thermodynamique, le nombre de particules \( N \) et la longueur \( L \) du système tendent vers l'infini de telle sorte que leur rapport reste fini :
\begin{eqnarray*}
	\lim_{N,\, L \to \infty} \frac{N}{L} = D < \infty,
\end{eqnarray*}
où \( D \) désigne la densité linéique de particules.

Considérons désormais le système à température nulle. L’état fondamental dans le secteur à nombre de particules fixé correspond à la configuration d’énergie minimale parmi les solutions des équations de Bethe \eqref{chap:1:eq:EBA}.

Dans la limite thermodynamique, les valeurs de \( \theta_a \) deviennent quasi-continues, avec un espacement \( \theta_{a+1} - \theta_a = \mathcal{O}(1/L) \), et se condensent dans un intervalle symétrique autour de zéro :
\[
\theta_a \in [-K, K],
\]
où \( K \) est le paramètre de Fermi (ou rapidité maximale), défini par \( K = \theta_N \). En supposant l'ordre \( I_a \geq I_b \Rightarrow \theta_a \geq \theta_b \), cet intervalle constitue ce qu'on appelle la {\em mer de Dirac} (ou sphère de Fermi en dimension un).

Nous introduisons la densité d’états \( \rho_s(\theta) \), définie par
\begin{eqnarray*}
	2\pi \rho_s(\theta_a) &=& \frac{2\pi}{L} \lim_{\text{therm}} \frac{|I_{a+1} - I_a|}{|\theta_{a+1} - \theta_a|} = \frac{2\pi}{L} \frac{\partial I}{\partial \theta}(\theta_a),
\end{eqnarray*}
où \( I(\theta_a) = I_a \). L’application des équations de Bethe sous forme logarithmique conduit alors à
\begin{eqnarray*}
	2\pi \rho_s(\theta_a) = 1 + \frac{1}{L} \sum_{b = 1}^N \Delta(\theta_a - \theta_b),
\end{eqnarray*}
ce qui relie \( \rho_s \) à la fonction d’interaction \( \Delta \) entre les rapidités.

Intéressons-nous maintenant à la {\em densité de particules dans l’espace des moments}, notée \( \rho(\theta) \), définie par
\begin{eqnarray*}
	\rho(\theta_a) = \lim_{L \to \infty} \frac{1}{L} \cdot \frac{1}{\theta_{a+1} - \theta_a} > 0.
\end{eqnarray*}
Dans l’état fondamental, toutes les positions disponibles dans l’intervalle \( [-K, K] \) sont occupées. On a donc :
\begin{eqnarray}\label{chap.1.rho.2}
	\rho(\theta) = \rho_s(\theta).
\end{eqnarray}

La quantité \( L \rho(\theta) d\theta \) représente le nombre de rapidités dans la cellule infinitésimale \( [\theta, \theta + d\theta] \), tandis que
\(
	N = L \int_{-K}^{K} \rho(\theta)\, d\theta
\)
donne le nombre total de particules dans le système. Le passage de la somme discrète à l'intégrale dans le second membre de l'équation de Bethe permet d’écrire :
\begin{eqnarray*}
	\frac{1}{L} \sum_{b = 1}^N \Delta(\theta_a - \theta_b) \longrightarrow \int_{-K}^{K} \Delta(\theta_a - \theta)\, \rho(\theta)\, d\theta.
\end{eqnarray*}
Ainsi, l'équation pour la densité d'états devient :
\begin{eqnarray}\label{chap.1.rho.s.2}
	2\pi \rho_s(\theta) = 1 + \int_{-K}^{K} \Delta(\theta - \theta')\, \rho(\theta')\, d\theta',
\end{eqnarray}
et, comme \( \rho = \rho_s \), on obtient l’équation linéaire intégrale satisfaite par la densité de rapidités :
\begin{eqnarray}\label{chap.1.rho.3}
	\rho(\theta) - \int_{-K}^{K} \frac{\Delta(\theta - \theta')}{2\pi} \rho(\theta')\, d\theta' = \frac{1}{2\pi}.
\end{eqnarray}


\subsection{Excitations élémentaires à température nulle}




\chapter{Relaxation et Équilibre dans les Systèmes Quantiques Intégrables : Une Approche par la Thermodynamique de Bethe}\label{chap:relaxation}
\minitoc

%------------------------------------------------------------------
\section*{Introduction générale}

Dans les modèles quantiques intégrables, l’évolution vers l’équilibre, à partir d’un état initial arbitraire (et typiquement hors d’équilibre), ne conduit pas à une thermique de Gibbs classique.  
En effet, du fait de l’existence d’une infinité de charges conservées en involution, les systèmes intégrables n’explorent qu’une sous-partie contrainte de l’espace des états accessibles.  
Ils relaxent alors vers un état stationnaire décrit par une \emph{Ensemble Thermodynamique Généralisé} (GGE), qui encode la conservation de toutes ces quantités.

Cette section pose les fondations nécessaires à la description de ces états stationnaires dans le cadre de la \textbf{thermodynamique de Bethe} (TBA), qui généralise l’analyse au-delà de l’état fondamental.  
Nous considérons ici un régime macroscopique à température (ou entropie) finie, correspondant à des états hautement excités du spectre, mais toujours décrits dans le formalisme intégrable exact.

Notre point de départ est la relation constitutive entre la \emph{densité de quasi-particules} (ou \emph{rapidités}) $\rho(\theta)$ et la \emph{densité d’états} disponibles $\rho_s(\theta)$, qui encode le spectre accessible en présence d’interactions.  
Nous introduisons ensuite une opération clé de la TBA, appelée \emph{habillage} (\emph{dressing}), qui intervient systématiquement dans le calcul des observables physiques et permet de prendre en compte de manière non perturbative les effets des interactions.  
Cette construction sera illustrée dans le cadre du modèle intégrable de Lieb–Liniger, qui décrit un gaz unidimensionnel de bosons avec interaction delta répulsive.

Les outils développés ici seront fondamentaux pour formuler dans la section suivante le concept d’ensemble généralisé (GGE), et pour décrire la dynamique de relaxation des systèmes intégrables.



\section{Notion d’état d’équilibre généralisé (GGE)}

\paragraph{Introduction.}


\paragraph{Configuration des états.}\label{sec:config-etats}.
On désigne par $\boldsymbol{\{ \theta_a \}}\equiv \{ \theta_1 , \cdots , \theta_{N} \}$ la \emph{configuration de rapidités} caractérisant un état propre à $N\!\equiv\!N(\{ \theta_a \})$ particules – le nombre de particules n’est donc pas fixé \emph{a priori} mais dépend de la configuration.  
L’état propre correspondant est noté $\ket{\{ \theta_a \}}\;=\;\ket{\{\theta_1,\dots,\theta_N \}}$.

%%%%%%%%%%%%%%%%%%%%%%%%%%%%%%%%%%%%%%%%%%%%%%%%%%
\paragraph{Observables diagonales dans la base des états propres.}
Dans le chapitre précédent (\ref{chap:LL-BA}), on a vu que l'état $\ket{\{ \theta_a \}}$ associé à cette configuration est une état propre des observables nombre et quantité de mouvement  et  énergie cinétique \eqref{chap1:eq.Q.P.K.theta.1}. Ces observables sont diagonales dans la base des états propres :
\begin{eqnarray}
	\operator{Q}  =  \sum_{ \{\theta_a\} } \left ( \sum_{a = 1}^{N}  1 \right )  \vert \{ \theta_a\}\rangle	\langle \{ \theta_a \}\vert, \, 
	\operator{P}  =  \sum_{\{ \theta_a\}}\left( \sum_{a = 1}^{N}  \theta_a \right )   \vert \{ \theta_a\}\rangle	\langle \{ \theta_a \}\vert,\,\operator{K}  =  \sum_{\{ \theta_a\}}\left ( \sum_{a = 1}^{N} \frac{\theta_a^2}{2} \right )   \vert \{ \theta_a\}\rangle	\langle \{ \theta_a \}\vert.\label{chap.2.gge.1}		
\end{eqnarray}
avec $ \sum_{\{ \theta_a\}}$ une somme sur tous les configurations.\\
%\begin{eqnarray}
%	\operator{Q} \ket{\{ \theta_a\}}  =  \sum_{ \{\theta_a\} } \left ( \sum_{a = 1}^{N}  1 \right ) \ket{\{ \theta_a\}}, \, 
%	\operator{P} \ket{\{ \theta_a\}}  =  \sum_{\{ \theta_a\}}\left( \sum_{a = 1}^{N}  \theta_a \right ) \ket{\{ \theta_a\}},\,\operator{H} \ket{\{ \theta_a\}}  =  \sum_{\{ \theta_a\}}\left ( \sum_{a = 1}^{N} \frac{\theta_a^2}{2} \right )   \ket{\{ \theta_a\}}.		
%\end{eqnarray}

Nous avons introduit ces observables en injectant des opérateurs $\operator{f}$ proportionnels à des puissances de la quantité de mouvement d’une particule $\operator{p}$, respectivement $\propto \operator{p}^0$, $\propto \operator{p}^1$ et $\propto \operator{p}^2$, dans l’opérateur à un corps $\operator{F}$ défini dans l’équation \eqref{chap.1:eq.rapel.opp.1.second.2}. Écrit de cette manière, nous avons vu dans l’équation \eqref{chap.1:eq.rapel.opp.1.second.3} que pour $\operator{f} = \operator{p}^q$ avec $q$ entier, l’état de Bethe $\ket{\{ \theta_a \} }$ est un état propre de $\operator{F}$ :
\begin{eqnarray}\label{chap.2:eq.rapel.opp.1.second.1}
	 \operator{F} \ket{\{\theta_a\}} =   \sum_{ \{\theta_a\} }\left( \sum_{a = 1}^N \theta_a^q \right) \ket{\{\theta_a\}},
\end{eqnarray}
avec des valeurs propres données par des puissances de $\theta$. Cela motive l’étude d’états d’équilibre statistique au-delà de l’équilibre thermique, c’est-à-dire au-delà de l’ensemble de Gibbs.
   




%%%%%%%%%%%%%%%%%%%%%%%%%%%%%%%%%%%%%%%%%%%%
\paragraph{Contexte et GGE dans les systèmes intégrables.}

Dans un système quantique {\bf intégrable}, il existe une infinité de charges conservées locales $\operator{Q}_i$ commutant entre elles et avec l’Hamiltonien $\operator{H}$ ([Rigol et al. 2007] ) \cite{??}. Concrètement, chaque charge se présente sous la forme $\operator{Q}_i = \int dx \,\operator{q}_i(x)$, où $\operator{q}_i(x)$ est une densité d’observable locale à support borné. L’intégrabilité implique ainsi une caractérisation complète des états propres par un ensemble de paramètres (rapidités $\{\theta_j\}$ dans le modèle de Lieb-Liniger) \cite{??}. En particulier, contrairement aux systèmes génériques, un système intégrable ne thermalise pas au sens canonique classique, car la présence de toutes ces contraintes empêche l’oubli complet des conditions initiales. Les points clés sont alors :

\begin{itemize}[label = $\bullet$]
	\item {\bf Charges conservées} : infinité de locales $\operator{Q}_i$ satisfaisant et $[\operator{Q}_i , \operator{H} ] = 0$ et $[\operator{Q}_i , \operator{Q}_j ] = 0$.
	\item {\bf Densités locales} : chaque $\operator{Q}_i$ s’écrit $\operator{Q}_i = \int_\mathbb{R} dx \, \operator{q}_i(x)$ avec $\operator{q}_i(x)$ à support fini.
	\item {\bf Relaxation non canonique} : après un {\em quench} (changement brutal de paramètre), le système évolue vers un état stationnaire qui n’est pas décrit par l’ensemble canonique habituel.
\end{itemize}

Pour décrire cet état, on introduit l’{\bf ensemble de Gibbs généralisé (GGE)}. Rigol et al. ont montré qu’une « extension naturelle de l’ensemble de Gibbs aux systèmes intégrables » prédit correctement les valeurs moyennes des observables après relaxation \cite{??}.  Formellement, pour une région finie du système $\mathcal{S} \subset \mathbb{R}$, on définit la matrice densité locale :
\begin{eqnarray}
	\operator{\rho}^{(\mathcal{S})}_{\mathrm{GGE}} = \frac{1}{Z^{(\mathcal{S})}}\exp \left ( - \sum_i \beta_i \operator{Q}_i^{(\mathcal{S})} \right), \quad \operator{Q}_i^{(\mathcal{S})} = \int_\mathcal{S} dx \, \operator{q}_i(x), \label{chap.TBA.op.rho.S}	
\end{eqnarray}

où $\beta_i \in \mathbb{R}$ sont les multiplicateurs de Lagrange (ou « températures généralisées ») associés aux charges locales conservées $\{\operator{Q}_i\}$. La fonction de partition 
\begin{eqnarray}
	Z^{(\mathcal{S})} = \bm{\mathrm{Tr}}\left [\exp \left( - \sum_i \beta_i \operator{Q}_i^{(\mathcal{S})} \right ) \right ]  \label{chap.TBA.op.Z.S}	
\end{eqnarray}
 assure la normalisation. L’{\bf état GGE} ainsi défini est le seul permettant de prédire de manière cohérente les observables locales de $\mathcal{S}$ à long temps \cite{??}. Autrement dit, l’équilibre local après quench est un état stationnaire faisant perdurer la mémoire de chaque charge conservée, ce qui conduit à un nombre macroscopique de paramètres $\beta_i$ thermodynamiques (une « température » par charge) \cite{??}.

 \subparagraph{Interprétation des multiplicateurs de Lagrange.}
Les multiplicateurs de Lagranges $\beta_i$ apparaissent naturellement lors de l'optimisation sous contraintes, par exemple dans le formalisme de l'{\bf ensemble de Gibbs généralisé (GGE)}, oû il imposent la conservation des valeurs moyennes des charges $\langle \operator{Q}_i^{(\mathcal{S})} \rangle_{\operator{\rho}^{(\mathcal{S})}_{\mathrm{GGE}}} = \bm{\mathrm{Tr}}[\operator{\rho}^{(\mathcal{S})}_{\mathrm{GGE}} \operator{Q}_i^{(\mathcal{S})}]   $.\\

En résumé, la GGE généralise les ensembles canoniques standard : au lieu de retenir uniquement l’énergie, on impose la conservation de l’ensemble complet $\{\operator{Q}_i \}$. Cette construction rend compte du fait que, dans un système intégrable, les observables locaux convergent vers les valeurs moyennes de $\operator{\rho}^{(\mathcal{S})}_{\mathrm{GGE}}$ , et non vers celles d’un Gibbs thermique ordinaire \cite{??}\cite{??}. On comprend ainsi pourquoi la {\em thermalisation habituelle} (canonique ou microcanonique) échoue : seul l’ensemble de Gibbs généralisé peut intégrer toutes les contraintes locales.

\paragraph{Rappel sur le modèle de Lieb-Liniger et distribution de rapidités.}
Comme rappelé au chapitre précédent, {\bf le modèle de  Lieb-Liniger} (gaz bosonique 1D à interactions de contact) est un exemple paradigmatique d’un système intégrable \cite{??}. Ses états propres sont caractérisés par un ensemble de $N$  rapidités $\{ \theta_a \}$ , qui jouent le rôle de quasi-momenta ({\bf Bethe ansatz}). Dans ce contexte, l’état macroscopique du gaz après relaxation unitaire est entièrement déterminé par la {\bf distribution des rapidités}. Formellement, on définit $\rho(\theta)$ la distribution intensive des rapidités telle que $\rho(\theta) d \theta$ donne la fraction de particules par unité de longueur ayant une rapidité dans la cellule $[\theta , \theta + d \theta ] $.\\

Cette « distribution de rapidités » est d’autant plus pertinente qu’elle est {\em accessible expérimentalement}. En effet, lorsque le gaz bosonique 1D est libéré et laissé s’étendre, la distribution asymptotique des vitesses des atomes coïncide avec la distribution initiale des rapidités \cite{??} . Autrement dit, la GGE prédit un profil de vitesses observables en laboratoire. Léa Dubois souligne dans sa thèse que " la distribution de rapidités est la distribution asymptotique des vitesses des atomes après une expansion dans le guide 1D ", et qu’elle peut être extraite par l’hydrodynamique généralisée \cite{??}. \\

Dans la GGE, cette distribution macroscopique $\rho(\theta)$ est fixée par l’ensemble des charges conservées. Par exemple, on ajuste les $\beta_i$ de sorte que les valeurs moyennes $\langle \operator{Q}_i \rangle_{\operator{\rho}^{(\mathcal{S})}_{\mathrm{GGE}}}$ correspondent aux valeurs initiales. Ce processus détermine donc la fonction $\rho(\theta)$ décrivant l’état d’équilibre local. Les observables locaux du gaz (densité, corrélations, etc.) en découlent alors via les équations de Bethe ansatz. 


\paragraph{Convention pour les moyennes d'observables.}
Dans la suite du chapitre, nous noterons la moyenne d’une observable $\operator{\mathcal{O}}$ dans un état décrit par une matrice densité (ici noté) $\operator{\rho}$ par :
\begin{eqnarray}\label{chap.TBA.moy.dens}	
	\braket{\operator{\mathcal{O}}}_{\operator{\rho}} \doteq \bm{\mathrm{Tr}}[\operator{\rho} \, \operator{\mathcal{O}}],
\end{eqnarray}
En particulier, si la matrice densité est un projecteur, comme $\ket{\{\theta_a \}}\!\bra{\{\theta_a \}}$, $\bm{\mathrm{Tr}}[\ket{\{\theta_a \}}\!\bra{\{\theta_a \}} \operator{\mathcal{O}}] =  \bra{\{\theta_a \}}\operator{\mathcal{O}}\ket{\{\theta_a \}}$. dans ce cas on notera la moyenne :
\begin{eqnarray}\label{chap.TBA.moy.dens.pur}
	\braket{\operator{\mathcal{O}}}_{\{\theta_a \}} = \bra{\{\theta_a \}} \operator{\mathcal{O}} \ket{\{\theta_a \}},
\end{eqnarray}
où l’on note simplement l’ensemble des rapidité ${\theta_a}$ pour désigner l’état pur.

%%%%%%%%%%%%%%%%%%%%%%%%%%%%%%%%%%%%%%%%%%%%%%%%%%
\paragraph{Charges conservées locales diagonales dans la base des états propres.}
Les charges conservées locales $\operator{Q}_i^{(\mathcal{S})}$ est diagonale dans la base des  états propres $\ket{ \{ \theta_a \}}$ , avec pour valeurs propres $\langle \operator{Q}_i^{(\mathcal{S})} \rangle_{\{\theta_a \}} $ 	 :
%\begin{eqnarray}
%	\operator{Q}_i^{(\mathcal{S})} & = & \sum_{ \{\theta_a\} } \langle \operator{Q}_i^{(\mathcal{S})} \rangle_{\{\theta_a \}}  \ket{\{\theta_a \}}\!\bra{\{\theta_a \}}.		
%\end{eqnarray}
\begin{eqnarray}\label{chap.TBA.Qi.diag}
	\operator{Q}_i^{(\mathcal{S})}\ket{\{\theta_a \}} & = &  \langle \operator{Q}_i^{(\mathcal{S})} \rangle_{\{\theta_a \}}  \ket{\{\theta_a \}}.		
\end{eqnarray}
%%%%%%%%%%%%%%%%%%%%%%%%%%%%%%%%%%%%%%%%
\paragraph{Probabilité d’un état à rapidités fixées.}
On peut alors définir la probabilité d’occurrence d’un état $\ket{\{ \theta_a \} }$ comme la moyenne de la matrice densité locale $\operator{\rho}^{(\mathcal{S})}_{\mathrm{GGE}}$ définie dans \eqref{chap.TBA.op.rho.S}:
\begin{eqnarray}
	\mathbb{P}^{(\mathcal{S})}_{\{ \theta_a \}}  & \equiv &  \langle \operator{\rho}^{(\mathcal{S})}_{\mathrm{GGE}} \rangle_{\{\theta_a \}}, \label{chap.TBA.P.1}\\
	& = & 
	\frac{1}{Z^{(\mathcal{S})}} \exp \left (- \sum_i \beta_i \langle \operator{Q}_i^{(\mathcal{S})} \rangle_{\{\theta_a \}} \right ) \label{chap.TBA.P.2}.
\end{eqnarray}

%%%%%%%%%%%%%%%%%%%%%%%%%%%
\paragraph{Moyenne d’un charges conservées locales et dérivées de $Z^{(\mathcal{S})}$.} Les charges locales $\operator{Q}_i^{(\mathcal{S})}$ sont diagonale dans la bases \( \{ \ket{\{\theta_a \}} \}  \) [cf eq~ ~\eqref{chap.TBA.Qi.diag}]. 
On peut donc  écrire la moyenne d’une observable comme une somme pondérée par cette probabilité [cf eqs ~\eqref{chap.TBA.P.1}-\eqref{chap.TBA.P.2}] , ou encore comme une dérivée de la fonction de partition définie dans l'équation \eqref{chap.TBA.op.Z.S} :
\begin{eqnarray}
	\langle \operator{Q}_i^{(\mathcal{S})} \rangle_{\operator{\rho}^{(\mathcal{S})}_{\mathrm{GGE}}} &= & \sum_{\{ \theta_a\}} \langle \operator{Q}_i^{(\mathcal{S})} \rangle_{\{\theta_a \}} \mathbb{P}^{(\mathcal{S})}_{\{ \theta_a \}} \label{chap.TBA.moy.1}\\
	 & = &  \left. \frac{1}{Z^{(\mathcal{S})}} \frac{\partial Z^{(\mathcal{S})}}{\partial \beta_i} \right )_{\beta_{j \neq i }}	 \label{chap.TBA.moy.2}
\end{eqnarray}

Par le même raisonnement le moment non centré s'écrit :
\begin{eqnarray}
	\braket{ \operator{Q}_{i_1}^{(\mathcal{S})} \, \operator{Q}_{i_2}^{(\mathcal{S})} \cdots \operator{Q}_{i_q}^{(\mathcal{S})} }_{\operator{\rho}^{(\mathcal{S})}_{\mathrm{GGE}}} &= &  (-1)^q \frac{1}{Z^{(\mathcal{S})}} \left.\frac{\partial}{\partial \beta_{i_1}} \right )_{\beta_{j \neq i_1 }} \left.\frac{\partial}{\partial \beta_{i_2}} \right )_{\beta_{j \neq i_2 }} \cdots \left.\frac{\partial}{\partial \beta_{i_q}} \right )_{\beta_{j \neq i_q }} Z^{(\mathcal{S})} \label{chap.TBA.mom.1}.	
\end{eqnarray}

%%%%%%%%%%%%%%%%%%%%%%%%%%%%%%%
\paragraph{Moments d’ordre supérieur et fluctuations.} On s'avance sur le chapitre (\ref{chap:Fluctu}).
Le premier et second moments permettent d’accéder à la variance 
\begin{eqnarray}
	 \left \langle \left (\operator{Q}_i^{(\mathcal{S})} - \langle\operator{Q}_i^{(\mathcal{S})} \rangle_{\operator{\rho}^{(\mathcal{S})}_{\mathrm{GGE}}} \right )^2  \right \rangle_{\operator{\rho}^{(\mathcal{S})}_{\mathrm{GGE}}} = \langle(\operator{Q}_i^{(\mathcal{S})})^2 \rangle_{\operator{\rho}^{(\mathcal{S})}_{\mathrm{GGE}}}  -  \langle\operator{Q}_i^{(\mathcal{S})} \rangle_{\operator{\rho}^{(\mathcal{S})}_{\mathrm{GGE}}}^2	
\end{eqnarray}
de le charge locale $\operator{Q}_i^{(\mathcal{S})}$, en injectant \eqref{chap.TBA.moy.2} et \eqref{chap.TBA.mom.1} et en utilisant $\frac{1}{f} \partial_x^2 f - ( \frac{1}{f} \partial_x f ) = \partial_x^2 \ln f  $:
\begin{eqnarray}
	\left \langle \left (\operator{Q}_i^{(\mathcal{S})} - \langle\operator{Q}_i^{(\mathcal{S})} \rangle_{\operator{\rho}^{(\mathcal{S})}_{\mathrm{GGE}}} \right )^2  \right \rangle_{\operator{\rho}^{(\mathcal{S})}_{\mathrm{GGE}}}  &=&	  \left . \frac{\partial^2 \ln Z^{(\mathcal{S})}  }{{\partial \beta_i}^2 }  \right )_{\beta_{j\neq i}},\\
	& = &  - \left . 	\frac{\partial \langle\operator{Q}_i^{(\mathcal{S})} \rangle_{\operator{\rho}^{(\mathcal{S})}_{\mathrm{GGE}}} }{\partial \beta_i } \right )_{\beta_{j\neq i}}.	
\end{eqnarray}

%%%%%%%%%%%%%%%%%%%%%%%%%%%%%%
\paragraph{Cas particulier de l’équilibre thermique.}

Dans le cas particulier de l’équilibre thermique standard (\ie Gibbsien), le système est décrit par une seule contrainte d’énergie (ou d’énergie et de particule, dans le cas d’un grand canonique). Les multiplicateurs de Lagrange associés aux charges conservées peuvent alors être identifiés à des grandeurs thermodynamiques classiques.

\begin{itemize}[label=$\bullet$]
	\item Si la seule charge conservée est le nombre de particules $\operator{Q}_0^{(\mathcal{S})} = \operator{Q}$, le multiplicateur associé est $\beta_0 = -\beta \mu$, où $\mu$ est le potentiel chimique et $\beta = T^{-1}$ l’inverse de la température (avec $k_B = 1$).
	
	\item Si la charge conservée est $\operator{Q}_2^{(\mathcal{S})}-\mu\operator{Q}_0^{(\mathcal{S})}  = \operator{K} - \mu \operator{Q} $ (ensemble grand canonique), alors le multiplicateur est simplement $ \beta$.
\end{itemize}

Dans le cadre de l’équilibre thermique , les moyennes et les fluctuations thermodynamiques usuelles s’expriment naturellement comme dérivées du logarithme de la fonction de partition $Z^{(\mathcal{S})}$ :
\begin{eqnarray}
	\langle \operator{Q} \rangle_{\operator{\rho}^{(\mathcal{S})}_{\mathrm{GGE}}}  = \left .\frac{1}{\beta} \frac{ \partial \ln Z^{(\mathcal{S})}}{\partial \mu } \right )_{T},  & &  \left . \frac{1}{\beta} \frac{ \partial \langle \operator{Q} \rangle_{\operator{\rho}^{(\mathcal{S})}_{\mathrm{GGE}}}}{\partial \mu } \right )_{T} =  \left . \frac{1}{\beta^2} \frac{ \partial^2 \ln Z^{(\mathcal{S})}}{{\partial \mu}^2 } \right )_{T} \\
	\langle \operator{H} - \mu\operator{Q}  \rangle_{\operator{\rho}^{(\mathcal{S})}_{\mathrm{GGE}}}  = -\left . \frac{ \partial \ln Z^{(\mathcal{S})}}{\partial \beta } \right )_{\mu} ,  & & -\left .  \frac{ \partial \langle \operator{H} - \mu\operator{Q} \rangle_{\operator{\rho}^{(\mathcal{S})}_{\mathrm{GGE}}}}{\partial \beta } \right )_{\mu } = \left .  \frac{ \partial^2 \ln Z^{(\mathcal{S})}}{{\partial \beta}^2 } \right )_{\mu}   .		
\end{eqnarray}
En combinant ces relations, on peut également exprimer l’énergie moyenne et ses fluctuations comme :
\begin{eqnarray}
	\langle \operator{H} \rangle_{\operator{\rho}^{(\mathcal{S})}_{\mathrm{GGE}}}  = \left [ \left .\frac{\mu}{\beta} \frac{ \partial}{\partial \mu } \right )_{T} -\left . \frac{ \partial }{\partial \beta } \right )_{\mu}   \right ]\ln Z^{(\mathcal{S})},  \quad  -\left .  \frac{ \partial \langle \operator{H} \rangle_{\operator{\rho}^{(\mathcal{S})}_{\mathrm{GGE}}}}{\partial \beta } \right )_{-\mu \beta } = \left [ \left .\frac{\mu}{\beta} \frac{ \partial}{\partial \mu } \right )_{T} -\left . \frac{ \partial }{\partial \beta } \right )_{\mu}  \right ]^2\ln Z^{(\mathcal{S})}.		
\end{eqnarray}

%%%%%%%%%%%%%

\section{Remarques sur le formalisme}




%\input{preamble}

\begin{document}

\frontmatter
%\input{chapters/00_intro}
\tableofcontents
\mainmatter

\input{chapters/01_LL_BA}
\input{chapters/02_GGE_TBA}
\input{chapters/03_GHD}
%\input{chapters/97_GHD}
\input{chapters/04_GGE_Fluctuation}
\input{chapters/05_Disp_Exp}
\input{chapters/06_Bipart}
\input{chapters/07_Dipolaire}

%\input{chapters/08_conclusion}
%\appendix
%\input{chapters/99_annexes}

\bibliographystyle{abbrv}
\bibliography{thesis}

%\printbibliography

\end{document}

%| Style     | Description                                                             |
%| --------- | ----------------------------------------------------------------------- |
%| `plain`   | Tri alphabétique, numérotation croissante                               |
%| `unsrt`   | Même que `plain` mais sans tri, respecte l’ordre d’apparition           |
%| `abbrv`   | Comme `plain` mais avec prénoms et noms abrégés                         |
%| `alpha`   | Les références sont étiquetées par une combinaison du nom et de l’année |
%| `apalike` | Style APA simplifié                                                     |
%| `ieeetr`  | Style IEEE, tri par ordre d’apparition                                  |
%| `siam`    | Style SIAM (mathématiques appliquées)                                   |
%| `acm`     | Style ACM (informatique)                                                |
%



\section{Rôle des charges conservées extensives et quasi-locales}
%Dans les systèmes intégrables, l’état stationnaire atteint après une évolution hors d’équilibre n’est généralement pas décrit par un état de Gibbs classique, mais par un ensemble généralisé de Gibbs (GGE). Celui-ci est construit à partir de toutes les charges conservées du système

\paragraph{Écriture des observables thermodynamiques comme sommes sur les rapidités.}

%Dans le cas thermique, les valeurs moyennes des observables classiques telles que le nombre de particules et l'énergie peuvent s'exprimer comme des sommes de puissances des rapidités :
Dans un système à $N$ particules caractérisé par des rapidités $\{ \theta_a \}_{a = 1}^N$, les charges conservées classiques — telles que le nombre de particules, l’impulsion ou l’énergie — s’écrivent comme des sommes de puissances des rapidités :
\(
	\langle \operator{Q} \rangle_{\{ \theta_a\} } \propto \sum_{a = 1}^N \theta_a^0 , \,  \langle \operator{P} \rangle_{\{ \theta_a\} } \propto \sum_{a = 1}^N \theta_a^1  ,\,  \mbox{et} \langle \operator{K} \rangle_{\{ \theta_a\} } \propto \sum_{a = 1}^N \theta_a^2 .	
\)
(cf. équations \eqref{chap.2.gge.1})
Dans ce paragraphe précédent, nous avons sous-entendu — sans l’expliciter — qu’il est montré que l’ensemble des charges locales conservées forme une famille donnée par :
\begin{eqnarray}
	\operator{Q}_i^{(\mathcal{S})} \ket{\{\theta_a\} } & \propto & \sum_a \theta_a^i \ket{\{\theta_a\} }.
\end{eqnarray}
Ces charges agissent donc de manière diagonale sur les états de Bethe, avec des valeurs propres correspondant aux moments des rapidités.
%%%%%%%%%%%%%%%%%%%%%%%%%%%%%%%%%%%%%%%%%%%%%%%%%%
\paragraph{Charges locales conservées .\label{sec:charges-gen}}

%Les états propres du Hamiltonien de Lieb–Liniger~\eqref{eq:LL} sont les états de Bethe
%\(
%  \ket{\boldsymbol{\theta}}
%  =\ket{\theta_1,\dots,\theta_N}\!,
%\)
%déterminés par leurs rapidités \(\boldsymbol{\theta}\).

À toute fonction régulière
\(
  f:\mathbb R\!\to\!\mathbb R
\)
on associe un opérateur-charge loclal :
\begin{eqnarray}\label{chap.2.charge.f.1}
	\operator{\mathcal{Q}}^{(\mathcal{S})}[f] & = &  L \int_0^L d\theta \, f(\theta) \operator{\rho}^{(\mathcal{S})}(\theta).	
\end{eqnarray}
où $\operator{\rho}(\theta)$ agit sur une état de Bethe comme 
\begin{eqnarray}\label{chap.2.rho.1}
	 \operator{\rho}(\theta) \ket{ \{ \theta_a \} } &=& \frac{1}{L} \sum_{a = 1 }^N  \delta ( \theta - \theta_a ) \ket{ \{ \theta_a \} }.	
\end{eqnarray}
De sorte que $\operator{\mathcal{Q}}^{(\mathcal{S})}[f]$ agit sur une état de Bethe comme
\begin{eqnarray}\label{chap.2.charge.1}
	\operator{\mathcal{Q}}^{(\mathcal{S})}[f]\,\ket{\{\theta_a\} } =  \sum_{a=1}^{N}f(\theta_a)\,\ket{\{\theta_a\} } \quad \mbox{de sorte que} \quad \braket{\operator{\mathcal{Q}}^{(\mathcal{S})}[f]}_{\{\theta_a\}} = \sum_{a=1}^N f(\theta_a)
\end{eqnarray}
Les choix particuliers
\(
  f_0(\theta)=1
\)
,
\(
  f_1(\theta)=\theta
\)
et
\(
  f_2(\theta)=\theta^{2}/2
\)
redonnent respectivement l'opérateur nombre \(\operator{Q}=\operator{Q}_0^{(\mathcal{S})} = \operator{\mathcal{Q}}^{(\mathcal{S})}[1]\) , impulsion \(\operator{P}=\operator{Q}_1^{(\mathcal{S})} = \operator{\mathcal{Q}}^{(\mathcal{S})}[\theta]\) et énergie cinétique
\(\operator{K}=\operator{Q}_2^{(\mathcal{S})} = \operator{\mathcal{Q}}^{(\mathcal{S})}[\theta^2/2]\). Et dans le cadre des (GGE), pour tous les ordres $i$ on note :
\begin{eqnarray}\label{chap.2.charge.ordre.i.1}
	\operator{Q}^{(\mathcal{S})}_i = \operator{\mathcal{Q}}^{(\mathcal{S})}[f_i]	, \quad \mbox{de sorte que} \quad \braket{\operator{Q}^{(\mathcal{S})}_i}_{\{\theta_a\}} = \sum_{a=1}^N f_i(\theta_a)  
\end{eqnarray}
avec les densités spectrales $f_i(\theta) \propto \theta^i$ . 

Ces charges sont extensives : leur densité locale $\operator{q}^{(\mathcal{S})}_{[f]}$ permet d’écrire
\(
  \operator{\mathcal{Q}}^{(\mathcal{S})}[f]=\int_0^{L}\!dx\;\operator{q}^{(\mathcal{S})}_{[f]}(x).
\)

\paragraph{Charges conservées généralisée.\label{sec:charges-gen}}
Les fonction $f_i$ étant fixées, on note la fonction régulière
\(
  w:\mathbb R\!\to\!\mathbb R
\)
–– dorénavant appelée \emph{poids spectral}, ou \emph{potentiel spectral} ––
\begin{eqnarray}
	w = \sum_i \beta_i f_i \label{chap.2.w.1},	
\end{eqnarray}
on associe un opérateur-charge généralisé $\operator{\mathcal{Q}}^{(\mathcal{S})}[w]$ :
\begin{eqnarray}\label{chap.2.charge.gen.1}
	\operator{\mathcal{Q}}^{(\mathcal{S})}[w]\,\ket{\{\theta_a\} } =  \sum_{a=1}^{N}w(\theta_a)\,\ket{\{\theta_a\} } \quad \mbox{de sorte que} \quad \braket{\operator{\mathcal{Q}}^{(\mathcal{S})}[w]}_{\{\theta_a\}} = \sum_{i} \beta_i  \braket{\operator{Q}^{(\mathcal{S})}_i}_{\{\theta_a\}}
\end{eqnarray}

%%%%%%%%%%%%%%%%%%%%%%%%%%%%%%%%%%%%%%%%%
\paragraph{Expression de la matrice densité généralisée.}
La matrice densité  s’écrit sous la forme :
L’ensemble général défini par $\operator{\varrho}^{(\mathcal{S})}[w]$ 
\begin{eqnarray}\label{chap.2.densite.1}
	\operator{\varrho}^{(\mathcal{S})}[w]  =  \frac{e^{-\operator{\mathcal{Q}}^{(\mathcal{S})}[w]}}{Z^{(\mathcal{S})}[w]}, \, \mbox{avec} \quad e^{-\operator{\mathcal{Q}}^{(\mathcal{S})}[w]}  = 	\sum_{\{\theta_a \}} e^{- \sum_{a = 1}^N w(\theta_a) } \vert \{ \theta_a\} \rangle \langle  \{ \theta_a\}  \vert, 
\end{eqnarray}	
	%pour une certaine fonction $w$ relié à la charge% $\operator{\mathcal{Q}} [w]  = \sum_{\{\theta_a \}} \left ( \sum_{a = 1}^N w ( \theta_a )  \right ) \vert \{ \theta_a \} \rangle \langle \{ \theta_a \} \vert $.
%où l'opérateur de charge associé à $w$ s’écrit :
%\begin{eqnarray}
%	\operator{\mathcal{Q}} [w]   & = &  \sum_{\{\theta_a \}} \left ( \sum_{a = 1}^N w ( \theta_a )  \right ) \vert \{ \theta_a \} \rangle \langle \{ \theta_a \} \vert,	
%\end{eqnarray}
et la fonction de partition \eqref{chap.TBA.op.Z.S} s'écrit $Z^{(\mathcal{S})}[w]\doteq \bm{\mathrm{Tr}}\left [ e^{-\operator{\mathcal{Q}}^{(\mathcal{S})}[w]}\right ] $ vaux :
\begin{eqnarray}
	Z^{(\mathcal{S})}[w]   =  \sum_{\{\theta_a \}} e^{-\sum_{a = 1}^N w(\theta_a)},\label{chap.TBA.op.Z.S.1}	
\end{eqnarray}
devient un Generalized Gibbs Ensemble (GGE), $\operator{\rho}^{(\mathcal{S})}_{\mathrm{GGE}}$ (de l'équation \eqref{chap.TBA.op.rho.S})	 dès lors que $w(\theta) = \sum_i \beta_i f_i(\theta)$ (de l'équation \eqref{chap.2.w.1}) où $f_i$ sont les densités spectrales associées aux charges locales conservées (de l'équation \eqref{chap.2.charge.ordre.i.1}).


%%%%%%%%%%%%%%%%%%%%%%%%%%%%%%%%%%
\paragraph{Probabilité associée à une configuration de rapidités.}
	%Et on peut réecrire la probabilité de la configuration $\{\theta_a\}$ :% $ P_{\{ \theta_a \}} = \langle \{ \theta_a \}\vert \operator{\rho}_{GGE}[w] \vert  \{ \theta_a \} \rangle = e^{-\sum_{a = 1}^N w(\theta_a)} / Z $ avec $Z = \sum_{\{\theta_a \}} e^{-\sum_{a = 1}^N w(\theta_a)}$.\\
	%La probabilité d’occuper un état à $N$ particules caractérisé par les rapidités ${\theta_a}$ est alors :
Dans ce formalisme, la probabilité d’occuper l’état $\ket{\{\theta \}}$ \eqref{chap.TBA.P.1} est donc
\begin{eqnarray}
	\mathbb{P}^{(\mathcal{S})}_{\{ \theta_a \}} & = &  Z^{(\mathcal{S})}[w]^{-1}e^{-\sum_{a = 1}^N w(\theta_a)}\label{chap.TBA.P.w.2}. 		
\end{eqnarray}
%Cela montre que le poids statistique d’une configuration factorise naturellement sur les pseudo-moments, avec un poids spectrale / energie génralisé $w(\theta)$ attribué à chaque particule.
On voit ainsi que le poids statistique factorise naturellement sur les
pseudo‑moments, chaque particule étant pondérée par $w(\theta_a)$.

%avec 
%\begin{eqnarray}
%	Z  & = & \sum_{\{\theta_a \}} e^{-\sum_{a = 1}^N w(\theta_a)}.		
%\end{eqnarray}


%%%%%%%%%%%%%%%%%%%%%%%%
\paragraph{Moyennes d'observables dans le GGE.}
%La valeur moyenne d’un observable locale $\operator{\mathcal{O}}$ dans l’ensemble généralisé s’écrit :
Pour tout opérateur local $\operator{\mathcal{O}}$ diagonal dans la base de Bethe,
la moyenne généralisée vaut
\begin{eqnarray}\label{chap.2.moyenne.1}
	\langle \operator{\mathcal{O}}\rangle_{\operator{\varrho}^{(\mathcal{S})}[w]} & = & \displaystyle   \frac{\sum_{\{\theta_a \}} \braket{ \operator{\mathcal{O}}}_{\{ \theta_a\}} e^{- \sum_{a = 1}^N w(\theta_a) }  }{\sum_{\{\theta_a  \}} e^{- \sum_{a = 1}^N  w(\theta_a) } }
\end{eqnarray}
%Cette expression formelle montre que la connaissance de $w(\theta)$ suffit à déterminer les propriétés statistiques de toutes les observables diagonales dans cette base, incluant les charges conservées elles-mêmes.
Ainsi, la connaissance de la fonction $w(\theta)$ suffit à déterminer
les propriétés statistiques de toute observable diagonale,
y compris les charges conservées elles‑mêmes.	
	% Nous aimerions calculer les valeurs d'attente par rapport à cette matrice de densité, par exemple
	%La moyenne GGE d'un observable s'écrit ,
	%\begin{aff}
	%\begin{eqnarray}
	%	\langle \operator{\mathcal{O}} \rangle_{GGE} & \doteq & \displaystyle  \text{Tr} (\operator{\mathcal{O}}\operator{\rho}[w]) = \frac{\text{Tr} (\operator{\mathcal{O}}e^{-\operator{\mathcal{Q}}[w]})}{\text{Tr} (e^{-\operator{\mathcal{Q}}[w]})}	 = \frac{\sum_{\{\theta_a \}} \langle  \{ \theta_a\}  \vert   \operator{\mathcal{O}} \vert \{ \theta_a\} \rangle e^{- \sum_{a = 1}^N w(\theta_a) }  }{\sum_{\{\theta_a  \}} e^{- \sum_{a = 1}^N  f(\theta_a) } }
		%& =  & \frac{ \sum_{\pi} \sum_{\vert \{\theta_a \}\rangle \vert \Pi } \langle  \{ \theta_a\}  \vert   \operator{\mathcal{O}} \vert \{ \theta_a\} \rangle e^{- \sum_{a = 1}^N f(\theta_a) }  }{\sum_{\pi} \sum_{\vert \{\theta_a \}\rangle \vert \Pi }  e^{- \sum_{a = 1}^N  f(\theta_a) } }
	%\end{eqnarray}
	%pour une certaine observable $\operator{\mathcal{O}}$.\\
	%\end{aff}
	

\paragraph{Conclusion de la section : vers la thermodynamique de Bethe.}

Nous avons vu que, dans un système intégrable, la description correcte de l’équilibre stationnaire requiert l’introduction d’une \emph{famille infinie de charges conservées}, comprenant à la fois des charges strictement locales et des charges quasi‑locales.
Toutes ces charges se réunissent dans l’opérateur fonctionnel
\(
\operator{\mathcal{Q}}^{(\mathcal{S})}[w]
\)
, défini par un \emph{poids spectral}  $w(\theta)$ (cf. équations~\eqref{chap.2.charge.1}).
Cette construction conduit naturellement à la matrice densité généralisée
\(
\operator{\rho}^{(\mathcal{S})}_{\mathrm{GGE}}  \propto  e^{-\operator{\mathcal{Q}}^{(\mathcal{S})}[w]}
\) 
(cf. équations~\eqref{chap.2.densite.1}), et à la moyenne d’un opérateur local $\operator{\mathcal{O}}$ donnée par
\(
\langle \operator{\mathcal{O}}\rangle_{\operator{\rho}^{(\mathcal{S})}_{\mathrm{GGE}}}  =  \displaystyle  \text{Tr} (\operator{\mathcal{O}}\operator{\varrho}^{(\mathcal{S})}[w])
\)
(cf. équations~\eqref{chap.2.moyenne.1}).
La connaissance de $w(\theta)$ suffit donc pour prédire les valeurs moyennes de toutes les observables diagonales, y compris celles des charges elles‑mêmes ; c’est le cœur du {\bf Ensemble de Gibbs Généralisé (GGE pour Generalized Gibbs Ensemble)} .

\medskip
Cette base est désormais posée : dans la section suivante, nous passerons au \emph{thermodynamique de Bethe}.
Nous verrons comment, dans la limite thermodynamique, les sommes sur les configurations de rapidités se transforment en intégrales sur des densités continues, comment apparaît l’entropie de Yang–Yang, et comment les moyennes de l’ensemble généralisé se réexpriment à l’aide de ces densités macroscopiques.
C’est ce formalisme qui permettra d’analyser finement la relaxation post‑quench et de relier microscopie intégrable et hydrodynamique généralisée.



%\input{preamble}

\begin{document}

\frontmatter
%\input{chapters/00_intro}
\tableofcontents
\mainmatter

\input{chapters/01_LL_BA}
\input{chapters/02_GGE_TBA}
\input{chapters/03_GHD}
%\input{chapters/97_GHD}
\input{chapters/04_GGE_Fluctuation}
\input{chapters/05_Disp_Exp}
\input{chapters/06_Bipart}
\input{chapters/07_Dipolaire}

%\input{chapters/08_conclusion}
%\appendix
%\input{chapters/99_annexes}

\bibliographystyle{abbrv}
\bibliography{thesis}

%\printbibliography

\end{document}

%| Style     | Description                                                             |
%| --------- | ----------------------------------------------------------------------- |
%| `plain`   | Tri alphabétique, numérotation croissante                               |
%| `unsrt`   | Même que `plain` mais sans tri, respecte l’ordre d’apparition           |
%| `abbrv`   | Comme `plain` mais avec prénoms et noms abrégés                         |
%| `alpha`   | Les références sont étiquetées par une combinaison du nom et de l’année |
%| `apalike` | Style APA simplifié                                                     |
%| `ieeetr`  | Style IEEE, tri par ordre d’apparition                                  |
%| `siam`    | Style SIAM (mathématiques appliquées)                                   |
%| `acm`     | Style ACM (informatique)                                                |
%




\section{Thermodynamique de Bethe et relaxation}

%------------------------------------------------------------------
\subsection{Limite thermodynamique}

\paragraph{Observables locales dans la limite thermodynamique.}
%Lorsque l'observable $\operator{\mathcal{O}}$ est suffisamment local, on croit que la valeur d'attente $\langle  \{ \theta_a\}  \vert   \mathcal{O} \vert \{ \theta_a\} \rangle$ ne dépend pas de l'état microscopique spécifique du système, de sorte qu'elle devient une fonctionnelle de $\Pi$ dans la limite thermodynamique.
Dans la suite de ce chapitre, nous omettrons l’exposant $(\mathcal{S})$.
\vspace{0.2em}
Dans la base des états de Bethe \( \{ \ket{\{ \theta_a \}} \} \), l’opérateur \( \hat{\rho}(\theta) \) défini en \eqref{chap.2.rho.1} est diagonal, et agit comme un projecteur sur les valeurs de rapidité.

\vspace{0.5em}

Dans la limite thermodynamique, différentes configurations microscopiques \( \{ \theta_a \} \) peuvent correspondre à la même distribution de rapidité macroscopique \( \rho(\theta) \). Autrement dit, plusieurs états \( \ket{\{ \theta_a \}} \) partagent la même valeur propre \( \rho(\theta) \) de l’opérateur \( \operator{\rho}(\theta) \). Cela reflète une {\em dégénérescence macroscopique} induite par le passage à la limite thermodynamique (\( N, L \to \infty \) avec \( N/L \to \text{const} \)).

\vspace{0.5em}

Si l’observable $\mathcal{O}$ est suffisamment locale, sa valeur d’attente dans un état propre ne dépend pas des détails microscopiques, mais uniquement de la distribution de rapidité. On écrit alors :
\begin{eqnarray}
	\underset{\mbox{\tiny therm.}}{\lim} \braket{  \operator{\mathcal{O}} }_{\{ \theta_a\}}  & = & \langle \operator{\mathcal{O}}\rangle_{[\rho]},
\end{eqnarray}
où $\underset{\mbox{\tiny therm.}}{\lim}$ est la limite thermodynamique ($N,L \to \infty$ avec $N/L \to $ const) et où \( \langle \mathcal{O} \rangle_{[\rho]} \) désigne la valeur d’attente de \( \mathcal{O} \) dans un état macroscopique caractérisé par la distribution de rapidité \( \rho(\theta) \).


\medskip
Dans un ensemble général (GGE), la valeur moyenne de l’observable \eqref{chap.2.moyenne.1} devient alors :		
\begin{eqnarray}\label{chap.2.moyenne.2}
	\underset{\mbox{\tiny therm.}}{\lim} \langle \operator{\mathcal{O}} \rangle_{\operator{\varrho}[w]} & =  & \frac{  \displaystyle \sum_{\rho }  \langle \operator{\mathcal{O}}\rangle_{[\rho]} \Omega[\rho] e^{- \sum_{a = 1}^N  w(\theta_a)    }}{ \displaystyle \sum_{\rho}   \Omega[\rho]\,e^{- \sum_{a = 1}^N  w(\theta_a) } } ,
\end{eqnarray}
où $\sum_{\rho }$ est une somme sus tous les distribution de rapidité $\rho$ et 
où $\Omega[\rho]$ désigne le nombre de micro-états compatibles avec la distribution de rapidité $\rho$.

%où $\# \mbox{micro-états.}$ est les nombre de micro état associée àa la distribution de rapidité $\rho$.
%Avant de se plonger sur $\# \mbox{micro-états.}$, regardons le changement des équation de Bethes. 

\medskip
Pour établir la fonction $\Omega[\rho]$, reppelons-nons de la transformation des équations de Bethe dans dans la limite thermodynamique, hors état fondamentale \eqref{eq:TBA-nu} et \eqref{eq:TBA-rhos-2}.
\begin{equation}
	\nu = \frac{\rho}{\rho_s} \, , \qquad 2\pi \rho_s = 1^{\mathrm{dr}}_{[\nu]} 
\label{chap.2:eq:TBA-rhos}
\end{equation}
où $f^{\mathrm{dr}}_{[\nu]}$ est définie en \eqref{eq:dessing}.

\medskip

Cette formalisation constitue la brique de base de la \textbf{hydrodynamique généralisée} et, dans la section suivante, permet de définir rigoureusement l’\textbf{entropie de Yang–Yang}, indispensable pour décrire la relaxation hors d’équilibre des systèmes intégrables.

%\vspace{1ex}
%La formalisation ci‑dessus fournit la brique de base pour la
%\textbf{hydrodynamique généralisée} et, dans la section suivante, pour la
%définition précise de l’\textbf{entropie de Yang-Yang}
%assurant la relaxation des systèmes intégrables hors‑équilibre.

%\input{preamble}

\begin{document}

\frontmatter
%\input{chapters/00_intro}
\tableofcontents
\mainmatter

\input{chapters/01_LL_BA}
\input{chapters/02_GGE_TBA}
\input{chapters/03_GHD}
%\input{chapters/97_GHD}
\input{chapters/04_GGE_Fluctuation}
\input{chapters/05_Disp_Exp}
\input{chapters/06_Bipart}
\input{chapters/07_Dipolaire}

%\input{chapters/08_conclusion}
%\appendix
%\input{chapters/99_annexes}

\bibliographystyle{abbrv}
\bibliography{thesis}

%\printbibliography

\end{document}

%| Style     | Description                                                             |
%| --------- | ----------------------------------------------------------------------- |
%| `plain`   | Tri alphabétique, numérotation croissante                               |
%| `unsrt`   | Même que `plain` mais sans tri, respecte l’ordre d’apparition           |
%| `abbrv`   | Comme `plain` mais avec prénoms et noms abrégés                         |
%| `alpha`   | Les références sont étiquetées par une combinaison du nom et de l’année |
%| `apalike` | Style APA simplifié                                                     |
%| `ieeetr`  | Style IEEE, tri par ordre d’apparition                                  |
%| `siam`    | Style SIAM (mathématiques appliquées)                                   |
%| `acm`     | Style ACM (informatique)                                                |
%






\subsection{Statistique des macro-états : entropie de Yang-Yang}

%\paragraph{Macro-états et entropie dans la TBA.}

%Dans la limite thermodynamique, dans le modèle statistique (GGE) , les moyenne, observables physiques deviennent des fonctionnelles de la {\bf distribution de rapidité}  $\rho(\theta)$ et du {\bf poing spectrale} $w(\theta)$ . Cette description est efficace car elle permet d’échapper au détail de chaque état propre. 
%Toutefois, cette simplification laisse en suspens une question cruciale : 
%Mais dans ce modelle qui simplifie on veux {\bf la distribution de rapidité d’un système à l'équilibre thermique à température finie} que l'on notera $\langle \rho \rangle$ pour dire la dansité moyenne. Et les lien entre  $w$ et $\langle \rho \rangle$.  Le problème est étudier par par Yang et Yang en 1969. Pour saisir l'enssentielle, nous devons comprendre la {\bf structure statistique des états propres} associés à une même distribution $\rho(\theta)$. Nous nous interrensons comme promis plus haut : à $\Omega(\theta)$ dans l'équation de moyenne \eqref{chap.2.moyenne.2}  ,  {\bf  nombre états propres microscopiques correspondent à une même distribution $\rho(\theta)$}.
%{\bf quelle est la distribution de rapidité d’un système à l'équilibre thermique à température finie ?}. 
%La question a été répondue dans les travaux pionniers de Yang et Yang (1969), que nous allons maintenant examiner brièvement. Pour répondre à cette question, nous devons comprendre la {\bf structure statistique des états propres} associés à une même distribution $\rho(\theta)$.

\paragraph{Motivation.}

Dans la limite thermodynamique, une observable locale dans un \textit{Generalized Gibbs Ensemble} (GGE) dépend uniquement de deux objets continus :  (i)  la \textbf{distribution de rapidité} $\rho(\theta)$, (ii) le \textbf{poids spectral} $w(\theta)$, c.-à-d.\ la " température généralisée " assignée à chaque quasi‑particule.
Cette reformulation est puissante car elle fait disparaître les détails d’un état propre individuel.  

\medskip
Cependant, pour décrire un \emph{vrai} équilibre à température finie, il faut la distribution à l'équilibre :
\begin{eqnarray}\label{chap.2:eq.rho.eq.1}
	\rho_{\mathrm{eq}}(\theta)\;\doteq\;\braket{\operator{\rho}(\theta)}_{\operator{\varrho}[w]}	,  
\end{eqnarray}
donc le lien entre $\rho_{\mathrm{eq}}$ et $w$.
La réponse fut donnée dans les travaux pionniers de \textsc{Yang \& Yang} (1969).  
Leur approche repose sur l’analyse de la \textbf{structure statistique des états propres} partageant la même distribution $\rho(\theta)$.

% : combien d’états microscopiquement distincts correspondent à ce même « macro‑état » ?

\paragraph{Distribution de rapidité comme macro-état.}

Chaque distribution de rapidité $\rho(\theta)$ ne correspond pas à un état propre unique, mais à un grand {\bf ensemble de micro-états} : différents choix des ensembles de quasi-moments $(\{\theta_a\}_{a \in \llbracket 1 , N \rrbracket })_{N \in \mathbb{Z}} $ peuvent conduire à la même densité de distribution à l’échelle macroscopique. Ainsi, $\rho(\theta)$ doit être interprétée comme un {\bf macro-état}, qui agrège un très grand nombre d’états propres microscopiques.

La question thermodynamique devient alors : {\bf Combien de micro-états microscopiquement distincts sont compatibles avec un même macro-état $\rho(\theta)$ ?} 

\medskip
Plus précisément, dans l’expression de moyenne des operateurs locaux \eqref{chap.2.moyenne.2}, apparaît le facteur
\(
\Omega[\rho]
\),
qui compte ces états propres.  
La détermination de $\Omega[\rho]$ (ou équivalemment de l’entropie de Yang–Yang $\mathcal{S}_{YY}[\rho]$ car 
\(
\Omega[\rho] = e^{L\mathcal{S}_{YY}[\rho]}
\)
avec $L$ la taille du système
) est donc la clé pour relier \emph{(i)} le poids spectral $w(\theta)$ imposé dans le GGE et \emph{(ii)} la distribution de rapidité moyenne $\rho_{\mathrm{eq}}(\theta)$ observée à l’équilibre.

\paragraph{Dénombrement local des configurations microcanoniques.}
Pour répondre à cette question, on subdivise l’axe des rapidités en petites tranches ou cellules de largeur $\delta \theta$, chacune centrée en un point $\theta_a$. Dans une tranche $[\theta_a, \theta_a + \delta\theta]$, on suppose que la densité $\rho(\theta)$ est à peu près constante. Le nombre de quasi-particules dans cette tranche est alors approximativement :
\begin{eqnarray*}
	N_a = L\rho(\theta_a) \delta \theta,
\end{eqnarray*}
et le nombre total d'états disponibles (\ie, le nombre d’états possibles si toutes les positions en moment étaient disponibles) est donné par la densité totale de niveaux 
\begin{eqnarray*}
	M_a = L\rho_s(\theta_a) \delta \theta.
\end{eqnarray*}
%La densité de niveaux $\rho_s(\theta)$ tient compte du fait que les moments sont quantifiés de manière discrète, en raison des équations de Bethe (voir équation (??)).

Les particules occupent ces niveaux de manière analogue à des fermions libres (principe d’exclusion de Pauli), le nombre de manières différentes de choisir $N_a$ niveaux parmi $M_a$ est donné par :
	
	
	\begin{figure}[H]
		\centering 
		\begin{tikzpicture}
			%\input{figures/04_GGE_Fluctuation/Occupation_code}	
			\begin{scope}[transform canvas={scale=0.6}]
			\input{figures/04_GGE_Fluctuation/Occupation_theta_code}	
			\end{scope}
			
			\draw[color = red , scale = 0.5 , draw = none ] (-13.5 , -1) rectangle (13 , 10) ; 
				
			
		\end{tikzpicture}	
		\captionsetup{skip=10pt} % Ajoute de l’espace après la légende
	\end{figure}
	
	
\begin{eqnarray}
	\Omega(\theta_a) & \approx  & \binom{M_a}{N_a} ~= ~   \frac{[ L\rho_s ( \theta ) \delta \theta ] ! }{ [ L\rho ( \theta ) \delta \theta ] ! [( L\rho_s ( \theta ) - L\rho ( \theta ) )  \delta \theta ] ! }. 	
\end{eqnarray}

\paragraph{Estimation asymptotique à l’aide de Stirling.}

En utilisant la formule de Stirling :
\begin{eqnarray}
	n! & \underset{n \to \infty}{\sim} &  n^n e^{-n} \sqrt{2\pi n}.,
\end{eqnarray}	
composé du fonction logarithmique, il vient cette équivalence : 
\begin{eqnarray}
	\ln n! & \underset{n \to \infty}{\rightarrow} & n \ln n \underbrace{- n + \ln \sqrt{2 \pi n }}_{o \left ( n \ln n \right ) } ,\\
	&  \underset{n \to \infty}{\sim} & n \ln n  
\end{eqnarray}
	
$\# \mbox{conf.}$ est jamais null donc on peut approximer, pour de grandes valeurs de $L$ et de $\delta\theta$  : 
\begin{eqnarray}
    \ln \Omega(\theta) & \underset{\underset{\rho (\theta )\leq  \rho_s (\theta )}{\rho \delta \theta  \to \infty}}{\sim}   & L [ \rho_s\ln \rho_s - \rho \ln \rho - (\rho_s - \rho ) \ln ( \rho_s - \rho) ] (\theta )\delta \theta .
\end{eqnarray}

Cette expression donne la contribution par unité de $\theta$ à l’{\bf entropie}  associée à la cellule autour de $\theta_a$.

\paragraph{Entropie de Yang-Yang : définition .}
%L'entropie totale du macro-état $\rho(\theta)$, notée $\mathcal{S}_{YY}[\rho]$, est obtenue en sommant sur toutes les tranches. Pour alléger la notation, nous écrivons cette somme comme :
%Le nombre total de micro-états est le produit de toutes ces configurations pour toutes les cellules de rapidité $[\theta, \theta + \delta \theta]$. %En prenant le logarithme et en remplaçant la somme par une intégrale sur $ \theta$, nous obtenons l'entropie de Yang-Yang :

%L'entropie totale du macro-état $\rho(\theta)$, notée $\mathcal{S}_{YY}[\rho]$, est obtenue en sommant sur toutes les tranches. Pour alléger la notation, nous écrivons cette somme comme :
%Le nombre total de micro-états compatibles avec une distribution macroscopique $\rho(\theta)$ est donné par le produit des nombres de configurations pour chaque cellule de rapidité $[\theta, \theta + \delta \theta]$.

%En prenant la sum le logarithme des $\Omega(\theta)$ , on obtient l'entropie totale de Yang-Yang. Pour alléger la notation, cette somme sur les tranches est notée :

Le nombre total de micro-états compatibles avec une distribution macroscopique donnée $\rho(\theta)$ est obtenu en prenant le produit des nombres de configurations pour chaque cellule de rapidité $[\theta_a, \theta_a + \delta \theta]$ : $ \Omega(\theta_a)$ .
En prenant le logarithme de ce produit, on accède à l'entropie totale. Pour alléger la notation, cette somme sur les cellules est notée
\(
	\sum_a^{\theta-\mbox{\tiny cellules}}	
\)
où chaque $a$ indexe une cellule de rapidité $[\theta_a, \theta_a + \delta\theta]$.
On écrit alors :
\begin{eqnarray}
    \ln \Omega[\rho] & = & \sum_a^{\theta-\mbox{\tiny cellules}} \ln \Omega(\theta_a), \\
    & \approx &   L\mathcal{S}_{YY} [ \rho ] , 	
\end{eqnarray}
où l’on définit l’\textbf{entropie de Yang–Yang} par la formule discrétisée :
\begin{eqnarray}
    \mathcal{S}_{YY}[\rho] & \doteq & \sum_a^{\theta-\mbox{\tiny cellules}} \, [ \rho_s\ln \rho_s - \rho \ln \rho - ( \rho_s - \rho ) \ln ( \rho_s - \rho ) ] (\theta_a) \delta \theta .
\end{eqnarray}

%\paragraph{Énergie généralisée.}	
%Les variations de $w(\theta)$ étant négligeables sur chaque tranche de largeur $\delta\theta$, on peut approximer l’énergie généralisée comme :%  $\sum_{a = 1}^N  f(\theta_a) = \sum_{a \vert tranche } f(\theta_a) \Pi( \theta_a)\delta \theta$.

%\begin{eqnarray}
%	 \mathcal{W} & = & \sum_{a = 1}^N  w(\theta_a)	 ~ \sim ~ L\mathcal{W}[\rho] ~=~ L \sum_a^{\theta-\mbox{\tiny tranches}}	 w(\theta_a) \rho(\theta_a) \delta \theta.
%\end{eqnarray}

\paragraph{Énergie généralisée par unité de longueur : définition.}

Dans le cadre du Generalized Gibbs Ensemble (GGE), l’\textbf{énergie généralisée} associée à une distribution de rapidité $\rho(\theta)$ et à un poids spectral $w(\theta)$ est définie comme la somme des poids assignés à chaque quasi-particule. 
Dans la limite thermodynamique, en supposant que $w(\theta)$ varie lentement sur chaque tranche $[\theta_a, \theta_a + \delta\theta]$ ,  cette somme soit l’\textbf{énergie généralisée par unité de longueur} $\mathcal{W}$ se se définit par :
\begin{eqnarray}
	L \mathcal{W}(\{\theta_a\}) \doteq  \sum_{a = 1}^N w(\theta_a) 
	 \underset{\mbox{\tiny therm .}}{\sim}  L \mathcal{W}[\rho]  \doteq  L \sum_a^{\theta\text{-cellules}} w(\theta_a) \rho(\theta_a)\, \delta\theta. 
\end{eqnarray} 
%La fonctionnelle
%\(
%\mathcal{W}[\rho] = \int d\theta\, w(\theta)\, \rho(\theta)
%\)
%représente donc l’énergie généralisée par unité de longueur, dans l’état macroscopique défini par la distribution $\rho$.


\paragraph{Moyenne des Observables locales dans la limite thermodynamique.}

Dans un ensemble général (GGE), la valeur moyenne de l’observable \eqref{chap.2.moyenne.2} devient :	
	
\begin{eqnarray}\label{chap.2.moyenne.3}
	\underset{\mbox{\tiny therm.}}{\lim} \langle \operator{\mathcal{O}} \rangle_{\operator{\varrho}[w]} &  \approx &  ~ \frac{  \displaystyle \sum_{\rho }  \langle \operator{\mathcal{O}}\rangle_{[\rho]}  e^{L(\mathcal{S}_{YY}[\rho] -  \mathcal{W}[\rho]) }}{ \displaystyle \sum_{\rho } e^{L(\mathcal{S}_{YY}[\rho] -  \mathcal{W}[\rho]) } },
\end{eqnarray}
où la somme $\sum\rho$ porte sur toutes les distributions possibles de rapidité $\rho$

%%%%%%%%%%%%%%%%%%%%%%%%%%%%%%%%%%%%%%%%%
\paragraph{Passage à la limite continue.}
%En faisant tandre $\delta \theta \to 0 $ , les somme devienen des integrales 
En faisant tendre $\delta\theta \to 0$, les sommes deviennent des intégrales 
%\(
%\sum_a^{\theta-\mbox{\tiny tranches}}\delta \theta   \underset{\delta \theta \to 0 }{\rightarrow}  \int d \theta ,	
%\)
et l'entropie de Yang-Yang ainsi que l’énergie généralisée par unité de longueur prennent la forme :
\begin{eqnarray}
	\mathcal{S}_{YY}[\rho] & = & \int d \theta  \, [ \rho_s\ln \rho_s - \rho \ln \rho - ( \rho_s - \rho ) \ln ( \rho_s - \rho ) ] (\theta) , \label{chap.2.entropi.int}\\
	\mathcal{W}[\rho] & = & \int   w(\theta) \rho(\theta) \, d \theta \label{chap.2.W.int}		
\end{eqnarray}

%%%%%%%%%%%%%%%%%%%%%%%%%%%%%%%%%%%%%%%%%%
\paragraph{Formule fonctionnelle pour les moyennes.}

%et la valeur moyenne des opservables $\langle \operator{\mathcal{O}} \rangle$ s'écrit commes une intégrale de chemin/formelle
Dans la limite thermodynamique $L \to \infty$, la somme sur les distributions de rapidité $\rho$ admissibles peut être approximée par une intégrale fonctionnelle sur l’espace des densités de rapidité continues, munie d’une mesure fonctionnelle $\mathcal{D}\rho$ : 
\(
\sum_{\rho } \sim \int \mathcal{D} \rho .
\)
Cette correspondance repose sur l’idée que les macro-états admissibles deviennent denses dans l’espace fonctionnel, et que le poids statistique associé à chaque configuration est donné par l’entropie de Yang–Yang.
La mesure fonctionnelle $\mathcal{D}\rho$ parcourt l’espace des densités
$\rho(\theta)$ continues, \emph{chaque configuration étant pondérée par le
facteur exponentiel}
\(
e^{\,L(\mathcal{S}_{YY}[\rho]-\mathcal{W}[\rho])}.
\)
Finalement, la moyenne d'une observable dans le GGE \eqref{chap.2.moyenne.3} s’écrit comme une intégrale fonctionnelle/de chemin :
\begin{eqnarray}
	\underset{\mbox{\tiny therm.}}{\lim} \langle \operator{\mathcal{O}} \rangle_{\operator{\varrho}[w]} & = & \frac{\int \mathcal{D} \rho \; e^{L (\mathcal{S}_{YY}[\rho] - \mathcal{W}[\rho])} \, \langle\operator{\mathcal{O}}\rangle_{[\rho]}}{\int \mathcal{D} \rho \; e^{L (\mathcal{S}_{YY}[\rho] - \mathcal{W}[\rho])}}. \label{chap:TBA:eq:ensemble_average}
\end{eqnarray}


%----------------------
%------------------------------------------------------------------
%\paragraph{Passage de la somme discrète à l’intégrale fonctionnelle.}

%Dans la limite thermodynamique $L\to\infty$, l’ensemble (discret) des
%distributions de rapidité admissibles devient dense dans l’espace
%fonctionnel ; la somme correspondante peut donc s’approximer par une
%intégrale fonctionnelle :
%\[
%\sum_{\rho}\; \longrightarrow\; \int\! \mathcal{D}\rho .
%\]
%La mesure fonctionnelle $\mathcal{D}\rho$ parcourt l’espace des densités
%$\rho(\theta)$ continues, \emph{chaque configuration étant pondérée par le
%facteur exponentiel}
%\(
%e^{\,L\bigl[\mathcal{S}_{YY}[\rho]-\mathcal{W}[\rho]\bigr]},
%\)
%qui combine
%\begin{itemize}
%\item l’\textbf{entropie de Yang–Yang}
%      $\displaystyle
%        \mathcal{S}_{YY}[\rho]
%        =\!\int d\theta\,
%          \bigl[
%            \rho_s\ln\rho_s
%            -\rho\ln\rho
%            -(\rho_s-\rho)\ln(\rho_s-\rho)
%          \bigr]$,
%\item le \textbf{coût énergétique généralisé}
%      $\displaystyle
%        \mathcal{W}[\rho]
%        =\!\int d\theta\, w(\theta)\,\rho(\theta)$,
%\end{itemize}
%où $w(\theta)$ est le \emph{poids spectral} fixé par le GGE.

%------------------------------------------------------------------
%\paragraph{Moyenne d’une observable dans le GGE.}

%On obtient alors la formule de champ moyen
%\begin{equation}\label{eq:GGE-functional-average}
%\bigl\langle\mathcal{O}\bigr\rangle_{\!{\rm GGE}}
%=
%\frac{\displaystyle
%      \int \mathcal{D}\rho\;
%      e^{L\bigl[\mathcal{S}_{YY}[\rho]-\mathcal{W}[\rho]\bigr]}\,
%      \langle\mathcal{O}\rangle_{[\rho]}}
%     {\displaystyle
%      \int \mathcal{D}\rho\;
%      e^{L\bigl[\mathcal{S}_{YY}[\rho]-\mathcal{W}[\rho]\bigr]}}.
%\end{equation}

%------------------------------------------------------------------
\paragraph{Interprétation thermodynamique.}

\begin{itemize}[label = $\bullet$] 
\item $\mathcal{S}_{YY}[\rho]$ \emph{compte} le logarithme du nombre de
      micro-états réalisant la distribution $\rho(\theta)$ :
      c’est l’\textbf{entropie combinatoire}.
\item $\mathcal{W}[\rho]$ mesure le \emph{coût énergétique généralisé}
      associé à cette distribution, dicté par le poids spectral $w(\theta)$.
\end{itemize}

Leur différence
\[
(\mathcal{S}_{YY}-\mathcal{W})[\rho]
\]
joue donc le rôle d’une \emph{fonction thermodynamique effective}
(analogue à une entropie libre).  
L’exposant $e^{L(\mathcal{S}_{YY}-\mathcal{W})[\rho]}$ fixe la \textbf{probabilité relative} d’un
macro-état $\rho(\theta)$ dans le GGE : le terme entropique favorise la
multiplicité des états, tandis que le terme énergétique pénalise les
configurations coûteuses — d’où la compétition caractéristique de
l’équilibre statistique.





%avec $\mathcal{O}[\rho]$ la valeur de l’observable dans un état propre caractérisé par la distribution de rapidité $\rho$.	
%où $\mathcal{O}[\rho]$ est la valeur de l’observable dans un état propre caractérisé par la distribution $\rho$.

%\input{preamble}

\begin{document}

\frontmatter
%\input{chapters/00_intro}
\tableofcontents
\mainmatter

\input{chapters/01_LL_BA}
\input{chapters/02_GGE_TBA}
\input{chapters/03_GHD}
%\input{chapters/97_GHD}
\input{chapters/04_GGE_Fluctuation}
\input{chapters/05_Disp_Exp}
\input{chapters/06_Bipart}
\input{chapters/07_Dipolaire}

%\input{chapters/08_conclusion}
%\appendix
%\input{chapters/99_annexes}

\bibliographystyle{abbrv}
\bibliography{thesis}

%\printbibliography

\end{document}

%| Style     | Description                                                             |
%| --------- | ----------------------------------------------------------------------- |
%| `plain`   | Tri alphabétique, numérotation croissante                               |
%| `unsrt`   | Même que `plain` mais sans tri, respecte l’ordre d’apparition           |
%| `abbrv`   | Comme `plain` mais avec prénoms et noms abrégés                         |
%| `alpha`   | Les références sont étiquetées par une combinaison du nom et de l’année |
%| `apalike` | Style APA simplifié                                                     |
%| `ieeetr`  | Style IEEE, tri par ordre d’apparition                                  |
%| `siam`    | Style SIAM (mathématiques appliquées)                                   |
%| `acm`     | Style ACM (informatique)                                                |
%



\subsection{Équations intégrales de la TBA}

\paragraph{Moyenne des observables dans l’ensemble généralisé de Gibbs.}

\paragraph{Approximation au point selle («\,méthode de la selle statique\,»)}

Dans la limite thermodynamique \( L \to \infty \), cette intégrale est dominée par la configuration \( \rho_{eq} \) qui maximise le poids exponentiel $e^{L(\mathcal{S}_{YY}-\mathcal{W})[\rho]}$  dans l'expression \eqref{chap:TBA:eq:ensemble_average}. Il s’agit de la densité de rapidité la plus probable, solution d’un problème de maximisation. On obtient à l’ordre principal
\begin{eqnarray}
	\underset{\mbox{\tiny therm.}}{\lim} \langle \operator{\mathcal{O}} \rangle_{\operator{\varrho}[w]} & \approx &  \langle\operator{\mathcal{O}}\rangle_{[\rho_{eq} ]},	
	\label{chap:TBA:eq:ensemble_average:approx}
\end{eqnarray}
où $\rho_{eq}$ est la distribution de rapidité à l'équilibre \eqref{chap.2:eq.rho.eq.1}.
Cette approximation correspond à une méthode de \textit{selle statique}, où l’on développe la \emph{fonction thermodynamique effective}, $\mathcal{S}_{YY}-\mathcal{W}$  au voisinage de la distribution dominante.


\paragraph{Développement fonctionnel au premier ordre.}

%On effectue un développement de Taylor fonctionnel de l'action à l’ordre linéaire en $\rho = \rho_{eq} + \delta \rho$ :
Écrivons
\(
\rho=\rho_{\text{eq}}+\delta\rho
\)
et développons $(\mathcal{S}_{YY}-\mathcal{W})[\rho]$ à l’ordre linéaire :
\begin{eqnarray*}
	\mathcal{S}_{YY}[\rho] - \mathcal{W}[\rho] & \approx & \mathcal{S}_{YY}[ \rho_{eq}] - \mathcal{W}[ \rho_{eq}] +  \left. \frac{\delta (\mathcal{S}_{YY}[\rho] - \mathcal{W}[\rho]) }{\delta \rho} \right|_{\rho = \rho_{eq} }	(\delta \rho) + \mathcal{O}(\delta \rho^2 ) ,
	\label{chap:TBA:eq:action}	
\end{eqnarray*}	
La condition de stationnarité au point selle impose :
\(
	\left. \frac{\delta (\mathcal{S}_{YY}[\rho] - \mathcal{W}[\rho]) }{\delta \rho} \right|_{\rho = \rho_{eq} }	  =  0  	
\)
soit 
\begin{equation}
\left. \frac{\delta \mathcal{S}_{YY}}{\delta \rho} \right|_{\rho = \rho_{eq}} = \left. \frac{\delta \mathcal{W}}{\delta \rho} \right|_{\rho = \rho_{eq}}. \label{chap:TBA:eq:stationnarite}
\end{equation}

%%%%%%%%%%%%%%%%
%-----------------------------------------------------

%------------------------------------------------------------------
%\subsection{Équations intégrales de la TBA}

%\paragraph{Moyenne des observables dans le Generalized Gibbs Ensemble.}

%Dans la limite thermodynamique, la moyenne d’une observable locale
%s’écrit formellement comme une intégrale fonctionnelle sur les densités de
%rapidité\,\footnote{%
%La mesure fonctionnelle $\mathcal{D}\rho$ est la limite continue de la
%somme discrète sur les macro-états admissibles, chacun étant pondéré par
%le facteur combinatoire $e^{L\mathcal{S}_{YY}[\rho]}$.}
%
%\begin{equation}\label{eq:TBA:ensemble_average}
%\left\langle \mathcal{O} \right\rangle_{\!\text{GGE}}
%=\frac{\displaystyle
%      \int\!\mathcal{D}\rho\;
%      e^{L\bigl[\mathcal{S}_{YY}[\rho]-\mathcal{W}[\rho]\bigr]}\;
%      \langle\mathcal{O}\rangle_{[\rho]}}
%     {\displaystyle
%      \int\!\mathcal{D}\rho\;
%      e^{L\bigl[\mathcal{S}_{YY}[\rho]-\mathcal{W}[\rho]\bigr]}} .
%\end{equation}

%------------------------------------------------------------------
%\paragraph{Approximation au point selle («\,méthode de la selle statique\,»).}

%Lorsque $L\to\infty$, les intégrales \eqref{eq:TBA:ensemble_average}
%sont dominées par la distribution
%$\rho_{\text{eq}}$ qui \emph{maximise} l’exposant
%\(
%\Phi[\rho]=\mathcal{S}_{YY}[\rho]-\mathcal{W}[\rho].
%\)
%On obtient à l’ordre principal
%\begin{equation}
%\left\langle \mathcal{O} \right\rangle_{\!\text{GGE}}
%\;\simeq\;
%\langle \mathcal{O} \rangle_{[\rho_{\text{eq}}]} .
%\label{eq:TBA:saddle_average}
%\end{equation}

%------------------------------------------------------------------
%\paragraph{Condition de stationnarité et équation variationnelle.}

%Écrivons
%\(
%\rho=\rho_{\text{eq}}+\delta\rho
%\)
%et développons $\Phi[\rho]$ à l’ordre linéaire :
%\[
%\Phi[\rho]\;=\;
%\Phi[\rho_{\text{eq}}]
%+
%\int d\theta\,
%\left.
%\frac{\delta\Phi}{\delta\rho(\theta)}
%\right|_{\rho_{\text{eq}}}
%\delta\rho(\theta)
%+O(\delta\rho^{2}).
%\]
%La stationnarité impose
%\(
%\dfrac{\delta\Phi}{\delta\rho(\theta)}\bigl|_{\rho_{\text{eq}}}=0,
%\)
%soit
%\begin{equation}
%\left.
%\frac{\delta\mathcal{S}_{YY}}{\delta\rho(\theta)}
%\right|_{\rho_{\text{eq}}}
%=
%\left.
%\frac{\delta\mathcal{W}}{\delta\rho(\theta)}
%\right|_{\rho_{\text{eq}}}.
%\label{eq:TBA:variational_condition}
%\end{equation}

%------------------------------------------------------------------
%\paragraph{Forme explicite : introduction de la pseudo-énergie.}

%Pour le modèle de Lieb–Liniger (et, plus généralement, pour un modèle
%intégrable à noyau $\Delta$), on introduit la \emph{pseudo-énergie}
%\[
%\varepsilon(\theta)
%\;=\;
%w(\theta)
%\;+\;\Bigl[\Delta\star\ln\!\bigl(1+e^{-\varepsilon}\bigr)\Bigr](\theta),
%\]
%obtenue en réécrivant \eqref{eq:TBA:variational_condition}.
%Le \emph{facteur d’occupation}
%\(
%\nu(\theta)=\rho(\theta)/\rho_s(\theta)
%\)
%se donne alors par la statistique de type Fermi-Dirac
%\[
%\nu(\theta)=\frac1{1+e^{\varepsilon(\theta)}}.
%\]

%Les équations intégrales complètes de la \textbf{Thermodynamique de Bethe}
%(TBA) sont donc
%\begin{align}
%2\pi\rho_s(\theta) &= 1 + \bigl[\Delta \star \rho\bigr](\theta),
%\label{eq:TBA:rho_s}\\[4pt]
%\rho(\theta) &= \frac{\rho_s(\theta)}{1+e^{\varepsilon(\theta)}},
%\qquad
%\varepsilon(\theta)=w(\theta)+\bigl[\Delta\star\ln(1+e^{-\varepsilon})\bigr](\theta).
%\label{eq:TBA:epsilon}
%\end{align}
%Elles déterminent sans ambiguïté la distribution d’équilibre
%$\rho_{\text{eq}}(\theta)$ en fonction du poids spectral $w(\theta)$.

%\medskip
%Ainsi, la méthode du point selle relie le \emph{poids spectral}
%(caractéristique du GGE) à la distribution de rapidité la plus probable,
%et permet d’évaluer les observables par la formule
%\label{chap:TBA:eq:ensemble_average:approx}.


%-----------------------------------------------------
%%%%%%%%%%%%%%%%

%\paragraph{Équation intégrale de la TBA.}

%Cette égalité donne naissance à une équation intégrale pour le poids spectral \( w \), défini comme la dérivée fonctionnelle de l'énergie généralisée pris en $\rho_{eq}$ :
%\(
%w ~=~ \left. \frac{\delta \mathcal{W}[\rho]}{\delta \rho} \right|_{\rho =  \rho_{eq} }
%\)
%qui par stationnarité (cf équation \eqref{chap:TBA:eq:stationnarite}) est égale à la dérivée fonctionnelle de l'entropie de Yang-Yang pris en $\rho_{eq}$ :
%\(
%\left. \frac{\delta \mathcal{S}_{YY}[\rho]}{\delta \rho} \right|_{\rho = \rho_{eq} }
%\) 
%qui lui vaux 
%\(
%\ln ( \nu_{eq}^{-1}  - 1 ) - \frac{\Delta}{2\pi} \star \ln ( 1 -  \nu_{eq })
%\)
%avec le facteur d'ocupation à l'équilibre $\nu_{eq} = \rho_{eq}/{\rho_{eq}}_s$. Ainci on peux s'arreter sur l'équation 
%\begin{eqnarray}
%	w & = & \ln ( \nu_{eq}^{-1}  - 1 ) - \frac{\Delta}{2\pi} \star \ln ( 1 -  \nu_{eq }).\label{chap:TBA:eq:w}
%\end{eqnarray}

%\medskip
%Ainsi, la méthode du point selle relie le \emph{poids spectral}
%(caractéristique du GGE) à la distribution de rapidité la plus probable,
%et permet d’évaluer les observables par la formule
%\eqref{chap:TBA:eq:ensemble_average:approx}.\\

%\paragraph{Forme explicite : introduction de la pseudo-énergie.}

%Le \emph{facteur d’occupation}
%\(
%\nu_{eq}
%\)
%se donne alors par la statistique de type Fermi-Dirac
%\begin{eqnarray}
%	\nu_{eq}=\frac1{1+e^{\epsilon}},\label{chap:TBA:eq:nu_eq}
%\end{eqnarray}
%où \emph{pseudo-énergie} 
%\(
%\epsilon
%\)
%se définie en intectant \eqref{chap:TBA:eq:nu_eq} dans \eqref{chap:TBA:eq:w} : 
%\begin{eqnarray}
%	\epsilon & = & w + \frac{\Delta}{2\pi} \star \ln ( 1  + e^{-\epsilon}).\label{chap:TBA:eq:e}	
%\end{eqnarray}


%---------------------------------
%------------------------------------------------------------------
\paragraph{Équation intégrale de la TBA.}

La condition de stationnarité au point selle \(\rho=\rho_{\mathrm{eq}}\) \eqref{chap:TBA:eq:stationnarite} implique :
\begin{eqnarray}
	\left.\frac{\delta\mathcal{S}_{YY}}{\delta\rho(\theta)}\right|_{\rho_{\mathrm{eq}}} = \left.\frac{\delta\mathcal{W}}{\delta\rho(\theta)}\right|_{\rho_{\mathrm{eq}}}\;\doteq\;w(\theta),
\end{eqnarray}
En utilisant l’expression explicite de l’entropie de Yang–Yang \eqref{chap.2.entropi.int}, on obtient l’identité fonctionnelle
\begin{eqnarray}
	w & = & \ln ( \nu_{\!eq}^{-1}  - 1 ) - \frac{\Delta}{2\pi} \star \ln ( 1 -  \nu_{\!eq}).\label{chap:TBA:eq:w}
\end{eqnarray}
où
\(
\nu_{\!eq}=\rho_{\!eq}/\rho_{s,\!eq}
\)
est le \textbf{facteur d’occupation} à l’équilibre.
%------------------------------------------------------------------
\paragraph{Forme pseudo-énergie.}
La \textbf{pseudo-énergie} $\epsilon$ se donne alors par la statistique de type Fermi-Dirac
\begin{eqnarray}
	\epsilon =\ln(\nu^{-1}_{\!eq}-1),\qquad\nu_{\!eq}=\frac{1}{1+e^{\epsilon}}.\label{chap:TBA:eq:nu}%\tag{\text{TBA--$\nu$}} 
\end{eqnarray}
En réinjectant \eqref{chap:TBA:eq:nu} dans \eqref{chap:TBA:eq:w} on obtient
l’équation intégrale canonique de la thermodynamique de Bethe :
\begin{eqnarray}
	\epsilon & = & w - \frac{\Delta}{2\pi} \star \ln ( 1  + e^{-\epsilon}).\label{chap:TBA:eq:e}%\tag{\text{TBA–-$\varepsilon$}}	
\end{eqnarray}
%\[
%\boxed{\;
%\varepsilon(\theta)
%=
%w(\theta)
%+\frac{\Delta}{2\pi}\star\ln\!\bigl[1+e^{-\varepsilon(\theta)}\bigr]
%\;}
%\tag{TBA–$\varepsilon$}\label{eq:TBA:eq:e}
%\]

Les relations \eqref{chap:TBA:eq:nu}–\eqref{chap:TBA:eq:e} déterminent de façon univoque la distribution de rapidité d’équilibre \(\rho_{\!eq}\) à partir du poids spectral \(w\), caractéristique du GGE.

\medskip
Ainsi, la méthode du point selle relie \emph{explicitement} le {\em poids spectral}, $w$  (caractéristique du GGE) au \emph{macro-état le plus probable}, $\rho_{eq}$ , et permet d’évaluer les observables par la formule d’ensemble \eqref{chap:TBA:eq:ensemble_average:approx}.


\paragraph{Résolution numérique de l’équation TBA.}\label{para-algho-TBA}

Prenons un poids spectrale quelconque, par exemple : 
\begin{equation}
  w(\theta)= \theta^2 .\label{eq:TBA:w:quadra}
\end{equation} 
En injectant $w$ dans l’équation intégrale pour lapseudo-énergie \eqref{chap:TBA:eq:e}, on obtient l’équation non linéaire.
Cette équation définit un opérateur contractant sur l’espace des fonctions
\( \epsilon(\theta) \) ; son Jacobien a une norme strictement
inférieure à 1, garantissant la convergence de l’itération de Picard.

\medskip
\subparagraph{Algorithme d’itération.}  
La structure contractante de l’équation garantit l’absence de cycles ou de points fixes multiples, assurant la convergence de l’itération vers l’unique solution admissible.
L’équation \eqref{eq:num:TBA} est non linéaire ; pour la résoudre numériquement, on utilise une méthode itérative de type Picard. On initialise
\(
  \epsilon_0 = w ,
\)
puis on construit une suite de fonctions \(\varepsilon_n\) définie par
\begin{eqnarray*}
	\epsilon_{n+1} & = & \epsilon_0 -   \frac{\Delta}{2\pi} \star \ln \left( 1 + e^{-\epsilon_n} \right) ,\quad n\ge0
\end{eqnarray*}
L’itération est poursuivie jusqu’à convergence, que l’on peut tester via le critère numérique
\(
  \beta \left\| \varepsilon_{n+1} - \varepsilon_n \right\|_\infty < 10^{-12},
\)
où \(\|\cdot\|_\infty\) désigne la norme \(L^\infty\) (ou un maximum discret après discrétisation).


\medskip
\subparagraph{Facteur d’occupation et densités.}  
Une fois la pseudo-énergie \( \epsilon(\theta) \) convergée, le facteur d’occupation  à l'équilibre est obtenu en injectant $\epsilon$ dans l’équation \eqref{chap:TBA:eq:nu}, ce qui donne  $\nu_{\!eq}$.
 
On en déduit ensuite la densité d'état à l'équilibre $\rho_{s,eq}$ via le {\bf dressing}  de la fonction constante $f(\theta) = 1$, selon \eqref{eq:TBA-rhos-2}, rappelée ici pour mémoire : $ 2\pi \rho_{s,eq}  =  1^{\mathrm{dr}}_{[\nu_{\! eq}]}$.\\

L’opérateur de dressing \eqref{eq:dressing} étant linéaire, il se résout numériquement sous la forme :
\begin{eqnarray*}
	\left\{ \mathrm{id} - \frac{\Delta}{2\pi} \star ( \nu \ast \cdot ) \right\} f^{\mathrm{dr}}_{[\nu]} & = & f,\label{eq:TBA:rho_s:num}
\end{eqnarray*}
où $\mathrm{id} \colon f \mapsto f$ est l’identité fonctionnelle, et $\ast$ désigne la multiplication.
Après discrétisation de la variable $\theta$, cette équation devient un système linéaire de type $Ax=b$ , facilement résoluble numériquement.

La distribution de rapidité est alors obtenue par $\rho_{\!\mathrm{eq}} = \nu_{\!\mathrm{eq}} \ast \rho_{\! s,\mathrm{eq}}$.\\

\medskip
Ainsi en fixant le poids spectral $w(\theta)$, l’algorithme fournit la pseudo-énergie \( \epsilon \), le facteur d’occupation \( \nu_{\mathrm{eq}} \) et la distribution de rapidité \( \rho_{\!\mathrm{eq}} \).

\medskip
\subparagraph{À l'équilibre thermique.} 
Si on se place à l’équilibre canonique, caractérisé par la température \( T \) et le potentiel chimique \( \mu \).  Dans ce cadre, le poids spectral vaut
\begin{equation}
  w(\theta)=\beta\bigl[\varepsilon(\theta)-\mu\bigr],\qquad\beta=\tfrac1T\; (k_B = 1 ),\quad\varepsilon(\theta)=\tfrac{\theta^{2}}{2}\;(m=1).\label{eq:TBA:w:canonical}
\end{equation}
%En injectant \eqref{eq:TBA:w:canonical} dans l’équation intégrale pour lapseudo-énergie \eqref{chap:TBA:eq:e}, on obtient l’équation non linéaire :
%\begin{eqnarray*}
%	\epsilon & = & \beta(\varepsilon - \mu)  -  \frac{\Delta}{2\pi} \star \ln \left( 1 + e^{-\epsilon} \right) ,\label{eq:num:TBA}
%\end{eqnarray*}
%Ainsi, pour tout couple \((T,\mu)\), l’algorithme fournit la pseudo-énergie \( \epsilon \), le facteur d’occupation \( \nu_{\mathrm{eq}} \) et la distribution de rapidité \( \rho_{\mathrm{eq}} \) à l’équilibre thermique, prêts à être utilisés pour le calcul des observables.
%
%\medskip
%Pour $w$ quelconque , l'algorythme est identique.




		


\chapter{Dynamique hors-équilibre et hydrodynamique généralisée}
\label{chap:GHD}
\minitoc

%\chapter{Hydrodynamique généralisée (GHD)}

\section*{Introduction}


\paragraph{De l’état stationnaire à la dynamique}  
Après avoir étudié les propriétés stationnaires des gaz de bosons unidimensionnels, nous nous tournons désormais vers leur évolution temporelle. Ce chapitre s’appuie sur une approche hydrodynamique adaptée aux systèmes intégrables : la théorie dite d’Hydrodynamique Généralisée (GHD). Celle-ci est largement documentée dans la littérature (voir par exemple [50, 24, 51, 52]) et nous en présentons ici les concepts essentiels.

\paragraph{Principe général d’une approche hydrodynamique}  
De manière générale, l’hydrodynamique vise à décrire la dynamique à grande échelle (\emph{coarse grained dynamics}) d’un système, également appelée « échelle d’Euler ». L’idée consiste à découper l’espace-temps d’un système de taille $L$ en cellules de dimensions $\ell \times \tau$, comme illustré en Fig.~???.  
La longueur $\ell$ est choisie de sorte que $L \gg \ell \gg \ell_c$, où $\ell_c$ désigne une longueur microscopique caractéristique, par exemple la distance inter-particule. On peut alors considérer que la densité est uniforme à l’intérieur de chaque cellule, ce qui correspond à l’Approximation de Densité Locale.

\paragraph{Choix des échelles spatio-temporelles}  
Le temps $\tau$ est fixé pour être beaucoup plus grand que le temps caractéristique de relaxation. Ainsi, chaque cellule de l’espace-temps est supposée décrire un état localement relaxé. La notion de relaxation occupe donc une place centrale dans la construction des approches hydrodynamiques.

\paragraph{Particularités pour les systèmes quantiques isolés}  
Dans le cadre de systèmes quantiques isolés, la relaxation n’est pas un concept trivial, qu’il s’agisse de systèmes chaotiques ou intégrables. La section suivante s’attache à définir plus précisément cette notion, avant de présenter les approches hydrodynamiques adaptées à chaque cas. Pour les systèmes intégrables, une attention particulière est portée à la formulation et aux implications de l’Hydrodynamique Généralisée.

\paragraph{Équations hydrodynamiques de type Euler}  
Les équations hydrodynamiques de type Euler sont des équations hyperboliques qui décrivent la dynamique émergente des systèmes à plusieurs corps à grandes échelles d’espace et de temps~\cite{ref1}. Elles rendent compte de la propagation de la relaxation locale, c’est-à-dire la séparation entre une dynamique lente, émergente, et la projection rapide des observables locales sur les quantités conservées. En une dimension d’espace, elles prennent la forme locale de conservation
\begin{equation}\label{chap:GHD:eq.conserv.1}
	\partial_t q_i + \partial_x j_i = F_i,	
\end{equation}
où l’indice $i$ énumère les lois de conservation locales admises, et où $F_i$ représente les contributions provenant de champs de force externes, qui rompent en général la conservation stricte.

\paragraph{Relations constitutives et exemples}  
Les flux $j_i$ et les termes de force $F_i$ dépendent uniquement des densités conservées $q_i$ (équations d’état), et sont déterminés à partir de considérations thermodynamiques, telles que la maximisation de l’entropie. Les équations d’Euler pour un fluide galiléen, ou encore l’hydrodynamique relativiste, constituent des exemples classiques de ce type d’équations.

\paragraph{Cas intégrable et hydrodynamique généralisée}  
En dimension un, de nombreux systèmes à plusieurs corps présentent une propriété d’intégrabilité~\cite{ref2,ref3}. Dans ce contexte, il existe une infinité de lois de conservation, et la théorie universelle qui décrit leur hydrodynamique à l’échelle d’Euler est l’Hydrodynamique Généralisée (GHD)~\cite{ref4,ref5}. Cette approche englobe les équations connues pour les bâtons durs~\cite{ref1,ref6} et les gaz de solitons~\cite{ref7,ref8,ref9}, tout en s’appliquant plus largement, aussi bien à des systèmes classiques que quantiques : particules en interaction, chaînes de spins ou théories des champs quantiques (voir~\cite{ref10} pour des revues).

\paragraph{Paramétrisation spectrale et densité conservée}  
La GHD reformule l’infinité de lois de conservation (éventuellement rompues) en une famille indexée par un paramètre spectral continu $\theta$, plutôt que par un indice discret $i$. On note $\rho(x,\theta,t)$ la densité conservée en espace réel, espace spectral et temps. Le paramètre spectral énumère les objets asymptotiques issus de la théorie de diffusion correspondante (particules, solitons, etc.), incluant leur quantité de mouvement et leurs éventuels degrés internes. Dans de nombreux cas simples, $\theta$ appartient à un sous-ensemble de $\mathbb{R}$, représentant les moments asymptotiques, et les coordonnées $(x,\theta)$ forment un « espace des phases spectral » sur lequel $\rho$ joue le rôle de densité.

\paragraph{Prise en compte des champs de force}  
L’inclusion de champs de force externes couplés aux densités conservées a été introduite dans~\cite{ref11}, où il est montré que la GHD s’écrit
\begin{equation}\label{chap:GHD:eq.GHD.1}
	\partial_t \rho + \partial_x(v^{\text{eff}} \rho) + \partial_\theta(a^{\text{eff}} \rho) = 0.
\end{equation}
Ici, $v^{\text{eff}}$ et $a^{\text{eff}}$ sont des fonctionnels appropriés de $\rho(x,\cdot,t)$, et le dernier terme représente la contribution des champs de force. D’autres types de forces ont été étudiés~\cite{Bastianello2019a,Bastianello2019b}, mais ne seront pas considérés ici.

%\section{Formulation hamiltonienne de la GHD}
%
%\subsection{Crochet de Poisson fonctionnel}
%
%\paragraph{Définition générale}
%Bonnemain \emph{et al.}~\cite{bonnemain2024hamiltonian} définissent un crochet de Poisson fonctionnel agissant sur les fonctionnelles $F$ et $G$ de la distribution de rapidité, avec interactions :
%\begin{equation}\label{chap:GHD:eq.chochet.bonnemain.1}
%	\{F,G\}=\iint dx\,d\theta\;\frac{\nu}{2\pi}\,\left[\partial_x \left ( \frac{\delta F}{\delta \rho(x,\theta)} \right )\,\left(\partial_\theta \left ( \frac{\delta G}{\delta \rho(x,\theta)} \right ) \right)^{\mathrm{dr}}_{[\nu]} -\partial_x \left ( \frac{\delta G}{\delta \rho (x,\theta)} \right ) \,\left( \partial_\theta \left ( \frac{\delta F}{\delta \rho (x,\theta)} \right )\right)^{\mathrm{dr}}_{[\nu]} \right],
%\end{equation}
%où $\nu$ est la fonction d’occupation. L'application de l’opérateur de \emph{dressing} dans ce crochet traduit les interactions entre particules.
%
%\paragraph{Cas des charges globales}
%Les charges locales conservées ont été définies en \eqref{chap.2.charge.f.1}.  
%Avec le même formalisme, les charges globales conservées se définissent comme fonctionnelles linéaires d’une fonction réelle et régulière \( f(x, \theta) \) définie sur \( \mathbb{R}^2 \) :
%\begin{equation}\label{chap:GHD:eq.charge.global.1}
%	\mathcal{Q}[f] = \int_{\mathbb{R}^2} dx\, d\theta\, f(x, \theta)\, \rho(x, \theta),
%\end{equation}
%qui représente la charge totale associée à une quantité prenant la valeur \( f(x, \theta) \) pour chaque quasi-particule.
%
%Dans notre étude de la dynamique, nous n’avons pas besoin de l’information sur le poids spectral.  
%On notera donc, dans la limite thermodynamique, les moyennes d’opérateurs simplement en retirant leur chapeau :
%\[
%\underset{\mathrm{therm}}{\lim} \braket{\mathcal{O}}_{\varrho[w]} \equiv \mathcal{O}.
%\]
%Ainsi, dans cette limite, la charge globale \eqref{chap:GHD:eq.charge.global.1} s’écrit directement comme ci-dessus.
%
%Le crochet de Poisson \eqref{chap:GHD:eq.chochet.bonnemain.1} appliqué à deux charges globales \( \mathcal{Q}[f] \) et \( \mathcal{Q}[g]\) s’écrit :
%\begin{equation}\label{chap:GHD:eq.chochet.bonnemain.2}
%	\{\mathcal{Q}[f], \mathcal{Q}[g]\} = \int_{\mathbb{R}^2} dx\, d\theta \frac{\nu}{2\pi}  \left( \partial_x f  (\partial_\theta g )^{\mathrm{dr}}_{[\nu]}  - \partial_x g (\partial_\theta f)^{\mathrm{dr}}_{[\nu]}  \right).
%\end{equation}
%L’application du dressing satisfait la symétrie~\cite{doyon2020lecture} :
%\begin{equation}\label{chap:GHD:eq.sym.dr.1}
%	\int_{\mathbb{R}^2}	 dx\, d\theta \, \nu f g^{\mathrm{dr}}_{[\nu]} = \int_{\mathbb{R}^2}	 dx\, d\theta \, \nu f^{\mathrm{dr}}_{[\nu]} g.
%\end{equation}
%Par intégration par parties, le crochet \eqref{chap:GHD:eq.chochet.bonnemain.2} devient :
%\begin{equation}\label{chap:GHD:eq.chochet.bonnemain.3}
%	\{ \mathcal{Q}[f] , \mathcal{Q}[g]\} = \int_{\mathbb{R}^2} dx\, d\theta \,   f  \left( \partial_\theta \left ( \frac{\nu }{2\pi}  (\partial_x g )^{\mathrm{dr}}_{[\nu]} \right )   - \partial_x  \left ( \frac{\nu}{2\pi}  (\partial_\theta g )^{\mathrm{dr}}_{[\nu]} \right )  \right).
%\end{equation}
%
%\subsection{Crochet avec l’Hamiltonien}
%
%\paragraph{Densité hamiltonienne et grandeurs effectives}
%On note $h(x,\theta)$ la densité associée à la moyenne de l’Hamiltonien :
%\begin{equation}\label{chap:GHD:eq.ham.1}
%	H = \mathcal{Q}[h].
%\end{equation}
%La fonction d’occupation $\nu$, la vitesse effective $v^{\mathrm{eff}}$ et l’accélération effective $a^{\mathrm{eff}}$ sont définies par :
%\begin{equation}\label{chap:GHD:eq.nu.v.a.1}
%	\nu = 2\pi \frac{\rho}{1^{\mathrm{dr}}_{[\nu]}}, \quad  
%	v^{\mathrm{eff}} = \frac{(\partial_\theta h )^{\mathrm{dr}}_{[\nu]}}{1^{\mathrm{dr}}_{[\nu]}}, \quad  
%	a^{\mathrm{eff}} = -\frac{(\partial_x h )^{\mathrm{dr}}_{[\nu]}}{1^{\mathrm{dr}}_{[\nu]}},
%\end{equation}
%fonctions de $\rho(x,\theta,t)$.
%
%Le crochet \eqref{chap:GHD:eq.chochet.bonnemain.3} appliqué à $(f,h)$ devient :
%\begin{equation}\label{chap:GHD:eq.chochet.bonnemain.4}
%	\{\mathcal{Q}[f] , \mathcal{Q}[h]\} = -\int_{\mathbb{R}^2} dx\, d\theta \,   f  \left[ \partial_x \left ( \rho  v^{\mathrm{eff}} \right )   +  \partial_\theta   \left ( \rho  a^{\mathrm{eff}} \right )  \right].
%\end{equation}
%
%\paragraph{Forme locale : densités conservées}
%En choisissant $f(x,\theta) \mapsto \delta(\cdot - x)f(\theta)$ dans \eqref{chap:GHD:eq.charge.global.1}, on obtient la densité conservée :
%\[
%q_{[f]}(x) = \mathcal{Q}[(x,\theta) \mapsto \delta(\cdot - x) f(\theta)].
%\]
%Appliquée à \eqref{chap:GHD:eq.chochet.bonnemain.4}, cette prescription donne :
%\begin{equation}\label{chap:GHD:eq.chochet.bonnemain.5}
%	\{ q_{[f]}(x) , \mathcal{Q}[h]\} = - \partial_x \left ( \int_{\mathbb{R}} d\theta \,   f  \,  \rho  \,  v^{\mathrm{eff}} \right ) + \int_{\mathbb{R}} d\theta \, f' \,    \rho \, a^{\mathrm{eff}}.
%\end{equation}
%En utilisant l’équation de Liouville \eqref{chap:GHD:eq.Liouv.1}, on retrouve la forme de convection :
%\begin{equation}\label{chap:GHD:eq.conserv.2}
%	\partial_t q_{[f]} + \partial_x j_{[f]} = F_{[f]},
%\end{equation}
%avec
%\begin{equation}\label{chap:GHD:eq.conserv.2.1}
%	j_{[f]} = \int_{\mathbb{R}} d\theta \,v^{\mathrm{eff}} \, f \, \rho, 
%	\quad F_{[f]} = \int_{\mathbb{R}} d\theta \,  a^{\mathrm{eff}} \, f' \, \rho.
%\end{equation}
%
%\paragraph{Forme locale : équation sur \texorpdfstring{$\rho$}{rho}}
%En prenant $\rho(x,\theta) = \mathcal{Q}[\delta(\cdot - x)\delta(\cdot - \theta)]$ et en l’appliquant à \eqref{chap:GHD:eq.chochet.bonnemain.4}, on obtient :
%\begin{equation}\label{chap:GHD:eq.chochet.bonnemain.6}
%	\{ \rho ( x , \theta ) , \mathcal{Q}[h]\} = - \partial_x \left (  v^{\mathrm{eff}} \,  \rho   \right ) - \partial_\theta \left (  a^{\mathrm{eff}}  \,  \rho  \right).
%\end{equation}
%En appliquant l’équation de Liouville \eqref{chap:GHD:eq.Liouv.1}, on retrouve l’équation GHD :
%\begin{equation}\label{chap:GHD:eq.conserv.3}
%	\partial_t \rho + \partial_x(v^{\mathrm{eff}} \rho) + \partial_\theta(a^{\mathrm{eff}} \rho) = 0.
%\end{equation}
%
%---------------------
\section{Formulation hamiltonienne de la GHD}

\subsection{Crochet de Poisson fonctionnel}

\paragraph{Interprétation et limite non-interactive}  
À ce niveau de généralité, l'équation de l’Hydrodynamique Généralisée (GHD) \eqref{chap:GHD:eq.GHD.1} peut être interprétée comme la dynamique hydrodynamique d’un fluide bidimensionnel dont la densité est conservée dans l’espace des phases spectral.  
Les effets d’interaction se traduisent par un couplage non local dans la direction des rapidités $\theta$, reflétant les processus de diffusion élastique entre quasi-particules possédant des paramètres spectraux distincts.

\medskip

Dans le cas limite d’un système \emph{sans interactions}, l’espace spectral coïncide avec l’espace des phases classique, et l’équation de GHD se réduit alors à l’équation de Liouville (ou, de façon équivalente, à l’équation de Boltzmann sans terme de collisions) issue de la théorie cinétique élémentaire.

\medskip

En l’absence de phénomènes dissipatifs, la densité de distribution $\rho$ est conservée le long du flot hamiltonien associé à l’énergie $H$, ce qui s’exprime par
\begin{equation}\label{chap:GHD:eq.Liouv.1}
	\frac{d \rho}{dt} 
	= \frac{\partial \rho}{\partial t } + \{ \rho , H \} = 0,
\end{equation}
où $\{\cdot , \cdot\}$ désigne le crochet de Poisson canonique dans l’espace des phases.  
Dans cette perspective, l’Hydrodynamique Généralisée apparaît comme une extension naturelle de l’équation de Liouville aux systèmes intégrables, incorporant les effets collectifs induits par les interactions tout en préservant une description exacte à grande échelle.


\paragraph{Structure hamiltonienne et crochet de Poisson fonctionnel}  
Bonnemain \emph{et al.} \cite{bonnemain2024hamiltonian} introduisent un crochet de Poisson fonctionnel agissant sur des fonctionnelles $F$ et $G$ de la distribution de rapidité $\rho(x,\theta)$ en présence d’interactions. Celui-ci s’écrit
\begin{equation}\label{chap:GHD:eq.chochet.bonnemain.1}
	\{F,G\}
	=\iint dx\,d\theta\;\frac{\nu}{2\pi}\,
	\left[
		\partial_x \left( \frac{\delta F}{\delta \rho(x,\theta)} \right)
		\left( \partial_\theta \left( \frac{\delta G}{\delta \rho(x,\theta)} \right) \right)^{\mathrm{dr}}_{[\nu]}
		-
		\partial_x \left( \frac{\delta G}{\delta \rho(x,\theta)} \right)
		\left( \partial_\theta \left( \frac{\delta F}{\delta \rho(x,\theta)} \right) \right)^{\mathrm{dr}}_{[\nu]}
	\right],
\end{equation}
où $\nu$ désigne la fonction d’occupation. Dans ce crochet l'application de l’opérateur de \emph{dressing} $(\cdot)^{\mathrm{dr}}_{[\nu]}$ (introduit dans \eqref{eq:dessing})  traduit les interactions entre particules.

%L’opérateur de \emph{dressing} $(\cdot)^{\mathrm{dr}}_{[\nu]}$ agit ici sur les dérivées fonctionnelles dans la variable spectrale $\theta$ ; il encode les effets des interactions à longue portée dans l’espace des rapidités. Cette structure hamiltonienne permet de reformuler la GHD comme une équation de type Liouville sur l’espace fonctionnel des distributions $\rho$, mais avec un crochet de Poisson modifié par le \emph{dressing}, traduisant la nature intégrable et non-locale des interactions.

\medskip

\paragraph{Charges globales conservées}  
Les charges locales conservées ont été définies dans les équations~\eqref{chap.2.charge.f.1}.  
Dans le même formalisme, on définit les \emph{charges globales conservées} comme des fonctionnelles linéaires agissant sur une fonction réelle et régulière $f(x,\theta)$ définie sur $\mathbb{R}^2$, selon
\begin{equation}\label{chap:GHD:eq.charge.global.0}
	\operator{\mathcal{Q}}[f] 
	= \int_{\mathbb{R}^2} dx\, d\theta\, f(x, \theta)\, \operator{\rho}(x, \theta),
\end{equation}
où $\operator{\rho}(x,\theta)$ est l'opérateur distribution de rapidité.  
Cette quantité correspond à la charge totale associée à une observable prenant la valeur $f(x,\theta)$ pour chaque quasi-particule.

\medskip

La valeur moyenne $\langle \operator{\mathcal{Q}}[f] \rangle_{\operator{\varrho}[w]}$ a été définie en~\eqref{chap.TBA.moy.dens}.  
La matrice densité locale $\operator{\varrho}^{(\mathcal{S})}[w]$ a été introduite en~\eqref{chap.2.densite.1}.  
De manière analogue, la \emph{matrice densité globale} $\operator{\varrho}[w]$ s’écrit
\begin{equation}\label{chap:GHD:eq.charge.global.2}
	\operator{\varrho}[w] 
	= \frac{1}{Z[w]}\, e^{-\operator{\mathcal{Q}}[w]}, 
	\qquad  
	Z[w] = \mathrm{Tr} \left[ e^{-\operator{\mathcal{Q}}[w]} \right],
\end{equation}
où la charge globale $\operator{\mathcal{Q}}[w]$ est définie par~\eqref{chap:GHD:eq.charge.global.0}, et $w$ désigne le poids spectral.  

%Cette formulation met en évidence le lien entre la description statistique du système et la conservation des charges globales, en généralisant le principe de Gibbs aux systèmes intégrables par l’introduction de l’ensemble d’observables $\operator{\mathcal{Q}}[f]$ sur l’espace spectral.

\medskip

\paragraph{Crochet de Poisson entre charges globales}  
Dans notre étude de la dynamique, nous n’avons pas besoin de l’information détaillée sur le poids spectral $w$.  
Nous noterons donc, dans ce chapitre, et dans la limite thermodynamique, les moyennes des opérateurs en supprimant leur chapeau, \emph{i.e.}
\begin{equation}
\underset{\mathrm{therm}}{\lim} \, \langle \operator{\mathcal{O}} \rangle_{\varrho[w]} \; \equiv \; \mathcal{O},
\end{equation}
de sorte que, dans cette limite, la moyenne de la charge globale s’écrit
\begin{equation}\label{chap:GHD:eq.charge.global.1}
	\mathcal{Q}[f] 
	= \int_{\mathbb{R}^2} dx\, d\theta\, f(x, \theta)\, \rho(x, \theta),
\end{equation}
où $f$ est une fonction régulière sur $\mathbb{R}^2$.

\medskip

Le crochet de Poisson (défini en~\eqref{chap:GHD:eq.chochet.bonnemain.1}) entre deux charges $\mathcal{Q}[f]$ et $\mathcal{Q}[g]$ prend la forme
\begin{equation}\label{chap:GHD:eq.chochet.bonnemain.2}
	\{\mathcal{Q}[f], \mathcal{Q}[g]\}
	= \int_{\mathbb{R}^2} dx\, d\theta\, \frac{\nu}{2\pi} 
	\left[ \partial_x f \, (\partial_\theta g)^{\mathrm{dr}}_{[\nu]} 
	     - \partial_x g \, (\partial_\theta f)^{\mathrm{dr}}_{[\nu]} \right].
\end{equation}
%où $\nu$ est la fonction d’occupation et $(\cdot)^{\mathrm{dr}}_{[\nu]}$ désigne l’application de \emph{dressing} associée à $\nu$.

\medskip

Cette application de \emph{dressing} satisfait la relation de symétrie~\cite{doyon2020lecture} :
\begin{equation}\label{chap:GHD:eq.sym.dr.1}
	\int_{\mathbb{R}^2} dx\, d\theta \; \nu \, f \, g^{\mathrm{dr}}_{[\nu]} 
	= \int_{\mathbb{R}^2} dx\, d\theta \; \nu \, f^{\mathrm{dr}}_{[\nu]} \, g.
\end{equation}

Pour appliquer la relation de symétrie~\eqref{chap:GHD:eq.sym.dr.1} au crochet~\eqref{chap:GHD:eq.chochet.bonnemain.2}, il est nécessaire de vérifier que les fonctions impliquées satisfont les conditions requises sur leurs types tensoriels.
\footnote{
La relation de symétrie~\eqref{chap:GHD:eq.sym.dr.1} est valable lorsque la somme des types tensoriels de $f$ et $g$ est $(1,1)$ dans le sens de~\cite{doyon2020lecture}. Dans ce formalisme, le type $(a,b)$ caractérise la transformation d'un objet vis-à-vis de $x$ (première entrée) et de $\theta$ (seconde entrée). Si $f$ est de type $(p,q)$ et $g$ de type $(r,s)$, alors leur somme est $(p+r,q+s)$. La condition $(1,1)$ garantit que l'intégrande $\nu\, f\, g^{\mathrm{dr}}$ est un scalaire invariant, rendant l'intégrale bien définie. Dans~\eqref{chap:GHD:eq.chochet.bonnemain.2}, $\partial_x f$ est de type $(1,0)$ et $\partial_\theta g$ de type $(0,1)$, ce qui satisfait cette condition et permet l'utilisation de~\eqref{chap:GHD:eq.sym.dr.1}.
}

En utilisant cette symétrie ainsi qu’une intégration par parties, le crochet~\eqref{chap:GHD:eq.chochet.bonnemain.2} se réécrit
\begin{equation}\label{chap:GHD:eq.chochet.bonnemain.3}
	\{\mathcal{Q}[f], \mathcal{Q}[g]\}
	= \int_{\mathbb{R}^2} dx\, d\theta \; f \,
	\left[
		\partial_\theta \left( \frac{\nu}{2\pi} \, (\partial_x g)^{\mathrm{dr}}_{[\nu]} \right)
		- \partial_x \left( \frac{\nu}{2\pi} \, (\partial_\theta g)^{\mathrm{dr}}_{[\nu]} \right)
	\right].
\end{equation}

\medskip

\subsection{Crochet avec l’Hamiltonien}

\paragraph{Densité hamiltonienne et grandeurs effectives} 
On note $h(x,\theta)$ la densité associée à la moyenne de l’Hamiltonien, telle que
\begin{equation}\label{chap:GHD:eq.ham.1}
	H = \mathcal{Q}[h].
\end{equation}

La fonction d’occupation $\nu$, la vitesse effective $v^{\mathrm{eff}}$ et l’accélération effective $a^{\mathrm{eff}}$ sont définies par
%\begin{equation}\label{chap:GHD:eq.nu.v.a.1}
%	\nu = 2\pi \frac{\rho}{1^{\mathrm{dr}}_{[\nu]}}, 
%	\quad v^{\mathrm{eff}} = \frac{(\partial_\theta h )^{\mathrm{dr}}_{[\nu]}}{1^{\mathrm{dr}}_{[\nu]}}, 
%	\quad a^{\mathrm{eff}} = -\frac{(\partial_x h )^{\mathrm{dr}}_{[\nu]}}{1^{\mathrm{dr}}_{[\nu]}},
%\end{equation}
\begin{equation}\label{chap:GHD:eq.nu.v.a.1}
	2 \pi \rho =  1^{\mathrm{dr}}_{[\nu]} \, \nu , 
	\quad 2 \pi \, v^{\mathrm{eff}} \, \rho  =(\partial_\theta h )^{\mathrm{dr}}_{[\nu]} \, \nu , 
	\quad 2 \pi \, a^{\mathrm{eff}} \, \rho  = -(\partial_x h )^{\mathrm{dr}}_{[\nu]}\, \nu ,
\end{equation}
toutes trois étant des fonctions de $\rho(\cdot,\cdot,t)$. Ces quantités interviennent dans les équations de mouvement
\begin{equation}
	\dot{x} = v^{\mathrm{eff}}, \qquad \dot{\theta} = a^{\mathrm{eff}},
\end{equation}
montrant que les dérivées $\partial_x$ et $\partial_\theta$ présentes dans le crochet de Poisson correspondent respectivement à l'action de l'accélération effective sur $\theta$ et de la vitesse effective sur $x$.

\medskip 

%Avec ces définitions, le crochet~\eqref{chap:GHD:eq.chochet.bonnemain.3} s’écrit
Le crochet \eqref{chap:GHD:eq.chochet.bonnemain.3} appliqué à $(f,h)$ devient :
\begin{equation}\label{chap:GHD:eq.chochet.bonnemain.4}
	\{\mathcal{Q}[f], \mathcal{Q}[h]\} 
	= - \int_{\mathbb{R}^2} dx\, d\theta \; f \left[ \partial_x \big( \rho \, v^{\mathrm{eff}} \big) 
	+ \partial_\theta \big( \rho \, a^{\mathrm{eff}} \big) \right].
\end{equation}

\paragraph{Forme locale : densités conservées .} 
En choisissant $(x,\theta) \mapsto \delta(\cdot - x)f(\theta)$ dans \eqref{chap:GHD:eq.charge.global.1}, on obtient la moyenne de la densité conservée \ie
%On remarque que les moyennes des densités conservées $q_{[f]}(x)$ s’obtiennent en appliquant la prescription
%\[
%(x,\theta) \mapsto \delta(\cdot - x) \, f(\theta)
%\]
%dans~\eqref{chap:GHD:eq.charge.global.1}, \emph{i.e.}
\begin{equation}
	q_{[f]}(x) = \mathcal{Q} \big[ (x,\theta) \mapsto \delta(\cdot - x) \, f(\theta) \big].
\end{equation}

Appliqué à~\eqref{chap:GHD:eq.chochet.bonnemain.4}, on obtient
\begin{equation}\label{chap:GHD:eq.chochet.bonnemain.5}
	\{q_{[f]}(x), \mathcal{Q}[h]\} 
	= - \partial_x \left[ \int_{\mathbb{R}} d\theta \; f \, \rho \, v^{\mathrm{eff}} \right]
	+ \int_{\mathbb{R}} d\theta \; f' \, \rho \, a^{\mathrm{eff}}.
\end{equation}

En appliquant l’équation de Liouville~\eqref{chap:GHD:eq.Liouv.1}, on retrouve la forme de convection~\eqref{chap:GHD:eq.conserv.1} :
\begin{equation}\label{chap:GHD:eq.conserv.2}
	\partial_t q_{[f]} + \partial_x j_{[f]} = F_{[f]},
\end{equation}
où le flux $j_{[f]}$ et le terme de force $F_{[f]}$ sont donnés par
\begin{equation}\label{chap:GHD:eq.conserv.2.1}
	j_{[f]} = \int_{\mathbb{R}} d\theta \; v^{\mathrm{eff}} \, f \, \rho,
	\quad F_{[f]} = \int_{\mathbb{R}} d\theta \; a^{\mathrm{eff}} \, f' \, \rho.
\end{equation}

\paragraph{Forme locale : équation sur \texorpdfstring{$\rho$}{rho}} 
De manière analogue, pour la distribution de rapidité à l’équilibre thermodynamique, on note
\begin{equation}
	\rho(x,\theta) = \mathcal{Q}\big[ \delta(\cdot - x) \, \delta(\cdot - \theta) \big].
\end{equation}
Appliqué à~\eqref{chap:GHD:eq.chochet.bonnemain.4}, on obtient
\begin{equation}\label{chap:GHD:eq.chochet.bonnemain.6}
	\{\rho(x,\theta), \mathcal{Q}[h]\} 
	= - \partial_x \big( v^{\mathrm{eff}} \, \rho \big)
	  - \partial_\theta \big( a^{\mathrm{eff}} \, \rho \big).
\end{equation}

En appliquant l’équation de Liouville~\eqref{chap:GHD:eq.Liouv.1}, on retrouve l’équation GHD~\eqref{chap:GHD:eq.GHD.1} :
\begin{equation}\label{chap:GHD:eq.conserv.3}
	\partial_t \rho + \partial_x \big( v^{\mathrm{eff}} \rho \big)
	+ \partial_\theta \big( a^{\mathrm{eff}} \rho \big) = 0.
\end{equation}

Le résultat remarquable \eqref{chap:GHD:eq.conserv.3} a été obtenu pour la première fois par Bertini et al. (2016) et Castro-Alvaredo et al. (2016). Cette observation clé a déclenché l’ensemble des développements ultérieurs de la dynamique hydrodynamique généralisée (GHD) dans les systèmes quantiques intégrables. Les travaux de Bertini et al. (2016) s’appuient en partie sur ceux de Bonnes et al. (2014), où la formule donnant la vitesse effective \eqref{chap:GHD:eq.nu.v.a.1} était apparue pour la première fois dans le contexte d’un système quantique intégrable.\\

Les équation \eqref{chap:GHD:eq.Liouv.1},  \eqref{chap:GHD:eq.conserv.2} et \eqref{chap:GHD:eq.conserv.3} décrivent la dynamique au régime d’Euler. En dehors de cette approximation, il est nécessaire de prendre en compte des contributions supplémentaires liées aux effets diffusifs \cite{DeNardis2018}.




\section{Cas particuliers et interpretations}


\subsection{Cas sans interaction}

En l’absence d’interaction, l’opérateur de \emph{dressing} se réduit à l’identité.  
Dans ce cas, la fonction d’occupation \eqref{chap:GHD:eq.nu.v.a.1} devient :
\begin{equation}
	\nu = 2\pi \rho,
\end{equation}
et le crochet \eqref{chap:GHD:eq.chochet.bonnemain.1} se simplifie en :
\begin{equation}
	\{F,G\} = \iint dx\,d\theta\;\rho \,\left[\partial_x \!\left( \frac{\delta F}{\delta \rho (x , \theta)} \right)\, \partial_\theta \!\left( \frac{\delta G}{\delta \rho (x , \theta)} \right) - \partial_x \!\left( \frac{\delta G}{\delta \rho (x , \theta)} \right) \, \partial_\theta \!\left( \frac{\delta F}{\delta \rho(x , \theta) } \right) \right].
\end{equation}

Les flux et termes de force \eqref{chap:GHD:eq.conserv.2.1} s’expriment alors en remplaçant la vitesse effective $v^{\mathrm{eff}}$ et l’accélération effective $a^{\mathrm{eff}}$ (de \eqref{chap:GHD:eq.nu.v.a.1}) par leurs expressions issues de la dynamique hamiltonienne libre :
\begin{equation}
	v^{\mathrm{eff}} \to \partial_\theta h, 
	\quad a^{\mathrm{eff}} \to  -\partial_x h.
\end{equation}

Dans le cadre de \eqref{chap:GHD:eq.ham.2} \(\partial_\theta h = \theta\) et   \(\partial_x h  = V' \). De plus en ne considérant que les premières charges conservées associées à $f(\theta) = 1$, $\theta$ et $\theta^2/2$ dans \eqref{chap:GHD:eq.conserv.2} et \eqref{chap:GHD:eq.conserv.2.1}, on retrouve les équations d’Euler classiques :
\begin{eqnarray}\label{chap:3:eq:hydro.1}
	\left\{
	\begin{array}{rcl}
	\partial_t n + \partial_x (n u) &=& 0, \\
	\partial_t (n u) + \partial_x (n u^2 + \mathcal{P}) &=& -n \, \partial_x V(x), \\
	\partial_t E + \partial_x (u(E+\mathcal{P})) &=& 0,
	\end{array}
	\right .  
\end{eqnarray}
avec la densité de particule $n(x, t) = q_{[1]}$, la vitesse moyenne du fluide $u = \frac{q_{[\theta]}}{n}$ , la pression cinétique du fluide $\mathcal{P}(x, t) = \left( q_{[\theta^2]} - \frac{q_{[\theta]}^2}{n} \right)$ , l'énergie totale $E = nu^2/2 + nV + ne$ où $e(x,t)$ est l'énergie interne d'une particule.



\paragraph{Remarques sur les charges globales}
En l’absence de potentiel externe ($V = 0$), le système conserve certaines charges globales. Dans un système classique non intégrable, seules ces quelques charges sont conservées. Par exemple dans un système de Gibbs sont conservé , nombre total de particules , quantité de mouvement totale , énergie cinétique totale soit respectivement $Q[1]  =  \int dx \, q[1]$ , $Q[\theta] = \int dx \, q[\theta$ et $Q\left[\frac{\theta^2}{2}\right] = \int dx \, q\left[\frac{\theta^2}{2}\right] $. 
%\begin{equation}
%	Q[1]  &= & \int dx \, q[1] \, \text{(nombre total de particules)},\\ 
%	Q[\theta] = \int dx \, q[\theta] \, \text{(quantité de mouvement totale)},\\ 
%	Q\left[\frac{\theta^2}{2}\right] = \int dx \, q\left[\frac{\theta^2}{2}\right] \text{(énergie cinétique totale)}.
%\end{equation}

\medskip

En revanche, dans un système intégrable, une infinité de charges sont conservées. En particulier, pour tout $\theta \in \mathbb{R}$:
\(
\rho(\theta , t ) = Q[\delta(\cdot - \theta)] = \int dx \, \rho(x, \theta, t),
\)
et les charges associées à une observable quelconque $f(\theta)$ s’écrivent :
\(
Q[f] = \int d\theta \, f(\theta) \, \rho(\theta, t).
\)

\medskip

Cette structure est l’analogue classique de la description en termes de {\bf distribution de rapidité} dans le cadre intégrable. Elle constitue le point de départ naturel pour développer une description hydrodynamique généralisée (GHD) dans le cas intégrable.


\subsection{la vitesse effectif}

En partant de la définition de la vitesse effectif en \eqref{chap:GHD:eq.nu.v.a.1} et en utilisant la définition de l'appliocation \emph{dressing} \eqref{eq:dessing}, il vient que 
\begin{eqnarray}\label{chap:GHD:veff.1}
	2 \pi \, v^{\mathrm{eff}} \,   \rho   = \nu  \,  \partial_\theta h   + \nu  \,  \left ( \Delta \star ( \rho \,  v^{\mathrm{eff}} )  \right ) ,
\end{eqnarray}
où $\Delta$ désigne le décalage en diffusion défini dans le modèle LL  en Eq.\eqref{eq:I-1-16}
et en soustraiant $v^{\mathrm{eff}} ( \theta ) \nu (\theta) \left ( \Delta \star  \rho  \right )( \theta )$ des deux cotés et on obtiens 

\begin{eqnarray}\label{chap:GHD:veff.2}
	v^{\mathrm{eff}} ( \theta )  \left (  2\pi \, \rho ( \theta ) - \nu (\theta) \left ( \Delta \star  \rho  \right )( \theta ) \right )    = \nu( \theta )  \,  \left (  \partial_\theta h ( \theta )  +  \int d \theta' \, \Delta(\theta - \theta') \rho ( \theta') ( v^{\mathrm{eff}} ( \theta' ) - v^{\mathrm{eff}} ( \theta )  )   \right )  ,
\end{eqnarray}

En partant de la l'écriture de la fonction d'ocumation en \eqref{chap:GHD:eq.nu.v.a.1} et en utilisant la définition de l'appliocation \emph{dressing} \eqref{eq:dessing}, il vient que 
\begin{eqnarray}\label{chap:GHD:veff.3}
	2 \pi \rho - \nu \, 	\Delta \star  \rho = \nu, 
\end{eqnarray}
On obtient 
\begin{eqnarray}\label{chap:GHD:veff.4}
	v^{\mathrm{eff}} ( \theta )      =  \partial_\theta h ( \theta )   +  \int d \theta' \, \Delta(\theta - \theta') \rho ( \theta') ( v^{\mathrm{eff}} ( \theta' ) - v^{\mathrm{eff}} ( \theta )  )   ,
\end{eqnarray}
et dans le modèle de Lieb-Liniger $\partial_\theta h ( \theta )   = \theta$.
Sur le plan physique, le premier terme peut être interprété comme un décalage spatial induit par un processus de diffusion à deux corps. Le second terme quantifie le taux de ces processus de diffusion par unité de temps. L’équation ainsi obtenue correspond à l’équation hydrodynamique généralisée (GHD), formulée pour la première fois en 2016\cite{Bertini2016,CastroAlvaredo2016}.

Le résultat remarquable~\eqref{eq:46} a été obtenu pour la première fois par Bertini \emph{et al.}~\cite{Bertini2016} et Castro-Alvaredo \emph{et al.}~\cite{CastroAlvaredo2016}. Cette observation a constitué le point de départ des développements ultérieurs de l’hydrodynamique généralisée (GHD) dans les systèmes quantiques intégrables. Les travaux de Bertini \emph{et al.}~\cite{Bertini2016} s’appuient en partie sur ceux de Bonnes \emph{et al.}~\cite{Bonnes2014}, où la formule de la vitesse effective~\eqref{eq:47} avait été présentée pour la première fois dans le cadre d’un système quantique intégrable.

Dans le cadre général de l’hydrodynamique généralisée (GHD), l’équation~(48) s’interprète comme une extension naturelle du résultat classique obtenu pour le gaz de tiges dures. La distinction essentielle réside dans le fait que le décalage de diffusion \(\Delta(\theta - \theta_0)\) dépend désormais explicitement de la rapidité relative entre les quasi-particules, alors que, dans le cas du gaz de tiges dures, \(\Delta\) est une constante égale à l’opposé du diamètre des particules.

Sur le plan cinématique, on peut décrire la situation de la manière suivante~: un quasi-particule \emph{traceur} de rapidité \(\theta\) --- c’est-à-dire de moment asymptotique \(\theta\) en l’absence d’interactions --- se déplacerait, dans le vide, à vitesse constante \( \theta \). En présence d’une densité finie \(\rho(\theta_0)\) de quasi-particules de rapidité \(\theta_0\), cette vitesse est modifiée par les processus de diffusion à deux corps.

Pendant un intervalle de temps infinitésimal \([t,\, t + \delta t]\), le traceur subit en moyenne
\[
\delta t \times \left| v_{\mathrm{eff}}[\rho](\theta) - v_{\mathrm{eff}}[\rho](\theta_0) \right| \, \rho(\theta_0)
\]
collisions avec des quasi-particules de rapidité \(\theta_0\). Chaque interaction provoque un décalage spatial \(\Delta(\theta - \theta_0)\) vers l’arrière. L’équation~(48) formalise précisément cet effet cumulatif, résultant de l’intégration des contributions de toutes les collisions binaires sur l’espace des rapidités.

Cette analyse microscopique s’étend naturellement aux modèles à \(N\) corps, où les processus de diffusion se combinent et interagissent de manière non triviale, la fonction \(\Delta(\theta - \theta_0)\) encapsulant alors l’intégralité de la structure intégrable du système.

\subsection{Modèle de Lieb–Liniger}

Les informations relatives aux interactions entre particules sont contenues dans la définition du crochet de Poisson \eqref{chap:GHD:eq.chochet.bonnemain.1}, associée à l’opérateur de \emph{dressing} spécifique au modèle de Lieb–Liniger, défini en \eqref{eq:dessing}.  
L’Hamiltonien $H = \mathcal{Q}[h]$ \eqref{chap:GHD:eq.ham.1} s’écrit ici :
\begin{equation}\label{chap:GHD:eq.ham.2}
	h(x , \theta ) = \varepsilon(\theta) + V(x),
\end{equation}
où l’énergie cinétique est $\varepsilon(\theta) = \theta^2 / 2$ et $V(x)$ représente le potentiel extérieur.

\medskip

Dans ce modèle, la vitesse effective et l’accélération effective de \eqref{chap:GHD:eq.nu.v.a.1} se réécrivent :
\begin{equation}
	v^{\mathrm{eff}} = \frac{(\mathrm{id})^{\mathrm{dr}}_{[\nu]}}{1^{\mathrm{dr}}_{[\nu]}}, 
	\quad a^{\mathrm{eff}} = - V'(x).
\end{equation}

Avec l'equation \eqref{chap:GHD:veff.4} la vitesse effectiff dans le modèle de LL s'écrit 
\begin{equation}
	v^{\mathrm{eff}} = \theta +	\int d \theta' \, \Delta(\theta - \theta') \rho ( \theta') ( v^{\mathrm{eff}} ( \theta' ) - v^{\mathrm{eff}} ( \theta )  ) 
\end{equation}


Ainsi, les termes de force dans \eqref{chap:GHD:eq.conserv.2} et \eqref{chap:GHD:eq.conserv.2.1} prennent la forme :
\begin{equation}
	F_{[f]} = -V'(x) \int_{\mathbb{R}} d\theta \, f'(\theta) \, \rho(x, \theta).
\end{equation}

L’équation GHD \eqref{chap:GHD:eq.conserv.3} devient alors :
\begin{equation}\label{chap:GHD:eq.conserv.3.1}
	\partial_t \rho + \partial_x\!\left(v^{\mathrm{eff}} \rho\right) - V'(x) \, \partial_\theta \rho = 0.
\end{equation}


\subsection{Diagonalisation et invariants de Riemann dans la GHD spatiale étendue}

En dérivant la définition de l'application \emph{dressing} \eqref{eq:dessing}, on obtient la relation suivante :
\begin{equation}\label{chap:GHD:d.dressing}
	\partial_X(f^{\mathrm{dr}}) = \left (\partial_X f \right )^{\mathrm{dr}} + \frac{\Delta}{2\pi} \star ( f^{\mathrm{dr}} \partial_X \nu ), 	
\end{equation}
où les variables \(X = t, x, \theta\).

\medskip

En injectant les définitions \eqref{chap:GHD:eq.nu.v.a.1} dans l'équation GHD \eqref{chap:GHD:eq.conserv.3} puis en appliquant les dérivées à l'application \emph{dressing} conformément à \eqref{chap:GHD:d.dressing}, on obtient :  
\begin{eqnarray}
	\begin{array}{c}\left ( \left(\partial_t 1 \right)^{\mathrm{dr}} + \left(\partial_x  \partial_\theta h  \right)^{\mathrm{dr}} - \left(\partial_\theta  \partial_x h  \right)^{\mathrm{dr}} \right ) \nu + \left ( 1 + \nu \,  \frac{\Delta}{2 \pi } \star  \right ) \left ( 1^{\mathrm{dr}} \partial_t v +  \left ( \partial_\theta h \right )^{\mathrm{dr}} \partial_x \nu -  \left ( \partial_x h \right )^{\mathrm{dr}} \partial_\theta \nu\right ) = 0  \end{array}. 
\end{eqnarray}
Or, on a \(\partial_t 1 = 0\) et \(\partial_x \partial_\theta h = \partial_\theta \partial_x h\). Il en résulte donc l'équation locale de conservation :  
\begin{equation}
	\partial_t \nu + v^{\mathrm{eff}}\partial_x \nu
	+ a^{\mathrm{eff}} \partial_\theta \nu = 0.
\end{equation}  

\medskip

Dans les systèmes hyperboliques, la \emph{diagonalisation} d'une équation consiste à trouver une transformation des variables qui permet de décomposer le système couplé en un ensemble de modes indépendants, appelés \emph{invariants de Riemann} ou \emph{modes normaux}. 

\medskip

Dans le cadre de la GHD spatiale étendue, l'équation d'évolution de la densité \(\rho(x,\theta,t)\) est couplée de manière non triviale en \((x,\theta)\) par la vitesse effective \(v^{\mathrm{eff}}\) et l'accélération effective \(a^{\mathrm{eff}}\). La fonction d'occupation \(\nu(x,\theta,t)\) est définie par une transformation non locale dite \emph{dressing} qui incorpore les interactions du système.

\medskip

Grâce à cela, on peut affirmer que la fonction $\nu (x , \theta)$  s’interprète comme un continuum d’{\bf invariants de Riemann}, c’est-à-dire des variables normales qui restent constantes le long des caractéristiques du système.

\medskip

Cette diagonalisation est essentielle pour comprendre la structure hamiltonienne du système et simplifier l'analyse de sa dynamique, notamment dans le cadre spatialement étendu avec un dressing dépendant de la position. 





\section{Interpretation}

 






%\section{Equation Hydrodynamique Généralisé}
%
%\subsection{Description classique sans interaction}
%Considérons une distribution classique de particules dans l’espace des phases, notée $\varphi(x, p, t)$, représentant la densité de particules autour du point $(x, p)$ à l’instant $t$. En l’absence de phénomènes dissipatifs, cette densité est conservée le long du flot hamiltonien, c’est-à-dire \(
%\frac{d\varphi}{dt} = \frac{\partial \varphi}{\partial t} + \{ \varphi , H \} = 0,
%\)
%où $\{ \cdot , \cdot \}$ désigne le crochet de Poisson canonique :
%\begin{equation}
%\{ \varphi , H \} = \frac{\partial \varphi}{\partial x} \frac{\partial H}{\partial p} - \frac{\partial \varphi}{\partial p} \frac{\partial H}{\partial x}.
%\end{equation}
%Pour $d \varphi /dt = 0 $ , 
%\begin{equation}
%	\frac{\partial \varphi}{\partial t} + \{ \varphi , H \} = 0	
%\end{equation}
%
%
%Ce résultat exprime que la distribution $\varphi$ est constante le long des trajectoires dans l’espace des phases générées par le hamiltonien $H$. Sous cette hypothèse, on peut réécrire l’équation de conservation sous forme différentielle :
%
%\begin{equation}
%\partial_t \varphi + \partial_x ( \dot{x} \varphi ) + \partial_p ( \dot{p} \varphi ) = 0,
%\end{equation}
%
%où les équations du mouvement hamiltonien sont :
%\(
%\dot{x} = \frac{\partial H}{\partial p}, \qquad \dot{p} = -\frac{\partial H}{\partial x}.
%\)
%
%Cette équation prend alors la forme d’une équation de continuité dans l’espace des phases :
%
%\begin{equation}
%\partial_t \varphi + \partial_x j_x + \partial_p j_p = 0,
%\end{equation}
%
%où les densités de courant sont données par :
%\(
%j_x = \dot{x} \varphi, \qquad j_p = \dot{p} \varphi.
%\)
%
%\paragraph{Exemple : particules libres dans un potentiel externe}
%Prenons pour Hamiltonien :
%
%\begin{equation}
%H = \varepsilon(p) + V(x), \qquad \text{où } \varepsilon(p) = \frac{p^2}{2m},
%\end{equation}
%
%correspondant à un système de particules classiques de masse $m$ soumises à un potentiel externe $V(x)$, sans interaction entre particules.
%
%L'équation de conservation s’écrit alors :
%
%\begin{equation}
%\partial_t \varphi + v(p) , \partial_x \varphi - \partial_x V(x) , \partial_p \varphi = 0,
%\end{equation}
%
%où $v(p) = \partial_p \varepsilon(p) = p/m$ est la vitesse du flot hamiltonien dans l’espace des phases.
%
%\paragraph{Charges locales conservées et équations hydrodynamiques}
%On définit une observable locale (ou charge locale) $q[f](x, t)$ associée à une fonction test $f(p)$ par :
%
%\begin{equation}
%q[f](x, t) = \frac{1}{m} \int_{\mathbb{R}} dp \, f(p) \, \varphi(x, p, t).
%\end{equation}
%
%Cette quantité représente la moyenne locale de $f(p)$ pondérée par la distribution $\varphi$. En particulier : la densité de particules : $n(x, t) = q[1]$,l’impulsion moyenne locale : $u(x, t) = \frac{q[p]}{n m}$, la pression cinétique : $\mathcal{P}(x, t) = \frac{1}{m} \left( q[p^2] - \frac{q[p]^2}{q[1]} \right)$.
%
%Les courants associés à ces charges s’écrivent :
%\begin{equation}
%j[f](x, t) = \frac{1}{m} \int dp \, f(p) , \partial_p H(x, p) \, \varphi(x, p, t).
%\end{equation}
%
%En prenant la dérivée temporelle de $q[f]$ et en utilisant l’équation de Liouville, on obtient une équation de conservation de la forme :
%
%\begin{equation}
%\partial_t q[f] + \partial_x j[f] = \frac{1}{m} \int dp \, f(p) , \partial_p \left( \partial_x V(x) \, \varphi \right),
%\end{equation}
%
%qui ne s’annule en général que si $V(x)$ est constant. Toutefois, dans le régime dit hydrodynamique, où $\varphi(x,p,t)$ varie lentement en espace, cette équation devient fermée sur les seules densités $q[f]$, en négligeant les dérivées spatiales d'ordre élevé.
%
%Dans ce cadre, et en ne retenant que les premières charges conservées associées à $f(p) = 1$, $p$, $p^2$, on retrouve les équations d’Euler classiques :
%
%\begin{eqnarray*}
%	\partial_t n + \partial_x (n u) &=& 0, \\
%	\partial_t (m n u) + \partial_x (m n u^2 + \mathcal{P}) &=& -n \, \partial_x V(x), \\
%	\partial_t \mathcal{E} + \partial_x j[\varepsilon(p)] &=& -\partial_x V(x) \cdot q[p],
%\end{eqnarray*}
%
%où $\mathcal{E} = q[\varepsilon(p)]$ est la densité d'énergie, et $j[\varepsilon(p)]$ le courant d'énergie.
%
%\paragraph{Remarques sur les charges globales}
%En l’absence de potentiel externe ($V = 0$), le système conserve certaines charges globales. Dans un système classique non intégrable, seules ces quelques charges sont conservées. Par exemple dans un système de Gibbs sont conservé
%\(
%	Q[1] = \int dx \, q[1] \quad \text{(nombre total de particules)}, 
%	Q[p] = \int dx \, q[p] \quad \text{(quantité de mouvement totale)}, 
%	Q\left[\frac{p^2}{2m}\right] = \int dx \, q\left[\frac{p^2}{2m}\right] \text{(énergie cinétique totale)}.
%\)
%En revanche, dans un système intégrable, une infinité de charges sont conservées. En particulier, pour tout $p \in \mathbb{R}$:
%
%\begin{equation}
%Q[\delta(\cdot - p)] = \frac{1}{m} \int dx \, \varphi(x, p, t),
%\end{equation}
%
%%et les charges associées à une observable quelconque $f(p)$ s’écrivent :
%%
%%\begin{equation}
%%Q[f] = \int dp , f(p) , \rho(p, t).
%%\end{equation}
%%
%%Cette structure est l’analogue classique de la description en termes de "rapidity distribution function" dans le cadre quantique. Elle constitue le point de départ naturel pour développer une description hydrodynamique généralisée (GHD) dans le cas intégrable.
%%
%%
%
%\subsection{Description classique avec interactions}
%
%%Dans la formulation hamiltonienne de la GHD, le champ dynamique est la densité fluide à deux variables  . 
%On définit un crochet de Poisson fonctionnel agissant sur les fonctionnelles $F[\rho]$ et $G[\rho]$  de la distribution de rapidité , avec intéraction. Conformément à Bonnemain et al.\cite{bonnemain2024hamiltonian}  :
%\begin{equation}
%	\{F,G\}\;=\;\iint dx\,d\theta\;\frac{\nu(\theta)}{2\pi}\,\Bigl[\partial_x\frac{\delta F}{\delta \rho(x,\theta)}\,\left(\partial_\theta \left ( \frac{\delta G}{\delta \rho(x,\theta)} \right ) \right)^{\mathrm{dr}} -\partial_x\frac{\delta G}{\delta \rho (x,\theta)}\,\left( \partial_\theta \left ( \frac{\delta F}{\delta \rho (x,\theta)} \right )\right)^{\mathrm{dr}} \Bigr]\,,	
%\end{equation}
%où $\nu$ est la fonction d’occupation.
%%\cite{bonnemain2024hamiltonian,doyon2020lecture}
%
%Pour toute fonction réelle et régulière \( f(x, \theta) \) définie sur \( \mathbb{R}^2 \), on associe le fonctionnel linéaire suivant :
%\begin{equation}
%	Q[f] = \int_{\mathbb{R}^2} dx\, d\theta\, f(x, \theta)\, \rho(x, \theta).
%\end{equation}
%Il s'agit de la charge totale associée à une quantité prenant la valeur \( f(x, \theta) \) pour chaque quasi-particule.  Le crochet de Poisson entre deux charges \( Q[f] \) et \( Q[g]\) s’écrit :
%\begin{equation}
%	\{ Q[f] , Q[g] \} = \int_{\mathbb{R}^2} \frac{dx\, d\theta}{2\pi} \nu  \left( \partial_x f  (\partial_\theta g )^{\mathrm{dr}}  - \partial_x g (\partial_\theta f)^{\mathrm{dr}}  \right),
%\end{equation}
%or l'application dressing satisfait la relation de symétrie \cite{doyon2020lecture}:
%\begin{equation}
%	\int_{\mathbb{R}^2}	 dx\, d\theta \, \nu f g^{\mathrm{dr}} = \int_{\mathbb{R}^2}	 dx\, d\theta \, \nu f^{\mathrm{dr}} g,
%\end{equation}
%soit avec une integration part partie, on réécrit le crochet 
%\begin{equation}
%	\{ Q[f] , Q[g]\} = \int_{\mathbb{R}^2} \frac{dx\, d\theta}{2\pi} f  \left( \partial_\theta ( \nu   (\partial_x g )^{\mathrm{dr}} )   - \partial_x ( \nu   (\partial_\theta g )^{\mathrm{dr}} )  \right).
%\end{equation}
%
%La distribution de rapidité $\rho( x , \theta )  = Q[\delta( \cdot - x )\delta( \cdot - \theta  )  ]$ et pour un hamiltinien $H = Q[h]$ avec $h(x , \theta ) = \varepsilon(\theta) + V(x)$ avec $\varepsilon(\theta) = m \theta^2/2$.
%
%\begin{equation}
%	\{ \rho(x, \theta), Q[h] \} + \partial_x (v^{\mathrm{eff}} \rho) + \partial_\theta (a^{\mathrm{eff}} \rho) = 0.
%\end{equation}
%
%Nous avons ici utilisé les identités (2.29), ainsi que la définition de la fonction d’occupation (rappelée pour commodité) :
%
%\begin{equation}
%	v^{\mathrm{eff}} = \frac{\varepsilon'^{\mathrm{dr}}}{1^{\mathrm{dr}}}, 
%	\quad 
%	a^{\mathrm{eff}} = -V, 
%	\quad 
%	\nu = \frac{\rho}{\rho_s}.
%\end{equation}
%
%Ainsi, en posant \( \partial_t \rho(x, \theta) = \{ \rho(x, \theta), Q[h] \} \), on retrouve bien les équations de la GHD sous forme hamiltonienne étendue à l’espace :
%
%\begin{equation}
%	\partial_t \rho(x, \theta) = -\partial_x (v^{\mathrm{eff}} \rho) - \partial_\theta (a^{\mathrm{eff}} \rho).
%\end{equation}
%
%











%\input{chapters/97_GHD}
\input{chapters/04_GGE_Fluctuation}
\chapter{Dispositif expérimental et méthodes d’analyse}
\label{chap:disp.exp}
\minitoc

%\section{Présentation de l’expérience}
%\section*{Introduction}
%
%\section{Refroidissement}
%
%\section{Imagerie}
%\subsection{Prubleme d'iamgerie et idée numerique}
%
%\section{Confinement transverse}
%
%\section{Confinement longitudinale}
%
%\subsection{Evolution logitudinale}
%
%\section{Outil de sélection spatial}
%
%\subsection{Mesure de distribution de rapidités locales $\rho(x , \theta ) $  pour des systèmes en équilibre}
%
%%\subsection{Piégeage transverses et longitudinale}
%%\section{Outil de sélection spatial}
%%%\section{Mesure de $\rho(x , \theta ) $ }
%
%%\section{Mesure de distribution de rapidités locales $\rho(x , \theta ) $  pour des systèmes en équilibre}

\section*{Introduction}

%\begin{itemize}
%	\item Objectif du chapitre : présentation synthétique de l’expérience
%	\item Distinction claire des contributions : mise en place initiale (précédents doctorants), développement (travail de Léa Dubois), contribution personnelle (prise de données, analyses spécifiques, participation à certaines manipulations)
%	\item Rôle de l’expérience dans l’étude de la dynamique des gaz de Bose 1D
%\end{itemize}

Ce chapitre présente l’expérience utilisée pour étudier les gaz unidimensionnels de rubidium ultra-froids. Nous décrivons l’architecture du dispositif, les méthodes d’imagerie et d’analyse, ainsi que les protocoles expérimentaux auxquels j’ai participé. Le développement initial du refroidissement et du piégeage avant la puce a été réalisé par d’anciens doctorants. La mise en place du piégeage sur la puce et du système de sélection spatiale à l’aide d’un DMD a été initiée par Léa Dubois, alors en première année de doctorat à mon arrivée. Mon travail s’est concentré principalement sur la prise de données, l’analyse et la participation à certaines expériences spécifiques telles que l’expansion longitudinale et la mesure locale de la distribution de rapidité.


\paragraph{Objectif du chapitre}  
Ce chapitre a pour objectif de fournir une présentation synthétique et structurée du dispositif expérimental utilisé pour étudier la dynamique de gaz de Bose unidimensionnels ultra-froids. Il constitue un socle indispensable pour comprendre les protocoles expérimentaux développés au cours de ma thèse et les analyses présentées dans les chapitres suivants.

\paragraph{Architecture générale}  
Nous présentons d'abord l’architecture complète de l’expérience, depuis la production des atomes jusqu’à leur imagerie, en passant par les étapes de refroidissement, de piégeage magnétique sur puce, de manipulation optique, et de génération de potentiels. Cette description s’accompagne d’une mise en contexte des contributions historiques au dispositif.

\paragraph{Contributions successives et personnelles}  
Une attention particulière est portée à la répartition chronologique des contributions. Les étapes initiales (source atomique, MOT, piège DC) ont été développées par d’anciens doctorants. La mise en place du piégeage 1D sur puce ainsi que l’utilisation du DMD pour la sélection spatiale ont été réalisées au cours de la thèse de Léa Dubois. Mon travail s’inscrit dans cette continuité et concerne principalement la prise de données, l’analyse de protocoles dynamiques, ainsi que la participation à certaines opérations de maintenance et d’optimisation du système.

\paragraph{Rôle du dispositif dans la thèse}  
Ce dispositif permet d’explorer des phénomènes hors équilibre dans des gaz quantiques 1D. Il constitue une plateforme particulièrement adaptée à l’étude de protocoles d’expansion, de sondes locales, ou de dynamiques guidées par la théorie hydrodynamique généralisée (GHD), qui sont au cœur de cette thèse.




%\section{Présentation générale de l’expérience}
%\subsection{Vue d’ensemble du dispositif}
%\begin{itemize}
%    \item Architecture générale : production, piégeage, manipulation et imagerie.
%    \item Systèmes étudiés : gaz de rubidium 87 dans des pièges 1D.
%    \item Objectifs : exploration de dynamiques hors équilibre.
%\end{itemize}
%
%\subsection{Historique et contributions successives}
%\begin{itemize}
%    \item Étapes de refroidissement et piégeage initial : travaux antérieurs (voir thèses citées).
%    \item Développement du piégeage 1D sur puce et du DMD : thèse de Léa Dubois.
%    \item Contributions personnelles : prise de données, protocoles dynamiques, analyse.
%\end{itemize}

\section{Le dispositif expérimental}
\subsection{Système laser et contrôle de fréquence}
\label{sec:systeme_laser}

%\paragraph{Laser maître 1 : référence de fréquence}
%La référence principale de fréquence pour l'ensemble des faisceaux utilisés dans l'expérience est fournie par un laser à cavité étendue, développé au SYRTE. Ce laser est asservi par spectroscopie d’absorption saturée sur la transition D2 du $^{87}$Rb, au croisement des transitions $|F=2\rangle \rightarrow |F'=2,3\rangle$. Ce signal de référence est utilisé pour verrouiller les autres sources laser par battement optique.

\paragraph{Laser maître 1 : référence de fréquence}
La stabilité en fréquence de l’ensemble des faisceaux employés dans l’expérience est assurée par un laser à cavité étendue conçu au SYRTE. Ce laser est verrouillé par spectroscopie d’absorption saturée sur la raie D2 du $^{87}$Rb, en ciblant le croisement des transitions $|F=2\rangle \rightarrow |F'=2,3\rangle$. Ce verrouillage fournit la référence absolue de fréquence à partir de laquelle les autres sources laser sont synchronisées par battement optique.

%\paragraph{Laser repompeur}
%Un laser DFB (Distributed Feedback Diode) est utilisé pour produire le faisceau repompeur, permettant de transférer les atomes retombés dans l’état $|F=1\rangle$ vers l’état $|F=2\rangle$. Ce laser est asservi à une fréquence distante de 6\,GHz de celle du maître 1, en utilisant un montage de battement optique et mélange avec un oscillateur à 6.6\,GHz. Une diode Fabry-Perot injectée par la DFB permet d’amplifier la puissance au-delà de 100\,mW.
%
%\paragraph{Laser repompeur}
%Le faisceau de repompage, qui permet de transférer les atomes piégés dans l’état $|F=1\rangle$ vers l’état $|F=2\rangle$, est généré par une diode DFB (Distributed Feedback). Sa fréquence est décalée de 6,GHz par rapport au maître 1 grâce à un système de battement optique combiné à un mélange avec un oscillateur micro-onde à 6.6,GHz. Une diode Fabry–Perot, injectée par la DFB, permet d’augmenter la puissance de sortie au-delà de 100,mW.

\paragraph{Laser repompeur}
Le faisceau de repompage, qui transfère les atomes tombé  dans l’état $|F=1\rangle$ vers l’état $|F=2\rangle$, est produit par une diode DFB (Distributed Feedback). Sa fréquence est décalée de 6 GHz par rapport au maître 1 par battement optique et mélange avec un oscillateur à micro-ondes de 6.6 GHz. Une diode Fabry–Perot, injectée par la DFB, élève la puissance de sortie au-delà de 100 mW.

%\paragraph{Laser maître 2 : laser principal de manipulation}
%Un second laser à cavité étendue, identique au maître 1, est asservi par battement optique à la fréquence du maître 1. Il est amplifié par un amplificateur à semi-conducteur évasé (Tapered Amplifier), permettant d’atteindre une puissance de sortie supérieure à 1\,W. Ce faisceau est ensuite divisé en plusieurs branches pour alimenter :
%\begin{itemize}
%    \item le Piège Magnéto-Optique (PMO),
%    \item la mélasse optique,
%    \item le pompage optique,
%    \item l’imagerie par absorption,
%    \item le faisceau de sélection.
%\end{itemize}

\paragraph{Laser maître 2 : source principale de manipulation}
Un second laser à cavité étendue, est verrouillé par battement optique sur la fréquence du maître 1. L’émission est amplifiée au moyen d’un amplificateur à semi-conducteur évasé (Tapered Amplifier), fournissant plus de 1\,W en sortie. Le faisceau ainsi produit est distribué vers différentes parties de l’installation expérimentale : alimentation du piège magnéto-optique (PMO), formation de la mélasse optique, réalisation du pompage optique, imagerie par absorption,génération du faisceau de sélection.


%\paragraph{Contrôle de fréquence et polarisation}
%Les fréquences des différents faisceaux sont ajustées via des Modulateurs Acousto-Optiques (AOM), tandis que leur polarisation et leur intensité sont contrôlées à l’aide de cubes PBS en combinaison avec des lames demi-onde motorisées ou fixes. Cette configuration assure une grande flexibilité dans la mise en œuvre des différentes phases expérimentales.

\paragraph{Gestion des fréquences et polarisations}
%Les ajustements de fréquence des divers faisceaux sont réalisés à l’aide de modulateurs acousto-optiques (AOM).
Les faisceaux peuvent être interrompus soit à l’aide d’obturateurs mécaniques, soit via des modulateurs acousto-optiques (AOM). Ces derniers offrent un temps de commutation beaucoup plus court que les systèmes mécaniques, car ils permettent de sélectionner uniquement un ordre de diffraction non nul et d’éteindre instantanément le faisceau en interrompant l’alimentation radiofréquence. L’intensité et la polarisation sont réglées via des cubes séparateurs PBS associés à des lames demi-onde, fixes ou motorisées. Ce dispositif offre une grande souplesse pour adapter la configuration optique aux différentes étapes de l’expérience.

%\paragraph{Remarque}
%Une description plus détaillée du montage laser et de son verrouillage peut être trouvée dans la thèse de A.~Johnson~\cite{Johnson2016}. L’ensemble a été maintenu et utilisé sans modifications majeures au cours de ma thèse.

\paragraph{Note}
Une présentation plus exhaustive du montage laser et de son système de verrouillage est disponible dans la thèse de A.Johnson\cite{Johnson2016}. Le dispositif a été conservé dans son architecture d’origine tout au long de mes travaux, avec seulement un entretien régulier.


\subsection{Production et refroidissement des atomes (non détaillé ici, renvoi à d'autres travaux)}
{\color{blue}
\begin{itemize}
    \item Source chaude de rubidium, MOT, molasses optique.
    \item Refroidissement à des températures sub-$\mu~K$ Refroidissement sub-Doppler (détails renvoyés aux travaux précédents).
\end{itemize}
}
%Le dispositif expérimental permet de produire des gaz de rubidium ultra-froids, avec pour objectif final l’obtention de gaz unidimensionnels dans le régime quantique dégénéré. La production suit une séquence expérimentale déjà bien établie, initialement développée par d’anciens doctorants (voir par exemple la thèse d’A. Johnson~\cite{Johnson2016}), puis réoptimisée au début de la thèse de Léa-Dubois ~\cite{L.Dubois2024} sous la supervision d’I. Bouchoule.

Le dispositif expérimental permet de produire des gaz ultra-froids de rubidium, en vue d’obtenir des gaz unidimensionnels dans le régime quantique dégénéré. La séquence expérimentale suit un protocole établi, initialement développé par d’anciens doctorants (voir par exemple la thèse d’A. Johnson~\cite{Johnson2016}) et réoptimisé au début de la thèse de Léa. Dubois~\cite{L.Dubois2024} sous la supervision d’I. Bouchoule.

%\paragraph{Libération des atomes de rubidium}
%Les atomes de $^{87}$Rb sont libérés à partir d’un \emph{dispenser}, placé directement dans l’enceinte à vide, sur le côté de la monture de la puce atomique. Ce composant, parcouru par un courant de \( 4.5\,\mathrm{A} \) pendant environ \( 5\,\mathrm{s} \), émet un flux d’atomes thermiques dans la chambre à vide.

\paragraph{Libération des atomes de rubidium}
Les atomes de $^{87}$Rb sont émis à partir d’un \emph{dispenser} placé directement dans l’enceinte à vide, à proximité de la monture de la puce atomique. Un courant de \( 4.5\,\mathrm{A} \)  est appliqué pendant environ \( 5\,\mathrm{s} \), générant un flux d’atomes thermiques dans la chambre à vide.

%
%\paragraph{Capture par piège magnéto-optique (PMO)}
%Les atomes thermiques sont ralentis et piégés à l’aide d’un piège magnéto-optique. Celui-ci utilise quatre faisceaux laser (dont deux sont réfléchis par la puce) et un champ quadrupolaire magnétique généré par des bobines. Le nuage ainsi formé se situe à quelques millimètres de la surface de la puce.

\paragraph{Capture par le piège magnéto-optique (PMO)}
Les atomes thermiques sont ralentis et confinés dans un piège magnéto-optique. Quatre faisceaux laser (dont deux réfléchis par la puce) combinés à un champ quadrupolaire magnétique produit par des bobines permettent de former un nuage atomique situé à quelques millimètres de la surface de la puce.

%\paragraph{Rapprochement vers la puce}
%Pour rapprocher les atomes de la puce, on transfère le champ quadrupolaire depuis les bobines vers un champ généré par le fil en forme de U de la puce (fil bleu dans la Fig.~\ref{fig:puce}). Ce fil est parcouru par un courant variant de \( 3.6\,\mathrm{A} \) à \( 1.5\,\mathrm{A} \), ce qui rapproche le nuage à quelques centaines de micromètres de la surface.

\paragraph{Rapprochement vers la puce}
Le nuage est rapproché de la surface de la puce en transférant le champ quadrupolaire depuis les bobines vers le champ produit par le fil en forme de U de la puce (fil bleu, Fig.~\ref{fig:puce}). Le courant dans ce fil est ajusté lentement de \( 3.6\,\mathrm{A} \) à \( 1.5\,\mathrm{A} \), ce qui positionne le nuage à quelques centaines de micromètres de la surface.

%\paragraph{Mélasse optique}
%Une phase de mélasse optique permet un refroidissement sub-Doppler des atomes capturés. Un système d’imagerie provisoire est utilisé à cette étape pour visualiser le nuage atomique, dont la taille dépasse le champ d’observation du système d’imagerie final.

%\paragraph{Mélasse optique}
%Une étape de mélasse optique est ensuite appliquée pour atteindre un refroidissement sub-Doppler des atomes capturés. %Un système d’imagerie provisoire permet de visualiser le nuage, dont la taille dépasse le champ d’observation du dispositif final.

%\paragraph{Pompage optique}
%Afin de polariser les atomes dans l’état magnétique \( |F=2,\,m_F=2\rangle \), un pompage optique est effectué avec un faisceau circulairement polarisé \( \sigma^+ \), résonant sur la transition \( |F=2\rangle \rightarrow |F'=2\rangle \).

\paragraph{Pompage optique}
Enfin, les atomes sont préparés dans l’état magnétique \( |F=2,\,m_F=2\rangle \) par pompage optique. Un faisceau circulairement polarisé \( \sigma^+ \), résonant sur la transition \( |F=2\rangle \rightarrow |F'=2\rangle \), assure la polarisation du nuage.

\paragraph{Mélasse optique}
Après la capture dans le PMO, une étape de mélasse optique est appliquée pour refroidir davantage le nuage atomique, au-delà de la limite de Doppler. La mélasse optique repose sur l’utilisation de faisceaux laser légèrement désaccordés en fréquence et polarisés de manière appropriée, qui interagissent avec les atomes selon le mécanisme de refroidissement sub-Doppler.

Le principe physique est le suivant : les atomes en mouvement voient les faisceaux laser avec un décalage Doppler, ce qui modifie la probabilité d’absorption selon leur vitesse et leur position. Combiné avec les effets de polarisation (notamment les forces de type Sisyphus dans un champ de polarisation variable), cela crée un potentiel de friction optique qui ralentit les atomes. Contrairement au refroidissement Doppler standard, la mélasse optique permet de réduire l’énergie cinétique des atomes en dessous de la limite Doppler, atteignant des températures beaucoup plus basses.

Ainsi, cette étape permet d’obtenir un nuage plus dense et plus froid, condition essentielle pour les manipulations ultérieures et la formation de gaz unidimensionnels dans le régime quantique dégénéré.






\subsection{Piégeage magnétique sur puce}
%{\color{blue}
%\begin{itemize}
%    \item Présentation de la puce atomique.
%    \item Confinement transverse et longitudinal.
%    \item Régime 1D : conditions d’accès (\(\hbar \omega_\perp \gg k_B T\)).
%    \item Problèmes de rugosité, stabilité magnétique.
%\end{itemize}
%}

\subsubsection{Piégeage magnétique sur puce}
\label{sec:piegeage_puce}

%On peut utiliser des piégeage optique pour produire des stracture atomique longitudinale alongé. Certaines groupe de recherche utilise un redeau optique 2D pour obtenir un réseau 2D de tube longitudinaaux \cite{Kinoshita2004,LaburtheTolra2004,Paredes2004,Moritz2003}. Ce raseaux 2D produit un grand nombre de systéme atomique propise è l'étude de de gase 1D. Avec ce genre de dispositif on peux etudier des gas 1D peut dense car les densité peut etre moyenné sur tous les tudes. Mais avec ce genre de dispositif on ne peut pas étudier experimentalement les fluctudation dans le systéme. Nous pour gièger les atomes on utilise une puce atomique.
%
%\paragraph{Principe général}
%Les atomes de rubidium sont piégés grâce à une puce atomique intégrée dans l’enceinte à vide. Une puce atomique est un circuit microfabriqué contenant des micro-fils dans lesquels circulent des courants permettant de générer des champs magnétiques à géométrie contrôlée. Ce dispositif, développé dans les années 1990 \cite{Denschlag1999,Fortagh1998}, permet une miniaturisation du système de piégeage \cite{Folman2000,Reichel1999}, les premiers condensats sur puce ont été obtenus en 2001 \cite{Haensel2001,Ott2001} et la premièref fois aux laboratoir Charles Fabry (LCF) \cite{Aussibal2003} et un accès à des confinements forts, particulièrement adaptés à l'étude de gaz de Bose unidimensionnels \cite{Schumm2005,Trebbia2006}.

-------

On peut créer des structures atomiques allongées en utilisant des techniques de piégeage optique. Par exemple, plusieurs groupes de recherche ont recours à des réseaux optiques bidimensionnels (2D) pour former un ensemble de tubes atomiques longitudinaux \cite{Kinoshita2004,LaburtheTolra2004,Paredes2004,Moritz2003}. Ces réseaux 2D permettent de produire un grand nombre de systèmes atomiques quasi-unidimensionnels, offrant ainsi une plateforme idéale pour l’étude des gaz 1D. Ce type de dispositif est particulièrement adapté à l’étude de gaz faiblement denses, car les densités peuvent être moyennées sur l’ensemble des tubes. Cependant, l’étude expérimentale des fluctuations locales dans chaque tube reste difficile avec ce genre de configuration. Pour surmonter cette limitation, on utilise le piégeage à l’aide de puces atomiques.

\paragraph{Principe général}
Les atomes de rubidium sont confinés par une puce atomique intégrée dans l’enceinte à vide. Une puce atomique est un circuit microfabriqué comportant de fins micro-fils parcourus par des courants électriques, ce qui permet de générer des champs magnétiques à géométrie contrôlée. Cette technologie, développée dans les années 1990 \cite{Denschlag1999,Fortagh1998}, offre une miniaturisation significative des dispositifs de piégeage \cite{Folman2000,Reichel1999}. Les premiers condensats de Bose–Einstein sur puce ont été réalisés en 2001 \cite{Haensel2001,Ott2001}, puis ultérieurement au Laboratoire Charles Fabry \cite{Aussibal2003}. Les puces atomiques permettent d’accéder à des confinements très forts, particulièrement adaptés à l’étude des gaz de Bose unidimensionnels et à l’exploration de leurs propriétés quantiques locales \cite{Schumm2005,Trebbia2006}.


-----
Des structures atomiques allongées peuvent être réalisées par piégeage optique. Dans ce cadre, des réseaux optiques bidimensionnels (2D) permettent de créer un ensemble de tubes atomiques quasi-unidimensionnels \cite{Kinoshita2004,LaburtheTolra2004,Paredes2004,Moritz2003}. Ces réseaux offrent un grand nombre de systèmes atomiques identiques, facilitant l’étude statistique de gaz 1D faiblement dense. Toutefois, l’accès expérimental aux fluctuations locales dans chaque tube reste limité.

Pour contourner cette contrainte, les puces atomiques offrent une solution efficace. Ces dispositifs microfabriqués intègrent de fins micro-fils parcourus par des courants, générant des champs magnétiques de géométrie contrôlée et permettant des confinements très forts \cite{Denschlag1999,Fortagh1998,Folman2000,Reichel1999}. La miniaturisation ainsi obtenue a permis l’obtention des premiers condensats de Bose–Einstein sur puce dès 2001 \cite{Haensel2001,Ott2001}, et dés 2003 au Laboratoire Charles Fabry \cite{Aussibal2003}. Grâce à ces confinements, il devient possible d’étudier expérimentalement les propriétés de gaz de Bose unidimensionnels et leurs fluctuations locales \cite{Schumm2005,Trebbia2006}.

-------

\paragraph{Structure de la puce utilisée}
La puce utilisée au cours de cette expérience a été conçue en collaboration avec S.~Bouchoule, A.~Durnez et A.~Harouri (C2N). Elle repose sur un substrat de carbure de silicium sur lequel est déposé le circuit électrique. Ce dernier est recouvert d’une couche de résine BCB, aplanie par des cycles d’enduction et d’attaque plasma. Une fine couche d’or (\(\sim200\,\mathrm{nm}\)) est finalement évaporée afin de permettre l’utilisation de la puce comme miroir pour l’imagerie à \(780\,\mathrm{nm}\). La puce est soudée à l’indium sur une monture en cuivre inclinée à \(45^\circ\) par rapport à l’axe optique.

%\paragraph{Fils de piégeage et géométrie des champs}
%Plusieurs fils sont intégrés à la puce pour assurer les différentes étapes du piégeage et du transport des atomes : un fil en forme de Z est utilisé pour le piégeage initial (DC), tandis que trois micro-fils (symétriques et parallèles) sont utilisés pour former un guide unidimensionnel par courants alternatifs (AC). La géométrie des fils a été optimisée pour minimiser la dissipation de chaleur, limiter les couplages parasites et améliorer la symétrie du piège. Dans la zone d’intérêt, les atomes sont piégés à environ \(15\,\mu\mathrm{m}\) au-dessus des fils, soit à \(8\,\mu\mathrm{m}\) au-dessus de la surface de la puce.

%\paragraph{Fils de piégeage et géométrie des champs}
%La puce atomique comporte plusieurs ensembles de fils, chacun jouant un rôle précis dans les différentes étapes de la capture, du transport et du confinement des atomes.
\paragraph{Fils de piégeage et géométrie des champs}
La puce atomique intègre plusieurs ensembles de conducteurs, chacun conçu pour une étape spécifique de la capture, du transport et du confinement des atomes. L’ensemble de la séquence de transfert, depuis le piège magnéto-optique (PMO) jusqu’au guide unidimensionnel, repose sur une succession de configurations magnétiques générées par ces différents fils.

%\medskip
%\subparagraph{Fil en forme de U .}
%Après la phase de pré-refroidissement, le nuage est initialement capturé dans un piège magnéto-optique (PMO) situé au-dessus de la puce. Il est ensuite approché de la surface en transférant progressivement le champ quadrupolaire des bobines externes vers celui produit par un fil en forme de U intégré à la puce (phase \textit{U} : transfert du PMO vers la puce + mélace optique + ponpage optique). 

\subparagraph{Phase U : approche de la surface}
Après la phase de pré-refroidissement, le nuage est initialement capturé dans un PMO situé au-dessus de la puce. Il est ensuite rapproché de la surface en transférant progressivement le champ quadrupolaire des bobines externes vers celui produit par un fil en forme de U intégré à la puce (fils bleus dans la Fig.~\ref{fig:puce}). Cette étape (\textit{phase U}) est accompagnée d’un mélange optique et d’un pompage optique afin de préparer les atomes pour le piégeage magnétique.


%\medskip
%\subparagraph{Fil en forme de Z : Chargement dans le piège DC .}
%Après le pompage optique, les atomes sont transférés dans un piège magnétique combinant un courant continu circulant dans le fil en forme de Z de la puce (fil orange dans la Fig.~\ref{fig:puce}) et un champ magnétique externe. Ce piège, noté \emph{piège DC}, permet un confinement transverse important. Un refroidissement par évaporation radiofréquence est alors réalisé pendant environ \( 2.3\,\mathrm{s} \), ce qui abaisse la température du nuage à environ \( 1\,\mu\mathrm{K} \), pour un nombre d’atomes typiquement autour de \( 2.5 \times 10^5 \).

\subparagraph{Phase Z : piège DC et refroidissement}
À l’issue du pompage optique, les atomes sont transférés dans un piège magnétique combinant un courant continu circulant dans un fil en forme de Z (fil orange) et un champ magnétique externe. Ce \emph{piège DC} assure un confinement transverse fort. Un refroidissement par évaporation radiofréquence, d’une durée d’environ \(2.3\,\mathrm{s}\), abaisse la température du nuage à environ \(1\,\mu\mathrm{K}\), pour un nombre typique d’atomes de l’ordre de \(2.5\times 10^5\).
%\medskip
%Une fois chargé dans ce piège intermédiaire, le nuage est transporté vers la zone expérimentale. Dans cette région, trois micro-fils parallèles et symétriques (jaune), parcourus par des courants alternatifs (AC), créent un guide magnétique unidimensionnel assurant le confinement transversal des atomes. Le confinement longitudinal est obtenu grâce à deux paires de fils : d/d′ (rose) et D/D′ (vert).

\subparagraph{Transfert vers le guide unidimensionnel}
Une fois refroidi, le nuage est acheminé vers la zone expérimentale où trois micro-fils parallèles et symétriques (fils jaunes) parcourus par des courants alternatifs (AC) génèrent un guide magnétique unidimensionnel assurant le confinement transverse. Le confinement longitudinal est fourni par deux paires de fils : $d/d'$ (rose) et $D/D'$ (vert).

Le passage du piège DC au guide 1D est réalisé de manière adiabatique grâce à cinq rampes linéaires de courant d’une durée comprise entre \(50\) et \(60\,\mathrm{ms}\) chacune. Durant cette opération :  
(i) le courant dans le fil Z est progressivement réduit,  
(ii) le courant dans les micro-fils du guide est augmenté jusqu’à environ \(50\,\mathrm{mA}\),  
(iii) un courant initial de \(0.5\,\mathrm{A}\) est appliqué dans les fils $D$ et $D'$, puis ajusté pour maintenir fixe la position du centre de masse du nuage.  

Ce protocole minimise les oscillations résiduelles dans le guide et assure un découplage efficace entre la dynamique longitudinale et le confinement transverse. Ce dispositif a été développé au cours de la thèse de Léa Dubois~\cite{TheseLea} et a été utilisé dans le cadre de mes protocoles expérimentaux sur l’expansion longitudinale et les sondes locales de distribution de rapidité.

\subparagraph{Optimisation géométrique}
La géométrie des conducteurs de la puce a été conçue pour réduire la dissipation thermique, limiter les couplages parasites et garantir une bonne symétrie des champs magnétiques. Dans la zone expérimentale, les atomes sont piégés à environ \(15\,\mu\mathrm{m}\) au-dessus des fils, soit \(8\,\mu\mathrm{m}\) au-dessus de la surface de la puce.


\paragraph{Refroidissement final et accès au régime unidimensionnel}
Une dernière phase de refroidissement par évaporation radiofréquence est effectuée directement dans le guide AC. Grâce à l’anisotropie marquée du piège, le confinement transverse atteint une fréquence \(\omega_\perp\) telle que l’énergie quantique \(\hbar \omega_\perp\) dépasse largement les énergies thermique et chimique du système. On atteint ainsi le régime unidimensionnel, caractérisé par la hiérarchie d’énergies :
\[
k_B T, \mu \ll \hbar \omega_\perp,
\]
où \(\mu\) désigne le potentiel chimique et \(T\) la température du gaz.

Dans ce régime, le confinement transverse est assuré principalement par la géométrie des micro-fils et la présence de champs magnétiques externes, tandis que le confinement longitudinal, plus faible, est ajustable via une combinaison de champs magnétiques externes et de courants circulant dans des fils additionnels ($d/d'$ et $D/D'$). 

Les gaz obtenus contiennent typiquement entre \(3\times 10^3\) et \(1.5\times 10^4\) atomes, pour des températures de l’ordre de \(50\) à \(200\,\mathrm{nK}\). La Fig.~\ref{fig:gaz1D} illustre un exemple de nuage dans ce régime, observé avec le système d’imagerie final.



%\paragraph{Confinement transverse et longitudinal}
%Le confinement transverse est assuré principalement par la géométrie des fils et la présence de champs magnétiques externes. Sa fréquence élevée permet d’atteindre des énergies de confinement \(\hbar \omega_\perp\) bien supérieures aux énergies thermiques et chimiques du système, condition nécessaire à l’accès au régime 1D :
%\[
%k_B T, \mu \ll \hbar \omega_\perp.
%\]
%Le confinement longitudinal, plus faible, est modulable par combinaison de champs magnétiques externes et courants dans les fils additionnels.

\paragraph{Avantages du piégeage sur puce}
Comparé aux systèmes utilisant des réseaux optiques 2D, le piégeage sur puce ne fournit qu’un seul tube, ce qui permet un meilleur accès aux fluctuations locales de densité et aux observables résolues spatialement. Ce type de dispositif est ainsi particulièrement adapté à l'étude de la thermodynamique et de la dynamique de gaz 1D isolés.

\paragraph{Limitations et effets parasites}
Parmi les limitations spécifiques au piégeage sur puce figurent la rugosité des potentiels magnétiques due aux imperfections des fils, qui peut induire des modulations parasites du confinement longitudinal. De plus, la stabilité du dispositif est sensible aux champs parasites magnétiques externes ainsi qu’aux échauffements dus aux courants continus.





\paragraph{Imagerie finale}
À l’issue de ce refroidissement, les atomes sont observés avec le système d’imagerie final (voir Fig.~\ref{fig:imagerieFinale}), adapté aux tailles caractéristiques du gaz dans le piège. Une image typique de ce nuage est présentée en Fig.~\ref{fig:nuageDC}.



%\paragraph{Refroidissement final et accès au régime unidimensionnel}
%Une dernière phase de refroidissement par évaporation radiofréquence est ensuite réalisée dans le guide AC. Ce refroidissement, mené dans le piège à forte anisotropie, permet d’atteindre le régime unidimensionnel, caractérisé par la hiérarchie d’énergies :
%\[
%k_B T, \mu \ll \hbar \omega_\perp
%\]
%où \( \omega_\perp \) est la fréquence de confinement transverse, \( \mu \) le potentiel chimique et \( T \) la température du gaz.
%
%Les gaz obtenus contiennent typiquement entre \( 3 \times 10^3 \) et \( 1.5 \times 10^4 \) atomes, pour des températures de l’ordre de \( 50 \text{ à } 200\,\mathrm{nK} \). La Fig.~\ref{fig:gaz1D} montre un exemple de tel gaz observé avec le système d’imagerie final.


\paragraph{Remarques expérimentales}
Lorsque j’ai rejoint l’équipe, la première année thèse de Léa Dubois touchait à sa fin et le dispositif expérimental était en fonctionnement stable. Les différentes étapes du cycle (dispenser, PMO, mélasse, pompage optique, piège DC, transfert vers le guide, évaporation finale) avaient été mises en place et optimisées pendant les premières années de sa thèse, sous la supervision d’I. Bouchoule.Le cycle expérimental complet dure environ 15 secondes. Une description plus détaillée peut être trouvée dans la thèse d’A. Johnson~\cite{Johnson2016}.


Pendant ma première année, j’ai principalement participé à la prise de données en collaboration avec Léa. Grâce à la qualité de son travail, le dispositif était globalement très fiable, ce qui a permis de mener des campagnes expérimentales riches sans intervention lourde. Néanmoins, cette stabilité avait pour contrepartie que je n’ai pas été directement impliqué dans la résolution des pannes complexes ou dans le reconditionnement complet de la manipulation, ce qui a limité ma formation sur les aspects de maintenance approfondie du dispositif.

En revanche, peu avant la fin de la thèse de Léa et au début de ma troisième année, nous avons observé une chute significative du nombre d’atomes capturés. Sous la supervision d’I. Bouchoule, une intervention lourde a alors été décidée : nous avons cassé le vide pour diagnostiquer le problème. Il s’est avéré que les connecteurs du dispenser étaient endommagés. L’opération a été mise à profit pour installer un nouveau dispenser et remplacer la puce atomique.

Cette opération a mobilisé plusieurs personnes du laboratoire et de ses partenaires : S. Bouchoule (C2N) et Anne [Nom complet à préciser] ont participé à la manipulation et à la pose de la puce, tandis que j’ai pu assister à l’étuvage de l’enceinte à vide avec F. Nogrette. Après cette intervention, j’ai suivi avec I. Bouchoule le réajustement progressif de la séquence de refroidissement : alignement des faisceaux, réglages de la mélasse, optimisation du chargement dans le piège DC, puis dans le guide.

Cet épisode m’a permis de me confronter plus directement aux paramètres critiques du cycle d’évaporation et à la reprise d’une séquence complète. Toutefois, le départ de Léa, qui maîtrisait tous les aspects de la manipulation, a marqué une rupture importante dans la continuité des savoir-faire pratiques liés à cette expérience.


\begin{center}
	({fig:puce} — Schéma de la puce atomique avec fils U, Z, AC, D et D'.)
\end{center}
\begin{center}
	({fig:imagerieFinale} — Schéma optique du système d’imagerie final)
\end{center}
\begin{center}
	[{fig:nuageDC} — Image du gaz dans le piège DC après évaporation]
\end{center}
\begin{center}
	[{fig:gaz1D} — Image typique d’un gaz dans le régime 1D]
\end{center}



\subsection{Génération de potentiels modulés}
%\begin{itemize}
%    \item Courants modulés pour créer des pièges harmoniques ou quartiques.
%    \item Découplage transverse/longitudinal.
%\end{itemize}

\paragraph{Champ des micro-fils.}
Puisque que $m_F = 2 $, (état assuré par pompage optique), le potentiel magnétique $-\vec{\mu} \vec{B}(\vec{r}) $ (avec moment dipolaire magnétique alors $\vec{\mu}$ et le champs magnetque totale que resente les atomes$\vec{B}(\vec{r})$) est proportionnel à $\vert \vec{B}(\vec{r}) \vert$  de sorte que les atomes, en état low-field seeking, sont attirés vers les régions de champ magnétique minimal. Les micro-fils, alignés selon l’axe horizontal $\vec{e}_x$, sont parcourus par des courants alternatifs $\pm I$ (déphasés) produisant le champ magnétique de confinement : un fil central parcouru par un courant \( I \), et deux fils latéraux par des courants opposés \(-I\). 

\paragraph{Champ de biais.}
Un champ de biais transverse $\vec{B}_{\mathrm{biais}} = {B}_{\mathrm{biais}} \, \vec{e}_y$ , avec l'axe verticale par $\vec{e}_y$ , est appliqué afin de régler la distance des atomes par rapport aux micro-fils. En notant $\vec{e}_z$ l’axe horizontal perpendiculaire à $\vec{e}_x$ et $\vec{e}_y$ l’annulation du champ total a lieu en
%Dans cette configuration, un champ de biais transverse est appliqué pour ajuster la distance des atomes au-dessus des micro-fils. Pour y avoir une idée notons l'axe verticale par $\vec{e}_y$, et $\vec{B}_{biais} = {B}_{biais} \, \vec{e}_y$. Alors en notan $\vec{e}_z$ l'axe hortisontale perpetdiculaire à $\vec{e}_x$ et $\vec{e}_y$, le champs totale s'anume en ​
  %permet ainsi de positionner précisément le minimum du potentiel à une hauteur
$z_0 = \mu_0 I / (2 \pi {B}_{\mathrm{biais}} ) $ avec $\mu_0$ la perméabilité du vide . La modulation de ${B}_{\mathrm{biais}}$ permet de déplacer le point où le champ total s’annule, ce qui permet de positionner précisément le minimum du potentiel à une distance $d$ du plan des fils. 

\paragraph{Champ d’Ioffe.}
Afin d’éviter les pertes de Majorana liées à la présence d’un champ nul, un champ longitudinal $B_0 \, \vec{e}_x$ est ajouté, garantissant que le minimum de champ reste non nul.%.Un champ longitudinal (selon $\vec{e}_x$) $B_0$ est ajouté afin que ce minimum ne corresponde pas à un champ nul, ce qui supprime les pertes de Majorana dues aux inversions de spin au voisinage d’un zéro de champ. 

%L’intérêt de ces pièges est que les atomes peuvent être confinés très près des micro-fils — ici à $ d = 15\, \mu m$ , soit l’espacement entre deux fils — ce qui maximise le gradient de champ et donc la fréquence de piégeage transverse

\paragraph{Fréquence de piégeage transverse.}
Dans la configuration étudiée, les atomes sont confinés à $ d = 15\, \mu m$ au-dessus de la puce, soit l’espacement entre deux micro-fils. Cette faible distance maximise le gradient de champ et donc la fréquence de piégeage transverse, qui s’écrit
\begin{eqnarray*}
	\omega_\perp^{(0)} =  \sqrt{\frac{\mu_B}{mB_0}} \frac{\mu_0 I }{2\pi d^2} 
\end{eqnarray*}
avec $\mu_B$ le magnéton de Bohr, $m$ la masse atomique et $\mu_0$ la perméabilité du vide.
%Pour éviter que les atomes ne perçoivent les rugosités magnétiques dues aux défauts des conducteurs, on fait circuler dans les fils un courant alternatif à haute fréquence ($\sim 400\,KHz$) : le potentiel est alors moyenné temporellement, produisant un confinement plus lisse. À $15\, \mu m$ au-dessus de la puce, le profil de champ est localement harmonique, et la fréquence de piégeage transverse devient

\paragraph{Rugosité et suppression par modulation}
Les imperfections géométriques des micro-fils engendrent des fluctuations parasites du champ magnétique le long du guide, créant une rugosité du potentiel. Pour la supprimer, les courants sont modulés à haute fréquence ($\sim 400\,KHz$), bien au-delà des fréquences de piégeage. Dans ce régime, les atomes ne perçoivent que le potentiel moyenné temporellement, où la composante parasite longitudinale est fortement réduite. Le confinement effectif reste harmonique, avec une fréquence transverse donnée par
\begin{eqnarray*}
	\omega_\perp = \frac{\omega_\perp^{(0)}}{\sqrt{2}}.		
\end{eqnarray*}




%\paragraph{Découplage des confinements transverses et longitudinaux.}
%Les courants qui parcourent les fils D, D', d, d' sons selon $\vec{e}_u$ donc les chanps induit sont selon $\vec{e}_x$ noté $B_\parallel^x$ et $\vec{e}_v$ (axex normale à la puce) , noté $B_\parallel^v$. Si les champs selon $\vec{e}_x$ est négligeable devant $B_0$ alors la moyenne de pottenstelle presente une partie transverce et longitudinale decouplés : $\braket{V} = V_\perp ( y , z ) + V_\parallel(x) $.
\paragraph{Découplage des confinements transverses et longitudinaux.}
Les courants qui parcourent les fils $D$, $D'$, $d$ et $d'$ sont orientés selon $\vec{e}_u$. 
Les champs magnétiques induits possèdent alors une composante selon $\vec{e}_x$, notée $B_\parallel^x$, et une composante selon $\vec{e}_v$ (axe normal à la puce), notée $B_\parallel^v$. 
Si le champ selon $\vec{e}_x$ est négligeable devant $B_0$, alors le potentiel moyen se sépare en une partie transverse et une partie longitudinale découplées : 
\(
\braket{V} = V_\perp(y,z) + V_\parallel(x) .
\)


\paragraph{Potentiel longitudinal harmonique.}
Dans la configuration où seuls les fils $D$ et $D'$ sont utilisés, le potentiel longitudinal peut, à l’ordre 2 en $x$, être considéré comme harmonique :
\begin{eqnarray*}
	V_\parallel (x) = V_0 + \frac{1}{2} m \omega_\parallel^2 x^2 ,
\end{eqnarray*}
On note  $2L=1.89 \,mm$ est la distance séparant les fils $D$ et $D'$. Les courants circulant dans ces deux fils sont identiques et notés $I_D = I_{D'}$. Si la condition $B_0 \gg \mu_0 I_D d /(\pi L)^2 $ est vérifiée, alors le terme constant du potentiel vaut approximativement $V_0 \simeq \mu_B B_0$.

\medskip

La pulsation longitudinale totale $\omega_\parallel$ se décompose en deux contributions : (i) une pulsation $\omega_\parallel^x = \sqrt{\frac{6\, d \, \mu_B \, \mu_0 \,I_D }{\pi \, L^4 \, m}}$ induite par le champ longitudinal $B_\parallel^x$ et (ii) une pulsation $\omega_\parallel^v = \sqrt{\frac{\mu_B }{m \, B_0}}\frac{\mu_0 \, I_D }{\pi \, L^2}$ liée au champ  $B_\parallel^v$. Pour des courants $I>1A$ , on a $\omega_\parallel^v \gg \omega_\parallel^x$, et ainsi :  
\begin{eqnarray*}
	\omega_\parallel \propto \frac{I_D}{\sqrt{B_0} L^2}.
\end{eqnarray*} 
La fréquence longitudinale est donc réglée expérimentalement en ajustant $I_D$.

\medskip

Avec les dimensions caractéristiques de la puce et des fils, il est possible d’atteindre des confinements longitudinaux de fréquence $f_\parallel = \omega_\parallel/ 2 \pi$ allant jusqu’à $\sim 150 \, H_z$, la limite étant imposée par le chauffage des fils pour $I_D \leq =4 \, A$.
 
 \medskip
 
 \subparagraph{Mesure de la fréquence transverse et longitudinale}
Pour la caractérisation, la pulsation transverse $\omega_\perp$ a été mesurée par la méthode du mode de respiration transverse \cite{Kagan1996}, tandis que $\omega_\parallel$ a été obtenue à partir des oscillations dipolaires longitudinales. Les détails expérimentaux de ces méthodes figurent dans le manuscrit de thèse de Léa Dubois \cite{L.Dubois2024}, p. 73 et p. 78.

\medskip

 \paragraph{Potentiel longitudinal quartic.}
 Si on ajoute du courand  dans les fils $d$ et $d'$.  Alors on peux avoir un potentiel non gégligeable à l'ordre 4 . Pour simmplifier, les courants dans ces fils $I_d$ et $I_{d'}$ sont identique. et le potentiel s'écrit : 
 \paragraph{Potentiel longitudinal quartique.}
Si l’on ajoute un courant dans les fils $d$ et $d'$, on peut générer un potentiel longitudinal comportant un terme significatif à l’ordre 4 en $x$ . Pour simplifier, on suppose $I_d=I_{d'}$. On obtient alors : 
 \begin{eqnarray*}
 	V_\parallel(x) \, = \, \mu_B B_0  & + & 	 \frac{\mu_B \, \mu_0}{\pi} d  \left [ \frac{I_D}{L^2} + \frac{I_d}{l^2} + 3 \left ( \frac{I_D}{L^4} + \frac{I_d}{l^4} \right ) x^2  +  5 \left ( \frac{I_D}{L^6} + \frac{I_d}{l^6} \right ) x^4 \right ] \\
 	& + & \frac{\mu_B}{B_0} \left ( \frac{\mu_0}{\pi} \right )^2  \left [ \left ( \frac{I_D}{L^2} + \frac{I_d}{l^2} \right ) x^2  + 2 \left ( \frac{I_D}{L^2} + \frac{I_d}{l^2} \right )\left ( \frac{I_D}{L^4} + \frac{I_d}{l^4} \right ) x^4 \right ].
 \end{eqnarray*}
 
 En ajustant $I_D$ et $i_d$, on peut réaliser par exemple un double puits \cite{Schemmer2019}, ou bien supprimer le terme quadratique $x^2$ afin d’obtenir un potentiel quartique pur :
\begin{eqnarray*}
	V_\parallel(x) = a_0 + a_4 x^4 	
\end{eqnarray*}
comme on le fais dans \cite{Dubois2025}.

En pratique, la puce présente des dimensions finies et n’est pas parfaitement symétrique. Un calcul plus précis, prenant en compte la géométrie exacte (disposition et épaisseur des fils), est présenté en annexe de la thèse de Thibault Jacqmin \cite{???}, p. 151. Cela impose un ajustement fin et asymétrique des courants $I_D$, $I_{D'}$, $I_d$ et $I_{d'}$.


On ajuste les courant $I_D$ et $i_d$ pour par exemple fais des douple puit \cite{Schemmer2019} ou en supriment le terme en $x^2$  d’obtenir un potentiel longitudinal quartique de la forme $V_\parallel(x) = a_0 + a_4 x^4$ \cite{Dubois2025}.\\

En réalité la puce presente des dimention finie, Un calcul plus précis prenant en compte la géométrie exacte des fils (disposition sur la
puce, épaisseur finie) se trouve en appendice de la thèse de Thibault Jacqmin [112] , page 151. De plus la pude n'est pas pardetement symetrique donc on doit ajuster les courant $I_D$, $I_{D'}$, $I_d$ et $I_{d'}$.


\paragraph{Caractérisation des potentiels longitudinal et transverse.}
Pour atteindre le régime unidimensionnel, les confinements doivent être fortement anisotropes : un piégeage transverse très fort et un piégeage longitudinal faible. La condition \(\mu, k_B T \ll \hbar \omega_\perp\) garantit le gel des degrés de liberté transverses.

\medskip

Cette configuration est particulièrement adaptée pour obtenir des profils de densité homogènes, nécessaires à certaines expériences de transport. Le transfert des atomes du piège harmonique vers le piège quartique est réalisé de manière \emph{diabatique} (changement rapide du potentiel), car un transfert adiabatique entraîne des pertes importantes.

\paragraph{Caractérisation des potentiels longitudinal et transverse.}
Pour atteindre le régime unidimensionnel, les potentiels de piégeage doivent être très asymétriques : un confinement transverse fort et un confinement longitudinal faible. La fréquence transverse \(\omega_\perp\) doit être suffisamment élevée pour geler les degrés de liberté dans cette direction, avec la condition \(\mu, k_B T \ll \hbar \omega_\perp\).

%\paragraph{Potentiel longitudinal}
%
%Le confinement longitudinal est produit par des courants continus ou modulés dans certains fils. Dans certains protocoles spécifiques, on utilise un potentiel quartique \( V_\parallel(x) = c_4 x^4 \). Le système reste dans le régime 1D tant que la longueur caractéristique longitudinale reste beaucoup plus grande que la transverse.
%
%\paragraph{Potentiel transverse}
%
%Le confinement transverse est réalisé à l’aide de trois micro-fils parallèles situés sur la puce : un fil central parcouru par un courant \( I \), et deux fils latéraux par des courants opposés \(-I\). Cette configuration crée un piège transverse harmonique avec une fréquence \(\omega_\perp\) contrôlable par la valeur du champ \( B_0 \) et le courant. Les atomes sont piégés à environ \( d = 15~\mu\text{m} \) au-dessus de la puce. La fréquence maximale accessible expérimentalement est de l’ordre de \( \sim 100~\text{kHz} \).
%
%\paragraph{Effet de rugosité et suppression par modulation}
%
%La rugosité des micro-fils induit des fluctuations parasites du champ magnétique le long du guide. Pour supprimer cet effet, les courants sont modulés à haute fréquence (environ 400~kHz). Grâce à cette modulation rapide, les atomes ne ressentent que le potentiel moyen, dans lequel la composante parasite longitudinale du champ s’annule. Ce procédé permet d’obtenir un potentiel transverse régulier et stable, avec une fréquence efficace \[ f_\perp = \frac{f_\perp^{(0)}}{\sqrt{2}}. \]
%
%\paragraph{Découplage des confinements transverse et longitudinal.}
%Dans notre dispositif, le confinement transverse est assuré par les micro-fils modulés, tandis que le confinement longitudinal est généré par quatre fils extérieurs (D, D', d, d'). L’analyse du potentiel magnétique moyen montre que, sous l’hypothèse d’un champ de bobine homogène et dominant, les contributions transverse et longitudinale du potentiel sont découplées. Cette propriété est cruciale pour nos expériences : elle permet de modifier la géométrie du potentiel longitudinal sans perturber le confinement transverse, facilitant ainsi l’exploration de différentes configurations dynamiques.
%
%\paragraph{Piégeage longitudinal harmonique.}
%Un piège longitudinal harmonique est réalisé en appliquant des courants égaux dans les fils D et D', disposés de manière symétrique. Le champ magnétique longitudinal produit conduit à un potentiel quadratique local :
%\[
%V_\parallel(x) = V_0 + \frac{1}{2} m \omega_\parallel^2 x^2,
%\]
%avec une fréquence $\omega_\parallel$ contrôlée par le courant et la géométrie de la puce. En pratique, des fréquences jusqu’à 150 Hz sont atteintes pour des courants de 4 A. Une correction peut être nécessaire pour prendre en compte un champ magnétique résiduel $B_{0v}$, responsable d’un déplacement du centre du nuage atomique.
%
%\paragraph{Piégeage longitudinal quartique.}
%L’ajout de deux fils supplémentaires (d et d') permet de modifier la forme du potentiel longitudinal jusqu’à l’ordre 4. En ajustant les courants dans les quatre fils, on peut annuler le terme quadratique et obtenir un potentiel quartique :
%\[
%V_\parallel(x) = a_0 + a_4 x^4.
%\]
%Cette configuration est particulièrement adaptée pour générer des profils de densité homogènes, comme requis dans certaines expériences de transport. Le transfert des atomes du piège harmonique vers le piège quartique est réalisé de manière diabatique (changement rapide du potentiel), car un transfert adiabatique entraînait des pertes importantes.



\section{Sélection spatiale avec DMD}
\subsection{Motivation et principe}
{\color{blue}
\begin{itemize}
    \item Besoin de préparer des tranches homogènes.
    \item Intérêt dans les protocoles hors équilibre.
\end{itemize}
}

\paragraph{Objectif du dispositif de sélection}

L’outil de sélection spatiale a été conçu pour permettre une action locale sur le gaz atomique. Il présente deux objectifs principaux. D’une part, il permet de mesurer la distribution de rapidité localement résolue, en sélectionnant une tranche du gaz avant de la libérer et de suivre son expansion. D’autre part, il offre la possibilité de créer des situations hors équilibre en retirant une partie du gaz à l’équilibre, ce qui perturbe la configuration initiale et initie une dynamique.

\paragraph{Intérêt pour les protocoles hors équilibre}

Ce dispositif permet ainsi de générer des protocoles analogues à des configurations classiques comme le pendule de Newton, ou de sonder directement la dynamique d’un gaz de Lieb-Liniger dans des conditions contrôlées. Il constitue une brique essentielle pour les expériences de dynamique et de transport quantique.


\subsection{Mise en place technique (initiée par Léa Dubois)}

{\color{blue}
\begin{itemize}
    \item Dispositif optique de projection.
    \item Contrôle numérique des motifs.
    \item Calibration et stabilité.
\end{itemize}
}

\paragraph{Principe de sélection par pression de radiation}

La sélection repose sur l’illumination d’une zone définie du gaz avec un faisceau quasi-résonant avec la transition cyclique \( F=2 \rightarrow F'=3 \) de la ligne D2 du rubidium. Les atomes subissent une pression de radiation due aux cycles absorption/émission spontanée, ce qui les pousse hors du piège ou les amène dans un état non piégé.

\paragraph{Façonnage spatial du faisceau}

La sélection doit être spatialement résolue. Le profil d’intensité dans le plan des atomes est de type binaire :
\[
I(x) = 
\begin{cases}
0 & \text{si } x \in [x_1, x_2] \\
I_0 & \text{sinon}
\end{cases}
\]
ce qui permet de préserver ou d’éjecter les atomes selon leur position longitudinale.

\paragraph{Utilisation du DMD}

Pour générer ce profil, un DMD (Digital Micromirror Device) est utilisé. Il s’agit d’une matrice de \(1024 \times 768\) micro-miroirs orientables individuellement (±12°). En inclinant ces miroirs, on contrôle localement la réflexion de la lumière. L’image du DMD est projetée directement sur le plan des atomes, en imagerie directe.

\paragraph{Avantages du DMD}

Le DMD permet une reconfiguration rapide et programmable du motif de lumière. Cette technologie est largement utilisée dans les expériences d’atomes froids pour produire des potentiels structurés, homogénéiser un faisceau ou adresser localement les atomes.

\paragraph{Alternatives possibles}

Il est possible, en théorie, d’atteindre un effet similaire par un transfert cohérent des atomes vers un état anti-piégé via un pulse micro-onde ou une transition Raman. Cependant, la méthode par pression de radiation est plus simple à mettre en œuvre et adaptée à nos objectifs expérimentaux.

\paragraph{Principe de l’expulsion par pression de radiation}

Un atome illuminé par un faisceau proche de la résonance peut être expulsé du piège soit par transition vers un état anti-piégé, soit par effet de pression de radiation. Cette dernière génère une accélération suffisante pour fournir une énergie cinétique supérieure à la profondeur du puits magnétique. Le nombre de photons diffusés nécessaire peut être estimé à partir de la conservation de l’impulsion : une vingtaine de photons suffisent typiquement à extraire un atome du piège dans nos conditions.

\paragraph{Modèle de diffusion et estimation du seuil}

Le taux de diffusion de photons est modélisé à l’aide d’un taux \(\Gamma_{\mathrm{sc}}\), dépendant de l’intensité \(I\), de l’intensité de saturation \(I_{\mathrm{sat}}\), d’un paramètre \(\alpha\) (lié à la polarisation et au champ magnétique) et du désaccord \(\delta\). À résonance, et pour un temps d’illumination \(\tau_p\), on peut estimer le nombre total de photons diffusés par atome par \(N_{\mathrm{sc}} = \tau_p \Gamma_{\mathrm{sc}}\).

\paragraph{Mesures expérimentales de la puissance nécessaire}

La puissance minimale nécessaire pour éjecter tous les atomes d’une zone illuminée est déterminée en fixant un temps d’illumination donné, puis en variant l’intensité du faisceau. L’analyse est réalisée après un délai d’attente de \(\sim 10\) ms, pour s’assurer que seuls les atomes encore piégés soient détectés. Il est observé que 99$\%$ des atomes sont retirés à partir d’un rapport \(I/I_{\mathrm{sat}} \simeq 0.12\).

\paragraph{Mesures de photons diffusés par fluorescence}

La quantité de photons diffusés est également mesurée par l’analyse du signal de fluorescence capté par la caméra. En calibrant le rapport entre photons détectés et photons diffusés (en tenant compte de l’efficacité optique du système), le nombre moyen de photons nécessaires pour éjecter un atome est confirmé expérimentalement autour de 20. Un ajustement du modèle de diffusion permet d’estimer le paramètre \(\alpha \simeq 0.4\).

\paragraph{Saturation et effets Doppler}

À fort temps d’illumination (\(\tau_p > 150\,\mu\)s), une saturation du nombre de photons diffusés est observée, interprétée comme un effet géométrique : les atomes accélérés atteignent physiquement la puce atomique et cessent de contribuer au signal. Une correction Doppler peut être introduite dans le modèle, mais reste négligeable (\(< 5\%\)) dans les régimes expérimentaux utilisés.

\paragraph{Limitations expérimentales de la sélection}

Plusieurs effets peuvent limiter l'efficacité ou la propreté de la sélection :
\begin{itemize}
    \item La diffraction liée à la taille finie de l’objectif entraîne un flou de l’ordre de \(1{-}2\,\mu\)m au bord des zones éclairées.
    \item Une diffusion parasite par la puce peut se produire à forte intensité si tout le DMD est illuminé ; cela est évité en réduisant la taille transverse du faisceau à quelques micro-miroirs seulement.
    \item Des inhomogénéités d’éclairement dues à la gaussienne du faisceau et au speckle peuvent conduire à une sur-illumination de certaines zones. Un effort a été fait pour homogénéiser l’intensité en sortie de fibre.
    \item La réabsorption des photons diffusés pourrait entraîner un échauffement du gaz restant. Un désaccord en fréquence de 15 MHz a été testé pour éviter ce phénomène, sans effet visible sur la température du gaz.
\end{itemize}

\paragraph{Mesures de l’impact sur le gaz restant}

La température du gaz sélectionné est comparée avant et après sélection via l’analyse des fluctuations de densité après temps de vol. Aucun changement significatif de température ni d’élargissement n’a été observé. Ces résultats suggèrent que, dans les conditions expérimentales utilisées, la sélection ne perturbe pas significativement les atomes restants.




\subsection{Utilisation dans les protocoles}

{\color{blue}
\begin{itemize}
    \item Formes utilisées : boîtes, barrières, coupures.
    \item Préparation initiale contrôlée du gaz.
    \item Exemples de protocoles expérimentaux utilisant le DMD
\end{itemize}
}

\paragraph{Sélection locale et mesure de rapidité}

En sélectionnant une tranche du gaz, on peut ensuite couper le confinement longitudinal et laisser cette tranche s’étendre. Le profil de densité asymptotique obtenu après un long temps d’expansion est proportionnel à la distribution de rapidité locale du gaz initial. Ce protocole permet ainsi une mesure résolue de \(\rho(x,t \to \infty) \sim \rho(v)\).

\paragraph{Génération d’états hors équilibre}

La sélection permet également de créer des discontinuités dans le profil de densité, et donc d’initier une dynamique hors équilibre. Par exemple, on peut ne conserver que deux paquets séparés de gaz, qui vont alors osciller l’un vers l’autre. Cette configuration est analogue à un pendule de Newton quantique.

\paragraph{Formes utilisées}

Les motifs projetés par le DMD peuvent prendre différentes formes : boîtes, barrières, coupures, etc. Cette flexibilité rend l’outil extrêmement précieux pour explorer diverses configurations initiales et protocoles dynamiques.

\paragraph{Contrôle logiciel du DMD}

Le pilotage du DMD repose sur l’utilisation d’un module intégré fourni par Vialux (V7001-SuperSpeed), qui comprend les bibliothèques logicielles ALP-4. Plusieurs configurations du DMD peuvent être chargées en mémoire au début de chaque cycle expérimental, puis sélectionnées en cours de séquence à l’aide d’un signal digital. Le temps de commutation des miroirs est inférieur à \(30\,\mu\mathrm{s}\), ce qui est compatible avec les protocoles étudiés.

\paragraph{Partage du faisceau avec la voie d’imagerie}

Le faisceau utilisé pour la sélection spatiale est prélevé à partir du faisceau sonde déjà accordé sur la transition \(F=2 \rightarrow F'=3\) de la raie D2. Le partage est réalisé à l’aide d’un cube séparateur de polarisation placé en aval d’une lame demi-onde, permettant de contrôler la puissance injectée dans la fibre optique. Ce choix simplifie la mise en œuvre en évitant d’ajouter une source laser supplémentaire.

\paragraph{Blocage du faisceau de sélection}

Deux systèmes permettent de couper le faisceau de sélection pendant le cycle expérimental :
\begin{itemize}
    \item un cache mécanique (type électro-aimant), utilisé pour un blocage longue durée ;
    \item un modulateur acousto-optique (AOM), permettant de produire des impulsions brèves de quelques dizaines de \(\mu\mathrm{s}\), en amont du séparateur.
\end{itemize}
Pour garantir que le faisceau ne perturbe pas l’imagerie, le cache mécanique reste fermé pendant l’utilisation du faisceau sonde.

\paragraph{Montage optique de projection}

Le faisceau façonné par le DMD est projeté dans le plan des atomes à l’aide d’un système optique permettant de sélectionner l’ordre 0 de diffraction. L’ensemble des optiques est dimensionné (diamètre \(50\,\mathrm{mm}\)) pour limiter la diffraction. L’alignement est effectué en superposant le faisceau de sélection à la voie d’imagerie.

\paragraph{Grandissement et champ couvert}

Le montage permet de couvrir une zone de l’ordre de \(600\,\mu\mathrm{m}\) dans le plan des atomes, soit plus que la longueur typique d’un nuage (\(\sim 400\,\mu\mathrm{m}\) pour \(f_{\parallel}=5\,\mathrm{Hz}\)). Le grandissement est déterminé par les focales utilisées : une focale \(f_1 = 750\,\mathrm{mm}\) du côté du DMD, et \(f = 32\,\mathrm{mm}\) pour l’objectif côté atomes, donnant \(G = f/f_1 \approx 0.043\).

\paragraph{Visualisation et interface}

Le contrôle du DMD s’effectue via une interface graphique permettant de prévisualiser les configurations de miroirs. Une capture d’écran de cette interface est présentée dans la Fig.~\ref{fig:dmd_interface}, où la zone active réfléchie est visualisée en rouge. Cette interface est pilotée de manière automatisée pendant le déroulement de la séquence expérimentale.


\section{Techniques d’imagerie et d’analyse}
\subsection{Imagerie par absorption}
{\color{blue}
\begin{itemize}
    \item Imagerie \textit{in situ} et après temps de vol.
    \item Résolution, limites instrumentales.
\end{itemize}
}

\paragraph{Système d’imagerie par absorption}

L’imagerie est réalisée à l’aide d’une caméra CCD à déplétion profonde, optimisée pour une grande efficacité quantique à la longueur d’onde de 780 nm. On utilise des techniques d’imagerie par absorption permettant d’extraire la densité optique \( D(x, z) \), elle-même reliée à la densité atomique 3D via la loi de Beer-Lambert. Le profil de densité linéaire \( n(x) \) est obtenu par intégration sur les directions transverses.

\paragraph{Imagerie après temps de vol}

En appliquant un champ magnétique vertical (\( B = 8\,\mathrm{G} \)), la polarisation du faisceau peut être rendue circulaire (\( \sigma^+ \)) pour adresser la transition fermée \( |F=2, m_F=2\rangle \rightarrow |F'=3, m_F'=3\rangle \). Cette configuration assure une meilleure définition de la section efficace d’absorption. Un temps de vol de quelques ms est utilisé avant l’imagerie, permettant également de décomprimer le nuage.

\paragraph{Imagerie in situ}

Sans champ magnétique, les atomes sont imagés à $7~\mu m$ de la puce, ce qui implique une double absorption du faisceau incident et réfléchi. Dans ce cas, la transition n’est pas fermée, ce qui nécessite une calibration du facteur de conversion entre la densité mesurée et la densité réelle. Un ajustement linéaire permet de relier les profils in situ aux profils obtenus après temps de vol.

\paragraph{Choix des paramètres d’imagerie}

L’intensité du faisceau sonde est choisie typiquement à \( I_0/I_{\mathrm{sat}} \approx 0.3 \) pour optimiser le rapport signal sur bruit tout en restant dans une zone de linéarité acceptable. Dans ces conditions, le nombre de photons diffusés est de l’ordre de \( N_{\mathrm{sc}} \approx 230 \) et le rayon de diffusion reste comparable à la résolution du système d’imagerie (\( \sim 2.6\,\mu \mathrm{m} \)).

\paragraph{Limites du modèle de Beer-Lambert}

La validité de la loi de Beer-Lambert repose sur une approximation à une particule. Dans le cas des gaz fortement denses ou quasi 1D, les effets collectifs, les réabsorptions et les couplages dipolaires peuvent invalider ce modèle. Pour cette raison, même pour l’imagerie in situ, un temps de vol court (\( \sim 1\,\mathrm{ms} \)) est souvent appliqué afin de diluer le gaz transversalement.

\paragraph{Défauts et instabilités expérimentales}

Plusieurs limitations instrumentales ont été identifiées :
\begin{itemize}
    \item La caméra initialement utilisée montrait des motifs parasites aléatoires ainsi qu’un offset variant au cours du temps. Le remplacement de la caméra a permis de résoudre ces problèmes.
    \item Des franges d’interférences apparaissaient lors de la division des images d’absorption, probablement dues à des effets Fabry-Pérot dans les optiques. Le désaxage du faisceau d’imagerie a permis d’en limiter l’impact.
    \item Des photons résiduels, même en l’absence de faisceau sonde, ont été détectés. Ces derniers proviennent vraisemblablement de diffusions multiples dans le système optique.
\end{itemize}

\paragraph{Conclusion}

La combinaison de l’imagerie in situ et après temps de vol, ainsi qu’une calibration soigneuse des paramètres optiques et expérimentaux, permettent d’accéder à des profils de densité fiables malgré les limites intrinsèques du système d’imagerie. Une attention particulière a été portée à la réduction des artefacts expérimentaux afin de garantir la précision des mesures.


\subsection{Analyse des profils}

{\color{blue}
\begin{itemize}
    \item Extraction des densités, tailles, températures.
    \item Distribution longitudinale.
    \item Estimation de la température par ajustement Yang-Yang (optionnel si pertinent).
\end{itemize}
}


\section{Expériences et protocoles étudiés}
Cette section peut être la plus personnelle, en précisant ton rôle à chaque fois.
\subsection{Expansion longitudinale}
\begin{itemize}
    \item Protocole d’expansion (libération longitudinale, maintien du confinement transverse).
    \item Suivi de l’évolution du profil.
    \item Analyse à différents temps d’expansion
    \item Comparaison aux modèles analytiques : solutions homothétiques, GP, asymptotiques.
\end{itemize}

\subsection{Motivation et protocole expérimental d’expansion longitudinale}

\paragraph{Motivation.}
Une partie essentielle de mon travail de thèse a consisté à sonder la distribution de rapidités résolue spatialement, ce qui constitue une information clé pour comprendre la dynamique hors équilibre d’un gaz quantique unidimensionnel. Pour accéder à cette observable, il est nécessaire de réaliser un protocole qui relie la distribution de rapidités à des profils de densité mesurables expérimentalement. L’expansion longitudinale dans le guide 1D s’impose alors comme un outil naturel : en laissant le nuage se dilater librement dans la direction longitudinale, on convertit en partie l’information contenue dans les phases et les excitations collectives du système en une dynamique de densité directement accessible par imagerie. Ce protocole permet ainsi de comparer les prédictions issues des équations effectives, comme l’équation de Gross–Pitaevskii dans différents régimes de confinement, avec des mesures expérimentales résolues spatialement.

\paragraph{Considérations physiques.}
Au-delà de son intérêt pratique, l’expansion longitudinale offre une fenêtre unique sur la physique des gaz bosoniques 1D. Elle permet d’étudier comment un système initialement confiné évolue vers un état dilué, révélant à la fois l’impact du régime transverse (TF 3D vs TF 1D) et l’influence des fluctuations de phase. Dans le régime TF 1D, ces fluctuations deviennent dominantes et se traduisent par des ondulations de densité mesurables. Leur analyse expérimentale, via le spectre de puissance, fournit un accès direct aux corrélations de phase et à la thermodynamique effective du gaz.

\paragraph{Protocole expérimental.}
Concrètement, l’expansion longitudinale est réalisée selon la séquence illustrée en Fig.~??? :
\begin{itemize}
  \item Le nuage est initialement piégé dans un potentiel magnétique caractérisé par une fréquence longitudinale $f_{\parallel} = 5.0$ ou $9.4\,\mathrm{Hz}$ selon les jeux de données, et une fréquence transverse $f_{\perp} = 2.56\,\mathrm{kHz}$.
  \item À $t=0$, le confinement longitudinal est éteint en annulant les courants $I_D=I_{D'}=0$. La coupure est réalisée sur un temps fini $t_{\parallel} = 70\,\mu\mathrm{s} \ll 1/f_{\parallel}$, ce qui évite un pic de courant parasite tout en préservant la dynamique du gaz.
  \item Le nuage se dilate librement dans la direction longitudinale pendant une durée $\tau$. Ensuite, le confinement transverse est relâché en annulant $I_{\perp}$, avec un temps de coupure $t_{\perp} = 5\,\mu\mathrm{s} \ll 1/f_{\perp}$.
  \item Une image par absorption est enfin prise après un temps de vol $t_v$. Pour l’étude des profils de densité, on utilise typiquement $t_v = 1\,\mathrm{ms}$.
\end{itemize}

%\paragraph{Découplage des confinements.}
%Comme discuté en Section~\ref{chap:...}, l’architecture expérimentale rend ce protocole particulièrement simple à mettre en œuvre. La modulation des courants transverses $I_{\perp}$ garantit que le potentiel longitudinal est découplé de celui transverse, ce qui permet un contrôle précis et indépendant des deux confinements.

\paragraph{Perspective.}
La mise en œuvre de ce protocole d’expansion longitudinale ne répond donc pas seulement à un besoin technique de mesure, mais s’inscrit dans une stratégie plus générale : relier les prédictions théoriques de la GHD et des modèles effectifs à des observables accessibles, et sonder directement l’évolution des fluctuations et des corrélations dans un système quantique 1D.

\paragraph{Équations Gross-Pitaevskii dépendantes du temps.}
La dynamique du système étudié est décrite par l’équation de Gross-Pitaevskii (GP) \eqref{chap.1:eq.GP.1} :
\begin{eqnarray*}
	i \partial_\tau\phi = \left \{ - \frac{1}{2}\Delta_{\vec{r}} + V(\vec{r}) + g_{\mathrm{3D}} N \vert \phi \vert^2 \right \} \phi,
\end{eqnarray*}
avec $g_{\mathrm{3D}} = 4 \pi a_{\mathrm{3D}}$ et en présence d’un potentiel externe (voir \eqref{} et \eqref{}) :
\begin{eqnarray*}
	V(\vec{r}) = V_\perp(\vec{r}_\perp) + V_\parallel(x), 
	\qquad 
	V_\perp(\vec{r}_\perp) = \tfrac{1}{2} \, \omega_\perp^2 \, \vec{r}_\perp^2, 
	\qquad 
	V_\parallel(x) = \tfrac{1}{2} \, \omega_\parallel^2 \, x^2.
\end{eqnarray*} 


\paragraph{Séparation des degrés de liberté.}
Dans un piège de type cigare, caractérisé par $\omega_\perp \gg \omega_\parallel$, la dynamique transverse se déroule sur des temps caractéristiques beaucoup plus courts que la dynamique longitudinale. On fait alors l’hypothèse d’un \emph{suivi adiabatique transverse} : l’état reste en permanence dans son état fondamental transverse. Ainsi, les degrés de liberté transverses et longitudinaux se découplent et la fonction d’onde peut se factoriser sous la forme
\begin{equation}
    \phi(r,\tau) = \psi(x,\tau)\,\Phi\!\left(\vec{r}_\perp, n(x,\tau)\right),
\end{equation}
où $\psi(x,\tau)$ décrit la dynamique longitudinale et $\Phi$ est la fonction d’onde transverse dépendant paramétriquement de la densité linéaire $n(x,\tau)$. La condition de normalisation 
\(
\int d \vec{r}_\perp \, \big|\Phi\!\left(\vec{r}_\perp, n\right)\big|^2 = 1
\)
permet de réécrire la densité linéaire définie par 
\(
n \doteq N \int d \vec{r}_\perp \, |\phi|^2
\)
sous la forme
\begin{eqnarray*}
	n(x,\tau) = N \, |\psi(x,\tau)|^2.
\end{eqnarray*}
L’équation de Gross-Pitaevskii se réécrit alors
\begin{eqnarray}
	\left( i \partial_\tau + \tfrac{1}{2} \partial_x^2 - V_\parallel(x) - \mu(n) \right) \psi = 0, 
	\qquad 
	\mu(n)\,\Phi = \left( - \tfrac{1}{2} \Delta_{\vec{r}_\perp} + V_\perp + g_{\mathrm{3D}} \, n \, \big|\Phi(\vec{r}_\perp,n)\big|^2 \right)\Phi.
\end{eqnarray}

\paragraph{Équations hydrodynamiques.}
En utilisant la transformation de Madelung 
\(
\psi(x,\tau) = \sqrt{n(x,\tau)} \, e^{i \vartheta(x,\tau)},
\)
et en introduisant la vitesse $u = \partial_x \vartheta$, on obtient les équations hydrodynamiques associées :
\begin{eqnarray}\label{chap:5:eq.hydro.1}
	\left\{
	\begin{array}{rcl}
		\partial_\tau n + \partial_x ( n u )	 & = & 0, \\[0.3em]
		\partial_\tau u + \partial_x \left( \tfrac{u^2}{2} + V_\parallel(x) + \mu(n) + Q(n) \right) & = & 0,
	\end{array} 
	\right.
\end{eqnarray}
où le terme de pression quantique est donné par
\(
Q(n) = - \frac{1}{2} \, \frac{\partial_x^2 \sqrt{n}}{\sqrt{n}}.
\)
Ces équations sont équivalentes aux deux premières de \eqref{chap:3:eq:hydro.1}, en tenant compte de la relation thermodynamique $dP = n \, d\mu$ et en négligeant le terme de pression quantique $Q(n)$.

\medskip

Pour notre protocole, pour $\tau < 0$ le système est à l’équilibre, avec la condition
\(
\mu(n) + V_\parallel(x) = \mu\bigl(n(x=0)\bigr).
\)
Pour $\tau \geq 0$, le potentiel longitudinal est éteint : $V_\parallel(x) = 0$.

\medskip

\paragraph{Solutions analytiques homothétique.}
Si $n$ est solution des equation hydrodynamique \eqref{chap:5:eq.hydro.1} , pour $\tau \geq 0$. On fais l'hypothèse que la densité linéaire suit une forme homothétique
\begin{eqnarray}
	n(x,\tau) = \frac{1}{\lambda(\tau)} n_0 \left ( \frac{x}{\lambda(\tau)} \right ) ,	
\end{eqnarray}
avec $n_0$ le profil de densité à $\tau = 0 $ et $\lambda(\tau)$ le facter d'echelle à une temps d'expension $\tau$. Avec les containtes $\lambda(0) = 1$ et $\lambda'(0) = 0$ et $N = \int dx \, n(x , \tau ) $. En injectant dans \eqref{chap:5:eq.hydro.1} il vient que 
\begin{eqnarray}\label{chap:5:eq.hydro.2}
	\left\{
	\begin{array}{rcl}
		u(x, \tau ) & = & \displaystyle \frac{\dot\lambda(\tau)}{\lambda(\tau)} x , \\[0.3em]
		\partial_x \mu ( n ( x , \tau ))  & = & - \displaystyle \frac{\ddot\lambda(\tau)}{\lambda(\tau)} x,
	\end{array} 
	\right.
\end{eqnarray}
(car \(\partial_\tau u=(\ddot\lambda/\lambda-\dot\lambda^2/\lambda^2)x\) et \(v\partial_x v=(\dot\lambda/\lambda)^2 x\), leur somme donne \((\ddot\lambda/\lambda)x\)) et initialement $\mu( n_0 ( x ) ) = \mu( n_0 ( x = 0  ) ) - \frac{1}{2} \omega_\parallel^2 x^2 $.

\medskip

Calculons maintenant \(\partial_x\mu(n(x))\). D'abord
\[
\partial_x n(x)=\frac{1}{\lambda^2}\,n_0'\!\Big(\frac{x}{\lambda}\Big).
\]
À l'équilibre \(\mu\big(n_0(y)\big)=\mu_0-\tfrac12 \omega_\parallel^2 y^2\), d'où
\[
\mu'(n_0(y))\,n_0'(y)=-\omega_\parallel^2 y
\quad\Rightarrow\quad
n_0'(y)=-\frac{\omega_\parallel^2\,y}{\mu'(n_0(y))}.
\]
En prenant \(y=x/\lambda\) on obtient
\[
n_0'\!\Big(\frac{x}{\lambda}\Big)
= -\frac{\omega_\parallel^2}{\lambda}\,\frac{x}{\mu'\big(n_0(x/\lambda)\big)}.
\]
Donc
\[
\partial_x n(x) = -\frac{\omega_\parallel^2\,x}{\lambda^3}\;
\frac{1}{\mu'\big(n_0(x/\lambda)\big)}.
\]
Puis
\[
\partial_x\mu(n(x))=\mu'\big(n(x)\big)\,\partial_x n(x)
= -\frac{m\omega_\parallel^2\,x}{\lambda^3}\;
\frac{\mu'\big(n(x)\big)}{\mu'\big(n_0(x/\lambda)\big)}.
\]
Or \(n_0(x/\lambda)=\lambda\,n(x)\), donc on définit
\[
f(\lambda)\equiv\frac{\mu'(n)}{\mu'(\lambda n)}.
\]
On obtient finalement
\[
\partial_x\mu(n(x)) = -\frac{\omega_\parallel^2}{\lambda^3}\,f(\lambda)\,x.
\]

%On souhaite calculer $\partial_x \mu\bigl(n(x,\tau)\bigr)$ en utilisant la règle de la chaîne et la forme homothétique. On a
%\[
%	\partial_x \mu\bigl(n(x)\bigr) 
%	= \mu'(n(x)) \, \partial_x n(x) 
%	= \frac{1}{\lambda^2} \, \mu'(n(x)) \, \partial_x n_0\!\left(\tfrac{x}{\lambda}\right),
%\]
%où le dernier terme s’écrit
%\[
%	\left. \frac{\partial n_0}{\partial x} \right|_{x/\lambda} 
%	= \left. \frac{\partial n_0}{\partial \mu} \right|_{\mu(n_0(x/\lambda))} 
%	\left. \frac{\partial \mu}{\partial x} \right|_{n_0(x/\lambda)} .
%\]
%On utilise alors 
%\[
%	\left. \frac{\partial \mu}{\partial x} \right|_{n_0(x/\lambda)} = - \frac{\omega_\parallel^2}{\lambda^2} \, x,
%	\qquad 
%	\left. \frac{\partial n_0}{\partial \mu} \right|_{\mu(n_0(x/\lambda))} 
%	= \left. \frac{\partial n}{\partial \mu} \right|_{\mu(\lambda n)} .
%\]
%Il vient donc, en utilisant de plus la deuxième équation de 
En remplaçant dans la deuxième d'Euler \eqref{chap:5:eq.hydro.2} et en simplifiant  \(x\),
\begin{eqnarray}\label{chap:5:eq.hydro.3}
	\frac{\ddot\lambda}{\lambda}  
	 =  \frac{\omega_\parallel^2}{\lambda^3} \, f(\lambda)  .
\end{eqnarray}

\paragraph{Proposition.}
Si le facteur
\(
f(\lambda)
\)
est bien défini indépendamment de \(n>0\) (ce qui est le cas pour les solutions homothétiques),
alors \(f\) est une loi de puissance.

\paragraph{Preuve.}
Posons \(g(n) = \mu'(n)>0\) ou \(<0\) (\ie $\mu$ strictement monotone). La définition de \(f\) équivaut à l’existence d’une fonction
\(\chi(\lambda)=1/f(\lambda)\) telle que
\[
g(\lambda n)=\chi(\lambda)\,g(n)\qquad(\forall\,\lambda,n>0).
\]
En prenant \(n=1\), on a \(\chi(\lambda)=g(\lambda)/g(1)\).
Donc, pour tous \(a,b>0\),
\[
\chi(ab)=\frac{g(ab)}{g(1)}=\frac{\chi(a)\,g(b)}{g(1)}=g(a)\,g(b),
\]
c’est-à-dire que \(\chi\) est \emph{multiplicative}. Sous une hypothèse physique très faible
(continuité/mesurabilité ou simple localement bornée), toute fonction multiplicative sur
\(\mathbb{R}_+^\ast\) est de la forme
\[
\chi(\lambda)=\lambda^{\alpha-1}
\quad\Rightarrow\quad
f(\lambda)=\lambda^{1-\alpha}.
\]
%En réintégrant \(g=\mu'\propto n^{\alpha-1}\), on retrouve \(\mu(n)\propto n^\alpha\) pour \(\alpha\neq 0\)
%(et \(\mu(n)\propto \ln n\) pour \(\alpha=0\)).
\qed


% --- Démonstration que f(λ)=λ^{1-\alpha} et réciproque ---
\paragraph{Proposition.}
$f(\lambda) = \lambda^{1-\alpha}$ et $\mu (n) \propto n^\alpha $ sont equivalents.
%
%On définit
%\[
%f(\lambda)=\frac{\mu'(n)}{\mu'(\lambda n)},
%\]
%en supposant \(\mu\in C^1\) et \(\mu'(n)>0\) pour \(n>0\).

\paragraph{1. Si \(\mu(n)=C\,n^\alpha\) (avec \(C\neq0\)) :}
Alors \(\mu'(n)=C\alpha\,n^{\alpha-1}\). Par conséquent
\[
f(\lambda)=\frac{C\alpha\,n^{\alpha-1}}{C\alpha\,(\lambda n)^{\alpha-1}}
=\lambda^{1-\alpha}.
\]

\paragraph{2. Réciproque : si \(f(\lambda)=\lambda^{1-\alpha}\) pour tout \(\lambda>0\) (et tout \(n>0\)) :}
Posons \(g(n)=\mu'(n)\). L'hypothèse s'écrit
\[
\frac{g(n)}{g(\lambda n)}=\lambda^{1-\alpha}
\quad\Longleftrightarrow\quad
g(\lambda n)=\lambda^{\alpha-1}\,g(n),
\]
pour tout \(n>0\) et tout \(\lambda>0\).

Fixons \(n_0>0\) et définissons \(\varphi(\lambda)\equiv g(\lambda n_0)\). La relation ci-dessus donne
\[
\varphi(\lambda)=\lambda^{\alpha-1}\,\varphi(1).
\]
Autrement dit \(\varphi(\lambda)=C_1\,\lambda^{\alpha-1}\) pour une constante \(C_1=\varphi(1)=g(n_0)\). En remplaçant \(\lambda=x/n_0\) on obtient pour tout \(x>0\)
\[
g(x)=C_1\,x^{\alpha-1}.
\]
Ainsi \(g(n)=\mu'(n)=C\,n^{\alpha-1}\) avec \(C\) constant.

En intégrant (en supposant \(\alpha\neq 0\)),
%\[
%\mu(n)=\int \mu'(n)\,dn = \int C\,n^{\alpha-1}\,dn = \frac{C}{\alpha}\,n^\alpha + \text{const},
%\]
%donc 
\(\mu(n)\propto n^\alpha\). (Pour \(\alpha=0\) on obtient \(\mu'(n)=C\,n^{-1}\) et \(\mu(n)=C\ln n+\text{const}\).)

\paragraph{Remarque sur les hypothèses.}
La démonstration utilise la propriété fonctionnelle multiplicative
\(g(\lambda n)=\lambda^{\alpha-1}g(n)\). Sous une hypothèse faible de continuité (ou dérivabilité) en \(n\) cette équation force la forme de puissance \(g(n)\propto n^{\alpha-1}\). Sans régularité, des solutions pathologiques peuvent exister mais ne sont pas physiquement pertinentes dans le contexte thermodynamique.

\qed

%% --- Bref argument (à insérer) ---
%En posant \(g(n)=\mu'(n)\) et \(f(\lambda)=\dfrac{g(n)}{g(\lambda n)}=\lambda^{1-\alpha}\), on obtient
%\[
%g(\lambda n)=\lambda^{\alpha-1}g(n)\quad\forall\,\lambda,n>0,
%\]
%d'où \(g(n)=C\,n^{\alpha-1}\) et, pour \(\alpha\neq0\), \(\mu(n)=\dfrac{C}{\alpha}n^\alpha+\mathrm{const}\), i.e. \(\mu\propto n^\alpha\).
%
%
%% --- Encadré pour le cas alpha = 0 ---
%\medskip
%\noindent\textbf{Remarque (cas \(\alpha=0\)).} Si \(\alpha=0\) alors la relation fonctionnelle donne \(g(n)=\mu'(n)=C\,n^{-1}\). En intégrant on obtient
%\[
%\mu(n)=C\ln n + \mathrm{const}.
%\]
%Ce cas correspond physiquement, par exemple, au gaz isotherme idéal en 1D (ou plus généralement à une dépendance logarithmique du potentiel chimique), où la compressibilité \(\mu'(n)\propto 1/n\).
%
%--------------------------------------------
%
%avec $f(\lambda) = \frac{\mu'(n)}{\mu'(\lambda n )}$ avec  $\mu(n)$ est continue et strictement monitone (donc inversible). Puisque $f(1)= 1$ et $f(\lambda_1 \lambda_2) =  f(\lambda_1) f( \lambda_2)$ alors $f$ est une fonction de puissace $f(\lambda) = \lambda^{1-\beta}$. Ainssi une solution de l'éqtation hydrondynamique homothètique donne  
%\begin{eqnarray}
%	\mu(n) = a n^\beta + b,  	
%\end{eqnarray}
%avec $a, b$ et $\beta$ des réelles. 

\paragraph{Cas particulier.}
Dans le régime quasi-1D on utilise l'expression d'interpolation (cf. Salasnich et al.)
\[
\mu(n)=\hbar\omega_\perp\Big(\sqrt{1+4\,a_{\mathrm{3D}}\,n}-1\Big),
\]
où \(n\) est la densité linéique et \(a_{\mathrm{3D}}\) le scattering length. De cette formule on obtient deux limites asymptotiques :

\begin{itemize}
\item \emph{Régime transverse Thomas--Fermi (TF), \(4a_{\mathrm{3D}}n\gg1\).} 
Alors \(\sqrt{1+4a_{\mathrm{3D}}n}\simeq 2\sqrt{a_{\mathrm{3D}}n}\) et
\[
\mu(n)\simeq 2\hbar\omega_\perp\sqrt{a_{\mathrm{3D}}\,n},
\]
ce qui correspond à \(\mu\propto n^{1/2}\) (donc \(\alpha=\tfrac12\)). Ce régime décrit la situation où \(\mu\gg\hbar\omega_\perp\) et de nombreux niveaux transverses sont excités.
\item \emph{Régime quasi-1D (transverse fondamental), \(4a_{\mathrm{3D}}n\ll1\).} 
Alors \(\sqrt{1+4a_{\mathrm{3D}}n}\simeq 1+2a_{\mathrm{3D}}n\) et
\[
\mu(n)\simeq 2\hbar\omega_\perp\,a_{\mathrm{3D}}\,n \equiv g\,n,
\]
avec \(g=2\hbar\omega_\perp a_{\mathrm{3D}}\). Ici \(\mu\propto n\) (donc \(\alpha=1\)) ; on est proche de l'état fondamental transverse (gaussien).
\end{itemize}

Les deux formes ci-dessus sont bien les limites asymptotiques de l'expression d'interpolation donnée plus haut.

\medskip

Enfin, l'équation d'évolution du facteur d'échelle obtenue précédemment s'écrit correctement
\[
\boxed{\qquad \ddot\lambda\,\lambda^{\alpha+1}=\omega_\parallel^2 \qquad}
\]

% --- Première intégration de l'équation ---
On part de
\[
\ddot\lambda\,\lambda^{\alpha+1}=\omega_\parallel^2,
\]
et on pose \(v=\dot\lambda\). Comme \(\ddot\lambda=\dot\lambda\frac{d\dot\lambda}{d\lambda}\), on obtient
\[
\dot\lambda\frac{d\dot\lambda}{d\lambda}=\omega_\parallel^2\,\lambda^{-(\alpha+1)}.
\]

\paragraph{Cas \(\alpha\neq0\).}
Intégration par rapport à \(\lambda\) :
\[
\frac{1}{2}\dot\lambda^2
= \omega_\parallel^2\int \lambda^{-(\alpha+1)}\,d\lambda
= -\frac{\omega_\parallel^2}{\alpha}\,\lambda^{-\alpha} + C,
\]
où \(C\) est une constante d'intégration déterminée par les conditions initiales \(\lambda(0)=\lambda_0\), \(\dot\lambda(0)=\dot\lambda_0\) :
\[
C=\frac{1}{2}\dot\lambda_0^2+\frac{\omega_\parallel^2}{\alpha}\,\lambda_0^{-\alpha}.
\]
On a donc la première intégrale
\[
\boxed{\; \dot\lambda^2
= \dot\lambda_0^2 + \frac{2\omega_\parallel^2}{\alpha}\big(\lambda_0^{-\alpha}-\lambda^{-\alpha}\big)\; }.
\]
%La solution en quadrature s'écrit alors
%\[
%t-t_0=\int_{\lambda_0}^{\lambda(t)}\frac{d\lambda}{\sqrt{\,v_0^2 + \dfrac{2\omega_\parallel^2}{\alpha}\big(\lambda_0^{-\alpha}-\lambda^{-\alpha}\big)\,}}.
%\]

\paragraph{Cas \(\alpha=0\).}
L'équation devient \(\ddot\lambda\,\lambda=\omega_\parallel^2\). On obtient
%\[
%\frac{1}{2}v^2=\omega_\parallel^2\ln\lambda + C,
%\]
%avec \(C=\tfrac12 v_0^2-\omega_\parallel^2\ln\lambda_0\). D'où
\[
\boxed{\; \dot\lambda^2 = \dot\lambda_0^2 + 2\omega_\parallel^2\ln\!\big(\tfrac{\lambda}{\lambda_0}\big)\; }.
\]

%La quadrature est
%\[
%t-t_0=\int_{\lambda_0}^{\lambda(t)}\frac{d\lambda}{\sqrt{\,v_0^2 + 2\omega_\parallel^2\ln(\lambda/\lambda_0)\,}}.
%\]

%\paragraph{Remarques.}
%\begin{itemize}
%\item Ces intégrales donnent la solution implicite \(t(\lambda)\). En général on ne dispose pas d'une primitive élémentaire fermée pour \(\lambda(t)\) (sauf cas particuliers de choix des conditions initiales), mais la première intégrale ci-dessus est très utile pour l'analyse qualitative (points de retournement, énergie effective, petites oscillations).
%\item Pour les petites oscillations autour de \(\lambda=1\) on peut linéariser et retrouver la fréquence \(\omega_{\rm breath}=\sqrt{4+\beta}\,\omega_\parallel=\sqrt{3+\alpha}\,\omega_\parallel\) (avec \(\beta=\alpha-1\)).
%\end{itemize}


%\subsubsection{Comportement asymptotique du facteur d'échelle}



%\paragraph{Régime à temps longs.} 
%On considère la condition initiale
%\(
%\lambda(0)=1,\,\dot\lambda(0)=0.
%\) 
%et pour $\alpha > 0$ 
%Pour \(\tau\) très grand, on a \(\lambda^{-\alpha}\ll 1\). On a 
%\[
%\lambda(\tau) \simeq \frac{2}{\alpha}\,\omega_\parallel \tau.
%\]
%En particulier :  
%\begin{itemize}
%\item TF 1D (\(\alpha=1\)) : \(\lambda(\tau)\simeq \sqrt{2}\,\omega_\parallel \tau\),  
%\item TF 3D (\(\alpha=1/2\)) : \(\lambda(\tau)\simeq 2\,\omega_\parallel \tau\).  
%\end{itemize}
%%Ces comportements sont observés sur la Fig.~7.4(b).
%
%\paragraph{Régime à temps courts.}  
%À temps courts, on peut approximer \(\mu(x,\tau)\simeq \mu_0(x)=\mu_p-\frac{1}{2}m\omega_\parallel^2 x^2\). Le comportement initial est alors indépendant de l'équation d'état \(\mu(n)\). L'équation d'Euler sans potentiel extérieur donne
%\[
%\frac{d^2 x}{d\tau^2} \simeq \omega_\parallel^2 x \quad\Rightarrow\quad v(x,\tau)\simeq \omega_\parallel^2 x \,\tau \quad (\tau\to 0).
%\]
%En réinjectant ce profil dans l'équation de continuité et en intégrant, on obtient
%\[
%\frac{1}{\lambda(\tau)} \simeq 1 - \frac{\omega_\parallel^2 \tau^2}{2} + \mathcal{O}(\tau^4) \quad\Rightarrow\quad
%\lambda(\tau)\simeq 1 + \frac{\omega_\parallel^2 \tau^2}{2} + \mathcal{O}(\tau^4),
%\]
%ce qui correspond au comportement observé à temps courts sur la Fig.~7.4(a), identique pour les régimes TF 1D et TF 3D.
%
%
%------------------
%
%\paragraph{Régime à temps courts.}  
%Pour \(\tau \to 0\), on linéarise le facteur d'échelle autour de l'équilibre \(\lambda=1\) en posant
%\(\lambda(\tau) = 1 + \epsilon(\tau)\) avec \(|\epsilon|\ll 1\). L'équation de mouvement devient alors
%\[
%\ddot \epsilon + (1+\alpha)\,\omega_\parallel^2 \epsilon - \omega_\parallel^2 = 0,
%\]
%équivalente à un oscillateur harmonique forcé. La solution pour des conditions initiales
%\(\epsilon(0)=0\), \(\dot\epsilon(0)=0\) est
%\[
%\epsilon(\tau) \simeq \frac{\omega_\parallel^2}{1+\alpha}\left[1 - \cos\left(\sqrt{1+\alpha}\,\omega_\parallel \tau\right)\right].
%\]
%Ainsi, à temps très courts \(\tau\ll 1/\omega_\parallel\), on retrouve
%\[
%\epsilon(\tau) \simeq \frac{1}{2}\,\omega_\parallel^2 \tau^2 + \mathcal{O}(\tau^4),
%\]
%et donc
%\[
%\lambda(\tau) \simeq 1 + \frac{\omega_\parallel^2 \tau^2}{2} + \mathcal{O}(\tau^4),
%\]
%ce qui coïncide avec le comportement universel observé à temps courts pour tous les régimes TF, indépendamment de \(\alpha\) et de l'équation d'état \(\mu(n)\).
%
%----------------------------

On impose les conditions initiales
\[
\lambda(0)=1, \qquad \dot\lambda(0)=0,
\]
et l'on considère le cas \(\alpha>0\).

\paragraph{Régime à temps courts (\(\tau \ll 1/\omega_\parallel\)).}  
On linéarise autour de l'équilibre \(\lambda=1\) en posant \(\lambda(\tau)=1+\epsilon(\tau)\) avec \(|\epsilon|\ll 1\). L'équation de mouvement devient un oscillateur harmonique forcé :
\[
\ddot \epsilon + (1+\alpha)\,\omega_\parallel^2 \epsilon - \omega_\parallel^2 = 0.
\]
Pour les conditions initiales choisies, la solution à petits temps est
\[
\epsilon(\tau) \simeq \frac{1}{2}\,\omega_\parallel^2 \tau^2 \quad\Rightarrow\quad
\lambda(\tau) \simeq 1 + \frac{\omega_\parallel^2 \tau^2}{2},
\]
indépendamment de l'équation d'état \(\mu(n)\). Ce comportement correspond au profil universel observé à temps courts (Fig.~7.4(a)) pour tous les régimes TF.

\paragraph{Régime à temps longs (\(\tau \gg 1/\omega_\parallel\)).}  
Pour \(\lambda^{-\alpha}\ll 1\), l'équation intégrée donne
\[
\dot\lambda \simeq \sqrt{\frac{2\omega_\parallel^2}{\alpha}} \quad\Rightarrow\quad
\lambda(\tau) \simeq \frac{2}{\alpha}\,\omega_\parallel \tau.
\]
En particulier :
\begin{itemize}
\item TF 1D (\(\alpha=1\)) : \(\lambda(\tau)\simeq \sqrt{2}\,\omega_\parallel \tau\),  
\item TF 3D (\(\alpha=1/2\)) : \(\lambda(\tau)\simeq 2\,\omega_\parallel \tau\).  
\end{itemize}
Ces comportements sont bien observés sur la Fig.~7.4(b) et correspondent à l’expansion asymptotique du gaz.





% --- Table révisée (petites corrections typographiques) ---
\begin{table}[h]
\centering
\begin{tabular}{l c c c}
\hline
Système & loi pour $\mu(n)$ & $\beta$ (avec $f(\lambda)=\lambda^{-\beta}$) & $\displaystyle \omega_{\rm breath}/\omega_\parallel$ \\
\hline
Gaz classique isotherme (1D, $\mu\propto\ln n$) 
& $\mu'(n)\propto 1/n$ 
& $-1$ 
& $\sqrt{3}\approx1.732$ \\[4pt]

Gaz de Bose 1D en régime moyen (GP, $\mu\propto n$) 
& $\alpha=1$ 
& $0$ 
& $2$ \\[4pt]

Tonks--Girardeau (1D, $\mu\propto n^2$) 
& $\alpha=2$ 
& $1$ 
& $\sqrt{5}\approx2.236$ \\[4pt]

Gaz de Fermi unitaire (ex. 3D, $\mu\propto n^{2/3}$) 
& $\alpha=\tfrac{2}{3}$ 
& $-\tfrac{1}{3}$ 
& $\sqrt{3+\tfrac{2}{3}}\approx1.915$ \\[4pt]

Cas général (loi de puissance) 
& $\mu\propto n^\alpha$ 
& $\beta=\alpha-1$ 
& $\displaystyle \sqrt{3+\alpha}$ \\
\hline
\end{tabular}
\caption{Valeurs de $\beta$ et fréquences du mode de souffle pour quelques régimes usuels.}
\label{tab:breathing}
\end{table}
 



\subsection{Sonde locale de distribution de rapidité}
\begin{itemize}
    \item Principe de la mesure : coupure d’une tranche puis expansion.
    \item Rôle du DMD dans la sélection.
    \item Accès à la distribution de vitesse locale.
    \item Comparaison avec les prédictions GHD.
    \item Limites et incertitudes
\end{itemize}

\subsubsection{Distribution de rapidités locale dans les gaz 1D}

\paragraph{Motivation.}  
La compréhension des gaz de bosons 1D avec interactions de contact répulsives repose sur la notion de distribution de rapidités \(\rho(\theta)\). Chaque état propre du système peut être paramétré par un ensemble de rapidités \(\{\theta_i\}\) (Ansatz de Bethe), ou interprété comme les vitesses de quasi-particules à durée de vie infinie. Ici, on utilise la définition pratique issue des expansions 1D : les rapidités correspondent aux vitesses asymptotiques des atomes après une expansion, avec \(x_j \simeq \tau \theta_j\) pour un temps \(\tau\) long. Cette définition est directement applicable à des mesures expérimentales de distribution de rapidités locales.

\paragraph{Distribution locale et LDA.}  
Pour un nuage atomique piégé dans un potentiel longitudinal variant lentement, on peut appliquer l’Approximation de Densité Locale (LDA). Le gaz est alors vu comme un fluide décomposé en cellules mésoscopiques de densité homogène et relaxée. Dans chaque cellule, l’état d’équilibre est décrit par un Ensemble de Gibbs Généralisé (GGE), ou équivalemment par une distribution de rapidités locale \(\rho(x,\theta)\). Cette description permet d’étudier non seulement l’équilibre, mais aussi la dynamique hors équilibre à grandes échelles spatiales et temporelles, via la théorie Hydrodynamique Généralisée (GHD).

\paragraph{Protocole expérimental.}  
Pour mesurer \(\rho(x,\theta)\) localement :  
\begin{enumerate}
    \item Une zone du nuage atomique de taille \(\ell\) centrée en \(x_0\) est sélectionnée à l’aide d’un dispositif de micromiroirs digitaux (DMD). La pression de radiation supprime instantanément les atomes en dehors de la zone, laissant uniquement ceux de la cellule.
    \item Après la sélection, le confinement longitudinal est relâché, tandis que le confinement transverse reste actif. Les atomes réalisent une expansion 1D pendant un temps \(\tau\), puis le profil de densité est imagé (typiquement pour \(\tau\sim 1\) ms).
    \item Le protocole est répété pour plusieurs positions \(x_0\), permettant d’obtenir la distribution de rapidités locale sur l’ensemble du nuage.
\end{enumerate}

\paragraph{Mesures à l’équilibre.}  
Pour un gaz initialement à l’équilibre dans un piège harmonique, le profil de densité de chaque zone sélectionnée est analysé via la thermodynamique Yang-Yang et la LDA, donnant température \(T_{\rm YY}\) et potentiel chimique \(\mu_{\rm YY}\). Après un temps d’expansion long, le profil devient homothétique à la distribution de rapidités locale \(\rho(x,\theta)\). La comparaison avec les prédictions numériques montre une bonne cohérence, confirmant que le protocole permet de sonder efficacement \(\rho(x,\theta)\).

\paragraph{Résumé.}  
— Une sonde locale de distribution de rapidités a été mise en place grâce au DMD.  
— Les atomes sélectionnés réalisent une expansion dans le guide 1D.  
— Après un temps long, le profil de densité reflète la distribution de rapidités locale.  
— Ce protocole a été appliqué avec succès sur un nuage atomique initialement à l’équilibre.


\section{Discussion sur les limites et les perspectives}
\begin{itemize}
    \item Contraintes techniques (bruit, alignement, stabilité de la puce…).
    \item Améliorations potentielles (résolution, contrôle du potentiel, automatisation).
    \item Perspectives pour d’autres types d’expériences (étude de chocs, turbulence quantique, etc.)
\end{itemize}

\section*{Conclusion}
\begin{itemize}
    \item Résumé de l’architecture du dispositif).
    \item Méthodes d’analyse utilisées et robustesse.
    \item Importance de l’expérience dans le contexte de l’étude des gaz quantiques unidimensionnels
\end{itemize}
Ce chapitre a présenté les éléments essentiels du dispositif expérimental, les méthodes d’imagerie, ainsi que les expériences auxquelles j’ai participé. L’ensemble constitue une plateforme performante pour l’étude de la dynamique de gaz 1D hors équilibre.

\paragraph{Résumé de l’architecture expérimentale}  
Nous avons décrit les éléments clés du dispositif utilisé : un système de refroidissement laser basé sur trois sources couplées, un piégeage magnétique sur puce optimisé pour réaliser des géométries unidimensionnelles, une plateforme de modulation de potentiel via un DMD, et un système d’imagerie haute résolution. L’ensemble permet une manipulation fine des nuages atomiques dans un cadre reproductible et stable.

\paragraph{Méthodes d’analyse et robustesse}  
L’imagerie par absorption, couplée à une analyse rigoureuse des profils atomiques, fournit des outils fiables pour extraire les grandeurs pertinentes : densités, tailles, températures, distributions de vitesses. Ces méthodes ont permis de confronter les résultats expérimentaux à des prédictions théoriques de type GHD ou Yang-Yang.

\paragraph{Importance du dispositif pour la thèse}  
Ce dispositif a été essentiel pour mener à bien les expériences présentées dans cette thèse. Il offre à la fois un contrôle local (grâce au DMD), un bon confinement transverse (grâce à la puce) et une imagerie précise. La plateforme est ainsi bien adaptée pour étudier des systèmes 1D fortement corrélés hors équilibre, et pour tester les prédictions de la physique statistique intégrable.

\paragraph{Perspectives}  
Malgré ses atouts, le dispositif présente des limitations techniques (rugosité magnétique, sensibilité à l’alignement, etc.) qui laissent entrevoir des pistes d’amélioration. Des développements futurs pourraient notamment viser à augmenter la résolution spatiale, automatiser davantage les séquences, ou explorer d'autres régimes dynamiques comme la turbulence ou les collisions de chocs quantiques.



%\appendix
\section*{Annexes}
\begin{itemize}
    \item Schémas techniques (puce, DMD, optique).
    \item Tableaux de paramètres expérimentaux.
    \item Exemples de motifs DMD utilisés.
\end{itemize}
\input{chapters/06_Bipart}
\input{chapters/07_Dipolaire}

%\chapter*{Conclusion}
\addcontentsline{toc}{chapter}{Conclusion}

Conclusion de la thèse.


%\appendix
%\chapter{Annexes}

Informations complémentaires.



\bibliographystyle{abbrv}
\bibliography{thesis}

%\printbibliography

\end{document}

%| Style     | Description                                                             |
%| --------- | ----------------------------------------------------------------------- |
%| `plain`   | Tri alphabétique, numérotation croissante                               |
%| `unsrt`   | Même que `plain` mais sans tri, respecte l’ordre d’apparition           |
%| `abbrv`   | Comme `plain` mais avec prénoms et noms abrégés                         |
%| `alpha`   | Les références sont étiquetées par une combinaison du nom et de l’année |
%| `apalike` | Style APA simplifié                                                     |
%| `ieeetr`  | Style IEEE, tri par ordre d’apparition                                  |
%| `siam`    | Style SIAM (mathématiques appliquées)                                   |
%| `acm`     | Style ACM (informatique)                                                |
%






\subsection{Statistique des macro-états : entropie de Yang-Yang}

%\paragraph{Macro-états et entropie dans la TBA.}

%Dans la limite thermodynamique, dans le modèle statistique (GGE) , les moyenne, observables physiques deviennent des fonctionnelles de la {\bf distribution de rapidité}  $\rho(\theta)$ et du {\bf poing spectrale} $w(\theta)$ . Cette description est efficace car elle permet d’échapper au détail de chaque état propre. 
%Toutefois, cette simplification laisse en suspens une question cruciale : 
%Mais dans ce modelle qui simplifie on veux {\bf la distribution de rapidité d’un système à l'équilibre thermique à température finie} que l'on notera $\langle \rho \rangle$ pour dire la dansité moyenne. Et les lien entre  $w$ et $\langle \rho \rangle$.  Le problème est étudier par par Yang et Yang en 1969. Pour saisir l'enssentielle, nous devons comprendre la {\bf structure statistique des états propres} associés à une même distribution $\rho(\theta)$. Nous nous interrensons comme promis plus haut : à $\Omega(\theta)$ dans l'équation de moyenne \eqref{chap.2.moyenne.2}  ,  {\bf  nombre états propres microscopiques correspondent à une même distribution $\rho(\theta)$}.
%{\bf quelle est la distribution de rapidité d’un système à l'équilibre thermique à température finie ?}. 
%La question a été répondue dans les travaux pionniers de Yang et Yang (1969), que nous allons maintenant examiner brièvement. Pour répondre à cette question, nous devons comprendre la {\bf structure statistique des états propres} associés à une même distribution $\rho(\theta)$.

\paragraph{Motivation.}

Dans la limite thermodynamique, une observable locale dans un \textit{Generalized Gibbs Ensemble} (GGE) dépend uniquement de deux objets continus :  (i)  la \textbf{distribution de rapidité} $\rho(\theta)$, (ii) le \textbf{poids spectral} $w(\theta)$, c.-à-d.\ la " température généralisée " assignée à chaque quasi‑particule.
Cette reformulation est puissante car elle fait disparaître les détails d’un état propre individuel.  

\medskip
Cependant, pour décrire un \emph{vrai} équilibre à température finie, il faut la distribution à l'équilibre :
\begin{eqnarray}\label{chap.2:eq.rho.eq.1}
	\rho_{\mathrm{eq}}(\theta)\;\doteq\;\braket{\operator{\rho}(\theta)}_{\operator{\varrho}[w]}	,  
\end{eqnarray}
donc le lien entre $\rho_{\mathrm{eq}}$ et $w$.
La réponse fut donnée dans les travaux pionniers de \textsc{Yang \& Yang} (1969).  
Leur approche repose sur l’analyse de la \textbf{structure statistique des états propres} partageant la même distribution $\rho(\theta)$.

% : combien d’états microscopiquement distincts correspondent à ce même « macro‑état » ?

\paragraph{Distribution de rapidité comme macro-état.}

Chaque distribution de rapidité $\rho(\theta)$ ne correspond pas à un état propre unique, mais à un grand {\bf ensemble de micro-états} : différents choix des ensembles de quasi-moments $(\{\theta_a\}_{a \in \llbracket 1 , N \rrbracket })_{N \in \mathbb{Z}} $ peuvent conduire à la même densité de distribution à l’échelle macroscopique. Ainsi, $\rho(\theta)$ doit être interprétée comme un {\bf macro-état}, qui agrège un très grand nombre d’états propres microscopiques.

La question thermodynamique devient alors : {\bf Combien de micro-états microscopiquement distincts sont compatibles avec un même macro-état $\rho(\theta)$ ?} 

\medskip
Plus précisément, dans l’expression de moyenne des operateurs locaux \eqref{chap.2.moyenne.2}, apparaît le facteur
\(
\Omega[\rho]
\),
qui compte ces états propres.  
La détermination de $\Omega[\rho]$ (ou équivalemment de l’entropie de Yang–Yang $\mathcal{S}_{YY}[\rho]$ car 
\(
\Omega[\rho] = e^{L\mathcal{S}_{YY}[\rho]}
\)
avec $L$ la taille du système
) est donc la clé pour relier \emph{(i)} le poids spectral $w(\theta)$ imposé dans le GGE et \emph{(ii)} la distribution de rapidité moyenne $\rho_{\mathrm{eq}}(\theta)$ observée à l’équilibre.

\paragraph{Dénombrement local des configurations microcanoniques.}
Pour répondre à cette question, on subdivise l’axe des rapidités en petites tranches ou cellules de largeur $\delta \theta$, chacune centrée en un point $\theta_a$. Dans une tranche $[\theta_a, \theta_a + \delta\theta]$, on suppose que la densité $\rho(\theta)$ est à peu près constante. Le nombre de quasi-particules dans cette tranche est alors approximativement :
\begin{eqnarray*}
	N_a = L\rho(\theta_a) \delta \theta,
\end{eqnarray*}
et le nombre total d'états disponibles (\ie, le nombre d’états possibles si toutes les positions en moment étaient disponibles) est donné par la densité totale de niveaux 
\begin{eqnarray*}
	M_a = L\rho_s(\theta_a) \delta \theta.
\end{eqnarray*}
%La densité de niveaux $\rho_s(\theta)$ tient compte du fait que les moments sont quantifiés de manière discrète, en raison des équations de Bethe (voir équation (??)).

Les particules occupent ces niveaux de manière analogue à des fermions libres (principe d’exclusion de Pauli), le nombre de manières différentes de choisir $N_a$ niveaux parmi $M_a$ est donné par :
	
	
	\begin{figure}[H]
		\centering 
		\begin{tikzpicture}
			%\input{figures/04_GGE_Fluctuation/Occupation_code}	
			\begin{scope}[transform canvas={scale=0.6}]
			\input{figures/04_GGE_Fluctuation/Occupation_theta_code}	
			\end{scope}
			
			\draw[color = red , scale = 0.5 , draw = none ] (-13.5 , -1) rectangle (13 , 10) ; 
				
			
		\end{tikzpicture}	
		\captionsetup{skip=10pt} % Ajoute de l’espace après la légende
	\end{figure}
	
	
\begin{eqnarray}
	\Omega(\theta_a) & \approx  & \binom{M_a}{N_a} ~= ~   \frac{[ L\rho_s ( \theta ) \delta \theta ] ! }{ [ L\rho ( \theta ) \delta \theta ] ! [( L\rho_s ( \theta ) - L\rho ( \theta ) )  \delta \theta ] ! }. 	
\end{eqnarray}

\paragraph{Estimation asymptotique à l’aide de Stirling.}

En utilisant la formule de Stirling :
\begin{eqnarray}
	n! & \underset{n \to \infty}{\sim} &  n^n e^{-n} \sqrt{2\pi n}.,
\end{eqnarray}	
composé du fonction logarithmique, il vient cette équivalence : 
\begin{eqnarray}
	\ln n! & \underset{n \to \infty}{\rightarrow} & n \ln n \underbrace{- n + \ln \sqrt{2 \pi n }}_{o \left ( n \ln n \right ) } ,\\
	&  \underset{n \to \infty}{\sim} & n \ln n  
\end{eqnarray}
	
$\# \mbox{conf.}$ est jamais null donc on peut approximer, pour de grandes valeurs de $L$ et de $\delta\theta$  : 
\begin{eqnarray}
    \ln \Omega(\theta) & \underset{\underset{\rho (\theta )\leq  \rho_s (\theta )}{\rho \delta \theta  \to \infty}}{\sim}   & L [ \rho_s\ln \rho_s - \rho \ln \rho - (\rho_s - \rho ) \ln ( \rho_s - \rho) ] (\theta )\delta \theta .
\end{eqnarray}

Cette expression donne la contribution par unité de $\theta$ à l’{\bf entropie}  associée à la cellule autour de $\theta_a$.

\paragraph{Entropie de Yang-Yang : définition .}
%L'entropie totale du macro-état $\rho(\theta)$, notée $\mathcal{S}_{YY}[\rho]$, est obtenue en sommant sur toutes les tranches. Pour alléger la notation, nous écrivons cette somme comme :
%Le nombre total de micro-états est le produit de toutes ces configurations pour toutes les cellules de rapidité $[\theta, \theta + \delta \theta]$. %En prenant le logarithme et en remplaçant la somme par une intégrale sur $ \theta$, nous obtenons l'entropie de Yang-Yang :

%L'entropie totale du macro-état $\rho(\theta)$, notée $\mathcal{S}_{YY}[\rho]$, est obtenue en sommant sur toutes les tranches. Pour alléger la notation, nous écrivons cette somme comme :
%Le nombre total de micro-états compatibles avec une distribution macroscopique $\rho(\theta)$ est donné par le produit des nombres de configurations pour chaque cellule de rapidité $[\theta, \theta + \delta \theta]$.

%En prenant la sum le logarithme des $\Omega(\theta)$ , on obtient l'entropie totale de Yang-Yang. Pour alléger la notation, cette somme sur les tranches est notée :

Le nombre total de micro-états compatibles avec une distribution macroscopique donnée $\rho(\theta)$ est obtenu en prenant le produit des nombres de configurations pour chaque cellule de rapidité $[\theta_a, \theta_a + \delta \theta]$ : $ \Omega(\theta_a)$ .
En prenant le logarithme de ce produit, on accède à l'entropie totale. Pour alléger la notation, cette somme sur les cellules est notée
\(
	\sum_a^{\theta-\mbox{\tiny cellules}}	
\)
où chaque $a$ indexe une cellule de rapidité $[\theta_a, \theta_a + \delta\theta]$.
On écrit alors :
\begin{eqnarray}
    \ln \Omega[\rho] & = & \sum_a^{\theta-\mbox{\tiny cellules}} \ln \Omega(\theta_a), \\
    & \approx &   L\mathcal{S}_{YY} [ \rho ] , 	
\end{eqnarray}
où l’on définit l’\textbf{entropie de Yang–Yang} par la formule discrétisée :
\begin{eqnarray}
    \mathcal{S}_{YY}[\rho] & \doteq & \sum_a^{\theta-\mbox{\tiny cellules}} \, [ \rho_s\ln \rho_s - \rho \ln \rho - ( \rho_s - \rho ) \ln ( \rho_s - \rho ) ] (\theta_a) \delta \theta .
\end{eqnarray}

%\paragraph{Énergie généralisée.}	
%Les variations de $w(\theta)$ étant négligeables sur chaque tranche de largeur $\delta\theta$, on peut approximer l’énergie généralisée comme :%  $\sum_{a = 1}^N  f(\theta_a) = \sum_{a \vert tranche } f(\theta_a) \Pi( \theta_a)\delta \theta$.

%\begin{eqnarray}
%	 \mathcal{W} & = & \sum_{a = 1}^N  w(\theta_a)	 ~ \sim ~ L\mathcal{W}[\rho] ~=~ L \sum_a^{\theta-\mbox{\tiny tranches}}	 w(\theta_a) \rho(\theta_a) \delta \theta.
%\end{eqnarray}

\paragraph{Énergie généralisée par unité de longueur : définition.}

Dans le cadre du Generalized Gibbs Ensemble (GGE), l’\textbf{énergie généralisée} associée à une distribution de rapidité $\rho(\theta)$ et à un poids spectral $w(\theta)$ est définie comme la somme des poids assignés à chaque quasi-particule. 
Dans la limite thermodynamique, en supposant que $w(\theta)$ varie lentement sur chaque tranche $[\theta_a, \theta_a + \delta\theta]$ ,  cette somme soit l’\textbf{énergie généralisée par unité de longueur} $\mathcal{W}$ se se définit par :
\begin{eqnarray}
	L \mathcal{W}(\{\theta_a\}) \doteq  \sum_{a = 1}^N w(\theta_a) 
	 \underset{\mbox{\tiny therm .}}{\sim}  L \mathcal{W}[\rho]  \doteq  L \sum_a^{\theta\text{-cellules}} w(\theta_a) \rho(\theta_a)\, \delta\theta. 
\end{eqnarray} 
%La fonctionnelle
%\(
%\mathcal{W}[\rho] = \int d\theta\, w(\theta)\, \rho(\theta)
%\)
%représente donc l’énergie généralisée par unité de longueur, dans l’état macroscopique défini par la distribution $\rho$.


\paragraph{Moyenne des Observables locales dans la limite thermodynamique.}

Dans un ensemble général (GGE), la valeur moyenne de l’observable \eqref{chap.2.moyenne.2} devient :	
	
\begin{eqnarray}\label{chap.2.moyenne.3}
	\underset{\mbox{\tiny therm.}}{\lim} \langle \operator{\mathcal{O}} \rangle_{\operator{\varrho}[w]} &  \approx &  ~ \frac{  \displaystyle \sum_{\rho }  \langle \operator{\mathcal{O}}\rangle_{[\rho]}  e^{L(\mathcal{S}_{YY}[\rho] -  \mathcal{W}[\rho]) }}{ \displaystyle \sum_{\rho } e^{L(\mathcal{S}_{YY}[\rho] -  \mathcal{W}[\rho]) } },
\end{eqnarray}
où la somme $\sum\rho$ porte sur toutes les distributions possibles de rapidité $\rho$

%%%%%%%%%%%%%%%%%%%%%%%%%%%%%%%%%%%%%%%%%
\paragraph{Passage à la limite continue.}
%En faisant tandre $\delta \theta \to 0 $ , les somme devienen des integrales 
En faisant tendre $\delta\theta \to 0$, les sommes deviennent des intégrales 
%\(
%\sum_a^{\theta-\mbox{\tiny tranches}}\delta \theta   \underset{\delta \theta \to 0 }{\rightarrow}  \int d \theta ,	
%\)
et l'entropie de Yang-Yang ainsi que l’énergie généralisée par unité de longueur prennent la forme :
\begin{eqnarray}
	\mathcal{S}_{YY}[\rho] & = & \int d \theta  \, [ \rho_s\ln \rho_s - \rho \ln \rho - ( \rho_s - \rho ) \ln ( \rho_s - \rho ) ] (\theta) , \label{chap.2.entropi.int}\\
	\mathcal{W}[\rho] & = & \int   w(\theta) \rho(\theta) \, d \theta \label{chap.2.W.int}		
\end{eqnarray}

%%%%%%%%%%%%%%%%%%%%%%%%%%%%%%%%%%%%%%%%%%
\paragraph{Formule fonctionnelle pour les moyennes.}

%et la valeur moyenne des opservables $\langle \operator{\mathcal{O}} \rangle$ s'écrit commes une intégrale de chemin/formelle
Dans la limite thermodynamique $L \to \infty$, la somme sur les distributions de rapidité $\rho$ admissibles peut être approximée par une intégrale fonctionnelle sur l’espace des densités de rapidité continues, munie d’une mesure fonctionnelle $\mathcal{D}\rho$ : 
\(
\sum_{\rho } \sim \int \mathcal{D} \rho .
\)
Cette correspondance repose sur l’idée que les macro-états admissibles deviennent denses dans l’espace fonctionnel, et que le poids statistique associé à chaque configuration est donné par l’entropie de Yang–Yang.
La mesure fonctionnelle $\mathcal{D}\rho$ parcourt l’espace des densités
$\rho(\theta)$ continues, \emph{chaque configuration étant pondérée par le
facteur exponentiel}
\(
e^{\,L(\mathcal{S}_{YY}[\rho]-\mathcal{W}[\rho])}.
\)
Finalement, la moyenne d'une observable dans le GGE \eqref{chap.2.moyenne.3} s’écrit comme une intégrale fonctionnelle/de chemin :
\begin{eqnarray}
	\underset{\mbox{\tiny therm.}}{\lim} \langle \operator{\mathcal{O}} \rangle_{\operator{\varrho}[w]} & = & \frac{\int \mathcal{D} \rho \; e^{L (\mathcal{S}_{YY}[\rho] - \mathcal{W}[\rho])} \, \langle\operator{\mathcal{O}}\rangle_{[\rho]}}{\int \mathcal{D} \rho \; e^{L (\mathcal{S}_{YY}[\rho] - \mathcal{W}[\rho])}}. \label{chap:TBA:eq:ensemble_average}
\end{eqnarray}


%----------------------
%------------------------------------------------------------------
%\paragraph{Passage de la somme discrète à l’intégrale fonctionnelle.}

%Dans la limite thermodynamique $L\to\infty$, l’ensemble (discret) des
%distributions de rapidité admissibles devient dense dans l’espace
%fonctionnel ; la somme correspondante peut donc s’approximer par une
%intégrale fonctionnelle :
%\[
%\sum_{\rho}\; \longrightarrow\; \int\! \mathcal{D}\rho .
%\]
%La mesure fonctionnelle $\mathcal{D}\rho$ parcourt l’espace des densités
%$\rho(\theta)$ continues, \emph{chaque configuration étant pondérée par le
%facteur exponentiel}
%\(
%e^{\,L\bigl[\mathcal{S}_{YY}[\rho]-\mathcal{W}[\rho]\bigr]},
%\)
%qui combine
%\begin{itemize}
%\item l’\textbf{entropie de Yang–Yang}
%      $\displaystyle
%        \mathcal{S}_{YY}[\rho]
%        =\!\int d\theta\,
%          \bigl[
%            \rho_s\ln\rho_s
%            -\rho\ln\rho
%            -(\rho_s-\rho)\ln(\rho_s-\rho)
%          \bigr]$,
%\item le \textbf{coût énergétique généralisé}
%      $\displaystyle
%        \mathcal{W}[\rho]
%        =\!\int d\theta\, w(\theta)\,\rho(\theta)$,
%\end{itemize}
%où $w(\theta)$ est le \emph{poids spectral} fixé par le GGE.

%------------------------------------------------------------------
%\paragraph{Moyenne d’une observable dans le GGE.}

%On obtient alors la formule de champ moyen
%\begin{equation}\label{eq:GGE-functional-average}
%\bigl\langle\mathcal{O}\bigr\rangle_{\!{\rm GGE}}
%=
%\frac{\displaystyle
%      \int \mathcal{D}\rho\;
%      e^{L\bigl[\mathcal{S}_{YY}[\rho]-\mathcal{W}[\rho]\bigr]}\,
%      \langle\mathcal{O}\rangle_{[\rho]}}
%     {\displaystyle
%      \int \mathcal{D}\rho\;
%      e^{L\bigl[\mathcal{S}_{YY}[\rho]-\mathcal{W}[\rho]\bigr]}}.
%\end{equation}

%------------------------------------------------------------------
\paragraph{Interprétation thermodynamique.}

\begin{itemize}[label = $\bullet$] 
\item $\mathcal{S}_{YY}[\rho]$ \emph{compte} le logarithme du nombre de
      micro-états réalisant la distribution $\rho(\theta)$ :
      c’est l’\textbf{entropie combinatoire}.
\item $\mathcal{W}[\rho]$ mesure le \emph{coût énergétique généralisé}
      associé à cette distribution, dicté par le poids spectral $w(\theta)$.
\end{itemize}

Leur différence
\[
(\mathcal{S}_{YY}-\mathcal{W})[\rho]
\]
joue donc le rôle d’une \emph{fonction thermodynamique effective}
(analogue à une entropie libre).  
L’exposant $e^{L(\mathcal{S}_{YY}-\mathcal{W})[\rho]}$ fixe la \textbf{probabilité relative} d’un
macro-état $\rho(\theta)$ dans le GGE : le terme entropique favorise la
multiplicité des états, tandis que le terme énergétique pénalise les
configurations coûteuses — d’où la compétition caractéristique de
l’équilibre statistique.





%avec $\mathcal{O}[\rho]$ la valeur de l’observable dans un état propre caractérisé par la distribution de rapidité $\rho$.	
%où $\mathcal{O}[\rho]$ est la valeur de l’observable dans un état propre caractérisé par la distribution $\rho$.

%% !TEX encoding = IsoLatin

%\documentclass[11pt,a4paper]{report}
%\documentclass[11pt,a4paper]{book}
\documentclass[10pt, titlepage]{book} % Taille de base des caractères (12pt recommandée pour lecture)


% -------------------------------------
% Encodage et langue
% -------------------------------------
\usepackage[utf8]{inputenc}
\usepackage[T1]{fontenc}
\usepackage[french]{babel}

% -------------------------------------
% Marges et dimensions
% -------------------------------------
\usepackage[a4paper, top=1.0cm, bottom=1.0cm, left=1cm, right=1cm]{geometry} 
% Ajuste ici les marges selon tes préférences

% -------------------------------------
% Interligne
% -------------------------------------
\usepackage{setspace}
%\onehalfspacing  % Interligne 1.5 
%\doublespacing %(utilise \doublespacing pour double interligne)

% -------------------------------------
% Police (facultatif)
% -------------------------------------
%\usepackage{mathptmx} % Police Times (ancienne)
%\usepackage{libertine} % Police élégante

\usepackage{newtxtext,newtxmath} % Times moderne pour texte et maths

% -------------------------------------
% Paquets utiles
% -------------------------------------
\let\Bbbk\relax
\let\openbox\relax
\usepackage{amsmath, amssymb, amsthm}
\usepackage{graphicx}
\usepackage{hyperref}
\usepackage{xcolor}
\usepackage{braket}
\usepackage{tikz}
\usepackage{pgfplots}
\usepackage{float}
\usepackage{enumitem}
\usepackage{caption}
\usepackage{subcaption}
\usepackage{algorithm2e}
\usepackage{cancel}
\usepackage{bm}
\usepackage{listings}
\usepackage{pdfpages}
\usepackage{mdframed}
\usepackage{braket}
\usepackage{stmaryrd} 
\usetikzlibrary {datavisualization}
\usetikzlibrary {arrows.meta,bending,positioning}
\usetikzlibrary {datavisualization.formats.functions}
%PREAMBULE pour schÃéma
\usepackage{pgfplots}
\usepackage{tikz}
\usepackage[european resistor, european voltage, european current]{circuitikz}
\usetikzlibrary{arrows,shapes,positioning}
\usetikzlibrary{decorations.markings,decorations.pathmorphing,
decorations.pathreplacing}
\usetikzlibrary{calc,patterns,shapes.geometric}
\usepackage{anyfontsize}


% -------------------------------------
% Pour les chapitres
% -------------------------------------
\usepackage[Glenn]{fncychap} % Style de chapitres


% -------------------------------------
% Largeur du texte (évite de le redéfinir si tu utilises geometry)
% -------------------------------------
%\setlength\textwidth{20.5cm}
%\setlength\textheight{22cm}

% -------------------------------------
% Optionnel : si tu veux jouer avec les marges manuellement
% -------------------------------------
% \setlength\topmargin{-1cm}
% \setlength\evensidemargin{-2cm}
% \setlength\oddsidemargin{\evensidemargin}

\usepackage{mdframed}

\usepackage{scalerel}
\usepackage{xcolor}
\usepackage{stackengine}
\usepgflibrary {shadings}


\usetikzlibrary {decorations.pathmorphing}

\usepackage{tikz}

\usepackage{marvosym}
\usepackage{changepage}

\usepackage{minitoc}
\usepackage{tocloft}
%\renewcommand{\cfttoctitle}{\hspace{-2em}}
% Nastaveni obsahu
% Nastaveni obsahu

\usepackage{imakeidx}
\usepackage{fancyhdr}

%\usepackage{makeidx}
\makeindex[intoc=true]
\makeindex[name=pers, title=Index of person names, intoc=true]

\usepackage{xcolor}

\usepackage{hyperref}

%%%%%%%%%%%%%%%%%%%%%
%\definecolor{linkcolor}{RGB}{0,0,180}
\usepackage{titlesec}

\usepackage{tocloft}
\usepackage{datetime} % Pour une date personnalisée
\usepackage[useregional]{datetime2}

\usepackage{mathrsfs}

% -------------------------------------
% Pour les mini-tables des matières
% -------------------------------------
\usepackage{minitoc}
\dominitoc

%\usepackage[most]{tcolorbox}

%%%%%%%%%%%%%%%%%%%%%%%%%%%%%
%\usepackage[utf8]{inputenc}
%\usepackage[T1]{fontenc}
%\usepackage[french]{babel}
%\usepackage{amsmath, amssymb}
%\usepackage{graphicx}
%\usepackage{hyperref}
%\usepackage{tikz}
%\usepackage{physics}
%\usepackage{float}


%\newcommand{\ket}[1]{\left|#1\right\rangle}
%\newcommand{\bra}[1]{\left\langle#1\right|}
%\newcommand{\mean}[1]{\left\langle#1\right\rangle}
%\newcommand{\dd}{\mathrm{d}}

% Activer \frontmatter, \mainmatter et \appendix pour la classe report
%\newcommand{\frontmatter}{%
%  \pagenumbering{roman}%
%  \setcounter{page}{1}%
%  \renewcommand{\chaptermark}[1]{\markboth{##1}{}}
%  \renewcommand{\sectionmark}[1]{\markright{##1}}
%}
%
%\newcommand{\mainmatter}{%
%  \pagenumbering{arabic}%
%  \setcounter{page}{1}
%}


% \appendix est déjà défini dans report, inutile de le redéfinir


% Figures flottantes:
% fraction maximale d'une page pouvant etre occupe par une figure:
\renewcommand{\topfraction}{0.8}
% fraction minimale d'une page reservee pour le texte:
\renewcommand{\textfraction}{0.2}
% fraction minimale d'occupation de la page par une figure pleine page:
\renewcommand{\floatpagefraction}{0.7}

%%%%%%%%%%%%%%%%%%%%%%%%%%%%%%%%%%%%%%%%
%         D\'ecoupage des mots           %
%%%%%%%%%%%%%%%%%%%%%%%%%%%%%%%%%%%%%%%%
\hyphenation{}

%%%%%%%%%%%%%%%%%%%%%%%%%%%%%%%%%%%%%%%%
%%%%  Th\'eor\`emes, d\'efinitions, etc.
%%%%%%%%%%%%%%%%%%%%%%%%%%%%%%%%%%%%%%%%


% Il y a diffÃérents types d'ÃénoncÃés qui mÃéritent un environnement spÃécifique, voici une liste assez exhaustive.
\theoremstyle{plain}
    \newtheorem{Theo}{Th\'eor\`eme}[section] %compteur commençant par le numÃéro de la section (on pourrait aussi faire commencer par le numÃéro de la sous-section - remplacer "section" par "subsection")
    \newtheorem{Prop}[Theo]{Proposition}        %mÃême compteur que pour les thÃéorÃèmes
    \newtheorem{Prob}[Theo]{Probl\`eme}        %idem
    \newtheorem{Lemm}[Theo]{Lemme}            %etc...
    \newtheorem{Coro}[Theo]{Corollaire}
    \newtheorem{Propr}[Theo]{Propri\'et\'e}
    \newtheorem{Conj}[Theo]{ Conjecture}
    \newtheorem{Aff}[Theo]{Affirmation}

    \newtheorem{TheoPrinc}{Th\'eor\`eme}     %compteur spÃécifique pour les thÃéorÃèmes les plus importants du papier
        
\theoremstyle{definition}
    \newtheorem{Defi}[Theo]{D\'efinition}
    \newtheorem{Exem}[Theo]{Exemple}
    \newtheorem{Nota}[Theo]{\Large Notation}

\theoremstyle{remark}
    \newtheorem{Rema}[Theo]{Remarque}
    \newtheorem{NB}[Theo]{N.B.}
    \newtheorem{Comm}[Theo]{Commentaire}
    \newtheorem{question}[Theo]{$\ast$ Question}
    \newtheorem{exer}[Theo]{Exercice}
    \newtheorem{Consequence}[Theo]{Conséquence}
    \newtheorem{Rap}[Theo]{Rappel}
    \newtheorem*{Merci}{Remerciements}

\mdfdefinestyle{propstyle}{%
linecolor=black,linewidth=2pt,%
hidealllines=true,
frametitlerule=true,%
frametitlebackgroundcolor=gray!20,
backgroundcolor=gray!10!white,
roundcorner=5pt,
innertopmargin=\topskip,
}

%\mdtheorem[style=propstyle]{prop}{Property}[chapter]
\mdtheorem[style=propstyle]{lemma}[prop]{Lemma}
\mdtheorem[style=propstyle]{TheoPrinc}{Th\'eor\`eme}[chapter]

% Définition d'un style personnalisé pour les Affirmations
\mdfdefinestyle{affirmestyle}{%
    linecolor=gray, % Couleur de la bordure
    linewidth=1pt, % Épaisseur de la bordure
    backgroundcolor=gray!10, % Couleur de fond (gris clair)
    roundcorner=5pt, % Coins arrondis
    innertopmargin=0pt, % Marge intérieure au-dessus du cadre
    innerbottommargin=10pt, % Marge intérieure en-dessous du cadre
    innerleftmargin=10pt, % Marge intérieure à gauche
    innerrightmargin=10pt, % Marge intérieure à droite
    skipabove=10pt, % Espace au-dessus du cadre
    skipbelow=10pt % Espace en-dessous du cadre
}

% Définition de l'environnement Affirmation
\theoremstyle{definition} % Style de théorème pour les affirmations
\newmdtheoremenv[style=affirmestyle]{aff}{Point clé n$^{\circ}$} % Environnement Affirmation avec le style personnalisé
    
\newcommand\dangersign[1]{%
    \renewcommand\stacktype{L}%
    \scaleto{\stackon[1.3pt]{\color{red}$\triangle$}{\tiny !}}{#1}%
}

\tikzset{every picture/.style={execute at begin picture={\shorthandoff{:;!?};}}}
\tikzstyle{every picture}+=[remember picture]
\tikzstyle{na} = [shape=rectangle,inner sep=0pt]

% Commandes pour les flèches textuelles
\newcommand{\ptFleche}[2]{        % Déclaration d'une extrémité de flèche
    \tikz[baseline=(#1.base)]\node[na](#1){#2};
  }
%\newcommand{\Fleche}[5][thick]{    % Dessin de la flèche
%    \begin{tikzpicture}[overlay]
%        \path[->,#1](#2) edge [out=#4, in=#5] (#3);
%    \end{tikzpicture}
%  }
  
% \newcommand{\Flecheprim}[5][thick]{    % Dessin de la flèche
%    \begin{tikzpicture}[overlay]
%        \path[->,#1](#2) edge [out=#4, in=#5] (#3);
%    \end{tikzpicture}
%  }
%



\definecolor{linkcolor}{RGB}{0,0,180}
\PassOptionsToPackage{
    colorlinks=true,
    linkcolor=linkcolor,
    citecolor=linkcolor,
    urlcolor=linkcolor
}{hyperref}

% Appliquer la couleur à tous les niveaux de titre
\titleformat{\section}{\normalfont\color{colorSix!90!black}\Large\bfseries}{\thesection}{1em}{}
\titleformat{\subsection}{\normalfont\color{colorSix!70!black}\large\bfseries}{\thesubsection}{1em}{}
\titleformat{\subsubsection}{\normalfont\color{colorSix!50!black}\normalsize\bfseries}{\thesubsubsection}{1em}{}
\titleformat{\paragraph}[runin]{\normalfont\color{colorOne!30!black}\bfseries}{\theparagraph}{1em}{}
\titleformat{\subparagraph}[runin]{\normalfont\color{colorOne!10!black}\itshape}{\thesubparagraph}{1em}{}
%%%%%%%%%%%%%%%%%%%%%%%

%%Couleurs dans la table des matières

% Modifier la couleur des entrées de la TOC
\renewcommand{\cftsecfont}{\color{linkcolor!90!black}}
\renewcommand{\cftsubsecfont}{\color{linkcolor!70!black}}
\renewcommand{\cftsubsubsecfont}{\color{linkcolor!50!black}}
\renewcommand{\cftparafont}{\color{linkcolor!30!black}}
\renewcommand{\cftsubparafont}{\color{linkcolor!10!black}}
%%%%%%%%%%%%%%%%%%%%%%%%%%%%%
% Reglages:
%
%\pagestyle{fancyplain}
%\addtolength{\headwidth}{\marginparsep}
%\addtolength{\headwidth}{\marginparwidth}
%\renewcommand{\chaptermark}[1]{\markboth{#1}{}}
%\renewcommand{\sectionmark}[1]{\markright{\thesection\ #1}}
%\lhead[\fancyplain{}{\bfseries\thepage}]{}
%\rhead[]{\fancyplain{}{\bfseries\thepage}}
%\chead[\fancyplain{}{\bfseries\leftmark}]{\fancyplain{}{\bfseries\rightmark}}
%\cfoot{}
%

%usepackage{titlesec}
% Changer la couleur des paragraphes en rouge par exemple :
%\titleformat{\paragraph}[runin] % ou [block] selon ce que tu veux
%  {\normalfont\color{red}\bfseries}
%  {\theparagraph}{1em}{}

% Définition des couleurs avec les codes HTML
\definecolor{colorOne}{HTML}{443E46}
\definecolor{colorTwo}{HTML}{F6DEB8}
\definecolor{colorThree}{HTML}{908CA4}
\definecolor{colorFour}{HTML}{57659E}
\definecolor{colorFive}{HTML}{C57284}
\definecolor{colorSix}{HTML}{FF5B69}

% Raccourcis pour les couleurs
\def\colorOne{colorOne}
\def\colorTwo{colorTwo}
\def\colorThree{colorThree}
\def\colorFour{colorFour}
\def\colorFive{colorFive}
\def\colorSix{colorSix}

%%% ===== Index principal + index secondaire (noms propres) =====
\makeindex[intoc=true]
\makeindex[name=pers, title=Index des noms propres, intoc=true]

%%% ===== Couleur des liens =====
\definecolor{linkcolor}{RGB}{0,0,180}
\PassOptionsToPackage{
    colorlinks=true,
    linkcolor=linkcolor,
    citecolor=linkcolor,
    urlcolor=linkcolor
}{hyperref}
\usepackage{hyperref}

%%% ===== Réglages hyperref =====
\hypersetup{
  pdftitle={Étude de la dynamique hors équilibre de bosons unidimensionnels},
  pdfsubject={Quantum Physics},
  pdfauthor={Guillaume THEMEZE <guillaume.themeze@gmail.fr>},
  pdfkeywords={LaTeX, quantum, bosons, dynamique},
  colorlinks=true
}

%%% ===== Style des titres (colorés) =====
\titleformat{\chapter}[display]{\normalfont\sffamily\huge\bfseries\color{colorSix}}{\chaptertitlename\ \thechapter}{20pt}{\Huge}
\titleformat{\section}{\normalfont\color{colorSix!90!colorFour}\Large\bfseries}{\thesection}{1em}{}
\titleformat{\subsection}{\normalfont\color{colorSix!70!colorFour}\large\bfseries}{\thesubsection}{1em}{}
\titleformat{\subsubsection}{\normalfont\color{colorSix!50!colorFour}\normalsize\bfseries}{\thesubsubsection}{1em}{}
\titleformat{\paragraph}[runin]{\normalfont\color{colorSix!30!colorFour}\bfseries}{\theparagraph}{1em}{}
\titleformat{\subparagraph}[runin]{\normalfont\color{colorSix!10!colorFour}\itshape}{\thesubparagraph}{1em}{}

%%% ===== Couleurs de la table des matières =====
\renewcommand{\cftsecfont}{\color{linkcolor!90!black}}
\renewcommand{\cftsubsecfont}{\color{linkcolor!70!black}}
\renewcommand{\cftsubsubsecfont}{\color{linkcolor!50!black}}
\renewcommand{\cftparafont}{\color{linkcolor!30!black}}
\renewcommand{\cftsubparafont}{\color{linkcolor!10!black}}

%%% ===== En-têtes et pieds de page =====
\pagestyle{fancy}
\fancyhf{}
\setlength{\headheight}{14pt}

\fancyhead[RO,LE]{\thepage}
\fancyhead[LO]{\scshape \nouppercase{\rightmark}}  % Section
\fancyhead[RE]{\scshape \nouppercase{\leftmark}}  % Chapitre
\renewcommand{\headrulewidth}{.4pt}


\newdateformat{mydate}{\THEDAY~\monthname[\THEMONTH]~\THEYEAR}
\newdateformat{mydatetime}{\THEDAY~\monthname[\THEMONTH]~\THEYEAR~à~\currenttime}

%\DTMsetstyle{french} % ou autre style
%\DTMsetup{showtimezone=false}

\fancyfoot[L]{Thèse}
%\fancyfoot[R]{Paris, \mydatetime\today{} -- Période 2022--2025}
\fancyfoot[R]{Paris, \DTMnow -- Période 2022--2025}
%\fancyfoot[R]{Paris, le \DTMdate\today{} à \DTMcurrenttime -- Période 2022--2025}
\renewcommand{\footrulewidth}{.4pt}

% Supprimer les numéros sur la première page de chaque chapitre
\makeatletter
\let\ps@plain=\ps@empty
\makeatother

%%% ===== Réglages des titres de sections dans les en-têtes =====
\renewcommand{\chaptermark}[1]{\markboth{#1}{}}
\renewcommand{\sectionmark}[1]{\markright{\thesection\ #1}}

%%% ===== Notes de bas de page à la française =====
\usepackage[french]{babel}
%\usepackage[frenchfootnotes]{french}
%\FrenchFootnotes
%\AddThinSpaceBeforeFootnotes

%%%%%%%%%%%%%%%%%%%%%%%%%%%%%%%%%%
\newcommand{\operatorvec}[1]{\vec{{\bm{#1}}}} % pour les operateur
\newcommand{\operator}[1]{\hat{\bm{#1}}} % pour les operaeur vecteur
\newcommand{\operatormat}[1]{\operatorname{#1}} % pour les operaeur vecteur
\newcommand{\operatortilde}[1]{\tilde{\bm{#1}}} % pour les opetateur avec un tilde
\newcommand{\operatortildevec}[1]{\tilde{\bm{#1}}}% pour les opetateur avec un tilde et vecteur
\newcommand{\dfonc}[1]{\mathscr{D}_{[#1]}}
%%%%%%%%%%%%%%%%%%%%%%%%%%%%%%%%%%

%🔤 2. Abréviations classiques
% Mathématiques générales
\newcommand{\dd}{\mathrm{d}}           % différentielle droite
\newcommand{\ii}{\mathrm{i}}           % unité imaginaire
\newcommand{\ee}{\mathrm{e}}           % exponentielle

% Pour les ensembles usuels
\newcommand{\R}{\mathbb{R}}            % réels
\newcommand{\C}{\mathbb{C}}            % complexes
\newcommand{\Z}{\mathbb{Z}}            % entiers
\newcommand{\N}{\mathbb{N}}            % naturels

% Délimiteurs automatiques
%\newcommand{\abs}[1]{\left|#1\right|}
%\newcommand{\norm}[1]{\left\lVert#1\right\rVert}
%\newcommand{\paren}[1]{\left(#1\right)}
%\newcommand{\bracket}[1]{\left[#1\right]}
%\newcommand{\set}[1]{\left\{#1\right\}}

%⚛️ 3. Physique quantique
% Bra-ket
%\newcommand{\ketbra}[2]{\ket{#1}\!\bra{#2}}
%\newcommand{\braket}[2]{\left\langle #1 \middle| #2 \right\rangle}
%\newcommand{\ketproj}[1]{\ket{#1}\!\bra{#1}}

% Hamiltonien, opérateurs
\newcommand{\Ham}{\mathcal{H}}
\newcommand{\Op}[1]{\hat{#1}}
\newcommand{\Tr}{\mathrm{Tr}}

% Commutateurs et anticommutateurs
\newcommand{\comm}[2]{\left[#1, #2\right]}
\newcommand{\acomm}[2]{\left\{#1, #2\right\}}

%🌡️ 4. GHD ou dynamique intégrable
\newcommand{\rhoP}{\rho_{\mathrm{p}}}       % densité de particules
\newcommand{\rhoT}{\rho_{\mathrm{t}}}       % densité totale
\newcommand{\veff}{v^{\mathrm{eff}}}        % vitesse efficace
\newcommand{\dr}{\partial}                  % dérivée
\newcommand{\nustar}{\nu^\ast}              % solution auto-similaire

%✍️ 5. Utilisation typographique
\newcommand{\eg}{\emph{e.g.}\xspace}
\newcommand{\ie}{\emph{i.e.}\xspace}
\newcommand{\etal}{\emph{et al.}\xspace}



% Commandes spécifiques ou pour la mise en forme

\makeatletter
\newcommand\xleftrightarrow[2][]{%
  \ext@arrow 9999{\longleftrightarrowfill@}{#1}{#2}}
\newcommand\longleftrightarrowfill@{%
  \arrowfill@\leftarrow\relbar\rightarrow}
\makeatother


\newacronym{LL}{LL}{Lieb-Liniger}
\newacronym{NS}{NS}{Schrödinger non linéaire}
\newacronym{GP}{GP}{Gross–Pitaevskii}
\newacronym{GGE}{GGE}{Generalized Gibbs Ensemble}

\title{Titre de la thèse}
\author{Prénom NOM}
\date{\today}



\begin{document}

\frontmatter
%\chapter*{Introduction}
\addcontentsline{toc}{chapter}{Introduction}
\minitoc

Ceci est l’introduction de la thèse.


\tableofcontents
\mainmatter

\chapter{Modèle de Lieb-Liniger et approche Bethe Ansatz}
\minitoc

\section*{Introduction}

Dans ce chapitre, nous introduisons progressivement le modèle de Lieb-Liniger et l'Ansatz de Bethe, outils fondamentaux pour décrire un gaz de bosons unidimensionnel avec interactions delta. L'objectif est d'accompagner pas à pas le lecteur depuis la formulation du problème quantique en champ de bosons jusqu'aux solutions exactes obtenues par l'Ansatz de Bethe.

Nous commençons par écrire l'équation du champ de bosons, exprimée à l’aide des opérateurs de création et d’annihilation en représentation de position. Pour des raisons pédagogiques, nous abordons d’abord le cas d’une seule particule, sans interaction. Cela permet d’introduire naturellement les états de position et leur évolution sous l’action du Hamiltonien libre.

Ensuite, nous étudions le cas de deux particules, cette fois en tenant compte de l’interaction locale. Cela nous amène à considérer les états de position dans le cas général, y compris lorsque les deux particules peuvent occuper la même position. Cette situation, bien plus subtile qu’il n’y paraît, met en évidence la complexité introduite par l’interaction, et justifie que l’on commence par analyser les configurations où les particules sont à des positions distinctes.

Dans le référentiel du centre de masse, le problème à deux corps avec interaction devient équivalent à un problème à une seule particule en interaction avec une barrière delta au centre. Cette reformulation permet d’interpréter l’effet de l’interaction comme une condition de raccord sur la fonction d’onde, tout en respectant la symétrie bosonique.

Nous revenons ensuite aux coordonnées du laboratoire afin d’introduire naturellement la forme des solutions imposée par l’Ansatz de Bethe. Cela nous conduit aux équations dites de Bethe, qui relient les quasimoments des particules à travers des conditions de périodicité modifiées par l’interaction.

Une fois les notations bien établies, nous généralisons le raisonnement au cas de \(N\) particules, pour obtenir l’Hamiltonien de Lieb-Liniger complet ainsi que la forme générale de l’Ansatz de Bethe. Les solutions ainsi construites permettent non seulement de déterminer le spectre de l’Hamiltonien, mais aussi de calculer des observables physiques importantes, telles que l’impulsion totale ou le nombre de particules.

Enfin, nous introduisons la notion de distribution de rapidité, outil essentiel dans l’étude des états d’énergie minimale (états fondamentaux) et dans la description thermodynamique du système. Ce cadre servira de base aux développements ultérieurs sur les gaz intégrables à température finie et les états stationnaires après quench quantique.

\section{Description du modèle de Lieb-Liniger}

\subsection{Introduction au modèle de gaz de Bose unidimensionnel et Hamiltonien du modèle}

\subsubsection{De la première à la seconde quantification}

\paragraph{Introduction.}

La mécanique quantique se développe historiquement en deux grandes étapes : la \emph{première quantification}, aussi appelée quantification canonique, et la \emph{seconde quantification}. Comprendre ces deux cadres est essentiel pour aborder les systèmes quantiques complexes, en particulier ceux où le nombre de particules peut varier.

%La mécanique quantique s’est historiquement développée en deux étapes : la \emph{première quantification}, aussi appelée quantification canonique, puis la \emph{seconde quantification}. Comprendre ces deux cadres est essentiel pour aborder les systèmes à nombre de particules variable.


%\vspace{0.5cm}

\paragraph{Première quantification (quantification canonique, particule unique).}

La première quantification est la mécanique quantique standard, celle que vous avez rencontrée dès vos premiers cours. Elle consiste à quantifier un système classique décrit par des variables dynamiques telles que la position $x$ et la quantité de mouvement $p$. On procède en remplaçant ces variables par des {\bf opérateurs hermitiens} $\operator{x}$ et %$\operator{p}$
\begin{eqnarray}
	\operator{p} \doteq -i\hbar \operator{\partial}_x,	\label{chap.1.rapel.1}
\end{eqnarray}
où $\hbar$ est la constante de Planck réduite, satisfaisant la {\bf relation de commutation canonique} fondamentale $[\operator{x}, \operator{p}] = i\hbar$. L’état du système est alors décrit par une {\bf fonction d’onde} $\psi(x,t)$, solution de {\bf l’équation de  Schrödinger} indépendante du nombre de particules :
\begin{eqnarray}
\quad i \hbar \frac{\partial \psi }{\partial t}  &= \operator{\mathcal{H}} \psi,\label{chap.1.rapel.2}
\end{eqnarray}

avec $\operator{\mathcal{H}}$ l’opérateur hamiltonien. 

\begin{mdframed}[
	linewidth=0.5pt, 
	backgroundcolor=gray!5, 
	roundcorner=50pt,	
	innerleftmargin=5pt,
    innerrightmargin=5pt,
    innertopmargin=-10pt,
    innerbottommargin=2pt,
    leftmargin=2pt,
    rightmargin=2pt
	]
\subparagraph{Exemple : particule libre en une boite à une dimension.} 
	{~}\\
	
	Dans le cas d’une particule libre de masse $m$ se déplaçant en une dimension, l’Hamiltonien est constitué uniquement du terme cinétique $\operator{\mathcal{H}} = \operator{p}^2 / 2m$. En représentation position, où l’opérateur quantité de mouvement s’écrit comme dans l’équation \eqref{chap.1.rapel.1}, l’Hamiltonien prend alors la forme différentielle :
	\begin{eqnarray}
		\operator{\mathcal{H}} = -\frac{\hbar^2}{2m} \partial_x^2.
	\end{eqnarray}
	Les états propres stationnaires de \eqref{chap.1.rapel.2} dépendant du temps sont de la forme $\psi_k(x,t) = \varphi_k(x)\,e^{-i\varepsilon(k)t/\hbar}$ où $\varphi_k(x)$ est une fonction propre de l’hamiltonien,  soit de  l’équation stationnaire  $\operator{\mathcal{H}}\varphi_k = \varepsilon(k)\varphi_k$ \ie pour une particule libre:
	\begin{eqnarray}
		\frac{\hbar^2}{2m} \partial_x^2 \varphi_k = \varepsilon(k) \varphi_k,
	\end{eqnarray}
	avec $\varepsilon(k)$ l’énergie associée à une onde plane de nombre d’onde $k$
	\begin{eqnarray}
		\varepsilon(k) = \frac{\hbar^2 k^2 }{2 m}.
	\end{eqnarray}
	Les fonctions propres spatiales $\varphi_k(x)$ de l’hamiltonien libre s’écrivent comme des combinaisons linéaires d’ondes planes  
	\begin{eqnarray}
		\varphi_k(x) = a e^{-i k x} + b e^{i k x},~~ \mbox{avec}\quad  (a,b) \in \mathbb{C}^2.
	\end{eqnarray}
	Si la particule est confinée dans une boîte de longueur $L$ avec des conditions aux limites périodiques (ie $\varphi_k(x+L) = \varphi_k(x)$), alors le spectre de $k$ est quantifié : 
	\begin{eqnarray}
		kL \in 2\pi\mathbb{Z}.
	\end{eqnarray}
	Le problème est équivalent à celui d’une particule libre sur un cercle de périmètre $L$.\\
	
	\medskip
	
	Les solutions générales de l’équation de Schrödinger s’écrivent alors comme une superposition d’états propres  $\psi = \sum_k c_k \psi_k $.  

%On résume :
%\begin{eqnarray}
%	,~~ , ~~\varphi_k(x) = a e^{-i k x} + b e^{i k x},~~ kL \in 2\pi\mathbb{Z}.\label{chap.1.recap}
%\end{eqnarray}
\end{mdframed}

Ces solutions correspondent à des {\bf états non liés} (ou états de diffusion) : la particule est délocalisée sur tout l’espace (le cercle), sans structure particulière.

La fonction $\varphi_k(x)$ est supposée normalisée dans l’espace des états quantifiés (boîte finie) :
\(
\int_0^L dx \, \varphi_{k'}^\ast(x)\, \varphi_k(x) = \delta_{k,\pm k'}.
\)
avec  $ \vert a \vert^2 + \vert b \vert^2 = L^{-1}$.
Et dans le sous espace engendré pas $x \mapsto e^{-ikx}$ et $x \mapsto e^{ikx}$ (de deux dimension si $k \neq 0$ et une dimension si $k$ =0), $x \mapsto \pm ( b^\ast e^{-ikx} - a^\ast e^{ikx} )$ est orthogonale à  $\varphi_k$ pour $k \neq 0$.
\begin{mdframed}[
	linewidth=0.5pt, 
	backgroundcolor=gray!5, 
	roundcorner=50pt,	
	innerleftmargin=5pt,
    innerrightmargin=5pt,
    innertopmargin=-10pt,
    innerbottommargin=2pt,
    leftmargin=2pt,
    rightmargin=2pt
	]
\subparagraph{Remarque.} On choisie  \( a = \frac{1}{\sqrt{L}} \) et \( b = 0 \)), alors
\(
\varphi_k(x) = \frac{1}{\sqrt{L}}\, e^{i k x}
\)
est une onde plane. 

\end{mdframed}

Avec le formalisme de Dirac, la fonction d’onde $\varphi_k$ est représentée par le ket $\ket{k}$ normé (i.e. $\langle k' \vert k \rangle = \delta_{k, k'}$, où $\delta_{p,q}$ est le symbole de Kronecker)
, et l’équation de Schrödinger s’écrit :
\(
\operator{\mathcal{H}}_1 \ket{k} = \varepsilon(k) \ket{k}.
\)
En appliquant le bra $\bra{x}$ de part et d’autre, on obtient :
\(
\bra{x} \operator{\mathcal{H}}_1 \ket{k} = \varepsilon(k) \langle x \vert k \rangle,
\)
où $\varphi_k(x) = \langle x \vert k \rangle$ est la représentation positionnelle de l’état $\ket{k}$.



%\begin{mdframed}[
%	linewidth=0.5pt, 
%	backgroundcolor=gray!5, 
%	roundcorner=50pt,	
%	innerleftmargin=5pt,
%    innerrightmargin=5pt,
%    innertopmargin=-10pt,
%    innerbottommargin=2pt,
%    leftmargin=2pt,
%    rightmargin=2pt
%	]
%\subparagraph{Remarque.} Si l’on choisit une base orthonormée d’états propres (par exemple en fixant \( a = \frac{1}{\sqrt{L}}, b = 0 \)), alors
%\(
%\varphi_k(x) = \frac{1}{\sqrt{L}}\, e^{i k x}, \quad \text{et donc} \quad \langle k \vert x \rangle = \varphi_k^\ast(x) = \frac{1}{\sqrt{L}}\, e^{-i k x},
%\)
%ce qui est bien une onde plane. 
%En revanche, dans l’écriture générale \( \varphi_k(x) = a\, e^{i k x} + b\, e^{-i k x} \), la fonction \( \langle k \vert x \rangle = \varphi_k^\ast(x) \) n’est \emph{pas nécessairement} une onde plane, car \( \varphi_k(x) \) n’est pas normalisée.
%\end{mdframed}

\begin{mdframed}[
	linewidth=0.5pt, 
	backgroundcolor=gray!5, 
	roundcorner=50pt,	
	innerleftmargin=5pt,
    innerrightmargin=5pt,
    innertopmargin=1pt,
    innerbottommargin=2pt,
    leftmargin=2pt,
    rightmargin=2pt
	]
La base $\{\ket{x}\}$ étant continue, et les états $\{\ket{k}\}$ quantifiés (par exemple dans une boîte de taille finie avec conditions aux limites périodiques), les relations de changement de base s’écrivent :
\begin{eqnarray}
	\ket{k} = \int dx \, \varphi_k(x) \ket{x}, \qquad   
	\ket{x} = \sum_k \varphi_k^\ast(x) \ket{k},
\end{eqnarray}
avec $\varphi_k^\ast(x) = \langle k \vert x \rangle$. L’état $\ket{x}$ est relié aux états $\ket{k}$ par une transformation de Fourier discrète. Ces formules montrent que les états $\ket{k}$ sont les composantes de Fourier de l’état $\ket{x}$.
\end{mdframed}






\subparagraph{De la particule unique aux systèmes à $N$ particules.}

Pour un système composé de $N$ particules identiques, une approche naturelle consiste à introduire une fonction d’onde $\varphi(x_1, \dots, x_N)$ dépendant de $N$ variables , symétrique pour des bosons ou antisymétrique pour des fermions sous l’échange de deux coordonnées $x_i \leftrightarrow x_j$, solution de l’équation de Schrödinger à $N$ corps. %Dans le cas bosonique, des interactions à courte portée peuvent être modélisées par un potentiel de type Dirac :

%\begin{equation}
%V_{\text{int}}(x_1, \dots, x_N) = g \sum_{i<j} \delta(x_i - x_j),
%\end{equation}

%où $g$ est un paramètre d’interaction contrôlant l’intensité des collisions. 
Toutefois, cette description devient rapidement inextricable lorsque le nombre de particules augmente, ou lorsque le système permet la création et l’annihilation de particules, comme dans un milieu ouvert ou en contact avec un bain thermique.





%{\color{blue} \paragraph{Seconde quantification.}
%
%%Dans ce formalisme, l’espace des états est une **somme directe d’espaces à $N$ particules**, et chaque état est décrit par son occupation des modes quantiques. Les opérateurs $\hat{a}_k^\dagger$ et $\hat{a}_k$ créent et détruisent une particule dans l’état d’onde plane de moment $k$, satisfaisant les relations de commutation (bosons) ou d’anticommutation (fermions) :
%%\begin{equation}
%%[\hat{a}_k, \hat{a}_{k'}^\dagger] = \delta_{k,k'} \quad \text{(bosons)}.
%%\end{equation}
%
%%L’hamiltonien d’un gaz de particules libres s’écrit alors simplement :
%%\begin{equation}
%%\hat{\mathcal{H}} = \sum_k \varepsilon(k) \, \hat{a}_k^\dagger \hat{a}_k,
%%\end{equation}
%%avec $\varepsilon(k) = \frac{\hbar^2 k^2}{2m}$ comme dans la quantification canonique.
%
%\paragraph{Vers le Bethe ansatz.}
%
%Ce formalisme devient particulièrement utile lorsque des interactions entre particules sont introduites. Dans le cas d’un **gaz de bosons en une dimension avec interactions de contact**, le système reste exactement soluble : la solution repose sur une **superposition cohérente d’ondes planes symétrisées**, ajustées par les conditions d’interaction.
%
%C’est l’idée fondamentale du **Bethe ansatz**, qui généralise la solution d’une particule libre sur un cercle à $N$ particules **avec collisions élastiques**. On y retrouve des relations de quantification du type :
%\begin{equation}
%k_j L + \sum_{\substack{\ell=1 \\ \ell \neq j}}^N \theta(k_j - k_\ell) = 2\pi n_j,
%\end{equation}
%où $\theta$ est une phase de diffusion et les $n_j \in \mathbb{Z}$.
%
%On passe ainsi des conditions de périodicité simples à des **conditions de type Bethe**, qui encodent les effets des interactions sous forme de **conditions de compatibilité entre les moments**.
%
%}

\subsubsection{Seconde quantification}

Pour dépasser ces limitations, on adopte le \textbf{formalisme de la seconde quantification}, dans lequel l’état du système est décrit non plus par une fonction d’onde mais par un vecteur dans un espace de Fock. Les opérateurs de création et d’annihilation remplacent alors les variables dynamiques classiques et permettent une description unifiée et élégante des systèmes à nombre variable de particules.

%\paragraph{Structure de l’espace des états de Fock.}
%Dans ce formalisme, l’espace des états est une {\bf somme directe d’espaces à $N$ particules}, et chaque état est décrit par l’occupation des différents modes quantiques. Les opérateurs $\operator{a}_k^\dagger$ et $\operator{a}_k^{}$ créent et annihilent une particule dans l’état d’onde plane de moment $k$ :
%\begin{eqnarray*}
%	\ket{k} & = & \operator{a}_k^\dagger \ket{\emptyset} ~=~ \text{état avec une particule dans le mode } k,	
%\end{eqnarray*}
%où \(\ket{\emptyset}\) est le vide quantique de Fock, défini par :
%\begin{eqnarray}
%	\forall k \in \mathbb{R}\colon \qquad \operator{a}_k \ket{\emptyset} = 0 ,\quad  \langle \emptyset\vert \emptyset \rangle = 1, \label{chap:eq.vide.fock.k}
%\end{eqnarray}
%où \( \operator{a}_\lambda \) peut ici estre une notation générique désignant \( \operator{b}_\lambda \) pour les bosons, ou \( \operator{c}_\lambda \) pour les fermions,
%et satisfaisant les relations de commutation (pour les bosons) ou d’anticommutation (pour les fermions). Dans ce qui suit, nous nous restreignons au cas bosonique. \subparagraph{Relations de commutation bosoniques.} Les relations de commutation bosoniques fondamentales sont alors :
%\begin{eqnarray}
%[\operator{a}_k^{}, \operator{a}_{k'}^{}]= [\operator{a}_k^\dagger, \operator{a}_{k'}^\dagger]= 0 ,\qquad [\operator{a}_k^{}, \operator{a}_{k'}^\dagger] = \operator{\delta}_{k,k'},\label{chap:1:com.1.k}
%\end{eqnarray}
%où $\operator{\delta}_{k,k'}$ est le symbole de Kronecker, qui vaut $1$ si $k = k'$ et $0$ sinon.\\

%%%%%%%%%%%%%%%%%%%%%%%%
\paragraph{Structure de l’espace des états de Fock.}
Dans ce formalisme, l’espace des états est une {\bf somme directe d’espaces à $N$ particules}, et chaque état est décrit par l’occupation des différents modes quantiques. Les opérateurs $\operator{a}_k^\dagger$ et $\operator{a}_k$ créent et annihilent une particule dans l’état d’onde plane de moment $k$ :
\begin{eqnarray*}
	\ket{k} & = & \operator{a}_k^\dagger \ket{\emptyset} ~=~ \text{état avec une particule dans le mode } k,	
\end{eqnarray*}
où \(\ket{\emptyset}\) désigne le vide quantique de Fock, défini par :
\begin{eqnarray}
	\forall k \in \mathbb{R}\colon \qquad \operator{a}_k \ket{\emptyset} = 0 ,\quad  \langle \emptyset \vert \emptyset \rangle = 1. \label{chap:eq.vide.fock.k}
\end{eqnarray}
Le symbole \( \operator{a}_\lambda \) représente ici de manière générique soit l’opérateur \( \operator{b}_\lambda \) pour les bosons, soit \( \operator{c}_\lambda \) pour les fermions, et satisfait respectivement les relations de commutation (pour les bosons) ou d’anticommutation (pour les fermions). Dans ce qui suit, nous nous restreignons au cas bosonique.

\subparagraph{Relations de commutation bosoniques.} Les relations de commutation fondamentales pour les bosons sont :
\begin{eqnarray}
	[\operator{b}_k, \operator{b}_{k'}] = [\operator{b}_k^\dagger, \operator{b}_{k'}^\dagger] = 0 ,\qquad [\operator{b}_k, \operator{b}_{k'}^\dagger] = \operator{\delta}_{k,k'}, \label{chap:1:com.1.k}
\end{eqnarray}
où $\operator{\delta}_{k,k'}$ est le symbole de Kronecker, valant $1$ si $k = k'$ et $0$ sinon.
%%%%%%%%%%%%%%%%%%%%%%%%%%%%%%%%%%%%%%%%

%\vspace{1em}
\paragraph{Nature du champ quantique.}
La seconde quantification généralise ce cadre en permettant de traiter des systèmes où le nombre de particules n’est pas fixé, ce qui est fréquent en physique des particules, des champs quantiques, ou des gaz quantiques.

L’idée principale est de ne plus quantifier directement les particules, mais le \emph{champ quantique} associé. Les états d’une particule unique deviennent alors des états d’occupation dans un espace de Fock, qui décrit l’ensemble des configurations possibles avec zéro, une, ou plusieurs particules.



\subparagraph{Champs de Bose.}
Le gaz de Bose unidimensionnel est décrit dans le cadre de la théorie quantique des champs par un champ bosonique canonique \( \operator{\Psi}(x) \), qui agit sur l’espace de Fock des états du système. Ce champ quantique encode l’annihilation d’une particule en \( x \), et son adjoint \( \operator{\Psi}^\dag(x) \) correspond à la création d’une particule en ce point. 
\begin{eqnarray}
	\vert x \rangle  & = & \operator{\Psi}^\dag (x)\ket{\emptyset} ~=~ \text{état avec une particule en } x,
\end{eqnarray}
et \(\ket{\emptyset}\) est le vide quantique de Fock défini par :
\begin{eqnarray}
	\forall x \in \mathbb{R}, \qquad \operator{\Psi}(x) \ket{\emptyset} = 0. \label{chap:eq.vide.fock}
\end{eqnarray}

\subparagraph{Relations de commutation bosoniques.}
Ces champs satisfont les relations de commutation canoniques à temps égal :
%\begin{eqnarray}
%	\left . \begin{array}{rcl}
%		[ \operator{\Psi}(x),  \operator{\Psi}^\dagger(y) ]  &=&  \operator{\delta}(x - y) \\
%		\left [ \operator{\Psi}(x),  \operator{\Psi}(y) \right ]   =  [ \operator{\Psi}^\dag(x),  \operator{\Psi}^\dag(y) ]  &=&  0 
%	\end{array} \right . \label{chap:1:com.1}
%\end{eqnarray}
\begin{eqnarray}
	 [ \operator{\Psi}(x),  \operator{\Psi}(y)  ]   =  [ \operator{\Psi}^\dag(x),  \operator{\Psi}^\dag(y) ]  =  0,   & & [ \operator{\Psi}(x),  \operator{\Psi}^\dagger(y) ]  =  \operator{\delta}(x - y) ,\label{chap:1:com.1}
\end{eqnarray}
où $\operator{\delta}(x - y)$ est la fonction delta de Dirac.  
Ces relations expriment le caractère bosonique des excitations du champ.

\paragraph{État à $N$ particules.} Soient $N$ bosons dans les états $\{ k_1 , \cdots , k_N \}$ (un boson dans l’état $k_1$, un autre dans $k_2$, etc.) et aux positions $\{ x_1 , \cdots , x_N \}$ (un boson en $x_1$, un autre en $x_2$, etc.). Leurs états s’écrivent alors :
\begin{eqnarray}
	\ket{ \{ k_1 , \cdots , k_N \}} = \frac{1}{\sqrt{N!}} \operator{b}_{k_1}^\dag\, \cdots \, \operator{b}_{k_N}^\dag \ket{\emptyset}, \quad \ket{\{x_1 , \cdots , x_N\}} = \frac{1}{\sqrt{N!}} \operator{\Psi}^\dag(x_1)\, \cdots \, \operator{\Psi}^\dag(x_N) \ket{\emptyset}	, \label{eq.chap.1.ket.N}
\end{eqnarray}
où le facteur \( 1/\sqrt{N!} \) traduit le caractère d’indiscernabilité des bosons et garantit la symétrisation correcte de l’état.

\subparagraph{Changement de base.}
On peut relier les opérateurs de création/annihilation dans la base des ondes planes aux opérateurs de champ via :
\begin{eqnarray}
	\operator{b}_k^\dagger = \int dx \, \varphi_k(x) \operator{\Psi}^\dagger(x), \qquad 
	\operator{\Psi}^\dagger(x) = \sum_k \varphi_k^\ast(x)\operator{b}_k^\dagger.\label{eq.chap.1.TF.1}
\end{eqnarray}
Le champ quantique $\operator{\Psi}(x)$ est relié aux opérateurs de moment $\operator{b}_k$ par une transformation de Fourier. Ces formules montrent que les opérateurs $\operator{b}_{k}$ sont les composantes de Fourier du champ $\operator{\Psi}(x)$.

%où $\varphi_k(x)$ est la fonction d’onde d’un état d’énergie bien définie \( \ket{k} \) dans la représentation positionnelle.
Ainsi, un état à \(N\) bosons dans la base \( \ket{k}^{\otimes N} \) peut s’écrire :
\begin{eqnarray}
	\ket{\{k_1 , \cdots , k_N\}} = \frac{1}{\sqrt{N!}} \int dx_1 \cdots dx_N \, \varphi_{\{k_a\}} ( x_1 , \cdots , x_N ) \, \hat{\Psi}^\dag(x_1) \cdots \hat{\Psi}^\dag(x_N) \ket{\emptyset},
\end{eqnarray}
où \( \{k_a\} \equiv \{k_1, \dots, k_N\} \), et la fonction d’onde symétrisée s’écrit :
\(
	\varphi_{\{k_a\}} ( x_1 , \cdots , x_N ) = \frac{1}{\sqrt{N!}} \sum_{\sigma \in \operator{S}_N } \prod_{i=1}^N \varphi_{k_{\sigma(i)}}(x_i),
\) 
avec $\operator{S}_N $  le groupe symétrique d'ordre $N$ mais aussi :
\begin{eqnarray}
	\varphi_{\{k_a\}} ( x_1 , \cdots , x_N ) = \frac{1}{\sqrt{N!}} \bra{\emptyset} \hat{\Psi}(x_1) \cdots \hat{\Psi}(x_N) \ket{\{k_1, \cdots , k_N\}}.
\end{eqnarray}



\subsubsection{Operateur. }


\paragraph{Opérateur à un corps.}

Soit \( \operator{f} \) un opérateur à une particule, dont les éléments de matrice dans une base orthonormée \( \{ \ket{k} \} \) sont donnés par \( f_{\lambda\nu} = \langle \lambda \vert \operator{f} \vert \nu \rangle \). Un opérateur symétrique à \( N \) particules correspondant à la somme des actions de \( \operator{f} \) sur chacune des particules s’écrit en première configuration  :
\(
	\operator{F} = \sum_{i=1}^N \operator{f}^{(i)},
\)
où \( \operator{f}^{(i)} \) désigne l’action de \( \operator{f} \) sur la $i^\text{e}$ particule uniquement. En base de Dirac, cela donne :
\(
	\operator{f}^{(i)} = \sum_{\lambda, \nu} f_{\lambda\nu} \, \ket{i\!:\!\lambda} \bra{i\!:\!\nu},
\)
où \( \ket{i\!:\!\lambda} \) représente un état où seule la $i^\text{e}$ particule est dans l’état \( \lambda \). (Par construction, l’opérateur \( \operator{F} \) commute avec les projecteurs de symétrisation \( \operator{S}_N \) (bosons) et d’antisymétrisation \( \operator{A}_N \) (fermions).)
On peut montrer que la somme des projecteurs agissant sur chaque particule s’identifie à une combinaison d’opérateurs de création et d’annihilation :
\(
	\sum_{i=1}^N \ket{i\!:\!\lambda} \bra{i\!:\!\nu} = \operator{a}^\dagger_\lambda \operator{a}_\nu^{},
\)
(où \( \operator{a}_\lambda \) peut ici estre une notation générique désignant \( \operator{b}_\lambda \) pour les bosons, ou \( \operator{c}_\lambda \) pour les fermions).

On en déduit que l’opérateur à un corps \( \operator{F} \) peut se réécrire dans le formalisme de la seconde quantification comme :
\begin{eqnarray}
	\operator{F} = \sum_{\lambda, \nu} f_{\lambda\nu} \, \operator{a}^\dagger_\lambda \operator{a}_\nu^{}.
\end{eqnarray}


\subparagraph{Exemples d’opérateurs à un corps.}

Si l’on sait diagonaliser l’opérateur \( \operator{f} \), c’est-à-dire si l’on peut écrire :
\(
	\operator{f} = \sum_k f_k \ket{k} \bra{k},
\)
alors l’opérateur à $N$ corps associé s’écrit :
\(
	\operator{F} = \sum_k f_k \, \operator{a}^\dagger_k \operator{a}_k^{} = \sum_k f_k \, \operator{n}_k,
\)
où \( \operator{n}_k = \operator{a}^\dagger_k \operator{a}_k \) est l’opérateur nombre de particules dans le mode \( k \). On obtient ainsi une forme diagonale de \( \operator{F} \) en seconde quantification.
\begin{mdframed}[linewidth=0.5pt, backgroundcolor=gray!5, roundcorner=5pt]
Un exemple immédiat est celui des particules libres. Si l’on diagonalise le problème à une particule selon :
\(
	\operator{\mathcal{H}}_1 \ket{k} = \varepsilon(k) \ket{k},
\)
alors l’énergie totale du système correspond ici uniquement à son énergie cinétique, et s’écrit :
\begin{equation}
	\operator{K} = \sum_{k} \varepsilon(k) \, \operator{a}^\dagger_k \operator{a}_k^{}.\label{eq.chap.1.cinietique.1}
\end{equation}

Et pour $N$ particules, en écrivant l’état sous la forme~\eqref{eq.chap.1.ket.N}, en utilisant les relations de commutation~\eqref{chap:1:com.1.k} et la définition de l’état de Fock~\eqref{chap:eq.vide.fock.k}, on trouve que $\ket{\{k_1, \cdots, k_N\}}$ est un état propre de $\operator{K}$ associé à l'énergie $\left( \sum_{i = 1}^N \varepsilon(k_i) \right)$, c’est-à-dire :
\begin{eqnarray}
	\operator{K} \ket{\{k_1, \cdots, k_N\}} = \left( \sum_{i = 1}^N \varepsilon(k_i) \right) \ket{\{k_1, \cdots, k_N\}}.\label{eq.chap.1.cinietique.2}
\end{eqnarray}
\end{mdframed}

\paragraph{Forme champ des opérateurs à un corps.}

Les opérateurs à plusieurs corps peuvent être exprimés de manière remarquable à l’aide des opérateurs de champ, d’une façon physiquement transparente qui rappelle les formules bien connues du cas à une particule.

La forme générale d’un opérateur à un corps s’écrit :
\begin{eqnarray}
\operator{F} = \int dx \, dx' \, \operator{\Psi}^\dagger(x) \, \bra{ x} \operator{f} \ket{x'} \, \operator{\Psi}(x').
\end{eqnarray}%où \( \hat{f} \) est l’opérateur à un corps exprimé dans la base position, et \( \hat{\psi}^\dagger(\vec{r}) \), \( \hat{\psi}(\vec{r}) \) sont les opérateurs de création et d’annihilation d’une particule au point \( \vec{r} \).
\begin{mdframed}[
	linewidth=0.5pt, 
	backgroundcolor=gray!5, 
	roundcorner=50pt,	
	innerleftmargin=5pt,
    innerrightmargin=5pt,
    innertopmargin=-10pt,
    innerbottommargin=2pt,
    leftmargin=2pt,
    rightmargin=2pt
]
\subparagraph{Énergie cinétique totale.}

Pour des particules non relativistes, l’énergie cinétique élémentaire s’écrit $\operator{f} = \frac{\hbar^2 \operator{p}^2}{2m}$. À l’échelle du champ quantique, l’énergie cinétique totale prend la forme opératorielle :
\begin{eqnarray}
\operator{K} =  -\frac{\hbar^2}{2m} \int dx \, \operator{\Psi}^\dagger(x) \, \operator{\partial}_x^2 \operator{\Psi}(x)
= \frac{\hbar^2}{2m} \int dx \, \operator{\partial}_x \operator{\Psi}^\dagger(x) \cdot \operator{\partial}_x \operator{\Psi}(x). \label{eq.chap.1.cinietique.3}
\end{eqnarray}

Le champ quantique $\operator{\Psi}(x)$ est relié aux opérateurs de moment $\operator{b}_k$ par une transformation de Fourier. En injectant l'expression \eqref{eq.chap.1.TF.1} dans \eqref{eq.chap.1.cinietique.3}, on retrouve la forme discrète \eqref{eq.chap.1.cinietique.1}, cette fois exprimée en termes des opérateurs $\operator{b}_k$.

Lorsque cet Hamiltonien agit sur l’état de Fock à $N$ particules $\ket{\{k_1 , \cdots , k_N\}}$, les règles de commutation (\ref{chap:1:com.1}) ainsi que la définition des états de Fock (\ref{chap:eq.vide.fock}) impliquent (cf. Annexe \ref{annex:N.part}) :
\begin{eqnarray}
\operator{K}\ket{k_1 , \cdots , k_N } =  \int d^N z \, \operator{\mathcal{K}}_N \, \varphi_{\{k_a\}}(z_1 , \cdots , z_N ) \operator{\Psi}(z_1) \cdots \operator{\Psi}^\dag(z_N) \ket{\emptyset}
\end{eqnarray}
avec :
\[
	\operator{\mathcal{K}}_N = \sum_{i=1}^N \frac{\operator{p}_i^2}{2m},
\]
où \( \operator{p}_i \) désigne l’opérateur impulsion de la \( i^\text{ème} \) particule.
\end{mdframed}




\paragraph{Opérateurs à deux corps}

Nous considérons à présent les termes d’interaction impliquant deux particules , $\operator{v}$ , dont les éléments de matrices sont donnés par $v_{\alpha \beta \gamma \delta} = \bra{ 1 : \alpha; 2 : \beta } \operator{v}\ket{ 1 : \gamma; 2 : \delta }$ , où $\ket{ i : \gamma; j : \delta }$ représente l'état où la $i^\text{e}$  particules est dans l'état $\gamma$ et la $j^\text{e}$ dans l'état $\delta$  . Ceux-ci correspondent à des opérateurs de la forme :
\(
    \operator{V} = \sum_{j < i} \operator{v}^{(i, j)} = \frac{1}{2} \sum_{i, j \ne i} \operator{v}^{(i, j)}
    \label{eq:V2corps}.
\)
avec $\operator{v}^{(i, j)}$ désigne l’interaction à deux corps entre les $i^\text{e}$ et $j^\text{e}$ particules , exprimés dans la base à deux états :
\(
	\operator{v}^{(i, j)} = \sum_{\alpha,\beta,\delta,\gamma} \ket{i : \alpha; j : \beta }v_{\alpha \beta \gamma \delta} \bra{ i : \gamma; j : \delta }.
    %v_{\alpha \beta \gamma \delta} = \bra{ i : \alpha; j : \beta } \operator{v}^{(i,j)} \ket{ i : \gamma; j : \delta }.
    \label{eq:matriceV}
\)
On peut réécrire l’opérateur \( \operator{V} \) en termes d’opérateurs de création et d’annihilation comme suit :
\begin{equation}
    \operator{V} = \frac{1}{2} \sum_{\alpha, \beta, \gamma, \delta} v_{\alpha \beta \gamma \delta} \, \operator{a}^\dagger_\alpha \operator{a}^\dagger_\beta \operator{a}_\delta^{} \operator{a}_\gamma^{}.
    \label{eq:Vcreation}
\end{equation}

Cette forme est particulièrement utile pour le traitement des interactions dans l’espace de Fock, notamment en théorie des champs et en physique des gaz quantiques.

\subparagraph{Expression générale d’un terme à deux corps. }

Un terme d’interaction à deux corps général peut s’écrire :
\begin{equation}
    \operator{V} = \frac{1}{2} \int dx_1^{} \, dx_2^{} \, dx_1' \, dx_2' \; 
    \bra{ 1 : x_1^{}, 2 : x_2^{} } \operator{v} \ket{ 1 : x_1', 2 : x_2' } \,
    \operator{\Psi}^\dagger(x_1^{}) \, \operator{\Psi}^\dagger(x_2^{}) \, 
    \operator{\Psi}(x_2') \, \operator{\Psi}(x_1')
    \label{eq:V_general}
\end{equation}

\begin{mdframed}[
	linewidth=0.5pt, 
	backgroundcolor=gray!5, 
	roundcorner=50pt,	
	innerleftmargin=5pt,
    innerrightmargin=5pt,
    innertopmargin=-10pt,
    innerbottommargin=2pt,
    leftmargin=2pt,
    rightmargin=2pt
	]
\subparagraph{Interactions ponctuelles.} 
Dans le cas d’une interaction ne dépendant que de la distance relative entre deux particules, cette expression se simplifie :
\(
     \operator{V} = \frac{1}{2} \sum_{i, j \ne i}  \operator{v}(x_i^{} - x_j^{}) = 
    \frac{1}{2} \int dx_1^{} \, dx_2^{} \; v(x_1^{} - x_2^{}) \,
    \operator{\Psi}^\dagger(x_1^{}) \, \operator{\Psi}^\dagger(x_2^{}) \, 
    \operator{\Psi}(x_2^{}) \, \operator{\Psi}(x_1^{})
    \label{eq:V_local}
\) soit pour des interactions ponctuelles :	
\begin{eqnarray}
	\quad \operator{V}  = \frac{g}{2} \int dx \,
    \operator{\Psi}^\dagger(x) \, \operator{\Psi}^\dagger(x) \, 
    \operator{\Psi}(x) \, \operator{\Psi}(x)  		
\end{eqnarray}
et quand on l'applique à l'état $\ket{\{k_1 , \cdots , k_N\}} $ , les règles de commutations (\ref{chap:1:com.1}) et la définition d'état de Fock (\ref{chap:eq.vide.fock}) impliquent que (cf Annex \ref{annex:N.part})
\begin{eqnarray}
\operator{V}\ket{\{k_1 , \cdots , k_N\}} =  \int d^Nz \, \operator{\mathcal{V}}_N \varphi_{\{k_a\}}(z_1 , \cdots , z_N )\operator{\Psi}(z_1)\cdots \operator{\Psi}^\dag(z_N) \ket{\emptyset} 
\end{eqnarray}
avec 
\(
	\operator{\mathcal{V}}_N 	
 = g\sum_{1\leq i < j \leq N } \operator{\delta}(x_i - x_j)	
\)
où \( g \) est la constante de couplage.
\end{mdframed}


%Le hamiltonien général décrivant des particules identiques en interaction s’écrit alors :
%\begin{equation}
%    \hat{H} = \int d\vec{r} \; \hat{\psi}^\dagger(\vec{r}) 
%    \left( -\frac{\hbar^2}{2m} \Delta + u(\vec{r}) - \mu \right) 
%    \hat{\psi}(\vec{r})
%    + \frac{1}{2} \int d\vec{r} \, d\vec{r}' \; v(\vec{r} - \vec{r}') \,
%    \hat{\psi}^\dagger(\vec{r}') \, \hat{\psi}^\dagger(\vec{r}) \,
%    \hat{\psi}(\vec{r}) \, \hat{\psi}(\vec{r}')
%    \label{eq:H_general}
%\end{equation}

%\noindent
%Bien que cette expression ait une interprétation physique très claire, il est important de garder à l'esprit que \( \hat{H} \) et \( \hat{\psi} \) sont des objets du formalisme à plusieurs corps.



%%%%%%%%%%%%%%%%
%........................

%\subsubsection{Seconde quantification}



%\paragraph{Hamiltoniens en seconde quantification.}
%\subparagraph{Terme à un corps.}
%Un hamiltonien à un corps, correspondant à une énergie cinétique ou un potentiel externe, s’écrit :
%\[
%\hat{\mathcal{H}}_1 = \int dx\, \operator{\Psi}^\dagger(x) \hat{h}(x) \operator{\Psi}(x),
%\]
%où \( \hat{h}(x) \) est l’opérateur d’un corps (ex. : \( -\frac{\hbar^2}{2m} \partial_x^2 + V(x) \)).

%\subparagraph{Terme à deux corps.}
%Les interactions entre particules, modélisées par une interaction à deux corps \( V(x - y) \), s’expriment comme :
%\[
%\hat{\mathcal{H}}_2 = \frac{1}{2} \int dx\,dy\, \operator{\Psi}^\dagger(x) \operator{\Psi}^\dagger(y) V(x - y) \operator{\Psi}(y) \operator{\Psi}(x).
%\]


%.......................


\paragraph{Expression de l’Hamiltonien. }
L’hamiltonien dans ce formalisme s’écrit en termes des opérateurs de champ, par exemple pour l’énergie cinétique et les interactions ponctuelles avec $\hbar = m = 1 $  :

%Le Hamiltonien du modèle est donné par

%\begin{eqnarray}
%	\operator{H} & = & \int dx \, \left [ \operator{\partial}_x \operator{\Psi}^\dag (x)\operator{\partial}_x \operator{\Psi}(x) + c \operator{\Psi}^\dag (x) \operator{\Psi}^\dag (x) \operator{\Psi} (x) \operator{\Psi} (x) \right ] \label{chap:1:ham.mod}
%\end{eqnarray}

\begin{eqnarray}
	\operator{H} & = & \int dx \, \operator{\Psi}^\dag (x)\left [-\frac{1}{2}\operator{\partial}_x^2 + \frac{g}{2}  \operator{\Psi}^\dag (x) \operator{\Psi} (x) \right ] \operator{\Psi} (x) \label{chap:1:ham.mod}.
\end{eqnarray}

Quand on l'applique à l'état $\ket{\{\theta_1 , \cdots , \theta_N \}} $, avec $\theta_i$ homogène à des nombres d'onde ou à des vitesse , il vient que %, les règles de commutations (\ref{chap:1:com.1}) et la définition d'état de Fock (\ref{chap:eq.vide.fock}) impliquent que (cf Annex \ref{annex:N.part})
\begin{eqnarray}
\operator{H}\ket{\{\theta_1 , \cdots , \theta_N\}} =  \int d^Nz \, \operator{\mathcal{H}}_N \varphi_{\{\theta_a\}}(z_1 , \cdots , z_N )\operator{\Psi}(z_1)\cdots \operator{\Psi}^\dag(z_N) \ket{\emptyset} 
\end{eqnarray}
avec 
\(
	\operator{\mathcal{H}}_N 	
 =  \operator{\mathcal{K}}_N  +  \operator{\mathcal{V}}_N .	
\)


%où \( g \) est la constante de couplage. %Dans ce chapitre, nous considérons uniquement les propriétés du système à un instant donné, de sorte que la dépendance temporelle des champs est omise pour alléger l’écriture.

Ce formalisme est ainsi adapté pour décrire des condensats de Bose, des gaz quantiques, ou la création/annihilation de particules dans les champs quantiques.

\paragraph{Équation du mouvement associée.}

L’équation du mouvement du champ \( \Psi(x) \) est obtenue à partir de l’équation de Heisenberg :

\begin{eqnarray}
	i\operator{\partial}_t	\operator{\Psi} & = & [ \operator{\Psi} , \operator{H} ]
\end{eqnarray}

ce qui, après évaluation explicite du commutateur (\ref{chap:1:com.1}), conduit à :


%\begin{eqnarray}
%	i \operator{\partial}_t \operator{\Psi}	 & = & - \operator{\partial}_x^2 \operator{\Psi} + 2c \operator{\Psi}^\dag\operator{\Psi} \operator{\Psi}
%\end{eqnarray}

\begin{eqnarray}
	i \operator{\partial}_t \operator{\Psi}	 & = & - \frac{1}{2}\operator{\partial}_x^2 \operator{\Psi} + g \operator{\Psi}^\dag\operator{\Psi} \operator{\Psi}
\end{eqnarray}

est appelée l'équation de \textbf{Schrödinger non linéaire (NS)}.

Pour $g > 0$, l'état fondamental à température nulle est une sphère de Fermi. Seul ce cas sera considéré par la suite.

%\vspace{0.5cm}

\subsubsection*{Conclusion}

La première quantification est la base indispensable qui permet de comprendre le comportement quantique d’un nombre fixé de particules. La seconde quantification en est une extension naturelle, nécessaire pour décrire des systèmes plus complexes où le nombre de particules peut varier. Elle repose sur la quantification des champs, et l’introduction d’opérateurs créant ou détruisant ces particules, ouvrant ainsi la voie à la physique quantique des champs et à de nombreuses applications modernes.


\subsection{Opérateurs nombre de particules et moment dans la formulation quantique du gaz de Lieb-Liniger}

Dans cette section, nous nous intéressons aux opérateurs fondamentaux que sont le {\em nombre total de particules} $\operator{Q}$ et le {\em moment total} $\operator{P}$, dans le cadre du gaz de bosons unidimensionnel décrit par l’Hamiltonien de Lieb-Liniger. Après avoir introduit ces opérateurs dans le langage de la seconde quantification, nous montrons qu’ils sont {\em conservés} par la dynamique, et qu’ils admettent les {\em mêmes états propres} que l’Hamiltonien. Nous donnons ensuite leur expression dans la représentation à  $N$ particules, ainsi que la forme explicite de leurs valeurs propres en fonction des {\em rapidités} $\theta_a$ , illustrant la structure polynomiale typique des intégrales du mouvement dans les systèmes intégrables.

\subsubsection{Définition en seconde quantification}

Les opérateurs du nombre total de particules $\operator{Q}$ et du moment total $\operator{P}$ s’écrivent en seconde quantification comme suit :
\begin{eqnarray}
	\operator{Q}  =  \int \operator{\Psi}^\dag (x) \operator{\Psi} (x) \, d x, \quad 
	\operator{P}  =  - \frac{i}2 \int \left \{  \operator{\Psi}^\dag(x) \operator{\partial}_x \operator{\Psi}(x) - \left [ \operator{\partial}_x \operator{\Psi}^\dag(x)\right ] \operator{\Psi}(x)\right \} dx \label{eq.1.7}
\end{eqnarray}
Ces deux opérateurs sont {\em hermitiens}, et représentent des observables physiques fondamentales : le nombre de particules et la quantité de mouvement du système.

\subsubsection{Conservation et commutation}
Ces opérateurs commutent avec l’Hamiltonien $\operator{H}$ du modèle de Lieb-Liniger :
\begin{eqnarray}
[ \operator{H} , \operator{Q} ] = 0, \quad [ \operator{H} , \operator{P} ] = 0.
\end{eqnarray}
Ils constituent ainsi des intégrales du mouvement. Cette propriété est une manifestation de la symétrie translationnelle du système (pour $\operator{P}$) et de la conservation du nombre total de particules (pour $\operator{Q}$).

\begin{mdframed}[
	linewidth=0.5pt, 
	backgroundcolor=gray!5, 
	roundcorner=50pt,	
	innerleftmargin=5pt,
    innerrightmargin=5pt,
    innertopmargin=5pt,
    innerbottommargin=2pt,
    leftmargin=2pt,
    rightmargin=2pt
	]
	Nous verrons au chapitre 2 que cette situation s’étend à une {\bf \em infinité d’intégrales du mouvement} dans les systèmes intégrables, ce qui permettra de construire l’ensemble de Gibbs généralisé (GGE).
\end{mdframed}

\subsubsection{États propres et valeurs propres}
Les états propres $\ket{\{\theta_a\}}$, construits dans le cadre de la seconde quantification à partir de la solution du modèle de Lieb-Liniger, sont simultanément fonctions propres des opérateurs $\operator{Q}$, $\operator{P}$ et $\operator{H}$ :
\begin{eqnarray}
\operator{Q} \ket{\{\theta_a\}} = N \ket{\{\theta_a\}}, \quad
\operator{P} \ket{\{\theta_a\}} = \left( \sum_{a=1}^N \theta_a \right) \ket{\{\theta_a\}}, \
\operator{H} \ket{\{\theta_a\}} = \left( \frac{1}{2} \sum_{a=1}^N \theta_a^2 \right) \ket{\{\theta_a\}}.
\end{eqnarray}
Autrement dit, les valeurs propres associées à ces trois opérateurs sont données par :
\begin{eqnarray}
N = \sum_{a = 1}^N \theta_a^0, \quad p = \sum_{a = 1}^N \theta_a, \quad e = \frac{1}{2} \sum_{a = 1}^N \theta_a^2.
\end{eqnarray}
Cela illustre que les trois premières intégrales du mouvement du système — nombre, moment, énergie — peuvent être exprimées comme des {\bf \em moments successifs} des rapidités.	

\subsubsection{Forme en première quantification}
En utilisant la représentation en espace de configuration $\{z_a\} \equiv \{z_1 , \cdots , z_N \}$, les opérateurs $\operator{Q}$ et $\operator{P}$ agissent comme suit sur les fonctions d’onde $\varphi_{\{\theta_a\}}(\{z_a\})$ :
\begin{eqnarray}
	\operator{Q}\ket{\{\theta_a\}} =  \sqrt{N!}\int d^Nz \, \operator{\mathcal{N}} \varphi_{\{\theta_a\}}(\{z_a\} )\ket{\{z_a\}}, \, \operator{P}\ket{\{\theta_a\}} =  \sqrt{N!}\int d^Nz \, \operator{\mathcal{P}}_N \varphi_{\{\theta_a\}}(\{z_a\} )\ket{\{z_a\}} 
\end{eqnarray}
où les opérateurs associés agissant sur les fonctions d’onde à $N$ particules sont :
\begin{eqnarray}
	\operator{ \mathcal{N}}  =  \sum_{k = 1}^N 1 = N ,~\operator{ \mathcal{P}}_N  = -i \sum_{k = 1}^N k =- i\sum_{k = 1}^N \operator{\partial}_{z_k}	
\end{eqnarray}

Ces formes découlent directement des règles de commutation canonique (\ref{chap:1:com.1}) et de la définition des opérateurs en seconde quantification (\ref{chap:eq.vide.fock}) (cf. annexes \ref{annex:N.part}).

\subsubsection{Conclusion}
Ainsi, les opérateurs $\operator{Q}$ , $\operator{P}$ et $\operator{H}$ possèdent une structure diagonale commune dans la base des états propres $\ket{\{\theta_a\}}$, révélant la nature intégrable du modèle de Lieb-Liniger. Leurs valeurs propres sont respectivement les 0e, 1er et 2e moments des rapidités. Cette structure permet de généraliser la construction à une hiérarchie complète d’observables conservées, qui seront présentées au chapitre suivant.


\subsection{Fonction d’onde et Hamiltonien et moment à 2 corps}

%Nous considérons à présent le cas de deux bosons quantiques dans la même boîte unidimensionnelle de longueur \(L\), avec des conditions aux limites périodiques. Contrairement au cas à une particule, le terme d’interaction à contact intervient dans la dynamique. L'hamiltonien à 2 particule s'écrit :
%En première quantification, en utilisant les coordonnées du centre de masse et relatives $Z = (z_1 + z_2)/2$ et $Y = z_1 - z_2$, il vient que
%l'hamiltonien (\ref{chap:1:hal.mod.2.part.3}) se divise en une somme de deux problèmes indépendants à une seule particule.
%Les états propres de l'hamiltonien du centre de masse de masse $\overline{m}= 2$, $-\frac{1}{4} \partial_Z^2$, sont des ondes planes, et l'hamiltonien pour la coordonnée relative $Y$ correspond à celui d'une particule de masse réduite $\tilde{m} = 1/2$ en présence d'un potentiel delta en $Y = 0$. 
%\paragraph{Introduction au système à deux bosons avec interaction de contact.}
%Nous considérons à présent le cas de deux bosons quantiques dans une même boîte unidimensionnelle de longueur \(L\), avec des conditions aux limites périodiques. Contrairement au cas à une particule, un terme d’interaction de contact intervient ici dans la dynamique. L’Hamiltonien à deux particules s’écrit :
%\begin{eqnarray}
%	\operator{\mathcal{H}}_2  =  \operator{\mathcal{K}}_2 +\operator{\mathcal{V}}_2  & avec & \operator{\mathcal{K}}_2 =   - \frac{1}{2} \partial_{z_1}^2 - \frac{1}{2} \partial_{z_2}^2,  \quad \mbox{et} \quad  \operator{\mathcal{V}}_2  =  	g  \delta(z_1 - z_2). \label{chap:1:hal.mod.2.part.3} 		
%\end{eqnarray}
%On rappelle que l'énergies propres de  $\operator{\mathcal{K}}_2$ associées aux fonction d'ondes $\varphi_{\{ \theta_1 , \theta_2 \}}$ , la masse des particule étant égale à 1 (ie $\hbar= m=1$) s'écrit 
%\begin{eqnarray}
%	\varepsilon(\theta_1) + 	\varepsilon(\theta_2) & = & \frac{\theta_1^2}{2} + \frac{\theta_2^2}{2} 
%\end{eqnarray}
%On vas travailler dans le centre de masse.

%\paragraph{Changement de variables : coordonnées du centre de masse et relatives.}
 
%En première quantification, en introduisant les coordonnées du centre de masse \(Z = \frac{z_1 + z_2}{2}\) et relative \(Y = z_1 - z_2\), on obtient :
%\(
%	\partial_{z_1}^2 + \partial_{z_2}^2 = \frac{1}{2} \partial_Z^2 + 	2\partial_Y^2.  
%\)
%L’Hamiltonien~\eqref{chap:1:hal.mod.2.part.3} se décompose alors en une somme de deux problèmes indépendants à une seule variable :

%\begin{eqnarray}\label{chap:1:hal.mod.2.part.4}
%	\operator{\mathcal{H}}_2  =  	-\frac{1}{4} \partial_Z^2 + \operator{\mathcal{H}}_{rel} , \quad \mbox{avec}\quad  \operator{\mathcal{H}}_{rel} =  - 	\partial_Y^2 + g \delta ( Y ). 
%\end{eqnarray}

%\paragraph{Résolution du problème de centre de masse et de coordonnée relative.}

%Les états propres de l’Hamiltonien associé au centre de masse, \(-\frac{1}{4} \partial_Z^2\), correspondant à une particule de masse totale \(\bar{m} = 2\), sont des ondes planes associés à l'énergie $\overline{\theta}^2$ avec $\overline{\theta} = \frac{ \theta_1 + \theta_2}{2}$. L’Hamiltonien, $\operator{\mathcal{H}}_{rel}$, associé à la coordonnée relative \(Y\) correspond quant à lui à celui d’une particule de masse réduite \(\tilde{m} = \frac{1}{2}\), soumise à un potentiel delta en \(Y = 0\) :
%\begin{eqnarray}\label{chap:1:hal.mod.2.part.5}
%	- 	\partial_Y^2 \tilde{\varphi}(Y) + g \delta ( Y )\tilde{\varphi}(Y) & = & \tilde{\varepsilon}\,\tilde{\varphi}(Y),
%\end{eqnarray}
%où $\tilde{\varepsilon}$ est l’énergie propre du problème relatif.

%%%%%%
\paragraph{Introduction au système de deux bosons avec interaction de contact.}

Considérons maintenant un système de deux bosons quantiques confinés dans une boîte unidimensionnelle de longueur \(L\), avec des conditions aux limites périodiques. Contrairement au cas à une seule particule, une interaction de contact intervient ici dans la dynamique. L’Hamiltonien à deux particules s’écrit :
\begin{eqnarray}
	\operator{\mathcal{H}}_2 = \operator{\mathcal{K}}_2 + \operator{\mathcal{V}}_2, \quad \text{avec} \quad \operator{\mathcal{K}}_2 = - \frac{1}{2} \partial_{z_1}^2 - \frac{1}{2} \partial_{z_2}^2, \quad \text{et} \quad \operator{\mathcal{V}}_2 = g \, \delta(z_1 - z_2). \label{chap:1:hal.mod.2.part.3}
\end{eqnarray}

On rappelle que, pour des particules de masse unitaire (i.e., \(\hbar = m = 1\)), les énergies propres de l’opérateur cinétique \(\operator{\mathcal{K}}_2\), associées aux fonctions d’onde symétrisées \(\varphi_{\{ \theta_1 , \theta_2 \}}\), sont données par :
\begin{eqnarray}
	\varepsilon(\theta_1) + \varepsilon(\theta_2) = \frac{\theta_1^2}{2} + \frac{\theta_2^2}{2}.
\end{eqnarray}

Afin de simplifier le problème, nous nous plaçons dans le référentiel du centre de masse.

\paragraph{Changement de variables : coordonnées du centre de masse et relative.}

En première quantification, on introduit les nouvelles variables :
\(
Z = \frac{z_1 + z_2}{2} \, \text{(centre de masse)}, \qquad Y = z_1 - z_2 \, \text{(coordonnée relative)}.
\)
Dans ce changement de variables, l’opérateur laplacien total devient :
\(
\partial_{z_1}^2 + \partial_{z_2}^2 = \frac{1}{2} \partial_Z^2 + 2 \, \partial_Y^2.
\)
L’Hamiltonien~\eqref{chap:1:hal.mod.2.part.3} se décompose alors en la somme de deux Hamiltoniens agissant respectivement sur \(Z\) et \(Y\) :
\begin{eqnarray}\label{chap:1:hal.mod.2.part.4}
	\operator{\mathcal{H}}_2 = -\frac{1}{4} \partial_Z^2 + \operator{\mathcal{H}}_{\text{rel}}, \qquad \text{avec} \quad \operator{\mathcal{H}}_{\text{rel}} = - \partial_Y^2 + g \, \delta(Y).
\end{eqnarray}

\paragraph{Résolution du problème du centre de masse et de la coordonnée relative.}

L’Hamiltonien du centre de masse, \(-\frac{1}{4} \partial_Z^2\), décrit une particule de masse totale \(\bar{m} = 2\). Ses états propres sont des ondes planes associées à une énergie \(\overline{\theta}^2\), avec :
\(
\overline{\theta} = \frac{\theta_1 + \theta_2}{2},
\)
jouant ici un rôle analogue à celui d’un pseudo-moment associé dans le référentielle de laboratoire.
Le Hamiltonien relatif, \(\operator{\mathcal{H}}_{\text{rel}}\), correspond quant à lui à une particule de masse réduite \(\tilde{m} = \frac{1}{2}\) soumise à un potentiel delta centré en \(Y = 0\). Son équation propre s’écrit :
\begin{eqnarray}\label{chap:1:hal.mod.2.part.5}
	- \partial_Y^2 \, \tilde{\varphi}(Y) + g \, \delta(Y) \, \tilde{\varphi}(Y) = \tilde{\varepsilon} \, \tilde{\varphi}(Y),
\end{eqnarray}
où \(\tilde{\varepsilon}\) désigne l’énergie associée au mouvement relatif.
%%%%%%%%%%%%%%

\paragraph{Forme symétrique de la fonction d'onde pour bosons.}
Dans le référentiel du centre de masse. Le système est le même que que celuis d'un particules de masse $\tilde{m}= \frac{1}{2}$.
Le système étant composé de particules bosoniques, on cherche une solution symétrique que l’on écrit sous la forme  :
\begin{eqnarray}
	\tilde{\varphi}(Y) ~=~a~e^{i\frac{1}{2} \tilde{\theta} \vert Y \vert } + b~e^{-i\frac{1}{2} \tilde{\theta}\vert Y \vert } ~\propto~  \sin\left( \frac{1}{2} (\tilde{\theta} |Y| + \Phi ) \right). \label{eq:ansatz.boson}
\end{eqnarray}
Le paramètre \( \tilde{\theta} = \theta_1 - \theta_2 \) joue ici un rôle analogue à celui d’un pseudo-moment associé à la coordonnée relative,
est  la phase s'écrit
\begin{eqnarray}
	\Phi(\tilde{\theta}) &=& 2 \arctan\left (\frac{1}{i} \frac{a+b}{a-b}\right),	\label{chap:1:dif.mod.2.part.1} 
\end{eqnarray}
car \( a\exp(ix)+b\exp(-ix) = 2\sqrt{ab}\sin\left(x+\arctan\left(-i\, \frac{a+b}{a-b}\right)\right) \). Pour $\tilde{\theta}<0$, les termes exponentiels \( \exp(i\tilde{\theta} \vert Y \vert/2 ) \) et \( \exp(-i\tilde{\theta} \vert Y \vert/2 ) \) correspondent aux paires de particules entrantes et sortantes d’un processus de diffusion à deux corps.


%En réinjectant l'équation \eqref{eq:ansatz.boson} dans l’équation \eqref{chap:1:hal.mod.2.part.5}, on obtient l’énergie propre du problème réduit $\tilde{\varepsilon}$ associé à l’état lié. Celle-ci prend la forme classique de l’énergie cinétique d’une particule, \( \frac{1}{2} \times \text{masse} \times \text{vitesse}^2 \), la masse réduite du problème étant ici \( \tilde{m} = \frac{1}{2} \), et où \( \tilde{\theta} \) joue un rôle analogue à celui d’une vitesse. On en déduit :
%\begin{eqnarray}\tilde{\varepsilon}(\tilde{\theta})  & = &  \frac{1}{2} \cdot \tilde{m} \cdot \tilde{\theta}^2 = \frac{1}{2} \cdot \frac{1}{2} \cdot \tilde{\theta}^2 = \frac{\tilde{\theta}^2}{4}.\end{eqnarray}
%\begin{eqnarray}
%	\tilde{\varepsilon}(\tilde{\theta})  & = &  \frac{\tilde{\theta}^2}{4}.
%\end{eqnarray}
% Il encode la décroissance exponentielle de la fonction d’onde liée dans l’espace relatif, et sa valeur est directement reliée à la profondeur de l’état lié. Une valeur plus grande de \( \tilde{\theta} \) correspond à un état plus fortement lié, c’est-à-dire plus localisé autour de \( Y = 0 \), ce qui reflète une interaction plus attractive entre les deux particules. $\overline{\theta}^2 +  \tilde{\varepsilon}(\tilde{\theta}) = \varepsilon{\theta_1} + \varepsilon{\theta_2}$.
En réinjectant l’ansatz~\eqref{eq:ansatz.boson} dans l’équation relative
\eqref{chap:1:hal.mod.2.part.5}, on obtient l’énergie propre
\(\tilde{\varepsilon}\) du problème réduit.  
Elle prend la forme cinétique usuelle
\(\tfrac{1}{2}\times\text{masse}\times\text{vitesse}^{2}\).  
La masse réduite vaut ici \(\tilde{m}= \frac{1}{2}\) et le paramètre
\(\tilde{\theta}\) joue le rôle d’une impulsion ; ainsi
\begin{equation}
   \tilde{\varepsilon}(\tilde{\theta})
   \;=\;
   \frac{1}{2}\,\tilde{m}\,\tilde{\theta}^{2}
   \;=\;
   \frac{1}{2}\times\frac{1}{2}\,\tilde{\theta}^{2}
   \;=\;
   \frac{\tilde{\theta}^{2}}{4}.
   \label{eq:energie_relative}
\end{equation}

Cette énergie gouverne la décroissance exponentielle de la fonction
d’onde dans la coordonnée relative : plus \(\tilde{\theta}\) est grand,
plus l’état est localisé autour de \(Y=0\), signe d’une interaction
attractive plus forte entre les deux bosons.

L’énergie totale se décompose enfin en la somme du mouvement du centre
de masse et du mouvement relatif :
\(
   \overline{\theta}^{2}
   \;+\;
   \tilde{\varepsilon}(\tilde{\theta})
   \;=\;
   \varepsilon(\theta_{1})
   \;+\;
   \varepsilon(\theta_{2}),
\)
où \(\overline{\theta}= \tfrac{\theta_{1}+\theta_{2}}{2}\) et
\(\varepsilon(\theta)=\theta^{2}/2\).






%%%%%%%%%%%%%%%%%%%%%%%%%%%
\paragraph{Condition de discontinuité à cause du potentiel delta.}
En raison de la présence du potentiel delta centré en $Y = 0$, la dérivée première de la fonction d’onde $\tilde{\varphi}(Y)$ présente une discontinuité en ce point. En effet, le potentiel étant infini en $Y = 0$, la phase $\Phi$ du régime symétrique est déterminée en intégrant l’équation du mouvement autour de la singularité. En intégrant entre $- \epsilon$ et $+ \epsilon$ et en faisant tendre $\epsilon \to 0$, on obtient la condition de saut de la dérivée :

%avec $\Phi$ une phase à déterminer. %\begin{equation}
%	E = \frac{\tilde{m} \theta^2}{2}.
%\end{equation}

%La dérivée de la fonction d’onde n’est pas continue en $Y = 0$. Le potentiel étant infini en $Y = 0$, la phase $\Phi$ est obtenue en intégrant l’équation du mouvement entre $- \epsilon$ et $+ \epsilon$ et en faisant tendre $\epsilon$ vers zéro :


%En raison de ce potentiel delta, la dérivée première de la fonction d'onde $\varphi(Y)$ doit avoir une discontinuité en $Y = 0$ : 

%{\color{lightgray} 
%\begin{eqnarray*}
%	\underset{ \epsilon \to 0 }{\lim} \int_{-\epsilon}^{+\epsilon}  	-\underbrace{\cancel{\frac{1}{4} \partial_Z^2\varphi(Y)}}_{0} - 	\partial_Y^2\varphi(Y) + c \delta ( Y )\varphi(Y) \, dY  & = & \underset{ \epsilon \to 0 }{\lim}  \int_{-\epsilon}^{+\epsilon}  E d Y , \\
%	\underset{ \epsilon \to 0 }{\lim}  \left [ \varphi'(\epsilon) - \varphi'(-\epsilon) \right ] - c \varphi (  0 ) & =  &  -\underset{ \epsilon \to 0 }{\lim}  \int_{-\epsilon}^{+\epsilon}  E d Y,\\
%	 \varphi'(0^+) - \varphi'(0^-) - c \varphi (  0 ) & = & 0 .
%\end{eqnarray*}


%}

\begin{eqnarray*}
	\underset{ \epsilon \to 0 }{\lim} \int_{-\epsilon}^{+\epsilon}  - 	\partial_Y^2\tilde{\varphi}(Y) + g \delta ( Y )\tilde{\varphi}(Y) \, dY  & = & \underset{ \epsilon \to 0 }{\lim}  \int_{-\epsilon}^{+\epsilon}  \tilde{\varepsilon}(\tilde{\theta})d Y ,\\
	\\
	\tilde{\varphi}'(0^+) - \tilde{\varphi}'(0^-) - g \tilde{\varphi} (  0 ) & = & 0 .
\end{eqnarray*}


%soit $\tilde{\varphi}'(0^+) - \tilde{\varphi}'(0^-) - c \tilde{\varphi} (  0 )  =  0 $ .

%%%%%%%%%%%%%%%
\paragraph{Détermination de la phase $\Phi$.}
Et en évaluant la discontinuité de sa dérivée au point $Y = 0$, on trouve que la phase $\Phi$ satisfait la condition :

%\begin{equation}
%	\tan\left( \frac{\Phi}{2} \right) = \frac{\tilde{\theta}}{c}.
%\end{equation}

\begin{eqnarray}\label{chap:1:dif.mod.2.part.2}
	\Phi(\tilde{\theta}) & = & 2 \arctan (\tilde{\theta}/g) \in [ - \pi , +\pi ].
\end{eqnarray}

%{\color{red}( à revoir)} Cette relation exprime l’impact de l’interaction delta sur le déphasage de la solution liée. On en déduit que plus le couplage $g$ est fort ($g \to \infty$), plus la phase $\Phi$ se rapproche de $0$, ce qui correspond à une fonction d’onde présentant s'annulant en $Y = 0$. En revanche, dans la limite d’interaction faible ($g \to 0$), la phase $\Phi$ tend vers $\pm \pi$ et la discontinué de la dérivé de la fonction d'onde devient négligeable.
%Cette relation exprime l’impact de l’interaction de type delta sur le déphasage de la fonction d’onde liée.On en déduit que plus le couplage $g$ est fort ($g \to \infty$), la phase $\Phi$ se rapproche de $0$, ce qui correspond à une fonction d’onde présentant s'annulant en $Y = 0$, à l’image du régime d’imperméabilité totale.
%À l’inverse, dans la limite d’interaction faible (\( g \to 0 \)), si bien que \( \Phi \) tend vers $\pi$ (ou \( -\pi \), selon le signe de \( \tilde{\theta} \)). Dans ce cas, la discontinuité de la dérivée de la fonction d’onde au point \( Y = 0 \) devient négligeable, ce qui traduit un couplage quasi inexistant entre les deux particules.
%Cette relation exprime l’impact de l’interaction de type delta sur le déphasage de la fonction d’onde liée. Lorsque le couplage \( g \) devient très fort (\( g \to \infty \)), la fraction \( \tilde{\theta}/g \to 0 \), et la phase \( \Phi(\tilde{\theta}) \to 0 \). Cela correspond à une situation dans laquelle la fonction d’onde est fortement contrainte à s’annuler en \( Y = 0 \), à l’image du régime d’imperméabilité totale.
%À l’inverse, dans la limite d’interaction faible (\( g \to 0 \)), la fraction \( \tilde{\theta}/g \to \infty \), si bien que \( \Phi(\tilde{\theta}) \to \pi \) (ou \( -\pi \), selon le signe de \( \tilde{\theta} \)). Dans ce cas, la discontinuité de la dérivée de la fonction d’onde au point \( Y = 0 \) devient négligeable, ce qui traduit un couplage quasi inexistant entre les deux particules.

Cette relation exprime l’impact de l’interaction de type delta sur le déphasage de la fonction d’onde liée. On en déduit que plus le couplage \( g \) est fort (\( g \to \infty \)), plus la phase \( \Phi \) se rapproche de zéro. Cela correspond à une fonction d’onde qui s’annule en \( Y = 0 \), caractéristique d’un régime d’imperméabilité totale.

À l’inverse, dans la limite d’une interaction faible (\( g \to 0 \)), la phase \( \Phi \) tend vers \( \pi \) (ou \( -\pi \), selon le signe de \( \tilde{\theta} \)). Dans ce cas, la discontinuité de la dérivée de la fonction d’onde au point \( Y = 0 \) devient négligeable, ce qui traduit une interaction presque absente entre les deux particules.


%%%%%%%%%%%%%%%%%%%%%%%%%%%%%%%%%%%
%\paragraph{Phase de diffusion à un corp.}
%Les équations \eqref{chap:1:dif.mod.2.part.1} et \eqref{chap:1:dif.mod.2.part.2}  et en remarquant que pour $z \in \mathbb{C} \backslash \{ \pm i \} 2\artan(z) = i \ln \left( \frac{ 1 - i z }{1+iz} \right ) $ soit $\exp(2i\arctan(x)) = (1 + ix)/(1 - ix)$ et $\Phi(\tilde{\theta}) = i \ln ( - b/a ) $  donne rapport entre les amplitudes $a$ et $b$ de la fonction d'onde \eqref{eq:ansatz.boson} définit la phase de diffusion / {\em matrice diffusion} $S( \tilde{\theta}) \doteq e^{i\Phi ( \tilde{\theta}  ) }$  :

%\begin{eqnarray}
%	e^{i\Phi ( \tilde{\theta}  ) } &=& -\frac{a}{b} ~=~\frac{1 +i\tilde{\theta}/g} { 1 - i\tilde{\theta}/g} .\label{chap:1:dif.mod.2.part.3}
%\end{eqnarray}

\paragraph{Phase de diffusion à deux corps.}

En combinant les équations~\eqref{chap:1:dif.mod.2.part.1} et~\eqref{chap:1:dif.mod.2.part.2} avec l’identité analytique valable pour tout
\(z \in \mathbb{C}\setminus\{\pm i\}\),
\(
2\arctan(z)=i\ln\!\left(\frac{1-iz}{1+iz}\right)
\Leftrightarrow
e^{2i\arctan(z)}=\frac{1+iz}{1-iz},
\)
on obtient que le rapport des amplitudes \(a\) et \(b\) de la fonction
d’onde relative~\eqref{eq:ansatz.boson} définit la {\em phase de diffusion }
\(
\Phi(\tilde{\theta}) = i\ln\!\left(-\frac{b}{a}\right).
\)
On introduit alors la {\em matrice de diffusion} (ou facteur de diffusion)
\begin{eqnarray}
	S(\tilde{\theta}) \;\doteq\; e^{i\Phi(\tilde{\theta})}= -\frac{a}{b}= \frac{1 + i\,\tilde{\theta}/g}{1 - i\,\tilde{\theta}/g}.%\tag{\ref{chap:1:dif.mod.2.part.3}}
\end{eqnarray}
%où \(g\) est le paramètre d’interaction et
%\(\tilde{\theta} = \theta_1 - \theta_2\) le pseudo‑moment relatif.  
Cette expression, unitaire et analytique, caractérise entièrement la diffusion élastique à deux corps dans le modèle considéré.



\paragraph{Lien entre phase de diffusion et décalage temporel : interprétation semi-classique. {\color{red}(à revoir)}}

Il a été souligné par {\color{black}Eisenbud (1948)} et {\color{black}Wigner (1955)} que la phase de diffusion peut être interprétée, de manière semi-classique, comme un {\em décalage temporel}. Esquissons brièvement l'argument de {\color{black}Wigner (1955)}.Tout d'abord, notons que, pour une particule unique, une approximation simple d’un paquet d’ondes peut être obtenue en superposant deux ondes planes avec des moments $\tilde{\theta}/2$ et $\tilde{\theta}/2 + \delta \tilde{\theta}$, respectivement :
\begin{eqnarray}
	\tilde{\varphi}_{\text{inc}}(Y) & \propto & e^{i\frac{1}{2}\tilde{\theta} \vert Y\vert} + e^{i\frac{1}{2}\left(\tilde{\theta} + 2\delta \tilde{\theta} \right) \vert Y\vert}.
\end{eqnarray}
Cette superposition évolue dans le temps comme :
\begin{eqnarray}
\tilde{\varphi}_{\text{inc}}(Y, t) &\propto &  e^{i\left( \frac{1}{2} \tilde{\theta}\vert Y\vert - t\,\tilde{\varepsilon}(\tilde{\theta}) \right)} + e^{i\left( \frac{1}{2}\left(  \tilde{\theta} + 2\delta \tilde{\theta} \right) \vert Y\vert - t\,\tilde{\varepsilon}(\tilde{\theta} + 2\delta \tilde{\theta}) \right)}.
\end{eqnarray}
%où l'on a utilisé l'expression de l'énergie réduite : $\tilde{\varepsilon}(\tilde{\theta}) = \tilde{\theta}^2 / 4$.
Le centre de ce 'paquet d'ondes' se situe à la position où les phases des deux termes coïncident, c'est-à-dire au point où $\vert Y\vert\delta \tilde{\theta}  - t[\tilde{\varepsilon}(\tilde{\theta} + 2\delta \tilde{\theta} ) - \tilde{\varepsilon}(\tilde{\theta})] = 0$, ce qui donne $\vert Y\vert \simeq \tilde{\theta} t$ avec la vitesse réduite $\tilde{\theta} = 1/2 \varepsilon'(\tilde{\theta}) $. %Ainsi, il s'agit effectivement d'un 'paquet d'ondes' se déplaçant à la vitesse $\theta$. Ensuite, considérons deux particules entrantes dans un état tel que le centre de masse $Z = (z_1 + z_2)/2$ ait une impulsion $\theta_1 - \theta_2$, tandis que la coordonnée relative $Y = z_1 - z_2$ se trouve dans un 'paquet d'ondes' se déplaçant à la vitesse $ (\theta_1 - \theta_2)/2$,
Selon les équations (\ref{eq:ansatz.boson}) et (\ref{chap:1:dif.mod.2.part.3}), l'état sortant de la diffusion correspondant serait :
\begin{eqnarray}
	\tilde{\varphi}_{outc} ( Y, t ) & \propto & -e^{i\Phi(\tilde{\theta})}e^{-i\frac{1}{2}\tilde{\theta} \vert Y\vert} - e^{i\Phi(\tilde{\theta} + 2 \delta \tilde{\theta} )}e^{-i\frac{1}{2}\left(\tilde{\theta} + 2\delta \tilde{\theta} \right) \vert Y\vert}. %\tag{2}
\end{eqnarray}
En répétant l'argument précédent de la stationnarité de phase, on trouve que la coordonnée relative est à la position $\vert Y \vert  \simeq \tilde{\theta} t - 2\Phi'( \tilde{\theta})$ au moment $t$. %Étant donné que le centre de masse n'est pas affecté par la collision et se déplace à la vitesse de groupe $\tilde{\theta} =(\theta_1 + \theta_2)/2$, nous constatons que la position des deux particules semiclassiques après la collision sera
\begin{eqnarray}
	\vert Y \vert & \simeq & 	\tilde{\theta} t  - 2 \Delta (\tilde{\theta} )
\end{eqnarray}
où le déplacement de diffusion $\Delta (\theta)$ est donné par la dérivée de la phase de diffusion,
\begin{eqnarray}\label{eq:I-1-16}
	\Delta ( \theta ) & \doteq & \frac{ d \Phi }{ d \theta } ( \theta )= \frac{ 2 g }{ g^2 + \theta^2} . 	
\end{eqnarray}


%\paragraph{Retour aux coordonnées du laboratoire.}
%En revenant aux coordonnées d'origine (celles du laboratoire), on constate que la fonction d'onde à deux corps 
%\(
%	\varphi_{\{\theta_1 , \theta_2\}} (z_1, z_2) = \langle \emptyset \vert \operator{\Psi} (z_1)\operator{\Psi} (z_2) \vert \{\theta_1, \theta_2\} \rangle,
%\)
%avec \(z_1 < z_2\) , (ie $Y>0$) . Et le centre de masse sur le mouvement
%\(
%	Z  =  \overline{\theta} t.	
%\)
%avec,  on rappelle , $\overline{\theta}$ la vitesse de groupe dans le référentielle de laboratoire.\\
%Nous constatons que la position des deux particules semiclassiques après la collision sera
%\begin{eqnarray}
%	z_1 ~=~ Z + \frac{Y}2 ~\simeq ~ \theta_1 t - \Delta(\theta_1 - \theta_2), & & 	z_2 ~=~ Z - \frac{Y}2 ~\simeq ~ \theta_2t + \Delta(\theta_1 - \theta_2),
%\end{eqnarray}

%avec  $\theta_1$ et $\theta_2$ on rappelle définie tel que 
%\(
%	\tilde{\theta} ~=~\theta_1 - \theta_2 , \,	\overline{\theta}~=~\frac{\theta_1 + \theta_2}{2}.	
%\)
%On remarquant que 
%\begin{eqnarray*}
%	z_1 \theta_1  + z_2  \theta_2 ~=~ 2Z\overline{\theta} + \frac{1}{2}Y\tilde{\theta}, & & z_1 \theta_2  + z_2  \theta_1 ~=~ 2Z\overline{\theta} - \frac{1}{2}Y\tilde{\theta}. 
%\end{eqnarray*}
%Ce qui est en accod avec la masse total $\overline{m} = 2$ et la masse résuite $\tilde{m} = \frac{1}{2}$ \\
%Ce qui nous motive à multiplier la fonction d'onde dans le référentiel du centre de masse \eqref{eq:ansatz.boson} par $\exp(2iZ\overline{\theta})$ pour obtenir 

%\begin{eqnarray}\label{eq:I-1-10}
%	\varphi_{\{\theta_1 , \theta_2\}}(z_1 , z_2) & \propto &  \left \{ \begin{array} { c cl} ( \theta_2 - \theta_1 - ic) e^{ i z_1 \theta_1 + iz_2 \theta_2 } - ( \theta_1 - \theta_2 - ic) e^{ i z_1 \theta_2 + iz_2 \theta_1} & \mbox{si} & z_1 < z_2 \\ (z_1 \leftrightarrow z_2) & \mbox{si} & z_1 > z_2 \end{array} \right.
%\end{eqnarray}

%correspondant aux valeurs propres

%\begin{eqnarray}
%	\varepsilon(\theta_1 , \theta_2) ~=~ \overbrace{ \overline{\theta}^2}^{\overline{\varepsilon}(\overline{\theta})}	 + \overbrace{\frac{1}{4} \tilde{\theta}^2}^{\tilde{\varepsilon}(\tilde{\theta})} ~=~ \frac{\theta_1}{2} + \frac{\theta_2}{2}.	
%\end{eqnarray}

%Pour $\theta_1 > \theta_2$, les deux termes $e^{iz_1 \theta_1 + iz_2 \theta_2 }$ et $e^{iz_1 \theta_2 + iz_2 \theta_1 }$ correspondent aux paires de particules entrantes et sortantes dans un processus de diffusion à deux corps. Le rapport de leurs amplitudes est la phase de diffusion à deux corps \eqref{chap:1:dif.mod.2.part.3} reste inchangé

%\begin{eqnarray}\label{chap:1:dif.mod.2.part.4}
%	e^{i\Phi ( \theta_1 - \theta_2  ) }~=~ -\frac{a}{b} ~=~\frac{\theta_1 - \theta_2  -ic} { \theta_2 - \theta_1  - ic}. 
%\end{eqnarray}


%%%%%%%%%%%%%%%%%%%%%%%%%%
\paragraph{Retour aux coordonnées du laboratoire.}

En revenant aux coordonnées du laboratoire, la fonction d’onde à deux corps s’écrit
\(
	\varphi_{\{\theta_1 , \theta_2\}} (z_1, z_2) 
	= \langle \emptyset \vert \operator{\Psi} (z_1)\operator{\Psi} (z_2) \vert \{\theta_1, \theta_2\} \rangle/\sqrt{2},
\)
dans le cas \(z_1 < z_2\), c’est-à-dire pour une séparation relative \(Y = z_1 - z_2 < 0\) (on pourra symétriser ultérieurement).  
Dans le référentiel du laboratoire, le centre de masse évolue selon
\(
	Z = \frac{z_1 + z_2}{2} = \overline{\theta}\,t.
\)
%où l’on rappelle que \(\overline{\theta} = \frac{\theta_1 + \theta_2}{2}\) est la vitesse de groupe du système dans le référentiel laboratoire.
Ainsi, la position semi-classique des deux particules après la collision s’écrit
\begin{eqnarray}
	z_1 = Z + \frac{Y}{2} \;\simeq\; \theta_1 t - \Delta(\theta_1 - \theta_2),\quad
	z_2 = Z - \frac{Y}{2} \;\simeq\; \theta_2 t + \Delta(\theta_1 - \theta_2),
\end{eqnarray}
%où \(\Delta(\theta_1 - \theta_2)\) représente le décalage dû à l’interaction entre les deux particules.
%On rappelle les définitions :
%\[
%	\tilde{\theta} = \theta_1 - \theta_2, 
%	\quad
%	\overline{\theta} = \frac{\theta_1 + \theta_2}{2}.
%\]
On peut vérifier les identités utiles suivantes :
\begin{eqnarray*}
	z_1 \theta_1 + z_2 \theta_2 = 2Z \overline{\theta} + \frac{1}{2} Y \tilde{\theta}, \quad
	z_1 \theta_2 + z_2 \theta_1 &=& 2Z \overline{\theta} - \frac{1}{2} Y \tilde{\theta},
\end{eqnarray*}
ce qui est en accord avec les masses associées : masse totale \(\overline{m} = 2\), masse réduite \(\tilde{m} = \frac{1}{2}\).

Cela nous motive à multiplier l’ansatz dans le référentiel du centre de masse (équation~\eqref{eq:ansatz.boson}) par un facteur de phase globale \(\exp(2iZ\overline{\theta})\) pour revenir à la représentation dans le laboratoire. On obtient alors l’expression de la fonction d’onde :
\begin{eqnarray}\label{eq:I-1-10}
	\varphi_{\{\theta_1 , \theta_2\}}(z_1 , z_2) & \propto &  \left \{ \begin{array} { c cl} ( \theta_2 - \theta_1 - ig) e^{ i z_1 \theta_1 + iz_2 \theta_2 } - ( \theta_1 - \theta_2 - ig) e^{ i z_1 \theta_2 + iz_2 \theta_1} & \mbox{si} & z_1 < z_2 \\ (z_1 \leftrightarrow z_2) & \mbox{si} & z_1 > z_2 \end{array} \right.
\end{eqnarray}

%Cette fonction d’onde correspond à une valeur propre d’énergie donnée par la somme des énergies associées aux deux degrés de liberté :

%\begin{equation}
%	\varepsilon(\theta_1 , \theta_2) 
%	= \underbrace{\overline{\theta}^2}_{\overline{\varepsilon}(\overline{\theta})}
%	+ \underbrace{\frac{1}{4} \tilde{\theta}^2}_{\tilde{\varepsilon}(\tilde{\theta})}
%	= \frac{\theta_1^2}{2} + \frac{\theta_2^2}{2}.
%\end{equation}

Pour \(\theta_1 > \theta_2\), les deux termes exponentiels 
\(e^{i z_1 \theta_1 + i z_2 \theta_2}\) et \(e^{i z_1 \theta_2 + i z_2 \theta_1}\)
correspondent respectivement aux ondes entrantes et sortantes dans le canal de diffusion à deux corps.  
Le rapport de leurs amplitudes définit la phase de diffusion / matrice diffusion $e^{i\Phi ( \tilde{\theta}  ) }$  à deux corps \eqref{chap:1:dif.mod.2.part.3} , reste inchangé :

\begin{equation}\label{chap:1:dif.mod.2.part.4}
	S(\theta_1- \theta_2) \doteq e^{i\Phi(\theta_1 - \theta_2)} 
	= \frac{\theta_1 - \theta_2 - ig}{\theta_2 - \theta_1 - ig}.
\end{equation}

Cette phase caractérise entièrement le processus de diffusion dans le modèle de Lieb-Liniger à deux particules.

\paragraph{Conditions périodiques et équations de Bethe pour deux bosons {\color{red}(à révoir)}.}

%La fonction d’onde obtenue par Bethe ansatz (voir
%\eqref{eq:I-1-10}) est, pour $z_{1}<z_{2}$,
%\[
%	\varphi_{\{\theta_{1},\theta_{2}\}}(z_{1},z_{2})
%		= a\,e^{i\theta_{1}z_{1}+i\theta_{2}z_{2}}
%		+b\,e^{i\theta_{2}z_{1}+i\theta_{1}z_{2}},
%	\quad
%	a=\theta_{2}-\theta_{1}-ic,\;
%	b=-(\theta_{1}-\theta_{2}-ic).
%\]

%\medskip
%\subparagraph{Périodicité sur $z_{2}$.}  
%On impose à la fonction d’onde obtenue par Bethe ansatz (voir
%\eqref{eq:I-1-10})
%\(
%	\varphi_{\{\theta_{1},\theta_{2}\}}(z_{1},z_{2}\!=\!L)
%	=
%	\varphi_{\{\theta_{1},\theta_{2}\}}(z_{1},z_{2}\!=\!0)
%\)
%avec $0<z_{1}<z_{2}=L$.  
%Au point $z_{2}=L$ on reste dans le secteur $z_{1}<z_{2}$, tandis qu’au point $z_{2}=0$ le domaine pertinent devient $z_{2}<z_{1}$;  la fonction d’onde y est obtenue en échangeant $z_{1}\leftrightarrow z_{2}$ , soit 
%\(
%	\varphi_{\{\theta_{1},\theta_{2}\}}(z_{1},\!L)
%	=
%	\varphi_{\{\theta_{1},\theta_{2}\}}(0 , z_{1})
%\)
%.  
%On obtient ainsi
%\begin{eqnarray*}
%	a\,e^{i\theta_{1}z_{1}+i\theta_{2}L}+b\,e^{i\theta_{2}z_{1}+i\theta_{1}L} & = &
%	a\,e^{i\theta_{2}z_{1}}\,e^{i\theta_{1}\! \cdot 0} + b \,e^{i\theta_{1}z_{1}}\,e^{i\theta_{2}\! \cdot 0},	
%\end{eqnarray*}
%avec la condition $z_1< z_2$, avec le rapport $a$ et $b$ vérifiant \eqref{chap:1:dif.mod.2.part.4} de la sorte $-b/a = e^{i\Phi(\theta_1 - \theta_2)}$ .

%%%%%%%%%%%%%%%%

\subparagraph{Périodicité en \( z_2 \).}  
On impose une condition de périodicité sur la fonction d’onde obtenue par ansatz de Bethe (voir équation~\eqref{eq:I-1-10}) :
\(
	\varphi_{\{\theta_1,\theta_2\}}(z_1, z_2 = L) = \varphi_{\{\theta_1,\theta_2\}}(z_1, z_2 = 0),
\)
avec \( 0 < z_1 < z_2 = L \).  
Au point \( z_2 = L \), la configuration reste dans le secteur \( z_1 < z_2 \), tandis qu’à \( z_2 = 0 \), on entre dans le secteur \( z_2 < z_1 \). La continuité de la fonction d’onde impose alors d’échanger les coordonnées \( z_1 \leftrightarrow z_2 \) :
\(
	\varphi_{\{\theta_1,\theta_2\}}(z_1, L) = \varphi_{\{\theta_1,\theta_2\}}(0, z_1).
\)
En utilisant l’expression explicite de l’ansatz dans les deux secteurs, on obtient l’égalité suivante :
\begin{eqnarray*}
	a\,e^{i\theta_1 z_1 + i\theta_2 L} + b\,e^{i\theta_2 z_1 + i\theta_1 L}
	&=& a\,e^{i\theta_2 z_1} + b\,e^{i\theta_1 z_1}.
\end{eqnarray*}
%où le second membre correspond à la fonction d’onde dans le secteur \( z_2 < z_1 \), évaluée en \( z_2 = 0 \) et \( z_1 = z_1 \).  
%La condition de périodicité impose donc :
%\[
%	a\,e^{i\theta_1 z_1 + i\theta_2 L} + b\,e^{i\theta_2 z_1 + i\theta_1 L}
%	= a\,e^{i\theta_2 z_1} + b\,e^{i\theta_1 z_1}.
%\]
Cette relation, valable pour tout \( z_1 \in (0,L) \), fixe une contrainte sur le rapport \( b/a \). En utilisant l’expression de la phase de diffusion introduite en \eqref{chap:1:dif.mod.2.part.4} pour $z_1<z_2$ :
\begin{eqnarray*}
	-\frac{b}{a} = e^{i\Phi(\theta_1 - \theta_2)},
\end{eqnarray*}
on obtient une condition quantique sur les phases \( \theta_1 \) et \( \theta_2 \), cœur de la quantification imposée par le formalisme de Bethe.

%\[
%	( \theta_2 - \theta_1 - ig)\,e^{i\theta_{1}z_{1}+i\theta_{2}L}
%	- ( \theta_1 - \theta_2 - ig)\,e^{i\theta_{2}z_{1}+i\theta_{1}L}
%	=
%	( \theta_2 - \theta_1 - ig)\,e^{i\theta_{2}z_{1}}\,e^{i\theta_{1}\! \cdot 0}
%	- ( \theta_1 - \theta_2 - ig)\,e^{i\theta_{1}z_{1}}\,e^{i\theta_{2}\! \cdot 0}.
%\]
En identifiant les coefficients de $e^{i\theta_{1}z_{1}}$ et
$e^{i\theta_{2}z_{1}}$ indépendamment, on obtient
\(
	e^{i\theta_{2}L}\;a = b, 
	\,
	e^{i\theta_{1}L}\;b = a,
\)
c’est‑à‑dire l'équations de Bethe
%\begin{equation}\label{eq:PC2}
%	e^{i\theta_{2}L} = \frac{b}{a}
%	= \frac{\theta_{1}-\theta_{2}+ic}{\theta_{2}-\theta_{1}+ic},
%\quad
%	e^{i\theta_{1}L} = \frac{a}{b}
%	= \frac{\theta_{2}-\theta_{1}+ic}{\theta_{1}-\theta_{2}+ic}.
%\end{equation}
\begin{eqnarray*}\label{eq:PC2}
	e^{i\theta_{1}L}\,e^{i\Phi(\theta_{1}-\theta_{2})} = -1,
	\qquad
	e^{i\theta_{2}L}\,e^{i\Phi(\theta_{2}-\theta_{1})} = -1.	
\end{eqnarray*}
En prenant le logarithme on obtient les \emph{équations de Bethe à deux
particules} :
\begin{equation}\label{eq:Bethe2}
	\theta_{1}L + \Phi(\theta_{1}-\theta_{2}) = 2\pi I_{1}, 
	\qquad
	\theta_{2}L + \Phi(\theta_{2}-\theta_{1}) = 2\pi I_{2},
\end{equation}
où $I_{1},I_{2}\in\mathbb{Z}$ sont les nombres quantiques entiers
(caractère bosonique). 

\subparagraph{Périodicité sur $z_{1}$.}  Le raisonnement symétrique conduit exactement aux mêmes égalités \eqref{eq:PC2}.  
\bigskip
Les équations \eqref{eq:Bethe2} constituent la quantification complète
du gaz de Lieb–Liniger à deux bosons sur un cercle de longueur $L$ et
seront le point de départ pour l’étude de l’état fondamental et des
excitations.



\begin{figure}[H]
	\centering
  %\includegraphics[width=0.5\textwidth]{}
  %\caption{Gauche : La fonction d'onde (\ref{eq:I-1-10}) sur la ligne infinie correspond à un processus de diffusion à deux corps. Semiclassiquement, la phase de diffusion dans ce processus à deux corps se reflète dans le décalage de diffusion (\ref{eq:I-1-16}) : après la collision, la position de la particule a été déplacée d'une distance $\Delta ( \theta_1 - \theta_2 )$ . Droite : La fonction d'onde de Bethe (\ref{eq:I-2-17}) sur la ligne infinie correspond à un processus de diffusion à $N$-corps qui se factorise en des processus à deux corps (le décalage de diffusion $\Delta$ est également présent ici, mais il n'est pas représenté dans la caricature). Dans ce processus à $N$-corps, les rapidités $\theta_j$ sont les moments asymptotiques des bosons.}
  \label{}	
\end{figure}



\section{Équation de Bethe et distribution de rapidité}

\subsection{Fonction d’onde dans le secteur ordonné et représentation de Gaudin}

Dans le domaine $z_1 < z_2 < \cdots < z_N$, la fonction d’onde pour un état de Bethe à $N$ particules s’écrit ({\color{blue}Gaudin 2014}, {\color{blue}Korepin et al. 1997}, {\color{black}Lieb et Liniger 1963}) :
\begin{eqnarray}
	\varphi_{\{\theta_a\}} ( z_1 , \cdots , z_N ) & = &  \frac{1}{\sqrt{N!}}\langle \emptyset \vert \operator{\Psi} ( z_1 ) \cdots \operator{\Psi} (z_N ) \vert \{ \theta_a \} \rangle \notag\\
	& \propto & \sum_\sigma (-1)^{|\sigma|} \left( \prod_{1 \leq a < b \leq N} (\theta_{\sigma(b)} - \theta_{\sigma(a)} - i g) \right) e^{i \sum_{j=1}^{N} z_j \theta_{\sigma(j)}},\label{eq:I-2-17}
\end{eqnarray}
où la somme s'étend sur toutes les permutations $\sigma$ de $\{1,\dots,N\}$. Le facteur $(-1)^{|\sigma|}$ est la signature de la permutation, et les amplitudes dépendent des différences de quasi-moments $\theta_j$ ainsi que du couplage $c$.
Cette fonction d’onde est ensuite étendue par symétrie aux autres domaines du type $z_{\pi(1)} < z_{\pi(2)} < \cdots < z_{\pi(N)}$ via des propriétés d’échange symétriques.

\vspace{1em}

\subsection{Conditions aux bords périodiques}

Les équations précédentes ont été établies pour un système défini sur la droite réelle. Cependant, dans une perspective thermodynamique, il est essentiel de considérer une densité finie $ N/L$. Cela peut être obtenu en compactifiant l’espace sur un cercle de longueur $L$, i.e. en imposant les {\em conditions aux bords périodiques}.

Concrètement, cela consiste à identifier $x = 0$ et $x = L$ et à exiger que la fonction d’onde soit périodique lorsqu’une particule fait le tour du système :
\begin{equation}\label{eq:periodic}
\varphi_{\{\theta_a\}}(x_1, \dots, x_{N-1}, L) = \varphi_{\{\theta_a\}}(0, x_1, \dots, x_{N-1}).
\end{equation}
Cette condition doit être satisfaite pour chaque particule. Or, déplacer la $j$-ième particule de $x_j$ à $x_j + L$ revient à la faire passer devant toutes les autres : cela introduit un facteur de diffusion à chaque croisement.

%\vspace{1em}

\subsection{Équations de Bethe exponentielles}

En imposant les conditions de périodicité sur la fonction d’onde de type Bethe~\eqref{eq:I-2-17}, on obtient que chaque moment $\theta_a$ doit satisfaire l’équation :
\begin{equation}
	e^{i \theta_a L} \prod_{b \ne a} S(\theta_a - \theta_b) = (-1)^{N-1}, \quad a = 1, \dots, N,
	\label{eq:bethe_exp}
\end{equation}
où la matrice diffusion $S(\theta) = \frac{\theta - i g}{-\theta - i g} = e^{i \Phi(\theta)}$ est l’amplitude de diffusion à deux corps, et $\Phi(\theta) = 2 \arctan\left( \frac{\theta}{c} \right)$ est la phase associée~\eqref{chap:1:eq:Phi}. Le signe $(-1)^{N-1}$ vient du fait que chaque permutation change la signature du déterminant dans la représentation de Gaudin.
%\vspace{1em}

\subsection{Équations de Bethe logarithmiques}

En prenant le logarithme du membre gauche et du membre droit de l’équation~\eqref{eq:bethe_exp}, on obtient :
\begin{equation}\label{chap:1:eq:EBA}
	L \theta_a + \sum_{b=1}^N \Phi(\theta_a - \theta_b) = 2\pi I_a, \qquad a = 1, \dots, N,
\end{equation}
où les $I_a \in \mathbb{Z}$ (ou $\mathbb{Z} + \tfrac{1}{2}$) sont des nombres quantiques entiers (ou demis entiers) . Dans la configuration d’état fondamental (ou de type “mer de Fermi”), ces nombres sont pris de manière symétrique autour de zéro :
\[
I_a = a - \frac{N+1}{2}, \quad \text{pour } a \in \llbracket 1 , N \rrbracket.
\]
Ce choix garantit une distribution uniforme des $\theta_a$ à l’état fondamental.
%\vspace{1em}

\subsection{Interprétation physique}

Les équations de Bethe~\eqref{chap:1:eq:EBA} représentent une {\em quantification des pseudo‑impulsions $\theta_a$} des particules en interaction, résultant d’un {\em interféromètre multi‑corps sur le cercle} : chaque particule accumule une phase $e^{i \theta_a L}$ due au mouvement libre, ainsi que des phases de diffusion lorsqu’elle croise les autres.

Ce système d'équations détermine les états propres du système de Lieb–Liniger en volume fini, et joue un rôle fondamental dans la description exacte de ses propriétés thermodynamiques et dynamiques.


\subsection{Thermodynamique du gaz de Lieb–Liniger à température nulle}

Dans la limite thermodynamique, le nombre de particules \( N \) et la longueur \( L \) du système tendent vers l'infini de telle sorte que leur rapport reste fini :
\begin{eqnarray*}
	\lim_{N,\, L \to \infty} \frac{N}{L} = D < \infty,
\end{eqnarray*}
où \( D \) désigne la densité linéique de particules.

Considérons désormais le système à température nulle. L’état fondamental dans le secteur à nombre de particules fixé correspond à la configuration d’énergie minimale parmi les solutions des équations de Bethe \eqref{chap:1:eq:EBA}.

Dans la limite thermodynamique, les valeurs de \( \theta_a \) deviennent quasi-continues, avec un espacement \( \theta_{a+1} - \theta_a = \mathcal{O}(1/L) \), et se condensent dans un intervalle symétrique autour de zéro :
\[
\theta_a \in [-K, K],
\]
où \( K \) est le paramètre de Fermi (ou rapidité maximale), défini par \( K = \theta_N \). En supposant l'ordre \( I_a \geq I_b \Rightarrow \theta_a \geq \theta_b \), cet intervalle constitue ce qu'on appelle la {\em mer de Dirac} (ou sphère de Fermi en dimension un).

Nous introduisons la densité d’états \( \rho_s(\theta) \), définie par
\begin{eqnarray*}
	2\pi \rho_s(\theta_a) &=& \frac{2\pi}{L} \lim_{\text{therm}} \frac{|I_{a+1} - I_a|}{|\theta_{a+1} - \theta_a|} = \frac{2\pi}{L} \frac{\partial I}{\partial \theta}(\theta_a),
\end{eqnarray*}
où \( I(\theta_a) = I_a \). L’application des équations de Bethe sous forme logarithmique conduit alors à
\begin{eqnarray*}
	2\pi \rho_s(\theta_a) = 1 + \frac{1}{L} \sum_{b = 1}^N \Delta(\theta_a - \theta_b),
\end{eqnarray*}
ce qui relie \( \rho_s \) à la fonction d’interaction \( \Delta \) entre les rapidités.

Intéressons-nous maintenant à la {\em densité de particules dans l’espace des moments}, notée \( \rho(\theta) \), définie par
\begin{eqnarray*}
	\rho(\theta_a) = \lim_{L \to \infty} \frac{1}{L} \cdot \frac{1}{\theta_{a+1} - \theta_a} > 0.
\end{eqnarray*}
Dans l’état fondamental, toutes les positions disponibles dans l’intervalle \( [-K, K] \) sont occupées. On a donc :
\begin{eqnarray}\label{chap.1.rho.2}
	\rho(\theta) = \rho_s(\theta).
\end{eqnarray}

La quantité \( L \rho(\theta) d\theta \) représente le nombre de rapidités dans la cellule infinitésimale \( [\theta, \theta + d\theta] \), tandis que
\(
	N = L \int_{-K}^{K} \rho(\theta)\, d\theta
\)
donne le nombre total de particules dans le système. Le passage de la somme discrète à l'intégrale dans le second membre de l'équation de Bethe permet d’écrire :
\begin{eqnarray*}
	\frac{1}{L} \sum_{b = 1}^N \Delta(\theta_a - \theta_b) \longrightarrow \int_{-K}^{K} \Delta(\theta_a - \theta)\, \rho(\theta)\, d\theta.
\end{eqnarray*}
Ainsi, l'équation pour la densité d'états devient :
\begin{eqnarray}\label{chap.1.rho.s.2}
	2\pi \rho_s(\theta) = 1 + \int_{-K}^{K} \Delta(\theta - \theta')\, \rho(\theta')\, d\theta',
\end{eqnarray}
et, comme \( \rho = \rho_s \), on obtient l’équation linéaire intégrale satisfaite par la densité de rapidités :
\begin{eqnarray}\label{chap.1.rho.3}
	\rho(\theta) - \int_{-K}^{K} \frac{\Delta(\theta - \theta')}{2\pi} \rho(\theta')\, d\theta' = \frac{1}{2\pi}.
\end{eqnarray}


\subsection{Excitations élémentaires à température nulle}




\chapter{Relaxation et Équilibre dans les Systèmes Quantiques Intégrables : Une Approche par la Thermodynamique de Bethe}\label{chap:relaxation}
\minitoc

%------------------------------------------------------------------
\section*{Introduction générale}

Dans les modèles quantiques intégrables, l’évolution vers l’équilibre, à partir d’un état initial arbitraire (et typiquement hors d’équilibre), ne conduit pas à une thermique de Gibbs classique.  
En effet, du fait de l’existence d’une infinité de charges conservées en involution, les systèmes intégrables n’explorent qu’une sous-partie contrainte de l’espace des états accessibles.  
Ils relaxent alors vers un état stationnaire décrit par une \emph{Ensemble Thermodynamique Généralisé} (GGE), qui encode la conservation de toutes ces quantités.

Cette section pose les fondations nécessaires à la description de ces états stationnaires dans le cadre de la \textbf{thermodynamique de Bethe} (TBA), qui généralise l’analyse au-delà de l’état fondamental.  
Nous considérons ici un régime macroscopique à température (ou entropie) finie, correspondant à des états hautement excités du spectre, mais toujours décrits dans le formalisme intégrable exact.

Notre point de départ est la relation constitutive entre la \emph{densité de quasi-particules} (ou \emph{rapidités}) $\rho(\theta)$ et la \emph{densité d’états} disponibles $\rho_s(\theta)$, qui encode le spectre accessible en présence d’interactions.  
Nous introduisons ensuite une opération clé de la TBA, appelée \emph{habillage} (\emph{dressing}), qui intervient systématiquement dans le calcul des observables physiques et permet de prendre en compte de manière non perturbative les effets des interactions.  
Cette construction sera illustrée dans le cadre du modèle intégrable de Lieb–Liniger, qui décrit un gaz unidimensionnel de bosons avec interaction delta répulsive.

Les outils développés ici seront fondamentaux pour formuler dans la section suivante le concept d’ensemble généralisé (GGE), et pour décrire la dynamique de relaxation des systèmes intégrables.



\section{Notion d’état d’équilibre généralisé (GGE)}

\paragraph{Introduction.}


\paragraph{Configuration des états.}\label{sec:config-etats}.
On désigne par $\boldsymbol{\{ \theta_a \}}\equiv \{ \theta_1 , \cdots , \theta_{N} \}$ la \emph{configuration de rapidités} caractérisant un état propre à $N\!\equiv\!N(\{ \theta_a \})$ particules – le nombre de particules n’est donc pas fixé \emph{a priori} mais dépend de la configuration.  
L’état propre correspondant est noté $\ket{\{ \theta_a \}}\;=\;\ket{\{\theta_1,\dots,\theta_N \}}$.

%%%%%%%%%%%%%%%%%%%%%%%%%%%%%%%%%%%%%%%%%%%%%%%%%%
\paragraph{Observables diagonales dans la base des états propres.}
Dans le chapitre précédent (\ref{chap:LL-BA}), on a vu que l'état $\ket{\{ \theta_a \}}$ associé à cette configuration est une état propre des observables nombre et quantité de mouvement  et  énergie cinétique \eqref{chap1:eq.Q.P.K.theta.1}. Ces observables sont diagonales dans la base des états propres :
\begin{eqnarray}
	\operator{Q}  =  \sum_{ \{\theta_a\} } \left ( \sum_{a = 1}^{N}  1 \right )  \vert \{ \theta_a\}\rangle	\langle \{ \theta_a \}\vert, \, 
	\operator{P}  =  \sum_{\{ \theta_a\}}\left( \sum_{a = 1}^{N}  \theta_a \right )   \vert \{ \theta_a\}\rangle	\langle \{ \theta_a \}\vert,\,\operator{K}  =  \sum_{\{ \theta_a\}}\left ( \sum_{a = 1}^{N} \frac{\theta_a^2}{2} \right )   \vert \{ \theta_a\}\rangle	\langle \{ \theta_a \}\vert.\label{chap.2.gge.1}		
\end{eqnarray}
avec $ \sum_{\{ \theta_a\}}$ une somme sur tous les configurations.\\
%\begin{eqnarray}
%	\operator{Q} \ket{\{ \theta_a\}}  =  \sum_{ \{\theta_a\} } \left ( \sum_{a = 1}^{N}  1 \right ) \ket{\{ \theta_a\}}, \, 
%	\operator{P} \ket{\{ \theta_a\}}  =  \sum_{\{ \theta_a\}}\left( \sum_{a = 1}^{N}  \theta_a \right ) \ket{\{ \theta_a\}},\,\operator{H} \ket{\{ \theta_a\}}  =  \sum_{\{ \theta_a\}}\left ( \sum_{a = 1}^{N} \frac{\theta_a^2}{2} \right )   \ket{\{ \theta_a\}}.		
%\end{eqnarray}

Nous avons introduit ces observables en injectant des opérateurs $\operator{f}$ proportionnels à des puissances de la quantité de mouvement d’une particule $\operator{p}$, respectivement $\propto \operator{p}^0$, $\propto \operator{p}^1$ et $\propto \operator{p}^2$, dans l’opérateur à un corps $\operator{F}$ défini dans l’équation \eqref{chap.1:eq.rapel.opp.1.second.2}. Écrit de cette manière, nous avons vu dans l’équation \eqref{chap.1:eq.rapel.opp.1.second.3} que pour $\operator{f} = \operator{p}^q$ avec $q$ entier, l’état de Bethe $\ket{\{ \theta_a \} }$ est un état propre de $\operator{F}$ :
\begin{eqnarray}\label{chap.2:eq.rapel.opp.1.second.1}
	 \operator{F} \ket{\{\theta_a\}} =   \sum_{ \{\theta_a\} }\left( \sum_{a = 1}^N \theta_a^q \right) \ket{\{\theta_a\}},
\end{eqnarray}
avec des valeurs propres données par des puissances de $\theta$. Cela motive l’étude d’états d’équilibre statistique au-delà de l’équilibre thermique, c’est-à-dire au-delà de l’ensemble de Gibbs.
   




%%%%%%%%%%%%%%%%%%%%%%%%%%%%%%%%%%%%%%%%%%%%
\paragraph{Contexte et GGE dans les systèmes intégrables.}

Dans un système quantique {\bf intégrable}, il existe une infinité de charges conservées locales $\operator{Q}_i$ commutant entre elles et avec l’Hamiltonien $\operator{H}$ ([Rigol et al. 2007] ) \cite{??}. Concrètement, chaque charge se présente sous la forme $\operator{Q}_i = \int dx \,\operator{q}_i(x)$, où $\operator{q}_i(x)$ est une densité d’observable locale à support borné. L’intégrabilité implique ainsi une caractérisation complète des états propres par un ensemble de paramètres (rapidités $\{\theta_j\}$ dans le modèle de Lieb-Liniger) \cite{??}. En particulier, contrairement aux systèmes génériques, un système intégrable ne thermalise pas au sens canonique classique, car la présence de toutes ces contraintes empêche l’oubli complet des conditions initiales. Les points clés sont alors :

\begin{itemize}[label = $\bullet$]
	\item {\bf Charges conservées} : infinité de locales $\operator{Q}_i$ satisfaisant et $[\operator{Q}_i , \operator{H} ] = 0$ et $[\operator{Q}_i , \operator{Q}_j ] = 0$.
	\item {\bf Densités locales} : chaque $\operator{Q}_i$ s’écrit $\operator{Q}_i = \int_\mathbb{R} dx \, \operator{q}_i(x)$ avec $\operator{q}_i(x)$ à support fini.
	\item {\bf Relaxation non canonique} : après un {\em quench} (changement brutal de paramètre), le système évolue vers un état stationnaire qui n’est pas décrit par l’ensemble canonique habituel.
\end{itemize}

Pour décrire cet état, on introduit l’{\bf ensemble de Gibbs généralisé (GGE)}. Rigol et al. ont montré qu’une « extension naturelle de l’ensemble de Gibbs aux systèmes intégrables » prédit correctement les valeurs moyennes des observables après relaxation \cite{??}.  Formellement, pour une région finie du système $\mathcal{S} \subset \mathbb{R}$, on définit la matrice densité locale :
\begin{eqnarray}
	\operator{\rho}^{(\mathcal{S})}_{\mathrm{GGE}} = \frac{1}{Z^{(\mathcal{S})}}\exp \left ( - \sum_i \beta_i \operator{Q}_i^{(\mathcal{S})} \right), \quad \operator{Q}_i^{(\mathcal{S})} = \int_\mathcal{S} dx \, \operator{q}_i(x), \label{chap.TBA.op.rho.S}	
\end{eqnarray}

où $\beta_i \in \mathbb{R}$ sont les multiplicateurs de Lagrange (ou « températures généralisées ») associés aux charges locales conservées $\{\operator{Q}_i\}$. La fonction de partition 
\begin{eqnarray}
	Z^{(\mathcal{S})} = \bm{\mathrm{Tr}}\left [\exp \left( - \sum_i \beta_i \operator{Q}_i^{(\mathcal{S})} \right ) \right ]  \label{chap.TBA.op.Z.S}	
\end{eqnarray}
 assure la normalisation. L’{\bf état GGE} ainsi défini est le seul permettant de prédire de manière cohérente les observables locales de $\mathcal{S}$ à long temps \cite{??}. Autrement dit, l’équilibre local après quench est un état stationnaire faisant perdurer la mémoire de chaque charge conservée, ce qui conduit à un nombre macroscopique de paramètres $\beta_i$ thermodynamiques (une « température » par charge) \cite{??}.

 \subparagraph{Interprétation des multiplicateurs de Lagrange.}
Les multiplicateurs de Lagranges $\beta_i$ apparaissent naturellement lors de l'optimisation sous contraintes, par exemple dans le formalisme de l'{\bf ensemble de Gibbs généralisé (GGE)}, oû il imposent la conservation des valeurs moyennes des charges $\langle \operator{Q}_i^{(\mathcal{S})} \rangle_{\operator{\rho}^{(\mathcal{S})}_{\mathrm{GGE}}} = \bm{\mathrm{Tr}}[\operator{\rho}^{(\mathcal{S})}_{\mathrm{GGE}} \operator{Q}_i^{(\mathcal{S})}]   $.\\

En résumé, la GGE généralise les ensembles canoniques standard : au lieu de retenir uniquement l’énergie, on impose la conservation de l’ensemble complet $\{\operator{Q}_i \}$. Cette construction rend compte du fait que, dans un système intégrable, les observables locaux convergent vers les valeurs moyennes de $\operator{\rho}^{(\mathcal{S})}_{\mathrm{GGE}}$ , et non vers celles d’un Gibbs thermique ordinaire \cite{??}\cite{??}. On comprend ainsi pourquoi la {\em thermalisation habituelle} (canonique ou microcanonique) échoue : seul l’ensemble de Gibbs généralisé peut intégrer toutes les contraintes locales.

\paragraph{Rappel sur le modèle de Lieb-Liniger et distribution de rapidités.}
Comme rappelé au chapitre précédent, {\bf le modèle de  Lieb-Liniger} (gaz bosonique 1D à interactions de contact) est un exemple paradigmatique d’un système intégrable \cite{??}. Ses états propres sont caractérisés par un ensemble de $N$  rapidités $\{ \theta_a \}$ , qui jouent le rôle de quasi-momenta ({\bf Bethe ansatz}). Dans ce contexte, l’état macroscopique du gaz après relaxation unitaire est entièrement déterminé par la {\bf distribution des rapidités}. Formellement, on définit $\rho(\theta)$ la distribution intensive des rapidités telle que $\rho(\theta) d \theta$ donne la fraction de particules par unité de longueur ayant une rapidité dans la cellule $[\theta , \theta + d \theta ] $.\\

Cette « distribution de rapidités » est d’autant plus pertinente qu’elle est {\em accessible expérimentalement}. En effet, lorsque le gaz bosonique 1D est libéré et laissé s’étendre, la distribution asymptotique des vitesses des atomes coïncide avec la distribution initiale des rapidités \cite{??} . Autrement dit, la GGE prédit un profil de vitesses observables en laboratoire. Léa Dubois souligne dans sa thèse que " la distribution de rapidités est la distribution asymptotique des vitesses des atomes après une expansion dans le guide 1D ", et qu’elle peut être extraite par l’hydrodynamique généralisée \cite{??}. \\

Dans la GGE, cette distribution macroscopique $\rho(\theta)$ est fixée par l’ensemble des charges conservées. Par exemple, on ajuste les $\beta_i$ de sorte que les valeurs moyennes $\langle \operator{Q}_i \rangle_{\operator{\rho}^{(\mathcal{S})}_{\mathrm{GGE}}}$ correspondent aux valeurs initiales. Ce processus détermine donc la fonction $\rho(\theta)$ décrivant l’état d’équilibre local. Les observables locaux du gaz (densité, corrélations, etc.) en découlent alors via les équations de Bethe ansatz. 


\paragraph{Convention pour les moyennes d'observables.}
Dans la suite du chapitre, nous noterons la moyenne d’une observable $\operator{\mathcal{O}}$ dans un état décrit par une matrice densité (ici noté) $\operator{\rho}$ par :
\begin{eqnarray}\label{chap.TBA.moy.dens}	
	\braket{\operator{\mathcal{O}}}_{\operator{\rho}} \doteq \bm{\mathrm{Tr}}[\operator{\rho} \, \operator{\mathcal{O}}],
\end{eqnarray}
En particulier, si la matrice densité est un projecteur, comme $\ket{\{\theta_a \}}\!\bra{\{\theta_a \}}$, $\bm{\mathrm{Tr}}[\ket{\{\theta_a \}}\!\bra{\{\theta_a \}} \operator{\mathcal{O}}] =  \bra{\{\theta_a \}}\operator{\mathcal{O}}\ket{\{\theta_a \}}$. dans ce cas on notera la moyenne :
\begin{eqnarray}\label{chap.TBA.moy.dens.pur}
	\braket{\operator{\mathcal{O}}}_{\{\theta_a \}} = \bra{\{\theta_a \}} \operator{\mathcal{O}} \ket{\{\theta_a \}},
\end{eqnarray}
où l’on note simplement l’ensemble des rapidité ${\theta_a}$ pour désigner l’état pur.

%%%%%%%%%%%%%%%%%%%%%%%%%%%%%%%%%%%%%%%%%%%%%%%%%%
\paragraph{Charges conservées locales diagonales dans la base des états propres.}
Les charges conservées locales $\operator{Q}_i^{(\mathcal{S})}$ est diagonale dans la base des  états propres $\ket{ \{ \theta_a \}}$ , avec pour valeurs propres $\langle \operator{Q}_i^{(\mathcal{S})} \rangle_{\{\theta_a \}} $ 	 :
%\begin{eqnarray}
%	\operator{Q}_i^{(\mathcal{S})} & = & \sum_{ \{\theta_a\} } \langle \operator{Q}_i^{(\mathcal{S})} \rangle_{\{\theta_a \}}  \ket{\{\theta_a \}}\!\bra{\{\theta_a \}}.		
%\end{eqnarray}
\begin{eqnarray}\label{chap.TBA.Qi.diag}
	\operator{Q}_i^{(\mathcal{S})}\ket{\{\theta_a \}} & = &  \langle \operator{Q}_i^{(\mathcal{S})} \rangle_{\{\theta_a \}}  \ket{\{\theta_a \}}.		
\end{eqnarray}
%%%%%%%%%%%%%%%%%%%%%%%%%%%%%%%%%%%%%%%%
\paragraph{Probabilité d’un état à rapidités fixées.}
On peut alors définir la probabilité d’occurrence d’un état $\ket{\{ \theta_a \} }$ comme la moyenne de la matrice densité locale $\operator{\rho}^{(\mathcal{S})}_{\mathrm{GGE}}$ définie dans \eqref{chap.TBA.op.rho.S}:
\begin{eqnarray}
	\mathbb{P}^{(\mathcal{S})}_{\{ \theta_a \}}  & \equiv &  \langle \operator{\rho}^{(\mathcal{S})}_{\mathrm{GGE}} \rangle_{\{\theta_a \}}, \label{chap.TBA.P.1}\\
	& = & 
	\frac{1}{Z^{(\mathcal{S})}} \exp \left (- \sum_i \beta_i \langle \operator{Q}_i^{(\mathcal{S})} \rangle_{\{\theta_a \}} \right ) \label{chap.TBA.P.2}.
\end{eqnarray}

%%%%%%%%%%%%%%%%%%%%%%%%%%%
\paragraph{Moyenne d’un charges conservées locales et dérivées de $Z^{(\mathcal{S})}$.} Les charges locales $\operator{Q}_i^{(\mathcal{S})}$ sont diagonale dans la bases \( \{ \ket{\{\theta_a \}} \}  \) [cf eq~ ~\eqref{chap.TBA.Qi.diag}]. 
On peut donc  écrire la moyenne d’une observable comme une somme pondérée par cette probabilité [cf eqs ~\eqref{chap.TBA.P.1}-\eqref{chap.TBA.P.2}] , ou encore comme une dérivée de la fonction de partition définie dans l'équation \eqref{chap.TBA.op.Z.S} :
\begin{eqnarray}
	\langle \operator{Q}_i^{(\mathcal{S})} \rangle_{\operator{\rho}^{(\mathcal{S})}_{\mathrm{GGE}}} &= & \sum_{\{ \theta_a\}} \langle \operator{Q}_i^{(\mathcal{S})} \rangle_{\{\theta_a \}} \mathbb{P}^{(\mathcal{S})}_{\{ \theta_a \}} \label{chap.TBA.moy.1}\\
	 & = &  \left. \frac{1}{Z^{(\mathcal{S})}} \frac{\partial Z^{(\mathcal{S})}}{\partial \beta_i} \right )_{\beta_{j \neq i }}	 \label{chap.TBA.moy.2}
\end{eqnarray}

Par le même raisonnement le moment non centré s'écrit :
\begin{eqnarray}
	\braket{ \operator{Q}_{i_1}^{(\mathcal{S})} \, \operator{Q}_{i_2}^{(\mathcal{S})} \cdots \operator{Q}_{i_q}^{(\mathcal{S})} }_{\operator{\rho}^{(\mathcal{S})}_{\mathrm{GGE}}} &= &  (-1)^q \frac{1}{Z^{(\mathcal{S})}} \left.\frac{\partial}{\partial \beta_{i_1}} \right )_{\beta_{j \neq i_1 }} \left.\frac{\partial}{\partial \beta_{i_2}} \right )_{\beta_{j \neq i_2 }} \cdots \left.\frac{\partial}{\partial \beta_{i_q}} \right )_{\beta_{j \neq i_q }} Z^{(\mathcal{S})} \label{chap.TBA.mom.1}.	
\end{eqnarray}

%%%%%%%%%%%%%%%%%%%%%%%%%%%%%%%
\paragraph{Moments d’ordre supérieur et fluctuations.} On s'avance sur le chapitre (\ref{chap:Fluctu}).
Le premier et second moments permettent d’accéder à la variance 
\begin{eqnarray}
	 \left \langle \left (\operator{Q}_i^{(\mathcal{S})} - \langle\operator{Q}_i^{(\mathcal{S})} \rangle_{\operator{\rho}^{(\mathcal{S})}_{\mathrm{GGE}}} \right )^2  \right \rangle_{\operator{\rho}^{(\mathcal{S})}_{\mathrm{GGE}}} = \langle(\operator{Q}_i^{(\mathcal{S})})^2 \rangle_{\operator{\rho}^{(\mathcal{S})}_{\mathrm{GGE}}}  -  \langle\operator{Q}_i^{(\mathcal{S})} \rangle_{\operator{\rho}^{(\mathcal{S})}_{\mathrm{GGE}}}^2	
\end{eqnarray}
de le charge locale $\operator{Q}_i^{(\mathcal{S})}$, en injectant \eqref{chap.TBA.moy.2} et \eqref{chap.TBA.mom.1} et en utilisant $\frac{1}{f} \partial_x^2 f - ( \frac{1}{f} \partial_x f ) = \partial_x^2 \ln f  $:
\begin{eqnarray}
	\left \langle \left (\operator{Q}_i^{(\mathcal{S})} - \langle\operator{Q}_i^{(\mathcal{S})} \rangle_{\operator{\rho}^{(\mathcal{S})}_{\mathrm{GGE}}} \right )^2  \right \rangle_{\operator{\rho}^{(\mathcal{S})}_{\mathrm{GGE}}}  &=&	  \left . \frac{\partial^2 \ln Z^{(\mathcal{S})}  }{{\partial \beta_i}^2 }  \right )_{\beta_{j\neq i}},\\
	& = &  - \left . 	\frac{\partial \langle\operator{Q}_i^{(\mathcal{S})} \rangle_{\operator{\rho}^{(\mathcal{S})}_{\mathrm{GGE}}} }{\partial \beta_i } \right )_{\beta_{j\neq i}}.	
\end{eqnarray}

%%%%%%%%%%%%%%%%%%%%%%%%%%%%%%
\paragraph{Cas particulier de l’équilibre thermique.}

Dans le cas particulier de l’équilibre thermique standard (\ie Gibbsien), le système est décrit par une seule contrainte d’énergie (ou d’énergie et de particule, dans le cas d’un grand canonique). Les multiplicateurs de Lagrange associés aux charges conservées peuvent alors être identifiés à des grandeurs thermodynamiques classiques.

\begin{itemize}[label=$\bullet$]
	\item Si la seule charge conservée est le nombre de particules $\operator{Q}_0^{(\mathcal{S})} = \operator{Q}$, le multiplicateur associé est $\beta_0 = -\beta \mu$, où $\mu$ est le potentiel chimique et $\beta = T^{-1}$ l’inverse de la température (avec $k_B = 1$).
	
	\item Si la charge conservée est $\operator{Q}_2^{(\mathcal{S})}-\mu\operator{Q}_0^{(\mathcal{S})}  = \operator{K} - \mu \operator{Q} $ (ensemble grand canonique), alors le multiplicateur est simplement $ \beta$.
\end{itemize}

Dans le cadre de l’équilibre thermique , les moyennes et les fluctuations thermodynamiques usuelles s’expriment naturellement comme dérivées du logarithme de la fonction de partition $Z^{(\mathcal{S})}$ :
\begin{eqnarray}
	\langle \operator{Q} \rangle_{\operator{\rho}^{(\mathcal{S})}_{\mathrm{GGE}}}  = \left .\frac{1}{\beta} \frac{ \partial \ln Z^{(\mathcal{S})}}{\partial \mu } \right )_{T},  & &  \left . \frac{1}{\beta} \frac{ \partial \langle \operator{Q} \rangle_{\operator{\rho}^{(\mathcal{S})}_{\mathrm{GGE}}}}{\partial \mu } \right )_{T} =  \left . \frac{1}{\beta^2} \frac{ \partial^2 \ln Z^{(\mathcal{S})}}{{\partial \mu}^2 } \right )_{T} \\
	\langle \operator{H} - \mu\operator{Q}  \rangle_{\operator{\rho}^{(\mathcal{S})}_{\mathrm{GGE}}}  = -\left . \frac{ \partial \ln Z^{(\mathcal{S})}}{\partial \beta } \right )_{\mu} ,  & & -\left .  \frac{ \partial \langle \operator{H} - \mu\operator{Q} \rangle_{\operator{\rho}^{(\mathcal{S})}_{\mathrm{GGE}}}}{\partial \beta } \right )_{\mu } = \left .  \frac{ \partial^2 \ln Z^{(\mathcal{S})}}{{\partial \beta}^2 } \right )_{\mu}   .		
\end{eqnarray}
En combinant ces relations, on peut également exprimer l’énergie moyenne et ses fluctuations comme :
\begin{eqnarray}
	\langle \operator{H} \rangle_{\operator{\rho}^{(\mathcal{S})}_{\mathrm{GGE}}}  = \left [ \left .\frac{\mu}{\beta} \frac{ \partial}{\partial \mu } \right )_{T} -\left . \frac{ \partial }{\partial \beta } \right )_{\mu}   \right ]\ln Z^{(\mathcal{S})},  \quad  -\left .  \frac{ \partial \langle \operator{H} \rangle_{\operator{\rho}^{(\mathcal{S})}_{\mathrm{GGE}}}}{\partial \beta } \right )_{-\mu \beta } = \left [ \left .\frac{\mu}{\beta} \frac{ \partial}{\partial \mu } \right )_{T} -\left . \frac{ \partial }{\partial \beta } \right )_{\mu}  \right ]^2\ln Z^{(\mathcal{S})}.		
\end{eqnarray}

%%%%%%%%%%%%%

\section{Remarques sur le formalisme}




%\input{preamble}

\begin{document}

\frontmatter
%\input{chapters/00_intro}
\tableofcontents
\mainmatter

\input{chapters/01_LL_BA}
\input{chapters/02_GGE_TBA}
\input{chapters/03_GHD}
%\input{chapters/97_GHD}
\input{chapters/04_GGE_Fluctuation}
\input{chapters/05_Disp_Exp}
\input{chapters/06_Bipart}
\input{chapters/07_Dipolaire}

%\input{chapters/08_conclusion}
%\appendix
%\input{chapters/99_annexes}

\bibliographystyle{abbrv}
\bibliography{thesis}

%\printbibliography

\end{document}

%| Style     | Description                                                             |
%| --------- | ----------------------------------------------------------------------- |
%| `plain`   | Tri alphabétique, numérotation croissante                               |
%| `unsrt`   | Même que `plain` mais sans tri, respecte l’ordre d’apparition           |
%| `abbrv`   | Comme `plain` mais avec prénoms et noms abrégés                         |
%| `alpha`   | Les références sont étiquetées par une combinaison du nom et de l’année |
%| `apalike` | Style APA simplifié                                                     |
%| `ieeetr`  | Style IEEE, tri par ordre d’apparition                                  |
%| `siam`    | Style SIAM (mathématiques appliquées)                                   |
%| `acm`     | Style ACM (informatique)                                                |
%



\section{Rôle des charges conservées extensives et quasi-locales}
%Dans les systèmes intégrables, l’état stationnaire atteint après une évolution hors d’équilibre n’est généralement pas décrit par un état de Gibbs classique, mais par un ensemble généralisé de Gibbs (GGE). Celui-ci est construit à partir de toutes les charges conservées du système

\paragraph{Écriture des observables thermodynamiques comme sommes sur les rapidités.}

%Dans le cas thermique, les valeurs moyennes des observables classiques telles que le nombre de particules et l'énergie peuvent s'exprimer comme des sommes de puissances des rapidités :
Dans un système à $N$ particules caractérisé par des rapidités $\{ \theta_a \}_{a = 1}^N$, les charges conservées classiques — telles que le nombre de particules, l’impulsion ou l’énergie — s’écrivent comme des sommes de puissances des rapidités :
\(
	\langle \operator{Q} \rangle_{\{ \theta_a\} } \propto \sum_{a = 1}^N \theta_a^0 , \,  \langle \operator{P} \rangle_{\{ \theta_a\} } \propto \sum_{a = 1}^N \theta_a^1  ,\,  \mbox{et} \langle \operator{K} \rangle_{\{ \theta_a\} } \propto \sum_{a = 1}^N \theta_a^2 .	
\)
(cf. équations \eqref{chap.2.gge.1})
Dans ce paragraphe précédent, nous avons sous-entendu — sans l’expliciter — qu’il est montré que l’ensemble des charges locales conservées forme une famille donnée par :
\begin{eqnarray}
	\operator{Q}_i^{(\mathcal{S})} \ket{\{\theta_a\} } & \propto & \sum_a \theta_a^i \ket{\{\theta_a\} }.
\end{eqnarray}
Ces charges agissent donc de manière diagonale sur les états de Bethe, avec des valeurs propres correspondant aux moments des rapidités.
%%%%%%%%%%%%%%%%%%%%%%%%%%%%%%%%%%%%%%%%%%%%%%%%%%
\paragraph{Charges locales conservées .\label{sec:charges-gen}}

%Les états propres du Hamiltonien de Lieb–Liniger~\eqref{eq:LL} sont les états de Bethe
%\(
%  \ket{\boldsymbol{\theta}}
%  =\ket{\theta_1,\dots,\theta_N}\!,
%\)
%déterminés par leurs rapidités \(\boldsymbol{\theta}\).

À toute fonction régulière
\(
  f:\mathbb R\!\to\!\mathbb R
\)
on associe un opérateur-charge loclal :
\begin{eqnarray}\label{chap.2.charge.f.1}
	\operator{\mathcal{Q}}^{(\mathcal{S})}[f] & = &  L \int_0^L d\theta \, f(\theta) \operator{\rho}^{(\mathcal{S})}(\theta).	
\end{eqnarray}
où $\operator{\rho}(\theta)$ agit sur une état de Bethe comme 
\begin{eqnarray}\label{chap.2.rho.1}
	 \operator{\rho}(\theta) \ket{ \{ \theta_a \} } &=& \frac{1}{L} \sum_{a = 1 }^N  \delta ( \theta - \theta_a ) \ket{ \{ \theta_a \} }.	
\end{eqnarray}
De sorte que $\operator{\mathcal{Q}}^{(\mathcal{S})}[f]$ agit sur une état de Bethe comme
\begin{eqnarray}\label{chap.2.charge.1}
	\operator{\mathcal{Q}}^{(\mathcal{S})}[f]\,\ket{\{\theta_a\} } =  \sum_{a=1}^{N}f(\theta_a)\,\ket{\{\theta_a\} } \quad \mbox{de sorte que} \quad \braket{\operator{\mathcal{Q}}^{(\mathcal{S})}[f]}_{\{\theta_a\}} = \sum_{a=1}^N f(\theta_a)
\end{eqnarray}
Les choix particuliers
\(
  f_0(\theta)=1
\)
,
\(
  f_1(\theta)=\theta
\)
et
\(
  f_2(\theta)=\theta^{2}/2
\)
redonnent respectivement l'opérateur nombre \(\operator{Q}=\operator{Q}_0^{(\mathcal{S})} = \operator{\mathcal{Q}}^{(\mathcal{S})}[1]\) , impulsion \(\operator{P}=\operator{Q}_1^{(\mathcal{S})} = \operator{\mathcal{Q}}^{(\mathcal{S})}[\theta]\) et énergie cinétique
\(\operator{K}=\operator{Q}_2^{(\mathcal{S})} = \operator{\mathcal{Q}}^{(\mathcal{S})}[\theta^2/2]\). Et dans le cadre des (GGE), pour tous les ordres $i$ on note :
\begin{eqnarray}\label{chap.2.charge.ordre.i.1}
	\operator{Q}^{(\mathcal{S})}_i = \operator{\mathcal{Q}}^{(\mathcal{S})}[f_i]	, \quad \mbox{de sorte que} \quad \braket{\operator{Q}^{(\mathcal{S})}_i}_{\{\theta_a\}} = \sum_{a=1}^N f_i(\theta_a)  
\end{eqnarray}
avec les densités spectrales $f_i(\theta) \propto \theta^i$ . 

Ces charges sont extensives : leur densité locale $\operator{q}^{(\mathcal{S})}_{[f]}$ permet d’écrire
\(
  \operator{\mathcal{Q}}^{(\mathcal{S})}[f]=\int_0^{L}\!dx\;\operator{q}^{(\mathcal{S})}_{[f]}(x).
\)

\paragraph{Charges conservées généralisée.\label{sec:charges-gen}}
Les fonction $f_i$ étant fixées, on note la fonction régulière
\(
  w:\mathbb R\!\to\!\mathbb R
\)
–– dorénavant appelée \emph{poids spectral}, ou \emph{potentiel spectral} ––
\begin{eqnarray}
	w = \sum_i \beta_i f_i \label{chap.2.w.1},	
\end{eqnarray}
on associe un opérateur-charge généralisé $\operator{\mathcal{Q}}^{(\mathcal{S})}[w]$ :
\begin{eqnarray}\label{chap.2.charge.gen.1}
	\operator{\mathcal{Q}}^{(\mathcal{S})}[w]\,\ket{\{\theta_a\} } =  \sum_{a=1}^{N}w(\theta_a)\,\ket{\{\theta_a\} } \quad \mbox{de sorte que} \quad \braket{\operator{\mathcal{Q}}^{(\mathcal{S})}[w]}_{\{\theta_a\}} = \sum_{i} \beta_i  \braket{\operator{Q}^{(\mathcal{S})}_i}_{\{\theta_a\}}
\end{eqnarray}

%%%%%%%%%%%%%%%%%%%%%%%%%%%%%%%%%%%%%%%%%
\paragraph{Expression de la matrice densité généralisée.}
La matrice densité  s’écrit sous la forme :
L’ensemble général défini par $\operator{\varrho}^{(\mathcal{S})}[w]$ 
\begin{eqnarray}\label{chap.2.densite.1}
	\operator{\varrho}^{(\mathcal{S})}[w]  =  \frac{e^{-\operator{\mathcal{Q}}^{(\mathcal{S})}[w]}}{Z^{(\mathcal{S})}[w]}, \, \mbox{avec} \quad e^{-\operator{\mathcal{Q}}^{(\mathcal{S})}[w]}  = 	\sum_{\{\theta_a \}} e^{- \sum_{a = 1}^N w(\theta_a) } \vert \{ \theta_a\} \rangle \langle  \{ \theta_a\}  \vert, 
\end{eqnarray}	
	%pour une certaine fonction $w$ relié à la charge% $\operator{\mathcal{Q}} [w]  = \sum_{\{\theta_a \}} \left ( \sum_{a = 1}^N w ( \theta_a )  \right ) \vert \{ \theta_a \} \rangle \langle \{ \theta_a \} \vert $.
%où l'opérateur de charge associé à $w$ s’écrit :
%\begin{eqnarray}
%	\operator{\mathcal{Q}} [w]   & = &  \sum_{\{\theta_a \}} \left ( \sum_{a = 1}^N w ( \theta_a )  \right ) \vert \{ \theta_a \} \rangle \langle \{ \theta_a \} \vert,	
%\end{eqnarray}
et la fonction de partition \eqref{chap.TBA.op.Z.S} s'écrit $Z^{(\mathcal{S})}[w]\doteq \bm{\mathrm{Tr}}\left [ e^{-\operator{\mathcal{Q}}^{(\mathcal{S})}[w]}\right ] $ vaux :
\begin{eqnarray}
	Z^{(\mathcal{S})}[w]   =  \sum_{\{\theta_a \}} e^{-\sum_{a = 1}^N w(\theta_a)},\label{chap.TBA.op.Z.S.1}	
\end{eqnarray}
devient un Generalized Gibbs Ensemble (GGE), $\operator{\rho}^{(\mathcal{S})}_{\mathrm{GGE}}$ (de l'équation \eqref{chap.TBA.op.rho.S})	 dès lors que $w(\theta) = \sum_i \beta_i f_i(\theta)$ (de l'équation \eqref{chap.2.w.1}) où $f_i$ sont les densités spectrales associées aux charges locales conservées (de l'équation \eqref{chap.2.charge.ordre.i.1}).


%%%%%%%%%%%%%%%%%%%%%%%%%%%%%%%%%%
\paragraph{Probabilité associée à une configuration de rapidités.}
	%Et on peut réecrire la probabilité de la configuration $\{\theta_a\}$ :% $ P_{\{ \theta_a \}} = \langle \{ \theta_a \}\vert \operator{\rho}_{GGE}[w] \vert  \{ \theta_a \} \rangle = e^{-\sum_{a = 1}^N w(\theta_a)} / Z $ avec $Z = \sum_{\{\theta_a \}} e^{-\sum_{a = 1}^N w(\theta_a)}$.\\
	%La probabilité d’occuper un état à $N$ particules caractérisé par les rapidités ${\theta_a}$ est alors :
Dans ce formalisme, la probabilité d’occuper l’état $\ket{\{\theta \}}$ \eqref{chap.TBA.P.1} est donc
\begin{eqnarray}
	\mathbb{P}^{(\mathcal{S})}_{\{ \theta_a \}} & = &  Z^{(\mathcal{S})}[w]^{-1}e^{-\sum_{a = 1}^N w(\theta_a)}\label{chap.TBA.P.w.2}. 		
\end{eqnarray}
%Cela montre que le poids statistique d’une configuration factorise naturellement sur les pseudo-moments, avec un poids spectrale / energie génralisé $w(\theta)$ attribué à chaque particule.
On voit ainsi que le poids statistique factorise naturellement sur les
pseudo‑moments, chaque particule étant pondérée par $w(\theta_a)$.

%avec 
%\begin{eqnarray}
%	Z  & = & \sum_{\{\theta_a \}} e^{-\sum_{a = 1}^N w(\theta_a)}.		
%\end{eqnarray}


%%%%%%%%%%%%%%%%%%%%%%%%
\paragraph{Moyennes d'observables dans le GGE.}
%La valeur moyenne d’un observable locale $\operator{\mathcal{O}}$ dans l’ensemble généralisé s’écrit :
Pour tout opérateur local $\operator{\mathcal{O}}$ diagonal dans la base de Bethe,
la moyenne généralisée vaut
\begin{eqnarray}\label{chap.2.moyenne.1}
	\langle \operator{\mathcal{O}}\rangle_{\operator{\varrho}^{(\mathcal{S})}[w]} & = & \displaystyle   \frac{\sum_{\{\theta_a \}} \braket{ \operator{\mathcal{O}}}_{\{ \theta_a\}} e^{- \sum_{a = 1}^N w(\theta_a) }  }{\sum_{\{\theta_a  \}} e^{- \sum_{a = 1}^N  w(\theta_a) } }
\end{eqnarray}
%Cette expression formelle montre que la connaissance de $w(\theta)$ suffit à déterminer les propriétés statistiques de toutes les observables diagonales dans cette base, incluant les charges conservées elles-mêmes.
Ainsi, la connaissance de la fonction $w(\theta)$ suffit à déterminer
les propriétés statistiques de toute observable diagonale,
y compris les charges conservées elles‑mêmes.	
	% Nous aimerions calculer les valeurs d'attente par rapport à cette matrice de densité, par exemple
	%La moyenne GGE d'un observable s'écrit ,
	%\begin{aff}
	%\begin{eqnarray}
	%	\langle \operator{\mathcal{O}} \rangle_{GGE} & \doteq & \displaystyle  \text{Tr} (\operator{\mathcal{O}}\operator{\rho}[w]) = \frac{\text{Tr} (\operator{\mathcal{O}}e^{-\operator{\mathcal{Q}}[w]})}{\text{Tr} (e^{-\operator{\mathcal{Q}}[w]})}	 = \frac{\sum_{\{\theta_a \}} \langle  \{ \theta_a\}  \vert   \operator{\mathcal{O}} \vert \{ \theta_a\} \rangle e^{- \sum_{a = 1}^N w(\theta_a) }  }{\sum_{\{\theta_a  \}} e^{- \sum_{a = 1}^N  f(\theta_a) } }
		%& =  & \frac{ \sum_{\pi} \sum_{\vert \{\theta_a \}\rangle \vert \Pi } \langle  \{ \theta_a\}  \vert   \operator{\mathcal{O}} \vert \{ \theta_a\} \rangle e^{- \sum_{a = 1}^N f(\theta_a) }  }{\sum_{\pi} \sum_{\vert \{\theta_a \}\rangle \vert \Pi }  e^{- \sum_{a = 1}^N  f(\theta_a) } }
	%\end{eqnarray}
	%pour une certaine observable $\operator{\mathcal{O}}$.\\
	%\end{aff}
	

\paragraph{Conclusion de la section : vers la thermodynamique de Bethe.}

Nous avons vu que, dans un système intégrable, la description correcte de l’équilibre stationnaire requiert l’introduction d’une \emph{famille infinie de charges conservées}, comprenant à la fois des charges strictement locales et des charges quasi‑locales.
Toutes ces charges se réunissent dans l’opérateur fonctionnel
\(
\operator{\mathcal{Q}}^{(\mathcal{S})}[w]
\)
, défini par un \emph{poids spectral}  $w(\theta)$ (cf. équations~\eqref{chap.2.charge.1}).
Cette construction conduit naturellement à la matrice densité généralisée
\(
\operator{\rho}^{(\mathcal{S})}_{\mathrm{GGE}}  \propto  e^{-\operator{\mathcal{Q}}^{(\mathcal{S})}[w]}
\) 
(cf. équations~\eqref{chap.2.densite.1}), et à la moyenne d’un opérateur local $\operator{\mathcal{O}}$ donnée par
\(
\langle \operator{\mathcal{O}}\rangle_{\operator{\rho}^{(\mathcal{S})}_{\mathrm{GGE}}}  =  \displaystyle  \text{Tr} (\operator{\mathcal{O}}\operator{\varrho}^{(\mathcal{S})}[w])
\)
(cf. équations~\eqref{chap.2.moyenne.1}).
La connaissance de $w(\theta)$ suffit donc pour prédire les valeurs moyennes de toutes les observables diagonales, y compris celles des charges elles‑mêmes ; c’est le cœur du {\bf Ensemble de Gibbs Généralisé (GGE pour Generalized Gibbs Ensemble)} .

\medskip
Cette base est désormais posée : dans la section suivante, nous passerons au \emph{thermodynamique de Bethe}.
Nous verrons comment, dans la limite thermodynamique, les sommes sur les configurations de rapidités se transforment en intégrales sur des densités continues, comment apparaît l’entropie de Yang–Yang, et comment les moyennes de l’ensemble généralisé se réexpriment à l’aide de ces densités macroscopiques.
C’est ce formalisme qui permettra d’analyser finement la relaxation post‑quench et de relier microscopie intégrable et hydrodynamique généralisée.



%\input{preamble}

\begin{document}

\frontmatter
%\input{chapters/00_intro}
\tableofcontents
\mainmatter

\input{chapters/01_LL_BA}
\input{chapters/02_GGE_TBA}
\input{chapters/03_GHD}
%\input{chapters/97_GHD}
\input{chapters/04_GGE_Fluctuation}
\input{chapters/05_Disp_Exp}
\input{chapters/06_Bipart}
\input{chapters/07_Dipolaire}

%\input{chapters/08_conclusion}
%\appendix
%\input{chapters/99_annexes}

\bibliographystyle{abbrv}
\bibliography{thesis}

%\printbibliography

\end{document}

%| Style     | Description                                                             |
%| --------- | ----------------------------------------------------------------------- |
%| `plain`   | Tri alphabétique, numérotation croissante                               |
%| `unsrt`   | Même que `plain` mais sans tri, respecte l’ordre d’apparition           |
%| `abbrv`   | Comme `plain` mais avec prénoms et noms abrégés                         |
%| `alpha`   | Les références sont étiquetées par une combinaison du nom et de l’année |
%| `apalike` | Style APA simplifié                                                     |
%| `ieeetr`  | Style IEEE, tri par ordre d’apparition                                  |
%| `siam`    | Style SIAM (mathématiques appliquées)                                   |
%| `acm`     | Style ACM (informatique)                                                |
%




\section{Thermodynamique de Bethe et relaxation}

%------------------------------------------------------------------
\subsection{Limite thermodynamique}

\paragraph{Observables locales dans la limite thermodynamique.}
%Lorsque l'observable $\operator{\mathcal{O}}$ est suffisamment local, on croit que la valeur d'attente $\langle  \{ \theta_a\}  \vert   \mathcal{O} \vert \{ \theta_a\} \rangle$ ne dépend pas de l'état microscopique spécifique du système, de sorte qu'elle devient une fonctionnelle de $\Pi$ dans la limite thermodynamique.
Dans la suite de ce chapitre, nous omettrons l’exposant $(\mathcal{S})$.
\vspace{0.2em}
Dans la base des états de Bethe \( \{ \ket{\{ \theta_a \}} \} \), l’opérateur \( \hat{\rho}(\theta) \) défini en \eqref{chap.2.rho.1} est diagonal, et agit comme un projecteur sur les valeurs de rapidité.

\vspace{0.5em}

Dans la limite thermodynamique, différentes configurations microscopiques \( \{ \theta_a \} \) peuvent correspondre à la même distribution de rapidité macroscopique \( \rho(\theta) \). Autrement dit, plusieurs états \( \ket{\{ \theta_a \}} \) partagent la même valeur propre \( \rho(\theta) \) de l’opérateur \( \operator{\rho}(\theta) \). Cela reflète une {\em dégénérescence macroscopique} induite par le passage à la limite thermodynamique (\( N, L \to \infty \) avec \( N/L \to \text{const} \)).

\vspace{0.5em}

Si l’observable $\mathcal{O}$ est suffisamment locale, sa valeur d’attente dans un état propre ne dépend pas des détails microscopiques, mais uniquement de la distribution de rapidité. On écrit alors :
\begin{eqnarray}
	\underset{\mbox{\tiny therm.}}{\lim} \braket{  \operator{\mathcal{O}} }_{\{ \theta_a\}}  & = & \langle \operator{\mathcal{O}}\rangle_{[\rho]},
\end{eqnarray}
où $\underset{\mbox{\tiny therm.}}{\lim}$ est la limite thermodynamique ($N,L \to \infty$ avec $N/L \to $ const) et où \( \langle \mathcal{O} \rangle_{[\rho]} \) désigne la valeur d’attente de \( \mathcal{O} \) dans un état macroscopique caractérisé par la distribution de rapidité \( \rho(\theta) \).


\medskip
Dans un ensemble général (GGE), la valeur moyenne de l’observable \eqref{chap.2.moyenne.1} devient alors :		
\begin{eqnarray}\label{chap.2.moyenne.2}
	\underset{\mbox{\tiny therm.}}{\lim} \langle \operator{\mathcal{O}} \rangle_{\operator{\varrho}[w]} & =  & \frac{  \displaystyle \sum_{\rho }  \langle \operator{\mathcal{O}}\rangle_{[\rho]} \Omega[\rho] e^{- \sum_{a = 1}^N  w(\theta_a)    }}{ \displaystyle \sum_{\rho}   \Omega[\rho]\,e^{- \sum_{a = 1}^N  w(\theta_a) } } ,
\end{eqnarray}
où $\sum_{\rho }$ est une somme sus tous les distribution de rapidité $\rho$ et 
où $\Omega[\rho]$ désigne le nombre de micro-états compatibles avec la distribution de rapidité $\rho$.

%où $\# \mbox{micro-états.}$ est les nombre de micro état associée àa la distribution de rapidité $\rho$.
%Avant de se plonger sur $\# \mbox{micro-états.}$, regardons le changement des équation de Bethes. 

\medskip
Pour établir la fonction $\Omega[\rho]$, reppelons-nons de la transformation des équations de Bethe dans dans la limite thermodynamique, hors état fondamentale \eqref{eq:TBA-nu} et \eqref{eq:TBA-rhos-2}.
\begin{equation}
	\nu = \frac{\rho}{\rho_s} \, , \qquad 2\pi \rho_s = 1^{\mathrm{dr}}_{[\nu]} 
\label{chap.2:eq:TBA-rhos}
\end{equation}
où $f^{\mathrm{dr}}_{[\nu]}$ est définie en \eqref{eq:dessing}.

\medskip

Cette formalisation constitue la brique de base de la \textbf{hydrodynamique généralisée} et, dans la section suivante, permet de définir rigoureusement l’\textbf{entropie de Yang–Yang}, indispensable pour décrire la relaxation hors d’équilibre des systèmes intégrables.

%\vspace{1ex}
%La formalisation ci‑dessus fournit la brique de base pour la
%\textbf{hydrodynamique généralisée} et, dans la section suivante, pour la
%définition précise de l’\textbf{entropie de Yang-Yang}
%assurant la relaxation des systèmes intégrables hors‑équilibre.

%\input{preamble}

\begin{document}

\frontmatter
%\input{chapters/00_intro}
\tableofcontents
\mainmatter

\input{chapters/01_LL_BA}
\input{chapters/02_GGE_TBA}
\input{chapters/03_GHD}
%\input{chapters/97_GHD}
\input{chapters/04_GGE_Fluctuation}
\input{chapters/05_Disp_Exp}
\input{chapters/06_Bipart}
\input{chapters/07_Dipolaire}

%\input{chapters/08_conclusion}
%\appendix
%\input{chapters/99_annexes}

\bibliographystyle{abbrv}
\bibliography{thesis}

%\printbibliography

\end{document}

%| Style     | Description                                                             |
%| --------- | ----------------------------------------------------------------------- |
%| `plain`   | Tri alphabétique, numérotation croissante                               |
%| `unsrt`   | Même que `plain` mais sans tri, respecte l’ordre d’apparition           |
%| `abbrv`   | Comme `plain` mais avec prénoms et noms abrégés                         |
%| `alpha`   | Les références sont étiquetées par une combinaison du nom et de l’année |
%| `apalike` | Style APA simplifié                                                     |
%| `ieeetr`  | Style IEEE, tri par ordre d’apparition                                  |
%| `siam`    | Style SIAM (mathématiques appliquées)                                   |
%| `acm`     | Style ACM (informatique)                                                |
%






\subsection{Statistique des macro-états : entropie de Yang-Yang}

%\paragraph{Macro-états et entropie dans la TBA.}

%Dans la limite thermodynamique, dans le modèle statistique (GGE) , les moyenne, observables physiques deviennent des fonctionnelles de la {\bf distribution de rapidité}  $\rho(\theta)$ et du {\bf poing spectrale} $w(\theta)$ . Cette description est efficace car elle permet d’échapper au détail de chaque état propre. 
%Toutefois, cette simplification laisse en suspens une question cruciale : 
%Mais dans ce modelle qui simplifie on veux {\bf la distribution de rapidité d’un système à l'équilibre thermique à température finie} que l'on notera $\langle \rho \rangle$ pour dire la dansité moyenne. Et les lien entre  $w$ et $\langle \rho \rangle$.  Le problème est étudier par par Yang et Yang en 1969. Pour saisir l'enssentielle, nous devons comprendre la {\bf structure statistique des états propres} associés à une même distribution $\rho(\theta)$. Nous nous interrensons comme promis plus haut : à $\Omega(\theta)$ dans l'équation de moyenne \eqref{chap.2.moyenne.2}  ,  {\bf  nombre états propres microscopiques correspondent à une même distribution $\rho(\theta)$}.
%{\bf quelle est la distribution de rapidité d’un système à l'équilibre thermique à température finie ?}. 
%La question a été répondue dans les travaux pionniers de Yang et Yang (1969), que nous allons maintenant examiner brièvement. Pour répondre à cette question, nous devons comprendre la {\bf structure statistique des états propres} associés à une même distribution $\rho(\theta)$.

\paragraph{Motivation.}

Dans la limite thermodynamique, une observable locale dans un \textit{Generalized Gibbs Ensemble} (GGE) dépend uniquement de deux objets continus :  (i)  la \textbf{distribution de rapidité} $\rho(\theta)$, (ii) le \textbf{poids spectral} $w(\theta)$, c.-à-d.\ la " température généralisée " assignée à chaque quasi‑particule.
Cette reformulation est puissante car elle fait disparaître les détails d’un état propre individuel.  

\medskip
Cependant, pour décrire un \emph{vrai} équilibre à température finie, il faut la distribution à l'équilibre :
\begin{eqnarray}\label{chap.2:eq.rho.eq.1}
	\rho_{\mathrm{eq}}(\theta)\;\doteq\;\braket{\operator{\rho}(\theta)}_{\operator{\varrho}[w]}	,  
\end{eqnarray}
donc le lien entre $\rho_{\mathrm{eq}}$ et $w$.
La réponse fut donnée dans les travaux pionniers de \textsc{Yang \& Yang} (1969).  
Leur approche repose sur l’analyse de la \textbf{structure statistique des états propres} partageant la même distribution $\rho(\theta)$.

% : combien d’états microscopiquement distincts correspondent à ce même « macro‑état » ?

\paragraph{Distribution de rapidité comme macro-état.}

Chaque distribution de rapidité $\rho(\theta)$ ne correspond pas à un état propre unique, mais à un grand {\bf ensemble de micro-états} : différents choix des ensembles de quasi-moments $(\{\theta_a\}_{a \in \llbracket 1 , N \rrbracket })_{N \in \mathbb{Z}} $ peuvent conduire à la même densité de distribution à l’échelle macroscopique. Ainsi, $\rho(\theta)$ doit être interprétée comme un {\bf macro-état}, qui agrège un très grand nombre d’états propres microscopiques.

La question thermodynamique devient alors : {\bf Combien de micro-états microscopiquement distincts sont compatibles avec un même macro-état $\rho(\theta)$ ?} 

\medskip
Plus précisément, dans l’expression de moyenne des operateurs locaux \eqref{chap.2.moyenne.2}, apparaît le facteur
\(
\Omega[\rho]
\),
qui compte ces états propres.  
La détermination de $\Omega[\rho]$ (ou équivalemment de l’entropie de Yang–Yang $\mathcal{S}_{YY}[\rho]$ car 
\(
\Omega[\rho] = e^{L\mathcal{S}_{YY}[\rho]}
\)
avec $L$ la taille du système
) est donc la clé pour relier \emph{(i)} le poids spectral $w(\theta)$ imposé dans le GGE et \emph{(ii)} la distribution de rapidité moyenne $\rho_{\mathrm{eq}}(\theta)$ observée à l’équilibre.

\paragraph{Dénombrement local des configurations microcanoniques.}
Pour répondre à cette question, on subdivise l’axe des rapidités en petites tranches ou cellules de largeur $\delta \theta$, chacune centrée en un point $\theta_a$. Dans une tranche $[\theta_a, \theta_a + \delta\theta]$, on suppose que la densité $\rho(\theta)$ est à peu près constante. Le nombre de quasi-particules dans cette tranche est alors approximativement :
\begin{eqnarray*}
	N_a = L\rho(\theta_a) \delta \theta,
\end{eqnarray*}
et le nombre total d'états disponibles (\ie, le nombre d’états possibles si toutes les positions en moment étaient disponibles) est donné par la densité totale de niveaux 
\begin{eqnarray*}
	M_a = L\rho_s(\theta_a) \delta \theta.
\end{eqnarray*}
%La densité de niveaux $\rho_s(\theta)$ tient compte du fait que les moments sont quantifiés de manière discrète, en raison des équations de Bethe (voir équation (??)).

Les particules occupent ces niveaux de manière analogue à des fermions libres (principe d’exclusion de Pauli), le nombre de manières différentes de choisir $N_a$ niveaux parmi $M_a$ est donné par :
	
	
	\begin{figure}[H]
		\centering 
		\begin{tikzpicture}
			%\input{figures/04_GGE_Fluctuation/Occupation_code}	
			\begin{scope}[transform canvas={scale=0.6}]
			\input{figures/04_GGE_Fluctuation/Occupation_theta_code}	
			\end{scope}
			
			\draw[color = red , scale = 0.5 , draw = none ] (-13.5 , -1) rectangle (13 , 10) ; 
				
			
		\end{tikzpicture}	
		\captionsetup{skip=10pt} % Ajoute de l’espace après la légende
	\end{figure}
	
	
\begin{eqnarray}
	\Omega(\theta_a) & \approx  & \binom{M_a}{N_a} ~= ~   \frac{[ L\rho_s ( \theta ) \delta \theta ] ! }{ [ L\rho ( \theta ) \delta \theta ] ! [( L\rho_s ( \theta ) - L\rho ( \theta ) )  \delta \theta ] ! }. 	
\end{eqnarray}

\paragraph{Estimation asymptotique à l’aide de Stirling.}

En utilisant la formule de Stirling :
\begin{eqnarray}
	n! & \underset{n \to \infty}{\sim} &  n^n e^{-n} \sqrt{2\pi n}.,
\end{eqnarray}	
composé du fonction logarithmique, il vient cette équivalence : 
\begin{eqnarray}
	\ln n! & \underset{n \to \infty}{\rightarrow} & n \ln n \underbrace{- n + \ln \sqrt{2 \pi n }}_{o \left ( n \ln n \right ) } ,\\
	&  \underset{n \to \infty}{\sim} & n \ln n  
\end{eqnarray}
	
$\# \mbox{conf.}$ est jamais null donc on peut approximer, pour de grandes valeurs de $L$ et de $\delta\theta$  : 
\begin{eqnarray}
    \ln \Omega(\theta) & \underset{\underset{\rho (\theta )\leq  \rho_s (\theta )}{\rho \delta \theta  \to \infty}}{\sim}   & L [ \rho_s\ln \rho_s - \rho \ln \rho - (\rho_s - \rho ) \ln ( \rho_s - \rho) ] (\theta )\delta \theta .
\end{eqnarray}

Cette expression donne la contribution par unité de $\theta$ à l’{\bf entropie}  associée à la cellule autour de $\theta_a$.

\paragraph{Entropie de Yang-Yang : définition .}
%L'entropie totale du macro-état $\rho(\theta)$, notée $\mathcal{S}_{YY}[\rho]$, est obtenue en sommant sur toutes les tranches. Pour alléger la notation, nous écrivons cette somme comme :
%Le nombre total de micro-états est le produit de toutes ces configurations pour toutes les cellules de rapidité $[\theta, \theta + \delta \theta]$. %En prenant le logarithme et en remplaçant la somme par une intégrale sur $ \theta$, nous obtenons l'entropie de Yang-Yang :

%L'entropie totale du macro-état $\rho(\theta)$, notée $\mathcal{S}_{YY}[\rho]$, est obtenue en sommant sur toutes les tranches. Pour alléger la notation, nous écrivons cette somme comme :
%Le nombre total de micro-états compatibles avec une distribution macroscopique $\rho(\theta)$ est donné par le produit des nombres de configurations pour chaque cellule de rapidité $[\theta, \theta + \delta \theta]$.

%En prenant la sum le logarithme des $\Omega(\theta)$ , on obtient l'entropie totale de Yang-Yang. Pour alléger la notation, cette somme sur les tranches est notée :

Le nombre total de micro-états compatibles avec une distribution macroscopique donnée $\rho(\theta)$ est obtenu en prenant le produit des nombres de configurations pour chaque cellule de rapidité $[\theta_a, \theta_a + \delta \theta]$ : $ \Omega(\theta_a)$ .
En prenant le logarithme de ce produit, on accède à l'entropie totale. Pour alléger la notation, cette somme sur les cellules est notée
\(
	\sum_a^{\theta-\mbox{\tiny cellules}}	
\)
où chaque $a$ indexe une cellule de rapidité $[\theta_a, \theta_a + \delta\theta]$.
On écrit alors :
\begin{eqnarray}
    \ln \Omega[\rho] & = & \sum_a^{\theta-\mbox{\tiny cellules}} \ln \Omega(\theta_a), \\
    & \approx &   L\mathcal{S}_{YY} [ \rho ] , 	
\end{eqnarray}
où l’on définit l’\textbf{entropie de Yang–Yang} par la formule discrétisée :
\begin{eqnarray}
    \mathcal{S}_{YY}[\rho] & \doteq & \sum_a^{\theta-\mbox{\tiny cellules}} \, [ \rho_s\ln \rho_s - \rho \ln \rho - ( \rho_s - \rho ) \ln ( \rho_s - \rho ) ] (\theta_a) \delta \theta .
\end{eqnarray}

%\paragraph{Énergie généralisée.}	
%Les variations de $w(\theta)$ étant négligeables sur chaque tranche de largeur $\delta\theta$, on peut approximer l’énergie généralisée comme :%  $\sum_{a = 1}^N  f(\theta_a) = \sum_{a \vert tranche } f(\theta_a) \Pi( \theta_a)\delta \theta$.

%\begin{eqnarray}
%	 \mathcal{W} & = & \sum_{a = 1}^N  w(\theta_a)	 ~ \sim ~ L\mathcal{W}[\rho] ~=~ L \sum_a^{\theta-\mbox{\tiny tranches}}	 w(\theta_a) \rho(\theta_a) \delta \theta.
%\end{eqnarray}

\paragraph{Énergie généralisée par unité de longueur : définition.}

Dans le cadre du Generalized Gibbs Ensemble (GGE), l’\textbf{énergie généralisée} associée à une distribution de rapidité $\rho(\theta)$ et à un poids spectral $w(\theta)$ est définie comme la somme des poids assignés à chaque quasi-particule. 
Dans la limite thermodynamique, en supposant que $w(\theta)$ varie lentement sur chaque tranche $[\theta_a, \theta_a + \delta\theta]$ ,  cette somme soit l’\textbf{énergie généralisée par unité de longueur} $\mathcal{W}$ se se définit par :
\begin{eqnarray}
	L \mathcal{W}(\{\theta_a\}) \doteq  \sum_{a = 1}^N w(\theta_a) 
	 \underset{\mbox{\tiny therm .}}{\sim}  L \mathcal{W}[\rho]  \doteq  L \sum_a^{\theta\text{-cellules}} w(\theta_a) \rho(\theta_a)\, \delta\theta. 
\end{eqnarray} 
%La fonctionnelle
%\(
%\mathcal{W}[\rho] = \int d\theta\, w(\theta)\, \rho(\theta)
%\)
%représente donc l’énergie généralisée par unité de longueur, dans l’état macroscopique défini par la distribution $\rho$.


\paragraph{Moyenne des Observables locales dans la limite thermodynamique.}

Dans un ensemble général (GGE), la valeur moyenne de l’observable \eqref{chap.2.moyenne.2} devient :	
	
\begin{eqnarray}\label{chap.2.moyenne.3}
	\underset{\mbox{\tiny therm.}}{\lim} \langle \operator{\mathcal{O}} \rangle_{\operator{\varrho}[w]} &  \approx &  ~ \frac{  \displaystyle \sum_{\rho }  \langle \operator{\mathcal{O}}\rangle_{[\rho]}  e^{L(\mathcal{S}_{YY}[\rho] -  \mathcal{W}[\rho]) }}{ \displaystyle \sum_{\rho } e^{L(\mathcal{S}_{YY}[\rho] -  \mathcal{W}[\rho]) } },
\end{eqnarray}
où la somme $\sum\rho$ porte sur toutes les distributions possibles de rapidité $\rho$

%%%%%%%%%%%%%%%%%%%%%%%%%%%%%%%%%%%%%%%%%
\paragraph{Passage à la limite continue.}
%En faisant tandre $\delta \theta \to 0 $ , les somme devienen des integrales 
En faisant tendre $\delta\theta \to 0$, les sommes deviennent des intégrales 
%\(
%\sum_a^{\theta-\mbox{\tiny tranches}}\delta \theta   \underset{\delta \theta \to 0 }{\rightarrow}  \int d \theta ,	
%\)
et l'entropie de Yang-Yang ainsi que l’énergie généralisée par unité de longueur prennent la forme :
\begin{eqnarray}
	\mathcal{S}_{YY}[\rho] & = & \int d \theta  \, [ \rho_s\ln \rho_s - \rho \ln \rho - ( \rho_s - \rho ) \ln ( \rho_s - \rho ) ] (\theta) , \label{chap.2.entropi.int}\\
	\mathcal{W}[\rho] & = & \int   w(\theta) \rho(\theta) \, d \theta \label{chap.2.W.int}		
\end{eqnarray}

%%%%%%%%%%%%%%%%%%%%%%%%%%%%%%%%%%%%%%%%%%
\paragraph{Formule fonctionnelle pour les moyennes.}

%et la valeur moyenne des opservables $\langle \operator{\mathcal{O}} \rangle$ s'écrit commes une intégrale de chemin/formelle
Dans la limite thermodynamique $L \to \infty$, la somme sur les distributions de rapidité $\rho$ admissibles peut être approximée par une intégrale fonctionnelle sur l’espace des densités de rapidité continues, munie d’une mesure fonctionnelle $\mathcal{D}\rho$ : 
\(
\sum_{\rho } \sim \int \mathcal{D} \rho .
\)
Cette correspondance repose sur l’idée que les macro-états admissibles deviennent denses dans l’espace fonctionnel, et que le poids statistique associé à chaque configuration est donné par l’entropie de Yang–Yang.
La mesure fonctionnelle $\mathcal{D}\rho$ parcourt l’espace des densités
$\rho(\theta)$ continues, \emph{chaque configuration étant pondérée par le
facteur exponentiel}
\(
e^{\,L(\mathcal{S}_{YY}[\rho]-\mathcal{W}[\rho])}.
\)
Finalement, la moyenne d'une observable dans le GGE \eqref{chap.2.moyenne.3} s’écrit comme une intégrale fonctionnelle/de chemin :
\begin{eqnarray}
	\underset{\mbox{\tiny therm.}}{\lim} \langle \operator{\mathcal{O}} \rangle_{\operator{\varrho}[w]} & = & \frac{\int \mathcal{D} \rho \; e^{L (\mathcal{S}_{YY}[\rho] - \mathcal{W}[\rho])} \, \langle\operator{\mathcal{O}}\rangle_{[\rho]}}{\int \mathcal{D} \rho \; e^{L (\mathcal{S}_{YY}[\rho] - \mathcal{W}[\rho])}}. \label{chap:TBA:eq:ensemble_average}
\end{eqnarray}


%----------------------
%------------------------------------------------------------------
%\paragraph{Passage de la somme discrète à l’intégrale fonctionnelle.}

%Dans la limite thermodynamique $L\to\infty$, l’ensemble (discret) des
%distributions de rapidité admissibles devient dense dans l’espace
%fonctionnel ; la somme correspondante peut donc s’approximer par une
%intégrale fonctionnelle :
%\[
%\sum_{\rho}\; \longrightarrow\; \int\! \mathcal{D}\rho .
%\]
%La mesure fonctionnelle $\mathcal{D}\rho$ parcourt l’espace des densités
%$\rho(\theta)$ continues, \emph{chaque configuration étant pondérée par le
%facteur exponentiel}
%\(
%e^{\,L\bigl[\mathcal{S}_{YY}[\rho]-\mathcal{W}[\rho]\bigr]},
%\)
%qui combine
%\begin{itemize}
%\item l’\textbf{entropie de Yang–Yang}
%      $\displaystyle
%        \mathcal{S}_{YY}[\rho]
%        =\!\int d\theta\,
%          \bigl[
%            \rho_s\ln\rho_s
%            -\rho\ln\rho
%            -(\rho_s-\rho)\ln(\rho_s-\rho)
%          \bigr]$,
%\item le \textbf{coût énergétique généralisé}
%      $\displaystyle
%        \mathcal{W}[\rho]
%        =\!\int d\theta\, w(\theta)\,\rho(\theta)$,
%\end{itemize}
%où $w(\theta)$ est le \emph{poids spectral} fixé par le GGE.

%------------------------------------------------------------------
%\paragraph{Moyenne d’une observable dans le GGE.}

%On obtient alors la formule de champ moyen
%\begin{equation}\label{eq:GGE-functional-average}
%\bigl\langle\mathcal{O}\bigr\rangle_{\!{\rm GGE}}
%=
%\frac{\displaystyle
%      \int \mathcal{D}\rho\;
%      e^{L\bigl[\mathcal{S}_{YY}[\rho]-\mathcal{W}[\rho]\bigr]}\,
%      \langle\mathcal{O}\rangle_{[\rho]}}
%     {\displaystyle
%      \int \mathcal{D}\rho\;
%      e^{L\bigl[\mathcal{S}_{YY}[\rho]-\mathcal{W}[\rho]\bigr]}}.
%\end{equation}

%------------------------------------------------------------------
\paragraph{Interprétation thermodynamique.}

\begin{itemize}[label = $\bullet$] 
\item $\mathcal{S}_{YY}[\rho]$ \emph{compte} le logarithme du nombre de
      micro-états réalisant la distribution $\rho(\theta)$ :
      c’est l’\textbf{entropie combinatoire}.
\item $\mathcal{W}[\rho]$ mesure le \emph{coût énergétique généralisé}
      associé à cette distribution, dicté par le poids spectral $w(\theta)$.
\end{itemize}

Leur différence
\[
(\mathcal{S}_{YY}-\mathcal{W})[\rho]
\]
joue donc le rôle d’une \emph{fonction thermodynamique effective}
(analogue à une entropie libre).  
L’exposant $e^{L(\mathcal{S}_{YY}-\mathcal{W})[\rho]}$ fixe la \textbf{probabilité relative} d’un
macro-état $\rho(\theta)$ dans le GGE : le terme entropique favorise la
multiplicité des états, tandis que le terme énergétique pénalise les
configurations coûteuses — d’où la compétition caractéristique de
l’équilibre statistique.





%avec $\mathcal{O}[\rho]$ la valeur de l’observable dans un état propre caractérisé par la distribution de rapidité $\rho$.	
%où $\mathcal{O}[\rho]$ est la valeur de l’observable dans un état propre caractérisé par la distribution $\rho$.

%\input{preamble}

\begin{document}

\frontmatter
%\input{chapters/00_intro}
\tableofcontents
\mainmatter

\input{chapters/01_LL_BA}
\input{chapters/02_GGE_TBA}
\input{chapters/03_GHD}
%\input{chapters/97_GHD}
\input{chapters/04_GGE_Fluctuation}
\input{chapters/05_Disp_Exp}
\input{chapters/06_Bipart}
\input{chapters/07_Dipolaire}

%\input{chapters/08_conclusion}
%\appendix
%\input{chapters/99_annexes}

\bibliographystyle{abbrv}
\bibliography{thesis}

%\printbibliography

\end{document}

%| Style     | Description                                                             |
%| --------- | ----------------------------------------------------------------------- |
%| `plain`   | Tri alphabétique, numérotation croissante                               |
%| `unsrt`   | Même que `plain` mais sans tri, respecte l’ordre d’apparition           |
%| `abbrv`   | Comme `plain` mais avec prénoms et noms abrégés                         |
%| `alpha`   | Les références sont étiquetées par une combinaison du nom et de l’année |
%| `apalike` | Style APA simplifié                                                     |
%| `ieeetr`  | Style IEEE, tri par ordre d’apparition                                  |
%| `siam`    | Style SIAM (mathématiques appliquées)                                   |
%| `acm`     | Style ACM (informatique)                                                |
%



\subsection{Équations intégrales de la TBA}

\paragraph{Moyenne des observables dans l’ensemble généralisé de Gibbs.}

\paragraph{Approximation au point selle («\,méthode de la selle statique\,»)}

Dans la limite thermodynamique \( L \to \infty \), cette intégrale est dominée par la configuration \( \rho_{eq} \) qui maximise le poids exponentiel $e^{L(\mathcal{S}_{YY}-\mathcal{W})[\rho]}$  dans l'expression \eqref{chap:TBA:eq:ensemble_average}. Il s’agit de la densité de rapidité la plus probable, solution d’un problème de maximisation. On obtient à l’ordre principal
\begin{eqnarray}
	\underset{\mbox{\tiny therm.}}{\lim} \langle \operator{\mathcal{O}} \rangle_{\operator{\varrho}[w]} & \approx &  \langle\operator{\mathcal{O}}\rangle_{[\rho_{eq} ]},	
	\label{chap:TBA:eq:ensemble_average:approx}
\end{eqnarray}
où $\rho_{eq}$ est la distribution de rapidité à l'équilibre \eqref{chap.2:eq.rho.eq.1}.
Cette approximation correspond à une méthode de \textit{selle statique}, où l’on développe la \emph{fonction thermodynamique effective}, $\mathcal{S}_{YY}-\mathcal{W}$  au voisinage de la distribution dominante.


\paragraph{Développement fonctionnel au premier ordre.}

%On effectue un développement de Taylor fonctionnel de l'action à l’ordre linéaire en $\rho = \rho_{eq} + \delta \rho$ :
Écrivons
\(
\rho=\rho_{\text{eq}}+\delta\rho
\)
et développons $(\mathcal{S}_{YY}-\mathcal{W})[\rho]$ à l’ordre linéaire :
\begin{eqnarray*}
	\mathcal{S}_{YY}[\rho] - \mathcal{W}[\rho] & \approx & \mathcal{S}_{YY}[ \rho_{eq}] - \mathcal{W}[ \rho_{eq}] +  \left. \frac{\delta (\mathcal{S}_{YY}[\rho] - \mathcal{W}[\rho]) }{\delta \rho} \right|_{\rho = \rho_{eq} }	(\delta \rho) + \mathcal{O}(\delta \rho^2 ) ,
	\label{chap:TBA:eq:action}	
\end{eqnarray*}	
La condition de stationnarité au point selle impose :
\(
	\left. \frac{\delta (\mathcal{S}_{YY}[\rho] - \mathcal{W}[\rho]) }{\delta \rho} \right|_{\rho = \rho_{eq} }	  =  0  	
\)
soit 
\begin{equation}
\left. \frac{\delta \mathcal{S}_{YY}}{\delta \rho} \right|_{\rho = \rho_{eq}} = \left. \frac{\delta \mathcal{W}}{\delta \rho} \right|_{\rho = \rho_{eq}}. \label{chap:TBA:eq:stationnarite}
\end{equation}

%%%%%%%%%%%%%%%%
%-----------------------------------------------------

%------------------------------------------------------------------
%\subsection{Équations intégrales de la TBA}

%\paragraph{Moyenne des observables dans le Generalized Gibbs Ensemble.}

%Dans la limite thermodynamique, la moyenne d’une observable locale
%s’écrit formellement comme une intégrale fonctionnelle sur les densités de
%rapidité\,\footnote{%
%La mesure fonctionnelle $\mathcal{D}\rho$ est la limite continue de la
%somme discrète sur les macro-états admissibles, chacun étant pondéré par
%le facteur combinatoire $e^{L\mathcal{S}_{YY}[\rho]}$.}
%
%\begin{equation}\label{eq:TBA:ensemble_average}
%\left\langle \mathcal{O} \right\rangle_{\!\text{GGE}}
%=\frac{\displaystyle
%      \int\!\mathcal{D}\rho\;
%      e^{L\bigl[\mathcal{S}_{YY}[\rho]-\mathcal{W}[\rho]\bigr]}\;
%      \langle\mathcal{O}\rangle_{[\rho]}}
%     {\displaystyle
%      \int\!\mathcal{D}\rho\;
%      e^{L\bigl[\mathcal{S}_{YY}[\rho]-\mathcal{W}[\rho]\bigr]}} .
%\end{equation}

%------------------------------------------------------------------
%\paragraph{Approximation au point selle («\,méthode de la selle statique\,»).}

%Lorsque $L\to\infty$, les intégrales \eqref{eq:TBA:ensemble_average}
%sont dominées par la distribution
%$\rho_{\text{eq}}$ qui \emph{maximise} l’exposant
%\(
%\Phi[\rho]=\mathcal{S}_{YY}[\rho]-\mathcal{W}[\rho].
%\)
%On obtient à l’ordre principal
%\begin{equation}
%\left\langle \mathcal{O} \right\rangle_{\!\text{GGE}}
%\;\simeq\;
%\langle \mathcal{O} \rangle_{[\rho_{\text{eq}}]} .
%\label{eq:TBA:saddle_average}
%\end{equation}

%------------------------------------------------------------------
%\paragraph{Condition de stationnarité et équation variationnelle.}

%Écrivons
%\(
%\rho=\rho_{\text{eq}}+\delta\rho
%\)
%et développons $\Phi[\rho]$ à l’ordre linéaire :
%\[
%\Phi[\rho]\;=\;
%\Phi[\rho_{\text{eq}}]
%+
%\int d\theta\,
%\left.
%\frac{\delta\Phi}{\delta\rho(\theta)}
%\right|_{\rho_{\text{eq}}}
%\delta\rho(\theta)
%+O(\delta\rho^{2}).
%\]
%La stationnarité impose
%\(
%\dfrac{\delta\Phi}{\delta\rho(\theta)}\bigl|_{\rho_{\text{eq}}}=0,
%\)
%soit
%\begin{equation}
%\left.
%\frac{\delta\mathcal{S}_{YY}}{\delta\rho(\theta)}
%\right|_{\rho_{\text{eq}}}
%=
%\left.
%\frac{\delta\mathcal{W}}{\delta\rho(\theta)}
%\right|_{\rho_{\text{eq}}}.
%\label{eq:TBA:variational_condition}
%\end{equation}

%------------------------------------------------------------------
%\paragraph{Forme explicite : introduction de la pseudo-énergie.}

%Pour le modèle de Lieb–Liniger (et, plus généralement, pour un modèle
%intégrable à noyau $\Delta$), on introduit la \emph{pseudo-énergie}
%\[
%\varepsilon(\theta)
%\;=\;
%w(\theta)
%\;+\;\Bigl[\Delta\star\ln\!\bigl(1+e^{-\varepsilon}\bigr)\Bigr](\theta),
%\]
%obtenue en réécrivant \eqref{eq:TBA:variational_condition}.
%Le \emph{facteur d’occupation}
%\(
%\nu(\theta)=\rho(\theta)/\rho_s(\theta)
%\)
%se donne alors par la statistique de type Fermi-Dirac
%\[
%\nu(\theta)=\frac1{1+e^{\varepsilon(\theta)}}.
%\]

%Les équations intégrales complètes de la \textbf{Thermodynamique de Bethe}
%(TBA) sont donc
%\begin{align}
%2\pi\rho_s(\theta) &= 1 + \bigl[\Delta \star \rho\bigr](\theta),
%\label{eq:TBA:rho_s}\\[4pt]
%\rho(\theta) &= \frac{\rho_s(\theta)}{1+e^{\varepsilon(\theta)}},
%\qquad
%\varepsilon(\theta)=w(\theta)+\bigl[\Delta\star\ln(1+e^{-\varepsilon})\bigr](\theta).
%\label{eq:TBA:epsilon}
%\end{align}
%Elles déterminent sans ambiguïté la distribution d’équilibre
%$\rho_{\text{eq}}(\theta)$ en fonction du poids spectral $w(\theta)$.

%\medskip
%Ainsi, la méthode du point selle relie le \emph{poids spectral}
%(caractéristique du GGE) à la distribution de rapidité la plus probable,
%et permet d’évaluer les observables par la formule
%\label{chap:TBA:eq:ensemble_average:approx}.


%-----------------------------------------------------
%%%%%%%%%%%%%%%%

%\paragraph{Équation intégrale de la TBA.}

%Cette égalité donne naissance à une équation intégrale pour le poids spectral \( w \), défini comme la dérivée fonctionnelle de l'énergie généralisée pris en $\rho_{eq}$ :
%\(
%w ~=~ \left. \frac{\delta \mathcal{W}[\rho]}{\delta \rho} \right|_{\rho =  \rho_{eq} }
%\)
%qui par stationnarité (cf équation \eqref{chap:TBA:eq:stationnarite}) est égale à la dérivée fonctionnelle de l'entropie de Yang-Yang pris en $\rho_{eq}$ :
%\(
%\left. \frac{\delta \mathcal{S}_{YY}[\rho]}{\delta \rho} \right|_{\rho = \rho_{eq} }
%\) 
%qui lui vaux 
%\(
%\ln ( \nu_{eq}^{-1}  - 1 ) - \frac{\Delta}{2\pi} \star \ln ( 1 -  \nu_{eq })
%\)
%avec le facteur d'ocupation à l'équilibre $\nu_{eq} = \rho_{eq}/{\rho_{eq}}_s$. Ainci on peux s'arreter sur l'équation 
%\begin{eqnarray}
%	w & = & \ln ( \nu_{eq}^{-1}  - 1 ) - \frac{\Delta}{2\pi} \star \ln ( 1 -  \nu_{eq }).\label{chap:TBA:eq:w}
%\end{eqnarray}

%\medskip
%Ainsi, la méthode du point selle relie le \emph{poids spectral}
%(caractéristique du GGE) à la distribution de rapidité la plus probable,
%et permet d’évaluer les observables par la formule
%\eqref{chap:TBA:eq:ensemble_average:approx}.\\

%\paragraph{Forme explicite : introduction de la pseudo-énergie.}

%Le \emph{facteur d’occupation}
%\(
%\nu_{eq}
%\)
%se donne alors par la statistique de type Fermi-Dirac
%\begin{eqnarray}
%	\nu_{eq}=\frac1{1+e^{\epsilon}},\label{chap:TBA:eq:nu_eq}
%\end{eqnarray}
%où \emph{pseudo-énergie} 
%\(
%\epsilon
%\)
%se définie en intectant \eqref{chap:TBA:eq:nu_eq} dans \eqref{chap:TBA:eq:w} : 
%\begin{eqnarray}
%	\epsilon & = & w + \frac{\Delta}{2\pi} \star \ln ( 1  + e^{-\epsilon}).\label{chap:TBA:eq:e}	
%\end{eqnarray}


%---------------------------------
%------------------------------------------------------------------
\paragraph{Équation intégrale de la TBA.}

La condition de stationnarité au point selle \(\rho=\rho_{\mathrm{eq}}\) \eqref{chap:TBA:eq:stationnarite} implique :
\begin{eqnarray}
	\left.\frac{\delta\mathcal{S}_{YY}}{\delta\rho(\theta)}\right|_{\rho_{\mathrm{eq}}} = \left.\frac{\delta\mathcal{W}}{\delta\rho(\theta)}\right|_{\rho_{\mathrm{eq}}}\;\doteq\;w(\theta),
\end{eqnarray}
En utilisant l’expression explicite de l’entropie de Yang–Yang \eqref{chap.2.entropi.int}, on obtient l’identité fonctionnelle
\begin{eqnarray}
	w & = & \ln ( \nu_{\!eq}^{-1}  - 1 ) - \frac{\Delta}{2\pi} \star \ln ( 1 -  \nu_{\!eq}).\label{chap:TBA:eq:w}
\end{eqnarray}
où
\(
\nu_{\!eq}=\rho_{\!eq}/\rho_{s,\!eq}
\)
est le \textbf{facteur d’occupation} à l’équilibre.
%------------------------------------------------------------------
\paragraph{Forme pseudo-énergie.}
La \textbf{pseudo-énergie} $\epsilon$ se donne alors par la statistique de type Fermi-Dirac
\begin{eqnarray}
	\epsilon =\ln(\nu^{-1}_{\!eq}-1),\qquad\nu_{\!eq}=\frac{1}{1+e^{\epsilon}}.\label{chap:TBA:eq:nu}%\tag{\text{TBA--$\nu$}} 
\end{eqnarray}
En réinjectant \eqref{chap:TBA:eq:nu} dans \eqref{chap:TBA:eq:w} on obtient
l’équation intégrale canonique de la thermodynamique de Bethe :
\begin{eqnarray}
	\epsilon & = & w - \frac{\Delta}{2\pi} \star \ln ( 1  + e^{-\epsilon}).\label{chap:TBA:eq:e}%\tag{\text{TBA–-$\varepsilon$}}	
\end{eqnarray}
%\[
%\boxed{\;
%\varepsilon(\theta)
%=
%w(\theta)
%+\frac{\Delta}{2\pi}\star\ln\!\bigl[1+e^{-\varepsilon(\theta)}\bigr]
%\;}
%\tag{TBA–$\varepsilon$}\label{eq:TBA:eq:e}
%\]

Les relations \eqref{chap:TBA:eq:nu}–\eqref{chap:TBA:eq:e} déterminent de façon univoque la distribution de rapidité d’équilibre \(\rho_{\!eq}\) à partir du poids spectral \(w\), caractéristique du GGE.

\medskip
Ainsi, la méthode du point selle relie \emph{explicitement} le {\em poids spectral}, $w$  (caractéristique du GGE) au \emph{macro-état le plus probable}, $\rho_{eq}$ , et permet d’évaluer les observables par la formule d’ensemble \eqref{chap:TBA:eq:ensemble_average:approx}.


\paragraph{Résolution numérique de l’équation TBA.}\label{para-algho-TBA}

Prenons un poids spectrale quelconque, par exemple : 
\begin{equation}
  w(\theta)= \theta^2 .\label{eq:TBA:w:quadra}
\end{equation} 
En injectant $w$ dans l’équation intégrale pour lapseudo-énergie \eqref{chap:TBA:eq:e}, on obtient l’équation non linéaire.
Cette équation définit un opérateur contractant sur l’espace des fonctions
\( \epsilon(\theta) \) ; son Jacobien a une norme strictement
inférieure à 1, garantissant la convergence de l’itération de Picard.

\medskip
\subparagraph{Algorithme d’itération.}  
La structure contractante de l’équation garantit l’absence de cycles ou de points fixes multiples, assurant la convergence de l’itération vers l’unique solution admissible.
L’équation \eqref{eq:num:TBA} est non linéaire ; pour la résoudre numériquement, on utilise une méthode itérative de type Picard. On initialise
\(
  \epsilon_0 = w ,
\)
puis on construit une suite de fonctions \(\varepsilon_n\) définie par
\begin{eqnarray*}
	\epsilon_{n+1} & = & \epsilon_0 -   \frac{\Delta}{2\pi} \star \ln \left( 1 + e^{-\epsilon_n} \right) ,\quad n\ge0
\end{eqnarray*}
L’itération est poursuivie jusqu’à convergence, que l’on peut tester via le critère numérique
\(
  \beta \left\| \varepsilon_{n+1} - \varepsilon_n \right\|_\infty < 10^{-12},
\)
où \(\|\cdot\|_\infty\) désigne la norme \(L^\infty\) (ou un maximum discret après discrétisation).


\medskip
\subparagraph{Facteur d’occupation et densités.}  
Une fois la pseudo-énergie \( \epsilon(\theta) \) convergée, le facteur d’occupation  à l'équilibre est obtenu en injectant $\epsilon$ dans l’équation \eqref{chap:TBA:eq:nu}, ce qui donne  $\nu_{\!eq}$.
 
On en déduit ensuite la densité d'état à l'équilibre $\rho_{s,eq}$ via le {\bf dressing}  de la fonction constante $f(\theta) = 1$, selon \eqref{eq:TBA-rhos-2}, rappelée ici pour mémoire : $ 2\pi \rho_{s,eq}  =  1^{\mathrm{dr}}_{[\nu_{\! eq}]}$.\\

L’opérateur de dressing \eqref{eq:dressing} étant linéaire, il se résout numériquement sous la forme :
\begin{eqnarray*}
	\left\{ \mathrm{id} - \frac{\Delta}{2\pi} \star ( \nu \ast \cdot ) \right\} f^{\mathrm{dr}}_{[\nu]} & = & f,\label{eq:TBA:rho_s:num}
\end{eqnarray*}
où $\mathrm{id} \colon f \mapsto f$ est l’identité fonctionnelle, et $\ast$ désigne la multiplication.
Après discrétisation de la variable $\theta$, cette équation devient un système linéaire de type $Ax=b$ , facilement résoluble numériquement.

La distribution de rapidité est alors obtenue par $\rho_{\!\mathrm{eq}} = \nu_{\!\mathrm{eq}} \ast \rho_{\! s,\mathrm{eq}}$.\\

\medskip
Ainsi en fixant le poids spectral $w(\theta)$, l’algorithme fournit la pseudo-énergie \( \epsilon \), le facteur d’occupation \( \nu_{\mathrm{eq}} \) et la distribution de rapidité \( \rho_{\!\mathrm{eq}} \).

\medskip
\subparagraph{À l'équilibre thermique.} 
Si on se place à l’équilibre canonique, caractérisé par la température \( T \) et le potentiel chimique \( \mu \).  Dans ce cadre, le poids spectral vaut
\begin{equation}
  w(\theta)=\beta\bigl[\varepsilon(\theta)-\mu\bigr],\qquad\beta=\tfrac1T\; (k_B = 1 ),\quad\varepsilon(\theta)=\tfrac{\theta^{2}}{2}\;(m=1).\label{eq:TBA:w:canonical}
\end{equation}
%En injectant \eqref{eq:TBA:w:canonical} dans l’équation intégrale pour lapseudo-énergie \eqref{chap:TBA:eq:e}, on obtient l’équation non linéaire :
%\begin{eqnarray*}
%	\epsilon & = & \beta(\varepsilon - \mu)  -  \frac{\Delta}{2\pi} \star \ln \left( 1 + e^{-\epsilon} \right) ,\label{eq:num:TBA}
%\end{eqnarray*}
%Ainsi, pour tout couple \((T,\mu)\), l’algorithme fournit la pseudo-énergie \( \epsilon \), le facteur d’occupation \( \nu_{\mathrm{eq}} \) et la distribution de rapidité \( \rho_{\mathrm{eq}} \) à l’équilibre thermique, prêts à être utilisés pour le calcul des observables.
%
%\medskip
%Pour $w$ quelconque , l'algorythme est identique.




		


\chapter{Dynamique hors-équilibre et hydrodynamique généralisée}
\label{chap:GHD}
\minitoc

%\chapter{Hydrodynamique généralisée (GHD)}

\section*{Introduction}


\paragraph{De l’état stationnaire à la dynamique}  
Après avoir étudié les propriétés stationnaires des gaz de bosons unidimensionnels, nous nous tournons désormais vers leur évolution temporelle. Ce chapitre s’appuie sur une approche hydrodynamique adaptée aux systèmes intégrables : la théorie dite d’Hydrodynamique Généralisée (GHD). Celle-ci est largement documentée dans la littérature (voir par exemple [50, 24, 51, 52]) et nous en présentons ici les concepts essentiels.

\paragraph{Principe général d’une approche hydrodynamique}  
De manière générale, l’hydrodynamique vise à décrire la dynamique à grande échelle (\emph{coarse grained dynamics}) d’un système, également appelée « échelle d’Euler ». L’idée consiste à découper l’espace-temps d’un système de taille $L$ en cellules de dimensions $\ell \times \tau$, comme illustré en Fig.~???.  
La longueur $\ell$ est choisie de sorte que $L \gg \ell \gg \ell_c$, où $\ell_c$ désigne une longueur microscopique caractéristique, par exemple la distance inter-particule. On peut alors considérer que la densité est uniforme à l’intérieur de chaque cellule, ce qui correspond à l’Approximation de Densité Locale.

\paragraph{Choix des échelles spatio-temporelles}  
Le temps $\tau$ est fixé pour être beaucoup plus grand que le temps caractéristique de relaxation. Ainsi, chaque cellule de l’espace-temps est supposée décrire un état localement relaxé. La notion de relaxation occupe donc une place centrale dans la construction des approches hydrodynamiques.

\paragraph{Particularités pour les systèmes quantiques isolés}  
Dans le cadre de systèmes quantiques isolés, la relaxation n’est pas un concept trivial, qu’il s’agisse de systèmes chaotiques ou intégrables. La section suivante s’attache à définir plus précisément cette notion, avant de présenter les approches hydrodynamiques adaptées à chaque cas. Pour les systèmes intégrables, une attention particulière est portée à la formulation et aux implications de l’Hydrodynamique Généralisée.

\paragraph{Équations hydrodynamiques de type Euler}  
Les équations hydrodynamiques de type Euler sont des équations hyperboliques qui décrivent la dynamique émergente des systèmes à plusieurs corps à grandes échelles d’espace et de temps~\cite{ref1}. Elles rendent compte de la propagation de la relaxation locale, c’est-à-dire la séparation entre une dynamique lente, émergente, et la projection rapide des observables locales sur les quantités conservées. En une dimension d’espace, elles prennent la forme locale de conservation
\begin{equation}\label{chap:GHD:eq.conserv.1}
	\partial_t q_i + \partial_x j_i = F_i,	
\end{equation}
où l’indice $i$ énumère les lois de conservation locales admises, et où $F_i$ représente les contributions provenant de champs de force externes, qui rompent en général la conservation stricte.

\paragraph{Relations constitutives et exemples}  
Les flux $j_i$ et les termes de force $F_i$ dépendent uniquement des densités conservées $q_i$ (équations d’état), et sont déterminés à partir de considérations thermodynamiques, telles que la maximisation de l’entropie. Les équations d’Euler pour un fluide galiléen, ou encore l’hydrodynamique relativiste, constituent des exemples classiques de ce type d’équations.

\paragraph{Cas intégrable et hydrodynamique généralisée}  
En dimension un, de nombreux systèmes à plusieurs corps présentent une propriété d’intégrabilité~\cite{ref2,ref3}. Dans ce contexte, il existe une infinité de lois de conservation, et la théorie universelle qui décrit leur hydrodynamique à l’échelle d’Euler est l’Hydrodynamique Généralisée (GHD)~\cite{ref4,ref5}. Cette approche englobe les équations connues pour les bâtons durs~\cite{ref1,ref6} et les gaz de solitons~\cite{ref7,ref8,ref9}, tout en s’appliquant plus largement, aussi bien à des systèmes classiques que quantiques : particules en interaction, chaînes de spins ou théories des champs quantiques (voir~\cite{ref10} pour des revues).

\paragraph{Paramétrisation spectrale et densité conservée}  
La GHD reformule l’infinité de lois de conservation (éventuellement rompues) en une famille indexée par un paramètre spectral continu $\theta$, plutôt que par un indice discret $i$. On note $\rho(x,\theta,t)$ la densité conservée en espace réel, espace spectral et temps. Le paramètre spectral énumère les objets asymptotiques issus de la théorie de diffusion correspondante (particules, solitons, etc.), incluant leur quantité de mouvement et leurs éventuels degrés internes. Dans de nombreux cas simples, $\theta$ appartient à un sous-ensemble de $\mathbb{R}$, représentant les moments asymptotiques, et les coordonnées $(x,\theta)$ forment un « espace des phases spectral » sur lequel $\rho$ joue le rôle de densité.

\paragraph{Prise en compte des champs de force}  
L’inclusion de champs de force externes couplés aux densités conservées a été introduite dans~\cite{ref11}, où il est montré que la GHD s’écrit
\begin{equation}\label{chap:GHD:eq.GHD.1}
	\partial_t \rho + \partial_x(v^{\text{eff}} \rho) + \partial_\theta(a^{\text{eff}} \rho) = 0.
\end{equation}
Ici, $v^{\text{eff}}$ et $a^{\text{eff}}$ sont des fonctionnels appropriés de $\rho(x,\cdot,t)$, et le dernier terme représente la contribution des champs de force. D’autres types de forces ont été étudiés~\cite{Bastianello2019a,Bastianello2019b}, mais ne seront pas considérés ici.

%\section{Formulation hamiltonienne de la GHD}
%
%\subsection{Crochet de Poisson fonctionnel}
%
%\paragraph{Définition générale}
%Bonnemain \emph{et al.}~\cite{bonnemain2024hamiltonian} définissent un crochet de Poisson fonctionnel agissant sur les fonctionnelles $F$ et $G$ de la distribution de rapidité, avec interactions :
%\begin{equation}\label{chap:GHD:eq.chochet.bonnemain.1}
%	\{F,G\}=\iint dx\,d\theta\;\frac{\nu}{2\pi}\,\left[\partial_x \left ( \frac{\delta F}{\delta \rho(x,\theta)} \right )\,\left(\partial_\theta \left ( \frac{\delta G}{\delta \rho(x,\theta)} \right ) \right)^{\mathrm{dr}}_{[\nu]} -\partial_x \left ( \frac{\delta G}{\delta \rho (x,\theta)} \right ) \,\left( \partial_\theta \left ( \frac{\delta F}{\delta \rho (x,\theta)} \right )\right)^{\mathrm{dr}}_{[\nu]} \right],
%\end{equation}
%où $\nu$ est la fonction d’occupation. L'application de l’opérateur de \emph{dressing} dans ce crochet traduit les interactions entre particules.
%
%\paragraph{Cas des charges globales}
%Les charges locales conservées ont été définies en \eqref{chap.2.charge.f.1}.  
%Avec le même formalisme, les charges globales conservées se définissent comme fonctionnelles linéaires d’une fonction réelle et régulière \( f(x, \theta) \) définie sur \( \mathbb{R}^2 \) :
%\begin{equation}\label{chap:GHD:eq.charge.global.1}
%	\mathcal{Q}[f] = \int_{\mathbb{R}^2} dx\, d\theta\, f(x, \theta)\, \rho(x, \theta),
%\end{equation}
%qui représente la charge totale associée à une quantité prenant la valeur \( f(x, \theta) \) pour chaque quasi-particule.
%
%Dans notre étude de la dynamique, nous n’avons pas besoin de l’information sur le poids spectral.  
%On notera donc, dans la limite thermodynamique, les moyennes d’opérateurs simplement en retirant leur chapeau :
%\[
%\underset{\mathrm{therm}}{\lim} \braket{\mathcal{O}}_{\varrho[w]} \equiv \mathcal{O}.
%\]
%Ainsi, dans cette limite, la charge globale \eqref{chap:GHD:eq.charge.global.1} s’écrit directement comme ci-dessus.
%
%Le crochet de Poisson \eqref{chap:GHD:eq.chochet.bonnemain.1} appliqué à deux charges globales \( \mathcal{Q}[f] \) et \( \mathcal{Q}[g]\) s’écrit :
%\begin{equation}\label{chap:GHD:eq.chochet.bonnemain.2}
%	\{\mathcal{Q}[f], \mathcal{Q}[g]\} = \int_{\mathbb{R}^2} dx\, d\theta \frac{\nu}{2\pi}  \left( \partial_x f  (\partial_\theta g )^{\mathrm{dr}}_{[\nu]}  - \partial_x g (\partial_\theta f)^{\mathrm{dr}}_{[\nu]}  \right).
%\end{equation}
%L’application du dressing satisfait la symétrie~\cite{doyon2020lecture} :
%\begin{equation}\label{chap:GHD:eq.sym.dr.1}
%	\int_{\mathbb{R}^2}	 dx\, d\theta \, \nu f g^{\mathrm{dr}}_{[\nu]} = \int_{\mathbb{R}^2}	 dx\, d\theta \, \nu f^{\mathrm{dr}}_{[\nu]} g.
%\end{equation}
%Par intégration par parties, le crochet \eqref{chap:GHD:eq.chochet.bonnemain.2} devient :
%\begin{equation}\label{chap:GHD:eq.chochet.bonnemain.3}
%	\{ \mathcal{Q}[f] , \mathcal{Q}[g]\} = \int_{\mathbb{R}^2} dx\, d\theta \,   f  \left( \partial_\theta \left ( \frac{\nu }{2\pi}  (\partial_x g )^{\mathrm{dr}}_{[\nu]} \right )   - \partial_x  \left ( \frac{\nu}{2\pi}  (\partial_\theta g )^{\mathrm{dr}}_{[\nu]} \right )  \right).
%\end{equation}
%
%\subsection{Crochet avec l’Hamiltonien}
%
%\paragraph{Densité hamiltonienne et grandeurs effectives}
%On note $h(x,\theta)$ la densité associée à la moyenne de l’Hamiltonien :
%\begin{equation}\label{chap:GHD:eq.ham.1}
%	H = \mathcal{Q}[h].
%\end{equation}
%La fonction d’occupation $\nu$, la vitesse effective $v^{\mathrm{eff}}$ et l’accélération effective $a^{\mathrm{eff}}$ sont définies par :
%\begin{equation}\label{chap:GHD:eq.nu.v.a.1}
%	\nu = 2\pi \frac{\rho}{1^{\mathrm{dr}}_{[\nu]}}, \quad  
%	v^{\mathrm{eff}} = \frac{(\partial_\theta h )^{\mathrm{dr}}_{[\nu]}}{1^{\mathrm{dr}}_{[\nu]}}, \quad  
%	a^{\mathrm{eff}} = -\frac{(\partial_x h )^{\mathrm{dr}}_{[\nu]}}{1^{\mathrm{dr}}_{[\nu]}},
%\end{equation}
%fonctions de $\rho(x,\theta,t)$.
%
%Le crochet \eqref{chap:GHD:eq.chochet.bonnemain.3} appliqué à $(f,h)$ devient :
%\begin{equation}\label{chap:GHD:eq.chochet.bonnemain.4}
%	\{\mathcal{Q}[f] , \mathcal{Q}[h]\} = -\int_{\mathbb{R}^2} dx\, d\theta \,   f  \left[ \partial_x \left ( \rho  v^{\mathrm{eff}} \right )   +  \partial_\theta   \left ( \rho  a^{\mathrm{eff}} \right )  \right].
%\end{equation}
%
%\paragraph{Forme locale : densités conservées}
%En choisissant $f(x,\theta) \mapsto \delta(\cdot - x)f(\theta)$ dans \eqref{chap:GHD:eq.charge.global.1}, on obtient la densité conservée :
%\[
%q_{[f]}(x) = \mathcal{Q}[(x,\theta) \mapsto \delta(\cdot - x) f(\theta)].
%\]
%Appliquée à \eqref{chap:GHD:eq.chochet.bonnemain.4}, cette prescription donne :
%\begin{equation}\label{chap:GHD:eq.chochet.bonnemain.5}
%	\{ q_{[f]}(x) , \mathcal{Q}[h]\} = - \partial_x \left ( \int_{\mathbb{R}} d\theta \,   f  \,  \rho  \,  v^{\mathrm{eff}} \right ) + \int_{\mathbb{R}} d\theta \, f' \,    \rho \, a^{\mathrm{eff}}.
%\end{equation}
%En utilisant l’équation de Liouville \eqref{chap:GHD:eq.Liouv.1}, on retrouve la forme de convection :
%\begin{equation}\label{chap:GHD:eq.conserv.2}
%	\partial_t q_{[f]} + \partial_x j_{[f]} = F_{[f]},
%\end{equation}
%avec
%\begin{equation}\label{chap:GHD:eq.conserv.2.1}
%	j_{[f]} = \int_{\mathbb{R}} d\theta \,v^{\mathrm{eff}} \, f \, \rho, 
%	\quad F_{[f]} = \int_{\mathbb{R}} d\theta \,  a^{\mathrm{eff}} \, f' \, \rho.
%\end{equation}
%
%\paragraph{Forme locale : équation sur \texorpdfstring{$\rho$}{rho}}
%En prenant $\rho(x,\theta) = \mathcal{Q}[\delta(\cdot - x)\delta(\cdot - \theta)]$ et en l’appliquant à \eqref{chap:GHD:eq.chochet.bonnemain.4}, on obtient :
%\begin{equation}\label{chap:GHD:eq.chochet.bonnemain.6}
%	\{ \rho ( x , \theta ) , \mathcal{Q}[h]\} = - \partial_x \left (  v^{\mathrm{eff}} \,  \rho   \right ) - \partial_\theta \left (  a^{\mathrm{eff}}  \,  \rho  \right).
%\end{equation}
%En appliquant l’équation de Liouville \eqref{chap:GHD:eq.Liouv.1}, on retrouve l’équation GHD :
%\begin{equation}\label{chap:GHD:eq.conserv.3}
%	\partial_t \rho + \partial_x(v^{\mathrm{eff}} \rho) + \partial_\theta(a^{\mathrm{eff}} \rho) = 0.
%\end{equation}
%
%---------------------
\section{Formulation hamiltonienne de la GHD}

\subsection{Crochet de Poisson fonctionnel}

\paragraph{Interprétation et limite non-interactive}  
À ce niveau de généralité, l'équation de l’Hydrodynamique Généralisée (GHD) \eqref{chap:GHD:eq.GHD.1} peut être interprétée comme la dynamique hydrodynamique d’un fluide bidimensionnel dont la densité est conservée dans l’espace des phases spectral.  
Les effets d’interaction se traduisent par un couplage non local dans la direction des rapidités $\theta$, reflétant les processus de diffusion élastique entre quasi-particules possédant des paramètres spectraux distincts.

\medskip

Dans le cas limite d’un système \emph{sans interactions}, l’espace spectral coïncide avec l’espace des phases classique, et l’équation de GHD se réduit alors à l’équation de Liouville (ou, de façon équivalente, à l’équation de Boltzmann sans terme de collisions) issue de la théorie cinétique élémentaire.

\medskip

En l’absence de phénomènes dissipatifs, la densité de distribution $\rho$ est conservée le long du flot hamiltonien associé à l’énergie $H$, ce qui s’exprime par
\begin{equation}\label{chap:GHD:eq.Liouv.1}
	\frac{d \rho}{dt} 
	= \frac{\partial \rho}{\partial t } + \{ \rho , H \} = 0,
\end{equation}
où $\{\cdot , \cdot\}$ désigne le crochet de Poisson canonique dans l’espace des phases.  
Dans cette perspective, l’Hydrodynamique Généralisée apparaît comme une extension naturelle de l’équation de Liouville aux systèmes intégrables, incorporant les effets collectifs induits par les interactions tout en préservant une description exacte à grande échelle.


\paragraph{Structure hamiltonienne et crochet de Poisson fonctionnel}  
Bonnemain \emph{et al.} \cite{bonnemain2024hamiltonian} introduisent un crochet de Poisson fonctionnel agissant sur des fonctionnelles $F$ et $G$ de la distribution de rapidité $\rho(x,\theta)$ en présence d’interactions. Celui-ci s’écrit
\begin{equation}\label{chap:GHD:eq.chochet.bonnemain.1}
	\{F,G\}
	=\iint dx\,d\theta\;\frac{\nu}{2\pi}\,
	\left[
		\partial_x \left( \frac{\delta F}{\delta \rho(x,\theta)} \right)
		\left( \partial_\theta \left( \frac{\delta G}{\delta \rho(x,\theta)} \right) \right)^{\mathrm{dr}}_{[\nu]}
		-
		\partial_x \left( \frac{\delta G}{\delta \rho(x,\theta)} \right)
		\left( \partial_\theta \left( \frac{\delta F}{\delta \rho(x,\theta)} \right) \right)^{\mathrm{dr}}_{[\nu]}
	\right],
\end{equation}
où $\nu$ désigne la fonction d’occupation. Dans ce crochet l'application de l’opérateur de \emph{dressing} $(\cdot)^{\mathrm{dr}}_{[\nu]}$ (introduit dans \eqref{eq:dessing})  traduit les interactions entre particules.

%L’opérateur de \emph{dressing} $(\cdot)^{\mathrm{dr}}_{[\nu]}$ agit ici sur les dérivées fonctionnelles dans la variable spectrale $\theta$ ; il encode les effets des interactions à longue portée dans l’espace des rapidités. Cette structure hamiltonienne permet de reformuler la GHD comme une équation de type Liouville sur l’espace fonctionnel des distributions $\rho$, mais avec un crochet de Poisson modifié par le \emph{dressing}, traduisant la nature intégrable et non-locale des interactions.

\medskip

\paragraph{Charges globales conservées}  
Les charges locales conservées ont été définies dans les équations~\eqref{chap.2.charge.f.1}.  
Dans le même formalisme, on définit les \emph{charges globales conservées} comme des fonctionnelles linéaires agissant sur une fonction réelle et régulière $f(x,\theta)$ définie sur $\mathbb{R}^2$, selon
\begin{equation}\label{chap:GHD:eq.charge.global.0}
	\operator{\mathcal{Q}}[f] 
	= \int_{\mathbb{R}^2} dx\, d\theta\, f(x, \theta)\, \operator{\rho}(x, \theta),
\end{equation}
où $\operator{\rho}(x,\theta)$ est l'opérateur distribution de rapidité.  
Cette quantité correspond à la charge totale associée à une observable prenant la valeur $f(x,\theta)$ pour chaque quasi-particule.

\medskip

La valeur moyenne $\langle \operator{\mathcal{Q}}[f] \rangle_{\operator{\varrho}[w]}$ a été définie en~\eqref{chap.TBA.moy.dens}.  
La matrice densité locale $\operator{\varrho}^{(\mathcal{S})}[w]$ a été introduite en~\eqref{chap.2.densite.1}.  
De manière analogue, la \emph{matrice densité globale} $\operator{\varrho}[w]$ s’écrit
\begin{equation}\label{chap:GHD:eq.charge.global.2}
	\operator{\varrho}[w] 
	= \frac{1}{Z[w]}\, e^{-\operator{\mathcal{Q}}[w]}, 
	\qquad  
	Z[w] = \mathrm{Tr} \left[ e^{-\operator{\mathcal{Q}}[w]} \right],
\end{equation}
où la charge globale $\operator{\mathcal{Q}}[w]$ est définie par~\eqref{chap:GHD:eq.charge.global.0}, et $w$ désigne le poids spectral.  

%Cette formulation met en évidence le lien entre la description statistique du système et la conservation des charges globales, en généralisant le principe de Gibbs aux systèmes intégrables par l’introduction de l’ensemble d’observables $\operator{\mathcal{Q}}[f]$ sur l’espace spectral.

\medskip

\paragraph{Crochet de Poisson entre charges globales}  
Dans notre étude de la dynamique, nous n’avons pas besoin de l’information détaillée sur le poids spectral $w$.  
Nous noterons donc, dans ce chapitre, et dans la limite thermodynamique, les moyennes des opérateurs en supprimant leur chapeau, \emph{i.e.}
\begin{equation}
\underset{\mathrm{therm}}{\lim} \, \langle \operator{\mathcal{O}} \rangle_{\varrho[w]} \; \equiv \; \mathcal{O},
\end{equation}
de sorte que, dans cette limite, la moyenne de la charge globale s’écrit
\begin{equation}\label{chap:GHD:eq.charge.global.1}
	\mathcal{Q}[f] 
	= \int_{\mathbb{R}^2} dx\, d\theta\, f(x, \theta)\, \rho(x, \theta),
\end{equation}
où $f$ est une fonction régulière sur $\mathbb{R}^2$.

\medskip

Le crochet de Poisson (défini en~\eqref{chap:GHD:eq.chochet.bonnemain.1}) entre deux charges $\mathcal{Q}[f]$ et $\mathcal{Q}[g]$ prend la forme
\begin{equation}\label{chap:GHD:eq.chochet.bonnemain.2}
	\{\mathcal{Q}[f], \mathcal{Q}[g]\}
	= \int_{\mathbb{R}^2} dx\, d\theta\, \frac{\nu}{2\pi} 
	\left[ \partial_x f \, (\partial_\theta g)^{\mathrm{dr}}_{[\nu]} 
	     - \partial_x g \, (\partial_\theta f)^{\mathrm{dr}}_{[\nu]} \right].
\end{equation}
%où $\nu$ est la fonction d’occupation et $(\cdot)^{\mathrm{dr}}_{[\nu]}$ désigne l’application de \emph{dressing} associée à $\nu$.

\medskip

Cette application de \emph{dressing} satisfait la relation de symétrie~\cite{doyon2020lecture} :
\begin{equation}\label{chap:GHD:eq.sym.dr.1}
	\int_{\mathbb{R}^2} dx\, d\theta \; \nu \, f \, g^{\mathrm{dr}}_{[\nu]} 
	= \int_{\mathbb{R}^2} dx\, d\theta \; \nu \, f^{\mathrm{dr}}_{[\nu]} \, g.
\end{equation}

Pour appliquer la relation de symétrie~\eqref{chap:GHD:eq.sym.dr.1} au crochet~\eqref{chap:GHD:eq.chochet.bonnemain.2}, il est nécessaire de vérifier que les fonctions impliquées satisfont les conditions requises sur leurs types tensoriels.
\footnote{
La relation de symétrie~\eqref{chap:GHD:eq.sym.dr.1} est valable lorsque la somme des types tensoriels de $f$ et $g$ est $(1,1)$ dans le sens de~\cite{doyon2020lecture}. Dans ce formalisme, le type $(a,b)$ caractérise la transformation d'un objet vis-à-vis de $x$ (première entrée) et de $\theta$ (seconde entrée). Si $f$ est de type $(p,q)$ et $g$ de type $(r,s)$, alors leur somme est $(p+r,q+s)$. La condition $(1,1)$ garantit que l'intégrande $\nu\, f\, g^{\mathrm{dr}}$ est un scalaire invariant, rendant l'intégrale bien définie. Dans~\eqref{chap:GHD:eq.chochet.bonnemain.2}, $\partial_x f$ est de type $(1,0)$ et $\partial_\theta g$ de type $(0,1)$, ce qui satisfait cette condition et permet l'utilisation de~\eqref{chap:GHD:eq.sym.dr.1}.
}

En utilisant cette symétrie ainsi qu’une intégration par parties, le crochet~\eqref{chap:GHD:eq.chochet.bonnemain.2} se réécrit
\begin{equation}\label{chap:GHD:eq.chochet.bonnemain.3}
	\{\mathcal{Q}[f], \mathcal{Q}[g]\}
	= \int_{\mathbb{R}^2} dx\, d\theta \; f \,
	\left[
		\partial_\theta \left( \frac{\nu}{2\pi} \, (\partial_x g)^{\mathrm{dr}}_{[\nu]} \right)
		- \partial_x \left( \frac{\nu}{2\pi} \, (\partial_\theta g)^{\mathrm{dr}}_{[\nu]} \right)
	\right].
\end{equation}

\medskip

\subsection{Crochet avec l’Hamiltonien}

\paragraph{Densité hamiltonienne et grandeurs effectives} 
On note $h(x,\theta)$ la densité associée à la moyenne de l’Hamiltonien, telle que
\begin{equation}\label{chap:GHD:eq.ham.1}
	H = \mathcal{Q}[h].
\end{equation}

La fonction d’occupation $\nu$, la vitesse effective $v^{\mathrm{eff}}$ et l’accélération effective $a^{\mathrm{eff}}$ sont définies par
%\begin{equation}\label{chap:GHD:eq.nu.v.a.1}
%	\nu = 2\pi \frac{\rho}{1^{\mathrm{dr}}_{[\nu]}}, 
%	\quad v^{\mathrm{eff}} = \frac{(\partial_\theta h )^{\mathrm{dr}}_{[\nu]}}{1^{\mathrm{dr}}_{[\nu]}}, 
%	\quad a^{\mathrm{eff}} = -\frac{(\partial_x h )^{\mathrm{dr}}_{[\nu]}}{1^{\mathrm{dr}}_{[\nu]}},
%\end{equation}
\begin{equation}\label{chap:GHD:eq.nu.v.a.1}
	2 \pi \rho =  1^{\mathrm{dr}}_{[\nu]} \, \nu , 
	\quad 2 \pi \, v^{\mathrm{eff}} \, \rho  =(\partial_\theta h )^{\mathrm{dr}}_{[\nu]} \, \nu , 
	\quad 2 \pi \, a^{\mathrm{eff}} \, \rho  = -(\partial_x h )^{\mathrm{dr}}_{[\nu]}\, \nu ,
\end{equation}
toutes trois étant des fonctions de $\rho(\cdot,\cdot,t)$. Ces quantités interviennent dans les équations de mouvement
\begin{equation}
	\dot{x} = v^{\mathrm{eff}}, \qquad \dot{\theta} = a^{\mathrm{eff}},
\end{equation}
montrant que les dérivées $\partial_x$ et $\partial_\theta$ présentes dans le crochet de Poisson correspondent respectivement à l'action de l'accélération effective sur $\theta$ et de la vitesse effective sur $x$.

\medskip 

%Avec ces définitions, le crochet~\eqref{chap:GHD:eq.chochet.bonnemain.3} s’écrit
Le crochet \eqref{chap:GHD:eq.chochet.bonnemain.3} appliqué à $(f,h)$ devient :
\begin{equation}\label{chap:GHD:eq.chochet.bonnemain.4}
	\{\mathcal{Q}[f], \mathcal{Q}[h]\} 
	= - \int_{\mathbb{R}^2} dx\, d\theta \; f \left[ \partial_x \big( \rho \, v^{\mathrm{eff}} \big) 
	+ \partial_\theta \big( \rho \, a^{\mathrm{eff}} \big) \right].
\end{equation}

\paragraph{Forme locale : densités conservées .} 
En choisissant $(x,\theta) \mapsto \delta(\cdot - x)f(\theta)$ dans \eqref{chap:GHD:eq.charge.global.1}, on obtient la moyenne de la densité conservée \ie
%On remarque que les moyennes des densités conservées $q_{[f]}(x)$ s’obtiennent en appliquant la prescription
%\[
%(x,\theta) \mapsto \delta(\cdot - x) \, f(\theta)
%\]
%dans~\eqref{chap:GHD:eq.charge.global.1}, \emph{i.e.}
\begin{equation}
	q_{[f]}(x) = \mathcal{Q} \big[ (x,\theta) \mapsto \delta(\cdot - x) \, f(\theta) \big].
\end{equation}

Appliqué à~\eqref{chap:GHD:eq.chochet.bonnemain.4}, on obtient
\begin{equation}\label{chap:GHD:eq.chochet.bonnemain.5}
	\{q_{[f]}(x), \mathcal{Q}[h]\} 
	= - \partial_x \left[ \int_{\mathbb{R}} d\theta \; f \, \rho \, v^{\mathrm{eff}} \right]
	+ \int_{\mathbb{R}} d\theta \; f' \, \rho \, a^{\mathrm{eff}}.
\end{equation}

En appliquant l’équation de Liouville~\eqref{chap:GHD:eq.Liouv.1}, on retrouve la forme de convection~\eqref{chap:GHD:eq.conserv.1} :
\begin{equation}\label{chap:GHD:eq.conserv.2}
	\partial_t q_{[f]} + \partial_x j_{[f]} = F_{[f]},
\end{equation}
où le flux $j_{[f]}$ et le terme de force $F_{[f]}$ sont donnés par
\begin{equation}\label{chap:GHD:eq.conserv.2.1}
	j_{[f]} = \int_{\mathbb{R}} d\theta \; v^{\mathrm{eff}} \, f \, \rho,
	\quad F_{[f]} = \int_{\mathbb{R}} d\theta \; a^{\mathrm{eff}} \, f' \, \rho.
\end{equation}

\paragraph{Forme locale : équation sur \texorpdfstring{$\rho$}{rho}} 
De manière analogue, pour la distribution de rapidité à l’équilibre thermodynamique, on note
\begin{equation}
	\rho(x,\theta) = \mathcal{Q}\big[ \delta(\cdot - x) \, \delta(\cdot - \theta) \big].
\end{equation}
Appliqué à~\eqref{chap:GHD:eq.chochet.bonnemain.4}, on obtient
\begin{equation}\label{chap:GHD:eq.chochet.bonnemain.6}
	\{\rho(x,\theta), \mathcal{Q}[h]\} 
	= - \partial_x \big( v^{\mathrm{eff}} \, \rho \big)
	  - \partial_\theta \big( a^{\mathrm{eff}} \, \rho \big).
\end{equation}

En appliquant l’équation de Liouville~\eqref{chap:GHD:eq.Liouv.1}, on retrouve l’équation GHD~\eqref{chap:GHD:eq.GHD.1} :
\begin{equation}\label{chap:GHD:eq.conserv.3}
	\partial_t \rho + \partial_x \big( v^{\mathrm{eff}} \rho \big)
	+ \partial_\theta \big( a^{\mathrm{eff}} \rho \big) = 0.
\end{equation}

Le résultat remarquable \eqref{chap:GHD:eq.conserv.3} a été obtenu pour la première fois par Bertini et al. (2016) et Castro-Alvaredo et al. (2016). Cette observation clé a déclenché l’ensemble des développements ultérieurs de la dynamique hydrodynamique généralisée (GHD) dans les systèmes quantiques intégrables. Les travaux de Bertini et al. (2016) s’appuient en partie sur ceux de Bonnes et al. (2014), où la formule donnant la vitesse effective \eqref{chap:GHD:eq.nu.v.a.1} était apparue pour la première fois dans le contexte d’un système quantique intégrable.\\

Les équation \eqref{chap:GHD:eq.Liouv.1},  \eqref{chap:GHD:eq.conserv.2} et \eqref{chap:GHD:eq.conserv.3} décrivent la dynamique au régime d’Euler. En dehors de cette approximation, il est nécessaire de prendre en compte des contributions supplémentaires liées aux effets diffusifs \cite{DeNardis2018}.




\section{Cas particuliers et interpretations}


\subsection{Cas sans interaction}

En l’absence d’interaction, l’opérateur de \emph{dressing} se réduit à l’identité.  
Dans ce cas, la fonction d’occupation \eqref{chap:GHD:eq.nu.v.a.1} devient :
\begin{equation}
	\nu = 2\pi \rho,
\end{equation}
et le crochet \eqref{chap:GHD:eq.chochet.bonnemain.1} se simplifie en :
\begin{equation}
	\{F,G\} = \iint dx\,d\theta\;\rho \,\left[\partial_x \!\left( \frac{\delta F}{\delta \rho (x , \theta)} \right)\, \partial_\theta \!\left( \frac{\delta G}{\delta \rho (x , \theta)} \right) - \partial_x \!\left( \frac{\delta G}{\delta \rho (x , \theta)} \right) \, \partial_\theta \!\left( \frac{\delta F}{\delta \rho(x , \theta) } \right) \right].
\end{equation}

Les flux et termes de force \eqref{chap:GHD:eq.conserv.2.1} s’expriment alors en remplaçant la vitesse effective $v^{\mathrm{eff}}$ et l’accélération effective $a^{\mathrm{eff}}$ (de \eqref{chap:GHD:eq.nu.v.a.1}) par leurs expressions issues de la dynamique hamiltonienne libre :
\begin{equation}
	v^{\mathrm{eff}} \to \partial_\theta h, 
	\quad a^{\mathrm{eff}} \to  -\partial_x h.
\end{equation}

Dans le cadre de \eqref{chap:GHD:eq.ham.2} \(\partial_\theta h = \theta\) et   \(\partial_x h  = V' \). De plus en ne considérant que les premières charges conservées associées à $f(\theta) = 1$, $\theta$ et $\theta^2/2$ dans \eqref{chap:GHD:eq.conserv.2} et \eqref{chap:GHD:eq.conserv.2.1}, on retrouve les équations d’Euler classiques :
\begin{eqnarray}\label{chap:3:eq:hydro.1}
	\left\{
	\begin{array}{rcl}
	\partial_t n + \partial_x (n u) &=& 0, \\
	\partial_t (n u) + \partial_x (n u^2 + \mathcal{P}) &=& -n \, \partial_x V(x), \\
	\partial_t E + \partial_x (u(E+\mathcal{P})) &=& 0,
	\end{array}
	\right .  
\end{eqnarray}
avec la densité de particule $n(x, t) = q_{[1]}$, la vitesse moyenne du fluide $u = \frac{q_{[\theta]}}{n}$ , la pression cinétique du fluide $\mathcal{P}(x, t) = \left( q_{[\theta^2]} - \frac{q_{[\theta]}^2}{n} \right)$ , l'énergie totale $E = nu^2/2 + nV + ne$ où $e(x,t)$ est l'énergie interne d'une particule.



\paragraph{Remarques sur les charges globales}
En l’absence de potentiel externe ($V = 0$), le système conserve certaines charges globales. Dans un système classique non intégrable, seules ces quelques charges sont conservées. Par exemple dans un système de Gibbs sont conservé , nombre total de particules , quantité de mouvement totale , énergie cinétique totale soit respectivement $Q[1]  =  \int dx \, q[1]$ , $Q[\theta] = \int dx \, q[\theta$ et $Q\left[\frac{\theta^2}{2}\right] = \int dx \, q\left[\frac{\theta^2}{2}\right] $. 
%\begin{equation}
%	Q[1]  &= & \int dx \, q[1] \, \text{(nombre total de particules)},\\ 
%	Q[\theta] = \int dx \, q[\theta] \, \text{(quantité de mouvement totale)},\\ 
%	Q\left[\frac{\theta^2}{2}\right] = \int dx \, q\left[\frac{\theta^2}{2}\right] \text{(énergie cinétique totale)}.
%\end{equation}

\medskip

En revanche, dans un système intégrable, une infinité de charges sont conservées. En particulier, pour tout $\theta \in \mathbb{R}$:
\(
\rho(\theta , t ) = Q[\delta(\cdot - \theta)] = \int dx \, \rho(x, \theta, t),
\)
et les charges associées à une observable quelconque $f(\theta)$ s’écrivent :
\(
Q[f] = \int d\theta \, f(\theta) \, \rho(\theta, t).
\)

\medskip

Cette structure est l’analogue classique de la description en termes de {\bf distribution de rapidité} dans le cadre intégrable. Elle constitue le point de départ naturel pour développer une description hydrodynamique généralisée (GHD) dans le cas intégrable.


\subsection{la vitesse effectif}

En partant de la définition de la vitesse effectif en \eqref{chap:GHD:eq.nu.v.a.1} et en utilisant la définition de l'appliocation \emph{dressing} \eqref{eq:dessing}, il vient que 
\begin{eqnarray}\label{chap:GHD:veff.1}
	2 \pi \, v^{\mathrm{eff}} \,   \rho   = \nu  \,  \partial_\theta h   + \nu  \,  \left ( \Delta \star ( \rho \,  v^{\mathrm{eff}} )  \right ) ,
\end{eqnarray}
où $\Delta$ désigne le décalage en diffusion défini dans le modèle LL  en Eq.\eqref{eq:I-1-16}
et en soustraiant $v^{\mathrm{eff}} ( \theta ) \nu (\theta) \left ( \Delta \star  \rho  \right )( \theta )$ des deux cotés et on obtiens 

\begin{eqnarray}\label{chap:GHD:veff.2}
	v^{\mathrm{eff}} ( \theta )  \left (  2\pi \, \rho ( \theta ) - \nu (\theta) \left ( \Delta \star  \rho  \right )( \theta ) \right )    = \nu( \theta )  \,  \left (  \partial_\theta h ( \theta )  +  \int d \theta' \, \Delta(\theta - \theta') \rho ( \theta') ( v^{\mathrm{eff}} ( \theta' ) - v^{\mathrm{eff}} ( \theta )  )   \right )  ,
\end{eqnarray}

En partant de la l'écriture de la fonction d'ocumation en \eqref{chap:GHD:eq.nu.v.a.1} et en utilisant la définition de l'appliocation \emph{dressing} \eqref{eq:dessing}, il vient que 
\begin{eqnarray}\label{chap:GHD:veff.3}
	2 \pi \rho - \nu \, 	\Delta \star  \rho = \nu, 
\end{eqnarray}
On obtient 
\begin{eqnarray}\label{chap:GHD:veff.4}
	v^{\mathrm{eff}} ( \theta )      =  \partial_\theta h ( \theta )   +  \int d \theta' \, \Delta(\theta - \theta') \rho ( \theta') ( v^{\mathrm{eff}} ( \theta' ) - v^{\mathrm{eff}} ( \theta )  )   ,
\end{eqnarray}
et dans le modèle de Lieb-Liniger $\partial_\theta h ( \theta )   = \theta$.
Sur le plan physique, le premier terme peut être interprété comme un décalage spatial induit par un processus de diffusion à deux corps. Le second terme quantifie le taux de ces processus de diffusion par unité de temps. L’équation ainsi obtenue correspond à l’équation hydrodynamique généralisée (GHD), formulée pour la première fois en 2016\cite{Bertini2016,CastroAlvaredo2016}.

Le résultat remarquable~\eqref{eq:46} a été obtenu pour la première fois par Bertini \emph{et al.}~\cite{Bertini2016} et Castro-Alvaredo \emph{et al.}~\cite{CastroAlvaredo2016}. Cette observation a constitué le point de départ des développements ultérieurs de l’hydrodynamique généralisée (GHD) dans les systèmes quantiques intégrables. Les travaux de Bertini \emph{et al.}~\cite{Bertini2016} s’appuient en partie sur ceux de Bonnes \emph{et al.}~\cite{Bonnes2014}, où la formule de la vitesse effective~\eqref{eq:47} avait été présentée pour la première fois dans le cadre d’un système quantique intégrable.

Dans le cadre général de l’hydrodynamique généralisée (GHD), l’équation~(48) s’interprète comme une extension naturelle du résultat classique obtenu pour le gaz de tiges dures. La distinction essentielle réside dans le fait que le décalage de diffusion \(\Delta(\theta - \theta_0)\) dépend désormais explicitement de la rapidité relative entre les quasi-particules, alors que, dans le cas du gaz de tiges dures, \(\Delta\) est une constante égale à l’opposé du diamètre des particules.

Sur le plan cinématique, on peut décrire la situation de la manière suivante~: un quasi-particule \emph{traceur} de rapidité \(\theta\) --- c’est-à-dire de moment asymptotique \(\theta\) en l’absence d’interactions --- se déplacerait, dans le vide, à vitesse constante \( \theta \). En présence d’une densité finie \(\rho(\theta_0)\) de quasi-particules de rapidité \(\theta_0\), cette vitesse est modifiée par les processus de diffusion à deux corps.

Pendant un intervalle de temps infinitésimal \([t,\, t + \delta t]\), le traceur subit en moyenne
\[
\delta t \times \left| v_{\mathrm{eff}}[\rho](\theta) - v_{\mathrm{eff}}[\rho](\theta_0) \right| \, \rho(\theta_0)
\]
collisions avec des quasi-particules de rapidité \(\theta_0\). Chaque interaction provoque un décalage spatial \(\Delta(\theta - \theta_0)\) vers l’arrière. L’équation~(48) formalise précisément cet effet cumulatif, résultant de l’intégration des contributions de toutes les collisions binaires sur l’espace des rapidités.

Cette analyse microscopique s’étend naturellement aux modèles à \(N\) corps, où les processus de diffusion se combinent et interagissent de manière non triviale, la fonction \(\Delta(\theta - \theta_0)\) encapsulant alors l’intégralité de la structure intégrable du système.

\subsection{Modèle de Lieb–Liniger}

Les informations relatives aux interactions entre particules sont contenues dans la définition du crochet de Poisson \eqref{chap:GHD:eq.chochet.bonnemain.1}, associée à l’opérateur de \emph{dressing} spécifique au modèle de Lieb–Liniger, défini en \eqref{eq:dessing}.  
L’Hamiltonien $H = \mathcal{Q}[h]$ \eqref{chap:GHD:eq.ham.1} s’écrit ici :
\begin{equation}\label{chap:GHD:eq.ham.2}
	h(x , \theta ) = \varepsilon(\theta) + V(x),
\end{equation}
où l’énergie cinétique est $\varepsilon(\theta) = \theta^2 / 2$ et $V(x)$ représente le potentiel extérieur.

\medskip

Dans ce modèle, la vitesse effective et l’accélération effective de \eqref{chap:GHD:eq.nu.v.a.1} se réécrivent :
\begin{equation}
	v^{\mathrm{eff}} = \frac{(\mathrm{id})^{\mathrm{dr}}_{[\nu]}}{1^{\mathrm{dr}}_{[\nu]}}, 
	\quad a^{\mathrm{eff}} = - V'(x).
\end{equation}

Avec l'equation \eqref{chap:GHD:veff.4} la vitesse effectiff dans le modèle de LL s'écrit 
\begin{equation}
	v^{\mathrm{eff}} = \theta +	\int d \theta' \, \Delta(\theta - \theta') \rho ( \theta') ( v^{\mathrm{eff}} ( \theta' ) - v^{\mathrm{eff}} ( \theta )  ) 
\end{equation}


Ainsi, les termes de force dans \eqref{chap:GHD:eq.conserv.2} et \eqref{chap:GHD:eq.conserv.2.1} prennent la forme :
\begin{equation}
	F_{[f]} = -V'(x) \int_{\mathbb{R}} d\theta \, f'(\theta) \, \rho(x, \theta).
\end{equation}

L’équation GHD \eqref{chap:GHD:eq.conserv.3} devient alors :
\begin{equation}\label{chap:GHD:eq.conserv.3.1}
	\partial_t \rho + \partial_x\!\left(v^{\mathrm{eff}} \rho\right) - V'(x) \, \partial_\theta \rho = 0.
\end{equation}


\subsection{Diagonalisation et invariants de Riemann dans la GHD spatiale étendue}

En dérivant la définition de l'application \emph{dressing} \eqref{eq:dessing}, on obtient la relation suivante :
\begin{equation}\label{chap:GHD:d.dressing}
	\partial_X(f^{\mathrm{dr}}) = \left (\partial_X f \right )^{\mathrm{dr}} + \frac{\Delta}{2\pi} \star ( f^{\mathrm{dr}} \partial_X \nu ), 	
\end{equation}
où les variables \(X = t, x, \theta\).

\medskip

En injectant les définitions \eqref{chap:GHD:eq.nu.v.a.1} dans l'équation GHD \eqref{chap:GHD:eq.conserv.3} puis en appliquant les dérivées à l'application \emph{dressing} conformément à \eqref{chap:GHD:d.dressing}, on obtient :  
\begin{eqnarray}
	\begin{array}{c}\left ( \left(\partial_t 1 \right)^{\mathrm{dr}} + \left(\partial_x  \partial_\theta h  \right)^{\mathrm{dr}} - \left(\partial_\theta  \partial_x h  \right)^{\mathrm{dr}} \right ) \nu + \left ( 1 + \nu \,  \frac{\Delta}{2 \pi } \star  \right ) \left ( 1^{\mathrm{dr}} \partial_t v +  \left ( \partial_\theta h \right )^{\mathrm{dr}} \partial_x \nu -  \left ( \partial_x h \right )^{\mathrm{dr}} \partial_\theta \nu\right ) = 0  \end{array}. 
\end{eqnarray}
Or, on a \(\partial_t 1 = 0\) et \(\partial_x \partial_\theta h = \partial_\theta \partial_x h\). Il en résulte donc l'équation locale de conservation :  
\begin{equation}
	\partial_t \nu + v^{\mathrm{eff}}\partial_x \nu
	+ a^{\mathrm{eff}} \partial_\theta \nu = 0.
\end{equation}  

\medskip

Dans les systèmes hyperboliques, la \emph{diagonalisation} d'une équation consiste à trouver une transformation des variables qui permet de décomposer le système couplé en un ensemble de modes indépendants, appelés \emph{invariants de Riemann} ou \emph{modes normaux}. 

\medskip

Dans le cadre de la GHD spatiale étendue, l'équation d'évolution de la densité \(\rho(x,\theta,t)\) est couplée de manière non triviale en \((x,\theta)\) par la vitesse effective \(v^{\mathrm{eff}}\) et l'accélération effective \(a^{\mathrm{eff}}\). La fonction d'occupation \(\nu(x,\theta,t)\) est définie par une transformation non locale dite \emph{dressing} qui incorpore les interactions du système.

\medskip

Grâce à cela, on peut affirmer que la fonction $\nu (x , \theta)$  s’interprète comme un continuum d’{\bf invariants de Riemann}, c’est-à-dire des variables normales qui restent constantes le long des caractéristiques du système.

\medskip

Cette diagonalisation est essentielle pour comprendre la structure hamiltonienne du système et simplifier l'analyse de sa dynamique, notamment dans le cadre spatialement étendu avec un dressing dépendant de la position. 





\section{Interpretation}

 






%\section{Equation Hydrodynamique Généralisé}
%
%\subsection{Description classique sans interaction}
%Considérons une distribution classique de particules dans l’espace des phases, notée $\varphi(x, p, t)$, représentant la densité de particules autour du point $(x, p)$ à l’instant $t$. En l’absence de phénomènes dissipatifs, cette densité est conservée le long du flot hamiltonien, c’est-à-dire \(
%\frac{d\varphi}{dt} = \frac{\partial \varphi}{\partial t} + \{ \varphi , H \} = 0,
%\)
%où $\{ \cdot , \cdot \}$ désigne le crochet de Poisson canonique :
%\begin{equation}
%\{ \varphi , H \} = \frac{\partial \varphi}{\partial x} \frac{\partial H}{\partial p} - \frac{\partial \varphi}{\partial p} \frac{\partial H}{\partial x}.
%\end{equation}
%Pour $d \varphi /dt = 0 $ , 
%\begin{equation}
%	\frac{\partial \varphi}{\partial t} + \{ \varphi , H \} = 0	
%\end{equation}
%
%
%Ce résultat exprime que la distribution $\varphi$ est constante le long des trajectoires dans l’espace des phases générées par le hamiltonien $H$. Sous cette hypothèse, on peut réécrire l’équation de conservation sous forme différentielle :
%
%\begin{equation}
%\partial_t \varphi + \partial_x ( \dot{x} \varphi ) + \partial_p ( \dot{p} \varphi ) = 0,
%\end{equation}
%
%où les équations du mouvement hamiltonien sont :
%\(
%\dot{x} = \frac{\partial H}{\partial p}, \qquad \dot{p} = -\frac{\partial H}{\partial x}.
%\)
%
%Cette équation prend alors la forme d’une équation de continuité dans l’espace des phases :
%
%\begin{equation}
%\partial_t \varphi + \partial_x j_x + \partial_p j_p = 0,
%\end{equation}
%
%où les densités de courant sont données par :
%\(
%j_x = \dot{x} \varphi, \qquad j_p = \dot{p} \varphi.
%\)
%
%\paragraph{Exemple : particules libres dans un potentiel externe}
%Prenons pour Hamiltonien :
%
%\begin{equation}
%H = \varepsilon(p) + V(x), \qquad \text{où } \varepsilon(p) = \frac{p^2}{2m},
%\end{equation}
%
%correspondant à un système de particules classiques de masse $m$ soumises à un potentiel externe $V(x)$, sans interaction entre particules.
%
%L'équation de conservation s’écrit alors :
%
%\begin{equation}
%\partial_t \varphi + v(p) , \partial_x \varphi - \partial_x V(x) , \partial_p \varphi = 0,
%\end{equation}
%
%où $v(p) = \partial_p \varepsilon(p) = p/m$ est la vitesse du flot hamiltonien dans l’espace des phases.
%
%\paragraph{Charges locales conservées et équations hydrodynamiques}
%On définit une observable locale (ou charge locale) $q[f](x, t)$ associée à une fonction test $f(p)$ par :
%
%\begin{equation}
%q[f](x, t) = \frac{1}{m} \int_{\mathbb{R}} dp \, f(p) \, \varphi(x, p, t).
%\end{equation}
%
%Cette quantité représente la moyenne locale de $f(p)$ pondérée par la distribution $\varphi$. En particulier : la densité de particules : $n(x, t) = q[1]$,l’impulsion moyenne locale : $u(x, t) = \frac{q[p]}{n m}$, la pression cinétique : $\mathcal{P}(x, t) = \frac{1}{m} \left( q[p^2] - \frac{q[p]^2}{q[1]} \right)$.
%
%Les courants associés à ces charges s’écrivent :
%\begin{equation}
%j[f](x, t) = \frac{1}{m} \int dp \, f(p) , \partial_p H(x, p) \, \varphi(x, p, t).
%\end{equation}
%
%En prenant la dérivée temporelle de $q[f]$ et en utilisant l’équation de Liouville, on obtient une équation de conservation de la forme :
%
%\begin{equation}
%\partial_t q[f] + \partial_x j[f] = \frac{1}{m} \int dp \, f(p) , \partial_p \left( \partial_x V(x) \, \varphi \right),
%\end{equation}
%
%qui ne s’annule en général que si $V(x)$ est constant. Toutefois, dans le régime dit hydrodynamique, où $\varphi(x,p,t)$ varie lentement en espace, cette équation devient fermée sur les seules densités $q[f]$, en négligeant les dérivées spatiales d'ordre élevé.
%
%Dans ce cadre, et en ne retenant que les premières charges conservées associées à $f(p) = 1$, $p$, $p^2$, on retrouve les équations d’Euler classiques :
%
%\begin{eqnarray*}
%	\partial_t n + \partial_x (n u) &=& 0, \\
%	\partial_t (m n u) + \partial_x (m n u^2 + \mathcal{P}) &=& -n \, \partial_x V(x), \\
%	\partial_t \mathcal{E} + \partial_x j[\varepsilon(p)] &=& -\partial_x V(x) \cdot q[p],
%\end{eqnarray*}
%
%où $\mathcal{E} = q[\varepsilon(p)]$ est la densité d'énergie, et $j[\varepsilon(p)]$ le courant d'énergie.
%
%\paragraph{Remarques sur les charges globales}
%En l’absence de potentiel externe ($V = 0$), le système conserve certaines charges globales. Dans un système classique non intégrable, seules ces quelques charges sont conservées. Par exemple dans un système de Gibbs sont conservé
%\(
%	Q[1] = \int dx \, q[1] \quad \text{(nombre total de particules)}, 
%	Q[p] = \int dx \, q[p] \quad \text{(quantité de mouvement totale)}, 
%	Q\left[\frac{p^2}{2m}\right] = \int dx \, q\left[\frac{p^2}{2m}\right] \text{(énergie cinétique totale)}.
%\)
%En revanche, dans un système intégrable, une infinité de charges sont conservées. En particulier, pour tout $p \in \mathbb{R}$:
%
%\begin{equation}
%Q[\delta(\cdot - p)] = \frac{1}{m} \int dx \, \varphi(x, p, t),
%\end{equation}
%
%%et les charges associées à une observable quelconque $f(p)$ s’écrivent :
%%
%%\begin{equation}
%%Q[f] = \int dp , f(p) , \rho(p, t).
%%\end{equation}
%%
%%Cette structure est l’analogue classique de la description en termes de "rapidity distribution function" dans le cadre quantique. Elle constitue le point de départ naturel pour développer une description hydrodynamique généralisée (GHD) dans le cas intégrable.
%%
%%
%
%\subsection{Description classique avec interactions}
%
%%Dans la formulation hamiltonienne de la GHD, le champ dynamique est la densité fluide à deux variables  . 
%On définit un crochet de Poisson fonctionnel agissant sur les fonctionnelles $F[\rho]$ et $G[\rho]$  de la distribution de rapidité , avec intéraction. Conformément à Bonnemain et al.\cite{bonnemain2024hamiltonian}  :
%\begin{equation}
%	\{F,G\}\;=\;\iint dx\,d\theta\;\frac{\nu(\theta)}{2\pi}\,\Bigl[\partial_x\frac{\delta F}{\delta \rho(x,\theta)}\,\left(\partial_\theta \left ( \frac{\delta G}{\delta \rho(x,\theta)} \right ) \right)^{\mathrm{dr}} -\partial_x\frac{\delta G}{\delta \rho (x,\theta)}\,\left( \partial_\theta \left ( \frac{\delta F}{\delta \rho (x,\theta)} \right )\right)^{\mathrm{dr}} \Bigr]\,,	
%\end{equation}
%où $\nu$ est la fonction d’occupation.
%%\cite{bonnemain2024hamiltonian,doyon2020lecture}
%
%Pour toute fonction réelle et régulière \( f(x, \theta) \) définie sur \( \mathbb{R}^2 \), on associe le fonctionnel linéaire suivant :
%\begin{equation}
%	Q[f] = \int_{\mathbb{R}^2} dx\, d\theta\, f(x, \theta)\, \rho(x, \theta).
%\end{equation}
%Il s'agit de la charge totale associée à une quantité prenant la valeur \( f(x, \theta) \) pour chaque quasi-particule.  Le crochet de Poisson entre deux charges \( Q[f] \) et \( Q[g]\) s’écrit :
%\begin{equation}
%	\{ Q[f] , Q[g] \} = \int_{\mathbb{R}^2} \frac{dx\, d\theta}{2\pi} \nu  \left( \partial_x f  (\partial_\theta g )^{\mathrm{dr}}  - \partial_x g (\partial_\theta f)^{\mathrm{dr}}  \right),
%\end{equation}
%or l'application dressing satisfait la relation de symétrie \cite{doyon2020lecture}:
%\begin{equation}
%	\int_{\mathbb{R}^2}	 dx\, d\theta \, \nu f g^{\mathrm{dr}} = \int_{\mathbb{R}^2}	 dx\, d\theta \, \nu f^{\mathrm{dr}} g,
%\end{equation}
%soit avec une integration part partie, on réécrit le crochet 
%\begin{equation}
%	\{ Q[f] , Q[g]\} = \int_{\mathbb{R}^2} \frac{dx\, d\theta}{2\pi} f  \left( \partial_\theta ( \nu   (\partial_x g )^{\mathrm{dr}} )   - \partial_x ( \nu   (\partial_\theta g )^{\mathrm{dr}} )  \right).
%\end{equation}
%
%La distribution de rapidité $\rho( x , \theta )  = Q[\delta( \cdot - x )\delta( \cdot - \theta  )  ]$ et pour un hamiltinien $H = Q[h]$ avec $h(x , \theta ) = \varepsilon(\theta) + V(x)$ avec $\varepsilon(\theta) = m \theta^2/2$.
%
%\begin{equation}
%	\{ \rho(x, \theta), Q[h] \} + \partial_x (v^{\mathrm{eff}} \rho) + \partial_\theta (a^{\mathrm{eff}} \rho) = 0.
%\end{equation}
%
%Nous avons ici utilisé les identités (2.29), ainsi que la définition de la fonction d’occupation (rappelée pour commodité) :
%
%\begin{equation}
%	v^{\mathrm{eff}} = \frac{\varepsilon'^{\mathrm{dr}}}{1^{\mathrm{dr}}}, 
%	\quad 
%	a^{\mathrm{eff}} = -V, 
%	\quad 
%	\nu = \frac{\rho}{\rho_s}.
%\end{equation}
%
%Ainsi, en posant \( \partial_t \rho(x, \theta) = \{ \rho(x, \theta), Q[h] \} \), on retrouve bien les équations de la GHD sous forme hamiltonienne étendue à l’espace :
%
%\begin{equation}
%	\partial_t \rho(x, \theta) = -\partial_x (v^{\mathrm{eff}} \rho) - \partial_\theta (a^{\mathrm{eff}} \rho).
%\end{equation}
%
%











%\input{chapters/97_GHD}
\input{chapters/04_GGE_Fluctuation}
\chapter{Dispositif expérimental et méthodes d’analyse}
\label{chap:disp.exp}
\minitoc

%\section{Présentation de l’expérience}
%\section*{Introduction}
%
%\section{Refroidissement}
%
%\section{Imagerie}
%\subsection{Prubleme d'iamgerie et idée numerique}
%
%\section{Confinement transverse}
%
%\section{Confinement longitudinale}
%
%\subsection{Evolution logitudinale}
%
%\section{Outil de sélection spatial}
%
%\subsection{Mesure de distribution de rapidités locales $\rho(x , \theta ) $  pour des systèmes en équilibre}
%
%%\subsection{Piégeage transverses et longitudinale}
%%\section{Outil de sélection spatial}
%%%\section{Mesure de $\rho(x , \theta ) $ }
%
%%\section{Mesure de distribution de rapidités locales $\rho(x , \theta ) $  pour des systèmes en équilibre}

\section*{Introduction}

%\begin{itemize}
%	\item Objectif du chapitre : présentation synthétique de l’expérience
%	\item Distinction claire des contributions : mise en place initiale (précédents doctorants), développement (travail de Léa Dubois), contribution personnelle (prise de données, analyses spécifiques, participation à certaines manipulations)
%	\item Rôle de l’expérience dans l’étude de la dynamique des gaz de Bose 1D
%\end{itemize}

Ce chapitre présente l’expérience utilisée pour étudier les gaz unidimensionnels de rubidium ultra-froids. Nous décrivons l’architecture du dispositif, les méthodes d’imagerie et d’analyse, ainsi que les protocoles expérimentaux auxquels j’ai participé. Le développement initial du refroidissement et du piégeage avant la puce a été réalisé par d’anciens doctorants. La mise en place du piégeage sur la puce et du système de sélection spatiale à l’aide d’un DMD a été initiée par Léa Dubois, alors en première année de doctorat à mon arrivée. Mon travail s’est concentré principalement sur la prise de données, l’analyse et la participation à certaines expériences spécifiques telles que l’expansion longitudinale et la mesure locale de la distribution de rapidité.


\paragraph{Objectif du chapitre}  
Ce chapitre a pour objectif de fournir une présentation synthétique et structurée du dispositif expérimental utilisé pour étudier la dynamique de gaz de Bose unidimensionnels ultra-froids. Il constitue un socle indispensable pour comprendre les protocoles expérimentaux développés au cours de ma thèse et les analyses présentées dans les chapitres suivants.

\paragraph{Architecture générale}  
Nous présentons d'abord l’architecture complète de l’expérience, depuis la production des atomes jusqu’à leur imagerie, en passant par les étapes de refroidissement, de piégeage magnétique sur puce, de manipulation optique, et de génération de potentiels. Cette description s’accompagne d’une mise en contexte des contributions historiques au dispositif.

\paragraph{Contributions successives et personnelles}  
Une attention particulière est portée à la répartition chronologique des contributions. Les étapes initiales (source atomique, MOT, piège DC) ont été développées par d’anciens doctorants. La mise en place du piégeage 1D sur puce ainsi que l’utilisation du DMD pour la sélection spatiale ont été réalisées au cours de la thèse de Léa Dubois. Mon travail s’inscrit dans cette continuité et concerne principalement la prise de données, l’analyse de protocoles dynamiques, ainsi que la participation à certaines opérations de maintenance et d’optimisation du système.

\paragraph{Rôle du dispositif dans la thèse}  
Ce dispositif permet d’explorer des phénomènes hors équilibre dans des gaz quantiques 1D. Il constitue une plateforme particulièrement adaptée à l’étude de protocoles d’expansion, de sondes locales, ou de dynamiques guidées par la théorie hydrodynamique généralisée (GHD), qui sont au cœur de cette thèse.




%\section{Présentation générale de l’expérience}
%\subsection{Vue d’ensemble du dispositif}
%\begin{itemize}
%    \item Architecture générale : production, piégeage, manipulation et imagerie.
%    \item Systèmes étudiés : gaz de rubidium 87 dans des pièges 1D.
%    \item Objectifs : exploration de dynamiques hors équilibre.
%\end{itemize}
%
%\subsection{Historique et contributions successives}
%\begin{itemize}
%    \item Étapes de refroidissement et piégeage initial : travaux antérieurs (voir thèses citées).
%    \item Développement du piégeage 1D sur puce et du DMD : thèse de Léa Dubois.
%    \item Contributions personnelles : prise de données, protocoles dynamiques, analyse.
%\end{itemize}

\section{Le dispositif expérimental}
\subsection{Système laser et contrôle de fréquence}
\label{sec:systeme_laser}

%\paragraph{Laser maître 1 : référence de fréquence}
%La référence principale de fréquence pour l'ensemble des faisceaux utilisés dans l'expérience est fournie par un laser à cavité étendue, développé au SYRTE. Ce laser est asservi par spectroscopie d’absorption saturée sur la transition D2 du $^{87}$Rb, au croisement des transitions $|F=2\rangle \rightarrow |F'=2,3\rangle$. Ce signal de référence est utilisé pour verrouiller les autres sources laser par battement optique.

\paragraph{Laser maître 1 : référence de fréquence}
La stabilité en fréquence de l’ensemble des faisceaux employés dans l’expérience est assurée par un laser à cavité étendue conçu au SYRTE. Ce laser est verrouillé par spectroscopie d’absorption saturée sur la raie D2 du $^{87}$Rb, en ciblant le croisement des transitions $|F=2\rangle \rightarrow |F'=2,3\rangle$. Ce verrouillage fournit la référence absolue de fréquence à partir de laquelle les autres sources laser sont synchronisées par battement optique.

%\paragraph{Laser repompeur}
%Un laser DFB (Distributed Feedback Diode) est utilisé pour produire le faisceau repompeur, permettant de transférer les atomes retombés dans l’état $|F=1\rangle$ vers l’état $|F=2\rangle$. Ce laser est asservi à une fréquence distante de 6\,GHz de celle du maître 1, en utilisant un montage de battement optique et mélange avec un oscillateur à 6.6\,GHz. Une diode Fabry-Perot injectée par la DFB permet d’amplifier la puissance au-delà de 100\,mW.
%
%\paragraph{Laser repompeur}
%Le faisceau de repompage, qui permet de transférer les atomes piégés dans l’état $|F=1\rangle$ vers l’état $|F=2\rangle$, est généré par une diode DFB (Distributed Feedback). Sa fréquence est décalée de 6,GHz par rapport au maître 1 grâce à un système de battement optique combiné à un mélange avec un oscillateur micro-onde à 6.6,GHz. Une diode Fabry–Perot, injectée par la DFB, permet d’augmenter la puissance de sortie au-delà de 100,mW.

\paragraph{Laser repompeur}
Le faisceau de repompage, qui transfère les atomes tombé  dans l’état $|F=1\rangle$ vers l’état $|F=2\rangle$, est produit par une diode DFB (Distributed Feedback). Sa fréquence est décalée de 6 GHz par rapport au maître 1 par battement optique et mélange avec un oscillateur à micro-ondes de 6.6 GHz. Une diode Fabry–Perot, injectée par la DFB, élève la puissance de sortie au-delà de 100 mW.

%\paragraph{Laser maître 2 : laser principal de manipulation}
%Un second laser à cavité étendue, identique au maître 1, est asservi par battement optique à la fréquence du maître 1. Il est amplifié par un amplificateur à semi-conducteur évasé (Tapered Amplifier), permettant d’atteindre une puissance de sortie supérieure à 1\,W. Ce faisceau est ensuite divisé en plusieurs branches pour alimenter :
%\begin{itemize}
%    \item le Piège Magnéto-Optique (PMO),
%    \item la mélasse optique,
%    \item le pompage optique,
%    \item l’imagerie par absorption,
%    \item le faisceau de sélection.
%\end{itemize}

\paragraph{Laser maître 2 : source principale de manipulation}
Un second laser à cavité étendue, est verrouillé par battement optique sur la fréquence du maître 1. L’émission est amplifiée au moyen d’un amplificateur à semi-conducteur évasé (Tapered Amplifier), fournissant plus de 1\,W en sortie. Le faisceau ainsi produit est distribué vers différentes parties de l’installation expérimentale : alimentation du piège magnéto-optique (PMO), formation de la mélasse optique, réalisation du pompage optique, imagerie par absorption,génération du faisceau de sélection.


%\paragraph{Contrôle de fréquence et polarisation}
%Les fréquences des différents faisceaux sont ajustées via des Modulateurs Acousto-Optiques (AOM), tandis que leur polarisation et leur intensité sont contrôlées à l’aide de cubes PBS en combinaison avec des lames demi-onde motorisées ou fixes. Cette configuration assure une grande flexibilité dans la mise en œuvre des différentes phases expérimentales.

\paragraph{Gestion des fréquences et polarisations}
%Les ajustements de fréquence des divers faisceaux sont réalisés à l’aide de modulateurs acousto-optiques (AOM).
Les faisceaux peuvent être interrompus soit à l’aide d’obturateurs mécaniques, soit via des modulateurs acousto-optiques (AOM). Ces derniers offrent un temps de commutation beaucoup plus court que les systèmes mécaniques, car ils permettent de sélectionner uniquement un ordre de diffraction non nul et d’éteindre instantanément le faisceau en interrompant l’alimentation radiofréquence. L’intensité et la polarisation sont réglées via des cubes séparateurs PBS associés à des lames demi-onde, fixes ou motorisées. Ce dispositif offre une grande souplesse pour adapter la configuration optique aux différentes étapes de l’expérience.

%\paragraph{Remarque}
%Une description plus détaillée du montage laser et de son verrouillage peut être trouvée dans la thèse de A.~Johnson~\cite{Johnson2016}. L’ensemble a été maintenu et utilisé sans modifications majeures au cours de ma thèse.

\paragraph{Note}
Une présentation plus exhaustive du montage laser et de son système de verrouillage est disponible dans la thèse de A.Johnson\cite{Johnson2016}. Le dispositif a été conservé dans son architecture d’origine tout au long de mes travaux, avec seulement un entretien régulier.


\subsection{Production et refroidissement des atomes (non détaillé ici, renvoi à d'autres travaux)}
{\color{blue}
\begin{itemize}
    \item Source chaude de rubidium, MOT, molasses optique.
    \item Refroidissement à des températures sub-$\mu~K$ Refroidissement sub-Doppler (détails renvoyés aux travaux précédents).
\end{itemize}
}
%Le dispositif expérimental permet de produire des gaz de rubidium ultra-froids, avec pour objectif final l’obtention de gaz unidimensionnels dans le régime quantique dégénéré. La production suit une séquence expérimentale déjà bien établie, initialement développée par d’anciens doctorants (voir par exemple la thèse d’A. Johnson~\cite{Johnson2016}), puis réoptimisée au début de la thèse de Léa-Dubois ~\cite{L.Dubois2024} sous la supervision d’I. Bouchoule.

Le dispositif expérimental permet de produire des gaz ultra-froids de rubidium, en vue d’obtenir des gaz unidimensionnels dans le régime quantique dégénéré. La séquence expérimentale suit un protocole établi, initialement développé par d’anciens doctorants (voir par exemple la thèse d’A. Johnson~\cite{Johnson2016}) et réoptimisé au début de la thèse de Léa. Dubois~\cite{L.Dubois2024} sous la supervision d’I. Bouchoule.

%\paragraph{Libération des atomes de rubidium}
%Les atomes de $^{87}$Rb sont libérés à partir d’un \emph{dispenser}, placé directement dans l’enceinte à vide, sur le côté de la monture de la puce atomique. Ce composant, parcouru par un courant de \( 4.5\,\mathrm{A} \) pendant environ \( 5\,\mathrm{s} \), émet un flux d’atomes thermiques dans la chambre à vide.

\paragraph{Libération des atomes de rubidium}
Les atomes de $^{87}$Rb sont émis à partir d’un \emph{dispenser} placé directement dans l’enceinte à vide, à proximité de la monture de la puce atomique. Un courant de \( 4.5\,\mathrm{A} \)  est appliqué pendant environ \( 5\,\mathrm{s} \), générant un flux d’atomes thermiques dans la chambre à vide.

%
%\paragraph{Capture par piège magnéto-optique (PMO)}
%Les atomes thermiques sont ralentis et piégés à l’aide d’un piège magnéto-optique. Celui-ci utilise quatre faisceaux laser (dont deux sont réfléchis par la puce) et un champ quadrupolaire magnétique généré par des bobines. Le nuage ainsi formé se situe à quelques millimètres de la surface de la puce.

\paragraph{Capture par le piège magnéto-optique (PMO)}
Les atomes thermiques sont ralentis et confinés dans un piège magnéto-optique. Quatre faisceaux laser (dont deux réfléchis par la puce) combinés à un champ quadrupolaire magnétique produit par des bobines permettent de former un nuage atomique situé à quelques millimètres de la surface de la puce.

%\paragraph{Rapprochement vers la puce}
%Pour rapprocher les atomes de la puce, on transfère le champ quadrupolaire depuis les bobines vers un champ généré par le fil en forme de U de la puce (fil bleu dans la Fig.~\ref{fig:puce}). Ce fil est parcouru par un courant variant de \( 3.6\,\mathrm{A} \) à \( 1.5\,\mathrm{A} \), ce qui rapproche le nuage à quelques centaines de micromètres de la surface.

\paragraph{Rapprochement vers la puce}
Le nuage est rapproché de la surface de la puce en transférant le champ quadrupolaire depuis les bobines vers le champ produit par le fil en forme de U de la puce (fil bleu, Fig.~\ref{fig:puce}). Le courant dans ce fil est ajusté lentement de \( 3.6\,\mathrm{A} \) à \( 1.5\,\mathrm{A} \), ce qui positionne le nuage à quelques centaines de micromètres de la surface.

%\paragraph{Mélasse optique}
%Une phase de mélasse optique permet un refroidissement sub-Doppler des atomes capturés. Un système d’imagerie provisoire est utilisé à cette étape pour visualiser le nuage atomique, dont la taille dépasse le champ d’observation du système d’imagerie final.

%\paragraph{Mélasse optique}
%Une étape de mélasse optique est ensuite appliquée pour atteindre un refroidissement sub-Doppler des atomes capturés. %Un système d’imagerie provisoire permet de visualiser le nuage, dont la taille dépasse le champ d’observation du dispositif final.

%\paragraph{Pompage optique}
%Afin de polariser les atomes dans l’état magnétique \( |F=2,\,m_F=2\rangle \), un pompage optique est effectué avec un faisceau circulairement polarisé \( \sigma^+ \), résonant sur la transition \( |F=2\rangle \rightarrow |F'=2\rangle \).

\paragraph{Pompage optique}
Enfin, les atomes sont préparés dans l’état magnétique \( |F=2,\,m_F=2\rangle \) par pompage optique. Un faisceau circulairement polarisé \( \sigma^+ \), résonant sur la transition \( |F=2\rangle \rightarrow |F'=2\rangle \), assure la polarisation du nuage.

\paragraph{Mélasse optique}
Après la capture dans le PMO, une étape de mélasse optique est appliquée pour refroidir davantage le nuage atomique, au-delà de la limite de Doppler. La mélasse optique repose sur l’utilisation de faisceaux laser légèrement désaccordés en fréquence et polarisés de manière appropriée, qui interagissent avec les atomes selon le mécanisme de refroidissement sub-Doppler.

Le principe physique est le suivant : les atomes en mouvement voient les faisceaux laser avec un décalage Doppler, ce qui modifie la probabilité d’absorption selon leur vitesse et leur position. Combiné avec les effets de polarisation (notamment les forces de type Sisyphus dans un champ de polarisation variable), cela crée un potentiel de friction optique qui ralentit les atomes. Contrairement au refroidissement Doppler standard, la mélasse optique permet de réduire l’énergie cinétique des atomes en dessous de la limite Doppler, atteignant des températures beaucoup plus basses.

Ainsi, cette étape permet d’obtenir un nuage plus dense et plus froid, condition essentielle pour les manipulations ultérieures et la formation de gaz unidimensionnels dans le régime quantique dégénéré.






\subsection{Piégeage magnétique sur puce}
%{\color{blue}
%\begin{itemize}
%    \item Présentation de la puce atomique.
%    \item Confinement transverse et longitudinal.
%    \item Régime 1D : conditions d’accès (\(\hbar \omega_\perp \gg k_B T\)).
%    \item Problèmes de rugosité, stabilité magnétique.
%\end{itemize}
%}

\subsubsection{Piégeage magnétique sur puce}
\label{sec:piegeage_puce}

%On peut utiliser des piégeage optique pour produire des stracture atomique longitudinale alongé. Certaines groupe de recherche utilise un redeau optique 2D pour obtenir un réseau 2D de tube longitudinaaux \cite{Kinoshita2004,LaburtheTolra2004,Paredes2004,Moritz2003}. Ce raseaux 2D produit un grand nombre de systéme atomique propise è l'étude de de gase 1D. Avec ce genre de dispositif on peux etudier des gas 1D peut dense car les densité peut etre moyenné sur tous les tudes. Mais avec ce genre de dispositif on ne peut pas étudier experimentalement les fluctudation dans le systéme. Nous pour gièger les atomes on utilise une puce atomique.
%
%\paragraph{Principe général}
%Les atomes de rubidium sont piégés grâce à une puce atomique intégrée dans l’enceinte à vide. Une puce atomique est un circuit microfabriqué contenant des micro-fils dans lesquels circulent des courants permettant de générer des champs magnétiques à géométrie contrôlée. Ce dispositif, développé dans les années 1990 \cite{Denschlag1999,Fortagh1998}, permet une miniaturisation du système de piégeage \cite{Folman2000,Reichel1999}, les premiers condensats sur puce ont été obtenus en 2001 \cite{Haensel2001,Ott2001} et la premièref fois aux laboratoir Charles Fabry (LCF) \cite{Aussibal2003} et un accès à des confinements forts, particulièrement adaptés à l'étude de gaz de Bose unidimensionnels \cite{Schumm2005,Trebbia2006}.

-------

On peut créer des structures atomiques allongées en utilisant des techniques de piégeage optique. Par exemple, plusieurs groupes de recherche ont recours à des réseaux optiques bidimensionnels (2D) pour former un ensemble de tubes atomiques longitudinaux \cite{Kinoshita2004,LaburtheTolra2004,Paredes2004,Moritz2003}. Ces réseaux 2D permettent de produire un grand nombre de systèmes atomiques quasi-unidimensionnels, offrant ainsi une plateforme idéale pour l’étude des gaz 1D. Ce type de dispositif est particulièrement adapté à l’étude de gaz faiblement denses, car les densités peuvent être moyennées sur l’ensemble des tubes. Cependant, l’étude expérimentale des fluctuations locales dans chaque tube reste difficile avec ce genre de configuration. Pour surmonter cette limitation, on utilise le piégeage à l’aide de puces atomiques.

\paragraph{Principe général}
Les atomes de rubidium sont confinés par une puce atomique intégrée dans l’enceinte à vide. Une puce atomique est un circuit microfabriqué comportant de fins micro-fils parcourus par des courants électriques, ce qui permet de générer des champs magnétiques à géométrie contrôlée. Cette technologie, développée dans les années 1990 \cite{Denschlag1999,Fortagh1998}, offre une miniaturisation significative des dispositifs de piégeage \cite{Folman2000,Reichel1999}. Les premiers condensats de Bose–Einstein sur puce ont été réalisés en 2001 \cite{Haensel2001,Ott2001}, puis ultérieurement au Laboratoire Charles Fabry \cite{Aussibal2003}. Les puces atomiques permettent d’accéder à des confinements très forts, particulièrement adaptés à l’étude des gaz de Bose unidimensionnels et à l’exploration de leurs propriétés quantiques locales \cite{Schumm2005,Trebbia2006}.


-----
Des structures atomiques allongées peuvent être réalisées par piégeage optique. Dans ce cadre, des réseaux optiques bidimensionnels (2D) permettent de créer un ensemble de tubes atomiques quasi-unidimensionnels \cite{Kinoshita2004,LaburtheTolra2004,Paredes2004,Moritz2003}. Ces réseaux offrent un grand nombre de systèmes atomiques identiques, facilitant l’étude statistique de gaz 1D faiblement dense. Toutefois, l’accès expérimental aux fluctuations locales dans chaque tube reste limité.

Pour contourner cette contrainte, les puces atomiques offrent une solution efficace. Ces dispositifs microfabriqués intègrent de fins micro-fils parcourus par des courants, générant des champs magnétiques de géométrie contrôlée et permettant des confinements très forts \cite{Denschlag1999,Fortagh1998,Folman2000,Reichel1999}. La miniaturisation ainsi obtenue a permis l’obtention des premiers condensats de Bose–Einstein sur puce dès 2001 \cite{Haensel2001,Ott2001}, et dés 2003 au Laboratoire Charles Fabry \cite{Aussibal2003}. Grâce à ces confinements, il devient possible d’étudier expérimentalement les propriétés de gaz de Bose unidimensionnels et leurs fluctuations locales \cite{Schumm2005,Trebbia2006}.

-------

\paragraph{Structure de la puce utilisée}
La puce utilisée au cours de cette expérience a été conçue en collaboration avec S.~Bouchoule, A.~Durnez et A.~Harouri (C2N). Elle repose sur un substrat de carbure de silicium sur lequel est déposé le circuit électrique. Ce dernier est recouvert d’une couche de résine BCB, aplanie par des cycles d’enduction et d’attaque plasma. Une fine couche d’or (\(\sim200\,\mathrm{nm}\)) est finalement évaporée afin de permettre l’utilisation de la puce comme miroir pour l’imagerie à \(780\,\mathrm{nm}\). La puce est soudée à l’indium sur une monture en cuivre inclinée à \(45^\circ\) par rapport à l’axe optique.

%\paragraph{Fils de piégeage et géométrie des champs}
%Plusieurs fils sont intégrés à la puce pour assurer les différentes étapes du piégeage et du transport des atomes : un fil en forme de Z est utilisé pour le piégeage initial (DC), tandis que trois micro-fils (symétriques et parallèles) sont utilisés pour former un guide unidimensionnel par courants alternatifs (AC). La géométrie des fils a été optimisée pour minimiser la dissipation de chaleur, limiter les couplages parasites et améliorer la symétrie du piège. Dans la zone d’intérêt, les atomes sont piégés à environ \(15\,\mu\mathrm{m}\) au-dessus des fils, soit à \(8\,\mu\mathrm{m}\) au-dessus de la surface de la puce.

%\paragraph{Fils de piégeage et géométrie des champs}
%La puce atomique comporte plusieurs ensembles de fils, chacun jouant un rôle précis dans les différentes étapes de la capture, du transport et du confinement des atomes.
\paragraph{Fils de piégeage et géométrie des champs}
La puce atomique intègre plusieurs ensembles de conducteurs, chacun conçu pour une étape spécifique de la capture, du transport et du confinement des atomes. L’ensemble de la séquence de transfert, depuis le piège magnéto-optique (PMO) jusqu’au guide unidimensionnel, repose sur une succession de configurations magnétiques générées par ces différents fils.

%\medskip
%\subparagraph{Fil en forme de U .}
%Après la phase de pré-refroidissement, le nuage est initialement capturé dans un piège magnéto-optique (PMO) situé au-dessus de la puce. Il est ensuite approché de la surface en transférant progressivement le champ quadrupolaire des bobines externes vers celui produit par un fil en forme de U intégré à la puce (phase \textit{U} : transfert du PMO vers la puce + mélace optique + ponpage optique). 

\subparagraph{Phase U : approche de la surface}
Après la phase de pré-refroidissement, le nuage est initialement capturé dans un PMO situé au-dessus de la puce. Il est ensuite rapproché de la surface en transférant progressivement le champ quadrupolaire des bobines externes vers celui produit par un fil en forme de U intégré à la puce (fils bleus dans la Fig.~\ref{fig:puce}). Cette étape (\textit{phase U}) est accompagnée d’un mélange optique et d’un pompage optique afin de préparer les atomes pour le piégeage magnétique.


%\medskip
%\subparagraph{Fil en forme de Z : Chargement dans le piège DC .}
%Après le pompage optique, les atomes sont transférés dans un piège magnétique combinant un courant continu circulant dans le fil en forme de Z de la puce (fil orange dans la Fig.~\ref{fig:puce}) et un champ magnétique externe. Ce piège, noté \emph{piège DC}, permet un confinement transverse important. Un refroidissement par évaporation radiofréquence est alors réalisé pendant environ \( 2.3\,\mathrm{s} \), ce qui abaisse la température du nuage à environ \( 1\,\mu\mathrm{K} \), pour un nombre d’atomes typiquement autour de \( 2.5 \times 10^5 \).

\subparagraph{Phase Z : piège DC et refroidissement}
À l’issue du pompage optique, les atomes sont transférés dans un piège magnétique combinant un courant continu circulant dans un fil en forme de Z (fil orange) et un champ magnétique externe. Ce \emph{piège DC} assure un confinement transverse fort. Un refroidissement par évaporation radiofréquence, d’une durée d’environ \(2.3\,\mathrm{s}\), abaisse la température du nuage à environ \(1\,\mu\mathrm{K}\), pour un nombre typique d’atomes de l’ordre de \(2.5\times 10^5\).
%\medskip
%Une fois chargé dans ce piège intermédiaire, le nuage est transporté vers la zone expérimentale. Dans cette région, trois micro-fils parallèles et symétriques (jaune), parcourus par des courants alternatifs (AC), créent un guide magnétique unidimensionnel assurant le confinement transversal des atomes. Le confinement longitudinal est obtenu grâce à deux paires de fils : d/d′ (rose) et D/D′ (vert).

\subparagraph{Transfert vers le guide unidimensionnel}
Une fois refroidi, le nuage est acheminé vers la zone expérimentale où trois micro-fils parallèles et symétriques (fils jaunes) parcourus par des courants alternatifs (AC) génèrent un guide magnétique unidimensionnel assurant le confinement transverse. Le confinement longitudinal est fourni par deux paires de fils : $d/d'$ (rose) et $D/D'$ (vert).

Le passage du piège DC au guide 1D est réalisé de manière adiabatique grâce à cinq rampes linéaires de courant d’une durée comprise entre \(50\) et \(60\,\mathrm{ms}\) chacune. Durant cette opération :  
(i) le courant dans le fil Z est progressivement réduit,  
(ii) le courant dans les micro-fils du guide est augmenté jusqu’à environ \(50\,\mathrm{mA}\),  
(iii) un courant initial de \(0.5\,\mathrm{A}\) est appliqué dans les fils $D$ et $D'$, puis ajusté pour maintenir fixe la position du centre de masse du nuage.  

Ce protocole minimise les oscillations résiduelles dans le guide et assure un découplage efficace entre la dynamique longitudinale et le confinement transverse. Ce dispositif a été développé au cours de la thèse de Léa Dubois~\cite{TheseLea} et a été utilisé dans le cadre de mes protocoles expérimentaux sur l’expansion longitudinale et les sondes locales de distribution de rapidité.

\subparagraph{Optimisation géométrique}
La géométrie des conducteurs de la puce a été conçue pour réduire la dissipation thermique, limiter les couplages parasites et garantir une bonne symétrie des champs magnétiques. Dans la zone expérimentale, les atomes sont piégés à environ \(15\,\mu\mathrm{m}\) au-dessus des fils, soit \(8\,\mu\mathrm{m}\) au-dessus de la surface de la puce.


\paragraph{Refroidissement final et accès au régime unidimensionnel}
Une dernière phase de refroidissement par évaporation radiofréquence est effectuée directement dans le guide AC. Grâce à l’anisotropie marquée du piège, le confinement transverse atteint une fréquence \(\omega_\perp\) telle que l’énergie quantique \(\hbar \omega_\perp\) dépasse largement les énergies thermique et chimique du système. On atteint ainsi le régime unidimensionnel, caractérisé par la hiérarchie d’énergies :
\[
k_B T, \mu \ll \hbar \omega_\perp,
\]
où \(\mu\) désigne le potentiel chimique et \(T\) la température du gaz.

Dans ce régime, le confinement transverse est assuré principalement par la géométrie des micro-fils et la présence de champs magnétiques externes, tandis que le confinement longitudinal, plus faible, est ajustable via une combinaison de champs magnétiques externes et de courants circulant dans des fils additionnels ($d/d'$ et $D/D'$). 

Les gaz obtenus contiennent typiquement entre \(3\times 10^3\) et \(1.5\times 10^4\) atomes, pour des températures de l’ordre de \(50\) à \(200\,\mathrm{nK}\). La Fig.~\ref{fig:gaz1D} illustre un exemple de nuage dans ce régime, observé avec le système d’imagerie final.



%\paragraph{Confinement transverse et longitudinal}
%Le confinement transverse est assuré principalement par la géométrie des fils et la présence de champs magnétiques externes. Sa fréquence élevée permet d’atteindre des énergies de confinement \(\hbar \omega_\perp\) bien supérieures aux énergies thermiques et chimiques du système, condition nécessaire à l’accès au régime 1D :
%\[
%k_B T, \mu \ll \hbar \omega_\perp.
%\]
%Le confinement longitudinal, plus faible, est modulable par combinaison de champs magnétiques externes et courants dans les fils additionnels.

\paragraph{Avantages du piégeage sur puce}
Comparé aux systèmes utilisant des réseaux optiques 2D, le piégeage sur puce ne fournit qu’un seul tube, ce qui permet un meilleur accès aux fluctuations locales de densité et aux observables résolues spatialement. Ce type de dispositif est ainsi particulièrement adapté à l'étude de la thermodynamique et de la dynamique de gaz 1D isolés.

\paragraph{Limitations et effets parasites}
Parmi les limitations spécifiques au piégeage sur puce figurent la rugosité des potentiels magnétiques due aux imperfections des fils, qui peut induire des modulations parasites du confinement longitudinal. De plus, la stabilité du dispositif est sensible aux champs parasites magnétiques externes ainsi qu’aux échauffements dus aux courants continus.





\paragraph{Imagerie finale}
À l’issue de ce refroidissement, les atomes sont observés avec le système d’imagerie final (voir Fig.~\ref{fig:imagerieFinale}), adapté aux tailles caractéristiques du gaz dans le piège. Une image typique de ce nuage est présentée en Fig.~\ref{fig:nuageDC}.



%\paragraph{Refroidissement final et accès au régime unidimensionnel}
%Une dernière phase de refroidissement par évaporation radiofréquence est ensuite réalisée dans le guide AC. Ce refroidissement, mené dans le piège à forte anisotropie, permet d’atteindre le régime unidimensionnel, caractérisé par la hiérarchie d’énergies :
%\[
%k_B T, \mu \ll \hbar \omega_\perp
%\]
%où \( \omega_\perp \) est la fréquence de confinement transverse, \( \mu \) le potentiel chimique et \( T \) la température du gaz.
%
%Les gaz obtenus contiennent typiquement entre \( 3 \times 10^3 \) et \( 1.5 \times 10^4 \) atomes, pour des températures de l’ordre de \( 50 \text{ à } 200\,\mathrm{nK} \). La Fig.~\ref{fig:gaz1D} montre un exemple de tel gaz observé avec le système d’imagerie final.


\paragraph{Remarques expérimentales}
Lorsque j’ai rejoint l’équipe, la première année thèse de Léa Dubois touchait à sa fin et le dispositif expérimental était en fonctionnement stable. Les différentes étapes du cycle (dispenser, PMO, mélasse, pompage optique, piège DC, transfert vers le guide, évaporation finale) avaient été mises en place et optimisées pendant les premières années de sa thèse, sous la supervision d’I. Bouchoule.Le cycle expérimental complet dure environ 15 secondes. Une description plus détaillée peut être trouvée dans la thèse d’A. Johnson~\cite{Johnson2016}.


Pendant ma première année, j’ai principalement participé à la prise de données en collaboration avec Léa. Grâce à la qualité de son travail, le dispositif était globalement très fiable, ce qui a permis de mener des campagnes expérimentales riches sans intervention lourde. Néanmoins, cette stabilité avait pour contrepartie que je n’ai pas été directement impliqué dans la résolution des pannes complexes ou dans le reconditionnement complet de la manipulation, ce qui a limité ma formation sur les aspects de maintenance approfondie du dispositif.

En revanche, peu avant la fin de la thèse de Léa et au début de ma troisième année, nous avons observé une chute significative du nombre d’atomes capturés. Sous la supervision d’I. Bouchoule, une intervention lourde a alors été décidée : nous avons cassé le vide pour diagnostiquer le problème. Il s’est avéré que les connecteurs du dispenser étaient endommagés. L’opération a été mise à profit pour installer un nouveau dispenser et remplacer la puce atomique.

Cette opération a mobilisé plusieurs personnes du laboratoire et de ses partenaires : S. Bouchoule (C2N) et Anne [Nom complet à préciser] ont participé à la manipulation et à la pose de la puce, tandis que j’ai pu assister à l’étuvage de l’enceinte à vide avec F. Nogrette. Après cette intervention, j’ai suivi avec I. Bouchoule le réajustement progressif de la séquence de refroidissement : alignement des faisceaux, réglages de la mélasse, optimisation du chargement dans le piège DC, puis dans le guide.

Cet épisode m’a permis de me confronter plus directement aux paramètres critiques du cycle d’évaporation et à la reprise d’une séquence complète. Toutefois, le départ de Léa, qui maîtrisait tous les aspects de la manipulation, a marqué une rupture importante dans la continuité des savoir-faire pratiques liés à cette expérience.


\begin{center}
	({fig:puce} — Schéma de la puce atomique avec fils U, Z, AC, D et D'.)
\end{center}
\begin{center}
	({fig:imagerieFinale} — Schéma optique du système d’imagerie final)
\end{center}
\begin{center}
	[{fig:nuageDC} — Image du gaz dans le piège DC après évaporation]
\end{center}
\begin{center}
	[{fig:gaz1D} — Image typique d’un gaz dans le régime 1D]
\end{center}



\subsection{Génération de potentiels modulés}
%\begin{itemize}
%    \item Courants modulés pour créer des pièges harmoniques ou quartiques.
%    \item Découplage transverse/longitudinal.
%\end{itemize}

\paragraph{Champ des micro-fils.}
Puisque que $m_F = 2 $, (état assuré par pompage optique), le potentiel magnétique $-\vec{\mu} \vec{B}(\vec{r}) $ (avec moment dipolaire magnétique alors $\vec{\mu}$ et le champs magnetque totale que resente les atomes$\vec{B}(\vec{r})$) est proportionnel à $\vert \vec{B}(\vec{r}) \vert$  de sorte que les atomes, en état low-field seeking, sont attirés vers les régions de champ magnétique minimal. Les micro-fils, alignés selon l’axe horizontal $\vec{e}_x$, sont parcourus par des courants alternatifs $\pm I$ (déphasés) produisant le champ magnétique de confinement : un fil central parcouru par un courant \( I \), et deux fils latéraux par des courants opposés \(-I\). 

\paragraph{Champ de biais.}
Un champ de biais transverse $\vec{B}_{\mathrm{biais}} = {B}_{\mathrm{biais}} \, \vec{e}_y$ , avec l'axe verticale par $\vec{e}_y$ , est appliqué afin de régler la distance des atomes par rapport aux micro-fils. En notant $\vec{e}_z$ l’axe horizontal perpendiculaire à $\vec{e}_x$ et $\vec{e}_y$ l’annulation du champ total a lieu en
%Dans cette configuration, un champ de biais transverse est appliqué pour ajuster la distance des atomes au-dessus des micro-fils. Pour y avoir une idée notons l'axe verticale par $\vec{e}_y$, et $\vec{B}_{biais} = {B}_{biais} \, \vec{e}_y$. Alors en notan $\vec{e}_z$ l'axe hortisontale perpetdiculaire à $\vec{e}_x$ et $\vec{e}_y$, le champs totale s'anume en ​
  %permet ainsi de positionner précisément le minimum du potentiel à une hauteur
$z_0 = \mu_0 I / (2 \pi {B}_{\mathrm{biais}} ) $ avec $\mu_0$ la perméabilité du vide . La modulation de ${B}_{\mathrm{biais}}$ permet de déplacer le point où le champ total s’annule, ce qui permet de positionner précisément le minimum du potentiel à une distance $d$ du plan des fils. 

\paragraph{Champ d’Ioffe.}
Afin d’éviter les pertes de Majorana liées à la présence d’un champ nul, un champ longitudinal $B_0 \, \vec{e}_x$ est ajouté, garantissant que le minimum de champ reste non nul.%.Un champ longitudinal (selon $\vec{e}_x$) $B_0$ est ajouté afin que ce minimum ne corresponde pas à un champ nul, ce qui supprime les pertes de Majorana dues aux inversions de spin au voisinage d’un zéro de champ. 

%L’intérêt de ces pièges est que les atomes peuvent être confinés très près des micro-fils — ici à $ d = 15\, \mu m$ , soit l’espacement entre deux fils — ce qui maximise le gradient de champ et donc la fréquence de piégeage transverse

\paragraph{Fréquence de piégeage transverse.}
Dans la configuration étudiée, les atomes sont confinés à $ d = 15\, \mu m$ au-dessus de la puce, soit l’espacement entre deux micro-fils. Cette faible distance maximise le gradient de champ et donc la fréquence de piégeage transverse, qui s’écrit
\begin{eqnarray*}
	\omega_\perp^{(0)} =  \sqrt{\frac{\mu_B}{mB_0}} \frac{\mu_0 I }{2\pi d^2} 
\end{eqnarray*}
avec $\mu_B$ le magnéton de Bohr, $m$ la masse atomique et $\mu_0$ la perméabilité du vide.
%Pour éviter que les atomes ne perçoivent les rugosités magnétiques dues aux défauts des conducteurs, on fait circuler dans les fils un courant alternatif à haute fréquence ($\sim 400\,KHz$) : le potentiel est alors moyenné temporellement, produisant un confinement plus lisse. À $15\, \mu m$ au-dessus de la puce, le profil de champ est localement harmonique, et la fréquence de piégeage transverse devient

\paragraph{Rugosité et suppression par modulation}
Les imperfections géométriques des micro-fils engendrent des fluctuations parasites du champ magnétique le long du guide, créant une rugosité du potentiel. Pour la supprimer, les courants sont modulés à haute fréquence ($\sim 400\,KHz$), bien au-delà des fréquences de piégeage. Dans ce régime, les atomes ne perçoivent que le potentiel moyenné temporellement, où la composante parasite longitudinale est fortement réduite. Le confinement effectif reste harmonique, avec une fréquence transverse donnée par
\begin{eqnarray*}
	\omega_\perp = \frac{\omega_\perp^{(0)}}{\sqrt{2}}.		
\end{eqnarray*}




%\paragraph{Découplage des confinements transverses et longitudinaux.}
%Les courants qui parcourent les fils D, D', d, d' sons selon $\vec{e}_u$ donc les chanps induit sont selon $\vec{e}_x$ noté $B_\parallel^x$ et $\vec{e}_v$ (axex normale à la puce) , noté $B_\parallel^v$. Si les champs selon $\vec{e}_x$ est négligeable devant $B_0$ alors la moyenne de pottenstelle presente une partie transverce et longitudinale decouplés : $\braket{V} = V_\perp ( y , z ) + V_\parallel(x) $.
\paragraph{Découplage des confinements transverses et longitudinaux.}
Les courants qui parcourent les fils $D$, $D'$, $d$ et $d'$ sont orientés selon $\vec{e}_u$. 
Les champs magnétiques induits possèdent alors une composante selon $\vec{e}_x$, notée $B_\parallel^x$, et une composante selon $\vec{e}_v$ (axe normal à la puce), notée $B_\parallel^v$. 
Si le champ selon $\vec{e}_x$ est négligeable devant $B_0$, alors le potentiel moyen se sépare en une partie transverse et une partie longitudinale découplées : 
\(
\braket{V} = V_\perp(y,z) + V_\parallel(x) .
\)


\paragraph{Potentiel longitudinal harmonique.}
Dans la configuration où seuls les fils $D$ et $D'$ sont utilisés, le potentiel longitudinal peut, à l’ordre 2 en $x$, être considéré comme harmonique :
\begin{eqnarray*}
	V_\parallel (x) = V_0 + \frac{1}{2} m \omega_\parallel^2 x^2 ,
\end{eqnarray*}
On note  $2L=1.89 \,mm$ est la distance séparant les fils $D$ et $D'$. Les courants circulant dans ces deux fils sont identiques et notés $I_D = I_{D'}$. Si la condition $B_0 \gg \mu_0 I_D d /(\pi L)^2 $ est vérifiée, alors le terme constant du potentiel vaut approximativement $V_0 \simeq \mu_B B_0$.

\medskip

La pulsation longitudinale totale $\omega_\parallel$ se décompose en deux contributions : (i) une pulsation $\omega_\parallel^x = \sqrt{\frac{6\, d \, \mu_B \, \mu_0 \,I_D }{\pi \, L^4 \, m}}$ induite par le champ longitudinal $B_\parallel^x$ et (ii) une pulsation $\omega_\parallel^v = \sqrt{\frac{\mu_B }{m \, B_0}}\frac{\mu_0 \, I_D }{\pi \, L^2}$ liée au champ  $B_\parallel^v$. Pour des courants $I>1A$ , on a $\omega_\parallel^v \gg \omega_\parallel^x$, et ainsi :  
\begin{eqnarray*}
	\omega_\parallel \propto \frac{I_D}{\sqrt{B_0} L^2}.
\end{eqnarray*} 
La fréquence longitudinale est donc réglée expérimentalement en ajustant $I_D$.

\medskip

Avec les dimensions caractéristiques de la puce et des fils, il est possible d’atteindre des confinements longitudinaux de fréquence $f_\parallel = \omega_\parallel/ 2 \pi$ allant jusqu’à $\sim 150 \, H_z$, la limite étant imposée par le chauffage des fils pour $I_D \leq =4 \, A$.
 
 \medskip
 
 \subparagraph{Mesure de la fréquence transverse et longitudinale}
Pour la caractérisation, la pulsation transverse $\omega_\perp$ a été mesurée par la méthode du mode de respiration transverse \cite{Kagan1996}, tandis que $\omega_\parallel$ a été obtenue à partir des oscillations dipolaires longitudinales. Les détails expérimentaux de ces méthodes figurent dans le manuscrit de thèse de Léa Dubois \cite{L.Dubois2024}, p. 73 et p. 78.

\medskip

 \paragraph{Potentiel longitudinal quartic.}
 Si on ajoute du courand  dans les fils $d$ et $d'$.  Alors on peux avoir un potentiel non gégligeable à l'ordre 4 . Pour simmplifier, les courants dans ces fils $I_d$ et $I_{d'}$ sont identique. et le potentiel s'écrit : 
 \paragraph{Potentiel longitudinal quartique.}
Si l’on ajoute un courant dans les fils $d$ et $d'$, on peut générer un potentiel longitudinal comportant un terme significatif à l’ordre 4 en $x$ . Pour simplifier, on suppose $I_d=I_{d'}$. On obtient alors : 
 \begin{eqnarray*}
 	V_\parallel(x) \, = \, \mu_B B_0  & + & 	 \frac{\mu_B \, \mu_0}{\pi} d  \left [ \frac{I_D}{L^2} + \frac{I_d}{l^2} + 3 \left ( \frac{I_D}{L^4} + \frac{I_d}{l^4} \right ) x^2  +  5 \left ( \frac{I_D}{L^6} + \frac{I_d}{l^6} \right ) x^4 \right ] \\
 	& + & \frac{\mu_B}{B_0} \left ( \frac{\mu_0}{\pi} \right )^2  \left [ \left ( \frac{I_D}{L^2} + \frac{I_d}{l^2} \right ) x^2  + 2 \left ( \frac{I_D}{L^2} + \frac{I_d}{l^2} \right )\left ( \frac{I_D}{L^4} + \frac{I_d}{l^4} \right ) x^4 \right ].
 \end{eqnarray*}
 
 En ajustant $I_D$ et $i_d$, on peut réaliser par exemple un double puits \cite{Schemmer2019}, ou bien supprimer le terme quadratique $x^2$ afin d’obtenir un potentiel quartique pur :
\begin{eqnarray*}
	V_\parallel(x) = a_0 + a_4 x^4 	
\end{eqnarray*}
comme on le fais dans \cite{Dubois2025}.

En pratique, la puce présente des dimensions finies et n’est pas parfaitement symétrique. Un calcul plus précis, prenant en compte la géométrie exacte (disposition et épaisseur des fils), est présenté en annexe de la thèse de Thibault Jacqmin \cite{???}, p. 151. Cela impose un ajustement fin et asymétrique des courants $I_D$, $I_{D'}$, $I_d$ et $I_{d'}$.


On ajuste les courant $I_D$ et $i_d$ pour par exemple fais des douple puit \cite{Schemmer2019} ou en supriment le terme en $x^2$  d’obtenir un potentiel longitudinal quartique de la forme $V_\parallel(x) = a_0 + a_4 x^4$ \cite{Dubois2025}.\\

En réalité la puce presente des dimention finie, Un calcul plus précis prenant en compte la géométrie exacte des fils (disposition sur la
puce, épaisseur finie) se trouve en appendice de la thèse de Thibault Jacqmin [112] , page 151. De plus la pude n'est pas pardetement symetrique donc on doit ajuster les courant $I_D$, $I_{D'}$, $I_d$ et $I_{d'}$.


\paragraph{Caractérisation des potentiels longitudinal et transverse.}
Pour atteindre le régime unidimensionnel, les confinements doivent être fortement anisotropes : un piégeage transverse très fort et un piégeage longitudinal faible. La condition \(\mu, k_B T \ll \hbar \omega_\perp\) garantit le gel des degrés de liberté transverses.

\medskip

Cette configuration est particulièrement adaptée pour obtenir des profils de densité homogènes, nécessaires à certaines expériences de transport. Le transfert des atomes du piège harmonique vers le piège quartique est réalisé de manière \emph{diabatique} (changement rapide du potentiel), car un transfert adiabatique entraîne des pertes importantes.

\paragraph{Caractérisation des potentiels longitudinal et transverse.}
Pour atteindre le régime unidimensionnel, les potentiels de piégeage doivent être très asymétriques : un confinement transverse fort et un confinement longitudinal faible. La fréquence transverse \(\omega_\perp\) doit être suffisamment élevée pour geler les degrés de liberté dans cette direction, avec la condition \(\mu, k_B T \ll \hbar \omega_\perp\).

%\paragraph{Potentiel longitudinal}
%
%Le confinement longitudinal est produit par des courants continus ou modulés dans certains fils. Dans certains protocoles spécifiques, on utilise un potentiel quartique \( V_\parallel(x) = c_4 x^4 \). Le système reste dans le régime 1D tant que la longueur caractéristique longitudinale reste beaucoup plus grande que la transverse.
%
%\paragraph{Potentiel transverse}
%
%Le confinement transverse est réalisé à l’aide de trois micro-fils parallèles situés sur la puce : un fil central parcouru par un courant \( I \), et deux fils latéraux par des courants opposés \(-I\). Cette configuration crée un piège transverse harmonique avec une fréquence \(\omega_\perp\) contrôlable par la valeur du champ \( B_0 \) et le courant. Les atomes sont piégés à environ \( d = 15~\mu\text{m} \) au-dessus de la puce. La fréquence maximale accessible expérimentalement est de l’ordre de \( \sim 100~\text{kHz} \).
%
%\paragraph{Effet de rugosité et suppression par modulation}
%
%La rugosité des micro-fils induit des fluctuations parasites du champ magnétique le long du guide. Pour supprimer cet effet, les courants sont modulés à haute fréquence (environ 400~kHz). Grâce à cette modulation rapide, les atomes ne ressentent que le potentiel moyen, dans lequel la composante parasite longitudinale du champ s’annule. Ce procédé permet d’obtenir un potentiel transverse régulier et stable, avec une fréquence efficace \[ f_\perp = \frac{f_\perp^{(0)}}{\sqrt{2}}. \]
%
%\paragraph{Découplage des confinements transverse et longitudinal.}
%Dans notre dispositif, le confinement transverse est assuré par les micro-fils modulés, tandis que le confinement longitudinal est généré par quatre fils extérieurs (D, D', d, d'). L’analyse du potentiel magnétique moyen montre que, sous l’hypothèse d’un champ de bobine homogène et dominant, les contributions transverse et longitudinale du potentiel sont découplées. Cette propriété est cruciale pour nos expériences : elle permet de modifier la géométrie du potentiel longitudinal sans perturber le confinement transverse, facilitant ainsi l’exploration de différentes configurations dynamiques.
%
%\paragraph{Piégeage longitudinal harmonique.}
%Un piège longitudinal harmonique est réalisé en appliquant des courants égaux dans les fils D et D', disposés de manière symétrique. Le champ magnétique longitudinal produit conduit à un potentiel quadratique local :
%\[
%V_\parallel(x) = V_0 + \frac{1}{2} m \omega_\parallel^2 x^2,
%\]
%avec une fréquence $\omega_\parallel$ contrôlée par le courant et la géométrie de la puce. En pratique, des fréquences jusqu’à 150 Hz sont atteintes pour des courants de 4 A. Une correction peut être nécessaire pour prendre en compte un champ magnétique résiduel $B_{0v}$, responsable d’un déplacement du centre du nuage atomique.
%
%\paragraph{Piégeage longitudinal quartique.}
%L’ajout de deux fils supplémentaires (d et d') permet de modifier la forme du potentiel longitudinal jusqu’à l’ordre 4. En ajustant les courants dans les quatre fils, on peut annuler le terme quadratique et obtenir un potentiel quartique :
%\[
%V_\parallel(x) = a_0 + a_4 x^4.
%\]
%Cette configuration est particulièrement adaptée pour générer des profils de densité homogènes, comme requis dans certaines expériences de transport. Le transfert des atomes du piège harmonique vers le piège quartique est réalisé de manière diabatique (changement rapide du potentiel), car un transfert adiabatique entraînait des pertes importantes.



\section{Sélection spatiale avec DMD}
\subsection{Motivation et principe}
{\color{blue}
\begin{itemize}
    \item Besoin de préparer des tranches homogènes.
    \item Intérêt dans les protocoles hors équilibre.
\end{itemize}
}

\paragraph{Objectif du dispositif de sélection}

L’outil de sélection spatiale a été conçu pour permettre une action locale sur le gaz atomique. Il présente deux objectifs principaux. D’une part, il permet de mesurer la distribution de rapidité localement résolue, en sélectionnant une tranche du gaz avant de la libérer et de suivre son expansion. D’autre part, il offre la possibilité de créer des situations hors équilibre en retirant une partie du gaz à l’équilibre, ce qui perturbe la configuration initiale et initie une dynamique.

\paragraph{Intérêt pour les protocoles hors équilibre}

Ce dispositif permet ainsi de générer des protocoles analogues à des configurations classiques comme le pendule de Newton, ou de sonder directement la dynamique d’un gaz de Lieb-Liniger dans des conditions contrôlées. Il constitue une brique essentielle pour les expériences de dynamique et de transport quantique.


\subsection{Mise en place technique (initiée par Léa Dubois)}

{\color{blue}
\begin{itemize}
    \item Dispositif optique de projection.
    \item Contrôle numérique des motifs.
    \item Calibration et stabilité.
\end{itemize}
}

\paragraph{Principe de sélection par pression de radiation}

La sélection repose sur l’illumination d’une zone définie du gaz avec un faisceau quasi-résonant avec la transition cyclique \( F=2 \rightarrow F'=3 \) de la ligne D2 du rubidium. Les atomes subissent une pression de radiation due aux cycles absorption/émission spontanée, ce qui les pousse hors du piège ou les amène dans un état non piégé.

\paragraph{Façonnage spatial du faisceau}

La sélection doit être spatialement résolue. Le profil d’intensité dans le plan des atomes est de type binaire :
\[
I(x) = 
\begin{cases}
0 & \text{si } x \in [x_1, x_2] \\
I_0 & \text{sinon}
\end{cases}
\]
ce qui permet de préserver ou d’éjecter les atomes selon leur position longitudinale.

\paragraph{Utilisation du DMD}

Pour générer ce profil, un DMD (Digital Micromirror Device) est utilisé. Il s’agit d’une matrice de \(1024 \times 768\) micro-miroirs orientables individuellement (±12°). En inclinant ces miroirs, on contrôle localement la réflexion de la lumière. L’image du DMD est projetée directement sur le plan des atomes, en imagerie directe.

\paragraph{Avantages du DMD}

Le DMD permet une reconfiguration rapide et programmable du motif de lumière. Cette technologie est largement utilisée dans les expériences d’atomes froids pour produire des potentiels structurés, homogénéiser un faisceau ou adresser localement les atomes.

\paragraph{Alternatives possibles}

Il est possible, en théorie, d’atteindre un effet similaire par un transfert cohérent des atomes vers un état anti-piégé via un pulse micro-onde ou une transition Raman. Cependant, la méthode par pression de radiation est plus simple à mettre en œuvre et adaptée à nos objectifs expérimentaux.

\paragraph{Principe de l’expulsion par pression de radiation}

Un atome illuminé par un faisceau proche de la résonance peut être expulsé du piège soit par transition vers un état anti-piégé, soit par effet de pression de radiation. Cette dernière génère une accélération suffisante pour fournir une énergie cinétique supérieure à la profondeur du puits magnétique. Le nombre de photons diffusés nécessaire peut être estimé à partir de la conservation de l’impulsion : une vingtaine de photons suffisent typiquement à extraire un atome du piège dans nos conditions.

\paragraph{Modèle de diffusion et estimation du seuil}

Le taux de diffusion de photons est modélisé à l’aide d’un taux \(\Gamma_{\mathrm{sc}}\), dépendant de l’intensité \(I\), de l’intensité de saturation \(I_{\mathrm{sat}}\), d’un paramètre \(\alpha\) (lié à la polarisation et au champ magnétique) et du désaccord \(\delta\). À résonance, et pour un temps d’illumination \(\tau_p\), on peut estimer le nombre total de photons diffusés par atome par \(N_{\mathrm{sc}} = \tau_p \Gamma_{\mathrm{sc}}\).

\paragraph{Mesures expérimentales de la puissance nécessaire}

La puissance minimale nécessaire pour éjecter tous les atomes d’une zone illuminée est déterminée en fixant un temps d’illumination donné, puis en variant l’intensité du faisceau. L’analyse est réalisée après un délai d’attente de \(\sim 10\) ms, pour s’assurer que seuls les atomes encore piégés soient détectés. Il est observé que 99$\%$ des atomes sont retirés à partir d’un rapport \(I/I_{\mathrm{sat}} \simeq 0.12\).

\paragraph{Mesures de photons diffusés par fluorescence}

La quantité de photons diffusés est également mesurée par l’analyse du signal de fluorescence capté par la caméra. En calibrant le rapport entre photons détectés et photons diffusés (en tenant compte de l’efficacité optique du système), le nombre moyen de photons nécessaires pour éjecter un atome est confirmé expérimentalement autour de 20. Un ajustement du modèle de diffusion permet d’estimer le paramètre \(\alpha \simeq 0.4\).

\paragraph{Saturation et effets Doppler}

À fort temps d’illumination (\(\tau_p > 150\,\mu\)s), une saturation du nombre de photons diffusés est observée, interprétée comme un effet géométrique : les atomes accélérés atteignent physiquement la puce atomique et cessent de contribuer au signal. Une correction Doppler peut être introduite dans le modèle, mais reste négligeable (\(< 5\%\)) dans les régimes expérimentaux utilisés.

\paragraph{Limitations expérimentales de la sélection}

Plusieurs effets peuvent limiter l'efficacité ou la propreté de la sélection :
\begin{itemize}
    \item La diffraction liée à la taille finie de l’objectif entraîne un flou de l’ordre de \(1{-}2\,\mu\)m au bord des zones éclairées.
    \item Une diffusion parasite par la puce peut se produire à forte intensité si tout le DMD est illuminé ; cela est évité en réduisant la taille transverse du faisceau à quelques micro-miroirs seulement.
    \item Des inhomogénéités d’éclairement dues à la gaussienne du faisceau et au speckle peuvent conduire à une sur-illumination de certaines zones. Un effort a été fait pour homogénéiser l’intensité en sortie de fibre.
    \item La réabsorption des photons diffusés pourrait entraîner un échauffement du gaz restant. Un désaccord en fréquence de 15 MHz a été testé pour éviter ce phénomène, sans effet visible sur la température du gaz.
\end{itemize}

\paragraph{Mesures de l’impact sur le gaz restant}

La température du gaz sélectionné est comparée avant et après sélection via l’analyse des fluctuations de densité après temps de vol. Aucun changement significatif de température ni d’élargissement n’a été observé. Ces résultats suggèrent que, dans les conditions expérimentales utilisées, la sélection ne perturbe pas significativement les atomes restants.




\subsection{Utilisation dans les protocoles}

{\color{blue}
\begin{itemize}
    \item Formes utilisées : boîtes, barrières, coupures.
    \item Préparation initiale contrôlée du gaz.
    \item Exemples de protocoles expérimentaux utilisant le DMD
\end{itemize}
}

\paragraph{Sélection locale et mesure de rapidité}

En sélectionnant une tranche du gaz, on peut ensuite couper le confinement longitudinal et laisser cette tranche s’étendre. Le profil de densité asymptotique obtenu après un long temps d’expansion est proportionnel à la distribution de rapidité locale du gaz initial. Ce protocole permet ainsi une mesure résolue de \(\rho(x,t \to \infty) \sim \rho(v)\).

\paragraph{Génération d’états hors équilibre}

La sélection permet également de créer des discontinuités dans le profil de densité, et donc d’initier une dynamique hors équilibre. Par exemple, on peut ne conserver que deux paquets séparés de gaz, qui vont alors osciller l’un vers l’autre. Cette configuration est analogue à un pendule de Newton quantique.

\paragraph{Formes utilisées}

Les motifs projetés par le DMD peuvent prendre différentes formes : boîtes, barrières, coupures, etc. Cette flexibilité rend l’outil extrêmement précieux pour explorer diverses configurations initiales et protocoles dynamiques.

\paragraph{Contrôle logiciel du DMD}

Le pilotage du DMD repose sur l’utilisation d’un module intégré fourni par Vialux (V7001-SuperSpeed), qui comprend les bibliothèques logicielles ALP-4. Plusieurs configurations du DMD peuvent être chargées en mémoire au début de chaque cycle expérimental, puis sélectionnées en cours de séquence à l’aide d’un signal digital. Le temps de commutation des miroirs est inférieur à \(30\,\mu\mathrm{s}\), ce qui est compatible avec les protocoles étudiés.

\paragraph{Partage du faisceau avec la voie d’imagerie}

Le faisceau utilisé pour la sélection spatiale est prélevé à partir du faisceau sonde déjà accordé sur la transition \(F=2 \rightarrow F'=3\) de la raie D2. Le partage est réalisé à l’aide d’un cube séparateur de polarisation placé en aval d’une lame demi-onde, permettant de contrôler la puissance injectée dans la fibre optique. Ce choix simplifie la mise en œuvre en évitant d’ajouter une source laser supplémentaire.

\paragraph{Blocage du faisceau de sélection}

Deux systèmes permettent de couper le faisceau de sélection pendant le cycle expérimental :
\begin{itemize}
    \item un cache mécanique (type électro-aimant), utilisé pour un blocage longue durée ;
    \item un modulateur acousto-optique (AOM), permettant de produire des impulsions brèves de quelques dizaines de \(\mu\mathrm{s}\), en amont du séparateur.
\end{itemize}
Pour garantir que le faisceau ne perturbe pas l’imagerie, le cache mécanique reste fermé pendant l’utilisation du faisceau sonde.

\paragraph{Montage optique de projection}

Le faisceau façonné par le DMD est projeté dans le plan des atomes à l’aide d’un système optique permettant de sélectionner l’ordre 0 de diffraction. L’ensemble des optiques est dimensionné (diamètre \(50\,\mathrm{mm}\)) pour limiter la diffraction. L’alignement est effectué en superposant le faisceau de sélection à la voie d’imagerie.

\paragraph{Grandissement et champ couvert}

Le montage permet de couvrir une zone de l’ordre de \(600\,\mu\mathrm{m}\) dans le plan des atomes, soit plus que la longueur typique d’un nuage (\(\sim 400\,\mu\mathrm{m}\) pour \(f_{\parallel}=5\,\mathrm{Hz}\)). Le grandissement est déterminé par les focales utilisées : une focale \(f_1 = 750\,\mathrm{mm}\) du côté du DMD, et \(f = 32\,\mathrm{mm}\) pour l’objectif côté atomes, donnant \(G = f/f_1 \approx 0.043\).

\paragraph{Visualisation et interface}

Le contrôle du DMD s’effectue via une interface graphique permettant de prévisualiser les configurations de miroirs. Une capture d’écran de cette interface est présentée dans la Fig.~\ref{fig:dmd_interface}, où la zone active réfléchie est visualisée en rouge. Cette interface est pilotée de manière automatisée pendant le déroulement de la séquence expérimentale.


\section{Techniques d’imagerie et d’analyse}
\subsection{Imagerie par absorption}
{\color{blue}
\begin{itemize}
    \item Imagerie \textit{in situ} et après temps de vol.
    \item Résolution, limites instrumentales.
\end{itemize}
}

\paragraph{Système d’imagerie par absorption}

L’imagerie est réalisée à l’aide d’une caméra CCD à déplétion profonde, optimisée pour une grande efficacité quantique à la longueur d’onde de 780 nm. On utilise des techniques d’imagerie par absorption permettant d’extraire la densité optique \( D(x, z) \), elle-même reliée à la densité atomique 3D via la loi de Beer-Lambert. Le profil de densité linéaire \( n(x) \) est obtenu par intégration sur les directions transverses.

\paragraph{Imagerie après temps de vol}

En appliquant un champ magnétique vertical (\( B = 8\,\mathrm{G} \)), la polarisation du faisceau peut être rendue circulaire (\( \sigma^+ \)) pour adresser la transition fermée \( |F=2, m_F=2\rangle \rightarrow |F'=3, m_F'=3\rangle \). Cette configuration assure une meilleure définition de la section efficace d’absorption. Un temps de vol de quelques ms est utilisé avant l’imagerie, permettant également de décomprimer le nuage.

\paragraph{Imagerie in situ}

Sans champ magnétique, les atomes sont imagés à $7~\mu m$ de la puce, ce qui implique une double absorption du faisceau incident et réfléchi. Dans ce cas, la transition n’est pas fermée, ce qui nécessite une calibration du facteur de conversion entre la densité mesurée et la densité réelle. Un ajustement linéaire permet de relier les profils in situ aux profils obtenus après temps de vol.

\paragraph{Choix des paramètres d’imagerie}

L’intensité du faisceau sonde est choisie typiquement à \( I_0/I_{\mathrm{sat}} \approx 0.3 \) pour optimiser le rapport signal sur bruit tout en restant dans une zone de linéarité acceptable. Dans ces conditions, le nombre de photons diffusés est de l’ordre de \( N_{\mathrm{sc}} \approx 230 \) et le rayon de diffusion reste comparable à la résolution du système d’imagerie (\( \sim 2.6\,\mu \mathrm{m} \)).

\paragraph{Limites du modèle de Beer-Lambert}

La validité de la loi de Beer-Lambert repose sur une approximation à une particule. Dans le cas des gaz fortement denses ou quasi 1D, les effets collectifs, les réabsorptions et les couplages dipolaires peuvent invalider ce modèle. Pour cette raison, même pour l’imagerie in situ, un temps de vol court (\( \sim 1\,\mathrm{ms} \)) est souvent appliqué afin de diluer le gaz transversalement.

\paragraph{Défauts et instabilités expérimentales}

Plusieurs limitations instrumentales ont été identifiées :
\begin{itemize}
    \item La caméra initialement utilisée montrait des motifs parasites aléatoires ainsi qu’un offset variant au cours du temps. Le remplacement de la caméra a permis de résoudre ces problèmes.
    \item Des franges d’interférences apparaissaient lors de la division des images d’absorption, probablement dues à des effets Fabry-Pérot dans les optiques. Le désaxage du faisceau d’imagerie a permis d’en limiter l’impact.
    \item Des photons résiduels, même en l’absence de faisceau sonde, ont été détectés. Ces derniers proviennent vraisemblablement de diffusions multiples dans le système optique.
\end{itemize}

\paragraph{Conclusion}

La combinaison de l’imagerie in situ et après temps de vol, ainsi qu’une calibration soigneuse des paramètres optiques et expérimentaux, permettent d’accéder à des profils de densité fiables malgré les limites intrinsèques du système d’imagerie. Une attention particulière a été portée à la réduction des artefacts expérimentaux afin de garantir la précision des mesures.


\subsection{Analyse des profils}

{\color{blue}
\begin{itemize}
    \item Extraction des densités, tailles, températures.
    \item Distribution longitudinale.
    \item Estimation de la température par ajustement Yang-Yang (optionnel si pertinent).
\end{itemize}
}


\section{Expériences et protocoles étudiés}
Cette section peut être la plus personnelle, en précisant ton rôle à chaque fois.
\subsection{Expansion longitudinale}
\begin{itemize}
    \item Protocole d’expansion (libération longitudinale, maintien du confinement transverse).
    \item Suivi de l’évolution du profil.
    \item Analyse à différents temps d’expansion
    \item Comparaison aux modèles analytiques : solutions homothétiques, GP, asymptotiques.
\end{itemize}

\subsection{Motivation et protocole expérimental d’expansion longitudinale}

\paragraph{Motivation.}
Une partie essentielle de mon travail de thèse a consisté à sonder la distribution de rapidités résolue spatialement, ce qui constitue une information clé pour comprendre la dynamique hors équilibre d’un gaz quantique unidimensionnel. Pour accéder à cette observable, il est nécessaire de réaliser un protocole qui relie la distribution de rapidités à des profils de densité mesurables expérimentalement. L’expansion longitudinale dans le guide 1D s’impose alors comme un outil naturel : en laissant le nuage se dilater librement dans la direction longitudinale, on convertit en partie l’information contenue dans les phases et les excitations collectives du système en une dynamique de densité directement accessible par imagerie. Ce protocole permet ainsi de comparer les prédictions issues des équations effectives, comme l’équation de Gross–Pitaevskii dans différents régimes de confinement, avec des mesures expérimentales résolues spatialement.

\paragraph{Considérations physiques.}
Au-delà de son intérêt pratique, l’expansion longitudinale offre une fenêtre unique sur la physique des gaz bosoniques 1D. Elle permet d’étudier comment un système initialement confiné évolue vers un état dilué, révélant à la fois l’impact du régime transverse (TF 3D vs TF 1D) et l’influence des fluctuations de phase. Dans le régime TF 1D, ces fluctuations deviennent dominantes et se traduisent par des ondulations de densité mesurables. Leur analyse expérimentale, via le spectre de puissance, fournit un accès direct aux corrélations de phase et à la thermodynamique effective du gaz.

\paragraph{Protocole expérimental.}
Concrètement, l’expansion longitudinale est réalisée selon la séquence illustrée en Fig.~??? :
\begin{itemize}
  \item Le nuage est initialement piégé dans un potentiel magnétique caractérisé par une fréquence longitudinale $f_{\parallel} = 5.0$ ou $9.4\,\mathrm{Hz}$ selon les jeux de données, et une fréquence transverse $f_{\perp} = 2.56\,\mathrm{kHz}$.
  \item À $t=0$, le confinement longitudinal est éteint en annulant les courants $I_D=I_{D'}=0$. La coupure est réalisée sur un temps fini $t_{\parallel} = 70\,\mu\mathrm{s} \ll 1/f_{\parallel}$, ce qui évite un pic de courant parasite tout en préservant la dynamique du gaz.
  \item Le nuage se dilate librement dans la direction longitudinale pendant une durée $\tau$. Ensuite, le confinement transverse est relâché en annulant $I_{\perp}$, avec un temps de coupure $t_{\perp} = 5\,\mu\mathrm{s} \ll 1/f_{\perp}$.
  \item Une image par absorption est enfin prise après un temps de vol $t_v$. Pour l’étude des profils de densité, on utilise typiquement $t_v = 1\,\mathrm{ms}$.
\end{itemize}

%\paragraph{Découplage des confinements.}
%Comme discuté en Section~\ref{chap:...}, l’architecture expérimentale rend ce protocole particulièrement simple à mettre en œuvre. La modulation des courants transverses $I_{\perp}$ garantit que le potentiel longitudinal est découplé de celui transverse, ce qui permet un contrôle précis et indépendant des deux confinements.

\paragraph{Perspective.}
La mise en œuvre de ce protocole d’expansion longitudinale ne répond donc pas seulement à un besoin technique de mesure, mais s’inscrit dans une stratégie plus générale : relier les prédictions théoriques de la GHD et des modèles effectifs à des observables accessibles, et sonder directement l’évolution des fluctuations et des corrélations dans un système quantique 1D.

\paragraph{Équations Gross-Pitaevskii dépendantes du temps.}
La dynamique du système étudié est décrite par l’équation de Gross-Pitaevskii (GP) \eqref{chap.1:eq.GP.1} :
\begin{eqnarray*}
	i \partial_\tau\phi = \left \{ - \frac{1}{2}\Delta_{\vec{r}} + V(\vec{r}) + g_{\mathrm{3D}} N \vert \phi \vert^2 \right \} \phi,
\end{eqnarray*}
avec $g_{\mathrm{3D}} = 4 \pi a_{\mathrm{3D}}$ et en présence d’un potentiel externe (voir \eqref{} et \eqref{}) :
\begin{eqnarray*}
	V(\vec{r}) = V_\perp(\vec{r}_\perp) + V_\parallel(x), 
	\qquad 
	V_\perp(\vec{r}_\perp) = \tfrac{1}{2} \, \omega_\perp^2 \, \vec{r}_\perp^2, 
	\qquad 
	V_\parallel(x) = \tfrac{1}{2} \, \omega_\parallel^2 \, x^2.
\end{eqnarray*} 


\paragraph{Séparation des degrés de liberté.}
Dans un piège de type cigare, caractérisé par $\omega_\perp \gg \omega_\parallel$, la dynamique transverse se déroule sur des temps caractéristiques beaucoup plus courts que la dynamique longitudinale. On fait alors l’hypothèse d’un \emph{suivi adiabatique transverse} : l’état reste en permanence dans son état fondamental transverse. Ainsi, les degrés de liberté transverses et longitudinaux se découplent et la fonction d’onde peut se factoriser sous la forme
\begin{equation}
    \phi(r,\tau) = \psi(x,\tau)\,\Phi\!\left(\vec{r}_\perp, n(x,\tau)\right),
\end{equation}
où $\psi(x,\tau)$ décrit la dynamique longitudinale et $\Phi$ est la fonction d’onde transverse dépendant paramétriquement de la densité linéaire $n(x,\tau)$. La condition de normalisation 
\(
\int d \vec{r}_\perp \, \big|\Phi\!\left(\vec{r}_\perp, n\right)\big|^2 = 1
\)
permet de réécrire la densité linéaire définie par 
\(
n \doteq N \int d \vec{r}_\perp \, |\phi|^2
\)
sous la forme
\begin{eqnarray*}
	n(x,\tau) = N \, |\psi(x,\tau)|^2.
\end{eqnarray*}
L’équation de Gross-Pitaevskii se réécrit alors
\begin{eqnarray}
	\left( i \partial_\tau + \tfrac{1}{2} \partial_x^2 - V_\parallel(x) - \mu(n) \right) \psi = 0, 
	\qquad 
	\mu(n)\,\Phi = \left( - \tfrac{1}{2} \Delta_{\vec{r}_\perp} + V_\perp + g_{\mathrm{3D}} \, n \, \big|\Phi(\vec{r}_\perp,n)\big|^2 \right)\Phi.
\end{eqnarray}

\paragraph{Équations hydrodynamiques.}
En utilisant la transformation de Madelung 
\(
\psi(x,\tau) = \sqrt{n(x,\tau)} \, e^{i \vartheta(x,\tau)},
\)
et en introduisant la vitesse $u = \partial_x \vartheta$, on obtient les équations hydrodynamiques associées :
\begin{eqnarray}\label{chap:5:eq.hydro.1}
	\left\{
	\begin{array}{rcl}
		\partial_\tau n + \partial_x ( n u )	 & = & 0, \\[0.3em]
		\partial_\tau u + \partial_x \left( \tfrac{u^2}{2} + V_\parallel(x) + \mu(n) + Q(n) \right) & = & 0,
	\end{array} 
	\right.
\end{eqnarray}
où le terme de pression quantique est donné par
\(
Q(n) = - \frac{1}{2} \, \frac{\partial_x^2 \sqrt{n}}{\sqrt{n}}.
\)
Ces équations sont équivalentes aux deux premières de \eqref{chap:3:eq:hydro.1}, en tenant compte de la relation thermodynamique $dP = n \, d\mu$ et en négligeant le terme de pression quantique $Q(n)$.

\medskip

Pour notre protocole, pour $\tau < 0$ le système est à l’équilibre, avec la condition
\(
\mu(n) + V_\parallel(x) = \mu\bigl(n(x=0)\bigr).
\)
Pour $\tau \geq 0$, le potentiel longitudinal est éteint : $V_\parallel(x) = 0$.

\medskip

\paragraph{Solutions analytiques homothétique.}
Si $n$ est solution des equation hydrodynamique \eqref{chap:5:eq.hydro.1} , pour $\tau \geq 0$. On fais l'hypothèse que la densité linéaire suit une forme homothétique
\begin{eqnarray}
	n(x,\tau) = \frac{1}{\lambda(\tau)} n_0 \left ( \frac{x}{\lambda(\tau)} \right ) ,	
\end{eqnarray}
avec $n_0$ le profil de densité à $\tau = 0 $ et $\lambda(\tau)$ le facter d'echelle à une temps d'expension $\tau$. Avec les containtes $\lambda(0) = 1$ et $\lambda'(0) = 0$ et $N = \int dx \, n(x , \tau ) $. En injectant dans \eqref{chap:5:eq.hydro.1} il vient que 
\begin{eqnarray}\label{chap:5:eq.hydro.2}
	\left\{
	\begin{array}{rcl}
		u(x, \tau ) & = & \displaystyle \frac{\dot\lambda(\tau)}{\lambda(\tau)} x , \\[0.3em]
		\partial_x \mu ( n ( x , \tau ))  & = & - \displaystyle \frac{\ddot\lambda(\tau)}{\lambda(\tau)} x,
	\end{array} 
	\right.
\end{eqnarray}
(car \(\partial_\tau u=(\ddot\lambda/\lambda-\dot\lambda^2/\lambda^2)x\) et \(v\partial_x v=(\dot\lambda/\lambda)^2 x\), leur somme donne \((\ddot\lambda/\lambda)x\)) et initialement $\mu( n_0 ( x ) ) = \mu( n_0 ( x = 0  ) ) - \frac{1}{2} \omega_\parallel^2 x^2 $.

\medskip

Calculons maintenant \(\partial_x\mu(n(x))\). D'abord
\[
\partial_x n(x)=\frac{1}{\lambda^2}\,n_0'\!\Big(\frac{x}{\lambda}\Big).
\]
À l'équilibre \(\mu\big(n_0(y)\big)=\mu_0-\tfrac12 \omega_\parallel^2 y^2\), d'où
\[
\mu'(n_0(y))\,n_0'(y)=-\omega_\parallel^2 y
\quad\Rightarrow\quad
n_0'(y)=-\frac{\omega_\parallel^2\,y}{\mu'(n_0(y))}.
\]
En prenant \(y=x/\lambda\) on obtient
\[
n_0'\!\Big(\frac{x}{\lambda}\Big)
= -\frac{\omega_\parallel^2}{\lambda}\,\frac{x}{\mu'\big(n_0(x/\lambda)\big)}.
\]
Donc
\[
\partial_x n(x) = -\frac{\omega_\parallel^2\,x}{\lambda^3}\;
\frac{1}{\mu'\big(n_0(x/\lambda)\big)}.
\]
Puis
\[
\partial_x\mu(n(x))=\mu'\big(n(x)\big)\,\partial_x n(x)
= -\frac{m\omega_\parallel^2\,x}{\lambda^3}\;
\frac{\mu'\big(n(x)\big)}{\mu'\big(n_0(x/\lambda)\big)}.
\]
Or \(n_0(x/\lambda)=\lambda\,n(x)\), donc on définit
\[
f(\lambda)\equiv\frac{\mu'(n)}{\mu'(\lambda n)}.
\]
On obtient finalement
\[
\partial_x\mu(n(x)) = -\frac{\omega_\parallel^2}{\lambda^3}\,f(\lambda)\,x.
\]

%On souhaite calculer $\partial_x \mu\bigl(n(x,\tau)\bigr)$ en utilisant la règle de la chaîne et la forme homothétique. On a
%\[
%	\partial_x \mu\bigl(n(x)\bigr) 
%	= \mu'(n(x)) \, \partial_x n(x) 
%	= \frac{1}{\lambda^2} \, \mu'(n(x)) \, \partial_x n_0\!\left(\tfrac{x}{\lambda}\right),
%\]
%où le dernier terme s’écrit
%\[
%	\left. \frac{\partial n_0}{\partial x} \right|_{x/\lambda} 
%	= \left. \frac{\partial n_0}{\partial \mu} \right|_{\mu(n_0(x/\lambda))} 
%	\left. \frac{\partial \mu}{\partial x} \right|_{n_0(x/\lambda)} .
%\]
%On utilise alors 
%\[
%	\left. \frac{\partial \mu}{\partial x} \right|_{n_0(x/\lambda)} = - \frac{\omega_\parallel^2}{\lambda^2} \, x,
%	\qquad 
%	\left. \frac{\partial n_0}{\partial \mu} \right|_{\mu(n_0(x/\lambda))} 
%	= \left. \frac{\partial n}{\partial \mu} \right|_{\mu(\lambda n)} .
%\]
%Il vient donc, en utilisant de plus la deuxième équation de 
En remplaçant dans la deuxième d'Euler \eqref{chap:5:eq.hydro.2} et en simplifiant  \(x\),
\begin{eqnarray}\label{chap:5:eq.hydro.3}
	\frac{\ddot\lambda}{\lambda}  
	 =  \frac{\omega_\parallel^2}{\lambda^3} \, f(\lambda)  .
\end{eqnarray}

\paragraph{Proposition.}
Si le facteur
\(
f(\lambda)
\)
est bien défini indépendamment de \(n>0\) (ce qui est le cas pour les solutions homothétiques),
alors \(f\) est une loi de puissance.

\paragraph{Preuve.}
Posons \(g(n) = \mu'(n)>0\) ou \(<0\) (\ie $\mu$ strictement monotone). La définition de \(f\) équivaut à l’existence d’une fonction
\(\chi(\lambda)=1/f(\lambda)\) telle que
\[
g(\lambda n)=\chi(\lambda)\,g(n)\qquad(\forall\,\lambda,n>0).
\]
En prenant \(n=1\), on a \(\chi(\lambda)=g(\lambda)/g(1)\).
Donc, pour tous \(a,b>0\),
\[
\chi(ab)=\frac{g(ab)}{g(1)}=\frac{\chi(a)\,g(b)}{g(1)}=g(a)\,g(b),
\]
c’est-à-dire que \(\chi\) est \emph{multiplicative}. Sous une hypothèse physique très faible
(continuité/mesurabilité ou simple localement bornée), toute fonction multiplicative sur
\(\mathbb{R}_+^\ast\) est de la forme
\[
\chi(\lambda)=\lambda^{\alpha-1}
\quad\Rightarrow\quad
f(\lambda)=\lambda^{1-\alpha}.
\]
%En réintégrant \(g=\mu'\propto n^{\alpha-1}\), on retrouve \(\mu(n)\propto n^\alpha\) pour \(\alpha\neq 0\)
%(et \(\mu(n)\propto \ln n\) pour \(\alpha=0\)).
\qed


% --- Démonstration que f(λ)=λ^{1-\alpha} et réciproque ---
\paragraph{Proposition.}
$f(\lambda) = \lambda^{1-\alpha}$ et $\mu (n) \propto n^\alpha $ sont equivalents.
%
%On définit
%\[
%f(\lambda)=\frac{\mu'(n)}{\mu'(\lambda n)},
%\]
%en supposant \(\mu\in C^1\) et \(\mu'(n)>0\) pour \(n>0\).

\paragraph{1. Si \(\mu(n)=C\,n^\alpha\) (avec \(C\neq0\)) :}
Alors \(\mu'(n)=C\alpha\,n^{\alpha-1}\). Par conséquent
\[
f(\lambda)=\frac{C\alpha\,n^{\alpha-1}}{C\alpha\,(\lambda n)^{\alpha-1}}
=\lambda^{1-\alpha}.
\]

\paragraph{2. Réciproque : si \(f(\lambda)=\lambda^{1-\alpha}\) pour tout \(\lambda>0\) (et tout \(n>0\)) :}
Posons \(g(n)=\mu'(n)\). L'hypothèse s'écrit
\[
\frac{g(n)}{g(\lambda n)}=\lambda^{1-\alpha}
\quad\Longleftrightarrow\quad
g(\lambda n)=\lambda^{\alpha-1}\,g(n),
\]
pour tout \(n>0\) et tout \(\lambda>0\).

Fixons \(n_0>0\) et définissons \(\varphi(\lambda)\equiv g(\lambda n_0)\). La relation ci-dessus donne
\[
\varphi(\lambda)=\lambda^{\alpha-1}\,\varphi(1).
\]
Autrement dit \(\varphi(\lambda)=C_1\,\lambda^{\alpha-1}\) pour une constante \(C_1=\varphi(1)=g(n_0)\). En remplaçant \(\lambda=x/n_0\) on obtient pour tout \(x>0\)
\[
g(x)=C_1\,x^{\alpha-1}.
\]
Ainsi \(g(n)=\mu'(n)=C\,n^{\alpha-1}\) avec \(C\) constant.

En intégrant (en supposant \(\alpha\neq 0\)),
%\[
%\mu(n)=\int \mu'(n)\,dn = \int C\,n^{\alpha-1}\,dn = \frac{C}{\alpha}\,n^\alpha + \text{const},
%\]
%donc 
\(\mu(n)\propto n^\alpha\). (Pour \(\alpha=0\) on obtient \(\mu'(n)=C\,n^{-1}\) et \(\mu(n)=C\ln n+\text{const}\).)

\paragraph{Remarque sur les hypothèses.}
La démonstration utilise la propriété fonctionnelle multiplicative
\(g(\lambda n)=\lambda^{\alpha-1}g(n)\). Sous une hypothèse faible de continuité (ou dérivabilité) en \(n\) cette équation force la forme de puissance \(g(n)\propto n^{\alpha-1}\). Sans régularité, des solutions pathologiques peuvent exister mais ne sont pas physiquement pertinentes dans le contexte thermodynamique.

\qed

%% --- Bref argument (à insérer) ---
%En posant \(g(n)=\mu'(n)\) et \(f(\lambda)=\dfrac{g(n)}{g(\lambda n)}=\lambda^{1-\alpha}\), on obtient
%\[
%g(\lambda n)=\lambda^{\alpha-1}g(n)\quad\forall\,\lambda,n>0,
%\]
%d'où \(g(n)=C\,n^{\alpha-1}\) et, pour \(\alpha\neq0\), \(\mu(n)=\dfrac{C}{\alpha}n^\alpha+\mathrm{const}\), i.e. \(\mu\propto n^\alpha\).
%
%
%% --- Encadré pour le cas alpha = 0 ---
%\medskip
%\noindent\textbf{Remarque (cas \(\alpha=0\)).} Si \(\alpha=0\) alors la relation fonctionnelle donne \(g(n)=\mu'(n)=C\,n^{-1}\). En intégrant on obtient
%\[
%\mu(n)=C\ln n + \mathrm{const}.
%\]
%Ce cas correspond physiquement, par exemple, au gaz isotherme idéal en 1D (ou plus généralement à une dépendance logarithmique du potentiel chimique), où la compressibilité \(\mu'(n)\propto 1/n\).
%
%--------------------------------------------
%
%avec $f(\lambda) = \frac{\mu'(n)}{\mu'(\lambda n )}$ avec  $\mu(n)$ est continue et strictement monitone (donc inversible). Puisque $f(1)= 1$ et $f(\lambda_1 \lambda_2) =  f(\lambda_1) f( \lambda_2)$ alors $f$ est une fonction de puissace $f(\lambda) = \lambda^{1-\beta}$. Ainssi une solution de l'éqtation hydrondynamique homothètique donne  
%\begin{eqnarray}
%	\mu(n) = a n^\beta + b,  	
%\end{eqnarray}
%avec $a, b$ et $\beta$ des réelles. 

\paragraph{Cas particulier.}
Dans le régime quasi-1D on utilise l'expression d'interpolation (cf. Salasnich et al.)
\[
\mu(n)=\hbar\omega_\perp\Big(\sqrt{1+4\,a_{\mathrm{3D}}\,n}-1\Big),
\]
où \(n\) est la densité linéique et \(a_{\mathrm{3D}}\) le scattering length. De cette formule on obtient deux limites asymptotiques :

\begin{itemize}
\item \emph{Régime transverse Thomas--Fermi (TF), \(4a_{\mathrm{3D}}n\gg1\).} 
Alors \(\sqrt{1+4a_{\mathrm{3D}}n}\simeq 2\sqrt{a_{\mathrm{3D}}n}\) et
\[
\mu(n)\simeq 2\hbar\omega_\perp\sqrt{a_{\mathrm{3D}}\,n},
\]
ce qui correspond à \(\mu\propto n^{1/2}\) (donc \(\alpha=\tfrac12\)). Ce régime décrit la situation où \(\mu\gg\hbar\omega_\perp\) et de nombreux niveaux transverses sont excités.
\item \emph{Régime quasi-1D (transverse fondamental), \(4a_{\mathrm{3D}}n\ll1\).} 
Alors \(\sqrt{1+4a_{\mathrm{3D}}n}\simeq 1+2a_{\mathrm{3D}}n\) et
\[
\mu(n)\simeq 2\hbar\omega_\perp\,a_{\mathrm{3D}}\,n \equiv g\,n,
\]
avec \(g=2\hbar\omega_\perp a_{\mathrm{3D}}\). Ici \(\mu\propto n\) (donc \(\alpha=1\)) ; on est proche de l'état fondamental transverse (gaussien).
\end{itemize}

Les deux formes ci-dessus sont bien les limites asymptotiques de l'expression d'interpolation donnée plus haut.

\medskip

Enfin, l'équation d'évolution du facteur d'échelle obtenue précédemment s'écrit correctement
\[
\boxed{\qquad \ddot\lambda\,\lambda^{\alpha+1}=\omega_\parallel^2 \qquad}
\]

% --- Première intégration de l'équation ---
On part de
\[
\ddot\lambda\,\lambda^{\alpha+1}=\omega_\parallel^2,
\]
et on pose \(v=\dot\lambda\). Comme \(\ddot\lambda=\dot\lambda\frac{d\dot\lambda}{d\lambda}\), on obtient
\[
\dot\lambda\frac{d\dot\lambda}{d\lambda}=\omega_\parallel^2\,\lambda^{-(\alpha+1)}.
\]

\paragraph{Cas \(\alpha\neq0\).}
Intégration par rapport à \(\lambda\) :
\[
\frac{1}{2}\dot\lambda^2
= \omega_\parallel^2\int \lambda^{-(\alpha+1)}\,d\lambda
= -\frac{\omega_\parallel^2}{\alpha}\,\lambda^{-\alpha} + C,
\]
où \(C\) est une constante d'intégration déterminée par les conditions initiales \(\lambda(0)=\lambda_0\), \(\dot\lambda(0)=\dot\lambda_0\) :
\[
C=\frac{1}{2}\dot\lambda_0^2+\frac{\omega_\parallel^2}{\alpha}\,\lambda_0^{-\alpha}.
\]
On a donc la première intégrale
\[
\boxed{\; \dot\lambda^2
= \dot\lambda_0^2 + \frac{2\omega_\parallel^2}{\alpha}\big(\lambda_0^{-\alpha}-\lambda^{-\alpha}\big)\; }.
\]
%La solution en quadrature s'écrit alors
%\[
%t-t_0=\int_{\lambda_0}^{\lambda(t)}\frac{d\lambda}{\sqrt{\,v_0^2 + \dfrac{2\omega_\parallel^2}{\alpha}\big(\lambda_0^{-\alpha}-\lambda^{-\alpha}\big)\,}}.
%\]

\paragraph{Cas \(\alpha=0\).}
L'équation devient \(\ddot\lambda\,\lambda=\omega_\parallel^2\). On obtient
%\[
%\frac{1}{2}v^2=\omega_\parallel^2\ln\lambda + C,
%\]
%avec \(C=\tfrac12 v_0^2-\omega_\parallel^2\ln\lambda_0\). D'où
\[
\boxed{\; \dot\lambda^2 = \dot\lambda_0^2 + 2\omega_\parallel^2\ln\!\big(\tfrac{\lambda}{\lambda_0}\big)\; }.
\]

%La quadrature est
%\[
%t-t_0=\int_{\lambda_0}^{\lambda(t)}\frac{d\lambda}{\sqrt{\,v_0^2 + 2\omega_\parallel^2\ln(\lambda/\lambda_0)\,}}.
%\]

%\paragraph{Remarques.}
%\begin{itemize}
%\item Ces intégrales donnent la solution implicite \(t(\lambda)\). En général on ne dispose pas d'une primitive élémentaire fermée pour \(\lambda(t)\) (sauf cas particuliers de choix des conditions initiales), mais la première intégrale ci-dessus est très utile pour l'analyse qualitative (points de retournement, énergie effective, petites oscillations).
%\item Pour les petites oscillations autour de \(\lambda=1\) on peut linéariser et retrouver la fréquence \(\omega_{\rm breath}=\sqrt{4+\beta}\,\omega_\parallel=\sqrt{3+\alpha}\,\omega_\parallel\) (avec \(\beta=\alpha-1\)).
%\end{itemize}


%\subsubsection{Comportement asymptotique du facteur d'échelle}



%\paragraph{Régime à temps longs.} 
%On considère la condition initiale
%\(
%\lambda(0)=1,\,\dot\lambda(0)=0.
%\) 
%et pour $\alpha > 0$ 
%Pour \(\tau\) très grand, on a \(\lambda^{-\alpha}\ll 1\). On a 
%\[
%\lambda(\tau) \simeq \frac{2}{\alpha}\,\omega_\parallel \tau.
%\]
%En particulier :  
%\begin{itemize}
%\item TF 1D (\(\alpha=1\)) : \(\lambda(\tau)\simeq \sqrt{2}\,\omega_\parallel \tau\),  
%\item TF 3D (\(\alpha=1/2\)) : \(\lambda(\tau)\simeq 2\,\omega_\parallel \tau\).  
%\end{itemize}
%%Ces comportements sont observés sur la Fig.~7.4(b).
%
%\paragraph{Régime à temps courts.}  
%À temps courts, on peut approximer \(\mu(x,\tau)\simeq \mu_0(x)=\mu_p-\frac{1}{2}m\omega_\parallel^2 x^2\). Le comportement initial est alors indépendant de l'équation d'état \(\mu(n)\). L'équation d'Euler sans potentiel extérieur donne
%\[
%\frac{d^2 x}{d\tau^2} \simeq \omega_\parallel^2 x \quad\Rightarrow\quad v(x,\tau)\simeq \omega_\parallel^2 x \,\tau \quad (\tau\to 0).
%\]
%En réinjectant ce profil dans l'équation de continuité et en intégrant, on obtient
%\[
%\frac{1}{\lambda(\tau)} \simeq 1 - \frac{\omega_\parallel^2 \tau^2}{2} + \mathcal{O}(\tau^4) \quad\Rightarrow\quad
%\lambda(\tau)\simeq 1 + \frac{\omega_\parallel^2 \tau^2}{2} + \mathcal{O}(\tau^4),
%\]
%ce qui correspond au comportement observé à temps courts sur la Fig.~7.4(a), identique pour les régimes TF 1D et TF 3D.
%
%
%------------------
%
%\paragraph{Régime à temps courts.}  
%Pour \(\tau \to 0\), on linéarise le facteur d'échelle autour de l'équilibre \(\lambda=1\) en posant
%\(\lambda(\tau) = 1 + \epsilon(\tau)\) avec \(|\epsilon|\ll 1\). L'équation de mouvement devient alors
%\[
%\ddot \epsilon + (1+\alpha)\,\omega_\parallel^2 \epsilon - \omega_\parallel^2 = 0,
%\]
%équivalente à un oscillateur harmonique forcé. La solution pour des conditions initiales
%\(\epsilon(0)=0\), \(\dot\epsilon(0)=0\) est
%\[
%\epsilon(\tau) \simeq \frac{\omega_\parallel^2}{1+\alpha}\left[1 - \cos\left(\sqrt{1+\alpha}\,\omega_\parallel \tau\right)\right].
%\]
%Ainsi, à temps très courts \(\tau\ll 1/\omega_\parallel\), on retrouve
%\[
%\epsilon(\tau) \simeq \frac{1}{2}\,\omega_\parallel^2 \tau^2 + \mathcal{O}(\tau^4),
%\]
%et donc
%\[
%\lambda(\tau) \simeq 1 + \frac{\omega_\parallel^2 \tau^2}{2} + \mathcal{O}(\tau^4),
%\]
%ce qui coïncide avec le comportement universel observé à temps courts pour tous les régimes TF, indépendamment de \(\alpha\) et de l'équation d'état \(\mu(n)\).
%
%----------------------------

On impose les conditions initiales
\[
\lambda(0)=1, \qquad \dot\lambda(0)=0,
\]
et l'on considère le cas \(\alpha>0\).

\paragraph{Régime à temps courts (\(\tau \ll 1/\omega_\parallel\)).}  
On linéarise autour de l'équilibre \(\lambda=1\) en posant \(\lambda(\tau)=1+\epsilon(\tau)\) avec \(|\epsilon|\ll 1\). L'équation de mouvement devient un oscillateur harmonique forcé :
\[
\ddot \epsilon + (1+\alpha)\,\omega_\parallel^2 \epsilon - \omega_\parallel^2 = 0.
\]
Pour les conditions initiales choisies, la solution à petits temps est
\[
\epsilon(\tau) \simeq \frac{1}{2}\,\omega_\parallel^2 \tau^2 \quad\Rightarrow\quad
\lambda(\tau) \simeq 1 + \frac{\omega_\parallel^2 \tau^2}{2},
\]
indépendamment de l'équation d'état \(\mu(n)\). Ce comportement correspond au profil universel observé à temps courts (Fig.~7.4(a)) pour tous les régimes TF.

\paragraph{Régime à temps longs (\(\tau \gg 1/\omega_\parallel\)).}  
Pour \(\lambda^{-\alpha}\ll 1\), l'équation intégrée donne
\[
\dot\lambda \simeq \sqrt{\frac{2\omega_\parallel^2}{\alpha}} \quad\Rightarrow\quad
\lambda(\tau) \simeq \frac{2}{\alpha}\,\omega_\parallel \tau.
\]
En particulier :
\begin{itemize}
\item TF 1D (\(\alpha=1\)) : \(\lambda(\tau)\simeq \sqrt{2}\,\omega_\parallel \tau\),  
\item TF 3D (\(\alpha=1/2\)) : \(\lambda(\tau)\simeq 2\,\omega_\parallel \tau\).  
\end{itemize}
Ces comportements sont bien observés sur la Fig.~7.4(b) et correspondent à l’expansion asymptotique du gaz.





% --- Table révisée (petites corrections typographiques) ---
\begin{table}[h]
\centering
\begin{tabular}{l c c c}
\hline
Système & loi pour $\mu(n)$ & $\beta$ (avec $f(\lambda)=\lambda^{-\beta}$) & $\displaystyle \omega_{\rm breath}/\omega_\parallel$ \\
\hline
Gaz classique isotherme (1D, $\mu\propto\ln n$) 
& $\mu'(n)\propto 1/n$ 
& $-1$ 
& $\sqrt{3}\approx1.732$ \\[4pt]

Gaz de Bose 1D en régime moyen (GP, $\mu\propto n$) 
& $\alpha=1$ 
& $0$ 
& $2$ \\[4pt]

Tonks--Girardeau (1D, $\mu\propto n^2$) 
& $\alpha=2$ 
& $1$ 
& $\sqrt{5}\approx2.236$ \\[4pt]

Gaz de Fermi unitaire (ex. 3D, $\mu\propto n^{2/3}$) 
& $\alpha=\tfrac{2}{3}$ 
& $-\tfrac{1}{3}$ 
& $\sqrt{3+\tfrac{2}{3}}\approx1.915$ \\[4pt]

Cas général (loi de puissance) 
& $\mu\propto n^\alpha$ 
& $\beta=\alpha-1$ 
& $\displaystyle \sqrt{3+\alpha}$ \\
\hline
\end{tabular}
\caption{Valeurs de $\beta$ et fréquences du mode de souffle pour quelques régimes usuels.}
\label{tab:breathing}
\end{table}
 



\subsection{Sonde locale de distribution de rapidité}
\begin{itemize}
    \item Principe de la mesure : coupure d’une tranche puis expansion.
    \item Rôle du DMD dans la sélection.
    \item Accès à la distribution de vitesse locale.
    \item Comparaison avec les prédictions GHD.
    \item Limites et incertitudes
\end{itemize}

\subsubsection{Distribution de rapidités locale dans les gaz 1D}

\paragraph{Motivation.}  
La compréhension des gaz de bosons 1D avec interactions de contact répulsives repose sur la notion de distribution de rapidités \(\rho(\theta)\). Chaque état propre du système peut être paramétré par un ensemble de rapidités \(\{\theta_i\}\) (Ansatz de Bethe), ou interprété comme les vitesses de quasi-particules à durée de vie infinie. Ici, on utilise la définition pratique issue des expansions 1D : les rapidités correspondent aux vitesses asymptotiques des atomes après une expansion, avec \(x_j \simeq \tau \theta_j\) pour un temps \(\tau\) long. Cette définition est directement applicable à des mesures expérimentales de distribution de rapidités locales.

\paragraph{Distribution locale et LDA.}  
Pour un nuage atomique piégé dans un potentiel longitudinal variant lentement, on peut appliquer l’Approximation de Densité Locale (LDA). Le gaz est alors vu comme un fluide décomposé en cellules mésoscopiques de densité homogène et relaxée. Dans chaque cellule, l’état d’équilibre est décrit par un Ensemble de Gibbs Généralisé (GGE), ou équivalemment par une distribution de rapidités locale \(\rho(x,\theta)\). Cette description permet d’étudier non seulement l’équilibre, mais aussi la dynamique hors équilibre à grandes échelles spatiales et temporelles, via la théorie Hydrodynamique Généralisée (GHD).

\paragraph{Protocole expérimental.}  
Pour mesurer \(\rho(x,\theta)\) localement :  
\begin{enumerate}
    \item Une zone du nuage atomique de taille \(\ell\) centrée en \(x_0\) est sélectionnée à l’aide d’un dispositif de micromiroirs digitaux (DMD). La pression de radiation supprime instantanément les atomes en dehors de la zone, laissant uniquement ceux de la cellule.
    \item Après la sélection, le confinement longitudinal est relâché, tandis que le confinement transverse reste actif. Les atomes réalisent une expansion 1D pendant un temps \(\tau\), puis le profil de densité est imagé (typiquement pour \(\tau\sim 1\) ms).
    \item Le protocole est répété pour plusieurs positions \(x_0\), permettant d’obtenir la distribution de rapidités locale sur l’ensemble du nuage.
\end{enumerate}

\paragraph{Mesures à l’équilibre.}  
Pour un gaz initialement à l’équilibre dans un piège harmonique, le profil de densité de chaque zone sélectionnée est analysé via la thermodynamique Yang-Yang et la LDA, donnant température \(T_{\rm YY}\) et potentiel chimique \(\mu_{\rm YY}\). Après un temps d’expansion long, le profil devient homothétique à la distribution de rapidités locale \(\rho(x,\theta)\). La comparaison avec les prédictions numériques montre une bonne cohérence, confirmant que le protocole permet de sonder efficacement \(\rho(x,\theta)\).

\paragraph{Résumé.}  
— Une sonde locale de distribution de rapidités a été mise en place grâce au DMD.  
— Les atomes sélectionnés réalisent une expansion dans le guide 1D.  
— Après un temps long, le profil de densité reflète la distribution de rapidités locale.  
— Ce protocole a été appliqué avec succès sur un nuage atomique initialement à l’équilibre.


\section{Discussion sur les limites et les perspectives}
\begin{itemize}
    \item Contraintes techniques (bruit, alignement, stabilité de la puce…).
    \item Améliorations potentielles (résolution, contrôle du potentiel, automatisation).
    \item Perspectives pour d’autres types d’expériences (étude de chocs, turbulence quantique, etc.)
\end{itemize}

\section*{Conclusion}
\begin{itemize}
    \item Résumé de l’architecture du dispositif).
    \item Méthodes d’analyse utilisées et robustesse.
    \item Importance de l’expérience dans le contexte de l’étude des gaz quantiques unidimensionnels
\end{itemize}
Ce chapitre a présenté les éléments essentiels du dispositif expérimental, les méthodes d’imagerie, ainsi que les expériences auxquelles j’ai participé. L’ensemble constitue une plateforme performante pour l’étude de la dynamique de gaz 1D hors équilibre.

\paragraph{Résumé de l’architecture expérimentale}  
Nous avons décrit les éléments clés du dispositif utilisé : un système de refroidissement laser basé sur trois sources couplées, un piégeage magnétique sur puce optimisé pour réaliser des géométries unidimensionnelles, une plateforme de modulation de potentiel via un DMD, et un système d’imagerie haute résolution. L’ensemble permet une manipulation fine des nuages atomiques dans un cadre reproductible et stable.

\paragraph{Méthodes d’analyse et robustesse}  
L’imagerie par absorption, couplée à une analyse rigoureuse des profils atomiques, fournit des outils fiables pour extraire les grandeurs pertinentes : densités, tailles, températures, distributions de vitesses. Ces méthodes ont permis de confronter les résultats expérimentaux à des prédictions théoriques de type GHD ou Yang-Yang.

\paragraph{Importance du dispositif pour la thèse}  
Ce dispositif a été essentiel pour mener à bien les expériences présentées dans cette thèse. Il offre à la fois un contrôle local (grâce au DMD), un bon confinement transverse (grâce à la puce) et une imagerie précise. La plateforme est ainsi bien adaptée pour étudier des systèmes 1D fortement corrélés hors équilibre, et pour tester les prédictions de la physique statistique intégrable.

\paragraph{Perspectives}  
Malgré ses atouts, le dispositif présente des limitations techniques (rugosité magnétique, sensibilité à l’alignement, etc.) qui laissent entrevoir des pistes d’amélioration. Des développements futurs pourraient notamment viser à augmenter la résolution spatiale, automatiser davantage les séquences, ou explorer d'autres régimes dynamiques comme la turbulence ou les collisions de chocs quantiques.



%\appendix
\section*{Annexes}
\begin{itemize}
    \item Schémas techniques (puce, DMD, optique).
    \item Tableaux de paramètres expérimentaux.
    \item Exemples de motifs DMD utilisés.
\end{itemize}
\input{chapters/06_Bipart}
\input{chapters/07_Dipolaire}

%\chapter*{Conclusion}
\addcontentsline{toc}{chapter}{Conclusion}

Conclusion de la thèse.


%\appendix
%\chapter{Annexes}

Informations complémentaires.



\bibliographystyle{abbrv}
\bibliography{thesis}

%\printbibliography

\end{document}

%| Style     | Description                                                             |
%| --------- | ----------------------------------------------------------------------- |
%| `plain`   | Tri alphabétique, numérotation croissante                               |
%| `unsrt`   | Même que `plain` mais sans tri, respecte l’ordre d’apparition           |
%| `abbrv`   | Comme `plain` mais avec prénoms et noms abrégés                         |
%| `alpha`   | Les références sont étiquetées par une combinaison du nom et de l’année |
%| `apalike` | Style APA simplifié                                                     |
%| `ieeetr`  | Style IEEE, tri par ordre d’apparition                                  |
%| `siam`    | Style SIAM (mathématiques appliquées)                                   |
%| `acm`     | Style ACM (informatique)                                                |
%



\subsection{Équations intégrales de la TBA}

\paragraph{Moyenne des observables dans l’ensemble généralisé de Gibbs.}

\paragraph{Approximation au point selle («\,méthode de la selle statique\,»)}

Dans la limite thermodynamique \( L \to \infty \), cette intégrale est dominée par la configuration \( \rho_{eq} \) qui maximise le poids exponentiel $e^{L(\mathcal{S}_{YY}-\mathcal{W})[\rho]}$  dans l'expression \eqref{chap:TBA:eq:ensemble_average}. Il s’agit de la densité de rapidité la plus probable, solution d’un problème de maximisation. On obtient à l’ordre principal
\begin{eqnarray}
	\underset{\mbox{\tiny therm.}}{\lim} \langle \operator{\mathcal{O}} \rangle_{\operator{\varrho}[w]} & \approx &  \langle\operator{\mathcal{O}}\rangle_{[\rho_{eq} ]},	
	\label{chap:TBA:eq:ensemble_average:approx}
\end{eqnarray}
où $\rho_{eq}$ est la distribution de rapidité à l'équilibre \eqref{chap.2:eq.rho.eq.1}.
Cette approximation correspond à une méthode de \textit{selle statique}, où l’on développe la \emph{fonction thermodynamique effective}, $\mathcal{S}_{YY}-\mathcal{W}$  au voisinage de la distribution dominante.


\paragraph{Développement fonctionnel au premier ordre.}

%On effectue un développement de Taylor fonctionnel de l'action à l’ordre linéaire en $\rho = \rho_{eq} + \delta \rho$ :
Écrivons
\(
\rho=\rho_{\text{eq}}+\delta\rho
\)
et développons $(\mathcal{S}_{YY}-\mathcal{W})[\rho]$ à l’ordre linéaire :
\begin{eqnarray*}
	\mathcal{S}_{YY}[\rho] - \mathcal{W}[\rho] & \approx & \mathcal{S}_{YY}[ \rho_{eq}] - \mathcal{W}[ \rho_{eq}] +  \left. \frac{\delta (\mathcal{S}_{YY}[\rho] - \mathcal{W}[\rho]) }{\delta \rho} \right|_{\rho = \rho_{eq} }	(\delta \rho) + \mathcal{O}(\delta \rho^2 ) ,
	\label{chap:TBA:eq:action}	
\end{eqnarray*}	
La condition de stationnarité au point selle impose :
\(
	\left. \frac{\delta (\mathcal{S}_{YY}[\rho] - \mathcal{W}[\rho]) }{\delta \rho} \right|_{\rho = \rho_{eq} }	  =  0  	
\)
soit 
\begin{equation}
\left. \frac{\delta \mathcal{S}_{YY}}{\delta \rho} \right|_{\rho = \rho_{eq}} = \left. \frac{\delta \mathcal{W}}{\delta \rho} \right|_{\rho = \rho_{eq}}. \label{chap:TBA:eq:stationnarite}
\end{equation}

%%%%%%%%%%%%%%%%
%-----------------------------------------------------

%------------------------------------------------------------------
%\subsection{Équations intégrales de la TBA}

%\paragraph{Moyenne des observables dans le Generalized Gibbs Ensemble.}

%Dans la limite thermodynamique, la moyenne d’une observable locale
%s’écrit formellement comme une intégrale fonctionnelle sur les densités de
%rapidité\,\footnote{%
%La mesure fonctionnelle $\mathcal{D}\rho$ est la limite continue de la
%somme discrète sur les macro-états admissibles, chacun étant pondéré par
%le facteur combinatoire $e^{L\mathcal{S}_{YY}[\rho]}$.}
%
%\begin{equation}\label{eq:TBA:ensemble_average}
%\left\langle \mathcal{O} \right\rangle_{\!\text{GGE}}
%=\frac{\displaystyle
%      \int\!\mathcal{D}\rho\;
%      e^{L\bigl[\mathcal{S}_{YY}[\rho]-\mathcal{W}[\rho]\bigr]}\;
%      \langle\mathcal{O}\rangle_{[\rho]}}
%     {\displaystyle
%      \int\!\mathcal{D}\rho\;
%      e^{L\bigl[\mathcal{S}_{YY}[\rho]-\mathcal{W}[\rho]\bigr]}} .
%\end{equation}

%------------------------------------------------------------------
%\paragraph{Approximation au point selle («\,méthode de la selle statique\,»).}

%Lorsque $L\to\infty$, les intégrales \eqref{eq:TBA:ensemble_average}
%sont dominées par la distribution
%$\rho_{\text{eq}}$ qui \emph{maximise} l’exposant
%\(
%\Phi[\rho]=\mathcal{S}_{YY}[\rho]-\mathcal{W}[\rho].
%\)
%On obtient à l’ordre principal
%\begin{equation}
%\left\langle \mathcal{O} \right\rangle_{\!\text{GGE}}
%\;\simeq\;
%\langle \mathcal{O} \rangle_{[\rho_{\text{eq}}]} .
%\label{eq:TBA:saddle_average}
%\end{equation}

%------------------------------------------------------------------
%\paragraph{Condition de stationnarité et équation variationnelle.}

%Écrivons
%\(
%\rho=\rho_{\text{eq}}+\delta\rho
%\)
%et développons $\Phi[\rho]$ à l’ordre linéaire :
%\[
%\Phi[\rho]\;=\;
%\Phi[\rho_{\text{eq}}]
%+
%\int d\theta\,
%\left.
%\frac{\delta\Phi}{\delta\rho(\theta)}
%\right|_{\rho_{\text{eq}}}
%\delta\rho(\theta)
%+O(\delta\rho^{2}).
%\]
%La stationnarité impose
%\(
%\dfrac{\delta\Phi}{\delta\rho(\theta)}\bigl|_{\rho_{\text{eq}}}=0,
%\)
%soit
%\begin{equation}
%\left.
%\frac{\delta\mathcal{S}_{YY}}{\delta\rho(\theta)}
%\right|_{\rho_{\text{eq}}}
%=
%\left.
%\frac{\delta\mathcal{W}}{\delta\rho(\theta)}
%\right|_{\rho_{\text{eq}}}.
%\label{eq:TBA:variational_condition}
%\end{equation}

%------------------------------------------------------------------
%\paragraph{Forme explicite : introduction de la pseudo-énergie.}

%Pour le modèle de Lieb–Liniger (et, plus généralement, pour un modèle
%intégrable à noyau $\Delta$), on introduit la \emph{pseudo-énergie}
%\[
%\varepsilon(\theta)
%\;=\;
%w(\theta)
%\;+\;\Bigl[\Delta\star\ln\!\bigl(1+e^{-\varepsilon}\bigr)\Bigr](\theta),
%\]
%obtenue en réécrivant \eqref{eq:TBA:variational_condition}.
%Le \emph{facteur d’occupation}
%\(
%\nu(\theta)=\rho(\theta)/\rho_s(\theta)
%\)
%se donne alors par la statistique de type Fermi-Dirac
%\[
%\nu(\theta)=\frac1{1+e^{\varepsilon(\theta)}}.
%\]

%Les équations intégrales complètes de la \textbf{Thermodynamique de Bethe}
%(TBA) sont donc
%\begin{align}
%2\pi\rho_s(\theta) &= 1 + \bigl[\Delta \star \rho\bigr](\theta),
%\label{eq:TBA:rho_s}\\[4pt]
%\rho(\theta) &= \frac{\rho_s(\theta)}{1+e^{\varepsilon(\theta)}},
%\qquad
%\varepsilon(\theta)=w(\theta)+\bigl[\Delta\star\ln(1+e^{-\varepsilon})\bigr](\theta).
%\label{eq:TBA:epsilon}
%\end{align}
%Elles déterminent sans ambiguïté la distribution d’équilibre
%$\rho_{\text{eq}}(\theta)$ en fonction du poids spectral $w(\theta)$.

%\medskip
%Ainsi, la méthode du point selle relie le \emph{poids spectral}
%(caractéristique du GGE) à la distribution de rapidité la plus probable,
%et permet d’évaluer les observables par la formule
%\label{chap:TBA:eq:ensemble_average:approx}.


%-----------------------------------------------------
%%%%%%%%%%%%%%%%

%\paragraph{Équation intégrale de la TBA.}

%Cette égalité donne naissance à une équation intégrale pour le poids spectral \( w \), défini comme la dérivée fonctionnelle de l'énergie généralisée pris en $\rho_{eq}$ :
%\(
%w ~=~ \left. \frac{\delta \mathcal{W}[\rho]}{\delta \rho} \right|_{\rho =  \rho_{eq} }
%\)
%qui par stationnarité (cf équation \eqref{chap:TBA:eq:stationnarite}) est égale à la dérivée fonctionnelle de l'entropie de Yang-Yang pris en $\rho_{eq}$ :
%\(
%\left. \frac{\delta \mathcal{S}_{YY}[\rho]}{\delta \rho} \right|_{\rho = \rho_{eq} }
%\) 
%qui lui vaux 
%\(
%\ln ( \nu_{eq}^{-1}  - 1 ) - \frac{\Delta}{2\pi} \star \ln ( 1 -  \nu_{eq })
%\)
%avec le facteur d'ocupation à l'équilibre $\nu_{eq} = \rho_{eq}/{\rho_{eq}}_s$. Ainci on peux s'arreter sur l'équation 
%\begin{eqnarray}
%	w & = & \ln ( \nu_{eq}^{-1}  - 1 ) - \frac{\Delta}{2\pi} \star \ln ( 1 -  \nu_{eq }).\label{chap:TBA:eq:w}
%\end{eqnarray}

%\medskip
%Ainsi, la méthode du point selle relie le \emph{poids spectral}
%(caractéristique du GGE) à la distribution de rapidité la plus probable,
%et permet d’évaluer les observables par la formule
%\eqref{chap:TBA:eq:ensemble_average:approx}.\\

%\paragraph{Forme explicite : introduction de la pseudo-énergie.}

%Le \emph{facteur d’occupation}
%\(
%\nu_{eq}
%\)
%se donne alors par la statistique de type Fermi-Dirac
%\begin{eqnarray}
%	\nu_{eq}=\frac1{1+e^{\epsilon}},\label{chap:TBA:eq:nu_eq}
%\end{eqnarray}
%où \emph{pseudo-énergie} 
%\(
%\epsilon
%\)
%se définie en intectant \eqref{chap:TBA:eq:nu_eq} dans \eqref{chap:TBA:eq:w} : 
%\begin{eqnarray}
%	\epsilon & = & w + \frac{\Delta}{2\pi} \star \ln ( 1  + e^{-\epsilon}).\label{chap:TBA:eq:e}	
%\end{eqnarray}


%---------------------------------
%------------------------------------------------------------------
\paragraph{Équation intégrale de la TBA.}

La condition de stationnarité au point selle \(\rho=\rho_{\mathrm{eq}}\) \eqref{chap:TBA:eq:stationnarite} implique :
\begin{eqnarray}
	\left.\frac{\delta\mathcal{S}_{YY}}{\delta\rho(\theta)}\right|_{\rho_{\mathrm{eq}}} = \left.\frac{\delta\mathcal{W}}{\delta\rho(\theta)}\right|_{\rho_{\mathrm{eq}}}\;\doteq\;w(\theta),
\end{eqnarray}
En utilisant l’expression explicite de l’entropie de Yang–Yang \eqref{chap.2.entropi.int}, on obtient l’identité fonctionnelle
\begin{eqnarray}
	w & = & \ln ( \nu_{\!eq}^{-1}  - 1 ) - \frac{\Delta}{2\pi} \star \ln ( 1 -  \nu_{\!eq}).\label{chap:TBA:eq:w}
\end{eqnarray}
où
\(
\nu_{\!eq}=\rho_{\!eq}/\rho_{s,\!eq}
\)
est le \textbf{facteur d’occupation} à l’équilibre.
%------------------------------------------------------------------
\paragraph{Forme pseudo-énergie.}
La \textbf{pseudo-énergie} $\epsilon$ se donne alors par la statistique de type Fermi-Dirac
\begin{eqnarray}
	\epsilon =\ln(\nu^{-1}_{\!eq}-1),\qquad\nu_{\!eq}=\frac{1}{1+e^{\epsilon}}.\label{chap:TBA:eq:nu}%\tag{\text{TBA--$\nu$}} 
\end{eqnarray}
En réinjectant \eqref{chap:TBA:eq:nu} dans \eqref{chap:TBA:eq:w} on obtient
l’équation intégrale canonique de la thermodynamique de Bethe :
\begin{eqnarray}
	\epsilon & = & w - \frac{\Delta}{2\pi} \star \ln ( 1  + e^{-\epsilon}).\label{chap:TBA:eq:e}%\tag{\text{TBA–-$\varepsilon$}}	
\end{eqnarray}
%\[
%\boxed{\;
%\varepsilon(\theta)
%=
%w(\theta)
%+\frac{\Delta}{2\pi}\star\ln\!\bigl[1+e^{-\varepsilon(\theta)}\bigr]
%\;}
%\tag{TBA–$\varepsilon$}\label{eq:TBA:eq:e}
%\]

Les relations \eqref{chap:TBA:eq:nu}–\eqref{chap:TBA:eq:e} déterminent de façon univoque la distribution de rapidité d’équilibre \(\rho_{\!eq}\) à partir du poids spectral \(w\), caractéristique du GGE.

\medskip
Ainsi, la méthode du point selle relie \emph{explicitement} le {\em poids spectral}, $w$  (caractéristique du GGE) au \emph{macro-état le plus probable}, $\rho_{eq}$ , et permet d’évaluer les observables par la formule d’ensemble \eqref{chap:TBA:eq:ensemble_average:approx}.


\paragraph{Résolution numérique de l’équation TBA.}\label{para-algho-TBA}

Prenons un poids spectrale quelconque, par exemple : 
\begin{equation}
  w(\theta)= \theta^2 .\label{eq:TBA:w:quadra}
\end{equation} 
En injectant $w$ dans l’équation intégrale pour lapseudo-énergie \eqref{chap:TBA:eq:e}, on obtient l’équation non linéaire.
Cette équation définit un opérateur contractant sur l’espace des fonctions
\( \epsilon(\theta) \) ; son Jacobien a une norme strictement
inférieure à 1, garantissant la convergence de l’itération de Picard.

\medskip
\subparagraph{Algorithme d’itération.}  
La structure contractante de l’équation garantit l’absence de cycles ou de points fixes multiples, assurant la convergence de l’itération vers l’unique solution admissible.
L’équation \eqref{eq:num:TBA} est non linéaire ; pour la résoudre numériquement, on utilise une méthode itérative de type Picard. On initialise
\(
  \epsilon_0 = w ,
\)
puis on construit une suite de fonctions \(\varepsilon_n\) définie par
\begin{eqnarray*}
	\epsilon_{n+1} & = & \epsilon_0 -   \frac{\Delta}{2\pi} \star \ln \left( 1 + e^{-\epsilon_n} \right) ,\quad n\ge0
\end{eqnarray*}
L’itération est poursuivie jusqu’à convergence, que l’on peut tester via le critère numérique
\(
  \beta \left\| \varepsilon_{n+1} - \varepsilon_n \right\|_\infty < 10^{-12},
\)
où \(\|\cdot\|_\infty\) désigne la norme \(L^\infty\) (ou un maximum discret après discrétisation).


\medskip
\subparagraph{Facteur d’occupation et densités.}  
Une fois la pseudo-énergie \( \epsilon(\theta) \) convergée, le facteur d’occupation  à l'équilibre est obtenu en injectant $\epsilon$ dans l’équation \eqref{chap:TBA:eq:nu}, ce qui donne  $\nu_{\!eq}$.
 
On en déduit ensuite la densité d'état à l'équilibre $\rho_{s,eq}$ via le {\bf dressing}  de la fonction constante $f(\theta) = 1$, selon \eqref{eq:TBA-rhos-2}, rappelée ici pour mémoire : $ 2\pi \rho_{s,eq}  =  1^{\mathrm{dr}}_{[\nu_{\! eq}]}$.\\

L’opérateur de dressing \eqref{eq:dressing} étant linéaire, il se résout numériquement sous la forme :
\begin{eqnarray*}
	\left\{ \mathrm{id} - \frac{\Delta}{2\pi} \star ( \nu \ast \cdot ) \right\} f^{\mathrm{dr}}_{[\nu]} & = & f,\label{eq:TBA:rho_s:num}
\end{eqnarray*}
où $\mathrm{id} \colon f \mapsto f$ est l’identité fonctionnelle, et $\ast$ désigne la multiplication.
Après discrétisation de la variable $\theta$, cette équation devient un système linéaire de type $Ax=b$ , facilement résoluble numériquement.

La distribution de rapidité est alors obtenue par $\rho_{\!\mathrm{eq}} = \nu_{\!\mathrm{eq}} \ast \rho_{\! s,\mathrm{eq}}$.\\

\medskip
Ainsi en fixant le poids spectral $w(\theta)$, l’algorithme fournit la pseudo-énergie \( \epsilon \), le facteur d’occupation \( \nu_{\mathrm{eq}} \) et la distribution de rapidité \( \rho_{\!\mathrm{eq}} \).

\medskip
\subparagraph{À l'équilibre thermique.} 
Si on se place à l’équilibre canonique, caractérisé par la température \( T \) et le potentiel chimique \( \mu \).  Dans ce cadre, le poids spectral vaut
\begin{equation}
  w(\theta)=\beta\bigl[\varepsilon(\theta)-\mu\bigr],\qquad\beta=\tfrac1T\; (k_B = 1 ),\quad\varepsilon(\theta)=\tfrac{\theta^{2}}{2}\;(m=1).\label{eq:TBA:w:canonical}
\end{equation}
%En injectant \eqref{eq:TBA:w:canonical} dans l’équation intégrale pour lapseudo-énergie \eqref{chap:TBA:eq:e}, on obtient l’équation non linéaire :
%\begin{eqnarray*}
%	\epsilon & = & \beta(\varepsilon - \mu)  -  \frac{\Delta}{2\pi} \star \ln \left( 1 + e^{-\epsilon} \right) ,\label{eq:num:TBA}
%\end{eqnarray*}
%Ainsi, pour tout couple \((T,\mu)\), l’algorithme fournit la pseudo-énergie \( \epsilon \), le facteur d’occupation \( \nu_{\mathrm{eq}} \) et la distribution de rapidité \( \rho_{\mathrm{eq}} \) à l’équilibre thermique, prêts à être utilisés pour le calcul des observables.
%
%\medskip
%Pour $w$ quelconque , l'algorythme est identique.




		

