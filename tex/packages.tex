% -------------------------------------
% Encodage et langue
% -------------------------------------
\usepackage[utf8]{inputenc}
\usepackage[T1]{fontenc}
\usepackage[french]{babel}

% -------------------------------------
% Marges et dimensions
% -------------------------------------
\usepackage[a4paper, top=3.0cm, bottom=3.0cm, left=3cm, right=3cm]{geometry} 
% Ajuste ici les marges selon tes préférences

% -------------------------------------
% Interligne
% -------------------------------------
\usepackage{setspace}
%\onehalfspacing  % Interligne 1.5 
%\doublespacing %(utilise \doublespacing pour double interligne)

% -------------------------------------
% Police (facultatif)
% -------------------------------------
%\usepackage{mathptmx} % Police Times (ancienne)
%\usepackage{libertine} % Police élégante
%\usepackage{newtxtext,newtxmath} % Times moderne pour texte et maths

% -------------------------------------
% Paquets utiles
% -------------------------------------
\usepackage{amsmath, amssymb, amsthm}
\usepackage{graphicx}
\usepackage{hyperref}
\usepackage{xcolor}
\usepackage{braket}
\usepackage{tikz}
\usepackage{pgfplots}
\usepackage{float}
\usepackage{enumitem}
\usepackage{caption}
\usepackage{subcaption}
\usepackage{algorithm2e}
\usepackage{cancel}
\usepackage{bm}
\usepackage{listings}
\usepackage{pdfpages}
\usepackage{mdframed}
\usepackage{braket}
\usepackage{stmaryrd} 
\usetikzlibrary {datavisualization}
\usetikzlibrary {arrows.meta,bending,positioning}
\usetikzlibrary {datavisualization.formats.functions}
%PREAMBULE pour schÃéma
\usepackage{pgfplots}
\usepackage{tikz}
\usepackage[european resistor, european voltage, european current]{circuitikz}
\usetikzlibrary{arrows,shapes,positioning}
\usetikzlibrary{decorations.markings,decorations.pathmorphing,
decorations.pathreplacing}
\usetikzlibrary{calc,patterns,shapes.geometric}
\usepackage{anyfontsize}


% -------------------------------------
% Pour les chapitres
% -------------------------------------
\usepackage[Glenn]{fncychap} % Style de chapitres


% -------------------------------------
% Largeur du texte (évite de le redéfinir si tu utilises geometry)
% -------------------------------------
%\setlength\textwidth{20.5cm}
%\setlength\textheight{22cm}

% -------------------------------------
% Optionnel : si tu veux jouer avec les marges manuellement
% -------------------------------------
% \setlength\topmargin{-1cm}
% \setlength\evensidemargin{-2cm}
% \setlength\oddsidemargin{\evensidemargin}

\usepackage{mdframed}

\usepackage{scalerel}
\usepackage{xcolor}
\usepackage{stackengine}
\usepgflibrary {shadings}


\usetikzlibrary {decorations.pathmorphing}

\usepackage{tikz}

\usepackage{marvosym}
\usepackage{changepage}

\usepackage{minitoc}
\usepackage{tocloft}
%\renewcommand{\cfttoctitle}{\hspace{-2em}}
% Nastaveni obsahu
% Nastaveni obsahu

\usepackage{imakeidx}
\usepackage{fancyhdr}

%\usepackage{makeidx}
\makeindex[intoc=true]
\makeindex[name=pers, title=Index of person names, intoc=true]

\usepackage{xcolor}

\usepackage{hyperref}

%%%%%%%%%%%%%%%%%%%%%
%\definecolor{linkcolor}{RGB}{0,0,180}
\usepackage{titlesec}

\usepackage{tocloft}
\usepackage{datetime} % Pour une date personnalisée
\usepackage[useregional]{datetime2}

\usepackage{mathrsfs}

% -------------------------------------
% Pour les mini-tables des matières
% -------------------------------------
\usepackage{minitoc}
\dominitoc

%\usepackage[most]{tcolorbox}

%%%%%%%%%%%%%%%%%%%%%%%%%%%%%
%\usepackage[utf8]{inputenc}
%\usepackage[T1]{fontenc}
%\usepackage[french]{babel}
%\usepackage{amsmath, amssymb}
%\usepackage{graphicx}
%\usepackage{hyperref}
%\usepackage{tikz}
%\usepackage{physics}
%\usepackage{float}

