\def\colorslide{blue!50!black}

\def\Occupation{
	\def\traitx{0.3}
	\def\traity{0.5}
	\draw
		(-10.5 , 0 ) edge [thick,line width=0.8ex ]( -3.2  , 0 )
		( -3.2 - \traitx  , 0 - \traity ) edge [thick,line width=0.8ex ]( -3.2 + \traitx  , 0 + \traity  )
		( -2.8 - \traitx  , 0 - \traity ) edge [thick,line width=0.8ex ]( -2.8 + \traitx  , 0 + \traity  )
		(-2.8 , 0 ) edge [thick,line width=0.8ex ](2.8  , 0 )
		( 2.8 - \traitx  , 0 - \traity ) edge [thick,line width=0.8ex ]( 2.8 + \traitx  , 0 + \traity  )
		( 3.2 - \traitx  , 0 - \traity ) edge [thick,line width=0.8ex ]( 3.2 + \traitx  , 0 + \traity  )
		(3.2, 0 ) edge [thick,line width=0.8ex,->,>=triangle 45 , color = black ]node [pos=1.01,below  ]{\huge$a$}	( 11  , 0 )
	;
	
	
	% Graduation abcsisse 
	\foreach \r in { 0 ,... , 2   } {
		\ifnum\r=0
			\draw [color = gray  ]
				( \r  , -0.3 ) edge [thick,line width=0.5ex] node [pos=-0.5 ]{\large \r }	( \r  , 0.3 )
			;
			\filldraw [line width=0.5ex , color = red ,outer color=red,inner color=red ] 
				( \r  , 0 )  circle (4pt)
			;
		\else\ifnum\r>0
			\draw [color = gray  ]
				( \r  , -0.3 ) edge [thick,line width=0.5ex] node [pos=-0.5 ,right]{\large \r }	( \r  , 0.3 )
				( -\r  , -0.3 ) edge [thick,line width=0.5ex]node [pos=-0.5, left ]{\large -\r }	( -\r  , 0.3 ) 
			;
			\filldraw [ line width=0.5ex , color = red ,outer color=red,inner color=red ]
				( -\r  , 0 )  circle (5pt)
			;
			\filldraw [ line width=0.5ex , color = red ,outer color=red,inner color=red ] 
				( \r  , 0 )  circle (5pt)
			;
		\fi\fi
	}

	
	\def\nmax{6};
	\def\nsmax{10}; % les trous
	%\foreach \r in {\nmax + 1 ,...,\nsmax} { % proble dans le -\r 
	\foreach \r in {4 ,...,\nsmax} {
		\draw [color=gray] 
			(\r,-0.3) edge [thick,line width=0.5ex]  (\r,0.3)
			(-\r,-0.3) edge [thick,line width=0.5ex]  (-\r,0.3)
		;
		\filldraw [line width=0.5ex , color = red ,outer color=white,inner color=white ] (\r,0) circle (4pt);
		\filldraw [line width=0.5ex , color = red ,outer color=white,inner color=white] (-\r,0) circle (4pt);
	}
	
	
	\foreach \r in {4,...,\nmax} {
		\ifnum\r=\nmax
			\draw [color=gray] 
				(\r,-0.3) edge [thick,line width=0.5ex] node [pos=-0.5]{ $\frac{N}{2}$ } (\r,0.3)
				(-\r,-0.3) edge [thick,line width=0.5ex] node [pos=-0.5]{ $-\frac{N}{2}$ } (-\r,0.3)
			;
			\filldraw [line width=0.5ex , color = red ,outer color=red,inner color=red ] (\r,0) circle (4pt);
			\filldraw [line width=0.5ex , color = red ,outer color=red,inner color=red ] (-\r,0) circle (4pt);
		\else\ifnum\r>4
			%\def\nreste{\nmax-\r}
			\pgfmathtruncatemacro{\nreste}{\nmax-\r}

			\draw [color=gray] 
				(\r,-0.3) edge [thick,line width=0.5ex] node [pos=-0.5]{ $\frac{N}{2} - \nreste$ } (\r,0.3)
				(-\r,-0.3) edge [thick,line width=0.5ex] node [pos=-0.5]{$-\frac{N}{2}+\nreste$ } (-\r,0.3)
			;
			\filldraw [line width=0.5ex , color = red ,outer color=red,inner color=red ] (\r,0) circle (4pt);
			\filldraw [line width=0.5ex , color = red ,outer color=red,inner color=red ] (-\r,0) circle (4pt);
		\else
			\draw [color=gray] 
				(\r,-0.3) edge [thick,line width=0.5ex] (\r,0.3)
				(-\r,-0.3) edge [thick,line width=0.5ex] (-\r,0.3)
			;
			\filldraw [line width=0.5ex , color = red ,outer color=red,inner color=red ] (\r,0) circle (4pt);
			\filldraw [line width=0.5ex , color = red ,outer color=red,inner color=red ] (-\r,0) circle (4pt);
		\fi\fi
	}
	
	
			
}


\begin{scope}
	%\draw[help lines , width=1.5ex] (-8,-3) grid (8,3);\draw[help lines ,width=0.5ex , opacity = 0.5] (-3,-3) grid[step=0.1] (3,3));
	
	%\draw[help lines] 
	%	(-3,-3) edge[width=1.5ex] grid (3,3)	
	%	(-3,-3) edge[width=0.5ex , opacity = 0.5] grid (3,3)	
	%;
	\begin{scope}[shift={(0,1)},rotate=0,opacity=1,color=black]
		\Occupation	
		
		\node[anchor=east, font=\bfseries] at (-11, 0) {\color{red}\large (T = 0 )} ;	
	\end{scope}
	
	
	\begin{scope}[shift={(0,-1.0)},rotate=0,opacity=1,color=black]
		\Occupation
		
		\node[anchor=east, font=\bfseries] at (-11, 0) {\color{red}\large (T > 0 )} ;
		%\def\nmax{8};
	
		\def\r{1};
		\draw [color=gray] 
			(\nmax-\r-1,-0.3) edge [thick,line width=0.5ex]  (\nmax-\r-1,0.3)
			(-\nmax+\r,-0.3) edge [thick,line width=0.5ex]  (-\nmax+\r,0.3)
			;
		\filldraw [line width=0.5ex , color = red ,outer color=white,inner color=white ] (\nmax-\r-1,0) circle (4pt);
		\filldraw [line width=0.5ex , color = red ,outer color=white,inner color=white] (-\nmax+\r,0) circle (4pt);
	
		\def\r{-1};
		\draw [color=gray] 
			(\nmax-\r,-0.3) edge [thick,line width=0.5ex]  (\nmax-\r,0.3)
			(-\nmax+\r,-0.3) edge [thick,line width=0.5ex]  (-\nmax+\r,0.3)
			;
		\filldraw [line width=0.5ex , color = red ,outer color=red,inner color=red ] (\nmax-\r,0) circle (4pt);
		\filldraw [line width=0.5ex , color = red ,outer color=red,inner color=red] (-\nmax+\r,0) circle (4pt);
	
		\draw[line width=0.8ex , color = red ,->,>=latex] ( \nmax - 1.8  ,0.7) to[out=55,in=125] (\nmax + 0.8 ,0.7) ;
		\draw[line width=0.8ex , color = red ,->,>=latex] (-\nmax + 0.8 ,0.7) to[out=125,in=55] (-\nmax - 0.8 ,0.7);
	
	
	\end{scope}
	
	\begin{scope}[shift={(-10.5,3)},rotate=0,opacity=1,color=black]
	
	\begin{scope}[shift={(-0,0)},rotate=0,opacity=1,color=black]
	
		\draw[shift={(0,0)} ,line width=1ex,rounded corners = 1ex,color=\colorslide , opacity =1 ,fill=\colorslide!00 , pattern={north east lines} , pattern color=\colorslide!00 ]
			(0 , -1 ) rectangle (5,1)
		;
		
		\filldraw [line width=0.5ex , color = red ,outer color=red,inner color=red ] (0.5 , 0.5) circle (4pt); 
		\node[anchor=west, font=\bfseries] at (0.7, 0.5) {\color{\colorslide}\large : quasi-particule};
		
		\filldraw [line width=-0.5ex , color = red ,outer color=white,inner color=white ] (0.5 , -0.5) circle (4pt); 
		\node[anchor=west, font=\bfseries] at (0.7, -0.5) {\color{\colorslide}\large : hole} ;
	\end{scope}
	
	\begin{scope}[shift={(6,0)},rotate=0,opacity=1,color=black]	
		
		\draw[shift={(0,0)} ,line width=1ex,rounded corners = 1ex,color=\colorslide , opacity =1 ,fill=\colorslide!00 , pattern={north east lines} , pattern color=\colorslide!00 ]
			(0 , -1 ) rectangle (7.5,1)
		;
		
		\node[anchor=west] at (0.5, 0.5) {\color{\colorslide}\large $\rho$ };\node[anchor=west, font=\bfseries] at (0.9, 0.5) {\color{\colorslide}\large : quasi-particule distribution};
		
		\node[anchor=west] at (0.5, -0.5) {\color{\colorslide}\large $\rho_h$ };\node[anchor=west, font=\bfseries] at (0.9, -0.5) {\color{\colorslide}\large  : hole distribution};
		
	\end{scope}
	
	\begin{scope}[shift={(14.5,0)},rotate=0,opacity=1,color=black]	
		
		\draw[shift={(0,0)} ,line width=1ex,rounded corners = 1ex,color=\colorslide , opacity =1 ,fill=\colorslide!00 , pattern={north east lines} , pattern color=\colorslide!00 ]
			(0 , -0.5 ) rectangle (7.0,0.5)
		;
		
		\node[anchor=west] at (0.5, 0) {\color{\colorslide}\large $\rho_s = \rho + \rho_h $ };\node[anchor=west, font=\bfseries] at (2.5, 0) {\color{\colorslide}\large : density of states};
		
	\end{scope}
	
	
	\end{scope}


		
	
\end{scope}

